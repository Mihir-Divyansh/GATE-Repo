\iffalse
\title{Gate Questions 2}
\author{EE24BTECH11012 - Bhavanisankar G S}
\section{ae}
\chapter{2008}
\fi
	\item In a 3-D orthotropic material, the number of elastic constants in linear stress-strain relationship is
		\begin{enumerate}
				\begin{multicols}{4}
				\item 3
				\item 5
				\item 9
				\item 21
				\end{multicols}
		\end{enumerate}
	\item The compatibility conditioons in theory of wlasticity ensure that
		\begin{enumerate}
			\item there is compatibility between various direct and shear stress
			\item relationships between stresses and strains are consistent with constitutive relations
			\item displacemets are single-valued and continuous
			\item stresses satisfy bi-harmonic equation
		\end{enumerate}
	\item In a spring-mass-damper single degree of freedom system, the mass is 2kg and the undamped natural frequencyy is 20Hz. The critical damping constant of the system is
		\begin{enumerate}
				\begin{multicols}{4}
				\item $ 160 \pi N-s/m $
				\item $ 80 \pi N-s/m $
				\item $ 1 N-s/m $
				\item $ 0 N-s/m $
				\end{multicols}
		\end{enumerate}
	\item Which of the following quantities remains constant for a satellite in an elliptical orbit around the earth ?
		\begin{enumerate}
			\item Kinetic energy
			\item Product of speed and radial distance from the centre of the earth.
			\item Rate of area swept by the radial vector from the centre of the orbit.
			\item Rate of area by the radial vector from the centre of the earth.
		\end{enumerate}
	\item A planet is observed to be at its slowest when it is at a distance $r_1$ from the sum and its fastest when it is at a distance $r_2$ from the sun. The eccentricity $e$ of the planet's orbit is given by
		\begin{enumerate}
				\begin{multicols}{4}
			\item $ e = \frac{r_1}{r_2} $
			\item $ e = \frac{r_1 - r_2}{r_1 + r_2} $
			\item $ e = \frac{r_2}{r_1} $
			\item $ e = \frac{r_1 + r_2}{r_1 - r_2} $
				\end{multicols}
		\end{enumerate}
	\item The function $f\brak{x,y,z} = \frac{1}{2} x^2y^2z^2 $ satisfies
		\begin{enumerate}
				\begin{multicols}{4}
				\item grad $f=0$
				\item div(grad $f$) =0
				\item curl(grad $f$) = 0
				\item grad(div(grad $f$))=0
				\end{multicols}
		\end{enumerate}
	\item Which of the following is true for all choices of vectors $\vec{p}, \vec{q}, \vec{r}$ ?
		\begin{enumerate}
			\item $ \vec{p} \times \vec{q} + \vec{q} \times \vec{r} + \vec{p} \times \vec{r} = 0 $
			\item $ \brak{\vec{p} \cdot \vec{q}}\vec{r} +  \brak{\vec{q} \cdot \vec{r}}\vec{p} + \brak{\vec{p} \cdot \vec{r}}\vec{q} = 0 $
			\item $ \vec{p} \cdot \brak{\vec{q} \times \vec{r}} + \vec{q} \cdot \brak{\vec{r} \times \vec{p}} + \vec{r} \cdot \brak{\vec{p} \times \vec{q}} =0 $
			\item $ \vec{p} \times \brak{\vec{q} \times \vec{r}} + \vec{q} \times \brak{\vec{r} \times \vec{p}} + \vec{r} \times \brak{\vec{p} \times \vec{q}} =0 $
		\end{enumerate}
	\item The value of the line integral $ \frac{1}{2 \pi} \int \brak{xdy - ydx} $ taken anti-clockwise along a circle of unit radius is
		\begin{enumerate}
				\begin{multicols}{4}
			\item 0.5
			\item 1
			\item 2
			\item $ \pi$
				\end{multicols}
		\end{enumerate}
	\item Which of the following is a solution of $ \frac{d^2y}{dx^2} + 2 \frac{dy}{dx} + y = 0 $ ?
		\begin{enumerate}
				\begin{multicols}{4}
				\item $ e^{-x} + xe^{-x} $
				\item $ e^{x} + xe^x $
				\item $ e^x + e^{-x} $
				\item $ e^{-x} + xe^x $
				\end{multicols}
		\end{enumerate}
	\item Suppose the non-constant functions F(x) and G(t) satisfy $\frac{d^2F}{dx^2} + p^2F = 0$, $\frac{dG}{dt}+ c^2p^2G = 0$, where p and c are constants. Then the function $u(x,t) = F(x)G(t)$ definitely satisfies 
		\begin{enumerate}
				\begin{multicols}{2}
				\item $\frac{ \partial ^2u}{\partial t^2} = c^2 \frac{\partial ^u}{\partial x^2} $
				\item $\frac{\partial u}{\partial x} = c^2 \frac{\partial ^2u}{\partial x^2}$
				\item $\triangledown ^2 u = 0$
				\item $ \frac{\partial ^2u}{\partial t^2} + c^2u^2 = 0 $
				\end{multicols}
		\end{enumerate}
	\item The following set of equations $$ \myvec{1 && 1 && 2 \\ 1 && 0 && 1 \\ 0 && 1 && 1} \myvec{x_1 \\ x_2 \\ x_3} = \myvec{1 \\ -1 \\ 0} $$ has
		\begin{enumerate}
				\begin{multicols}{2}
				\item no solution
				\item a unique solution
				\item two solutions
				\item infinite solutions
				\end{multicols}
		\end{enumerate}
	\item The function $ f(x) = x^2 - 5x + 6 $
		\begin{enumerate}
				\item has its maximum value at x=2.0
				\item has its maximum value value at x=2.5
				\item is increasing on the interval (2.0, 2.5)
				\item is increasing on the interval (2.5, 3.0)
		\end{enumerate}
	\item Let Y(s) denote the Laplace Transform L(y(t)) of the function $ y(t) = cosh(at)sin(at) $. Then
		\begin{enumerate}
			\item $L\brak{\frac{dy}{dt}} = \frac{dY}{ds}, L(ty(t)) = sY(s) $
			\item $L\brak{\frac{dy}{dt}} = sY(s) , L(ty(t)) = -\frac{dY}{ds} $
			\item $L\brak{\frac{dy}{dt}} = \frac{dY}{ds}, L(ty(t)) = Y(s-1) $
			\item $L\brak{\frac{dy}{dt}} = \frac{dY}{ds}, L(ty(t)) = e^{as}Y(s) $
		\end{enumerate}
	\item The velocity required for a spacecraft to escape earth's gravitational field depends on
		\begin{enumerate}
			\item the mass of the spacecraft
			\item the distance between eartg's centre and the spacecraft
			\item the earth's rotational speed about its own axis
			\item the earth's orbital speed
		\end{enumerate}
	\item Which of the following statements is TRUE as the altitude increases in the stratosphere of the International Standard Atmosphere ?
		\begin{enumerate}
			\item Temperature increases and the dynamic viscosity decreases
			\item Temperature increases and pressure increases
			\item Temperature decreases and sound speed increases
			\item Temperature decreases and density decreases
		\end{enumerate}
	\item Which of the following statement is TRUE ?
		\begin{enumerate}
			\item Wing dihedral reduces roll stability while a low wing increases roll stability
			\item Wing dihedral increases roll stability while a low wing increases roll stability
			\item Wing dihedral, as well as long wing reduces roll stability
			\item Wing dihedral, as well as long wing increases roll staility.
		\end{enumerate}
	\item The figure below shows the variation of $C_m$ versus $\alpha$ for an aircraft for three combinations of elevator deflections and locationsof centre of gravity. In the figure, lines P and Q are parallel, while lines Q and R have the same intercept on the $C_m$ axis.
		\begin{tikzpicture}
\draw[->] (-1,0) -- (5,0) node[right] {$\alpha$} ;
\draw[->] (0,-1) -- (0,8) node[above] {$C_m$} ;

\draw[thick, blue] (-1,6) -- (3,0) node[above right] {$Q$} ;
\draw[thick, green] (-1,5) -- (4,0) node[above right] {$R$} ;
\draw[thick, red] (-1,3) -- (2,0) node[above right] {$P$} ;
\end{tikzpicture}
		Which of the following is true ?
		\begin{enumerate}
			\item Lines P and Q correspond to the same centre of gravity location.
			\item Lines Q and R correspond to the same centre of gravity location.
			\item Lines P and R correspond to the same elvator deflection.
			\item Lines P and R correspond to the same centre of gravity location.
		\end{enumerate}



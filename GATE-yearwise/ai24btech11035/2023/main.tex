\iffalse
\chapter{2023}
\author{ai24btech11035}
\section{ph}
\fi
\item A rod $PQ$ of proper length $L$ lies along the $x$-axis and moves towards the positive $x$ direction with speed $v = \frac{3c}{5}$ with respect to the ground, where $c$ is the speed of light in vacuum. An observer on the ground measures the positions of $P$ and $Q$ at different times $t_P$ and $t_Q$ respectively in the ground frame, and finds the difference between them to be $\frac{9L}{10}$. What is the value of $t_Q - t_P?$
\begin{figure}[H]
	\centering
	\resizebox{0.2\textwidth}{!}{%
\begin{circuitikz}
\tikzstyle{every node}=[font=\Large]
\draw [short] (5,10.25) -- (5,7.5);
\draw [short] (5,7.5) -- (10,7.5);
\draw [short] (6.25,8.25) -- (6.25,7.5);
\draw [short] (8.25,8.25) -- (8.25,7.5);
\draw [short] (6.25,8.25) -- (8.25,8.25);
\draw [->, >=Stealth] (8.25,8) -- (9.5,8);
\node [font=\Large] at (6.25,8.5) {P};
\node [font=\Large] at (8.25,8.5) {Q};
\node [font=\Large] at (9.75,8) {v};
\node [font=\Large] at (4.75,7.25) {O};
\end{circuitikz}
}%
\end{figure}
\begin{enumerate}
\item $\frac{L}{3c}$
\item $\frac{L}{5c}$
\item $\frac{L}{6c}$
\item $\frac{L}{3c}$
\end{enumerate}
\item A symmetric top has principal moments of inertia $I_1 = I_2 = \frac{2\alpha}{3}$, $I_3 = 2\alpha$ about a set of principal axes 1, 2, 3 respectively, passing through its center of mass, where $\alpha$ is a positive constant. There is no force acting on the body, and the angular speed of the body about the 3-axis is $\omega_3 = \frac{1}{8}$ rad/s. With what angular frequency in rad/s does the angular velocity vector $\vec{\omega}_1$ precess about the 3-axis?
\begin{enumerate}
\item 2
\item 3
\item 5
\item 7
\end{enumerate}
\item A particle of mass $m$ is free to move on a frictionless horizontal two-dimensional $\brak{r, \theta}$ plane and is acted upon by a force $F = -\frac{k}{2r^3} \, \hat{r}$ with $k$ being a positive constant. If $p_r$ and $p_\theta$ are the generalised momenta corresponding to $r$ and $\theta$ respectively, then what is the value of $\frac{dp_r}{dt}?$
\begin{enumerate}
\item $\frac{p_{\theta}^2 - 2mk}{2mr^3}$
\item $\frac{2p_{\theta}^2 - mk}{mr^3}$
\item $\frac{p_{\theta}^2 - 2mk}{mr^3}$
\item $\frac{2p_{\theta}^2 - mk}{2mr^3}$
\end{enumerate}
\item Consider two real functions 
\[
U\brak{x, y} = x y \brak{x^2 - y^2},
\]
\[
V\brak{x, y} = a x^4 + b y^4 + c x^2 y^2 + k,
\]
where $k$ is a real constant and $a$, $b$, $c$ are real coefficients. If $U\brak{x, y} + i V\brak{x, y}$ is analytic, then what is the value of $a \times b \times c?$
\begin{enumerate}
\item $\frac{1}{8}$
\item $\frac{3}{28}$
\item $\frac{5}{36}$
\item $\frac{3}{32}$
\end{enumerate}
\item Young’s double slit experiment is performed using a beam of $C_{60}$ (fullerene) molecules, each molecule being made up of 60 carbon atoms. When the slit separation is 50 nm, fringes are formed on a screen kept at a distance of 1 m from the slits. Now, the experiment is repeated with $C_{70}$ molecules with a slit separation of 92.5 nm. The kinetic energies of both the beams are the same. The position of the $4^{\text{th}}$ bright fringe for $C_{60}$ will correspond to the $n^{\text{th}}$ bright fringe for $C_{70}$. What is the value of $n$ (rounded off to the nearest integer)?
\begin{enumerate}
\item 5
\item 6
\item 7
\item 8
\end{enumerate}
\item A neutron beam with a wave vector $\vec{k}$ and an energy 20.4 meV diffracts from a crystal with an outgoing wave vector $\vec{k}'$. One of the diffraction peaks is observed for the reciprocal lattice vector $\vec{G}$ of magnitude 3.14 $Å^{-1}$. What is the diffraction angle in degrees (rounded off to the nearest integer) that $\vec{k}$ makes with the plane? (Use mass of neutron = $1.67 \times 10^{-27}$ kg)\begin{enumerate}
\item 15 
\item 30
\item 45
\item 60
\end{enumerate}
\item In the first Brillouin zone of a rectangular lattice (lattice constants $a = 6 \, \text{Å}$ and $b = 4 \, \text{Å}$), three incoming phonons with the same wave vector $\langle 1.2 \, \text{Å}^{-1}, 0.6 \, \text{Å}^{-1} \rangle$ interact to give one phonon. Which one of the following is the CORRECT wave vector of the resulting phonon?
\begin{enumerate}
\item $\langle 2.56 \, \text{Å}^{-1}, 0.23 \, \text{Å}^{-1} \rangle$
\item $\langle 3.60 \, \text{Å}^{-1}, 1.80 \, \text{Å}^{-1} \rangle $
\item $\langle 0.48 \, \text{Å}^{-1}, 0.23 \, \text{Å}^{-1} \rangle$
\item $\langle 3.60 \, \text{Å}^{-1}, -0.80 \, \text{Å}^{-1} \rangle$
\end{enumerate}
\item For a covalently bonded solid consisting of ions of mass $m$, the binding potential can be assumed to be given by 
\begin{align*}
	U(r) = -\epsilon \left ( \frac{r}{r_0} \right)e^{-\frac{r}{r_0}},
\end{align*}
where $\epsilon$ and $r_0$ are positive constants. What is the Einstein frequency of the solid in Hz?
\begin{enumerate}
\item $\frac{1}{2\pi} \sqrt{\frac{\epsilon e}{m r_0^2}}$
\item $\frac{1}{2\pi} \sqrt{\frac{\epsilon }{me r_0^2}}$
\item $\frac{1}{2\pi} \sqrt{\frac{\epsilon }{2me r_0^2}}$
\item $\frac{1}{2\pi} \sqrt{\frac{\epsilon e}{2m r_0^2}}$
\end{enumerate}
\item In a hadronic interaction, $\pi^0$s are produced with different momenta, and they immediately decay into two photons with an opening angle $\theta$ between them. Assuming that all these decays occur in one plane, which one of the following figures depicts the behaviour of $\theta$ as a function of the $\pi^0$ momentum $p$?
\begin{enumerate}
\item 
\begin{figure}[H]
	\centering
	\resizebox{0.2\textwidth}{!}{%
\begin{circuitikz}
\tikzstyle{every node}=[font=\large]
\draw [short] (5.25,9.25) -- (5.25,6);
\draw [short] (5.25,6) -- (9.25,6);
\draw [short] (5.25,8.75) .. controls (6.25,7.25) and (6.25,7.25) .. (7.25,6.25);
\node [font=\large] at (5,5.75) {0};
\node [font=\large] at (8,5.75) {p};
\node [font=\large] at (4.75,9.25) {\text{$\theta$}};
\node [font=\large] at (5,8.75) {\text{$\pi$}};
\end{circuitikz}
}%
\end{figure} 
\item 
\begin{figure}[H]
	\centering
	\resizebox{0.2\textwidth}{!}{%
\begin{circuitikz}
\tikzstyle{every node}=[font=\normalsize]
\draw [short] (5.75,10.25) -- (5.75,7.25);
\draw [short] (5.75,7.25) -- (9.5,7.25);
\draw [short] (5.75,7.25) .. controls (6.75,9) and (6.75,9.25) .. (8.5,9.75);
\draw [dashed] (5.75,9.75) -- (8.75,9.75);
\node [font=\normalsize] at (5.75,7) {0};
\node [font=\normalsize] at (8.75,7) {p};
\node [font=\normalsize] at (5.5,10.25) {$\theta$};
\node [font=\normalsize] at (5.5,9.75) {$\pi$};
\end{circuitikz}
}%
\end{figure}
\item \begin{figure}[H]
	\centering
	\resizebox{0.2\textwidth}{!}{%
\begin{circuitikz}
\tikzstyle{every node}=[font=\large]
\draw [short] (5.25,9.25) -- (5.25,6);
\draw [short] (5.25,6) -- (9.25,6);
\node [font=\large] at (5,5.75) {0};
\node [font=\large] at (8,5.75) {p};
\node [font=\large] at (4.75,9.25) {\text{$\theta$}};
\node [font=\large] at (5,8.75) {\text{$\pi$}};
\draw [short] (5.25,8.75) -- (8.75,8.75);
\end{circuitikz}
}%
\end{figure}  
\item \begin{figure}[H]
	\centering
	\resizebox{0.2\textwidth}{!}{%
\begin{circuitikz}
\tikzstyle{every node}=[font=\large]
\draw [short] (5.25,9.25) -- (5.25,6);
\draw [short] (5.25,6) -- (9.25,6);
\node [font=\large] at (5,5.75) {0};
\node [font=\large] at (8,5.75) {p};
\node [font=\large] at (4.75,9.25) {\text{$\theta$}};
\draw [short] (6.25,8) .. controls (5.75,7.25) and (6.25,6.75) .. (5.25,6);
\draw [short] (6.25,8) .. controls (7.5,8.25) and (6.5,7) .. (7.25,6);
\node [font=\large] at (5,8) {\text{$\pi$}};
\draw [dashed] (5.25,8) -- (6.25,8);
\end{circuitikz}
}%
\end{figure} 
\end{enumerate}
\item A particle has wavefunction 
\[
	\psi\brak{x, y, z} = N z e^{-\alpha\brak{x^2 + y^2 + z^2}},
\]
where $N$ is a normalization constant and $\alpha$ is a positive constant. In this state, which one of the following options represents the eigenvalues of $L^2$ and $L_z$ respectively? Some values of $Y_{\ell m}$ are:
\[
Y_{0}^{0} = \sqrt{\frac{1}{4\pi}}, Y_{1}^{0} = \sqrt{\frac{3}{4\pi}} \cos \theta,  Y_{1}^{\pm 1} = \mp \sqrt{\frac{3}{8\pi}} \sin \theta \, e^{\pm i\phi}.
\]
\begin{enumerate}
\item 0 and 0
\item $\hbar^2 \quad \text{and} \quad -\hbar$
\item $2\hbar^2 \quad \text{and} \quad 0$
\item $\hbar^2 \quad \text{and} \quad \hbar$
\end{enumerate}
\item The wavefunction of a particle in one dimension is given by
\[
	\psi\brak{x} = 
\begin{cases} 
M, & -a < x < a \\
0, & \text{otherwise} 
\end{cases}
\]
Here $M$ and $a$ are positive constants. If $\phi\brak{p}$ is the corresponding momentum space wavefunction, which one of the following plots best represents $|\phi\brak{p}|^2$?
\begin{enumerate}
\item \begin{figure}[H]
	\centering
	\resizebox{0.2\textwidth}{!}{%
\begin{circuitikz}
\tikzstyle{every node}=[font=\large]
\draw [short] (5.25,6) -- (9.25,6);
\draw [short] (7.25,8.5) .. controls (8.25,8.5) and (7.25,6.75) .. (9.25,6.25);
\draw [short] (7.25,9.25) -- (7.25,6);
\draw [short] (7.25,8.5) .. controls (6,8) and (7.25,6.5) .. (5.5,6.25);
\node [font=\large] at (9,5.75) {p};
\node [font=\large] at (7,9.5) {$|\phi(p)|^2$};
\end{circuitikz}
}%
\end{figure} 
\item \begin{figure}[H]
	\centering
	\resizebox{0.2\textwidth}{!}{%
\begin{circuitikz}
\tikzstyle{every node}=[font=\large]
\draw [short] (5.25,6) -- (9.25,6);
\draw [short] (7.25,9.25) -- (7.25,6);
\node [font=\large] at (9,5.75) {p};
\node [font=\large] at (7,9.5) {$|\phi(p)|^2$};
\draw [short] (7.25,7) -- (8.75,7);
\draw [short] (7.25,7) -- (5.75,7);
\draw [short] (8.75,7) -- (8.75,6);
\draw [short] (5.75,7) -- (5.75,6);
\end{circuitikz}
}%
\end{figure} 
\item \begin{figure}[H]
	\centering
	\resizebox{0.2\textwidth}{!}{%
\begin{circuitikz}
\tikzstyle{every node}=[font=\normalsize]
\draw [short] (50.25,-96.25) -- (50.25,-99);
\draw [short] (48.5,-99) -- (52,-99);
\draw [short] (50.25,-97) .. controls (50,-98.25) and (50,-98.25) .. (48.75,-98.75);
\draw [short] (50.25,-97) .. controls (50.5,-98.75) and (50.75,-98.25) .. (51.75,-98.75);
\node [font=\normalsize] at (51.75,-99.25) {p};
	\node [font=\normalsize] at (50.25,-96) {$|\phi(p)|^2$};
\end{circuitikz}
}%
\end{figure} 
\item \begin{figure}[H]
	\centering
	\resizebox{0.2\textwidth}{!}{%
\begin{circuitikz}
\tikzstyle{every node}=[font=\normalsize]
\draw [short] (7.75,10.75) -- (7.75,7.25);
\draw [short] (5.5,7.25) -- (10,7.25);
\draw [short] (7.75,10.25) .. controls (8.5,9.5) and (8.25,8.75) .. (8.5,7.25);
\draw [short] (7.75,10.25) .. controls (7.25,9.25) and (7.5,8.75) .. (7,7.25);
\draw [short] (8.5,7.25) .. controls (9,7.5) and (8.75,7.5) .. (9.5,7.25);
\draw [short] (6.25,7.25) .. controls (6.5,7.5) and (7,7.5) .. (7,7.25);
\node [font=\normalsize] at (9.25,7) {p};
	\node [font=\normalsize] at (7.5,11) {$|\phi(p)|^2$};
\draw [short] (9.5,7.25) .. controls (10,7.5) and (10,7.25) .. (10.25,7.25);
\draw [short] (5.75,7.25) -- (6.25,7.25);
\end{circuitikz}
}%
\end{figure} 
\end{enumerate}
\item Consider a particle in a two-dimensional infinite square well potential of side $L$, with $0 \leq x \leq L$ and $0 \leq y \leq L$. The wavefunction of the particle is zero only along the line $y = \frac{L}{2}$, apart from the boundaries of the well. If the energy of the particle in this state is $E$, what is the energy of the ground state?
\begin{enumerate}
\item $\frac{1}{4} E$
\item $\frac{2}{5} E$
\item $\frac{3}{8} E$
\item $\frac{1}{2} E$
\end{enumerate}
\item Consider two non-identical spin-$\frac{1}{2}$ particles labelled 1 and 2 in the spin product state $|\frac{1}{2}, \frac{1}{2}\rangle |\frac{1}{2}, -\frac{1}{2}\rangle$. The Hamiltonian of the system is 
\[
H = \frac{4\lambda}{\hbar^2} \mathbf{S}_1 \cdot \mathbf{S}_2,
\]
where $\mathbf{S}_1$ and $\mathbf{S}_2$ are the spin operators of particles 1 and 2, respectively, and $\lambda$ is a constant with appropriate dimensions. What is the expectation value of $H$ in the above state?
\begin{enumerate}
\item $-\lambda$
\item $-2\lambda$
\item $\lambda$
\item $2\lambda$
\end{enumerate}

\iffalse
\chapter{2009}
\author{ai24btech11035}
\section{me}
\fi
	\item A forged steel link with a uniform diameter of 30 mm at the centre is subjected to an axial force that varies from 40 kN in compression to 160 kN in tension. The tensile strength $\brak{S_u}$, yield strength $\brak{S_y}$, and corrected endurance strength $\brak{S_e}$ of the steel material are 600 MPa, 420 MPa, and 240 MPa, respectively.The factor of safety against fatigue endurance as per Soderberg's criterion is
\begin{enumerate}
\begin{multicols}{2}
\item 1.26
\item 1.37
\item 1.45
\item 2.00
\end{multicols}
\end{enumerate}
\item An automatic engine weighing 240kg is supported on four springs with linear characteristics.Each of the two front springs have a stiffness of 16 MN/m while the stiffness of each rear spring is 32 MN/m.The engine speed(in rpm),at which resonance is likely to occur,is
\begin{enumerate}
\begin{multicols}{4}
\item 6040
\item 3020
\item 1424
\item 955
\end{multicols}
\end{enumerate}
\item A vehicle suspension system consists of a spring and a damper.The stiffness of the spring is 3.6 kN/m and the damping constant of the damper is 400 Ns/m.If the mass is 50kg,then the damping factor $\brak{d}$ and damping natural frequency $\brak{f_n}$, respectively,are
\begin{enumerate}
\begin{multicols}{2}
\item  0.471 and 1.19 Hz
\item  0.471 and 7.48 Hz
\item  0.666 and 1.35 Hz
\item  0.666 and 8.50 Hz
\end{multicols}
\end{enumerate}
\item A frame of two arms of equal length $L$ is shown in adjacent figure.The flexual rigidity of each arm of the frame is $El$.The vertical deflection at the point of application of load $P$ is
	\begin{figure}[H]
		\centering
		\resizebox{0.5\textwidth}{!}{%
\begin{circuitikz}
\tikzstyle{every node}=[font=\normalsize]
\draw [short] (4,11.25) -- (4,7.25);
\draw [short] (3.75,7.25) .. controls (3.75,7.75) and (4.25,7.75) .. (4.25,7.25);
\draw [short] (2.75,7.25) -- (8.25,7.25);
\draw [short] (3.75,7.25) -- (5.75,7.25);
\draw [short] (3.75,7.25) -- (3.75,7);
\draw [short] (4.25,7.25) -- (4.25,7);
\draw [short] (3.75,7) -- (4.25,7);
\draw [short] (4,7.25) -- (4,6.25);
\draw [short] (4,7) -- (3.75,6.75);
\draw [short] (3.75,7) -- (3.5,6.75);
\draw [short] (4.25,7) -- (4,6.75);
\draw [short] (8.25,7.25) -- (8.25,6.25);
\draw [<->, >=Stealth] (4,6.5) -- (8.25,6.5);
\draw [->, >=Stealth] (8.25,9.25) -- (8.25,7.25);
\draw [short] (4,11.25) -- (2.5,11.25);
\draw [short] (4.5,11.5) .. controls (4,11.5) and (3.75,11.25) .. (4.5,10.75);
\draw [short] (4.5,11.5) -- (4.5,10.75);
\draw [short] (4.5,11.5) -- (4.5,11.25);
\draw [short] (4.5,11.5) -- (4.75,11.25);
\draw [short] (4.5,11.25) -- (4.75,11);
\draw [short] (4.5,11) -- (4.75,10.75);
\draw [<->, >=Stealth] (3.25,11.25) -- (3.25,7.25);
\node [font=\normalsize] at (6,6.75) {L};
\node [font=\normalsize] at (3,9.25) {L};
\node [font=\normalsize] at (8.5,9.25) {P};
\end{circuitikz}
}%
	\end{figure}

\begin{enumerate}
\begin{multicols}{4}
\item $\frac{PL^3}{3El}$
\item $\frac{2PL^3}{3El}$
\item $\frac{PL^3}{El}$
\item $\frac{4PL^3}{3El}$
\end{multicols}
\end{enumerate}
\item A uniform rigid rod of mass $M$ and length $L$ is hinged at one end as shown in the adjacent figure.A force $P$ is applied at a distance of $\frac{2L}{3}$ from the hinge so that the rod swings to the right.The reaction at the hinge is
	\begin{figure}[H]
		\centering
		\resizebox{0.5\textwidth}{!}{%
\begin{circuitikz}
\tikzstyle{every node}=[font=\Huge]
\draw [short] (3.75,11.5) -- (3.75,6.25);
\draw [short] (4,11.5) -- (4,6.25);
\draw [short] (3.75,6.25) -- (6,6.25);
\draw [short] (4,11.5) -- (3.75,11.5);
\draw [->, >=Stealth] (2.25,7.75) -- (3.75,7.75);
\draw [<->, >=Stealth] (4.75,11.25) -- (4.75,7.75);
\draw [<->, >=Stealth] (5.75,11.25) -- (5.75,6.25);
\draw [short] (4,7.75) -- (4.75,7.75);
\draw [short] (3.25,11.75) .. controls (3.75,10.75) and (4.25,10.75) .. (4.25,11.75);
\draw [short] (3.25,11.75) -- (4.25,11.75);
\draw [short] (3.5,12) -- (3.25,11.75);
\node [font=\normalsize] at (3,8) {P};
\node [font=\normalsize] at (6,8.5) {L};
\node [font=\normalsize] at (5.25,9.25) {2L/3};
\draw [short] (3.75,12) -- (3.5,11.75);
\draw [short] (4,12) -- (3.75,11.75);
\draw [short] (4.25,12) -- (4,11.75);
\draw [short] (3.75,11.25) -- (5.75,11.25);
\end{circuitikz}
}%
	\end{figure}
\begin{enumerate}
\begin{multicols}{4}
\item $-P$
\item 0
\item $\frac{P}{3}$
\item $\frac{2P}{3}$                         
\end{multicols}
\end{enumerate}
\item Match the approaches given below to perform stated kinematics / dynamics analysis of machine.
\begin{table}[h!]
        \centering
  \begin{tabular}{|c|c|}                                      \hline \textbf{Analysis} & \textbf{Approach} \\ \hline P. Continuous relative rotation &
     1. D'Alembert's principle \\ \hline Q. Velocity and
     acceleration & 2. Grubler's criterion \\ \hline R.
    Mobility & 3. Grashoff's law \\ \hline S. Dynamic-st
    atic analysis & 4. Kennedy's theorem \\ \hline
\end{tabular}
\end{table}
\begin{enumerate}
\begin{multicols}{2}
\item P-1,Q-2,R-3,S-4
\item P-3,Q-4,R-2,S-1
\item P-2,Q-3,R-4,S-1
\item P-4,Q-2,R-1,S-3
\end{multicols}
\end{enumerate}
\item A company uses 2555 units of an item anually.Delivery lead time is 8 days.The reorder point (in number of units) to achieve optimum inventory is 
\begin{enumerate}
\begin{multicols}{4}
\item 7
\item 8
\item 56
\item 60
\end{multicols}
\end{enumerate}
\item Consider the following Linear Programming Problem \brak{LPP} :
	\begin{align*}
		\text{Maximize} &\ z = 3x_1+2x_2 \\
		\text{Subject to} &\ x_1 \le 4 \\
		&\ x_2 \le 6 \\
		&\  3x_1+2x_2 \le 18 \\
		&\ x_1 \ge 0, x_2 \ge 0  
	\end{align*}
\begin{enumerate}
\item The LLP has a unique optimal solution 
\item The LLP is infeasible.
\item The LLP is unbounded.
\item The LLP has multiple optimal solutions.
\end{enumerate}
\item Six jobs arrived in a sequence as given below.
\begin{table}[h!]
        \centering
\begin{tabular}{|c|c|}

                                     \hline \textbf{
Jobs} & \textbf{Processing Time (days)} \\ \hline I &
 4 \\     \hline II & 9 \\ \hline III & 5 \\ \hline I
V & 10 \\ \hline V & 6 \\ \hline VI & 8 \\ \hline    
              
\end{tabular}
\end{table}
Average flow time (in days) for the above jobs using Shortest Processing Time rule is 
\begin{enumerate}
\begin{multicols}{4}
\item 20.83
\item 23.16
\item 125.00
\item 139.00
\end{multicols}
\end{enumerate}
\item Minimum shear strain in orthogonal turning with a cutting tool of zero rake angle is  
\begin{enumerate}
\begin{multicols}{4}
\item 0.0
\item 0.5
\item 1.0
\item 2.0
\end{multicols}
\end{enumerate}
\item Electrochemical machining is performed to remove material from an iron surface of $20 \, \text{mm} \times 20 \, \text{mm}$ under the following conditions:
	\begin{align*}
		\text{Inter electrode gap} &= 0.2 \, \text{mm} \\
		\text{Supply voltage} \brak{DC} & =12V \\
		\text{Specific resistance of electrolyte} &= 2 \, \Omega \\
		\text{Atomic weight of Iron} &= 55.85 \\
		\text{Valency of Iron} &=2 \\
		\text{Faraday's constant} &= 96540 \,\text{Coulombs}
	\end{align*}
		The material removal rate (in g/s) is
\begin{enumerate}
\begin{multicols}{4}
\item 0.3471
\item 3.471
\item 34.71
\item 347.1
\end{multicols}
\end{enumerate}
\item Match the following:
\begin{table}[h!]
        \centering
  \begin{tabular}{|c|c|}                             
        \hline \textbf{NC Code} & \textbf{Definition}
 \\ \hline P. M05 & 1. Absolute coordinate system
 \\ \hline Q. G01 & 2. Dwell \\ \hline
     R. G04 & 3. Spindle stop \\ \hline S. G90 & 4. L
ine    ar interpolation \\ \hline                    

                    

\end{tabular}   

\end{table}	
	
\begin{enumerate}
\begin{multicols}{2}
\item  P-2,Q-3,R-4,S-1
\item  P-3,Q-4,R-1,S-2
\item  P-3,Q-4,R-2,S-1
\item  P-4,Q-3,R-2,S-1
\end{multicols}
\end{enumerate}


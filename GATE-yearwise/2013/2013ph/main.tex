\iffalse
\section{ph}
\chapter{2013}
\author{ee24btech11017}
\fi

    \item For the $O^{17}$ nucleus $\brak{A = 17, Z=8}$, the effective magnetic moment is given by, 
    
    \begin{center}
        
    
       $ \Vec{\overrightarrow{\mu}{_{eff}}} = \frac{eh}{2Mc}g\Vec{\overrightarrow{J}}$
    \end{center}
    where $g$ is equal to, $\brak{g_s = 5.59 \text{ for proton and} -3.83 \text{ for neutron}}$
    \begin{multicols}{4}
    \begin{enumerate}
    
        \item $1.12$ 
        \item $-0.77$ 
        \item $-1.28$ 
        \item $1.28$ 
    \end{enumerate}
    \end{multicols}
      Consider the following circuit:
    \begin{figure}[!ht]
\centering
\resizebox{6cm}{!}{%
\begin{circuitikz}
\tikzstyle{every node}=[font=\large]
\draw (5.5,12.75) to[short, -o] (4.75,12.75) ;
\draw [ line width=0.5pt](5.5,12.75) to[R] (7.75,12.75);
\draw [ line width=0.5pt](7.75,12.75) to[short] (9.5,12.75);
\draw [line width=0.5pt](8,12.75) to[C] (8,10.75);
\draw [line width=0.5pt](8,10.75) to (8,10.25) node[ground]{};
\draw [ line width=0.5pt](11,12.25) node[op amp,scale=1] (opamp2) {};
\draw [ line width=0.5pt](opamp2.+) to[short] (9.5,11.75);
\draw [ line width=0.5pt] (opamp2.-) to[short] (9.5,12.75);
\draw [ line width=0.5pt](12.2,12.25) to[short](12.5,12.25);
\draw [ line width=0.5pt](9.5,11.75) to[short] (9.5,9);
\draw [ line width=0.5pt](12.5,12.25) to[R] (12.5,9);
\draw [ line width=0.5pt](9.5,9) to[short] (12.5,9);
\draw [ line width=0.5pt](12.5,9) to[R] (12.5,7.25);
\draw [line width=0.5pt](12.5,7.25) to (12.5,7) node[ground]{};
\draw [ line width=0.5pt](12.5,12.25) to[short, -o] (14.25,12.25) ;
\node [font=\large] at (4,12.75) {\textbf{V\brak{in}}};
\node [font=\large] at (6.5,13.5) {\textbf{10k$\Omega$}};
\node [font=\large] at (6.5,11.5) {\textbf{1000pF}};
\node [font=\large] at (15,12.5) {\textbf{V\brak{out}}};
\node [font=\large] at (13.25,10.75) {\textbf{1k$\Omega$}};
\node [font=\large] at (13.25,8) {\textbf{2k$\Omega$}};
\end{circuitikz}
}%

\label{fig:my_label}
\end{figure}
    \item  For this circuit the frequency above which the gain will decrease by $20$ $dB$ per decade is 
    \begin{multicols}{4}
    \begin{enumerate}
        \item $15.9$ kHz
        \item $1.2$ kHz
        \item $5.6$ kHz
        \item $22.5$ kHz
    \end{enumerate}
    \end{multicols}
    \item 
    At $1.2$kHz the closed loop gain is 
    \begin{multicols}{4}
        \begin{enumerate}
            \item $1$
            \item $1.5$
            \item $3$
            \item $0.5$
        \end{enumerate}
    \end{multicols}
  
      \textbf{General Aptitude \brak{\text{GA}} Questions}\\
    \textbf{Q.56-Q.60 carry one mark each.}
    \item A number is as much greater than $75$ as it is smaller than $117$. The number is:
    \begin{multicols}{4}
    \begin{enumerate}
        \item $91$
        \item $93$
        \item $89$
        \item $96$\\
    \end{enumerate}
        
    \end{multicols}
\item $\underset{\text{I}}{\text{\underline{The professor}}}$ $\underset{\text{II}}{\text{\underline{ordered to}}}$ $\underset{\text{III}}{\text{\underline{the students to go}}}$ $\underset{\text{IV}}{\text{\underline{out of the class.}}}$

\begin{multicols}{4}
\begin{enumerate}
    \item I
    \item II
    \item III
    \item IV
\end{enumerate}
    
\end{multicols}
\item Which of the following options is the closest in meaning to the word given below:\\
Primeval
\begin{multicols}{2}
\begin{enumerate}
        \item Modern
        \item Historic
        \item Primitive
        \item Antique
    \end{enumerate}
\end{multicols}
\item Friendship,no matter how \rule{1.7cm}{0.2mm} it is, has its limitations.
\begin{enumerate}
    \item cordial
    \item intimate
    \item secret
    \item pleasant
\end{enumerate}
\item Select the pair that best expresses a relationship similar to that expressed in the pair:\\
\textbf{Medicine: Health}
\begin{multicols}{2}
\begin{enumerate}
    \item Science:Experiment
    \item Wealth:Peace
    \item Education:Knowledge
    \item Money:Happiness
\end{enumerate}
\end{multicols}
\textbf{Q.61 to Q.65 carry two marks each.}
\item $X$ and $Y$ are two positive real numbers such that $2X + Y \leq 6$ and $X + 2Y \leq8.$ For which of the following values of $\brak{X,Y}$ the function $f\brak{X,Y} = 3X + 6Y$ will give maximum value?
\begin{enumerate}
    \item $\brak{\frac{4}{3},\frac{10}{3}}$
    \item $\brak{\frac{8}{3},\frac{20}{3}}$
    \item $\brak{\frac{8}{3},\frac{10}{3}}$
    \item $\brak{\frac{4}{3},\frac{20}{3}}$
\end{enumerate}
\item If $\abs{4X - 7} = 5$ then the values of $2\abs{X} - \abs{-X}$ is:
\begin{multicols}{4}
\begin{enumerate}
    \item $2,\frac{1}{3}$
    \item $\frac{1}{2},3$
    \item $\frac{3}{2},9$
    \item $\frac{2}{3},9$
\end{enumerate}
\end{multicols}
\item Following table provides figures $\brak{\text{in rupees}}$ on annual expenditure od a firm for two years$-2010 \text{ and } 2011$.\\

    \begin{center}
    \begin{tabular}{|l|r|r|}
        \hline
        \textbf{Category} & \textbf{2010 } & \textbf{2011 } \\
        \hline
        Raw material & 5200 & 6240 \\
        Power \& fuel & 7000 & 9450 \\
        Salary \& wages & 9000 & 12600 \\
        Plant \& machinery & 20000 & 25000 \\
        Advertising & 15000 & 19500 \\
        Research \& Development & 22000 & 26400 \\
        \hline
    \end{tabular}
\end{center}
In $2011$, which of the following two categories have registered increase by same percentage?
\begin{enumerate}
    \item Raw material and Salary $\&$ wages
    \item Salary $\&$ wages and Advertising
    \item Power $\&$ fuel and Advertising
    \item Raw material and Research $\&$ Development
\end{enumerate}
\item A firm is selling its product at $RS. 60$ per unit. The total cost of production is $Rs. 100$ and firm is earning total profit of $Rs. 500.$ Later, the total cost increased by $30\%$. By what percentage the price should be increased to maintained the same profit level.
\begin{multicols}{4}
    \begin{enumerate}
        \item $5$
        \item $10$
        \item $15$
        \item $30$
    \end{enumerate}
\end{multicols}
\item Abhishek is elder to Savar.\\
Savar is younger to Anshul.\\
Which of the given conclusions is logically valid and is inferred from the above statements?
\begin{enumerate}
    \item Abhishek is elder to Anshul
    \item  Anshul is elder to Abhishek
    \item Abhishek and Anshul are of the same age
    \item No conclusion follows
\end{enumerate}


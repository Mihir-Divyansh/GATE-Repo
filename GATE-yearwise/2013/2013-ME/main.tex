\iffalse
\chapter{2013}
\author{AI24BTECH11009}
\section{me}
\fi

\item The partial differential equation $\frac{\partial u}{\partial t} + u\frac{\partial u}{\partial x} = \frac{\partial^2 u}{\partial x^2}$ is a
    \begin{enumerate}
      \item linear equation of order 2 
      \item non-linear equation of order 1
      \item linear equation of order 1
      \item non-linear equation of order 2  \\
    \end{enumerate}
\item The eigenvalues of a symmetric matrix are all  
\begin{enumerate}
    \item complex with non-zero positive imaginary part. 
    \item complex with non-zero negative imaginary part.
    \item real. 
    \item pure imaginary. \\
\end{enumerate}
\item Match the \textbf{CORRECT} pairs.
\begin{table}[h!]
  \centering
  \begin{tabular}[12pt]{ |c| c|}
    \hline
    \textbf{Numerical Integration Scheme} & \textbf{Order of Fitting Polynomial} \\ 
    \hline
    P. Simpson's $\frac{3}{8}$ Rule  & 1. First \\
    \hline 
    Q. Trapezoidal Rule & 2. Second \\
    \hline
    R. Simpson's $\frac{1}{3}$ Rule  & 3. Third \\
    \hline
    \end{tabular}

\end{table}
\begin{enumerate}
    \item P-2, Q-1, R-3 
    \item P-3, Q-2, R-1
    \item P-1, Q-2, R-3 
    \item P-3, Q-1, R-2  \\
\end{enumerate}
\item A rod of length $L$ having uniform cross-sectional area $A$ is subjected to a tensile force $P$ as shown in the figure below. If the Young's modulus of the material varies linearly from $E_1$ to $E_2$ along the length of the rod, the normal stress developed at the section-SS is 
\begin{figure}[!ht]
\centering
\resizebox{0.5\textwidth}{!}{%
\begin{circuitikz}
\tikzstyle{every node}=[font=\normalsize]
\draw  (4.75,11) rectangle (10.25,10.25);
\draw [->, >=Stealth] (10.25,10.5) -- (11.25,10.5);
\draw [->, >=Stealth] (4.75,10.5) -- (3.5,10.5);
\draw (7.25,11.75) to[short] (7.25,9.5);
\draw [->, >=Stealth] (7.25,11.75) -- (8.5,11.75);
\draw [->, >=Stealth] (7.25,9.5) -- (8.5,9.5);
\draw (4.75,9) to[short] (4.75,8);
\draw (7.25,9) to[short] (7.25,8.5);
\draw (10.25,8.5) to[short] (10.25,8);
\draw [<->, >=Stealth] (4.75,8.75) -- (7.25,8.75);
\draw [<->, >=Stealth] (4.75,8.25) -- (10.25,8.25);
\node [font=\normalsize] at (8.75,11.75) {S};
\node [font=\normalsize] at (8.75,9.5) {S};
\node [font=\normalsize] at (6,9) {$L/2$};
\node [font=\normalsize] at (7.25,8) {$L$};
\node [font=\normalsize] at (11.5,10.5) {$P$};
\node [font=\normalsize] at (3.25,10.5) {$P$};
\node [font=\normalsize] at (4.75,11.25) {$E_1$};
\node [font=\normalsize] at (10.25,11.25) {$E_2$};
\end{circuitikz}

}%
\end{figure}
 \begin{enumerate}
     \item $\frac{P}{A}$
     \item $\frac{P\brak{E_1-E_2}}{A\brak{E_1+_2}}$
     \item $\frac{PE_2}{AE_1}$
     \item $\frac{PE_1}{AE_2}$ \\
 \end{enumerate}
\item Two threaded bolts A and B of same material and length are subjected to identical tensile load. If the elastic strain energy stored in bolt A is 4 times that of bolt B and the mean diameter of bolt A is 12 $mm$, the mean diameter of bolt B in $mm$ is 
\begin{enumerate}
   \item 16
   \item 24
   \item 36
   \item 48 \\
\end{enumerate}
\item A link OB is rotating with a constant angular velocity of 2 $rad/s$ in counter clockwise direction and a block is sliding radially outward on it with an uniform velocity of 0.75 $m/s$ with respect to the rod, as shown in the figure below. If OA = 1 $m$, the magnitude of the absolute acceleration of the
block at location A in $m/s^2$ is 
\begin{figure}[!ht]
\centering
\resizebox{0.25\textwidth}{!}{%
\begin{circuitikz}
\tikzstyle{every node}=[font=\normalsize]
\draw  (6.25,10) rectangle (7.75,9.75);
\draw [short] (6.5,10) .. controls (6.75,11) and (7.25,10.75) .. (7.5,10);
\draw  (7,10.25) circle (0.25cm);
\draw  (8.75,12.5) -- (9.125,12.125) -- (9.5,12.5) -- (9.125,12.875) -- cycle;
\draw [short] (7,10.25) -- (9,12.25);
\draw [short] (9.25,12.5) -- (10.25,13.5);
\draw [->, >=Stealth] (9.25,11.75) -- (9.75,12.25);
\draw [->, >=Stealth] (8,10.75) .. controls (8.25,11.75) and (8,11.5) .. (7.25,11.5) ;
\node [font=\normalsize] at (6.5,10.75) {O};
\node [font=\normalsize] at (8.75,12.75) {A};
\node [font=\normalsize] at (10,13.75) {B};
\end{circuitikz}

}%
\end{figure}
\begin{enumerate}
    \item 3
    \item 4
    \item 5
    \item 6 \\
\end{enumerate}
\item For steady, fully developed flow inside a straight pipe of diameter $D$, neglecting gravity effects, the pressure drop $\Delta p$ over a length $L$ and the wall shear stress $\tau_w$ are related by 
\begin{enumerate}
    \item $\tau_w = \frac{\Delta pD}{4L}$
    \item $\tau_w = \frac{\Delta pD^2}{4L^2}$
    \item $\tau_w = \frac{\Delta pD}{2L}$
    \item $\tau_w = \frac{4\Delta pL}{D}$ \\
 \end{enumerate}
\item The pressure, dry bulb temperature and relative humidity of air in a room are 1 $bar$, 30\degree C and 70\%, respectively. If the saturated steam pressure at 30\degree C is 4.25 $kPa$, the specific humidity of the room air in $kg\ water\ vapour/kg\ dry\ air$ is 
 \begin{enumerate}
    \item 0.0083
    \item 0.0101 
    \item 0.0191 
    \item 0.0232 \\
 \end{enumerate}
\item Consider one-dimensional steady state heat conduction, without heat generation, in a plane wall; with boundary conditions as shown in the figure below. The conductivity of the wall is given by $k = k_0 + bT$; where $k_0$ and $b$ are positive constants, and $T$ is temperature. 
\begin{figure}[!ht]
\centering
\resizebox{0.4\textwidth}{!}{%
\begin{circuitikz}
\tikzstyle{every node}=[font=\normalsize]
\draw [short] (5.75,11.5) -- (5.75,9.75);
\draw [short] (7,11.5) -- (7,9.75);
\draw [short] (5.75,9.75) -- (5.75,9.25);
\draw [->, >=Stealth] (5.75,9.5) -- (6.25,9.5);
\draw [short] (5.75,11.5) .. controls (6.25,11.75) and (6.25,11.75) .. (6.75,11.5);
\draw [short] (6.75,11.5) .. controls (6.75,11.75) and (7,11.75) .. (7,11.5);
\draw [short] (5.75,9.75) .. controls (6.5,9.75) and (6.25,9.5) .. (6.75,9.75);
\draw [short] (6.75,9.75) -- (7,9.75);
\draw [->, >=Stealth] (5.25,10) -- (5.75,10.25);
\draw [->, >=Stealth] (7.5,10.75) -- (7,11);
\node [font=\normalsize] at (6.5,9.5) {$x$};
\node [font=\normalsize] at (5,10) {$T_1$};
\node [font=\normalsize] at (7.75,10.75) {$T_2$};
\node [font=\normalsize] at (8.75,11) {where $T_2>T_1$};
\end{circuitikz}

}%
\end{figure}
As $x$ increases, the temperature gradient $\brak{\frac{dT}{dx}}$ will
\begin{enumerate}
     \item remain constant
     \item be zero
     \item increase
     \item decrease \\
 \end{enumerate}
\item In a rolling process, the state of stress of the material undergoing deformation is 
\begin{enumerate}
    \item pure compression
    \item pure shear
    \item compression and shear 
    \item tension and shear  \\
\end{enumerate}
\item Match the \textbf{CORRECT} pairs.
\begin{table}[h!]
  \centering
  \begin{tabular}[12pt]{ |c| c|}
    \hline
    \textbf{Processes} & \textbf{Characteristics / Applications} \\ 
    \hline
    P. Friction Welding & 1. Non-consumable electrode  \\
    \hline 
    Q. Gas Metal Arc Welding & 2. Joining of thick plates \\
    \hline
    R. Tungsten Inert Gas Welding & 3. Consumable electrode wire \\
    \hline
    S. Electroslag Welding & 4. Joining of cylindrical dissimilar materials \\
    \hline
    \end{tabular}

\end{table}
\begin{enumerate}
    \item P-4, Q-3, R-1, S-2
    \item P-4, Q-2, R-3, S-1
    \item P-2, Q-3, R-4, S-1 
    \item P-2, Q-4, R-1, S-3 \\
\end{enumerate}
\item A metric thread of pitch 2 $mm$ and thread angle 60\degree is inspected for its pitch diameter using 3-wire method. The diameter of the best size wire in $mm$ is 
  \begin{enumerate}
   \item 0.866
   \item 1.000
   \item 1.154
   \item 2.000 \\
\end{enumerate}
\item Customers arrive at a ticket counter at a rate of 50 per $hr$ and tickets are issued in the order of their arrival. The average time taken for issuing a ticket is 1 $min$. Assuming that customer arrivals form a Poisson process and service times are exponentially distributed, the average waiting time in queue in $min$ is
\begin{enumerate}
    \item 3
    \item 4
    \item 5
    \item 6 \\
\end{enumerate}

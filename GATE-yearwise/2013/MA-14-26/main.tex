\iffalse
\title{GATE Questions 9}
\author{EE24BTECH11012 - Bhavanisankar G S}
\section{ma}
\chapter{2013}
\fi
%\begin{enumerate}
	\item Let $u(x,t)$ be the solution to the wave equation
		$$ \frac{\partial ^2 u}{\partial x^2} \brak{x,t} = \frac{\partial ^2 u}{\partial t^2} \brak{x,t} ; u(x,0) = \cos{5 \pi x} ; \frac{\partial u}{\partial t} = 0 $$
		Then the value of $u(1,1)$ is
	\item Let $f(x) = \sum_{n=1}^{\infty} \frac{\sin{nx}}{n^2}$ Then
		\begin{enumerate}
				\begin{multicols}{2}
				\item $\lim_{x \to 0} f(x) = 0$
				\item $\lim_{x \to 0} f(x) = 1$
				\item $\lim_{x \to 0} f(x) = \frac{\pi ^2}{6}$
				\item $\lim_{x \to 0} f(x)$ does not exist
				\end{multicols}
		\end{enumerate}
	\item Suppose X is a random variable with $P(X=k) = (1-p)^kp$ for $k \in \cbrak{0,1,2,\dots}$ and for some $p \in \brak{0,1}$. For the hypothesis testing problem
		$$ H_0 : p = \frac{1}{2} H_1 : p \neq \frac{1}{2} $$
		Consider the test "Reject $H_0$ if $X \leq A$ or if $X \geq B$", where A $\leq$ B are given positive integers. The type-1 error of this test is
		\begin{enumerate}
				\begin{multicols}{2}
				\item $1 + 2^{-B} + 2^{-A}$
				\item $1 - 2^{-B} + 2^{-A}$
				\item $1 + 2^{-B} - 2^{-A-1}$
				\item $1 - 2^{-B} + 2^{-A-1}$
				\end{multicols}
		\end{enumerate}
	\item Let G be a group of order 231. The numberof elements of order 11 in G is
	\item Let $f:\mathbb{R}^2 \to \mathbb{R}^2$ be defined by $f(x,y) = \brak{e^{x+y}, e^{x-y}}$. The area of the image of the region $\cbrak{(x,y) \in \mathbb{R} : 0 \leq x,y \leq 1}$ under the mapping $f$ is
		\begin{enumerate}
				\begin{multicols}{4}
				\item 1
				\item $e-1$
				\item $e^2$
				\item $e^2 - 1$
				\end{multicols}
		\end{enumerate}
	\item Which of the following is a field ?
		\begin{enumerate}
				\begin{multicols}{4}
				\item $\frac{\mathbb{C}(x)}{x^2 + 2}$
				\item $\frac{\mathbb{C}(x)}{x^2 + 2}$
				\item $\frac{\mathbb{C}(x)}{x^2 - 2}$
				\item $\frac{\mathbb{C}(x)}{x^2 - 2}$
				\end{multicols}
		\end{enumerate}
	\item Let $x_0 = 0$ Define $x_{n+1} = \cos{x_n}$ for every $n \geq 0$. Then
		\begin{enumerate}
				\begin{multicols}{2}
				\item $\cbrak{x_{n}}$ is increasing and convergent
				\item $\cbrak{x_n}$ is decreasing and convergent
				\item $\cbrak{x_n}$ is convergent and $x_{2n} \leq \lim_{m \to \infty} \leq x_{2n+1}$ for every $n \in \mathbb{N}$
				\item $\cbrak{x_n}$ is not convergent
				\end{multicols}
		\end{enumerate}
	\item Let C be the contour $\abs{z} = 2$ oriented in the anti-clockwise direction. The value of the integral $\int z e^{\frac{3}{z}} dz$ is
		\begin{enumerate}
				\begin{multicols}{4}
				\item $3 \pi i$
				\item $5 \pi i$
				\item $7 \pi i$
				\item $9 \pi i$
				\end{multicols}
		\end{enumerate}
	\item For each $\lambda \geq 0$, let $X_{\lambda}$ be a random variable with exponential density $\lambda e^{-\lambda x}$ on $\brak{0,\infty}$ Then $Var(log X_{\lambda})$
		\begin{enumerate}
				\begin{multicols}{2}
				\item is strictly increasing in $\lambda$
				\item is strictly decreasing in $\lambda$
				\item does not depend on $\lambda$
				\item first increases and the decreases in $\lambda$
				\end{multicols}
		\end{enumerate}
	\item Let $\cbrak{a_n}$ be the sequence of consecutive positive solutions of the equation $\tan{x} = x$ and let $\cbrak{b_n}$ be the sequence of consecutive positive solutions of the equation $\tan{\sqrt{x}} = x$. Then
		\begin{enumerate}
				\item $\sum_{n=1}^{\infty} \frac{1}{a_n}$ converges but $\sum_{n=1}^{\infty} \frac{1}{b_n}$ diverges
				\item $\sum_{n=1}^{\infty} \frac{1}{a_n}$ diverges but $\sum_{n=1}^{\infty} \frac{1}{b_n}$ converges
				\item Both $\sum_{n=1}^{\infty} \frac{1}{a_n}$ and $\sum_{n=1}^{\infty} \frac{1}{b_n}$ converges
				\item Both  $\sum_{n=1}^{\infty} \frac{1}{a_n}$ and $\sum_{n=1}^{\infty} \frac{1}{b_n}$ diverges
		\end{enumerate}
	\item Letf be an analytical function on $\overline{D} = \cbrak{z \in \mathbb{C} : \abs{z} \leq 1}$ Assume that $\abs{f(z)} \leq 1$ for each $z \in \overline{D}$. Then, which of the following is NOT a possible value of $\brak{e^f} (0) $?
		\begin{enumerate}
				\begin{multicols}{4}
				\item 2
				\item 6
				\item $\frac{7}{9} e^{\frac{1}{9}}$
				\item $\sqrt{2} + i \sqrt{2}$
				\end{multicols}
		\end{enumerate}
	\item The number of non-isomorphic abelian groups of order 24 is
	\item Let V be the real vector space of all polynomials in one variable with real coefficiencts and having degree at most 20. Define the subspaces
		$$ W_1 = \cbrak{p \in V : p(1) = 0 ; p(\frac{1}{2}) = 0 ; p(5) = 0 ; p(7) = 0 } $$
		$$ W_2 = \cbrak{p \in V : p(\frac{1}{2}) = 0 ; p(3) = 0 ; p(4) = 0 ; p(7) = 0 } $$
		Then the dimension of $ W_1 \cap W_2 $ is


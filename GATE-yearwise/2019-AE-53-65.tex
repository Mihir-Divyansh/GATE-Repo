\iffalse
	\chapter{2019}
	\author{AI24BTECH11001}
	\section{ae}
\fi	
    \item A propeller driven airplane has a gross take-off weight of $4905 N$ with a wing area of $6.84 m^2$ Assume that the wings are operating at the maximum $C_{L}^{\frac{3}{2}} / C_{D}$ of $13$, the propeller efficiency is $0.9$ and the specific fuel consumption of the engine is $0.76 kg/kW-hr$. Given that the density of air at sea level is $1.225kg/m^3$ and the acceleration due to gravity is $9.81 m/s^2$ the weight of the fuel required for an endurance of 18 hours at sea level is (round off to the nearest integer).

    \item The design of an airpla is modified to increase the vertical tail area by $20 \%$ and decrease the moment arm from the aerodynamic centre of the vertical tail to the airplane centre of gravity by $20 \%$. Assuming all other factors remain unchanged, the ratio of the modified to the original directional static stability ( $C_{N_\beta}$ due to tail fin) is $\dots$ (round off to $2$ decimal places).

    \item For a rocket engine, the velocity ratio $r$ is $\frac{V_a}{V_e}$, where $V_a$ is the vehicle velocity and $V_e$ is the exit velocity of the exhaust gases. Assume the flow to be optimally expanded through the nozzle. For $r = 2$, if $F$ is the thrust produced and $m$ is the mass flow rate of exhaust gases, then, $\frac{F}{mV_e}$ is $\dots$

    \item The specific impulse of a rocket engine is $3000 Ns/kg$. The mass of the rocket at burnout is $1000 kg$. The propellant consumed in the process is $720 kg$. Assume all factors contributing to velocity loss to be negligible. The change in vehicle velocity $\Delta u$ is $\dots$ $km/s$ (round off to 2 decimal places).

    \item The combustion products of a gas turbine engine can be assumed to be a calorically perfect gas with $\gamma = 1.2$. The pressure ratio across the turbine stage is $0.14$. The measured turbine inlet and exit stagnation temperatures are $1200 K$ and $900 K$, respectively. The total-to- total turbine efficiency is $\dots$ $\%$ (round off to the nearest integer).

    \item The figure shows the velocity triangles for an axial compressor stage. The specific work input to the compressor stage is $\dots$ $kJ/kg$ (round off to $2$ decimal places).

    \begin{figure}[H]
\centering
\resizebox{0.5\textwidth}{!}{%
\begin{circuitikz}
\tikzstyle{every node}=[font=\small]
\draw [line width=0.8pt, short] (-5,10.75) -- (-2,10.75);
\draw [line width=0.8pt, short] (-5,10.75) -- (-4.5,12.75);
\draw [line width=0.8pt, short] (-4.5,12.75) -- (-2,10.75);
\draw [line width=0.8pt, short] (-5,10.75) -- (-3.75,12.75);
\draw [line width=0.8pt, short] (-3.75,12.75) -- (-2,10.75);
\draw [line width=0.3pt, ->, >=Stealth] (-1.5,12.25) -- (-1.5,11.25);
\draw [line width=0.3pt, ->, >=Stealth] (-1.5,12.25) -- (-0.5,12.25);
\draw [line width=0.3pt, short] (-2,13) -- (-2,10.75);
\draw [line width=0.3pt, short] (-3.75,13) -- (-3.75,12.75);
\draw [line width=0.3pt, short] (-5.75,12.75) -- (-4.5,12.75);
\draw [line width=0.3pt, short] (-4.5,12.75) -- (-4.5,11.75);
\draw [line width=0.3pt, short] (-5.75,10.75) -- (-5.25,10.75);
\draw [line width=0.3pt, <->, >=Stealth] (-5.5,12.75) -- (-5.5,10.75);
\draw [line width=0.3pt, <->, >=Stealth] (-3.75,13) -- (-2,13);
\draw [line width=0.3pt, <->, >=Stealth] (-5,10.5) -- (-2,10.5);
\draw [line width=0.3pt, short] (-2,10.75) -- (-2,10.25);
\draw [line width=0.3pt, short] (-5,10.75) -- (-5,10.25);
\node [font=\small] at (-1.25,11.25) {z};
\node [font=\small] at (-0.5,12.5) {$\theta$};
\node [font=\small] at (-2.75,12) {$w_2$};
\node [font=\small] at (-3.25,11.5) {$w_1$};
\node [font=\small] at (-5,12) {$c_1$};
\node [font=\small] at (-4.25,11.5) {$c_2$};
\node [font=\small] at (-3,13.25) {$40 m/s$};
\node [font=\small] at (-3.5,10.25) {$U =100m/s$};
\node [font=\small] at (-6,11.75) {$c_z = 60m/s$};
\node [font=\small] at (-4.25,12) {};
\node [font=\footnotesize] at (-4.5,12.25) {$\alpha_1$};
\node [font=\small] at (-5.5,13.25) {$\alpha_1 = 30^\circ$};
\end{circuitikz}
}%

\label{fig:my_label}
\end{figure}

    \item As shown in the figure, a rigid slab CD of weight W (distributed uniformly along its length) is hung from a ceiling using three cables of identical length and cross-sectional area. The central cable is made of steel (Young's modulus = 3E) and the other two cables are made of aluminium (Young's modulus = E). The percentage of the total weight taken by the central cable is $\%$ (round off to the nearest integer).
    \begin{figure}[H]
\centering
\resizebox{0.5\textwidth}{!}{%
\begin{circuitikz}
\tikzstyle{every node}=[font=\normalsize]
\draw [ fill={rgb,255:red,10; green,10; blue,10} ] (-5.25,10.5) rectangle (-1.25,10.25);
\draw [short] (-5.25,12.5) -- (-1.25,12.5);
\draw [short] (-5.25,12.5) -- (-5.25,10.5);
\draw [short] (-3.25,12.5) -- (-3.25,10.5);
\draw [short] (-1.25,12.5) -- (-1.25,10.5);
\draw [short] (-5.25,12.5) -- (-5,12.75);
\draw [short] (-5,12.5) -- (-4.75,12.75);
\draw [short] (-4.75,12.5) -- (-4.5,12.75);
\draw [short] (-4.5,12.5) -- (-4.25,12.75);
\draw [short] (-4.25,12.5) -- (-4,12.75);
\draw [short] (-4,12.5) -- (-3.75,12.75);
\draw [short] (-3.75,12.5) -- (-3.5,12.75);
\draw [short] (-3.5,12.5) -- (-3.25,12.75);
\draw [short] (-3.25,12.5) -- (-3,12.75);
\draw [short] (-3,12.5) -- (-2.75,12.75);
\draw [short] (-2.75,12.5) -- (-2.5,12.75);
\draw [short] (-2.5,12.5) -- (-2.25,12.75);
\draw [short] (-2.25,12.5) -- (-2,12.75);
\draw [short] (-2,12.5) -- (-1.75,12.75);
\draw [short] (-1.75,12.5) -- (-1.5,12.75);
\draw [short] (-1.5,12.5) -- (-1.25,12.75);
\draw [short] (-1.25,12.5) -- (-1,12.75);
\draw [<->, >=Stealth] (-5.25,13) -- (-3.25,13);
\draw [<->, >=Stealth] (-3.25,13) -- (-1.25,13);
\node [font=\normalsize] at (-4.25,13.25) {a};
\node [font=\normalsize] at (-2.25,13.25) {a};
\node [font=\normalsize] at (-5.5,10) {C};
\node [font=\normalsize] at (-1.25,10) {D};
\node [font=\normalsize] at (-5.5,11.5) {E};
\node [font=\normalsize] at (-1,11.5) {E};
\node [font=\normalsize] at (-3.5,11.5) {3E};
\end{circuitikz}
}%

\label{fig:my_label}
\end{figure}
    \item All the bars in the given truss are elastic with Young's modulus $200 GPa$, and have identical cross-sections with moment of inertia $0.1 cm^2$. The lowest value of the load $P$ at which the truss fails due to buckling is $\dots$ KN (round off to the nearest integer).

    \begin{figure}[H]
\centering
\resizebox{0.5\textwidth}{!}{%
\begin{circuitikz}
\tikzstyle{every node}=[font=\normalsize]
\draw [short] (3,6.5) -- (7.25,6.5);
\draw [short] (3,6.25) -- (7.25,6.25);
\draw [short] (3,6.5) .. controls (3,6.25) and (3,6.25) .. (3,6.25);
\draw [short] (7.25,6.5) -- (7.25,6.25);
\draw [short] (3,6.5) -- (5,9.25);
\draw [short] (3.25,6.5) -- (5,9);
\draw [short] (7.25,6.5) -- (5,9.25);
\draw [short] (7,6.5) -- (5,9);
\draw [short] (5,9) -- (5,9.25);
\draw [->, >=Stealth] (5,9.75) -- (5,9.25);
\foreach \x in {0,...,0}{
  \draw  (2.75+\x*0.6802721088435374,6) -- ++(0.3401360544217687,0.3) -- ++ (0.3401360544217687, -0.3);
}
\draw [short] (2.5,6) -- (3.75,6);
\draw [short] (2.5,6) -- (2.75,5.75);
\draw [short] (2.75,6) -- (3,5.75);
\draw [short] (3,6) -- (3.25,5.75);
\draw [short] (3.25,6) -- (3.5,5.75);
\draw [short] (3.5,6) -- (3.75,5.75);
\draw [short] (3.75,6) -- (4,5.75);
\draw [short] (4.75,8.75) -- (5,8.5);
\draw [short] (5,8.5) -- (5.25,8.75);
\foreach \x in {0,...,0}{
  \draw  (6.75+\x*0.6802721088435374,6) -- ++(0.3401360544217687,0.3) -- ++ (0.3401360544217687, -0.3);
}
\draw [short] (3.75,7.25) .. controls (4,6.75) and (4,6.75) .. (3.75,6.5);
\node [font=\normalsize] at (4.25,7) {$45^{\circ}$};
\draw [short] (6.5,7.25) .. controls (6.25,6.75) and (6.25,6.75) .. (6.25,6.5);
\node [font=\normalsize] at (5.75,7) {$45^{\circ}$};
\node [font=\normalsize] at (4.75,9.25) {A};
\node [font=\normalsize] at (2.75,6.5) {B};
\node [font=\normalsize] at (7.5,6.5) {C};
\draw [<->, >=Stealth] (3,5.25) -- (7.25,5.25);
\node [font=\normalsize] at (5,5.5) {10 cm};
\node [font=\normalsize] at (7.25,6) {0};
\node [font=\normalsize] at (7,6) {0};
\node [font=\normalsize] at (6.75,6) {0};
\node [font=\normalsize] at (7.5,6) {0};
\draw [short] (6.5,6) -- (6.75,5.75);
\draw [short] (6.75,6) -- (7,5.75);
\draw [short] (7,6) -- (7.25,5.75);
\draw [short] (7.25,6) -- (7.5,5.75);
\draw [short] (7.5,6) -- (7.75,5.75);
\end{circuitikz}
}%

\label{fig:my_label}
\end{figure}

    \item A solid circular shaft is designed to transmit a torque $T$ with a factor of safety of $2$. It is proposed to replace the solid shaft by a hollow shaft of the same material and identical outer radius. If the inner radius is half the outer radius, the factor of safety for the hollow shaft is $\dots$ (round off to 1 decimal place).

    \begin{figure}[H]
\centering
\resizebox{0.5\textwidth}{!}{%
\begin{circuitikz}
\tikzstyle{every node}=[font=\normalsize]
\draw  (-5.5,11.25) circle (1.25cm);
\draw  (-2.5,11.25) circle (1.25cm);
\draw  (-2.5,11.25) circle (0.75cm);
\draw [->, >=Stealth] (-5.5,11.25) -- (-4.75,12.25);
\draw [->, >=Stealth] (-2.5,11.25) -- (-2,10.75);
\draw [->, >=Stealth] (-2.5,11.25) -- (-1.75,12.25);
\draw [short] (-4.75,12.25) -- (-4.25,12.25);
\draw [short] (-1.75,12.25) -- (-1.25,12.25);
\node [font=\large] at (-4.5,12.5) {$R$};
\node [font=\large] at (-1.5,12.5) {$R$};
\node [font=\normalsize] at (-2.5,10.75) {$R/2$};
\end{circuitikz}
}%
\label{fig:my_label}
\end{figure}

    \item In the structure shown in the figure, bars $AB$ and $BC$ are made of identical material and have circular cross-sections of $10 mm$ radii. The yield stress of the material under uniaxial tension is $280 MPa$. Using the von Mises yield criterion, the maximum load along the z- direction (perpendicular to the plane of paper) that can be applied at $C$, such that $AB$ does not yield is $\dots$ $N$ (round off to the nearest integer).

    \begin{figure}[H]
\centering
\resizebox{0.3\textwidth}{!}{%
\begin{circuitikz}
\tikzstyle{every node}=[font=\normalsize]
\draw [short] (-6,11) -- (-6,9);
\draw [short] (-6,11) -- (-6.25,10.75);
\draw [short] (-6,10.75) -- (-6.25,10.5);
\draw [short] (-6,10.5) -- (-6.25,10.25);
\draw [short] (-6,10.25) -- (-6.25,10);
\draw [short] (-6,10) -- (-6.25,9.75);
\draw [short] (-6,9.75) -- (-6.25,9.5);
\draw [short] (-6,9.5) -- (-6.25,9.25);
\draw [short] (-6,9.25) -- (-6.25,9);
\draw [short] (-6,9) -- (-6.25,8.75);
\draw [short] (-6,10) -- (-2.25,10);
\draw [short] (-2.25,10) -- (-2.25,13.25);
\draw [->, >=Stealth] (-6,11) -- (-6,11.5);
\draw [->, >=Stealth] (-6,9.75) -- (-5.25,9.75);
\draw (-2.25,13) to[short, -o] (-2.25,13.25) ;
\node [font=\large] at (-5.75,10.25) {A};
\node [font=\large] at (-2.5,10.25) {B};
\node [font=\large] at (-2.5,13) {C};
\node [font=\normalsize] at (-5,9.75) {x};
\node [font=\normalsize] at (-6,11.75) {y};
\node [font=\normalsize] at (-4.75,11.25) {AB = 0.55m};
\node [font=\normalsize] at (-4.75,10.75) {BC = 0.5m};
\end{circuitikz}
}%

\label{fig:my_label}
\end{figure}

    \item A thin-walled tube, with the cross-section shown in the figure, is subjected to a torque of $T=1 kN-m$. The walls have uniform thickness $t = 1 mm$ and shear modulus $G = 26 GPa$. Assume that the curved portion is semi-circular. The shear stress in the wall is $\dots$ $MPa$ (round off to 1 decimal place).

    \begin{figure}[H]
\centering
\resizebox{0.5\textwidth}{!}{%
\begin{circuitikz}
\tikzstyle{every node}=[font=\small]
\draw [line width=0.8pt, short] (-5,13) .. controls (-5.5,12.25) and (-5.5,11.75) .. (-5,11);
\draw [line width=0.8pt, short] (-5,11) -- (0.25,11);
\draw [line width=0.8pt, short] (-5,13) -- (0.25,11);
\draw [line width=0.3pt, ->, >=Stealth] (-4.75,12) -- (-5.25,12.25);
\node [font=\small] at (-2.25,10.75) {0.5 m};
\node [font=\small] at (-4,11.75) {R =50mm};
\draw [line width=0.3pt, <->, >=Stealth] (-5,10.5) -- (0.25,10.5);
\draw [line width=0.3pt, short] (-5,10.75) -- (-5,10.25);
\draw [line width=0.3pt, short] (0.25,10.75) -- (0.25,10.25);
\end{circuitikz}
}%

\label{fig:my_label}
\end{figure}

    \item For a damped spring-mass system, mass $m = 10 kg$, stiffness $k = 10^3 N/m$, and damping coefficient $c = 20 kg/s$. The ratio of the amplitude of oscillation of the first cycle to that of the fifth cycle is $\dots$ (round off to 1 decimal place).

    \item For the system of springs and masses shown below, $k = 1250 N/m$ and $m = 10 kg$. The highest natural frequency, $\omega$, of the system is $\dots$ radians/s (round off to nearest integer).
    \begin{figure}[H]
\centering
\resizebox{0.5\textwidth}{!}{%
\begin{circuitikz}
\tikzstyle{every node}=[font=\large]
\draw (-6,10) to[R] (-3.5,10);
\draw [ fill={rgb,255:red,0; green,0; blue,0} ] (-3.5,10.5) rectangle (-2.5,9.5);
\draw (-2.5,10) to[R] (-0.25,10);
\node [font=\LARGE] at (-2.5,9.5) {};
\node [font=\LARGE] at (-2,9) {};
\draw [ fill={rgb,255:red,0; green,0; blue,0} ] (-0.25,10.5) rectangle (0.75,9.5);
\draw (0.75,10) to[R] (3,10);
\draw [short] (3,11) -- (3,9);
\draw [short] (-6,11) -- (-6,9);
\draw [short] (-3,9.75) -- (-3,8.75);
\draw [short] (0.25,9.75) -- (0.25,8.75);
\draw [->, >=Stealth] (-3,9.25) -- (-2.25,9.25);
\draw [->, >=Stealth] (0.25,9.25) -- (1,9.25);
\draw [short] (3,11) -- (3.25,10.75);
\draw [short] (3,10.75) -- (3.25,10.5);
\draw [short] (3,10.5) -- (3.25,10.25);
\draw [short] (3,10.25) -- (3.25,10);
\draw [short] (3,10) -- (3.25,9.75);
\draw [short] (3,9.75) -- (3.25,9.5);
\draw [short] (3,9.5) -- (3.25,9.25);
\draw [short] (3,9.25) -- (3.25,9);
\draw [short] (3,9) -- (3.25,8.75);
\draw [short] (-6,11) -- (-6.25,10.75);
\draw [short] (-6,10.75) -- (-6.25,10.5);
\draw [short] (-6,10.5) -- (-6.25,10.25);
\draw [short] (-6,10.25) -- (-6.25,10);
\draw [short] (-6,10) -- (-6.25,9.75);
\draw [short] (-6,9.75) -- (-6.25,9.5);
\draw [short] (-6,9.5) -- (-6.25,9.25);
\draw [short] (-6,9.25) -- (-6.25,9);
\draw [short] (-6,9) -- (-6.25,8.75);
\node [font=\large] at (-1.5,10.75) {2k};
\node [font=\large] at (1.75,10.5) {k};
\node [font=\large] at (-4.75,10.5) {k};
\node [font=\large] at (-3,10.75) {m};
\node [font=\large] at (0.25,10.75) {m};
\node [font=\large] at (-2.5,9) {$x_1$};
\node [font=\large] at (0.75,9) {$x_2$};
\end{circuitikz}
}%

\label{fig:my_label}
\end{figure}


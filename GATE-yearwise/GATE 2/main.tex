\iffalse
\title{'2013-PH-(14-26)'}
\author{EE24BTECH11014 - Deepak }
\section{ph}
\chapter{2013}
\fi


    \item Interference fringes are seen at an observation plane $z=0$, by the superposition of two plane waves $
    A_1 \exp[i(\overrightarrow{k}_1 \cdot \overrightarrow{r} - \omega t)] 
  \textbf{ and }   A_2 \exp[i(\overrightarrow{k}_2 \cdot \overrightarrow{r} - \omega t)]$, where $A_1$ and $A_2$ are real amplitudes. The condition for interference maximum is 

  \begin{enumerate}
    \item $(\overrightarrow{k}_1 - \overrightarrow{k}_2) \cdot \overrightarrow{r} = (2m+1)\pi$
    \item $(\overrightarrow{k}_1 - \overrightarrow{k}_2) \cdot \overrightarrow{r} = 2m\pi$
    \item $(\overrightarrow{k}_1 + \overrightarrow{k}_2) \cdot \overrightarrow{r} = (2m+1)\pi$
    \item $(\overrightarrow{k}_1 + \overrightarrow{k}_2) \cdot \overrightarrow{r} = 2m\pi$
\end{enumerate}


\item For a scalar function $\phi$ satisfying the Laplace equation, $\nabla \phi$ has

\begin{enumerate}
    \item zero curl and non-zero divergence
    \item non-zero curl and zero divergence
    \item zero curl and zero divergence
    \item non-zero curl and non-zero divergence
\end{enumerate}

 \item A circularly polarized monochromatic plane wave is incident on a dielectric interface at Brewster angle. Which one of the following statements is CORRECT?

\begin{enumerate}
    \item The reflected light is plane polarized in the plane of incidence and the transmitted light is circularly polarized.
    \item The reflected light is plane polarized perpendicular to the plane of incidence and the transmitted light is plane polarized in the plane of incidence.
    \item The reflected light is plane polarized perpendicular to the plane of incidence and the transmitted light is elliptically polarized.
    \item There will be no reflected light and the transmitted light is circularly polarized.
\end{enumerate}

\item Which one of the following commutation relations is NOT CORRECT? Here, symbols have their usual meanings.
\begin{enumerate}
    \item (A) $[L^2, L_z] = 0$
    \item (B) $[L_x, L_y] = ihL_z$
    \item (C) $[L_z, L_+] = hL_+$
    \item (D) $[L_z, L_-] = hL_-$
\end{enumerate}
\item The Lagrangian of a system with one degree of freedom $q$ is given by $L = \alpha \dot{q}^2 + \beta q^2$, where $\alpha$ and $\beta$ are non-zero constants. If $p_q$ denotes the canonical momentum conjugate to $q$ then which one of the following statements is CORRECT?
\begin{enumerate}
    \item $p_q = 2\beta q$ \quad and it is a conserved quantity.
    \item $p_q = 2\beta q$ \quad and it is not a conserved quantity.
    \item $p_q = 2\alpha \dot{q}$ \quad and it is a conserved quantity.
    \item $p_q = 2\alpha \dot{q}$ \quad and it is not a conserved quantity.
\end{enumerate} 
\item What should be the clock frequency of a $6$-bit A/D converter so that its maximum conversion time is $32 \mu$s?
\begin{multicols}{4}
\begin{enumerate}
    \item  $1$ MHz
    \item  $2$ MHz
    \item  $0.5$ MHz
    \item  $4$ MHz
\end{enumerate}
\end{multicols}

\item A phosphorous doped silicon semiconductor (doping density: $10^{17}/cm^3$) is heated from $100^\circ C$ to $200^\circ C$. Which one of the following statements is CORRECT? 

\begin{enumerate}
    \item Position of Fermi level moves towards conduction band
    \item Position of dopant level moves towards conduction band
    \item Position of Fermi level moves towards middle of energy gap
    \item Position of dopant level moves towards middle of energy gap
\end{enumerate}

\item Considering the BCS theory of superconductors, which one of the following statements is NOT CORRECT?\\
(h is the Planck's constant and $e$ is the electronic charge)

\begin{enumerate}
    \item Presence of energy gap at temperatures below the critical temperature
    \item Different critical temperatures for isotopes
    \item Quantization of magnetic flux in superconducting ring in the unit of $\left(\dfrac{h}{e}\right)$
    \item Presence of Meissner effect
\end{enumerate}

\item Group $\mathrm{I}$ contains elementary excitations in solids. Group $\mathrm{II}$ gives the associated fields with these excitations. MATCH the excitations with their associated field and select your answer as per codes given below.

\begin{tabular}{|c|c|} 
\hline
Group $\mathrm{I}$ & Group $\mathrm{II}$ \\
\hline
(P) phonon & (i) photon + lattice vibration \\
\hline
(Q) plasmon & (ii) electron + elastic deformation \\
\hline
(R) polaron & (iii) collective electron oscillations \\
\hline
(S) polariton & (iv) elastic wave \\
\hline
\end{tabular} \\

Codes 

\begin{enumerate}
    \item (P-iv), (Q-iii), (R-i), (S-ii)
    \item (P-iv), (Q-iii), (R-ii), (S-i)
    \item (P-i), (Q-iii), (R-ii), (S-iv)
    \item (P-iii), (Q-iv), (R-ii), (S-i)
\end{enumerate}

\item The number of distinct ways of placing four indistinguishable balls into five distinguishable boxes is $\underline{\hspace{2cm}}$.

\item A voltage regulator has ripple rejection of $-50$dB. If input ripple is 1 mV, what is the output ripple voltage in $\mu$V? The answer should be up to two decimal places.

\underline{\hspace{2cm}}

\item The number of spectral lines allowed in the spectrum for the $3^2D \rightarrow 3^2P$ transition in sodium is

\underline{\hspace{2cm}}

\item Which of the following pairs of the given function $F(t)$ and its Laplace transform $f(s)$ is NOT CORRECT?


\begin{enumerate}
    \item $F(t)=\delta(t), \quad f(s)=1, \quad (\text{Singularity at }+0)$
    \item $F(t)=1, \quad f(s)=\dfrac{1}{s}, \quad (s>0)$
    \item $F(t)=\sin kt, \quad f(s)=\dfrac{k}{s^2+k^2}, \quad (s>0)$
    \item $F(t)=te^{kt}, \quad f(s)=\dfrac{1}{(s-k)^2}, \quad (s>k, s>0)$
\end{enumerate}






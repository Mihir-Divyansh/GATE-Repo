
\iffalse
\title{2011-EE-53-65}
\author{EE24BTECH11010 - BALAJI B}
\section{ee}
\chapter{2011}
\fi

    \item In the above system the three-phase fault $MVA$
 at the bus $3$
 is \hfill(2011-EE)
 \begin{enumerate}
 \begin{multicols}{2}  
     \item $82.55MVA$
     \item $85.11MVA$
     \item $170.91MVA$
     \item $181.82MVA$
 \end{multicols}
 \end{enumerate}
 \textbf{Statement for Linked Answer Questions \ref{54} and \ref{55}} \\
 A solar energy installation utilize a three-phase bridge converter to feed energy into power system through a transformer of $400V/400V$
 as shown below. The energy is collected in a bank of 
 $400V$
 battery and is connected to converter through a large filter choke of resistance $10 \ohm$ 
 \begin{figure}[H]
\centering
\resizebox{10cm}{!}{%
\begin{circuitikz}
\tikzstyle{every node}=[font=\normalsize]
\draw (6,14) to[R] (8,14);
\draw (4.5,14) to[L ] (6.5,14);
\draw (8,14) to[short] (9,14);
\draw (9,14) to[short] (9,13.5);
\draw (9,13.5) to[european resistor] (9,12);
\draw (9,12) to[short] (4.5,12);
\draw (4.5,14.5) to[short] (4.5,11.75);
\draw (4.5,14.5) to[short] (1.75,14.5);
\draw (1.75,14.5) to[short] (1.75,11.75);
\draw (1.75,11.75) to[short] (4.5,11.75);
\draw [short] (3,14) -- (3,13.5);
\draw [short] (2.5,13.5) -- (3.5,13.5);
\draw [short] (3,13.5) -- (2.5,13);
\draw [short] (3,13.5) -- (3.5,13);
\draw [short] (2.5,13) -- (3.5,13);
\draw [short] (3,13) -- (3,12.5);
\draw [short] (3.25,13.5) -- (3.5,13.75);
\node [font=\normalsize] at (5.5,14.75) {Filter};
\node [font=\normalsize] at (7,14.75) {Choke};
\node [font=\normalsize] at (10,12.75) {Battery};
\draw [short] (-2,14.75) -- (-2,11.75);
\draw [short] (-2.5,14.75) -- (-2.5,11.75);
\draw (-3.25,13.25) to[tmultiwire] (-6.75,13.25);
\draw (1.75,13.25) to[tmultiwire] (-1.25,13.25);
\draw  (-7.5,13.25) circle (0.75cm);
\draw [short] (-8,13.25) .. controls (-7.5,13.75) and (-7.5,12.5) .. (-7,13.25);
\draw [decorate, decoration={coil, aspect=0.4, segment length=6pt, amplitude=10pt}] (-1.5,12) -- (-1.5,14.5);
\draw [decorate, decoration={coil, aspect=0.4, segment length=6pt, amplitude=10pt}] (-3,12) -- (-3,14.5);
\end{circuitikz}
}%

\label{fig:my_label}
\end{figure}
 \item The maximum current through the battery will be \label{54} \hfill(2011-EE)
 \begin{enumerate}
     \begin{multicols}{2}
         \item $14A$
         \item $40A$
         \item $80A$
         \item $94A$
     \end{multicols}
 \end{enumerate}
 \item The $kVA$
 rating of the input transformer is \label{55} \hfill (2011-EE)
 \begin{enumerate}
     \begin{multicols}{2}
         \item $53.2kVA$
         \item $46.0kVA$
         \item $22.6kVA$
         \item None
     \end{multicols}
 \end{enumerate}
 \textbf{General Aptitude (GA) Questions}\\
 \item Choose the most appropriate word from the options given below to complete the following sentence: \\
 \textbf{Under ethical guidelines recently adopted by the Indian Medical Association, human genes are to be manipulated only to correct diseases for which \rule{1.6cm}{0.5pt}
treatments are unsatisfactory.} \hfill(2011-EE)
\begin{enumerate}
    \begin{multicols}{2}
    \item similar
    \item most
    \item uncommon
    \item available        
    \end{multicols}
\end{enumerate}
\item The question below consists of a pair of related words followed by four pairs of words. Select the pair that best expresses the relation in the original pair: \\
\textbf{Gladiator : Arena} \hfill (2011-EE)
\begin{enumerate}
    \begin{multicols}{2}
    \item dancer : stage
    \item commuter : train
    \item  teacher : classroom
    \item lawyer : courtroom
    \end{multicols}
\end{enumerate}
\item There are two candidates $P$ and $Q$ in an election. During the campaign, $40\%$ of the voters promised to vote for $P$, and rest for $Q$. However, on the day of election $15\%$ of the voters went back on their promise to vote for $P$ and instead voted for $Q$. $25\%$ of the voters went back on their promise to vote for $Q$ and instead voted for $P$. Suppose, $P$ lost by 2 votes, then what was the total number of voters? \hfill(2011-EE)
\begin{enumerate}
    \begin{multicols}{4}
        \item 100
        \item 110 
        \item 90 
        \item 95
    \end{multicols}
\end{enumerate}
\item \textbf{Choose the most appropriate word from the options given below to complete the following sentence:} \\
It was her view that the country's problems had been \rule{1.6cm}{0.5pt} by foreign technocrats, so that to invite them to come back would be counter-productive.

\hfill(2011-EE)
\begin{enumerate}
    \begin{multicols}{2}
    \item identified
    \item ascertained
    \item exacerbated
    \item analysed
    \end{multicols}
\end{enumerate}
\item Choose the word from the options given below that is most nearly opposite in meaning to the given word: \\
\textbf{Frequency} \hfill(2011-EE)
\begin{enumerate}
    \begin{multicols}{2}
        \item periodicity
        \item rarity
        \item gradualness
        \item persistency
    \end{multicols}
\end{enumerate}
\item Three friends R, S and T shared from a bowl took $\frac{1}{3}^{rd}$ of the toffees, but returned four to the bowl. S took $\frac{1}{4}^{th}$ of what was left but returned three toffees to the bowl. T took half of the remainder but returned 2 back. If the had 17 toffees left, how many toffees were originally there in the bowl ? \hfill(2011-EE)
	\begin{enumerate}
			\begin{multicols}{4}
			\item 38
			\item 31
			\item 48
			\item 41
			\end{multicols}
	\end{enumerate}
\item The fuel consumed by a motorcycle during a journey while travelling at various speeds is indicated in the graph below.
	\begin{figure}[H]
		\centering
		\begin{figure}[H]
\centering
\resizebox{10cm}{!}{%
\begin{circuitikz}
\tikzstyle{every node}=[font=\footnotesize]
\draw [short] (3.25,17.75) -- (3.25,13.25);
\draw [short] (3.25,17.75) -- (9.75,17.75);
\draw [short] (9.75,17.75) -- (9.75,13.25);
\draw [short] (3.25,13.25) -- (9.75,13.25);
\draw [short] (3.25,13.25) -- (5.25,13.25);
\draw [short] (3.25,13.25) -- (3.25,13);
\draw [short] (4.5,13.25) -- (4.5,13);
\draw [short] (5.75,13.25) -- (5.75,13);
\draw [short] (7,13.25) -- (7,13);
\draw [short] (8,13.25) -- (8,13);
\draw [short] (9,13.25) -- (9,13);
\draw [short] (3,17.75) -- (3.25,17.75);
\draw [short] (3.25,13.25) -- (3,13.25);
\draw [short] (3.25,14) -- (3,14);
\draw [short] (3.25,15.25) -- (3,15.25);
\draw [short] (3.25,16.5) -- (3,16.5);
\node at (4.25,14) [circ] {};
\node at (4.5,15.25) [circ] {};
\node at (6.75,16.5) [circ] {};
\node at (9,16) [circ] {};
\draw [short] (4.5,15.25) -- (4.25,14);
\draw [short] (4.5,15.25) -- (6.75,16.5);
\draw [short] (6.75,16.5) -- (9,16);
\node [font=\footnotesize] at (3.25,12.75) {0};
\node [font=\footnotesize] at (4.5,12.75) {15};
\node [font=\footnotesize] at (5.75,12.75) {30};
\node [font=\footnotesize] at (7,12.75) {45};
\node [font=\footnotesize] at (8,12.75) {60};
\node [font=\footnotesize] at (9,12.75) {75};
\node [font=\footnotesize] at (2.75,13.25) {0};
\node [font=\footnotesize] at (2.75,14) {30};
\node [font=\footnotesize] at (2.75,15.25) {60};
\node [font=\footnotesize] at (2.75,16.5) {90};
\node [font=\footnotesize] at (2.75,17.75) {120};
\node [font=\footnotesize, rotate around={90:(0,0)}] at (2,15.5) {Fuel consumption};
\node [font=\footnotesize, rotate around={90:(0,0)}] at (2.25,15.5) {(Kilometers per litre)};
\node [font=\footnotesize] at (6,12.25) {(kilometers per hour)};
\node [font=\footnotesize] at (6,12.5) {Speed};
\end{circuitikz}
}%
\end{figure}

	\end{figure}
	The distances covered during four laps of the journey are listed in the table below.
	
\begin{table}[h!]
    \centering
    \begin{tabular}{|c|c|c|}
        \hline
        \textbf{LAP} & \textbf{Distance} (kilometres) & \textbf{Average Speed} (kilometres per hour) \\
        \hline
        P & 15 & 15 \\
        \hline
        Q & 75 & 45 \\
        \hline
        R & 40 & 75 \\
        \hline
        S & 10 & 10 \\
        \hline
    \end{tabular}
\end{table}


	From the given data, we can conclude that the fuel consumed per kilometre was least during the lap \hfill(2011-EE)
	\begin{enumerate}
			\begin{multicols}{4}
			\item P
			\item Q
			\item R
			\item S
			\end{multicols}
	\end{enumerate}
\item The horse has played a little known but very important role in the field of medicine. Horses were injected with toxins of diseases until their blood built p immunities. Then a serum was made from their blood. Serums to fight with diphtheria and tetanus were developed this way.
	It can be inferred from the passage that horses were

 \hfill(2011-EE)
	\begin{enumerate}
		\item given immunity to diseases
		\item generally quite immune to diseases
		\item given medicines to fight toxins
		\item given diphtheria and tetanus serums
	\end{enumerate}
\item The sum of n terms of the series 4 + 44 + 444 + $\dots$ is \hfill(2011-EE)
	\begin{enumerate}
			\begin{multicols}{2}
			\item $\frac{4}{81} \sbrak{10^{n+1} - 9n - 1}$
			\item $\frac{4}{81} \sbrak{10^{n-1} - 9n - 1}$
			\item $\frac{4}{81} \sbrak{10^{n+1} - 9n - 10}$
			\item $\frac{4}{81} \sbrak{10^{n} - 9n - 10}$
			\end{multicols}
	\end{enumerate}
\item Given that $f(y) = \frac{\abs{y}}{y}$ and q is any non-zero real number, the value of $\abs{f(q) - f(-q)}$ is \hfill(2011-EE)
	\begin{enumerate}
			\begin{multicols}{4}
			\item 0
			\item -1
			\item 1
			\item 2
			\end{multicols}
	\end{enumerate}



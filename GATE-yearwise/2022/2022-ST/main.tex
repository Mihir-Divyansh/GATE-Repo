\iffalse
\chapter{2022}
\author{AI24BTECH11009}
\section{st}
\fi

\item If the line $y = \alpha x$, $\alpha \geq \sqrt{2}$, divides the area of the region 
\begin{align*}
    R:= \{\brak{x,y} \in \mathbb{R}^2|\ 0 \leq x \leq \sqrt{y},\ 0 \leq y \leq 2\}
\end{align*}
into two equal parts, then the value of $\alpha$ is equal to
\begin{enumerate}
    \item $\frac{3}{\sqrt{2}}$
    \item $2\sqrt{2}$
    \item $\sqrt{2}$
    \item $\frac{5}{2\sqrt{2}}$ \\
\end{enumerate}
\item Let $\brak{X, Y, Z}$ be a random vector with the joint probability density function 
\begin{align*}
    f_{X, Y, Z}\brak{x, y, z} = \begin{cases}
        \frac{1}{3}\brak{2x + 3y + z}, & 0 < x < 1,0 < y < 1,0 < z < 1, \\
        0, & \text{elsewhere}.
    \end{cases}
\end{align*}
Then which one of the following points is on the regression surface of $X$ on $\brak{Y, Z}$ ?
\begin{enumerate}
    \item $\brak{\frac{4}{7}, \frac{1}{3}, \frac{1}{3}}$
    \item $\brak{\frac{6}{7}, \frac{2}{3}, \frac{2}{3}}$
    \item $\brak{\frac{1}{2}, \frac{1}{3}, \frac{2}{3}}$
    \item $\brak{\frac{1}{2}, \frac{2}{3}, \frac{1}{3}}$ \\
\end{enumerate}
\item A random sample $X$ of size one is taken from a distribution with the probability density function 
\begin{align*}
    f\brak{x; \theta} = \begin{cases}
        \frac{2x}{\theta^2}, & 0 < x < \theta, \\
        0, & \text{elsewhere}.
    \end{cases}
\end{align*}
If $\frac{X}{\theta}$ is used as a pivot for obtaining the confidence interval for $\theta$, then which one of the following is an 80\% confidence interval (confidence limits rounded off to three decimal places) for $\theta$ based on the observed sample value $x = 10$ ? 
\begin{enumerate}
    \item \brak{10.541, 31.623}
    \item \brak{10.987, 31.126} 
    \item \brak{11.345, 30.524}
    \item \brak{11.267, 30.542} \\
\end{enumerate}
\item Let $X_1$, $X_2$, $\cdots$, $X_7$ be a random sample from a normal population with mean 0 and variance $\theta > 0$. Let
\begin{align*}
    K = \frac{X_1^2 + X_2^2}{X_1^2 + X_2^2 + \cdots + X_7^2}.
\end{align*}
Consider the following statements:\\\\
(I) The statistics $K$ and $X_1^2 + X_2^2 + \cdots + X_7^2$ are independent.\\
(II) $\frac{7K}{2}$ has an $F$-distribution with 2 and 7 degrees of freedom.\\
(III) $E\brak{K^2} = \frac{8}{63}$. \\\\
Then which of the above statements is/are true ?
\begin{enumerate}
    \item (I) and (II) only
    \item (I) and (III) only
    \item (II) and (III) only
    \item (I) only \\
\end{enumerate}
\item Consider the following statements:\\\\
(I) Let a random variable $X$ have the probability density function
\begin{align*}
    f_X\brak{x} = \frac{1}{2}e^{-\abs{x}},\ -\infty < x < \infty.
\end{align*}
Then there exist $i.i.d.$ random variables $X_1$ and $X_2$ such that $X$ and
$X_1 - X_2$ have the same distribution.\\ 
(II) Let a random variable $Y$ have the probability density function 
\begin{align*}
    f_Y\brak{y} = \begin{cases}
        \frac{1}{4}, & -2 < y < 2, \\
        0, & \text{elsewhere}.
    \end{cases}
\end{align*}
Then there exist $i.i.d.$ random variables $Y_1$ and $Y_2$ such that $Y$ and
$Y_1 - Y_2$ have the same distribution. \\\\
Then which of the above statements is/are true ?
\begin{enumerate}
    \item (I) only
    \item (II) only
    \item Both (I) and (II)
    \item Neither (I) nor (II) \\
\end{enumerate}
\item Suppose $X_1$, $X_2$, $\cdots$ , $X_n$, $\cdots$ are independent exponential random variables with the mean $\frac{1}{2}$ . Let the notation $i.o.$ denote 'infinitely often'. Then which of the following is/are true ? 
\begin{enumerate}
    \item $P\brak{\left\{X_n > \frac{\epsilon}{2}\log_en\right\}i.o.} = 1$ for $0 < \epsilon \leq 1$
    \item $P\brak{\left\{X_n < \frac{\epsilon}{2}\log_en\right\}i.o.} = 1$ for $0 < \epsilon \leq 1$
    \item $P\brak{\left\{X_n > \frac{\epsilon}{2}\log_en\right\}i.o.} = 1$ for $\epsilon > 1$
    \item $P\brak{\left\{X_n < \frac{\epsilon}{2}\log_en\right\}i.o.} = 1$ for $\epsilon > 1$ \\ 
\end{enumerate}
\item Let $\{X_n\}, n \geq 1$, be a sequence of random variables with the probability mass functions
\begin{align*}
    p_{X_n}\brak{x} = \begin{cases}
        \frac{n}{n+1}, & x = 0, \\
        \frac{1}{n+1}, & x = n, \\
        0, & \text{elsewhere}.
    \end{cases}
\end{align*}
Let $X$ be a random variable with $P\brak{X = 0} = 1$. Then which of the following statements is/are true ? 
\begin{enumerate}
    \item $X_n$ converges to $X$ in distribution
    \item $X_n$ converges to $X$ in probability
    \item $E\brak{X_n} \rightarrow E\brak{X}$
    \item There exists a subsequence $\{X_{n_k}\}$ of $\{X_n\}$ such that $X_{n_k}$ converges to $X$ almost surely \\
\end{enumerate}
\item Let $\mathbf{M}$ be any $3 \times 3$ symmetric matrix with eigenvalues 1, 2 and 3. Let $\mathbf{N}$ be any $3 \times 3$ matrix with real eigenvalues such that $\mathbf{MN} + \mathbf{N}^T\mathbf{M} = 3\mathbf{I}$, where $\mathbf{I}$ is the $3 \times 3$ identity matrix. Then which of the following cannot be eigenvalue(s) of the matrix $\mathbf{N}$ ? 
\begin{enumerate}
    \item $\frac{1}{4}$
    \item $\frac{3}{4}$
    \item $\frac{1}{2}$
    \item $\frac{7}{4}$ \\
\end{enumerate}
\item Let $\mathbf{M}$ be a $3 \times 2$ real matrix having a singular value decomposition as $\mathbf{M} = \mathbf{USV}^T$, where the matrix $\mathbf{S} = \sbrak{\begin{matrix}
    \sqrt{3} & 0 & 0 \\ 0 & 1 & 0
\end{matrix}}^T$, $\mathbf{U}$ is a $3 \times 3$ orthogonal matrix, and $\mathbf{V}$ is a $2 \times 2$ orthogonal matrix. Then which of the following statements is/are true ? 
\begin{enumerate}
    \item The rank of the matrix $\mathbf{M}$ is 1
    \item The trace of the matrix $\mathbf{M}^T\mathbf{M}$ is 4
    \item The largest singular value of the matrix $\brak{\mathbf{M}^T\mathbf{M}^{-1}}\mathbf{M}^T$ is 1
    \item The nullity of the matrix $\mathbf{M}$ is 1 \\ 
\end{enumerate}
\item Let $X$ be a random variable such that 
\begin{align*}
    P\brak{\frac{a}{2\pi}X \in \mathbb{Z}} = 1,\ a > 0,
\end{align*}
where $\mathbb{Z}$ denotes the set of all integers. If $\phi_X\brak{t},\ t \in \mathbb{R}$, denotes the characteristic function of $X$, then which of the following is/are true ? 
\begin{enumerate}
    \item $\phi_X\brak{a} = 1$
    \item $\phi_X\brak{\cdot}$ is periodic with period $a$
    \item $\abs{\phi_X\brak{t}} < 1$ for all $t \neq a$
    \item $\int_{0}^{2\pi} e^{-itn}\phi_X\brak{t}dt = \pi P \brak{X = \frac{2 \pi n}{a}},\ n\in \mathbb{Z},\ i = \sqrt{-1}$ \\
\end{enumerate}
\item Which of the following real valued functions is/are uniformly continuous on $[0, \infty)$ ?
\begin{enumerate}
    \item $\sin^2\brak{x}$
    \item $x\sin\brak{x}$
    \item $\sin\brak{\sin\brak{x}}$
    \item $\sin\brak{x\sin\brak{x}}$ \\ 
\end{enumerate}
\item Two independent random samples, each of size 7, from two populations yield the following values : 
\begin{table}[h!]
  \centering
  \begin{tabular}[12pt]{ |c| c| c| c| c| c| c| c|}
    \hline
    Population 1 & 18 & 20 & 16 & 20 & 17 & 18 & 14 \\ 
    \hline
    Population 2 & 17 & 18 & 14 & 20 & 14 & 13 & 16 \\
    \hline 
    \end{tabular}


\end{table}\\
If Mann-Whitney $U$ test is performed at 5\% level of significance to test the null hypothesis $H_0$: Distributions of the populations are same, against the alternative hypothesis $H_1$: Distributions of the populations are not same, then the value of the test statistic $U$ (\textbf{in integer}) for the given data, is $\_\_\_\_$ \\
\item Consider the multiple regression model
\begin{align*}
    Y = \beta_0 + \beta_1X_1 + \beta_2X_2 +\beta_3X_3 + \epsilon,
\end{align*}
where $\epsilon$ is normally distributed with mean 0 and variance $\sigma^2 > 0$, and $\beta_0$, $\beta_1$, $\beta_2$, $\beta_3$ are unknown parameters. Suppose 52 observations of $\brak{Y, X_1, X_2, X_3}$ yield sum of squares due to regression as 18.6 and total sum of squares as 79.23. Then, for testing the null hypothesis $H_0$: $\beta_1$ = $\beta_2$ = $\beta_3$ = 0 against the alternative hypothesis $H_1$: $\beta_i \neq 0$ for some $i = 1, 2, 3,$ the value of the test statistic (\textbf{rounded off to three decimal places}), based on one way analysis of variance, is $\_\_\_\_$

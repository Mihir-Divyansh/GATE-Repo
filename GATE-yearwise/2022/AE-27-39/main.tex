\iffalse
\title{GATE Questions 19}
\author{EE24BTECH11012 - Bhavanisankar G S}
\section{ae}
\chapter{2022}
\fi
%\begin{enumerate}
	\item For INnternational Standard Atmosphere ( ISA ) upto 11 km, which of the following statement(s) is(are) true ?
		\begin{enumerate}
			\item The hydrostatic / aerostatic equation is used.
			\item The temperature lapse rate is taken as $-10^{-2} K/m$
			\item The sea level conditions are taken as $P_s = 1.01325 \times 10^5 Pa, T_s = 300 K, \rho_s = 1.225 kg/m^3$
			\item Air is treated as a perfect gas
		\end{enumerate}
	\item Let $\sigma$ and $\rho$ represent the normal stress and shear stress on a plane, respectively. The Mohr circle(s) that may possibly represent the state of stress at points in a beam of rectangular cross-section under $\textbf{pure blending}$ is/are :
				\begin{multicols}{2}
					\begin{enumerate}
				\item
					\begin{figure}[H]
						\centering
						\begin{circuitikz}
\tikzstyle{every node}=[font=\Large]
\draw [->, >=Stealth] (2.5,14) -- (2.5,17);
\draw [->, >=Stealth] (2,14.75) -- (5.5,14.75);
\draw  (3.25,15.5) circle (1.25cm);
\node [font=\Large] at (2,16.75) {$\tau$};
\node [font=\Large] at (5.25,14.5) {$\sigma$};
\end{circuitikz}
					\end{figure}
				\item
					\begin{figure}[H]
						\centering
						\begin{circuitikz}
\tikzstyle{every node}=[font=\Large]
\draw [->, >=Stealth] (4.5,13.75) -- (4.5,16.75);
\draw [->, >=Stealth] (2,14.75) -- (5.5,14.75);
\draw  (3.25,14.75) circle (1.25cm);
\node [font=\Large] at (4,16.75) {$\tau$};
\node [font=\Large] at (5.5,15) {$\sigma$};
\end{circuitikz}
					\end{figure}
				\item
					\begin{figure}[H]
						\centering
						\begin{circuitikz}
\tikzstyle{every node}=[font=\Large]
\draw [->, >=Stealth] (2,13.75) -- (2,16.75);
\draw [->, >=Stealth] (2,14.75) -- (5.5,14.75);
\draw  (3.25,14.75) circle (1.25cm);
\node [font=\Large] at (1.5,16.75) {$\tau$};
\node [font=\Large] at (5.5,15) {$\sigma$};
\end{circuitikz}
					\end{figure}
				\item
					\begin{figure}[H]
						\centering
						\begin{circuitikz}
\tikzstyle{every node}=[font=\Large]
\draw [->, >=Stealth] (2,13.75) -- (2,16.75);
\draw [->, >=Stealth] (2,14.75) -- (5.5,14.75);
\draw  (3.75,14.75) circle (1.25cm);
\node [font=\Large] at (1.5,16.75) {$\tau$};
\node [font=\Large] at (5.5,15) {$\sigma$};
\end{circuitikz}
					\end{figure}
					\end{enumerate}
				\end{multicols}
	\item An isotropic linear elastic material point under plane srain condition in the x-y plane always obeys :
		\begin{enumerate}
			\item out-of-plane normal strain $\epsilon_{zz} = 0$
			\item out-of-plane normal stress $\sigma_{zz} = 0$
			\item out-of-plane shear stress $\tau_{xz} = 0$
			\item out-of-plane shear strain $\gamma_{xz} = 0$
		\end{enumerate}
	\item A high-pressure-ratio multistage axial compressor encounters an extreme loading mismatch during starting. Which of the following technique(s) can be used to alleviate this problem ?
		\begin{enumerate}
				\begin{multicols}{2}
				\item Blade cooling
				\item Variable angle stator vanes
				\item Blow-off valves
				\item Multi-spool shaft
				\end{multicols}
		\end{enumerate}
	\item The arc length of the parametric curve: $x = \cos{\theta}, y = \sin{\theta}, z = \theta$ from $\theta = 0 \text{to} \theta = 2 \pi$ is equal to
	\item An unpowered glider is flying at a glide angle of $10 \degree$. Its lift-to-drag ratio is \underline{  }
	\item The two-dimensional plane-stress state at a point is
		$$ \sigma_{xx} = 110 MPa, \sigma_{yy} = 30 MPa, \tau_{xy} = 40 MPa $$
		The normal stress, $\sigma_n$ on a plane inclined at $45 \degree$ as shown in the figure is \underline{  } MPa .
		\begin{figure}[H]
						\centering
						\begin{circuitikz}
\tikzstyle{every node}=[font=\normalsize]
\draw (2.25,16.5) to[short] (2.25,13.5);
\draw (2.25,13.5) to[short] (5.25,13.5);
\draw (5.25,13.5) to[short] (2.25,16.5);
\draw [->, >=Stealth] (3.75,15) -- (4.5,15.75);
\draw [->, >=Stealth] (2.75,14.25) -- (3.75,14.25);
\draw [->, >=Stealth] (2.75,14.25) -- (2.75,15.25);
\node [font=\normalsize] at (4,14.25) {x};
\node [font=\normalsize] at (2.75,15.5) {y};
\node [font=\normalsize] at (4.75,15.5) {$\sigma_n$};
\node [font=\normalsize] at (4.5,13.75) {45 $\circ$};
\end{circuitikz}
					\end{figure}
	\item In a $\textbf{static}$ test, a turbofam engine with $\textbf{by-pass ratio}$ of 9 has core hot exhaust speed 1.5 times that of fan exhaust speed. The engine is operated at a fuel to air ratio of f = 0.03 . Both the fan and the core streams have no pressure thrust. The ratio of fan thrust to thrust from the core engine is \underline{   }
	\item In a single stage turbine, the hot gases come out of stator/nozzle at a speed of 500 m/s and at angle of $70 \degree$ with the turbine axis as shown. The design speed of the rotor blade is 250 m/s at the mean blade radius. The rotor blade angle, $\beta$, at the leading edge is \underline{  } degrees.
		\begin{figure}[H]
						\centering
						\begin{circuitikz}
\tikzstyle{every node}=[font=\normalsize]
\draw [->, >=Stealth] (3.5,11.5) -- (4.75,14.5);
\draw [dashed] (3.5,11.5) -- (7.5,11.5);
\draw [dashed] (4.5,12) -- (7.75,12);
\draw [->, >=Stealth] (5.25,12) -- (6.25,13);
\node [font=\normalsize] at (6.5,12.25) {$\beta$};
\draw [short] (4.25,14) .. controls (4,13) and (3.75,12.25) .. (3,11.25);
\draw [short] (1.75,11.25) .. controls (2.5,11) and (2.5,11) .. (3,11.25);
\draw [short] (1.75,11.25) .. controls (2.5,11.25) and (2.75,11.5) .. (3,12);
\draw [short] (3,12) .. controls (3.5,12.5) and (3.75,13) .. (4.25,14);
\draw [short] (7.5,14) .. controls (8.5,14.25) and (8.75,14.25) .. (9.25,14.25);
\draw [short] (9.25,14.25) .. controls (10,14) and (10.5,14) .. (11.25,13.5);
\draw [short] (7.5,14) .. controls (8,14.25) and (8,14.75) .. (9,14.75);
\draw [short] (9,14.75) .. controls (10.25,14.25) and (10.5,14.25) .. (11.25,13.5);
\draw [->, >=Stealth] (9,13.25) -- (9,15.5);
\node [font=\large] at (3.5,10.75) {STATOR/NOZZLE};
\node [font=\large] at (9,11.25) {ROTOR};
\node [font=\normalsize] at (4.5,14.75) {$500 m/s$};
\node [font=\normalsize] at (4,12) {$70 \circ$};
\node [font=\normalsize] at (8.75,13.5) {$U = 250 m/s$};
\end{circuitikz}
					\end{figure}
	\item The height of a right  circular cone of maximum volume that can be enclosed within a hollow sphere of radius R is
		\begin{enumerate}
				\begin{multicols}{4}
				\item R
				\item $\frac{5}{4}$ R
				\item $\frac{4}{3}$ R
				\item $\frac{3}{2}$ R
				\end{multicols}
		\end{enumerate}
	\item Consider the differential equation $\frac{d^2 y}{d x^2} - 2 \frac{dy}{dx} + y = 0$. The boundary conditions are $y=0$ and $\frac{dy}{dx} = 1 \text{at} x=0$. Then the value of $y$ at $x = \frac{1}{2}$ is
		\begin{enumerate}
				\begin{multicols}{4}
				\item 0
				\item $\sqrt{e}$
				\item $\frac{\sqrt{e}}{2}$
				\item $\sqrt{\frac{e}{2}}$
				\end{multicols}
		\end{enumerate}
	\item Consider the partial differential equation $\frac{\partial ^2 f}{\partial x^2} + \frac{\partial ^2 f}{\partial y^2} = 0$ where x, y are real. If $f(x,y) = a(x) b(y) $, where $a(x)$ and $b(y)$ are real functions, which one of the folowing statements can be true ?
		\begin{enumerate}
			\item $a(x)$ is a periodic function and $b(y)$ is a linear function.
			\item both $a(x)$ and $b(y)$ are exponential functions
			\item $a(x)$ is a periodic function and $b(y)$ is an exponential function
			\item both $a(x)$ and $b(y)$ are periodic functions
		\end{enumerate}
	\item A cyindrical object of diameter 900 mm is designed to move $\textbf{axially}$ in air at 60 m/s. Its drag is estimated on a geometrically half-scaled model in water, assuming flow similarity. \\
		Co-efficients of dynamic viscosity and densities for air and water are $1.86 \times 10^{-5} Pa-s, 1.2 kg/m^3 \text{and} 1.01 \times 10^{-3} Pa-s, 1000 kg/m^3$ respectively. \\
		Drag measured for the model is 2280 N. Drag experienced by the full-scale object is \underline{  } N
		\begin{enumerate}
				\begin{multicols}{4}
				\item 322
				\item 644
				\item 1288
				\item 2576
				\end{multicols}
		\end{enumerate}



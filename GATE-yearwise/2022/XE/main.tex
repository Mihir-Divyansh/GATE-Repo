
\iffalse
\chapter{2022}
\author{AI24BTECH11008}
\section{xe}
\fi

	\item Let $\Gamma$ be the positively oriented circle $x^2 + y^2 = 9$ in the xy-plane. If $$\oint_{\Gamma}\brak{3y + e^{xsinx}dx + \brak{7x+\sqrt{e^y + 2}}}dy =\alpha \pi,$$ where $\alpha$ is a real constant then $\alpha$ is equal to.............. \hfill (2022)
	\item Let $y_1\brak{x}$ and $y_2\brak{x}$ be two linearly independent solutions of $$x^2\frac{d^2y}{dx^2}-2x\frac{dy}{dx}+2y=, \hfill x>0.$$ Let $W\brak{y_1,y_2}\brak{x}$ denote the Wronskian of $y_1\brak{x}$ and $y_2\brak{x}$ at x. If $W\brak{y_1, y_2}\brak{1}=1$ then $W\brak{y_1, y_2}\brak{2}$ is equal to........\hfill (2022)
    \item Let $A = \begin{bmatrix}2 &0&1&1\\1&2&5&-5\\0&0&3&0\\0&0&1&3\end{bmatrix}$. Then the sum of the geometric multiplicities of the distinct eigenvalues of A is equal to ..........\hfill (2020)
    \item In a cosmopolitan city, the population comprises of 30\% female and 70\% male.
    Suppose that 5\% of female and 30\% of male in the population are foreigners. A person is selected at random from this population. Given that the selected person
    is a foreigner, the probability that the person is a female is ............. (round off to
    three decimal places). \hfill (2022)
    \item Let $f:\brak{0, \infty}\to \mathcal{R}$ be the continuous function such that $f\brak{x} = 2 + \frac{g\brak{x}}{x}$ for all $x>0$, where $g\brak{x}=\int_{1}^{x}f\brak{t}dt$ for all x>0. Then $f\brak{2}$ is equal to \hfill (2022)
    \begin{enumerate}[label = (\Alph*)]
        \item $2+\ln 2$
        \item $2-\ln 2$
        \item $2+\ln 4$
        \item $2-\ln 4$
    \end{enumerate}
    \item Let A and B be $n\times n$ matrices with real entries. Consider the following statements:\\
    P: If A is symmetric then rank(A) = Number of nonzero eigenvalues(counting multiplicity) of A.
    Q: If AB = 0 then $rank\brak{A} + rank\brak{B}\leq n$.\\
    Then \hfill (2022)
    \begin{enumerate}[label = (\Alph*)]
        \item both P and Q are TRUE
        \item P is TRUE and Q is FALSE 
        \item Q is TRUE and P is FALSE
        \item both P and Q are FALSE  
    \end{enumerate}
    \item Let $f:\mathcal{R}^2\to R$ be given by $f\brak{x,y} = 4xy-2x^2-y^4+1$. The number of critical points where $f$ has local maximum is equal to ...........\hfill (2022)
    \item If the quadrature rule $$\int_{-1}^{1}f\brak{x}dx\approx f\brak{\alpha} + \gamma f\brak{\beta},$$ where $\alpha,\beta$ and $\gamma$ are real constants, is exact for all polynomials for degree$\leq 3$, then $\gamma + 3\brak{\alpha^2 + \beta^2} + \brak{\alpha^3 + \beta^3}$ is equal to .............\hfill (2022)
    \item A heavy horizontal cylinder of diameter D supports a mass of liqud having density $\rho$ as shown in the figure. Find out the vertical compononent of force exerted by the liquid per unit length of the cylinder if g is the acceleration due to gravity.\hfill (2022)
    \begin{figure}[!ht]
        \centering
        \resizebox{0.4\textwidth}{!}{%
    \begin{circuitikz}
    \tikzstyle{every node}=[font=\normalsize]
    \draw [ fill={rgb,255:red,27; green,24; blue,24} ] (12,15.5) rectangle (19.25,15.25);
    \draw [ fill={rgb,255:red,14; green,201; blue,225} ] (12,17.5) rectangle (17.25,15.5);
    \draw [ fill={rgb,255:red,31; green,30; blue,30} ] (17.25,16.5) circle (1cm);
    \draw [ color={rgb,255:red,252; green,252; blue,252}, <->, >=Stealth] (17.25,15.5) -- (17.25,17.5);
    \node [font=\normalsize, color={rgb,255:red,252; green,252; blue,252}] at (17.5,16.5) {D};
    \end{circuitikz}
    }% % Specify the path to your TikZ file
        \caption{}
        %\label{fig2}
    \end{figure}
    \begin{enumerate}[label = (\Alph*)]
        \item $\frac{\pi D^2}{4}\rho g$
        \item $\frac{\pi D^2}{8}\rho g$
        \item $\frac{\pi D^2}{2}\rho g$
        \item $\frac{\pi D^2}{3}\rho g$
    \end{enumerate} 
    \item The figure shows the developing zone and the fully developed region in a pipe
    flow where the steady flow takes place from left to right. The wall shear stress in
    the sections A, B, C, and D are given by $\tau_A,\tau_B,\tau_C,\tau_D$, respectively. Select the
    correct statement. \hfill (2022)
    \begin{figure}[!ht]
        \centering
        \resizebox{0.4\textwidth}{!}{%
    \begin{circuitikz}
    \tikzstyle{every node}=[font=\normalsize]
    \draw [short] (3,9.75) -- (11.25,9.75);
    \draw [short] (3.25,6.5) -- (11.25,6.5);
    \draw [dashed] (8,10.5) -- (8,5.25);
    \draw [dashed] (5.75,9.75) -- (5.75,6.75);
    \draw [dashed] (4.25,9.75) -- (4.25,6.5);
    \draw [dashed] (9.25,9.75) -- (9.25,6.5);
    \draw [dashed] (10.25,9.75) -- (10.25,6.5);
    \draw [short] (3.25,6.5) .. controls (5,7.75) and (5,7.75) .. (8,8.25);
    \draw [short] (3,9.75) .. controls (4.75,8.5) and (5.25,8.5) .. (8,8.25);
    \draw [dashed] (2.75,8.25) -- (11.25,8.25);
    \node [font=\normalsize] at (5.25,10) {Developing zone};
    \node [font=\normalsize] at (10,10) {Fully developed region};
    \node [font=\normalsize] at (4.25,6.25) {A};
    \node [font=\normalsize] at (5.75,6.25) {B};
    \node [font=\normalsize] at (9.25,6.25) {C};
    \node [font=\normalsize] at (10.25,6.25) {D};
    \end{circuitikz}
    }% % Specify the path to your TikZ file
        \caption{}
        %\label{fig2}
    \end{figure}
    \begin{enumerate}[label = (\Alph*)]
        \item $\tau_A > \tau_B$
        \item $\tau_B > \tau_A$
        \item $\tau_C > \tau_B$
        \item $\tau_C > \tau_D$
    \end{enumerate}
    \item The left hand column lists some non-dimensional numbers and the right hand column lists some physical phenomena. Indicate the correct combination \hfill (2022)
    \begin{table}
        \centering
        \begin{tabular}{|c|c|}
            \hline
           1. Reynolds number& i.Wave drag \\ \hline
            2. Froude number& ii.compressible flow \\ \hline
            3. Mach number & iii.Viscous drag \\ \hline
            4. Weber number & iv. Spray formation \\ \hline
        \end{tabular} % Specify the path to your TikZ file
        \caption{}
        %\label{fig2}
    \end{table}
    \begin{enumerate}[label = (\Alph*)]
        \item 1-iii, 2-i, 3-ii, 4-iv
        \item 1-i, 2-ii, 3-iv, 4-iii
        \item 1-iv, 2-iii, 3-iv, 4-iii
        \item 2-iv, 1-iii, 3-ii, 4-i
    \end{enumerate}
    \item As temperature increases\hfill (2022)
    \begin{enumerate}[label = (\Alph*)]
        \item the dynamic viscosity of a gas increases. 
        \item the dynamic viscosity of a liquid decreases.
        \item the dynamic viscosity of a liquid does not change.
        \item the dynamic viscosity of a gas decreases. 
    \end{enumerate}
    \item Which of the following statement(s) regarding a venturimeter is/are correct? \hfill (2022)
    \begin{enumerate}[label = (\Alph*)]
        \item In the direction of flow, it consists of a converging section, a throat, and a
        diverging section
        \item In the direction of flow, it consists of a diverging section, a throat, and a
        converging section.
        \item It is used for flow measurement at a very low Reynolds number.
        \item Pressure tappings are provided just upstream of the venturimeter and at the throat.
    \end{enumerate}


\iffalse
\chapter{2018}
\author{AI24BTECH11009}
\section{ce}
\fi

\item "The driver applied the $\_\_\_\_$ as soon as she approached the hotel where she wanted to take a $\_\_\_\_$." \\\\
The words that best fill the blanks in the above sentence are
\begin{enumerate}
    \item brake, break
    \item break, break
    \item brake, brake
    \item break, brake \\
\end{enumerate}
\item "It is no surprise that every society has had codes of behaviour; however, the nature of these codes is often $\_\_\_\_$." \\\\
The word that best fills the blank in the above sentence is
\begin{enumerate}
    \item unpredictable
    \item simple
    \item expected
    \item strict \\
\end{enumerate}
\item Hema's age is 5 years more than twice Hari's age. Suresh's age is 13 years less than 10 times Hari's age. If Suresh is 3 times as old as Hema, how old is Hema ?
\begin{enumerate}
    \item 14
    \item 17
    \item 18
    \item 19 \\
\end{enumerate}
\item Tower A is 90 m tall and tower B is 140 m tall. They are 100 m apart. A horizontal skywalk connects the floors at 70 m in both the towers. If a taut rope connects the top of tower A to the bottom of tower B, at what distance (in meters) from tower A will the rope intersect the skywalk?
\begin{enumerate}
    \item 22.22
    \item 50
    \item 57.87
    \item 77.78 \\
\end{enumerate}
\item The temperature $T$ in a room varies as a function of the outside temperature $T_0$ and the number of persons in the room $p$, according to the relation $T = K \brak{\Theta p + T_0}$, where $\Theta$ and $K$ are constants. What would be the value of $\Theta$ given the following data ?
\begin{table}[h!]
  \centering
  \begin{tabular}[12pt]{ |c| c| c|}
    \hline
    \textbf{$T_0$} & \textbf{p} & \textbf{T} \\ 
    \hline
   25 & 2 & 32.4 \\
    \hline 
   30 & 5 & 42.0\\
    \hline    
    \end{tabular}

 \end{table}\\
 \begin{enumerate}
    \item 0.8
    \item 1.0
    \item 2.0
    \item 10.0 \\
\end{enumerate}
\item A fruit seller sold a basket of fruits at 12.5\% loss. Had he sold it for Rs. 108 more, he would have made a 10\% gain. What is the loss in Rupees incurred by the fruit seller?
\begin{enumerate}
    \item 48
    \item 52
     \item 60
    \item 108 \\
\end{enumerate}
\item The price of a wire made of a superalloy material is proportional to the square of its length. The price of 10 m length of the wire is Rs. 1600. What would be the total price (in Rs.) of two wires of lengths 4 m and 6 m ?
\begin{enumerate}
    \item 768
    \item 832
    \item 1440
    \item 1600 \\
\end{enumerate}
\item Which of the following function(s) is an accurate description of the graph for the range(s) indicated ?
\begin{figure}[!ht]
\centering
\resizebox{0.4\textwidth}{!}{%
\begin{circuitikz}
\tikzstyle{every node}=[font=\normalsize]
\draw [<->, >=Stealth] (7.5,2.5) -- (7.5,12.5);
\draw [<->, >=Stealth] (2.5,7.5) -- (12.5,7.5);
\node [font=\normalsize] at (7.75,7.25) {0};
\node [font=\normalsize] at (8.75,7.25) {1};
\node [font=\normalsize] at (10,7.25) {2};
\node [font=\normalsize] at (11.25,7.25) {3};
\node [font=\normalsize] at (7.75,10.25) {2};
\node [font=\normalsize] at (7.75,11.5) {3};
\node [font=\normalsize] at (6.25,7.25) {-1};
\node [font=\normalsize] at (5,7.25) {-2};
\node [font=\normalsize] at (3.75,7.25) {-3};
\node [font=\normalsize] at (7.75,6) {-1};
\node [font=\normalsize] at (7.75,4.75) {-2};
\node [font=\normalsize] at (7.75,3.5) {-3};
\node [font=\normalsize] at (7.75,9) {1};
\draw [short] (3.75,5) -- (6.25,10);
\draw [short] (6.25,10) -- (8.75,7.5);
\draw [short] (8.75,7.5) -- (10,8.75);
\draw [short] (10,8.75) -- (11.25,8.75);
\node at (3.75,5) [circ] {};
\node at (11.25,8.75) [circ] {};
\draw [short] (2.5,12.5) -- (12.5,12.5);
\draw [short] (12.5,12.5) -- (12.5,2.5);
\draw [short] (12.5,2.5) -- (2.5,2.5);
\draw [short] (2.5,12.5) -- (2.5,2.5);
\end{circuitikz}

}%
\end{figure}
\begin{align*}
\text{(i)}\ y & = 2x + 4\ \text{for}\ -3 \leq x \leq -1 \\
\text{(ii)}\ y & = \abs{x - 1}\ \text{for}\ -1 \leq x \leq 2 \\
\text{(iii)}\ y & = \abs{\abs{x} - 1}\ \text{for}\ -1 \leq x \leq 2 \\
\text{(iv)}\ y & = 1\ \text{for}\ 2 \leq x \leq 3 \\
\end{align*}
\begin{enumerate}
   \item (i), (ii) and (iii) only.
   \item (i), (ii) and (iv) only.
   \item (i) and (iv) only.
   \item (ii) and (iv) only. \\
\end{enumerate}
\item Consider a sequence of numbers $a_1$, $a_2$, $a_3$, $\cdots$, $a_n$ where $a_n = \frac{1}{n} - \frac{1}{n+2}$, for each integer $n > 0$. What is the sum of the first 50 terms ?
\begin{enumerate}
    \item $\brak{1 + \frac{1}{2}} - \frac{1}{50}$
    \item $\brak{1 + \frac{1}{2}} + \frac{1}{50}$
    \item $\brak{1 + \frac{1}{2}} - \brak{\frac{1}{51} + \frac{1}{52}}$
    \item $1 - \brak{\frac{1}{51} + \frac{1}{52}}$ \\
\end{enumerate}
\item Each of the letters arranged as below represents a unique integer from 1 to 9. The letters are positioned in the figure such that $\brak{\text{A} \times \text{B} \times \text{C}}$, $\brak{\text{B} \times \text{G} \times \text{E}}$ and $\brak{\text{D} \times \text{E} \times \text{F}}$ are equal. Which integer among the following choices cannot be represented by the letters A, B, C, D, E, F or G ?
\begin{figure}[!ht]
\centering
\resizebox{0.3\textwidth}{!}{%
\begin{circuitikz}
\tikzstyle{every node}=[font=\large]
\draw  (6.75,10.75) rectangle (7.75,10);
\draw  (7.75,10.75) rectangle (8.75,10);
\draw  (6.75,10.75) rectangle (5.75,10);
\draw  (7.75,10.75) rectangle (8.75,11.5);
\draw  (7.75,10) rectangle (8.75,9.25);
\draw  (5.75,10.75) rectangle (6.75,11.5);
\draw  (5.75,10) rectangle (6.75,9.25);
\node [font=\large] at (6.25,11) {A};
\node [font=\large] at (6.25,10.25) {B};
\node [font=\large] at (6.25,9.5) {C};
\node [font=\large] at (8.25,11) {D};
\node [font=\large] at (8.25,10.25) {E};
\node [font=\large] at (8.25,9.5) {F};
\node [font=\large] at (7.25,10.25) {G};
\end{circuitikz}

}%
\end{figure}
\begin{enumerate}
    \item 4
    \item 5
    \item 6
    \item 9 \\
\end{enumerate}
\item Which one of the following matrices is singular ?
\begin{enumerate}
    \item $\sbrak{\begin{matrix}
        2 & 5 \\ 1 & 3
    \end{matrix}}$
    \item $\sbrak{\begin{matrix}
        3 & 2 \\ 2 & 3
    \end{matrix}}$
    \item $\sbrak{\begin{matrix}
        2 & 4 \\ 3 & 6
    \end{matrix}}$
    \item $\sbrak{\begin{matrix}
        4 & 3 \\ 6 & 2 \\
    \end{matrix}}$
\end{enumerate}
\item For the given orthogonal matrix $Q$,
\begin{align*}
    Q = \sbrak{\begin{matrix}
        \frac{3}{7} & \frac{2}{7} & \frac{6}{7} \\
        -\frac{6}{7} & \frac{3}{7} & \frac{2}{7} \\
        \frac{2}{7} & \frac{6}{7} & -\frac{3}{7}
    \end{matrix}}
\end{align*}
The inverse is
\begin{enumerate}
    \item $\sbrak{\begin{matrix}
        \frac{3}{7} & \frac{2}{7} & \frac{6}{7} \\
        -\frac{6}{7} & \frac{3}{7} & \frac{2}{7} \\
        \frac{2}{7} & \frac{6}{7} & -\frac{3}{7}
    \end{matrix}}$
    \item $\sbrak{\begin{matrix}
        -\frac{3}{7} & -\frac{2}{7} & -\frac{6}{7} \\
        \frac{6}{7} & -\frac{3}{7} & -\frac{2}{7} \\
        -\frac{2}{7} & -\frac{6}{7} & \frac{3}{7}
    \end{matrix}}$
    \item $\sbrak{\begin{matrix}
        \frac{3}{7} & -\frac{6}{7} & \frac{2}{7} \\
        \frac{2}{7} & \frac{3}{7} & \frac{6}{7} \\
        \frac{6}{7} & \frac{2}{7} & -\frac{3}{7}
    \end{matrix}}$
    \item $\sbrak{\begin{matrix}
        -\frac{3}{7} & \frac{6}{7} & -\frac{2}{7} \\
        -\frac{2}{7} & -\frac{3}{7} & -\frac{6}{7} \\
        -\frac{6}{7} & -\frac{2}{7} & \frac{3}{7}
    \end{matrix}}$
\end{enumerate}
\item At the point $x = 0$, the function $f\brak{x} = x^3$ has
\begin{enumerate}
    \item local maximum
    \item local minimum
    \item both local maximum and minimum
    \item neither local maximum nor local minimum
\end{enumerate}


\iffalse
\chapter{2018}
\author{AI24BTECH11008}
\section{ph}
\fi

    \item A constant and uniform magnetic field $\overrightarrow{B}=B_0\hat{k}$ pervades all space. Which one of the
    following is the correct choice for the vector potential in Coulomb gauge? \hfill (2018)
    \begin{enumerate}[label = (\Alph*)]
        \item $-B_0\brak{x+y}\hat{i}$
        \item $B_0\brak{x+y}\hat{j}$
        \item $-B_0\brak{x\hat{j}}$
        \item $-\frac{1}{2}B_0\brak{x\hat{i}-y\hat{j}}$       
    \end{enumerate}
    \item If H is the Hamiltonian for a free particle with mass m , the commutator $[x,[x,H]]$ is \hfill (2018)
    \begin{enumerate}[label = (\Alph*)]
        \item $\frac{\hslash ^2}{m}$
        \item $-\frac{\hslash ^2}{m}$
        \item $-\frac{\hslash ^2}{2m}$
        \item $\frac{\hslash ^2}{2m}$
    \end{enumerate}
    \item  A long straight wire, having radius a and resistance per unit length r , carries a current
    I . The magnitude and direction of the Poynting vector on the surface of the wire is \hfill (2018)
    \begin{enumerate}[label = (\Alph*)]
        \item $\frac{I^2r}{2\pi a}$, perpendicular to axis of the wire and pointing inwards
        \item $\frac{I^2r}{2\pi a}$, perpendicular to axis of the wire and pointing outwards
        \item  $\frac{I^2r}{\pi a}$, perpendicular to axis of the wire and pointing inwards
        \item  $\frac{I^2r}{\pi a}$, perpendicular to axis of the wire and pointing outwards 
    \end{enumerate}
    \item Three particles are to be distributed in four non-degenerate energy levels. The possible
    number of ways of distribution: (i) for distinguishable particles, and (ii) for identical
    Boson, respectively, is \hfill (2018)
    \begin{enumerate}[label = (\Alph*)]
        \item  (i) 24, (ii) 4 
        \item  (i) 24, (ii) 20 
        \item  (i) 64, (ii) 20 
        \item  (i) 60, (ii) 16
    \end{enumerate}
    \item The term symbol for the electronic ground state of oxygen atom is    \hfill (2018)
    \begin{enumerate}[label=(\Alph*)]
        \item $^1S_0$
        \item $^1D_2$
        \item $^3P_0$
        \item $^3P_2$
    \end{enumerate}
    \item The energy dispersion for electrons in one dimensional lattice with lattice parameter a is
    given by $E\brak{k} = E_0-\frac{1}{2}W\cos ka$, where W and $E_0$ are constants. The effective mass of
    the electron near the bottom of the band is \hfill (2018)
    \begin{enumerate}[label=(\Alph*)]
        \item $\frac{2\hslash^2}{Wa^2}$
        \item $\frac{\hslash^2}{Wa^2}$
        \item $\frac{\hslash^2}{2Wa^2}$
        \item $\frac{\hslash^2}{4Wa^2}$
    \end{enumerate}
    \item Amongst electrical resistivity $\rho$ , thermal conductivity $\kappa$, specific heat $C$ , Young’s
    modulus Y  and magnetic susceptibility $\chi$ , which quantities show a sharp change at
    the superconducting transition temperature?  \hfill (2018)
    \begin{enumerate} [label = (\Alph*)]
     \item $\rho, \kappa, C,Y$ 
     \item $\rho,C,\chi$
     \item $\rho, \kappa, C,\chi$
     \item $\kappa,Y,\chi$ 
    \end{enumerate}
    \item A quarter wave plate introduces a path difference of $\frac{\lambda}{4}$ between the two components
    of polarization parallel and perpendicular to the optic axis. An electromagnetic wave with $\overrightarrow{E}=\brak{\hat{x}+\hat{y}}E_0e^{i\brak{kz-\omega t}}$is incident normally on a quarter wave plate which has its optic axis
    making an angle $135^{\circ}$ with the x - axis as shown.  \hfill (2018)
    \begin{figure}[!ht]
        \centering
        \resizebox{0.4\textwidth}{!}{%
    \begin{circuitikz}
    \tikzstyle{every node}=[font=\normalsize]
    \draw [->, >=Stealth] (14.25,8) -- (14.25,11.5);
    \draw [->, >=Stealth] (13.5,8.75) -- (18.5,8.75);
    \draw [dashed] (13,10) -- (15.25,7.75);
    \draw [->, >=Stealth] (14.5,8.75) .. controls (14.75,9.25) and (14.5,9.5) .. (13.75,9.25) ;
    \node [font=\normalsize] at (12.75,9) {optical axis};
    \node [font=\normalsize] at (14.5,11.5) {y};
    \node [font=\normalsize] at (18.75,8.75) {x};
    \node [font=\normalsize] at (15,9.25) {$135^{\circ}$};
    \end{circuitikz}
    }%  % Specify the path to your TikZ file
        \caption{1}
        %\label{fig2}
    \end{figure}
    The emergent electromagnetic wave would be 
    \begin{enumerate}[label=(\Alph*)]
        \item elliptically polarized 
        \item circularly polarized 
        \item  linearly polarized with polarization as that of incident wave 
        \item  linearly polarized but with polarization at $90^{\circ}$ to that of the incident wave 
    \end{enumerate}
    \item  A p - doped semiconductor slab carries a current $I=100mA$ in a magnetic field
    $B = 0.2T$ as shown. One measures $V_y$ = 0.25 mV and $V_x$ = 2mV . The mobility of holes
    in the semiconductor is ...........$m^2V^{-1}s^{-1}$ \hfill (2018)
    \begin{figure}[!ht]
        \centering
        \resizebox{0.4\textwidth}{!}{%
    \begin{circuitikz}
    \tikzstyle{every node}=[font=\normalsize]
    \draw  (17.5,9) -- (13.25,9) -- (13.75,10.5) -- (18,10.5) -- cycle;
    \draw [short] (13.25,9) -- (13.25,8.25);
    \draw [short] (13.25,8.25) -- (17.5,8.25);
    \draw [short] (17.5,9) -- (17.5,8.25);
    \draw [short] (17.5,8.25) -- (18,10);
    \draw [short] (18,10.5) -- (18,10);
    \draw [short] (13.75,10.5) -- (13,10.5);
    \draw [short] (13.25,9) -- (12.5,9);
    \draw [short] (13.25,9) -- (12.75,6.5);
    \draw [short] (17.5,9) -- (16.75,6.5);
    \draw [->, >=Stealth] (15.75,11) -- (15.75,10.25);
    \draw [->, >=Stealth] (17.25,11) -- (17.25,10);
    \draw [->, >=Stealth] (17.25,11) -- (17,10.25);
    \draw [->, >=Stealth] (17.25,11) -- (18.25,11);
    \draw [->, >=Stealth] (14,9.5) -- (14.5,9.5);
    \draw  (12.5,10) circle (0.25cm);
    \draw  (14.5,6.5) circle (0.25cm);
    \draw [short] (12.25,10) -- (12.5,9);
    \draw [short] (12.5,10.25) -- (13,10.5);
    \draw [short] (12.75,6.5) -- (14.25,6.5);
    \draw [short] (14.75,6.5) -- (16.75,6.5);
    \draw [<->, >=Stealth] (13.25,8) -- (17.5,8) node[pos=0.5, fill=white]{$l=10mm$};
    \draw [<->, >=Stealth] (17.75,8.25) -- (18.25,10) node[pos=0.5, fill=white]{$w=4mm$};
    
    \draw [<->, >=Stealth] (18.25,10.5) -- (18.25,9.75);
    \node [font=\normalsize] at (19,10) {$t=1mm$};
    \node [font=\normalsize] at (18.5,11) {$x$};
    \node [font=\normalsize] at (16.75,10.25) {$y$};
    \node [font=\normalsize] at (17.25,9.75) {$z$};
    \node [font=\normalsize] at (15.5,10.75) {$B$};
    \node [font=\normalsize] at (13.75,9.5) {$I$};
    \node [font=\normalsize] at (12.5,10) {$V_y$};
    \node [font=\normalsize] at (14.5,6.5) {$V_x$};
    \end{circuitikz}
    }%  % Specify the path to your TikZ file
        \caption{2}
        %\label{fig2}
    \end{figure}
    \item  An n - channel FET having Gate-Source switch-off voltage $V_{GS\brak{OFF} = -2V}$ is used to
    invert a 0-5 V square-wave signal as shown. The maximum allowed value of R would
    be ...........$k\Omega$\hfill (2018)
    \begin{figure}[!ht]
        \centering
        \resizebox{0.4\textwidth}{!}{%
        \begin{circuitikz}
        \tikzstyle{every node}=[font=\normalsize]
        \draw [short] (12,8.75) -- (12.75,8.75);
        \draw [short] (12.75,8.75) -- (12.75,9.25);
        \draw [short] (12.75,9.25) -- (13.25,9.25);
        \draw [short] (13.25,9.25) -- (13.25,8.75);
        \draw [short] (13.25,8.75) -- (13.75,8.75);
        \draw (14.5,9) to[short, -o] (14,9) ;
        \draw (14.5,9) to[R] (16.5,9);
        \draw (16.25,9) to[R] (16.25,7.5);
        \draw [->, >=Stealth] (16.5,9) -- (17.25,9);
        \draw [short] (17.25,9.25) -- (17.25,8.75);
        \draw [short] (17.25,9.25) -- (17.75,9.25);
        \draw [short] (17.25,8.75) -- (17.75,8.75);
        \draw (17.75,9.25) to[R] (17.75,12);
        \draw (17.75,8.75) to[R] (17.75,6.75);
        \draw (17.75,9.75) to[short, -o] (18.75,9.75) ;
        \draw [short] (19,9) -- (19.75,9);
        \draw [short] (19.75,9) -- (19.75,8.5);
        \draw [short] (19.75,8.5) -- (20.5,8.5);
        \draw [short] (20.5,8.5) -- (20.5,9);
        \draw [short] (20.5,9) -- (21.25,9);
        \draw (17.75,7) to (17.75,6.5) node[ground]{};
        \node [font=\normalsize] at (12,8.5) {0 V};
        \node [font=\normalsize] at (12,9.25) {5 V};
        \node [font=\normalsize] at (21.5,9) {5V};
        \node [font=\normalsize] at (21.5,8.5) {0 V};
        \node [font=\normalsize] at (18.25,7.75) {100$\Omega$};
        \node [font=\normalsize] at (18.25,10.5) {5k$\Omega$};
        \node [font=\normalsize] at (18.25,11.5) {+5 V};
        \node [font=\normalsize] at (14,9.5) {$V_{in}$};
        \node [font=\normalsize] at (15.5,9.5) {R};
        \node [font=\normalsize] at (15.75,8.25) {1k$\Omega$};
        \node [font=\normalsize] at (16,7.25) {-12 V};
        \node [font=\normalsize] at (19.25,9.75) {$V_{out}$};
        \end{circuitikz}
        }% % Specify the path to your TikZ file
        \caption{3}
        %\label{fig2}
    \end{figure}
    \item Inside a large nucleus, a nucleon with mass $939 MeVc^{-2}$ has Fermi momentum $1.40 fm^{-1}$
    at absolute zero temperature. Its velocity is Xc , where the value of X is.............. (up
    to two decimal places). \hfill (2018)
    \item 4MeV $\gamma$ - rays emitted by the de-excitation of $^{19}F$ are attributed, assuming spherical
    symmetry, to the transition of protons from $1d_{3/2}$ state to $1d_{5/2}$ state. If the contribution
    of spin-orbit term to the total energy is written as C$\langle \overrightarrow{l}.\overrightarrow{s}\rangle$
    the magnitude of C is ..........MeV (up to one decimal place). \hfill (2018)
    \item An  $\alpha$ particle is emitted by a $^{230}_{90}Th$ nucleus. Assuming the potential to be purely
    Coulombic beyond the point of separation, the height of the Coulomb barrier is.............. MeV (up to two decimal places).\hfill (2018);wq



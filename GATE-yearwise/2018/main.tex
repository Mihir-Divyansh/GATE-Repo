\iffalse
	\title{2018}
	\author{ee24btech11027}
	\section{ma}
	\chapter{2018}
\fi
\item  The image of the half plane $\operatorname{Re}\brak{z}+\operatorname{Im}\brak{z}>0$ under the map $\omega=\frac{z-1}{z+i}$ is given by 
\begin{multicols}{4}
\begin{enumerate}
    \item $\operatorname{Re}\brak{\omega}>0$
    \item $\operatorname{Im}\brak{\omega}>0$
    \item $\abs{\omega}>1$
    \item $\abs{\omega}<1$
\end{enumerate}
\end{multicols}
\item Let $D \subset \mathbb{R}^2$ denote the closed disc with center at the origin and radius 2. Then
\begin{align*}
    \displaystyle \iint\limits_{D} e^{-\brak{x^2+y^2}}dxdy=
\end{align*}
\begin{multicols}{4}
    \begin{enumerate}
        \item $\pi\brak{1-e^{-4}}$
        \item $\frac{\pi}{2}\brak{1-e^{-4}}$
        \item $\pi\brak{1-e^{-2}}$
        \item $\frac{\pi}{2}\brak{1-e^{-2}}$
    \end{enumerate}
\end{multicols}
\item Consider the polynomial $p\brak{X}=X^4+4$ in the ring $\mathbb{Q}\sbrak{X}$ of polynomials in the variable $X$ with coefficients in the field $\mathbb{Q}$ of rational numbers. Then
\begin{enumerate}
	\item the set of zeroes of $p\brak{X}$ in $\mathbb{C}$ forms a group under multiplication
    \item $p\brak{X}$ is reducible in the ring $\mathbb{Q}\sbrak{X}$
    \item the splitting field of $p\brak{X}$ has degree 3 over $\mathbb{Q}$
    \item the splitting field of $p\brak{X}$ has degree 4 over $\mathbb{Q}$
\end{enumerate}
\item Which one of the following statements is true?
\begin{enumerate}
    \item Every group of order 12 has a non trivial proper normal subgroup 
    \item Some group of order 12 does not have a non-trivial proper normal subgroup
    \item Every group of order 12 has a subgroup of order 6
    \item Every group of order 12 has an element of order 12
\end{enumerate}
\item For an odd prime $p$, consider the ring $\mathbb{Z}\sbrak{\sqrt{-p}}=\cbrak{a+b\sqrt{-p}: a,b\in \mathbb{Z}}\subseteq \mathbb{C}$. Then the element 2 in $\mathbb{Z}\sbrak{\sqrt{-p}}$ is 
\begin{multicols}{4}
\begin{enumerate}
    \item a unit 
    \item a square
    \item a prime 
    \item irreducible
\end{enumerate}
\end{multicols}
\item Consider the following two statements:\\
  P : The matrix \myvec{0&5\\0&7} has infinitely many LU factorizations, where $L$ is lower triangular with each diagonal entry 1 and $U$ is upper triangular.\\
  Q: The matrix \myvec{0&0\\2&5} has no LU factorization, where $L$ is lower triangular with each diagonal entry 1 and $U$ is upper triangular.\\
  Then which one of the following options is correct?
  \begin{enumerate}
      \item P is TRUE and Q is FALSE
      \item Both P and Q are TRUE
      \item P is FALSE and Q is TRUE
      \item Both P and Q are FALSE
  \end{enumerate}
\item If the characteristic curves of the partial differential equation $xu_{xx}+2x^2u_{xy}=u_x-1$ are $\mu\brak{x,y}=c_1$ and $\nu\brak{x,y}=c_2$, where $c_1$ and $c_2$ are constants, then
\begin{enumerate}
    \item $\mu\brak{x,y}=x^2-y,\nu\brak{x,y}=y$
    \item $\mu\brak{x,y}=x^2+y,\nu\brak{x,y}=y$
    \item $\mu\brak{x,y}=x^2+y,\nu\brak{x,y}=x^2$
    \item $\mu\brak{x,y}=x^2-y,\nu\brak{x,y}=x^2$
\end{enumerate}
\item Let $f : X\rightarrow Y$ be a continuous map from a Hausdorff topological space $X$ to a metric space $Y$. Consider the following two statements:\\
  P: $f$ is a closed map and the inverse image $f^{-1}\brak{y}=\cbrak{x\in X : f\brak{x}=y} $ is compact for each $y \in Y$.\\
  Q: For every compact subset $K \subset Y$, the inverse image $f^{-1}\brak{K}$ is a compact subset of $X$.\\which one of the following is true?
  \begin{enumerate}
      \item Q implies P but P does NOT imply Q
      \item P implies Q but Q does NOT imply P
      \item P and Q are equivalent
      \item neither P implies Q nor Q implies P
  \end{enumerate}
\item Let $X$ denote $\mathbb{R}^2$ endowed with the usual topology. Let $Y$ denote $\mathbb{R}$ endowed with the co-finite topology. If $Z$ is the product topological space $Y\times Y$, then 
\begin{enumerate}
    \item the topology of $X$ is the same as the topology of $Z$
    \item the topology of $X$ is strictly coarser (weaker) than that of $Z$
    \item the topology of $Z$ is strictly coarser (weaker) than that of $X$
    \item the topology of $X$ cannot be compared with that of $Z$
\end{enumerate}
\item Consider $\mathbb{R}^n$ with the usual topology for $n=1,2,3$. Each of the following options gives topological spaces $X$ and $Y$ with respective induced topologies. In which option is $X$ homeomorphic to $Y$?
\begin{enumerate}
	\item $X=\cbrak{\brak{x,y,z}\in \mathbb{R}^3 : x^2+y^2=1},Y=\cbrak{\brak{x,y,z}\in \mathbb{R}^3 : z=0, x^2+y^2 \neq 0}$
	\item $X=\cbrak{\brak{x,y}\in \mathbb{R}^2 : y=\sin\brak{\frac{1}{2}},0<x\le 1}\cup \cbrak{\brak{x,y}\in \mathbb{R}^2: x-0,-1\le y \le 1},Y=\sbrak{0,1}\subseteq \mathbb{R}$
	\item $X=\cbrak{\brak{x,y}\in \mathbb{R}^2:y=x\sin\brak{\frac{1}{x}},0<x \le 1}, Y=\sbrak{0,1}\subseteq \mathbb{R}$
	\item $X=\cbrak{\brak{x,y,z}\in \mathbb{R}^3 : x^2+y^2=1}, Y=\cbrak{\brak{x,y,z}\in \mathbb{R}^3: x^2+y^2=z^2\neq0}$
\end{enumerate}
\item Let $\cbrak{X_i}$ be a sequence of independent Poisson$\brak{\lambda}$ and let $W_n=\frac{1}{n}\sum_{i=1}^n X_i$. Then the limiting distribution of $\sqrt{n}\brak{W_n-\lambda}$ is the normal distribution with zero mean and variance given by 
\begin{multicols}{4}
    \begin{enumerate}
        \item 1
        \item $\sqrt{\lambda}$
        \item $\lambda$
        \item $\lambda^2$
    \end{enumerate}
\end{multicols}
\item Let $X_1,X_2,X_3,\dots,X_n$ be independent and identically distributed random variables with probability density function given by 
\begin{align*}
    f_X\brak{x,\theta}=
    \begin{cases}
        \theta e^{-\theta\brak{x-1}}, & x \ge 1,\\ 0 & \text{otherwise} 
    \end{cases}
\end{align*}
Also, let $\overline{X}=\frac{1}{n}\sum_{i=1}^nX_i$. Then the maximum likelihood estimator of $\theta$ is 
\begin{multicols}{4}
    \begin{enumerate}
        \item $\frac{1}{\overline{X}}$
        \item $\frac{1}{\overline{X}}-1$
        \item $\frac{1}{\brak{\overline{X}-1}}$
        \item $\overline{X}$
    \end{enumerate}
\end{multicols}
\item Consider the Linear Programming Problem \brak{\text{LPP}}:\\
Maximize : $\alpha X_1+X_2$\\
\vspace{-3em}
\begin{align*}
\text{Subject to: } 2X_1 + X_2 &\leq 6, \\
-X_1 + X_2 &\leq 1, \\
X_1 + X_2 &\leq 4, \\
X_1 &\geq 0, \\
X_2 &\geq 0
\end{align*}

where $\alpha$ is a constant. If the \brak{3,0} is the only optimal solution, then 
\begin{multicols}{4}
    \begin{enumerate}
        \item $\alpha <-2$
        \item $-2<\alpha<1$
        \item $1<\alpha<2$
        \item $\alpha>2$
    \end{enumerate}
\end{multicols}



\iffalse
    \author{EE24BTECH11043}
    \section{ph}
    \chapter{2020}
\fi
	\item Which one of the following is a solution of $\frac{d^2 u\brak{x}}{dx^2}=k^2u\brak{x}$, for k real?
		\begin{enumerate}
			\item $e^{-kx}$
			\item $\sin {kx}$
			\item $\cos {kx}$
			\item $\sinh {x}$
		\end{enumerate}
	\item A real, invertible $3\times 3$ matrix M has eigenvalues $\lambda,\brak{i=1,2,3}$ and the corresponding eigenvectors are $|e_i\rangle , \brak{i=1,2,3}$ respectively. Which one of the following is correct?
		\begin{enumerate}
			\item $M |e_i \rangle = \frac{1}{\lambda_i}|e_i\rangle$, for $i=1,2,3$
			\item $M^{-1}|e_i \rangle = \frac{1}{\lambda_i}|e_i\rangle$, for $i=1,2,3$
			\item $M^{-1}|e_i \rangle = \lambda_i|e_i\rangle$, for $i=1,2,3$
			\item The eigenvalues of $M$ and $M^{-1}$ are not related.
		\end{enumerate}
	\item A quantum particle is subjected to the potential \\
		$ V\brak{x} =
		\begin{cases}
			\infty, & x\le -\frac{a}{2} \\
			0, & -\frac{a}{2} < x < \frac{a}{2} \\
			\infty, & x\ge \frac{a}{2}
		\end{cases}$ \\
		The ground state wave function of the particle is proportional to
		\begin{enumerate}
			\item $\sin{\brak{\frac{\pi x}{2a}}}$
			\item $\sin{\brak{\frac{\pi x}{a}}}$
			\item $\cos{\brak{\frac{\pi x}{2a}}}$
			\item $\cos{\brak{\frac{\pi x}{a}}}$
		\end{enumerate}
	\item Let $\hat{a}$ and $\hat{a}^{+}$, respectively denote the lowering and raising operators of a one-dimensional simple harmonic oscillator. Let $|n\rangle$ be the energy eigenstate of the simple harmonic oscillator. Given that $|n\rangle$ is also an eigen state of $\hat{a}^{+}\hat{a}^{+}\hat{a}\hat{a}$ , the corresponding eigenvalue is
		\begin{enumerate}
	\item $n\brak{n-1}$
	\item $n\brak{n+1}$
	\item $\brak{n+1}^2$
	\item $n^2$
		\end{enumerate}
	\item Which one of the following is a universal logic gate?
		\begin{enumerate}
			\item AND
			\item NOT
			\item OR
			\item NAND
		\end{enumerate}
	\item Which one of the following is the correct binary equivalent of the hexadecimal $F6C$ ?
		\begin{enumerate}
			\item 0110 1111 1100 
			\item 1111 0110 1100
			\item 1100 0110 1111
			\item 0110 1100 0111
		\end{enumerate}
	\item The total angular momentum $j$ of the ground state of the $^{17}_{8}O$ nucleus is
		\begin{enumerate}
			\item $\frac{1}{2}$
			\item 1
			\item $\frac{3}{2}$
			\item $\frac{5}{2}$
		\end{enumerate}
	\item A particle $X$ is produced in the process $\pi ^{+}+p \to K^{+}+X$ via the strong interaction. If the quark content of the $K^{+}$ is $u\bar{s}$ , the quark content of $X$ is
		\begin{enumerate}
			\item $c\bar{s}$
			\item $uud$
			\item $uus$
			\item $u\bar{d}$
		\end{enumerate}
	\item A medium $\brak{\epsilon_r > 1, \mu_r =1,\sigma >0}$ is semi-transparent to an electromagnetic wave when
		\begin{enumerate}
			\item Conduction current $>>$ Displacement current
			\item Conduction current $<<$ Displacement current
			\item Conduction current = Displacement current
			\item Both Conduction current and Displacement current are zero
		\end{enumerate}
	\item A particle is moving in a central force field given by $\hat{F} =-\frac{k}{r^3}$, where $\hat{r}$ is the unit vector pointing away from the center of the field. The potential energy of the particle is given by
		\begin{enumerate}
			\item $\frac{k}{r^2}$
			\item $\frac{k}{2r^2}$
			\item $-\frac{k}{r^2}$
			\item $-\frac{k}{2r^2}$
		\end{enumerate}
	\item Choose the correct statement related to the Fermi energy $\brak{E_F}$ and the chemical potential $\brak{\mu}$ of a metal
		\begin{enumerate}
			\item $\mu = E_F$ only at $0k$
			\item $\mu = E_F$ at finite temperature
			\item $\mu< E_F$ at $0K$
			\item $\mu > E_F$ at finite temparature
		\end{enumerate}
	\item Consider a diatomic molecule formed by identical atoms. If $E_V$ and $E_C$ represent the energy of the vibrational nuclear motion and electronic motion respectively, then in terms of the electronic mass $m$ and nuclear mass $M$ , $\frac{E_V}{E_C}$ is proportional to
		\begin{enumerate}
			\item $\brak{\frac{m}{M}}^{1/2}$
			\item $\frac{m}{M}$
			\item $\brak{\frac{m}{M}}^{3/2}$
			\item $\brak{\frac{m}{M}}^2$
		\end{enumerate}
	\item Which one of the following relations determines the manner in which the electric field lines are refracted across the interface between two dielectric media having $\brak{\text{see figure}}$?
		\begin{figure}[!ht]
\centering
\resizebox{1\textwidth}{!}{%
\begin{circuitikz}
\tikzstyle{every node}=[font=\Large]

\draw [line width=1.5pt, short] (0,9) -- (21,9);
\draw [line width=1.4pt, dashed] (10.5,15.75) -- (10.75,1.5);
\draw [line width=1.2pt, short] (6.25,13.5) -- (10.5,9);
\draw [short] (10.5,9) -- (12.5,9);
\draw [line width=1.2pt, short] (10.5,9) -- (13.5,3);
\draw [line width=1.2pt, ->, >=Stealth] (5,14.75) -- (6.25,13.5);
\draw [line width=1.3pt, ->, >=Stealth] (13.5,3) -- (14.25,1.5);
\draw [line width=1pt, short] (9.75,9.75) .. controls (10,10.25) and (10.25,10.5) .. (10.5,10.25);
\draw [line width=1pt, short] (10.75,6.75) .. controls (11.25,6.75) and (11.25,6.75) .. (11.5,7);

\node [font=\Large] at (20.75,9.75) {$\epsilon_2$};
\node [font=\Large] at (20.5,8.5) {$\epsilon_2$};
\node [font=\Large] at (10,10.75) {$\theta_1$};
\node [font=\Large] at (11.25,6.25) {$\theta_2$};
\node [font=\Large] at (4.75,14) {$E_1$};
\node [font=\Large] at (14.75,2.5) {$E_2$};

\end{circuitikz}
}%

\end{figure}


		\begin{enumerate}
			\item $\epsilon_1 \sin{\theta_1}=\epsilon_2 \sin{\theta_2}$
			\item $\epsilon_1 \cos{\theta_1} = \epsilon_2 \cos{\theta_2}$
			\item $\epsilon_1 \tan{\theta_1} = \epsilon_2 \tan{\theta_2}$
			\item $\epsilon_1 \cot{\theta_1} = \epsilon_2 \cot{\theta_2}$
		\end{enumerate}





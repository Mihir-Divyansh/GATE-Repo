\iffalse
\title{2014-XE-14-26}
\author{EE24BTECH11001 -  ADITYA TRIPATHY}
\section{xe}
\chapter{2014}
\fi
    \item 
        Polymerized isotactic polybutadiene has a molecular weight of $3 \times 10^5 g/mol$. The degree of
        polymerization is
        \hfill{\brak{2014-XE}}
        \\
    \item  A bar of $Ti$ with Young's modulus of 110 $GPa$ and yield strength of $880 MPa$ is tested in tension. It
        is noticed that the alloy does not exhibit any strain hardening and fails at a total strain of 0.108. The
        mechanical energy that is necessary to break the material in $MJ/m^3$ is		
        \hfill{\brak{2014-XE}}
        \\
    \item A copper cup weighing 140 g contains 80 g of water at 4 $^{\degree}C$. Specific heats of water and copper are
        4.18 and 0.385 J/g $^{\degree}C$, respectively. If 100 g of water that is at 90 $^{\degree}C$ is added to the cup, the final
        temperature of water in $^{\degree}C$ is
        \hfill{\brak{2014-XE}}
        \\

    \item Match the reaction in $\textbf{Column I}$ with its name in $\textbf{Column II}$.\\
        L - liquid $\alpha, \beta, \gamma -$ different solid solution phases
        \begin{center}

            \begin{tabular}{|c|c|} 
                \hline
                \textbf{Column I} & \textbf{Column II}\\
                \hline
                P. L $\xrightarrow{cooling}$ $\quad \alpha + \beta$ & 1. peritectic\\
                \hline
                Q. L $+ \beta \xrightarrow{cooling}$ $\quad \gamma$ & 2. eutectic\\
                \hline
                R. $\alpha \xrightarrow{cooling}$ $\quad \beta + \gamma$ & 3. monotectic\\
                \hline
                \quad & 4. eutectoid\\
                \hline
            \end{tabular}
        \end{center}
        \hfill{\brak{2014-XE}}
        \begin{enumerate}
                \begin{multicols}{2}
                \item P-1, Q-4, R-3 \columnbreak 
                \item P-2, Q-1, R-4 
                \end{multicols} 
                \begin{multicols}{2}
                \item P-2, Q-3, R-1 \columnbreak 
                \item P-4, Q-2, R-3 
                \end{multicols}
        \end{enumerate}
    \item The Young's modulus of a unidirectional $SiC$ fiber reinforced $Ti$ matrix is 185 $GPa$.
        If the Young's modulii of $Ti$ and $SiC$ are 110 and 360 $GPa$ respectively, the volume fraction
        of fibers in the composite is 

        \hfill{\brak{2014-XE}}



    \item Match the composite in \textbf{Column I} with the most suitable application in 
        \textbf{Column II}
        \begin{center}
            \begin{tabular}{| c | c |}
                \hline
                \textbf{Column I}  & \textbf{Column II}\\
                P. Glass fibre reinforced plastic & 1.  Missile cone heads\\
                \hline
                Q. $SiC$ particle reiforced $Al$ alloy & 2.  Commercial automobile chasis\\
                \hline
                R. Carbon-carbon composite & 3. Airplane wheel tyres\\
                \hline
                S. Metal fibre reinforced rubber & 4. Car piston rings\\
                \hline
                \quad & 5. High performance skate boards\\

                \hline
            \end{tabular} 
        \end{center}

        \hfill{\brak{2014-XE}}
        \begin{enumerate}
                \begin{multicols}{2}
                \item P-4, Q-5, R-1, S-2 \columnbreak 
                \item P-3, Q-5, R-2, S-4 
                \end{multicols} 
                \begin{multicols}{2}
                \item P-5, Q-4, R-1, S-3 \columnbreak 
                \item P-4, Q-2, R-3, S-1
                \end{multicols}
        \end{enumerate}

    \item Which among the following rules need to be satisfied for obtaining an isomorphous phase diagram
        in a binary alloy system? 
        \begin{enumerate}
            \item[P.] The atomic size difference should be less than $15\%$ 
            \item[Q.] Both the end components should have the same crystal structure
            \item[R.] The valency of the end components should be the same
            \item[S.] The end components should have dissimilar electronegativities
        \end{enumerate}
        \hfill{\brak{2014-XE}}
        \begin{multicols}{4}
            \begin{enumerate}
                \item  P, Q, R \columnbreak
                \item  Q, R, S \columnbreak
                \item  R, S, P  \columnbreak
                \item  S, P, Q
            \end{enumerate}
        \end{multicols}


    \item The energy in $eV$ and the wavelength in $\mu$m, respectively, of the photon emitted when an electron
        in a hydrogen atom falls from $n = 4$ to $n = 2$ state is
        \hfill{\brak{2014-XE}}
        \begin{multicols}{4}
            \begin{enumerate}
                \item  3.0, 0.413 \columnbreak
                \item  2.55, 0.365 \columnbreak
                \item  2.75, 0.451  \columnbreak
                \item  2.55, 0.487
            \end{enumerate}
        \end{multicols}

    \item  The weight in $kg$ of gallium $\brak{Ga}$ to be mixed with arsenic $\brak{As}$ for obtaining 1.0 kg of gallium
        arsenide $\brak{GaAs}$ is
        %\raggedright{M_{Ga} = 69.72 g/mol\; M_{As} = 74.92 g/mol}
        \hfill{\brak{2014-XE}}

    \item Match the material in \textbf{Column I} with the property in \textbf{Column II}
        \begin{center}
            \begin{tabular}{| c | c |}
                \hline
                \textbf{Column I} & \textbf{Column II}\\
                P. $Pb\brak{Zr, Ti}O_3$ & 1. Shape memory alloy\\
                \hline
                Q. $Ni_{50}Ti_{50} $ & 2. Piezoelectric ceramic\\
                \hline
                R. $GaAs$ & 3. High temperature superconductor\\
                \hline
                S. $YBa_2Cu_3O_7$ & 4. Optoelectronic semiconductor\\
                \hline
            \end{tabular} 
        \end{center}


        \hfill{\brak{2014-XE}}
        \begin{enumerate}
                \begin{multicols}{2}
                \item P-4, Q-5, R-1, S-2 \columnbreak 
                \item P-3, Q-5, R-2, S-4 
                \end{multicols} 
                \begin{multicols}{2}
                \item P-5, Q-4, R-1, S-3 \columnbreak 
                \item P-4, Q-2, R-3, S-1
                \end{multicols}
        \end{enumerate}

        \hfill{\brak{2014-XE}}

    \item Relevant portion of a binary phase diagram of elements A and B is shown below. The mass fraction
        of liquid phase at $1000 ^ {\degree}C$ for an alloy with 15 wt.$\% B$ is
        \hfill{\brak{2014-XE}}
        \begin{center}
            \resizebox{0.5\textwidth}{!}{%
                \begin{circuitikz}
                    \tikzstyle{every node}=[font=\LARGE]
                    \draw [ line width=1.9pt](1.75,20.75) to[short] (1.75,11);
                    \draw [ line width=1.3pt](1.75,11) to[short] (10,11);
                    \draw [ line width=1.6pt](9.75,11) to[short] (13.25,11);
                    \draw [ line width=1.6pt](3.75,16.75) to[short] (9.75,16.75);
                    \draw [ line width=1.6pt](9.75,16.75) to[short] (12.75,16.75);
                    \draw [line width=1.6pt, short] (2.75,11) -- (3.75,16.75);
                    \draw [line width=1.6pt, short] (1.75,19.75) -- (3.75,16.75);
                    \draw [line width=1.6pt, short] (1.75,19.75) -- (12,16.75);
                    \draw [line width=1.6pt, dashed] (9.25,17.5) -- (9.25,11);
                    \draw [line width=1.6pt, dashed] (9.25,17.5) -- (1.75,17.5);
                    \draw [line width=1.6pt, dashed] (3.25,17.5) -- (3,11);
                    \node [font=\LARGE] at (6,19.5) {L};
                    \node [font=\LARGE] at (3.7,18.5) {$L + \alpha$};
                    \node [font=\LARGE] at (2.5,15) {$\alpha$};
                    \node [font=\LARGE] at (3,10.5) {10};
                    \node [font=\LARGE] at (9,10.5) {30};
                    \draw [line width=1.6pt, ->, >=Stealth] (3.75,9.5) -- (9.5,9.5);
                    \node [font=\LARGE] at (10.5,9.5) {$    wt. \% B$};
                    \node [font=\LARGE] at (0.5,17.5) {1000};
                    \node [font=\LARGE, rotate around={90:(0,0)}] at (0.5,15) {Temperature in$^{\degree}C\quad\quad$              };
                \end{circuitikz}

                }%
        \end{center}
    \item The expected diffraction angle $\brak{\textnormal{in degrees}}$for the first order reflection from the $\brak{113}$ set of planes
        for face centered cubic Pt $\brak{\textnormal{lattice parameter = 0.392 nm}}$ using monochromatic radiation of
        wavelength 0.1542 nm is
        \hfill{\brak{2014-XE}}
    \item The diffusion coefficients of i$Mg$ in $Al$ at 500 and $550^{\degree}C$ are $1.9\times 10^{-13}$
        and $5.8\times 10^{-13} m^2/s$ respectively. The activation energy for diffusion of $Mg$ in $Al$ in $kJ/mol$ is
        \hfill{\brak{2014-XE}}

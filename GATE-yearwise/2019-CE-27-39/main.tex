\iffalse
\title{2019-CE-27-39}
\author{EE24BTECH11001 -  ADITYA TRIPATHY}
\section{ce}
\chapter{2019}
\fi
    \item 
        A one-dimensional domain is discretized into $N$ sub-domains of width $\Delta x$ with node
        numbers $i = 0, 1, 2, 3, \cdots N$. If the time scale is discretized in steps of $\Delta t$
        , the forward time and centered-space finite difference approximation at $i^{th}$ node and
        $n^{th}$ time step, for the partial differential equaivalent 
        $\frac{\partial v}{\partial t} = \beta \frac{\partial ^2 v}{\partial x^2}$
        
        \hfill{\brak{2019-CE}}
        \begin{enumerate}
                \item $\frac{v_i ^{\brak{n+1}} - v_i ^{\brak{n}}}{\Delta t} = \beta \sbrak{\frac{v_{i+1}^{\brak{n}} - 2v_i ^{\brak{n}} + v_{i-1}^{\brak{n}}}{\brak{\Delta x}^2}}$
                \item $\frac{v_{i+1} ^{\brak{n+1}} - v_i ^{\brak{n}}}{\Delta t} = \beta \sbrak{\frac{v_{i+1}^{\brak{n}} - 2v_i ^{\brak{n}} + v_{i-1}^{\brak{n}}}{2\brak{\Delta x}}}$
                \item $\frac{v_i ^{\brak{n}} - v_i ^{\brak{n-1}}}{\Delta t} = \beta \sbrak{\frac{v_{i+1}^{\brak{n}} - 2v_i ^{\brak{n}} + v_{i-1}^{\brak{n}}}{\brak{\Delta x}^2}}$
                \item $\frac{v_i ^{\brak{n}} - v_i ^{\brak{n-1}}}{2\Delta t} = \beta \sbrak{\frac{v_{i+1}^{\brak{n}} - 2v_i ^{\brak{n}} + v_{i-1}^{\brak{n}}}{2\brak{\Delta x}}}$
            \end{enumerate}
    \item A rectangular open channel has a width of 5$m$ and a bed slope of 0.001. For a uniform
        flow of depth $2m$, the velocity is $2m/s$. The Manning's roughness coefficient for this channel
        is

        \hfill{\brak{2019-CE}}
        \begin{multicols}{4}
            \begin{enumerate}
                \item 0.002
                    \columnbreak
                \item 0.017
                    \columnbreak
                \item 0.033
                    \columnbreak
                \item 0.050
            \end{enumerate}
        \end{multicols}
    \item For the following statements:
        \begin{enumerate}
            \item[P.] The lateral stress in the soil while being tested in an oedometer is always at rest. 
            \item[Q.] For a perfectly rigid strip footing at deeper depths in a sand deposit, the vertical
                normal contact stress at the footing edge is greater that at its centre.
            \item[R.] The corrections for overburden pressure and dilatency are not applied to measures 
                SPT-N values in case of clay deposits.
        \end{enumerate}
        The corre combination of statements is
        \hfill{\brak{2019-CE}}
            \begin{enumerate}
                \item P-TRUE, Q-TRUE, R-TRUE
                \item P-FALSE, Q-FALSE, R-TRUE
                \item P-TRUE, Q-TRUE, R-FALSE
                \item P-FALSE, Q-FALSE, R-FALSE
            \end{enumerate}
    \item Consider two functions: $x = \Psi \ln\phi$ and $y = \phi \ln\Psi$. Which of the following
        is the correct expression for $\frac{\partial \Psi}{\partial x}$?
        \hfill{\brak{2019-CE}}
        \begin{multicols}{4}
            \begin{enumerate}
                \item $\frac{x \ln \Psi}{\ln\phi\ln\Psi - 1}$
                    \columnbreak
                \item $\frac{x \ln \phi}{\ln\phi\ln\Psi - 1}$
                    \columnbreak
                \item $\frac{ \ln \phi}{\ln\phi\ln\Psi - 1}$
                    \columnbreak
                \item $\frac{\ln \Psi}{\ln\phi\ln\Psi - 1}$
            \end{enumerate}
        \end{multicols}
    \item The cross-section of a built-up wooden beam as shown in the figure $\brak{\text{not drawn to scale}}$
        is subjected to a vertical shear force of $8kN$. The beam is a symmetrical about the neutral axis
        \brak{N.A.} shown, and the moment of inertia about $\brak{N.A.}$ is $1.5 \times 10^9 mm^4$.
        Considering that the nails at the location $P$ are spaced longitudinally (along the lenghth of the beam)
        at $60mm$, each of the nails at $P$ will be subjected to the shear force of 
        \begin{center}
            \resizebox{0.3\textwidth}{!}{
                \begin{circuitikz}
                    \tikzstyle{every node}=[font=\small]
                    \draw [ color={rgb,255:red,255; green,255; blue,255} ] (1.75,16.5) rectangle (0.5,17.25);
                    \draw [short] (1.75,17.25) -- (1.75,16.25);
                    \draw [short] (1.75,17.25) -- (8,17.25);
                    \draw [short] (1.75,16.5) -- (7.75,16.5);
                    \draw [short] (8,17.25) -- (8,16.25);
                    \draw [short] (7.75,16.5) -- (8,16.5);
                    \draw [short] (2.5,17.25) -- (2.5,16.25);
                    \draw [short] (1.75,16.25) -- (2.5,16.25);
                    \draw [short] (7.25,17.25) -- (7.25,16.25);
                    \draw [short] (7.25,16.25) -- (8,16.25);
                    \draw [short] (4.25,16.5) -- (4.25,12.75);
                    \draw [short] (5.5,16.5) -- (5.5,12.75);
                    \draw [short] (2,12.75) -- (8,12.75);
                    \draw [short] (2,12.75) -- (2,11.75);
                    \draw [short] (2,11.75) -- (8,11.75);
                    \draw [short] (8,11.75) -- (8,12.75);
                    \draw [short] (2,12.75) -- (2,13.25);
                    \draw [short] (2,13.25) -- (2.75,13.25);
                    \draw [short] (2.75,13.25) -- (2.75,12.75);
                    \draw [short] (2.75,12.75) -- (2.75,11.75);
                    \draw [short] (7.25,11.75) -- (7.25,13);
                    \draw [short] (7.25,11.75) -- (7.25,13.25);
                    \draw [short] (7.25,13.25) -- (8,13.25);
                    \draw [short] (8,13.25) -- (8,12.25);
                    \draw [<->, >=Stealth] (7.25,17.5) -- (8,17.5);
                    \draw [<->, >=Stealth] (1.75,17.5) -- (2.5,17.5);
                    \node [font=\small] at (4.75,18) {};
                    \node [font=\small] at (2,18) {50};
                    \node [font=\small] at (7.5,18) {50};
                    \node [font=\small] at (6.25,15) {};
                    \node [font=\small] at (4.5,10.5) {Text};
                    \draw [short] (3,17.25) -- (3,17.5);
                    \draw [short] (3,16.5) -- (3,16.25);
                    \node [font=\small] at (3.5,16.25) {50};
                    \draw [ fill={rgb,255:red,13; green,12; blue,12} ] (4.5,11.625) -- (4.75,9.625) -- (5,11.625) -- (4.75,13.625) -- cycle;
                    \draw [ color={rgb,255:red,255; green,255; blue,255} , fill={rgb,255:red,255; green,250; blue,250}] (3.75,11.75) rectangle (5.75,9.5);
                    \draw [short] (3.75,11.75) -- (6,11.75);
                    \draw [ fill={rgb,255:red,5; green,5; blue,5} ] (3.25,12.5) -- (2,12.375) -- (0.75,12.5) -- (2,12.625) -- cycle;
                    \draw [ fill={rgb,255:red,5; green,5; blue,5} ] (3,17) -- (1.75,16.875) -- (0.5,17) -- (1.75,17.125) -- cycle;
                    \draw [ fill={rgb,255:red,5; green,5; blue,5} ] (6.5,17) -- (8,16.875) -- (9.5,17) -- (8,17.125) -- cycle;
                    \draw [ fill={rgb,255:red,5; green,5; blue,5} ] (5,17.375) -- (4.75,18.625) -- (4.5,17.375) -- (4.75,16.125) -- cycle;
                    \draw [ fill={rgb,255:red,5; green,5; blue,5} ] (9.25,12.5) -- (8,12.375) -- (6.75,12.5) -- (8,12.625) -- cycle;
                    \draw [ fill={rgb,255:red,5; green,5; blue,5} ] (1.75,17.25) rectangle (3.75,17.25);
                    \draw [ color={rgb,255:red,255; green,255; blue,255} , fill={rgb,255:red,251; green,249; blue,249}] (1.75,17.25) rectangle (0.5,16.5);
                    \draw [ color={rgb,255:red,255; green,255; blue,255} , fill={rgb,255:red,255; green,254; blue,255}] (8,17.25) rectangle (9.5,16.5);
                    \draw [ color={rgb,255:red,255; green,255; blue,255} , fill={rgb,255:red,255; green,254; blue,255}] (8,12.75) rectangle (9.5,12);
                    \draw [ color={rgb,255:red,255; green,255; blue,255} , fill={rgb,255:red,255; green,254; blue,255}] (0.5,12.75) rectangle (2,12);
                    \draw [ color={rgb,255:red,255; green,255; blue,255} , fill={rgb,255:red,255; green,253; blue,255}] (3.5,18.75) rectangle (5.5,17.25);
                    \draw [short] (2.5,17.25) -- (8,17.25);
                    \draw [short] (8,17.25) -- (8,16.5);
                    \draw [short] (8,12.75) -- (8,12);
                    \draw [short] (2,12.75) -- (2,11.75);
                    \draw [short] (1.75,16.5) -- (1.75,17.25);
                    \draw [<->, >=Stealth] (2.5,17.5) -- (7,17.5);
                    \draw [<->, >=Stealth] (8.5,17.5) -- (8.5,11.75);
                    \draw [dashed] (2.75,14.75) -- (6.5,14.75);
                    \node [font=\small] at (9,14.25) {400};
                    \node [font=\small] at (3.5,15) {N.A.};
                    \node [font=\small] at (4.75,18) {300};
                    \node [font=\small] at (1,16.75) {100};
                    \node [font=\small] at (1.5,12.25) {P};
                    \draw [<->, >=Stealth] (1.5,17.25) -- (1.5,16.25);
                \end{circuitikz}
                } 
            \end{center}
            \hfill{\brak{2019-CE}}
            \begin{multicols}{4}
                \begin{enumerate}
                    \item 0.002
                        \columnbreak
                    \item 0.017
                        \columnbreak
                    \item 0.033
                        \columnbreak
                \item 0.050
            \end{enumerate}
        \end{multicols}


    \item The rigid-jointed plane frame $QRS$ shown in the figure is subjected to a load $P$ at the 
        joint $R$. Let the axial deformations in the frame be neglected. If the support $S$ undergoes
        a settlement of $\Delta = \frac{PL^3}{\beta EI}$, the vertical reaction at the support $S$
        will become zero when $\beta$ is equal to 
        \begin{center}
            \resizebox{0.3\textwidth}{!}{
                \begin{circuitikz}
                    \tikzstyle{every node}=[font=\small]
                    \draw [short] (2.5,18.25) -- (2.5,18.25);
                    \draw [short] (1.5,18.75) -- (1.5,17.75);
                    \draw [short] (1.5,18.25) -- (6.5,18.25);
                    \draw [short] (6.5,18.25) -- (6.5,13);
                    \draw [short] (5.75,13) -- (7.25,13);
                    \draw [<->, >=Stealth] (7.75,18.5) -- (7.75,13);
                    \draw [short] (5.75,13) -- (5.5,12.75);
                    \draw [short] (6,13) -- (5.75,12.75);
                    \draw [short] (6.25,13) -- (6,12.75);
                    \draw [short] (6.5,13) -- (6.25,12.75);
                    \draw [short] (6.75,13) -- (6.5,12.75);
                    \draw [short] (7,13) -- (6.75,12.75);
                    \draw [short] (7.25,13) -- (7,12.75);
                    \draw [short] (1.5,17.75) -- (1.25,17.5);
                    \draw [short] (1.5,18) -- (1.25,17.75);
                    \draw [short] (1.5,18.25) -- (1.25,18);
                    \draw [short] (1.5,18.5) -- (1.25,18.25);
                    \draw [short] (1.5,18.75) -- (1.25,18.5);
                    \draw [->, >=Stealth] (6.5,19) -- (6.5,18.5);
                    \node [font=\small] at (2,19) {\small Q};
                    \node [font=\small] at (6.5,19.5) {\small P};
                    \node [font=\small] at (8.5,15.75) {\small L};
                    \node [font=\small] at (6,15.75) {\small EI};
                    \node [font=\small] at (4,19) {\small EI};
                    \node [font=\small] at (6,13.5) {\small S};
                    \draw [<->, >=Stealth] (1.75,12.25) -- (6.5,12.25);
                    \node [font=\small] at (6,18.5) {\small R};
                    \node [font=\small] at (3.75,11.75) {\small L};
                \end{circuitikz}
            } 
        \end{center}
        \hfill{\brak{2019-CE}}
        \begin{multicols}{4}
            \begin{enumerate}
                \item 0.1
                    \columnbreak
                \item 3.0
                    \columnbreak
                \item 7.5
                    \columnbreak
                \item 48.0
            \end{enumerate}
        \end{multicols}

    \item If the section shown in the figure turns from fully-elastic to fully-plastic, the depth of 
        neutral axis $\brak{\text{N.A.}}, \vec{y},$ decreases by 
        \begin{center}
            \resizebox{0.3\textwidth}{!}{
                \begin{circuitikz}
                    \tikzstyle{every node}=[font=\small]
                    \draw [short] (2.75,17.5) -- (2.75,16.5);
                    \draw [short] (2.75,17.5) -- (7.25,17.5);
                    \draw [short] (7.25,17.5) -- (7.25,16.5);
                    \draw [short] (2.75,16.5) -- (4.5,16.5);
                    \draw [short] (7.25,16.5) -- (5.75,16.5);
                    \draw [short] (5.75,16.5) -- (5.75,13.25);
                    \draw [short] (4.5,16.5) -- (4.5,11.75);
                    \draw [short] (5.75,13.25) -- (5.75,11.75);
                    \draw [short] (5.75,11.75) -- (4.5,11.75);
                    \draw [<->, >=Stealth] (7,16.25) -- (7,11.75);
                    \draw [<->, >=Stealth] (8,17.5) -- (8,16.5);
                    \draw [<->, >=Stealth] (2.75,18) -- (7.75,18);
                    \draw [<->, >=Stealth] (2.25,17.5) -- (2.25,14.5);
                    \draw [dashed] (3,14.5) -- (7,14.5);
                    \node [font=\small] at (1.25,16.25) {y};
                    \node [font=\small] at (5,18.5) {60};
                    \node [font=\small] at (3.5,15) {N.A.};
                    \node [font=\small] at (7.75,14.75) {60};
                    \node [font=\small] at (9,17) {5};
                    \draw [<->, >=Stealth] (4.5,11.25) -- (5.75,11.25);
                    \node [font=\small] at (5,10.5) {5};
                \end{circuitikz}
            } 
        \end{center}
        \hfill{\brak{2019-CE}}
        \begin{multicols}{4}
            \begin{enumerate}
                \item $10.75mm$
                    \columnbreak
                \item $12.25mm$
                    \columnbreak
                \item $13.75mm$
                    \columnbreak
                \item $15.25mm$
            \end{enumerate}
        \end{multicols}

    \item Sedimentation basin in a water treatment plant is designed for a new flow rate of $0.2m^3/s$.
        The basin is rectangular with a lenght of 32$m$, width of $8m$, and depth of $4m$. Assume that the
        settling velocity of these particles is governed by Stoke's Law. Given: density of the particles = 
        $2.5g/cm^3$, density of water = $1 g/cm^3$, dynamic viscosity of water = $0.01 g/cm s$, gravitational
        acceleration = $980 cm/s$. If the incoming water contains particles of diameter $25 \mu m$ 
        $\brak{\text{spherical and uniform}}$, the removal efficiency of these particles is
        \hfill{\brak{2019-CE}}
     \begin{multicols}{4}
            \begin{enumerate}
                \item $51\%$
                    \columnbreak
                \item $65\%$
                    \columnbreak
                \item $78\%$
                    \columnbreak
                \item $100\%$
            \end{enumerate}
        \end{multicols}

        
    \item A survey line was measured to be $285.5m$ with a tape having a nominal length of 30m. On 
        checking, the true length was found to be $0.05$ too short. If the line lay on a slope of 
        1 in 10, the reduced length $\brak{\text{horizontal}}$ of the line for plotting of survey work
        would be
        \hfill{\brak{2019-CE}}
        \begin{multicols}{4}
            \begin{enumerate}
                \item 283.6m
                    \columnbreak
                \item 284.5m
                    \columnbreak
                \item 285.0m
                    \columnbreak
                \item 285.6m
            \end{enumerate}
        \end{multicols}
    \item A 16$mm$ thick gusset is connected to the 12$mm$ thick flange of an I-section using fillet welds on
        both sides as shown in the figure (not drawn to scale).The gusset plate is subjected to a point load 
        of $350kN$ acting at a distance of $100mm$ from the flange plate. Size of the fillet weld is 10$mm$.
        The maximum resultant stress (in MPa, round off to 1 decimal place) on the fillet weld along the vertical plane would be
        \begin{center}
            \resizebox{0.5\textwidth}{!}{
                \begin{circuitikz}
                    \tikzstyle{every node}=[font=\small]
                    \draw [short] (2,16.75) -- (2,9.75);
                    \draw [short] (2,16.75) -- (2.5,16.75);
                    \draw [short] (2.5,16.75) -- (2.5,9.75);
                    \draw [short] (2.5,9.75) -- (2,9.75);
                    \draw [short] (4,16.75) -- (4.5,16.75);
                    \draw [short] (4.5,16.75) -- (4.5,9.75);
                    \draw [short] (4.5,9.75) -- (4,9.75);
                    \draw [short] (4,9.75) -- (4,16.75);
                    \draw [dashed] (1,9.75) -- (2.75,9.75);
                    \draw [dashed] (2.75,9.75) -- (3,10.5);
                    \draw [dashed] (3,10.5) -- (3.25,9);
                    \draw [dashed] (3.25,9) -- (3.5,9.75);
                    \draw [dashed] (3.5,9.75) -- (5.75,9.75);
                    \draw [dashed] (0.75,16.75) -- (2.75,16.75);
                    \draw [dashed] (2.75,16.75) -- (3.25,17.5);
                    \draw [dashed] (3.25,17.5) -- (3.25,16);
                    \draw [dashed] (3.25,16) -- (3.75,16.75);
                    \draw [dashed] (3.75,16.75) -- (5.25,16.75);
                    \draw [ fill={rgb,255:red,0; green,0; blue,0} ] (4.5,15) rectangle (4.75,11.5);
                    \draw [short] (4.75,11.5) -- (6,11.5);
                    \draw [short] (6,11.5) -- (7,12.5);
                    \draw [short] (7,12.5) -- (7,15);
                    \draw [short] (7,15) -- (4.75,15);
                    \draw [->, >=Stealth] (6,17) -- (6,15);
                    \draw [<->, >=Stealth] (4.5,15.25) -- (5.75,15.25);
                    \draw [->, >=Stealth] (6,10.5) -- (5,12.25);
                    \draw [->, >=Stealth] (8.25,13.75) -- (7,12.75);
                    \draw [<->, >=Stealth] (3.75,15) -- (3.75,11.5);
                    \draw [->, >=Stealth] (5.5,10.25) -- (4.5,10.5);
                    \draw [short] (10.5,17) -- (10.5,10);
                    \draw [short] (12.5,17) -- (12.5,10);
                    \draw [dashed] (9.5,17) -- (10.75,17);
                    \draw [dashed] (10.75,17) -- (11.25,17.75);
                    \draw [dashed] (11.25,17.75) -- (11.75,16.5);
                    \draw [dashed] (11.75,16.5) -- (12,17);
                    \draw [dashed] (12,17) -- (13.25,17);
                    \draw [dashed] (9.5,10) -- (10.75,10);
                    \draw [dashed] (10.75,10) -- (11.25,11);
                    \draw [dashed] (11.25,11) -- (11.5,9.25);
                    \draw [dashed] (11.5,9.25) -- (12,10);
                    \draw [dashed] (12,10) -- (13.5,10);
                    \draw  (11.25,15) rectangle (11.75,12);
                    \draw [short] (11.25,15) -- (11,14.75);
                    \draw [short] (11.25,14.75) -- (11,14.5);
                    \draw [short] (11.25,14.5) -- (11,14.25);
                    \draw [short] (11.25,14) -- (11.25,14.25);
                    \draw [short] (11.25,14.25) -- (11,14);
                    \draw [short] (11.25,14) -- (11,13.75);
                    \draw [short] (11.25,13.75) -- (11,13.5);
                    \draw [short] (11.25,13.5) -- (11,13.25);
                    \draw [short] (11.25,13.25) -- (11,13);
                    \draw [short] (11.25,13) -- (11,12.75);
                    \draw [short] (11.25,12.75) -- (11,12.5);
                    \draw [short] (11.25,12.5) -- (11,12.25);
                    \draw [short] (11.25,12.25) -- (11,12);
                    \draw [short] (11.75,15) -- (12,15.25);
                    \draw [short] (11.75,14.75) -- (12,15);
                    \draw [short] (11.75,14.5) -- (12,14.75);
                    \draw [short] (11.75,14.25) -- (12,14.5);
                    \draw [short] (11.75,14) -- (12,14.25);
                    \draw [short] (11.75,13.75) -- (12,14);
                    \draw [short] (11.75,13.5) -- (12,13.75);
                    \draw [short] (11.75,13.25) -- (12,13.5);
                    \draw [short] (11.75,13) -- (12,13.25);
                    \draw [short] (11.75,12.75) -- (12,13);
                    \draw [short] (11.75,12.5) -- (12,12.75);
                    \draw [short] (11.75,12.25) -- (12,12.5);
                    \draw [short] (11.75,12) -- (12,12.25);
                    \draw [->, >=Stealth] (12.75,15) -- (12.75,12);
                    \draw [->, >=Stealth] (12.75,12) -- (12.75,15);
                    \draw [->, >=Stealth] (11.5,11.5) -- (11,12);
                    \draw [->, >=Stealth] (11.5,11.5) -- (12,12);
                    \node [font=\small] at (3.25,13.5) {500mm};
                    \node [font=\small] at (3.25,10.75) {I-Section};
                    \node [font=\small] at (5,15.75) {$\quad 100mm$};
                    \node [font=\small] at (6,17.5) {350 kN};
                    \node [font=\small] at (11.5,11.25) {$\text{Fillet Weld}$};
                    \node [font=\small] at (13.25,13.75) {$\quad 500mm$};
                    \node [font=\small] at (5.75,10) {$\quad \quad \text{Flange(12mm thick)}$};
                    \node [font=\small] at (6.5,10.5) {$\quad \quad \text{Fillet weld}$};
                    \node [font=\small] at (8.5,14) {$\text{16mm gusset plate}$};
                \end{circuitikz}
                } 
            \end{center}
            \hfill{\brak{2019-CE}}
        \item The network of a small construction project awarded to a contractor is shown in the following
            figure. The normal duration, crash duration, normal cost, and crash cost of all activities are
            shown in the table. The indirect cost incurred by the contractor is $INR 5000$ per day.
            \begin{center}
                \resizebox{0.5\textwidth}{!}{
                    \begin{circuitikz}
                        \tikzstyle{every node}=[font=\small]
                        \draw  (0,15.25) circle (0.25cm);
                        \draw  (2,15.25) circle (0.25cm);
                        \draw  (4,16.75) circle (0.25cm);
                        \draw  (6,15.25) circle (0.25cm);
                        \draw  (4,13.5) circle (0.25cm);
                        \draw [->, >=Stealth] (0.25,15.25) -- (1.75,15.25);
                        \draw [->, >=Stealth] (2.25,15.5) -- (3.75,16.75);
                        \draw [->, >=Stealth] (4.25,16.75) -- (5.75,15.5);
                        \draw [->, >=Stealth] (2.25,15) -- (3.75,13.5);
                        \draw [->, >=Stealth] (4.25,13.5) -- (5.75,15);
                        \node [font=\small] at (0.5,14.75) {};
                        \node [font=\small] at (0.5,14.75) {};
                        \node [font=\small] at (1,14.25) {};
                        \node [font=\small] at (1,14.25) {};
                        \node [font=\small] at (6.5,14.75) {};
                        \draw  (8,15.25) circle (0.25cm);
                        \draw [->, >=Stealth] (6.25,15.25) -- (7.75,15.25);
                        \draw [->, >=Stealth] (2.25,15.25) -- (5.75,15.25);
                        \node [font=\small] at (4,15.5) {S};
                        \node [font=\small] at (1,15.5) {P};
                        \node [font=\small] at (2.5,16) {Q};
                        \node [font=\small] at (5,16.5) {U};
                        \node [font=\small] at (2.75,14) {R};
                        \node [font=\small] at (5.25,14) {T};
                        \node [font=\small] at (6.75,15.5) {V};
                    \end{circuitikz}
                } 
            \end{center}

            \begin{tabular}{|c|c|c|c|c|}
                \hline
                Activity & Normal Duration (days) & Crash Duration (days) & Normal Cost (INR) & Crash Cost (INR)\\
                \hline
                P & 6 & 4 & 15000 & 25000\\
                \hline
                Q & 5 & 2 & 6000 & 12000\\
                \hline
                R & 5 & 3 & 8000 & 9500\\
                \hline
                S & 6 & 3 & 7000 & 10000\\
                \hline
                T & 3 & 2 & 6000 & 9000\\
                \hline
                U & 2 & 1 & 4000 & 6000\\
                \hline
                V & 4 & 2 & 20000 & 28000\\
                \hline

            

        \end{tabular}
        If the project is tabulated for completion in 16 days, the total cost (in INR) to be incurred 
        by the contractor would be
        \hfill{\brak{2019-CE}}
        
    \item A box measuring $50cm \times 50cm \times 50cm$ is filled to the top with dry coarse
        aggregate of mass 187.5$kg$. The water absorption and specific gravity of aggregate are 
        $0.5\%$ and 2.5 respectivley. The maximum quantity of water (in kg, round off to 2 decimal places)
        required to fill the box completely
        \hfill{\brak{2019-CE}}

    \item A portal frame shown (not drawn to scale) has a hinge support at joint $P$ and a roller
        support at joint $R$. A point load of $5kN$ is acting at joint $R$ in the horizontal direction. 
        The flexual rigidity, $EI$, of each member is $10^6 hN/m^2$ . Under the applied load, the horizontal
        displacement (in mm, round off to 1 decimal place) of joint $R$ would be
        \begin{center}
            \resizebox{0.3\textwidth}{!}{
                \begin{circuitikz}
                    \tikzstyle{every node}=[font=\small]
                    \draw [short] (1.5,18.25) -- (1.25,17.75);
                    \draw [short] (1.25,17.75) -- (1.75,17.75);
                    \draw [short] (1.75,17.75) -- (1.5,18.25);
                    \draw [short] (1.5,18.25) -- (5.25,18.25);
                    \draw [short] (5.25,18.25) -- (5.25,14);
                    \draw [short] (5.25,14) -- (4.75,13.25);
                    \draw [short] (4.75,13.25) -- (6,13.25);
                    \draw [short] (6,13.25) -- (5.25,14);
                    \draw  (5,13) circle (0.25cm);
                    \draw  (5.75,13) circle (0.25cm);
                    \draw [short] (4.25,12.75) -- (6.5,12.75);
                    \draw [short] (4.25,12.75) -- (4.25,12.5);
                    \draw [short] (4.5,12.75) -- (4.25,12.5);
                    \draw [short] (4.75,12.75) -- (4.5,12.5);
                    \draw [short] (5,12.75) -- (4.75,12.5);
                    \draw [short] (5.25,12.75) -- (5,12.5);
                    \draw [short] (5.5,12.75) -- (5.25,12.5);
                    \draw [short] (5.75,12.75) -- (5.5,12.5);
                    \draw [short] (6,12.75) -- (5.75,12.5);
                    \draw [short] (6,12.75) -- (6.25,12.75);
                    \draw [short] (6.25,12.75) -- (6,12.5);
                    \draw [short] (6.5,12.75) -- (6.25,12.5);
                    \draw [short] (1.25,17.75) -- (1,17.5);
                    \draw [short] (1.5,17.75) -- (1.25,17.5);
                    \draw [short] (1.75,17.75) -- (1.5,17.5);
                    \node [font=\small] at (1.5,18.5) {P};
                    \node [font=\small] at (3,18.5) {5m};
                    \node [font=\small] at (5,18.5) {Q};
                    \node [font=\small] at (4.75,16) {EI};
                    \node [font=\small] at (4.75,14) {R};
                    \node [font=\small] at (6.75,14) {50 kN};
                    \draw [->, >=Stealth] (5.25,14) -- (6.25,14);
                    \node [font=\small] at (5.75,16) {10m};
                \end{circuitikz}
            } 
        \end{center}
        \hfill{\brak{2019-CE}}


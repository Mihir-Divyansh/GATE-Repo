\iffalse
\author{EE24BTECH11062}
\section{ee}
\chapter{2013}
\fi

\item
\centering
\resizebox{1\textwidth}{!}{%
\begin{circuitikz}
\tikzstyle{every node}=[font=\small]
\draw (-1.75,16.75) to[american voltage source,l={ \small $V_{dc}$}] (-1.75,12.25);
\draw (0,14.75) to[Tnmos, transistors/scale=1.02] (0,16.75);
\draw (0,12.25) to[Tnmos, transistors/scale=1.02] (0,14.75);
\draw (1,15.25) to[D,l={ \small $Q_1      D_1$}] (1,16.5);
\draw (1,12.75) to[D,l={ \small $Q_4      D_4$}] (1,14.25);
\draw (0,14.75) to[short] (3.5,14.75);
\draw (0,14.25) to[short] (1,14.25);
\draw (0,12.5) to[short] (1,12.5);
\draw (1,13) to[short] (1,12.5);
\draw (0,15.25) to[short] (1,15.25);
\draw (0,16.5) to[short] (1,16.5);
\draw (2.5,14.75) to[Tnmos, transistors/scale=1.02] (2.5,16.75);
\node at (0,14.75) [circ] {};
\draw (2.5,12.25) to[Tnmos, transistors/scale=1.02] (2.5,14.75);
\draw (3.25,15) to[D,l={ \small $Q_3  D_3$}] (3.25,16.5);
\draw (3.25,12.75) to[D,l={ \small $Q_2   D_2$}] (3.25,14.25);
\draw (2.5,16.5) to[short] (3.25,16.5);
\draw (2.5,15) to[short] (3.25,15);
\draw (2.5,14.25) to[short] (3.25,14.25);
\draw (2.5,12.75) to[short] (3.25,12.75);
\draw (2.5,14.25) to[short] (3,14.25);
\draw (2.5,14.5) to[short] (3.75,14.5);
\draw (3.5,14.75) to[short] (4,14.75);
\draw (2.5,14.5) to[short] (4,14.5);
\draw (4,14.75) to[short] (4,15.5);
\draw (4,15.5) to[short] (4.5,15.5);
\draw (4,14.5) to[short] (4,13.5);
\draw (4,13.5) to[short] (4.5,13.5);
\draw (4.5,15.5) to[L,l={ \small $L$} ] (7,15.5);
\draw (7,15.5) to[R,l={ \small R}] (7,13.5);
\draw (4.25,13.5) to[short] (7,13.5);
\draw (-1.75,12.25) to[short] (2.5,12.25);
\draw (-1.75,16.75) to[short] (2.5,16.75);
\node at (2.5,14.5) [circ] {};
\draw [->, >=Stealth] (5,15) -- (6.5,15)node[above] {$i_0$};
\end{circuitikz}
}
If the base impedance and the line to line base voltage are $100\ohm$ and $100kV$, respectively, then the real power in MW delivered by the generator connected at the slack bus is \hfill{[2013]}
\begin{multicols}{4}
    a) -10\\
    b) 0\\
    c) 10\\
    d)  20
\end{multicols}
Statement for Linked Answer Question 54 and 55:\\
The Voltage Source Inverter \brak{VSI} shown in the figure below is switched to provide a 50 Hz, square-wave ac output voltage \brak{v_o} across an R-L load. Reference polarity of $v_o$ and reference direction of the output current $i_o$ are indicated i the figure. It is given that $R=3\ohm$, $L=9.55 mH$.\\
\begin{center}
   
\centering
\resizebox{1\textwidth}{!}{%
\begin{circuitikz}
\tikzstyle{every node}=[font=\small]
\draw (-1.75,16.75) to[american voltage source,l={ \small $V_{dc}$}] (-1.75,12.25);
\draw (0,14.75) to[Tnmos, transistors/scale=1.02] (0,16.75);
\draw (0,12.25) to[Tnmos, transistors/scale=1.02] (0,14.75);
\draw (1,15.25) to[D,l={ \small $Q_1      D_1$}] (1,16.5);
\draw (1,12.75) to[D,l={ \small $Q_4      D_4$}] (1,14.25);
\draw (0,14.75) to[short] (3.5,14.75);
\draw (0,14.25) to[short] (1,14.25);
\draw (0,12.5) to[short] (1,12.5);
\draw (1,13) to[short] (1,12.5);
\draw (0,15.25) to[short] (1,15.25);
\draw (0,16.5) to[short] (1,16.5);
\draw (2.5,14.75) to[Tnmos, transistors/scale=1.02] (2.5,16.75);
\node at (0,14.75) [circ] {};
\draw (2.5,12.25) to[Tnmos, transistors/scale=1.02] (2.5,14.75);
\draw (3.25,15) to[D,l={ \small $Q_3  D_3$}] (3.25,16.5);
\draw (3.25,12.75) to[D,l={ \small $Q_2   D_2$}] (3.25,14.25);
\draw (2.5,16.5) to[short] (3.25,16.5);
\draw (2.5,15) to[short] (3.25,15);
\draw (2.5,14.25) to[short] (3.25,14.25);
\draw (2.5,12.75) to[short] (3.25,12.75);
\draw (2.5,14.25) to[short] (3,14.25);
\draw (2.5,14.5) to[short] (3.75,14.5);
\draw (3.5,14.75) to[short] (4,14.75);
\draw (2.5,14.5) to[short] (4,14.5);
\draw (4,14.75) to[short] (4,15.5);
\draw (4,15.5) to[short] (4.5,15.5);
\draw (4,14.5) to[short] (4,13.5);
\draw (4,13.5) to[short] (4.5,13.5);
\draw (4.5,15.5) to[L,l={ \small $L$} ] (7,15.5);
\draw (7,15.5) to[R,l={ \small R}] (7,13.5);
\draw (4.25,13.5) to[short] (7,13.5);
\draw (-1.75,12.25) to[short] (2.5,12.25);
\draw (-1.75,16.75) to[short] (2.5,16.75);
\node at (2.5,14.5) [circ] {};
\draw [->, >=Stealth] (5,15) -- (6.5,15)node[above] {$i_0$};
\end{circuitikz}
}
\end{center}

 \item In the interval when $v_o<0$ and $i_o>0$ the pair of devices which conducts the load current is \hfill{[2013]}

 \begin{multicols}{4}
     a) Q1, Q2\\
     b) Q3, Q4\\
     c) D1, D2\\
     d) D3, D4
 \end{multicols}
 
 \item Appropriate transition i.e, Zero Voltage Switching \brak{ZVS}/Zero Current Switching \brak{ZCS} of the IGBTs during turn-on/turn-off is \hfill{[2013]}
 \begin{enumerate}
     \item ZVS during turn-off\\
     \item ZVS during turn-on\\
     \item ZCS during turn-off\\
     \item ZCS during turn-on
 \end{enumerate}
 General Aptitude \brak{GA} Questions
\item They were requested not to quarrel with others.\\
Which of the following options is the closest in meaning to the world quarrel? 
\hfill{[2013]}
\begin{multicols}{4}
    a) make out\\
    b) call out\\
    c) dig out\\
    d) fall out
\end{multicols}

\item In the summer of 2012, in New Delhi, the mean temperature of Monday to Wednesday was $41\degree C$. If the temperature on Thursday was 15\% higher than that of Monday, then the temperature in $\degree C$ on Thursday was  \hfill{[2013]}
\begin{multicols}{4}
    a) 40\\
    b) 43\\
    c) 46\\
    d) 49
\end{multicols}
\item Complete the sentence:\\
Dare \rule{1in}{0.4pt} mistakes.
\hfill{[2013]}
\begin{multicols}{4}
    a) commit\\
    b) to commit\\
    c) committed\\
    d) committing
\end{multicols}

\item Choose the grammatically CORRECT sentence:\hfill{[2013]}
\begin{enumerate}
    \item Two and two add four
    \item Two and two become four
    \item Two and two are four
    \item Two and two make four
\end{enumerate}
\item Statement: You can always give me a ring whenever you need.\\
Which one of the following is the best inference from the above statement?\hfill{[2013]}
\begin{enumerate}
    \item Because I have a nice caller tune.
    \item Because I have a better telephone facility.
    \item Because a friend in need is a friend indeed.
    \item Because you need not pay towards the telephone bills when you give me a ring.
\end{enumerate}

\item What is the chance that a leap year, selected at random, will contain 53 Saturdays?\hfill{[2013]}
\begin{multicols}{4}
     a) $\frac{2}{7}$\\
     b) $\frac{3}{7}$\\
     c) $\frac{1}{7}$\\
     d) $\frac{5}{7}$
 \end{multicols}

\item There were different streams of freedom movements in colonial India carried out by the moderates, liberals, radicals, socialists, and so on.\\
Which one of the following is the best inference from the above statement?\hfill{[2013]}
\begin{enumerate}
    \item The emergence of nationalism in colonial India led to our Independence.
    \item Nationalism in India emerged in the context of colonialism.
    \item Nationalism in India is homogeneous.
    \item Nationalism in India is heterogeneous.
\end{enumerate}
\item The set of values of $p$ for which the roots of the equation $3x^2+2x+p\brak{p-1}=0$ are of opposite sign is\hfill{[2013]}
\begin{multicols}{4}
     a) \brak{-\infty,0}\\
     b) \brak{0,1}\\
     c) \brak{1,\infty}\\
     d) \brak{0,\infty}
 \end{multicols}

\item A car travels 8 km in the first quarter of an hour, 6 km in the second quarter and 16 km in the third quarter. The average speed of the car in km per hour over the entire journey is\hfill{[2013]}
\begin{multicols}{4}
    a) 30\\
    b) 36\\
    c) 40\\
    d) 24
\end{multicols}
\item Find the sum to $n$ terms of the series $10+84+734+\dots$\hfill{[2013]}
\begin{multicols}{4}
     a) $\frac{9\brak{9^n+1}}{10}+1$\\
     b) $\frac{9\brak{9^n-1}}{8}+1$\\
     c) $\frac{9\brak{9^n-1}}{8}+n$\\
     d) $\frac{9\brak{9^n-1}}{8}+n^2$
 \end{multicols}




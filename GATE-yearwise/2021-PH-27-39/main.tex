\iffalse
\title{2021-PH-27-39}
\author{EE24BTECH11010 - BALAJI B}
\section{ph}
\chapter{2021}
\fi
    \item The donor concentration in a sample of $n$-type silicon is increased by a factor
of 100. Assuming the sample to be non-degenerate, the shift in the Fermi
level (in $meV$) at $300 K$ (rounded off to the nearest integer) is \rule{2cm}{0.4pt}\\ (Given $k_BT = 25 meV$ at $300K$)\hfill [2021 PH]
\item Two observers $O$ and $O^\prime$ observe two events $P$ and $Q$. The observers have a
constant relative speed of $0.5c$. In the units, where the speed of light, $c$, is
taken as unity, the observer $O$ obtained the following coordinates: \\
Event $P$ : $x = 5$, $y = 3$, $z = 5$, $t = 3$ \\
Event $Q$ : $x = 5$, $y = 1$, $z = 3$, $t = 5$\\
The length of the space-time interval between these two events, as measured by $O^\prime$, is $L$. The value of $\abs{L}$ (in integer) is \rule{2cm}{0.4pt}\hfill [2021 PH]
\item A light source having its intensity peak at the wavelength $289.8 nm$ is
calibrated as $10,000 K$ which is the temperature of an equivalent black body
radiation. Considering the same calibration, the temperature of light source
(in $K$) having its intensity peak at the wavelength $579.6 nm$ (rounded off to
the nearest integer) is \rule{2cm}{0.4pt}

\hfill [2021 PH]
\item A hoop of mass $M$ and radius $R$ rolls without slipping along a straight line
on a horizontal surface as shown in the figure. A point mass $m$ slides without
friction along the inner surface of the hoop, performing small oscillations
about the mean position. The number of degrees of freedom of the system
(in integer) is \rule{1.5cm}{0.4pt}

\hfill [2021 PH]
\begin{figure}[H]
\centering
\resizebox{4cm}{!}{%
\begin{circuitikz}
\tikzstyle{every node}=[font=\LARGE]
\draw [ line width=0.8pt ] (2.75,12.75) circle (1.5cm);
\node [font=\LARGE] at (2,12.75) {$R$};
\node [font=\LARGE] at (3.25,13.5) {$M$};
\node at (2.75,12.75) [circ] {};
\draw [line width=0.8pt, <->, >=Stealth] (2.75,12.75) -- (1.75,13.75);
\draw [line width=0.8pt, <->, >=Stealth] (2.25,11.5) .. controls (2.75,11.25) and (2.75,11.25) .. (3.25,11.5);
\node at (2.75,11.25) [circ] {};
\node [font=\LARGE] at (2.75,11.75) {$m$};
\draw [line width=0.8pt, ->, >=Stealth] (3.25,14.5) .. controls (4.25,14.25) and (4,14.25) .. (4.25,13.5) ;
\draw [line width=0.8pt, short] (0.5,11.25) -- (5.5,11.25);
\draw [line width=0.8pt, short] (0.75,11.25) -- (0.5,11);
\draw [line width=0.8pt, short] (1,11.25) -- (0.75,11);
\draw [line width=0.8pt, short] (1.25,11.25) -- (1,11);
\draw [line width=0.8pt, short] (1.5,11.25) -- (1.25,11);
\draw [line width=0.8pt, short] (1.75,11.25) -- (1.5,11);
\draw [line width=0.8pt, short] (2,11.25) -- (1.75,11);
\draw [line width=0.8pt, short] (2.25,11.25) -- (2,11);
\draw [line width=0.8pt, short] (2.5,11.25) -- (2.25,11);
\draw [line width=0.8pt, short] (2.75,11.25) -- (2.5,11);
\draw [line width=0.8pt, short] (3,11.25) -- (2.75,11);
\draw [line width=0.8pt, short] (3.25,11.25) -- (3,11);
\draw [line width=0.8pt, short] (3.5,11.25) -- (3.25,11);
\draw [line width=0.8pt, short] (3.75,11.25) -- (3.5,11);
\draw [line width=0.8pt, short] (4,11.25) -- (3.75,11);
\draw [line width=0.8pt, short] (4.25,11.25) -- (4,11);
\draw [line width=0.8pt, short] (4.5,11.25) -- (4.25,11);
\draw [line width=0.8pt, short] (4.75,11.25) -- (4.5,11);
\draw [line width=0.8pt, short] (5,11.25) -- (4.75,11);
\draw [line width=0.8pt, short] (5.25,11.25) -- (5,11);
\draw [line width=0.8pt, short] (5.5,11.25) -- (5.25,11);
\end{circuitikz}
}%

\label{fig:my_label}
\end{figure}
\item Three non-interacting bosonic particles of mass $m$ each, are in a one-
dimensional infinite potential well of width $a$. The energy of the third excited
state of the system is $x \times \frac{h^2 \pi^2}{ma^2}$. The value of $x$ (in integer) is \rule{2cm}{0.4pt} 

\hfill [2021 PH]
\item The spacing between two consecutive S- 
branch lines of the rotational Raman
spectra of hydrogen gas is $243. 2 cm^{-1}$
. After excitation with a laser of
wavelength $514.5 cm$, the Stoke's line appeared at $17611.4 cm^{-1}$
for a
particular energy level. The wavenumber (rounded off to the nearest
integer), in $cm^{-1}$
, at which Stoke's line will appear for the next higher energy
level is \rule{2cm}{0.4pt}\hfill [2021 PH]
\item The transition line, as shown in the figure, arises between $^2D_{\frac{3}{2}}$ and $^2P_\frac{1}{2}$ states without any external magnetic field. The number of
lines that will appear in the presence of a weak magnetic field (in integer) is \rule{2cm}{0.4pt} \hfill [2021 PH] 
\begin{figure}[H]
\centering
\resizebox{3cm}{!}{%
\begin{circuitikz}
\tikzstyle{every node}=[font=\LARGE]
\draw [line width=1.1pt, short] (1.25,13.5) -- (4.75,13.5);
\draw [line width=1.1pt, short] (1.25,11.25) -- (4.75,11.25);
\draw [line width=1.1pt, ->, >=Stealth] (3,13.5) -- (3,11.25);
\node [font=\LARGE] at (1.25,13) {$^2D_{\frac{3}{2}}$};
\node [font=\LARGE] at (1,10.75) {$^2P_{\frac{1}{2}}$};
\end{circuitikz}
}%

\label{fig:my_label}
\end{figure}
\item Consider the atomic system as shown in the figure, where the Einstein $A$
coefficients for spontaneous emission for the levels are $A_{2 \to 1} = 2 \times 10^7 s^{-1}$ and $A_{1 \to 0} = 10^8 s^{-1}$. If $10^{14}$ atoms/$cm^3$ are excited from level 0 to level 2 and a steady state population in level 2 is achieved, then the steady state
population at level 1 will be $x \times 10^{13} $
. The value of $x$ (in integer) is \rule{2cm}{0.4pt}\hfill [2021 PH] 
\begin{figure}[H]
\centering
\resizebox{3cm}{!}{%
\begin{circuitikz}
\tikzstyle{every node}=[font=\LARGE]
\draw [line width=1.1pt, short] (2,14.75) -- (5.25,14.75);
\draw [line width=1.1pt, short] (1,13.25) -- (3.5,13.25);
\draw [line width=1.1pt, short] (2,12) -- (5.25,12);
\draw [line width=1.1pt, ->, >=Stealth] (3.5,14.75) -- (2.75,13.25);
\draw [line width=1.1pt, ->, >=Stealth] (2.75,13.25) -- (4,12);
\node [font=\LARGE] at (1.5,14.75) {$2$};
\node [font=\LARGE] at (0.75,13.25) {$1$};
\node [font=\LARGE] at (1.75,12) {$0$};
\end{circuitikz}
}%

\label{fig:my_label}
\end{figure}
\item If $\vec{a}$ and $\vec{b}$ are constant vectors, $\vec{r}$ and $\vec{p}$ are generalized positions and
conjugate momenta, respectively, then for the transformation $Q = \vec{a}\cdot \vec{p}$ and
$P = \vec{b} \cdot \vec{r}$ to be canonical, the value of $\vec{a} \cdot \vec{b}$ (in integer) is \rule{2cm}{0.4pt}\hfill [2021 PH]

\item The below combination of logic gates represents the operation \\
\begin{figure}[H]
\centering
\resizebox{4cm}{!}{%
\begin{circuitikz}
\tikzstyle{every node}=[font=\LARGE]
\draw (1.75,15) node[ieeestd not port, anchor=in](port){} (port.out) to[short] (3.75,15);
\draw (port.in) to[short] (1.25,15);
\draw (1.75,13) node[ieeestd not port, anchor=in](port){} (port.out) to[short] (3.75,13);
\draw (port.in) to[short] (1.25,13);
\draw [line width=1.1pt, short] (3.75,15) -- (3.75,14.25);
\draw [line width=1.1pt, short] (3.75,13) -- (3.75,13.75);
\draw [line width=1.1pt, short] (3.75,13.75) -- (4.75,13.75);
\draw [line width=1.1pt, short] (3.75,14.25) -- (4.75,14.25);
\draw (4.75,14.25) to[short] (5,14.25);
\draw (4.75,13.75) to[short] (5,13.75);
\draw (5,14.25) node[ieeestd or port, anchor=in 1, scale=0.89](port){} (port.out) to[short] (7,14);
\end{circuitikz}
}%

\label{fig:my_label}
\end{figure} \hfill[2021 PH]
\item In a semiconductor, the ratio of the effective mass of hole to electron is $2:11$
and the ratio of average relaxation time for hole to electron is $1:2$. The ratio
of the mobility of the hole to electron is \hfill [2021 PH]
\begin{enumerate}
    \begin{multicols}{2}
        \item $4:9$
        \item $4:11$
        \item $9:4$
        \item $11:4$
    \end{multicols}
\end{enumerate}
\item Consider a spin $S = \frac{\hbar}{2}$ particle in the state $| \phi \rangle = \frac{1}{3} \myvec{2 + i \\ 2}$. The probability that a measurement finds the state with $S_x = + \frac{\hbar}{2}$ is  \hfill [2021 PH]
\begin{enumerate}
    \begin{multicols}{4}
        \item $\frac{5}{18}$
         \item $\frac{11}{18}$
          \item $\frac{15}{18}$
           \item $\frac{17}{18}$
    \end{multicols}
\end{enumerate}
\item An electromagnetic wave having electric field $E = 8 \cos\brak{{kz - \omega t}} \hat{y}V cm^{-1}$ is incident at $90 \degree$(normal incidence) on a square slab from vacuum (with
refractive index $n_0 = 1.0$) as shown in the figure. The slab is composed of
two different materials with refractive indices $n_1$ and $n_2$. Assume that the
permeability of each medium is the same. After passing through the slab for
the first time, the electric field amplitude, in $V cm^{-1}$, of of the electromagnetic
wave, which emerges from the slab in region 2, is closest to \hfill [2021 PH]
\begin{figure}[H]
\centering
\resizebox{5cm}{!}{%
\begin{circuitikz}
\tikzstyle{every node}=[font=\large]
\draw [line width=1.1pt, short] (1.5,14.5) -- (1.5,10.75);
\draw [line width=1.1pt, short] (1.5,10.75) -- (5.75,10.75);
\draw [line width=1.1pt, short] (5.75,10.75) -- (5.75,14.5);
\draw [line width=1.1pt, short] (1.5,14.5) -- (5.75,14.5);
\draw [line width=1.1pt, short] (5.75,14.5) -- (1.5,10.75);
\draw [ line width=0.6pt ] (3.5,15.25) circle (0.25cm);
\draw [ line width=0.6pt ] (0.75,13) circle (0.25cm);
\draw [ line width=0.6pt ] (3.5,10.25) circle (0.25cm);
\draw [ line width=0.6pt ] (6.25,12.75) circle (0.25cm);
\draw [ line width=0.6pt ] (8,12.25) circle (0.25cm);
\draw [line width=0.6pt, ->, >=Stealth] (2.5,10.75) -- (2,11.25);
\node [font=\normalsize] at (3,11.25) {$45^{\degree}$};
\node [font=\large] at (3.5,10.25) {$1$};
\node [font=\Large] at (6.25,12.75) {$2$};
\node [font=\Large] at (3.5,15.25) {$3$};
\node [font=\Large] at (0.75,13) {$4$};
\draw [line width=0.6pt, ->, >=Stealth] (8,12.25) -- (8,14.25);
\draw [line width=0.6pt, ->, >=Stealth] (8,12.25) -- (9.75,12.25);
\node [font=\large] at (7.5,12.25) {$x$};
\node [font=\large] at (9.5,12) {$y$};
\node [font=\large] at (7.5,14.25) {$z$};
\node [font=\large] at (2.5,14) {$n_2 = 1.1$};
\node [font=\large] at (4.75,11.25) {$n_1 = 2.2$};
\draw [line width=0.6pt, ->, >=Stealth] (4,9.25) -- (4,10.75);
\node [font=\large] at (6,9.75) {$n_0 = 1.0$};
\end{circuitikz}
}%

\label{fig:my_label}
\end{figure}
\begin{enumerate}
    \begin{multicols}{4}
        \item $\frac{11}{1.6}$
        \item $\frac{11}{3.2}$
        \item $\frac{11}{13.8}$
        \item $\frac{11}{25.6}$
    \end{multicols}
\end{enumerate}


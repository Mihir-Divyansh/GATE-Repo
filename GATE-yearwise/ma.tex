\iffalse
\author{EE24BTECH11062}
\section{ma}
\chapter{2022}
\fi

\item The value of the integral $\int_C \frac{z^{100}}{z^{101}+1} \, dz$ where $C$ is the circle of radius 2 centered at the origin taken in the anti-clockwise direction is \hfill{[2022]}
\begin{multicols}{4}
    a) $-2\pi i$\\
    b) $2\pi$\\
    c) 0\\
    d)  $2\pi i$
\end{multicols}

 \item Let $X$ be a real normed linear space. Let $X_0=\cbrak{x\in X:\|\mathbf{x}\|=1}$. If $X_0$ contains two distinct points $x$ and $y$ and the line segment joining them, then, which of the following statements is TRUE?\hfill{[2022]}

 \begin{enumerate}
     \item $\|\mathbf{x+y}\|=\|\mathbf{x}\|+\|\mathbf{y}\|$ and $x,y$ are linearly independent\\
     \item $\|\mathbf{x+y}\|=\|\mathbf{x}\|+\|\mathbf{y}\|$ and $x,y$ are linearly dependent\\
     \item $\|\mathbf{x+y}\|^2=\|\mathbf{x}\|^2+\|\mathbf{y}\|^2$ and $x,y$ are linearly independent\\
     \item $\|\mathbf{x+y}\|=2\|\mathbf{x}\|\|\mathbf{y}\|$ and $x,y$ are linearly dependent
 \end{enumerate}
 
\item Let \cbrak{e_k:k\in N} be an orthogonal basis for a Hilbert space $H$. Define $f_k=e_k+e_{k+1},k\in N$ and $g_j=\sum_{n=1}^j \brak{-1}^{n+1}e_n,j\in N$. Then $\sum_{n=1}^j \abs{\langle g_j, f_k \rangle
}^2=$

\hfill{[2022]}
 \begin{multicols}{4}
    a) 0\\
    b) $j^2$\\
    c) $4j^2$\\
    d) 1
\end{multicols}
\item Consider $R^2$ with the usual metric. Let $A=\cbrak{\brak{x,y}\in R^2:x^2+y^2\leq 1}$ and $B=\cbrak{\brak{x,y}\in R^2:\brak{x-2}^2+y^2\leq 1}$. Let $M=A\cup B$ and $N=$ interior\brak{A}$\cup$ interior\brak{B}. Then, which of the following statements is TRUE?
\hfill{[2022]}
\begin{enumerate}
    \item $M$ and $N$ are connected\\
    \item Neither $M$ nor $N$ is connected\\
    \item $M$ is connected and $N$ is not connected\\
    \item $M$ is not connected and $N$ is connected
\end{enumerate}

\item The real sequence generated by the iterative scheme $x_n=\frac{x_{n-1}}{2}+\frac{1}{x_{n-1}},n\geq 1$ \hfill{[2022]}
\begin{enumerate}
    \item converges to $\sqrt{2}$, for all $x_0\in R\setminus\cbrak{0}$\\
    \item converges to $\sqrt{2}$, whenever $x_0>\sqrt{\frac{2}{3}}$\\
    \item converges to $\sqrt{2}$, whenever $x_0\in \brak{-1,1}\setminus\cbrak{0}$ \\
    \item diverges for any $x_0\neq 0$
\end{enumerate}

\item The initial value problem $\frac{dy}{dx}=\cos\brak{xy}, x\in R, y\brak{0}=y_0$, where $y_0$ is a real constant, has
\hfill{[2022]}
\begin{enumerate}
    \item  a unique solution\\
    \item  exactly two solutions\\
    \item  infinitely many solutions\\
    \item  no solution
\end{enumerate}

\item If eigenfunctions corresponding to distinct eigenvalues $\lambda$ of the Sturm=Liouville problem $\frac{d^2y}{dx^2}-3\frac{dy}{dx}=\lambda y, 0<x<\pi,\\ y\brak{0}=y\brak{\pi}=0$ are orthogonal with respect to the weight function $w\brak{x}$ is\hfill{[2022]}
\begin{multicols}{4}
    a) $e^{-3x}$\\
    b) $e^{-2x}$\\
    c) $e^{2x}$\\
    d) $e^{3x}$
\end{multicols}
\item The steady state solution for the heat equation $\frac{\partial u}{\partial t}-\frac{\partial^2 u}{\partial x^2}=0,0<x<2,t>0$, with the initial condition $u\brak{0,t}=1$ and $u\brak{2,t}=3,t>0$, at $x=1$ is \hfill{[2022]}
\begin{multicols}{4}
    a) 1\\
    b) 2\\
    c) 3\\
    d) 4
\end{multicols}

\item Consider \brak{\sbrak{0,1},T_1}, where $T_1$ is the subspace topology induced by the Euclidean topology on $R$, and let $T_2$ be any topology on \sbrak{0,1}. Consider the following statements:\\
P: If $T_1$ is a proper subset of $T_2$, then \brak{\sbrak{0,1},T_2} is not compact.\\
Q: If $T_2$ is a proper subset of $T_1$, then \brak{\sbrak{0,1},T_2} is not Haudorff. Then\hfill{[2022]}
\begin{enumerate}
    \item P is TRUE and Q is FALSE.
    \item Both P and Q are TRUE.
    \item Both P and Q are FALSE.
    \item P is FALSE and Q is TRUE.
\end{enumerate}

\item Let $p:\brak{\sbrak{0,1},T_1}\rightarrow\brak{\cbrak{0,1},T_2}$ be the quotient map, arising from the characteristic function on \sbrak{\frac{1}{2},1}, where $T_1$ is the subspace topology induced by the Euclidean topology on $R$. Which of the following statements is TRUE?\hfill{[2022]}
\begin{enumerate}
    \item $p$ is an open map but not a closed map.
    \item $p$ is an closed map but not a open map.
    \item $p$ is an closed map as well as an open map.
    \item $p$ is neither an open map nor a closed map.
\end{enumerate}
\item Set $X_n:=R$ for each $n\in N$. Define $Y:=\prod\limits_{n \in \mathbb{N}} X_n$. Endow $Y$ with the product topology, where the topology on each $X_n$ is the Euclidean topology. Consider the set $\Delta=\cbrak{\brak{x,x,x,\dots}\mid x\in R}$ with the subspace topology induced from $Y$. Which of the following statements is TRUE?\hfill{[2022]}
\begin{enumerate}
     \item  $\Delta$ is open in $Y$\\
     \item  $\Delta$ is locally compact\\
     \item  $\Delta$ is dense in $Y$\\
     \item  $\Delta$ is disconnected.
 \end{enumerate}

\item Consider the linear system of equations $Ax=b$ with $\myvec{3 &&1&& 1\\ 1 &&4&& 1\\ 2&& 0&& 3}$ and $b=\myvec{2\\3\\4}$ WHich of the following statements are TRUE?\hfill{[2022]}
\begin{enumerate}
    \item The Jacobi iterative matrix is $\myvec{0&& \frac{1}{4}&& \frac{1}{3}\\ \frac{1}{3}&&0&&\frac{1}{3}\\\frac{2}{3}&&0&&0}$
    \item The Jacobi iterative method converges for any initial vector.
    \item The Gauss-Seidel iterative method converges for any initial vector.
    \item The spectral radius of the Jacobi iterative matrix is less than 1.
\end{enumerate}
\item The number of non-isomorphic abelian groups of order $2^2.3^3.5^4$ is\hfill{[2022]}





\iffalse
\title{GATE Questions 14}
\author{EE24BTECH11012 - Bhavanisankar G S}
\section{ee}
\chapter{2019}
\fi
%\begin{enumerate}
	\item I am not sure if the bus that has been booked will be able to $\underline{   }$ all the students.
		\begin{enumerate}
				\begin{multicols}{4}
				\item sit
				\item deteriorate
				\item fill
				\item accomodate
				\end{multicols}
		\end{enumerate}
	\item The passengers were angry $\underline{  }$ the airline staff about the delay.
		\begin{enumerate}
				\begin{multicols}{4}
				\item on
				\item about
				\item with
				\item towards
				\end{multicols}
		\end{enumerate}
	\item The missing number in the given sequence 343, 1331, $\underline{   }$, 4913 is
		\begin{enumerate}
				\begin{multicols}{4}
				\item 3375
				\item 2744
				\item 2197
				\item 4096
				\end{multicols}
		\end{enumerate}
	\item It takes two hours for a person X to mow the lawn. Y can mow the same lawn in four hours. How long ( in minutes ) will it take X and Y, if they work together to mow the lawn ?
		\begin{enumerate}
				\begin{multicols}{4}
				\item 60
				\item 80
				\item 90
				\item 120
				\end{multicols}
		\end{enumerate}
	\item Newspapers are a constant source of delight and recreation for me. The $\underline{   }$ trouble is that I read $\underline{   }$ many of them.
		\begin{enumerate}
				\begin{multicols}{2}
				\item even, quite
				\item even, too
				\item only, quite
				\item only, too
				\end{multicols}
		\end{enumerate}
	\item How many integers are there between 100 and 1000 all of whose digits are even ?
		\begin{enumerate}
				\begin{multicols}{4}
				\item 60
				\item 80
				\item 100
				\item 90
				\end{multicols}
		\end{enumerate}
	\item The ratio of the number of boys and girls who participated in an examination is 4:3. The total percentage of candidates who passed the examination is 80 and the percentage of girls who passed is 90. The percentage of boys who passed is
		\begin{enumerate}
				\begin{multicols}{4}
				\item 55.50
				\item 72.50
				\item 80.50
				\item 90.00
				\end{multicols}
		\end{enumerate}
	\item An award-winning study by a group of researchers suggests that men are as prone to buying on impulse as women but women feel more guilty about shopping.\\
		Which one of the following statements can be inferred from the given text ?
		\begin{enumerate}
			\item Some men and women indulge in buying on impulse
			\item All men and women indulge in buying on impulse
			\item Few men and women indulge in buying on impulse
			\item Many men and women indulge in buying on impulse
		\end{enumerate}
	\item Given two sets $X = \cbrak{1,2,3}$ and $Y = \cbrak{2,3,4}$ we construct a set Z of all possible fractions where the numerators belong to set X and the denominators belong to set Y. The product of elements having minimum and maximum values in the set Z is
		\begin{enumerate}
				\begin{multicols}{4}
				\item $\frac{1}{12}$
				\item $\frac{1}{8}$
				\item $\frac{1}{6}$
				\item $\frac{3}{8}$
				\end{multicols}
		\end{enumerate}
	\item Consider five people - Mita, Ganga, Rekha, Lakshmi and Sana. Ganga is taller than both Rekha and Lakshmi. Lakshmi is taller than Sana. Mita is taller than Ganga. Which of the following conclusions are true ? \\
		A. Lakshmi is taller than Rekha. \\
		B. Rekha is shorter than Mita. \\
		C. Rekha is taller than Sana. \\
		D. Sana is shorter than Ganga.
		\begin{enumerate}
				\begin{multicols}{2}
				\item 1 and 3
				\item 3 only
				\item 2 and 4
				\item 1 only
				\end{multicols}
		\end{enumerate}
	\item The inverse Laplace transform of $H(s) = \frac{s+3}{s^2 + 2s + 1}$ for $t \geq 0$ is
		\begin{enumerate}
				\begin{multicols}{2}
				\item $3te^{-t} + e^{-t}$
				\item $3e^{-t}$
				\item $2te^{-t} + e^{-t}$
				\item $4te^{-t} + e^{-t}$
				\end{multicols}
		\end{enumerate}
	\item $M$ is a $2 \times 2$ matrix with eigen-values 4 and 9. The eigen values of $M^2$ are
		\begin{enumerate}
				\begin{multicols}{2}
				\item 4 and 9
				\item 2 and 3
				\item -2 and -3
				\item 16 and 81
				\end{multicols}
		\end{enumerate}
	\item The partial differentiation equation 
		$$ \frac{\partial ^2 u}{\partial t^2} - c^2 \brak{\frac{\partial ^2 u}{\partial x^2} + \frac{\partial ^2 u}{\partial y^2}} = 0, c \neq 0 $$
		is known as
		\begin{enumerate}
				\begin{multicols}{2}
				\item heat equation
				\item wave equation
				\item Poisson's equation
				\item Laplace equation
				\end{multicols}
		\end{enumerate}


\iffalse
\chapter{2019}
\author{AI24BTECH11009}
\section{ce}
\fi

\item For a small value of $h$, the Taylor series expansion for $f\brak{x+h}$ is
\begin{enumerate}
    \item $f\brak{x} + hf'\brak{x} + \frac{h^2}{2!}f''\brak{x} + \frac{h^3}{3!}f'''\brak{x} + \cdots + \infty$
    \item $f\brak{x} - hf'\brak{x} + \frac{h^2}{2!}f''\brak{x} - \frac{h^3}{3!}f'''\brak{x} + \cdots + \infty$
    \item $f\brak{x} + hf'\brak{x} + \frac{h^2}{2}f''\brak{x} + \frac{h^3}{3}f'''\brak{x} + \cdots + \infty$
    \item $f\brak{x} - hf'\brak{x} + \frac{h^2}{2}f''\brak{x} - \frac{h^3}{3}f'''\brak{x} + \cdots + \infty$ \\
\end{enumerate}
\item A plane truss is shown in the figure \brak{not\ drawn\ to\ scale}.
\begin{figure}[!ht]
\centering
\resizebox{1\textwidth}{!}{%
\begin{circuitikz}
\tikzstyle{every node}=[font=\normalsize]
\draw [short] (2.5,6.75) -- (16.75,7);
\draw [short] (4.25,8) -- (4.25,6.75);
\draw [short] (5.75,9.25) -- (5.75,6.75);
\draw [short] (7.25,10.25) -- (7.25,6.75);
\draw [short] (2.5,6.75) -- (9,11.75);
\draw [short] (9,11.75) -- (16.5,7);
\draw [short] (10.75,10.5) -- (10.75,7);
\draw [short] (12.25,9.5) -- (12.25,7);
\draw [short] (14,8.5) -- (14,7);
\draw [short] (4.25,6.75) -- (5.75,9.25);
\draw [short] (5.75,6.75) -- (7.25,10.25);
\draw [short] (7.25,7) -- (9,11.75);
\draw [short] (9,11.75) -- (10.75,7);
\draw [short] (10.75,10.5) -- (12.25,7);
\draw [short] (12.25,9.5) -- (14,7);
\draw [->, >=Stealth] (5.75,10.25) -- (5.75,9.5);
\draw [->, >=Stealth] (7.25,11.25) -- (7.25,10.5);
\draw [->, >=Stealth] (10.75,11.5) -- (10.75,10.75);
\draw [->, >=Stealth] (12.25,10.75) -- (12.25,10);
\draw [short] (2.25,5.5) -- (16.75,5.75);
\draw [short] (2.5,5.75) -- (2.5,5.25);
\draw [short] (4.25,5.75) -- (4.25,5.25);
\draw [short] (5.75,5.75) -- (5.75,5.25);
\draw [short] (7.25,5.75) -- (7.25,5.25);
\draw [short] (10.75,5.75) -- (10.75,5.25);
\draw [short] (12.5,6) -- (12.5,5.25);
\draw [short] (14,6) -- (14,5.5);
\draw [short] (16.5,6) -- (16.5,5.5);
\draw [short] (2.5,6.75) -- (2.25,6.5);
\draw [short] (2.5,6.75) -- (2.75,6.5);
\draw [short] (2.25,6.5) -- (2.75,6.5);
\draw [short] (2,6.5) -- (3.25,6.5);
\draw [short] (2,6.5) -- (2.25,6.25);
\draw [short] (2.25,6.5) -- (2.5,6.25);
\draw [short] (2.5,6.5) -- (2.75,6.25);
\draw [short] (2.75,6.5) -- (3,6.25);
\draw [short] (3,6.5) -- (3.25,6.25);
\draw [short] (16.5,7) -- (16.25,6.75);
\draw [short] (16.5,7) -- (16.75,6.75);
\draw [short] (16,6.75) -- (17,6.75);
\draw  (16,6.5) circle (0.25cm);
\draw  (17,6.5) circle (0.25cm);
\draw [short] (15.25,6.25) -- (17.25,6.25);
\draw [short] (15.25,6.25) -- (15.5,6);
\draw [short] (15.5,6.25) -- (15.75,6);
\draw [short] (16,6.25) -- (16.25,6);
\draw [short] (15.75,6.25) -- (16,6);
\draw [short] (16.25,6.25) -- (16.5,6);
\draw [short] (16.5,6.25) -- (16.75,6);
\draw [short] (16.75,6.25) -- (17,6);
\draw [short] (17,6.25) -- (17.25,6);
\draw [short] (17.75,12.25) -- (17.75,7);
\draw [short] (17.5,7.25) -- (18,7.25);
\draw [short] (17.5,11.75) -- (18,11.75);
\draw [short] (17.5,10.75) -- (18,10.75);
\draw [short] (17.5,9.75) -- (18,9.75);
\draw [short] (17.5,8.5) -- (18,8.5);
\node [font=\normalsize] at (5.75,10.5) {20 kN};
\node [font=\normalsize] at (7.25,11.5) {20 kN};
\node [font=\normalsize] at (10.75,11.75) {20 kN};
\node [font=\normalsize] at (12.25,11) {20 kN};
\node [font=\normalsize] at (9,12) {L};
\node [font=\normalsize] at (3.25,5.25) {2 m};
\node [font=\normalsize] at (5,5.25) {2 m};
\node [font=\normalsize] at (6.5,5.25) {2 m};
\node [font=\normalsize] at (9,5.25) {2 m};
\node [font=\normalsize] at (11.5,5.25) {2 m};
\node [font=\normalsize] at (13.25,5.25) {2 m};
\node [font=\normalsize] at (15.25,5.25) {2 m};
\node [font=\normalsize] at (18.25,11.25) {1 m};
\node [font=\normalsize] at (18.25,10.25) {1 m};
\node [font=\normalsize] at (18.25,9) {1 m};
\node [font=\normalsize] at (18.25,8) {1 m};
\node [font=\normalsize] at (2.5,7) {E};
\node [font=\normalsize] at (16.75,7.25) {T};
\node [font=\normalsize] at (4.25,6.5) {F};
\node [font=\normalsize] at (4.25,8.25) {G};
\node [font=\normalsize] at (5.5,9.25) {I};
\node [font=\normalsize] at (5.75,6.5) {H};
\node [font=\normalsize] at (7.5,6.5) {J};
\node [font=\normalsize] at (7.5,10.75) {K};
\node [font=\normalsize] at (10.75,6.75) {M};
\node [font=\normalsize] at (11,10.75) {N};
\node [font=\normalsize] at (12.25,6.75) {O};
\node [font=\normalsize] at (12.5,10) {P};
\node [font=\normalsize] at (14,6.75) {R};
\node [font=\normalsize] at (14.25,8.75) {S};
\end{circuitikz}

}%
\end{figure}\\
Which one of the options contains ONLY zero force members in the truss ?
\begin{enumerate}
    \item FG, FI, HI, RS
    \item FG, FH, HI, RS
    \item FI, HI, PR, RS
    \item FI, FG, RS, PR \\
\end{enumerate}
\item An element is subjected to biaxial normal tensile strains of 0.0030 and 0.0020. The normal strain in the plane of maximum shear strain is
\begin{enumerate}
    \item Zero
    \item 0.0010
    \item 0.0025
    \item 0.0050 \\
\end{enumerate}
\item Consider the pin-jointed plane truss shown in the figure \brak{not\ drawn\ to\ scale}. Let $R_P$, $R_Q$, and $R_R$ denote the vertical reactions (upward positive) applied by the supports at P, Q, and R, respectively, on the truss. The correct combination of $\brak{R_P, R_Q, R_R}$ is represented by
\begin{figure}[!ht]
\centering
\resizebox{0.7\textwidth}{!}{%
\begin{circuitikz}
\tikzstyle{every node}=[font=\normalsize]
\draw  (6,11) rectangle (8,9);
\draw [short] (6,9) -- (4.5,8.25);
\draw [short] (4.5,8.25) -- (6,11);
\draw [short] (8,9) -- (9.75,9);
\draw [short] (8,11) -- (9.75,9);
\draw [short] (10.5,9) -- (11,9);
\draw [short] (10.5,11) -- (11,11);
\draw [short] (3.25,10.75) -- (3.75,10.75);
\draw [short] (3.25,8.25) -- (3.75,8.25);
\draw [<->, >=Stealth] (3.5,10.75) -- (3.5,8.25);
\draw [<->, >=Stealth] (10.75,11) -- (10.75,9);
\draw [<->, >=Stealth] (4.5,7.5) -- (6,7.5);
\draw [<->, >=Stealth] (6,7.5) -- (8,7.5);
\draw [<->, >=Stealth] (8,7.5) -- (9.75,7.5);
\draw [short] (4.5,7.75) -- (4.5,7.25);
\draw [short] (6,7.75) -- (6,7.25);
\draw [short] (8,7.75) -- (8,7.25);
\draw [short] (9.75,7.75) -- (9.75,7.25);
\draw [short] (4.5,8.25) -- (4.25,8);
\draw [short] (4.5,8.25) -- (4.75,8);
\draw [short] (4,8) -- (5,8);
\draw [short] (4,8) -- (4.25,7.75);
\draw [short] (4.25,8) -- (4.5,7.75);
\draw [short] (4.5,8) -- (4.75,7.75);
\draw [short] (4.75,8) -- (5,7.75);
\draw  (8,8.75) circle (0.25cm);
\draw  (9.75,8.75) circle (0.25cm);
\draw [short] (7.5,8.5) -- (8.5,8.5);
\draw [short] (9.25,8.5) -- (10.25,8.5);
\draw [short] (7.5,8.5) -- (7.75,8.25);
\draw [short] (7.75,8.5) -- (8,8.25);
\draw [short] (8.25,8.5) -- (8.5,8.25);
\draw [short] (8,8.5) -- (8.25,8.25);
\draw [short] (9.25,8.5) -- (9.5,8.25);
\draw [short] (9.75,8.5) -- (10,8.25);
\draw [short] (9.5,8.5) -- (9.75,8.25);
\draw [short] (10,8.5) -- (10.25,8.25);
\draw [->, >=Stealth] (6,9) -- (6,8.25);
\node [font=\normalsize] at (3,9.5) {3 m};
\node [font=\normalsize] at (11,10) {2 m};
\node [font=\normalsize] at (6,8) {30 kN};
\node [font=\normalsize] at (5.25,7.25) {3 m};
\node [font=\normalsize] at (7,7.25) {3 m};
\node [font=\normalsize] at (8.75,7.25) {3 m};
\node [font=\normalsize] at (4.25,8.5) {P};
\node [font=\normalsize] at (8.25,8) {Q};
\node [font=\normalsize] at (10,8) {R};
\end{circuitikz}

}%
\end{figure}
\begin{enumerate}
    \item \brak{30, -30, 30}kN
    \item \brak{20, 0, 10}kN
    \item \brak{10, 30, -10}kN
    \item \brak{0, 60, -30}kN \\
\end{enumerate}
\item Assuming that there is no possibility of shear buckling in the web, the maximum reduction permitted by IS 800-2007 in the (low-shear) design bending strength of a semi-compact steel section due to high shear is
 \begin{enumerate}
    \item zero
    \item 25\%
    \item 50\%
    \item governed by the area of the flange \\
\end{enumerate}
\item In the reinforced beam section shown in the figure \brak{not\ drawn\ to\ scale}, the nominal cover provided at the bottom of the beam as per IS 456-2000, is
\begin{figure}[!ht]
\centering
\resizebox{0.5\textwidth}{!}{%
\begin{circuitikz}
\tikzstyle{every node}=[font=\normalsize]
\draw  (6.5,10.75) rectangle (9.5,7.25);
\draw  (6,11.25) rectangle (10,6.75);
\draw  (6.75,10.5) circle (0.25cm);
\draw  (9.25,10.5) circle (0.25cm);
\draw  (6.75,7.5) circle (0.25cm);
\draw  (9.25,7.5) circle (0.25cm);
\draw  (8,7.5) circle (0.25cm);
\draw [short] (10.25,11.25) -- (10.75,11.25);
\draw [short] (10.25,10.75) -- (10.75,10.75);
\draw [short] (10.25,7.25) -- (10.75,7.25);
\draw [short] (10.25,6.75) -- (10.75,6.75);
\draw [short] (6,6.5) -- (6,6);
\draw [short] (6.5,6.5) -- (6.5,6);
\draw [<->, >=Stealth] (10.5,11.25) -- (10.5,10.75);
\draw [<->, >=Stealth] (10.5,7.25) -- (10.5,6.75);
\draw [<->, >=Stealth] (6,6.25) -- (6.5,6.25);
\draw [<->, >=Stealth] (6,5.25) -- (10,5.25);
\draw [short] (6,5.5) -- (6,5);
\draw [short] (10,5.5) -- (10,5);
\draw [->, >=Stealth] (7.75,12.25) -- (6.75,10.75);
\draw [->, >=Stealth] (7.75,12.25) -- (9.25,10.75);
\draw [->, >=Stealth] (7.75,6) -- (6.75,7.25);
\draw [->, >=Stealth] (7.75,6) -- (8,7.25);
\draw [->, >=Stealth] (7.75,6) -- (9.25,7.25);
\node [font=\normalsize] at (10.75,11) {50};
\node [font=\normalsize] at (10.75,7) {50};
\node [font=\normalsize] at (6.25,6) {50};
\node [font=\normalsize] at (7.75,5) {350};
\draw [->, >=Stealth] (11.25,9.25) -- (9.5,9);
\node [font=\normalsize] at (7.75,12.5) {2-28$\varphi$};
\node [font=\normalsize] at (7.75,5.75) {3-16$\varphi$};
\node [font=\normalsize] at (8,4.25) {All dimensions are in mm};
\node [font=\normalsize] at (12.5,9.75) {2-legged, 12$\varphi$};
\node [font=\normalsize] at (12.5,9.25) {stirrups @ 90 c/c};
\end{circuitikz}

}%
\end{figure}
\begin{enumerate}
    \item 30 mm
    \item 36 mm
    \item 42 mm
    \item 50 mm \\
\end{enumerate}
\item The interior angles of four triangles are given below:
\begin{table}[h!]
  \centering
  \begin{tabular}[12pt]{ |c| c|}
    \hline
    \textbf{Triangle} & \textbf{Interior Angles}\\ 
    \hline
    P & 85\degree, 50\degree, 45\degree \\
    \hline 
    Q & 100\degree, 55\degree, 25\degree \\
    \hline
    R & 100\degree, 45\degree, 35\degree \\
    \hline
    S & 130\degree, 30\degree, 20\degree \\
    \hline
    \end{tabular}


\end{table}\\
Which of the triangles are ill-conditioned and should be avoided in Triangulation surveys ?
\begin{enumerate}
   \item Both P and R 
   \item Both Q and R 
   \item Both P and S
   \item Both Q and S \\
\end{enumerate}
\item The coefficient of average rolling friction of a road is $f_r$ and its grade is +$G$\%. If the grade of this road is doubled, what will be the percentage change in the braking distance (for the design vehicle to come to a stop) measured along the horizontal (assume all other parameters are kept unchanged) ?
\begin{enumerate}
    \item $\frac{0.01 G}{f_r + 0.02 G} \times 100$
    \item $\frac{f_r}{f_r + 0.02 G} \times 100$
    \item $\frac{0.02 G}{f_r + 0.01 G} \times 100$
    \item $\frac{2 f_r}{f_r + 0.01 G} \times 100$ \\
\end{enumerate}
\item An isolated concrete pavement slab of length $L$ is resting on a frictionless base. The temperature of the top and bottom fibre of the slab are $T_t$ and $T_b$, respectively. Given: the coefficient of thermal expansion = $\alpha$ and the elastic modulus = $E$. Assuming $T_t > T_b$ and the unit weight of concrete as zero, the maximum thermal stress is calculated as
\begin{enumerate}
    \item $L \alpha \brak{T_t - T_b}$
    \item $E \alpha \brak{T_t - T_b}$
    \item $\frac{E \alpha \brak{T_t - T_b}}{2}$
    \item zero \\
\end{enumerate}
\item In a rectangular channel, the ratio of the velocity head to the flow depth for critical flow condition, is
\begin{enumerate}
    \item $\frac{1}{2}$
    \item $\frac{2}{3}$
    \item $\frac{3}{2}$
    \item 2 \\
\end{enumerate}
\item If the path of an irrigation canal is below the bed level of a natural stream, the type of cross-drainage structure provided is
\begin{enumerate}
    \item Aqueduct
    \item Level crossing
    \item Sluice gate
    \item Super passage \\
\end{enumerate}
\item A catchment may be idealised as a rectangle. There are three rain gauges located inside the catchment at arbitrary locations. The average precipitation over the catchment is estimated by two methods: (i) Arithmetic mean $\brak{P_A}$, and (ii) Thiessen polygon $\brak{P_T}$. Which one of the
following statements is correct ?
\begin{enumerate}
    \item $P_A$ is always smaller than $P_T$
    \item $P_A$ is always greater than $P_T$
    \item $P_A$ is always equal to $P_T$
    \item There is no definite relationship between $P_A$ and $P_T$ \\
\end{enumerate}
\item A retaining wall of height $H$ with smooth vertical backface supports a backfill inclined at an angle $\beta$ with the horizontal. The backfill consists of cohesionless soil having angle of internal friction $\phi$. If the active lateral thrust acting on the wall is $P_a$, which one of the following statements is TRUE ?
\begin{enumerate}
    \item $P_a$ acts at a height $\frac{H}{2}$ from the base of the wall and at an angle $\beta$ with the horizontal
    \item $P_a$ acts at a height $\frac{H}{2}$ from the base of the wall and at an angle $\phi$ with the horizontal
    \item $P_a$ acts at a height $\frac{H}{3}$ from the base of the wall and at an angle $\beta$ with the horizontal
    \item $P_a$ acts at a height $\frac{H}{3}$ from the base of the wall and at an angle $\phi$ with the horizontal \\
\end{enumerate}

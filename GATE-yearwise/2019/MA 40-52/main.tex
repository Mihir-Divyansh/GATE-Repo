\iffalse
\chapter{2019}
\author{AI24BTECH11032}
\section{ma}
\fi
    \item Consider the ordered square $I_{0}^{2},$ the set $\sbrak{0,1}\times\sbrak{0,1}$ with the dictionary order topology. Let the general element of   $I_{0}^{2},$ be denoted by $x\times y,$ where x,y$\in\sbrak{0,1}.$ Then the closure of the subset
    \begin{align*}
        \text{S}=\cbrak{x\times\frac{3}{4}:0<a<x<b<1}\text{ in } I_{0}^{2}
    \end{align*}
\begin{multicols}{2}
    \begin{enumerate}
         \item $\text{S}\bigcup\brak{(\text{a,b}]\times\cbrak{0}}\bigcup\brak{[a,b)\times\cbrak{1}}$
         \item  $\text{S}\bigcup\brak{[\text{a,b})\times\cbrak{0}}\bigcup\brak{(a,b]\times\cbrak{1}}$
         \item $\text{S}\bigcup\brak{\brak{\text{a,b}}\times\cbrak{0}}\bigcup\brak{\brak{a,b}\times\cbrak{1}}$
        \item $\text{S}\bigcup\brak{(\text{a,b}]\times\cbrak{0}}$
    \end{enumerate}
\end{multicols}   
\bigskip
\item Let $\text{P}_{2}$ be the vector space of all polynomials of degree at most $2$ over $\textbf{R}\brak{\text{ the set of real numbers }}.$ Let a linear transformation  $\text{T}:P_{2}\to P_{2}$ be defined by 
\begin{align*}
    T\brak{a+bx+cx^{2}}=\brak{a+b}+\brak{b-c}x+\brak{a+c}x^{2}
\end{align*}

consider the following statements:
\begin{itemize}
    \item[I.] The null space of  T is $\cbrak{\text{a}\brak{-1 + x + x^{2}} : a \in \mathbb{R}}$ .
    
    \item[II.] The range space of T  is spanned by the set $\cbrak{1+x^{2},1+x}$.
    
    \item[III.] $\text{T}\brak{\text{T}\brak{1+x}}=1+x^{2}$.
    
    \item[IV.] If M is the matrix representation of T with respect to the standard basis $\cbrak{1,x,x^{2}}$ of $P_{2}$, then the trace of the matrix M is 3.
\end{itemize}
Which of the above statement are TRUE?
\begin{multicols}{2}
    \begin{enumerate}
         \item I and II only 
         \item I ,III and IV only 
         \item I ,II and IV only 
        \item  II and IV only 
    \end{enumerate}
\end{multicols}   
\bigskip
\item Let $T_{1}\text{ and }T_{2}$ be two topologies defined on $\mathbb{N}\brak{\text{ the set of all natural number }},$ where $T_{1}$ is the topology  generated by B$=\cbrak{\cbrak{2n-1,2n}:n\in \mathbb{N}}\text{ and }T_{2}$ is the discrete  topology on $\mathbb{N}$.
Consider the following  statements:
\begin{itemize}
    \item [I.] IN $\brak{\mathbb{N},T_{1}},$ every infinite subset has a limit point.
    \item [II.] The function f:$\brak{\mathbb{N},T_{1}}\in\brak{\mathbb{N},T_{2}}$ is defined by
\end{itemize}
\begin{align*}
    f\brak{n}=\begin{cases} 
    \frac{n}{2},\text{if n is even}\\
    \frac{n+1}{2},\text{if n is odd}
\end{cases}
\end{align*}
is a continuous function\\
which of the above statement is/are TRUE?
\begin{multicols}{2}
    \begin{enumerate}
         \item  both I and II  
         \item I  only 
         \item II  only 
        \item  Neither I or II 
    \end{enumerate}
\end{multicols}   
\bigskip
\item Let $1 \leq p < q < \infty$  Consider the following statements:
\begin{itemize}
    \item [I] $\ell^{p} \subset \ell^{q}$
    \item [II] $L^{p}\sbrak{0,1} \subset L^{q}\sbrak{0,1},$
\end{itemize}
where $\ell^{p} = \cbrak{\brak{x_{1}, x_{2}, \dots} : x_{i} \in \mathbb{R}, \sum_{i=1}^{\infty} \abs{x_i}^{p} < \infty}$ and\\
{\small $L^{p}=\cbrak{f:\sbrak{0,1}\to\mathbb{R}:\text{ f is }\mu-\text{measurable},\int_{\sbrak{0,1}}\abs{f}^{p}d\mu<\infty, \text{ where } \mu \text{ is the Lebesgue measure }}$}\\


%is that how you needed the integral limits?

$\brak{\mathbb{R} \text{ is the set of all real number }}$\\
Which of the above statements is/are TRUE?
\begin{multicols}{2}
    \begin{enumerate}
         \item  both I and II  
         \item I  only 
         \item II  only 
        \item  Neither I or II 
    \end{enumerate}
\end{multicols}   
\bigskip
\item Consider the differential equation
\begin{align*}
    t\frac{d^{2}y}{dt^{2}}+2\frac{dy}{dt}+ty=0,t>0,y\brak{0+}=1,\brak{\frac{dy}{dt}}_{t=0+}=0.
\end{align*}
 If Y$\brak{\text{s}}$ is the Laplace transform of Y$\brak{\text{t}}$, then the value of Y$\brak{1} \text{is} \underline{\hspace{2cm}}\brak{\text{ round off to 2 places of decimal }}.$\\
 $\brak{\text{Here, the inverse trigonometric functions assume principal values only}}$
 \bigskip
 \item Let R be the in region in the xy-plane bounded by the curve 
 $y=x^{2},y=4x^{2},xy=1\text{ and } xy=5.$\\
 Then the value of the integral $\int\int_{R}\frac{y^{2}}{x}$dydx is equal to $\underline{\hspace{2cm}}.$
\bigskip
\item Let V be the vector space of all $3\times3$ matrices with complex entries over the real field.If
\begin{align*}
    W_{1}=\cbrak{A\in V:A=A^{T}} \text{ and } W_{2}=\cbrak{A\in V: \text{ trace of }A=0},
\end{align*}
then the dimension of $W_{1} + W_{2}$ is equal to $\underline{\hspace{2cm}}.$
$\brak A^{T}{\text{ denote the conjugate transpose of A}}$
\bigskip
\item The number of elements of order 15 in the additive group $Z_{10}\times Z_{10}$ is \rule{1cm}{0.15mm}. ($Z_{10}$ denotes the group of integers modulo n, under the operation of addition modulo n, for any positive integer n).

\bigskip
\item Consider the following cost matrix of assigning four jobs to four persons:
\begin{figure}[H]
\centering
\resizebox{0.3\textwidth}{!}{%
\begin{circuitikz}
\tikzstyle{every node}=[font=\Large]
% Outer rectangle for grid
\draw (2.5,11.5) rectangle (12.75,6.5);

% Vertical grid lines
\draw (4.25,11.5) -- (4.25,6.5);
\draw (6.25,11.5) -- (6.25,6.5);
\draw (8,11.5) -- (8,6.5);
\draw (10,11.5) -- (10,6.5);

% Horizontal grid lines
\draw (2.5,10.5) -- (12.75,10.5);
\draw (2.5,9.5) -- (12.75,9.5);
\draw (2.5,8.5) -- (12.75,8.5);
\draw (2.5,7.5) -- (12.75,7.5);

% Labels for Jobs (J1, J2, J3, J4) at the top
\node at (5,11) {J$_1$};
\node at (7.25,11) {J$_2$};
\node at (9,11) {J$_3$};
\node at (11.25,11) {J$_4$};

% Labels for Persons (P1, P2, P3, P4) on the left side
\node at (3.25,10) {P$_1$};
\node at (3.25,9) {P$_2$};
\node at (3.25,8) {P$_3$};
\node at (3.25,7) {P$_4$};

% Values inside the grid
\node at (5.25,10) {5};    % Row 1
\node at (7.25,10) {8};
\node at (9,10) {6};
\node at (11.25,10) {10};

\node at (5.25,9) {2};     % Row 2
\node at (7.25,9) {5};
\node at (9,9) {4};
\node at (11.25,9) {8};

\node at (5.25,8) {6};     % Row 3
\node at (7.25,8) {7};
\node at (9,8) {6};
\node at (11.25,8) {9};

\node at (5.25,7) {6};     % Row 4
\node at (7.25,7) {9};
\node at (9,7) {8};
\node at (11.25,7) {10};

% Titles for Jobs and Persons
\node [font=\Large] at (7.25,12.25) {Jobs};
\node [font=\Large, rotate=90] at (1.25,9) {Persons};
\end{circuitikz}
}%
\end{figure}
Then the minimum cost of the assignment problem subject to the constraint that job $\text{J}_{4}$ is assigned to person $\text{P}_{2}$ is $\underline{\hspace{2cm}}.$
\bigskip
\item Let $y:\sbrak{-1,1}\to\mathbb{R}\text{ with }\text{y}\brak{1}=1$ satisfy the Legendre differential equation 
\begin{align*}\brak{1-x^{2}}\frac{d^{2}y}{dx^{2}}-2x\frac{dy}{dx}+6y=0\text{ for } \abs{x}<1.
\end{align*}
Then the value of $\int_{-1}^{1}y\brak{x}\brak{x+x^{2}}$dx is equal to $\underline{\hspace{2cm}}\brak{\text{ round off to 2 places of decimal }}.$ 
\bigskip
\item Let $\mathbb{Z}_{125}$ be the ring of integer modulo 125 under the operations of addition modulo 125 and multiplication modulo 125. if m is the number if maximal ideals of $\mathbb{Z}_{125}$ and n is the number of non-units of $\mathbb{Z}_{125},\text{ then } m+n$ is equal to  $\underline{\hspace{2cm}}.$
\bigskip
\item The maximum value of the error term of the composite Trapezoidal rule when it is used to evaluate the definite integral 
\begin{align*}
    \int_{0.2}^{1.4}\brak{\sin x - \log_{e}x}dx
\end{align*}
with 12 sub-intervals of equal length, is equal to $\underline{\hspace{2cm}}.$ $\brak{\text{ round off to 3 places of decimal }}$
\bigskip
\item By the Simplex method, the optimal table of the linear programming problem:
\begin{align*}
    \text{ Maximize} Z=\alpha x_{1}+3x_{2}\\
    \text{ subject to } \beta x_{1}+x_{2}+x_{3}=8,\\
    2x_{1}+x_{2}+x_{4}=\gamma, x_{1},x_{2},x_{3},x_{4}\geq0,
\end{align*}
where $\alpha,\beta,\gamma$ are real constant is
\begin{figure}[H]
\centering
\resizebox{0.5\textwidth}{!}{%
\begin{circuitikz}
\tikzstyle{every node}=[font=\LARGE]

% Draw the outer rectangle
\draw (3.5,11.25) rectangle (15.5,6);

% Draw vertical lines
\draw (8,11.25) -- (8,6);
\draw (9,11.25) -- (9,6);
\draw (10,11.25) -- (10,6);
\draw (11,11.25) -- (11,6);
\draw (12,11.25) -- (12,6);

% Draw horizontal lines
\draw (3.5,10.25) -- (15.5,10.25);
\draw (3.5,9) -- (15.5,9);
\draw (3.5,7.75) -- (15.5,7.75);
\draw (3.5,7) -- (15.5,7);

% Node placements
\node at (5.25,10.75) {c$_j \to$};
\node at (5.75,9.75) {Basic variable};
\node at (5.5,8.25) {x$_2$};
\node at (5.5,7.5) {x$_1$};
\node at (5.5,6.5) {z$_j - c_j$};

\node at (8.5,10.75) {$\alpha$};
\node at (8.5,9.5) {x$_1$};
\node at (8.5,8.25) {1};
\node at (8.5,7.25) {0};
\node at (8.5,6.5) {0};

\node at (9.5,10.75) {3};
\node at (10.5,10.75) {0};
\node at (11.25,10.75) {0};

\node at (14,9.5) {Solution};
\node at (13.75,8.25) {6};
\node at (13.75,7.25) {2};
\node at (13.75,6.5) {-};

\node at (11.5,6.5) {1};
\node at (11.5,8.25) {-1};
\node at (11.25,9.75) {x$_4$};
\node at (9.5,9.75) {x$_2$};
\node at (10.5,9.5) {x$_3$};

\node at (9.5,8.25) {0};
\node at (9.5,7.25) {1};
\node at (9.5,6.5) {0};

\node at (10.5,8.25) {2};
\node at (10.5,7.25) {-1};
\node at (10.5,6.5) {2};
\node at (11.5,7.25) {1};

\end{circuitikz}
}%

\end{figure}
Then the value of $\alpha+\beta+\gamma$ is $\underline{\hspace{2cm}}$



\iffalse
\title{2024-PH- 40-52}
\author{EE24BTECH11016 - DHWANITH M DODDAHUNDI}
\section{ph}
\chapter{2024}
\fi
% \maketitle
% \newpage
% \bigskip

\item In a parallel plate capacitor, the plate at $x=0$ is grounded and the plate at $x=d$ is
maintained at a potential $V_{0}$ . The space between the two plates is filled with a linear
dielectric of permittivity $\epsilon=\epsilon_{0}\brak{1+\frac{x}{d}}$, where $\epsilon_{0}$ is the permittivity of free space.
Neglecting the edge effects, the electric field $(\Vec{\overrightarrow{E}})$ inside the capacitor is
\begin{enumerate}
    \item $-\frac{V_{0}}{(d+x)ln 2}\hat{x}$
    \item $-\frac{V_{0}}{d}\hat{x}$
    \item $-\frac{V_{0}}{(d+x)}\hat{x}$
    \item $-\frac{V_{0}d}{(d+x)x}\hat{x}$
\end{enumerate}
\item The equation of motion for the forced simple harmonic oscillator is \\
$\Ddot{x}(t)+\omega^{2}x(t)=F\cos{(\omega t)}$ \\
    where $x(t=0)=0$ and $\Dot{x}(t=0)=0$. Which one of the following options is
correct?
    \begin{enumerate}
        \item $x(t)\propto t\sin{(\omega t)}$
        \item $x(t)\propto t\cos{(\omega t)}$
        \item $x(t)=\infty$
        \item $x(t)\propto e^{\omega t}$
    \end{enumerate}
    \item An atom is subjected to a weak uniform magnetic field $\Vec{\overrightarrow{B}}$. The number of lines in its Zeeman spectrum for transition from $n=2,l=1$ to $n=1,l=0$ is
    \begin{enumerate}
        \item 8
        \item 10
        \item 12
        \item 5
    \end{enumerate}
    \item Consider two matrices: P=$\myvec{ 1 && 2 \\ 0 && 1}$ and Q=$\myvec{1 && 0 \\ 0 && 1}$ \\
    Which of the following statement is/are true?
    \begin{enumerate}
        \item P and Q have same set of eigenvalues
        \item P and Q commute with each other
        \item P and Q have different sets of linearly independent eigenvectors
        \item P is diagonalizable
    \end{enumerate}
    \item An infinite one dimensional lattice extends along $x$-axis. At each lattice site there
exists an ion with spin $\frac{1}{2}$. The spin can point either in +$z$ or -$z$ direction only. Let
$S_P, S_F,$ and $S_A$ denote the entropies of paramagnetic, ferromagnetic and
antiferromagnetic configurations, respectively. Which of the following relation is/are true?
\begin{enumerate}
    \item $S_P>S_F$
    \item $S_A>S_F$
    \item $S_A=4S_F$
    \item $S_P>S_A$
\end{enumerate}
\item Cosider a vector field $\Vec{\overrightarrow{F}}=(2xz+3y^2)\hat{y}+4yz^2\hat{z}$. The closed path 
($\Gamma:A\rightarrow B\rightarrow C\rightarrow D\rightarrow A$) in $z=0$ plane is shown in figure. \\
\begin{figure}[H]
    \centering
\begin{figure}[H]
\centering
\resizebox{0.5\textwidth}{!}{%
\begin{circuitikz}
\tikzstyle{every node}=[font=\normalsize]
\draw (7,11.25) to[short] (7,14.5);
\draw (7,14.5) to[short] (7,15.25);
\draw (6.75,11.5) to[short] (11.5,11.5);
\draw (7,14) to[short] (9.5,14);
\draw (9.5,14) to[short] (9.5,11.5);
\draw [line width=0.6pt, ->, >=Stealth] (8.75,14) -- (8.25,14);
\draw [line width=0.6pt, ->, >=Stealth] (9.5,12.75) -- (9.5,13.25);
\draw [line width=0.6pt, ->, >=Stealth] (8,11.5) -- (8.5,11.5);
\draw [line width=0.6pt, ->, >=Stealth] (7,12.75) -- (7,12.5);
\draw [line width=0.6pt, ->, >=Stealth] (7,15) -- (7,15.5);
\draw [line width=0.6pt, ->, >=Stealth] (11.5,11.5) -- (11.75,11.5);
\node [font=\normalsize] at (6.75,15.5) {y};
\node [font=\normalsize] at (11.75,11.25) {x};
\node [font=\normalsize] at (6.75,11.25) {A(0,0)};
\node [font=\normalsize] at (9.5,11.25) {B(1,0)};
\node [font=\normalsize] at (10.25,14) {C(1,1)};
\node [font=\normalsize] at (6.25,14) {D(0,1)};
\end{circuitikz}
}%

\end{figure}
\end{figure}

$\oint_\Gamma \Vec{\overrightarrow{F}}.\Vec{\overrightarrow{dl}}$ denotes the line integral of $\Vec{\overrightarrow{F}}$ along the closed path $\Gamma$. Which of the following option is/are true?
\begin{enumerate}
    \item $\oint_\Gamma \Vec{\overrightarrow{F}}\cdot\Vec{\overrightarrow{dl}}=0$
    \item $\Vec{\overrightarrow{F}}$ is non-conservative
    \item $\Vec{\overrightarrow{\nabla}}\cdot\Vec{\overrightarrow{F}}=0$
    \item  $\Vec{\overrightarrow{F}}$ can be written as the gradient of a scalar field
\end{enumerate}
\item Two point charges of charge +$q$ each are placed a distance 2$d$ apart. A grounded
solid conducting sphere of radius $a$ is placed midway between them.
Assume $a^2<<d^2$. Which of the following statement is/are true?
\begin{enumerate}
    \item  If $a>\frac{d}{8}$ the net force acting on the charges is directed towards each other
    \item The potential at the surface of the sphere is zero
    \item Total induced charge on the sphere is $\brak{-\frac{2aq}{d}}$
    \item The potential at the center of the sphere is non-zero
\end{enumerate}
\item A particle of mass $m$ is moving in the potential  \\
  $ V(x)= \begin{cases} V_0+\frac{1}{2}m\omega_0^2x^2 & x>0 \\ \infty & x\leq0 \end{cases}$ \\
Figures P, Q, R and S show different combinations of the values of $\omega_0$ and $V_0$ . \\ 
\begin{figure}[H]
    \centering
    \resizebox{0.7\textwidth}{!}{%
  \begin{circuitikz}
\tikzstyle{every node}=[font=\large]
\draw [ line width=0.6pt](6.75,15.5) to[short] (6.75,12.5);
\draw [ line width=0.6pt](6.75,12.75) to[short] (10,12.75);
\draw [ line width=0.6pt](6.5,12.75) to[short] (7,12.75);
\draw [->, >=Stealth] (9.75,12.75) -- (10,12.75);
\draw [->, >=Stealth] (6.75,15.25) -- (6.75,15.75);
\draw [->, >=Stealth] (14,12.5) -- (14,16);
\draw [->, >=Stealth] (13.75,12.75) -- (17.75,12.75);
\draw [->, >=Stealth] (6.75,7.75) -- (6.75,11);
\draw [->, >=Stealth] (6.5,8) -- (10,8);
\draw [->, >=Stealth] (14.25,8) -- (14.25,11.25);
\draw [->, >=Stealth] (14,8.25) -- (18,8.25);
\draw [line width=0.6pt, short] (6.75,12.75) .. controls (8.5,13.25) and (9,13.5) .. (9.25,15.25);
\draw [line width=0.5pt, short] (14,13.5) .. controls (16.25,13.5) and (16.5,14.5) .. (17,15.75);
\draw [line width=0.5pt, short] (6.75,8.75) .. controls (8.75,9) and (9.5,9) .. (9.75,11);
\draw [line width=0.5pt, short] (14.25,8.25) .. controls (16,9.25) and (16,9.25) .. (16,11.25);
\draw [line width=0.5pt, <->, >=Stealth] (6.5,8) -- (6.5,8.75);
\draw [line width=0.5pt, <->, >=Stealth] (13.75,12.75) -- (13.75,13.5);
\node [font=\large] at (6,8.25) {$V_0$};
\node [font=\large] at (13.25,13) {$V_0$};
\node [font=\normalsize] at (9,16) {\textbf{Figure P}};
\node [font=\normalsize] at (17,16.5) {\textbf{Figure Q}};
\node [font=\normalsize] at (9.5,11.5) {\textbf{Figure R}};
\node [font=\normalsize] at (16.75,11.5) {\textbf{Figure S}};
\node [font=\normalsize] at (18,8) {\textbf{x}};
\node [font=\normalsize] at (17.75,12.5) {\textbf{x}};
\node [font=\normalsize] at (10,12.5) {\textbf{x}};
\node [font=\normalsize] at (10,7.75) {\textbf{x}};
\node [font=\normalsize] at (6.25,15.75) {\textbf{V(x)}};
\node [font=\normalsize] at (13.5,16) {\textbf{V(x)}};
\node [font=\normalsize] at (13.75,11.25) {\textbf{V(x)}};
\node [font=\normalsize] at (6.25,10.75) {\textbf{V(x)}};
\node [font=\large] at (6.25,7.75) {\textbf{O}};
\node [font=\large] at (13.75,8) {\textbf{O}};
\node [font=\large] at (13.5,12.5) {\textbf{O}};
\node [font=\large] at (6.25,12.5) {\textbf{O}};
\node [font=\small] at (7.75,14.5) {};
\node [font=\small] at (8,14.5) {$\omega_0=12 rad s^{-1}$};
\node [font=\small] at (7.75,14) {$V_0=0$};
\node [font=\small] at (17,13.75) {$\omega_0=12 rad s^{-1}$};
\node [font=\small] at (17,13.25) {$V_0=3h Joules$};
\node [font=\small] at (9.5,8.75) {$\omega_0=4 rad s^{-1}$};
\node [font=\small] at (9.5,8.25) {$V_0=4h Joules$};
\node [font=\small] at (17,9.25) {$\omega_0=14 rad s^{-1}$};
\node [font=\small] at (17,8.75) {$V_0=0$};
\end{circuitikz}}%

\end{figure}

$E_j^{(P)},E_j^{(Q)},E_j^{(R)}$ and $E_j^{(S)}$ with $j=0,1,2,\cdots$ are the eigen-energies of the $j$-th level
for the potentials shown in figures P, Q, R and S, respectively. Which of the statement is/are true? 
\begin{enumerate}
\begin{multicols}{2}
    \item $E_0^{(P)}=E_0^{(Q)}$
    \item $E_0^{(Q)}=E_0^{(S)}$
    \item $E_0^{(P)}=E_0^{(R)}$
    \item $E_0^{(R)}\neq E_0^{(Q)}$
    \end{multicols}
\end{enumerate}
\item The non-relativistic Hamiltonian for a single electron atom is 
\begin{center}
    {$H_0=\frac{p^2}{2m}-V(r)$ \\}
\end{center}
where $V(r)$ is the Coulomb potential and $m$ is the mass of the electron. Considering
the spin-orbit interaction term
\begin{center}
     {$H'=\frac{1}{2m^2c^2}\frac{1}{r}\frac{dV}{dr}\Vec{\overrightarrow{L}}\cdot\Vec{\overrightarrow{S}} $ \\ }
\end{center}
  added to $H_0$ , which of the following statement is/are true?
  \begin{enumerate}
      \item $H'$ commutes with $L^2$
      \item $H'$ commutes with $L_z$ and $S_z$
      \item For a given value of principal quantum number $n$ and orbital angular momentum
quantum number $l$, there are $2(2l+1)$ degenerate eigenstates of $H_0$
\item $H_0,L^2,S^2,L_z$ and $S_z$ have a set of simultaneous eigenstates 
 \end{enumerate}
 \item Decays of mesons and baryons can be categorized as weak, strong and
electromagnetic decays depending upon the interactions involved in the processes.
Which of the following option is/are true?
\begin{enumerate}
    \item $\pi^0\rightarrow\gamma\gamma   $      is a weak decay
    \item $\Lambda^0\rightarrow\pi^0+p$     is an electromagnetic decay
    \item $K^0\rightarrow\pi^++\pi^-$  is a weak decay
    \item $\nabla^{++}\rightarrow p+\pi^+$ is a strong decay
\end{enumerate}
\item An extrinsic semiconductor shown in figure carries a current of $2mA$ along its
length parallel to +$x$ axis \\
\begin{figure}[H]
    \centering
    \begin{figure}[H]
\centering
\resizebox{0.6\textwidth}{!}{%
\begin{circuitikz}
\tikzstyle{every node}=[font=\normalsize]
\draw [line width=0.5pt, ->, >=Stealth] (8,12) -- (8,14.75);
\draw [line width=0.5pt, ->, >=Stealth] (8,11.5) -- (13.75,11.5);
\draw [line width=0.5pt, ->, >=Stealth] (8,11.5) -- (10.25,14.75);
\draw [line width=0.5pt, short] (8,11.5) -- (8,12);
\draw [line width=0.5pt, short] (8,11.5) -- (8,12);
\draw [line width=0.5pt, short] (8,11.5) -- (8,12);
\draw [line width=0.5pt, short] (9.5,13.75) -- (9.5,14.25);
\draw [line width=0.5pt, short] (8,12) -- (9.5,14.25);
\draw [line width=0.5pt, short] (9.5,14.25) -- (13.75,14.25);
\draw [line width=0.5pt, short] (8,12) -- (12.25,12);
\draw [line width=0.5pt, short] (12.25,12) -- (12.75,12);
\draw [line width=0.5pt, short] (12.75,12) -- (14.25,14.25);
\draw [line width=0.5pt, short] (13.5,14.25) -- (14.25,14.25);
\draw [line width=0.5pt, short] (9.5,13.75) -- (14.25,13.75);
\draw [line width=0.5pt, short] (14.25,14.25) -- (14.25,14);
\draw [line width=0.5pt, short] (14.25,13.75) -- (14.25,14.25);
\draw [line width=0.5pt, short] (12.75,12) -- (12.75,11.5);
\draw [line width=0.5pt, short] (12.75,11.5) -- (14.25,13.75);
\draw [line width=0.5pt, <->, >=Stealth] (7.75,11.5) -- (7.75,12);
\draw [line width=0.5pt, <->, >=Stealth] (13,11.5) -- (14.5,13.75);
\draw [line width=0.5pt, <->, >=Stealth] (8,11.25) -- (12.75,11.25);
\node [font=\normalsize] at (7,11.75) {0.5 cm};
\node [font=\normalsize] at (10.25,10.75) {5 cm};
\node [font=\normalsize] at (14.5,12.5) {3 cm};
\node [font=\normalsize] at (13.75,11.25) {\textbf{x}};
\node [font=\normalsize] at (10,15) {\textbf{y}};
\node [font=\normalsize] at (7.5,14.75) {\textbf{z}};
\draw [line width=0.5pt, dashed] (9.5,13.75) -- (9.5,14.25);
\draw [line width=0.5pt, dashed] (9.5,13.75) -- (14.5,13.75);
\end{circuitikz}
}%


\end{figure}
\end{figure}
When the majority charge carrier concentration is $12.5\times10^{13}$ $cm^{-3}$ and the
sample is exposed to a constant magnetic field applied along the +$z$ direction, a
Hall voltage of 20 mV is measured with the negative polarity at $y=0$ plane.
Take the electric charge as $1.6\times10^{-19}$C. The concentration of minority charge
carrier is negligible. Which of the following statement is/are true?
\begin{enumerate}
    \item The majority charge carrier is electron
    \item The magnitude of the applied magnetic field is 1 Tesla
    \item The electric field corresponding to the Hall voltage is in the +$y$ direction
    \item The magnitude of Hall coefficient is 50,000 $m^3C^{-1}$
\end{enumerate}
\item $A^\alpha$ and $B_\beta$ $(\alpha,\beta=1,2,3,\cdots,n)$ are contravariant and covariant vectors,
respectively. By convention, any repeated indices are summed over. Which of the
following expression is/are tensors?
\begin{enumerate}
    \item $A^\alpha B_\beta$
    \item $\frac{A^\alpha B_\beta}{A^\alpha B_\alpha}$
    \item $\frac{A_\alpha}{B_\beta}$
    \item $A^\alpha+ B_\beta$
\end{enumerate}
\item The temperature T dependence of magnetic susceptibility $\gamma$ (Column I) of certain
magnetic materials (Column II) are given below. Which of the following option
is/are correct? \\
\begin{figure}[H]
    \centering
  \begin{figure}[H]
\centering
\resizebox{0.7\textwidth}{!}{%
\begin{circuitikz}
\tikzstyle{every node}=[font=\small]
\draw [line width=0.5pt, short] (7,18) -- (7,4);
\draw [line width=0.5pt, short] (7,18) -- (17.75,18);
\draw [line width=0.5pt, short] (17.75,18) -- (17.75,4);
\draw [line width=0.5pt, short] (7,17.5) -- (17.75,17.5);
\draw [line width=0.5pt, short] (12.5,18) -- (12.5,16);
\draw [line width=0.5pt, short] (12.5,16) -- (12.5,3.75);
\draw [line width=0.5pt, short] (7,14.5) -- (17.75,14.5);
\draw [line width=0.5pt, short] (7,11.25) -- (17.75,11.25);
\draw [line width=0.5pt, short] (7,7.75) -- (17.75,7.75);
\draw [line width=0.5pt, short] (7,4) -- (17.75,4);
\node [font=\normalsize] at (7.75,17) {\textbf{(1)}};
\node [font=\normalsize] at (14.5,17) {\textbf{(P) Diamagnetic}};
\node [font=\normalsize] at (7.75,14) {\textbf{(2)}};
\node [font=\normalsize] at (14.5,14) {\textbf{(Q) Paramagnetic}};
\node [font=\normalsize] at (7.75,10.75) {\textbf{(3)}};
\node [font=\normalsize] at (14.5,10.75) {\textbf{(R) Ferromagnetic}};
\node [font=\normalsize] at (7.75,7.25) {\textbf{(4)}};
\node [font=\normalsize] at (14.75,7.25) {\textbf{(S) Antiferromagnetic}};
\node [font=\small] at (9.5,17.75) {\textbf{Column 1}};
\node [font=\small] at (15,17.75) {\textbf{Column 2}};
\draw [line width=0.5pt, short] (9,15) -- (9,17.25);
\draw [line width=0.5pt, short] (9,15) -- (11.25,15);
\draw [line width=0.5pt, ->, >=Stealth] (11,15) -- (11.25,15);
\draw [line width=0.3pt, ->, >=Stealth] (9,17) -- (9,17.5);
\draw [line width=0.3pt, dashed] (9.75,15) -- (9.75,15.75);
\draw [line width=0.3pt, short] (9.25,15.25) .. controls (9.5,15.5) and (9.75,15.5) .. (9.75,15.75);
\draw [line width=0.3pt, short] (9.75,15.75) .. controls (10.25,15.5) and (10.25,15.5) .. (11,15.25);
\draw [line width=0.3pt, short] (9,12) -- (9,14.25);
\draw [line width=0.3pt, short] (9,11.75) -- (9,12.25);
\draw [line width=0.3pt, short] (9,11.75) -- (11.5,11.75);
\draw [line width=0.3pt, ->, >=Stealth] (9,13.75) -- (9,14.25);
\draw [line width=0.3pt, ->, >=Stealth] (11.25,11.75) -- (11.5,11.75);
\draw [line width=0.3pt, dashed] (9.75,11.75) -- (9.75,14);
\draw [line width=0.3pt, short] (9.75,14) .. controls (10.25,12.5) and (10.25,12.5) .. (11.25,12);
\draw [line width=0.3pt, ->, >=Stealth] (9,8.25) -- (9,11);
\draw [line width=0.3pt, ->, >=Stealth] (9,8.25) -- (11.75,8.25);
\draw [line width=0.3pt, short] (9.25,8.5) -- (11.5,8.5);
\draw [line width=0.3pt, ->, >=Stealth] (9,4.5) -- (9,7.5);
\draw [line width=0.3pt, ->, >=Stealth] (9,4.5) -- (11.75,4.5);
\draw [line width=0.3pt, short] (9.25,7.25) .. controls (10,5.5) and (10.25,5.5) .. (11.5,4.75);
\node [font=\small] at (8.75,17.25) {$|\chi|$};
\node [font=\small] at (8.75,14) {$|\chi|$};
\node [font=\small] at (8.75,10.75) {$|\chi|$};
\node [font=\small] at (8.75,7.25) {$|\chi|$};
\node [font=\small] at (8.75,11.75) {$0$};
\node [font=\small] at (8.75,15) {$0$};
\node [font=\small] at (8.75,4.5) {$0$};
\node [font=\small] at (8.75,8.25) {$0$};
\node [font=\small] at (11.25,14.75) {$T$};
\node [font=\small] at (11.5,11.5) {$T$};
\node [font=\small] at (11.75,8) {$T$};
\node [font=\small] at (11.75,4.25) {$T$};
\end{circuitikz}
}%


\end{figure}
\end{figure}
\begin{enumerate}
    \item 2 - P, 4 - Q, 3 - S
        \item 4 - P, 1 - Q, 2 - R
    \item 4 - Q, 2 - R, 1 - S
    \item 3 - P, 4 - Q, 2 - R

\end{enumerate}


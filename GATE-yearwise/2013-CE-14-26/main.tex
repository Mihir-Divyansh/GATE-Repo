\iffalse
    \title{2014-CE-14-26}
    \author{EE24BTECH11001 -  ADITYA TRIPATHY}
    \section{ce}
    \chapter{2014}
\fi
\item[14.] 
        For a saturated cohesive soil, a triaxial test yields the angle of internal friction
        $\brak{\phi}$ as zero. The conducted soil test is
        \hfill{\brak{2014-CE}}
        \begin{enumerate}
                \begin{multicols}{2}
                \item Consolidated Drained $\brak{\textnormal{CD}}$ test
                    \columnbreak
                \item Consolidated Undrained $\brak{\textnormal{CU}}$ test
                \end{multicols}
                \begin{multicols}{2}
                \item Unconfined Compression $\brak{\textnormal{UC}}$ test
                    \columnbreak
                \item Unconsolidated Undrained $\brak{\textnormal{UU}}$ test
                \end{multicols}
        \end{enumerate}

    \item[15.] The action of negative skin friction on the pile is to 		
        \hfill{\brak{2014-CE}}
        \begin{enumerate}
            \item increase the ultimate load on the pile
            \item  reduce the allowable load on the pile 
            \item  maintain the working load on the pile
            \item  reduce the settlement of the pile
        \end{enumerate}


    \item[16.] A long slope is formed in a soil with shear strength parameters: $c^{\prime} = 0$ 
        and $\phi^{\prime} = 34^{\degree}$. A firm stratum lies below the slope and it is assumed that
        the water table may occasionaly rise to the surface, with seepage taking place parallelto the slope.
        Use $\gamma_{sat}= 18 kN/m^{3}$ and $\gamma_w = 10 kN/m^3$. The maximum slope angle $\brak{\textnormal{in degrees}}$
        to ensure a factor of safety of 1.5, assuming a potential failure surface parallel to the slope, 
        would be
        \hfill{\brak{2014-CE}}
        \begin{multicols}{4}

            \begin{enumerate}
                \item  45.3 \columnbreak
                \item  44.7 \columnbreak
                \item  12.3 \columnbreak
                \item  11.3
            \end{enumerate}
        \end{multicols}

    \item[17.] An incompressible homogenous fluid is flowing steadily in a variable diameter pipe
        having the large and small diameters as 15cm and 5cm, respectively. If the velocity at a
        section at the 15cm diameter portion of the ipe is 2.5m/s, the velocity of the fluid 
        $\brak{\textnormal{in m/s}}$ at a section falling in 5cm portion of the pipe is
        \hfill{\brak{2014-CE}}

    \item[18.] A conventional flow duration is a plot between 

        \hfill{\brak{2014-CE}}
        \begin{enumerate}
            \item Flow and percentage time flow is exceeded
            \item Duration of flooding and ground level elevation
            \item Duration of water supply in a city and proportion of area receiving supply exceeding
                this duration
            \item Flow rate and duration of time taken to empty a reservoir at that flow rate
        \end{enumerate}


    \item[19.] In reservoirs with an uncontrolled spillway, the peak of the plotted outflow
        hydrograph
        \hfill{\brak{2014-CE}}
        \begin{enumerate}
            \item lies outside the plotted inflow hydrograph
            \item lies on the recession limb of the plotted inflow hydrograph
            \item lies on the peak of the inflow hydrograph
            \item is higher thhan the peak iof the plotted inflow hydrograph
        \end{enumerate}
    \item[20.] The dimension for kinematic viscosity is 
        \hfill{\brak{2014-CE}}
        \begin{multicols}{4}

            \begin{enumerate}
                \item $\frac{L}{MT}$ 
                \item $\frac{L}{T^2}$ 
                \item $\frac{L^2}{T}$ 
                \item $\frac{ML}{T}$ 
            \end{enumerate}
        \end{multicols}

    \item[21.] Some of the nontoxic metals normally found in natural water are
        \hfill{\brak{2014-CE}}
        \begin{enumerate}
                \begin{multicols}{2}
                \item arsenic, lead and mercury
                    \columnbreak
                \item calcium, sodium and silver
                \end{multicols}
                \begin{multicols}{2}
                \item cadmium, chromium and copper 
                    \columnbreak
                \item iron, manganese and magnesium
                \end{multicols}
        \end{enumerate}


    \item[22.] The amount of $CO_2$ generated $\brak{\textnormal{in kg}}$ while completely oxidizing
        one kg of $CH_4$ to the end products is 
        \hfill{\brak{2014-CE}}

    \item[23.] The minimum value of 15 minute peak hour factor on a section of a road is
        \hfill{\brak{2014-CE}}
        \begin{multicols}{4}

            \begin{enumerate}
                \item  0.10 \columnbreak
                \item  0.20 \columnbreak
                \item  0.25  \columnbreak
                \item  0.33
            \end{enumerate}
        \end{multicols}
    \item[24.] The following statements are related to temperature stresses developed in concrete 
        pavement slabs with free edges $\textnormal{without any restraint}$
        \begin{enumerate}
            \item[P.] The temperature stresses will be zero during both day and night times if
                pavement slab is considered weightless
            \item[Q.] The temperature stresses will be compressive at the bottom of the slab during night
                time if the self-weight of the pavement slab is considered
            \item[R.] The temperature stresses will be compressive at the bottom of the slab during day time
                if the self-weight of the pavement is considered 
        \end{enumerate}
        The TRUE statement$\brak{\textnormal{s}}$ is $\brak{\textnormal{are}}$ 
        \hfill{\brak{2014-CE}}
        \begin{multicols}{4}
            \begin{enumerate}
                \item  P only \columnbreak
                \item  Q only \columnbreak
                \item  P and Q only  \columnbreak
                \item  P and R only
            \end{enumerate}
        \end{multicols}  
    \item[25.] The Reduced Levels $\brak{\textnormal{RLs}}$ of the points $P$ and $Q$ are $+49.600m$ and 
        $+51.870m$ respectively. Distance $PQ$ is 20m. The distance $\brak{\textnormal{in m from P}}$
        at which the $+51.000m$ contour cuts the line $PQ$ is
        \hfill{\brak{2014-CE}}
        \begin{multicols}{4}

            \begin{enumerate}
                \item  15.00 \columnbreak
                \item  12.33 \columnbreak
                \item  3.52  \columnbreak
                \item  2.27
            \end{enumerate}
        \end{multicols}
        \textbf{Q.26 - Q.55 carry two marks each.}\\
    \item[26.] If the following equation establishes equilibrium in slightly bent position, the mid-
        span deflection of a member shown in the figure is
        \begin{align}
            \frac{d^2 y}{dx^2} + \frac{P}{EI}y = 0
        \end{align}
        If $a$ is amplitude constany for y, then
        \begin{center}
            \resizebox{0.5\textwidth}{!}{
                \begin{circuitikz}
                    \tikzstyle{every node}=[font=\normalsize]
                    \draw [dashed] (2.75,13) -- (8.75,13);
                    \draw [->, >=Stealth] (2.75,13) -- (2.75,15.5);
                    \draw (2.75,13) to[short] (2.25,12.5);
                    \draw (2.75,13) to[short] (3.25,12.5);
                    \draw (2.25,12.5) to[short] (3.25,12.5);
                    \draw (8.5,13) to[short] (8,12.5);
                    \draw (8.5,13) to[short] (9,12.5);
                    \draw (8,12.5) to[short] (9,12.5);
                    \draw [short] (2.75,13) .. controls (5.75,12) and (5.75,11.75) .. (8.5,13);
                    \draw [dashed] (5.75,13) -- (5.75,12.25);
                    \draw [<->, >=Stealth] (8.5,13) -- (11.5,13);
                    \draw [->, >=Stealth] (1,13) -- (2.5,13);
                    \draw [<->, >=Stealth] (2.75,11.5) -- (8.5,11.5);
                    \draw [short] (2.75,11.75) -- (2.75,11.25);
                    \draw [short] (8.5,11.75) -- (8.5,11.25);
                    \node [font=\normalsize] at (5.5,11) {L};
                    \node [font=\normalsize] at (1,13.5) {P};
                    \node [font=\normalsize] at (6,13) {};
                    \node [font=\normalsize] at (2.5,16) {y};
                    \node [font=\normalsize] at (6,12.5) {y};
                    \node [font=\normalsize] at (5.5,13.5) {EI};
                    \node [font=\normalsize] at (9.5,13.5) {P};
                    \node [font=\normalsize] at (11.5,12.5) {x};
                \end{circuitikz}
                }
        \end{center}
        \hfill{\brak{2014-CE}}
        \begin{enumerate}
                \begin{multicols}{2}
                \item $y = \frac{1}{P}\brak{1 - a\cos \frac{2\pi x}{L}}$ 
                    \columnbreak
                \item $y = \frac{1}{P}\brak{1 - a\sin \frac{2\pi x}{L}}$ 
                \end{multicols}
                \begin{multicols}{2}
                \item $y = a\sin \frac{n\pi x}{L}$ 
                    \columnbreak
                \item $y = a\cos\frac{n\pi x}{L}$ 
                \end{multicols}
        \end{enumerate}

\iffalse
	\chapter{2021}
	\author{AI24BTECH11001}
	\section{ph}
\fi	
    \item As shown in the figure, two metal-semiconductor junctions are formed between an n-type semiconductor $S$ and metal $M$. The work functions of $S$ and $M$ are $\varphi_s$ and $\varphi_m$, respectively with $\varphi_m > \varphi_s$.
    \begin{figure}[H]
\centering
\resizebox{0.3\textwidth}{!}{%
\begin{circuitikz}
\tikzstyle{every node}=[font=\LARGE]
\draw [ fill={rgb,255:red,33; green,194; blue,196} ] (-2.25,7.5) rectangle (-1,6.5);
\draw [ fill={rgb,255:red,212; green,206; blue,206} ] (-1,7.5) rectangle (0.5,6.5);
\draw [ fill={rgb,255:red,20; green,219; blue,186} ] (0.5,7.5) rectangle (1.75,6.5);
\draw  (-3,7) rectangle (-2.25,7);
\draw  (1.75,7) rectangle (2.5,7);
\draw [short] (2.5,8.75) -- (2.5,7);
\draw [short] (-3,7) -- (-3,8.75);
\draw (-3,8.75) to[battery1] (2.5,8.75);
\node [font=\LARGE] at (-1.75,7) {M};
\node [font=\LARGE] at (-0.25,7) {S};
\node [font=\LARGE] at (1.25,7) {M};
\end{circuitikz}
}%
\label{fig:my_label}
\end{figure}
    The I-V characteristics (on linear scale) of the junctions is best represented by 
    \begin{enumerate}
        \item \begin{figure}[H]
\resizebox{0.3\textwidth}{!}{%
\begin{circuitikz}
\tikzstyle{every node}=[font=\normalsize]
\draw [->, >=Stealth] (-2.25,8.75) -- (3.75,8.75);
\draw [->, >=Stealth] (0.5,6) -- (0.5,11.5);
\begin{scope}[rotate around={46.75:(-1.5,6.5)}]
\draw[domain=-1.5:4.75,samples=100,smooth] plot (\x,{0.8*sin(1*\x r +1.5 r ) +6.5});
\end{scope}
\node [font=\normalsize] at (0.25,11.5) {I};
\node [font=\normalsize] at (3.75,8.25) {V};
\end{circuitikz}
}%

\label{fig:my_label}
\end{figure}

    \item \begin{figure}[H]
\resizebox{0.3\textwidth}{!}{%
\begin{circuitikz}
\tikzstyle{every node}=[font=\normalsize]
\draw [->, >=Stealth] (-2.25,8.75) -- (3.75,8.75);
\draw [->, >=Stealth] (0.5,6) -- (0.5,11.5);
\begin{scope}[rotate around={53.5:(-0.5,6.5)}]
\draw[domain=-0.5:5.5,samples=100,smooth] plot (\x,{0.8*sin(1*\x r +0.5 r ) +6.5});
\end{scope}
\node [font=\normalsize] at (0.25,11.5) {I};
\node [font=\normalsize] at (3.75,8.25) {V};
\end{circuitikz}
}%

\label{fig:my_label}
\end{figure}

    \item \begin{figure}[H]
\resizebox{0.3\textwidth}{!}{%
\begin{circuitikz}
\tikzstyle{every node}=[font=\normalsize]
\draw [->, >=Stealth] (-2,8.75) -- (3.25,8.75);
\draw [->, >=Stealth] (0.5,6.5) -- (0.5,11.5);
\node [font=\normalsize] at (0.25,11.5) {I};
\node [font=\normalsize] at (3.75,8.25) {V};
\draw [short] (-1.25,11.25) .. controls (0.5,7.25) and (0.75,8.5) .. (2.75,11.25);
\end{circuitikz}
}%

\label{fig:my_label}
\end{figure}

    \item \begin{figure}[h]
\resizebox{0.3\textwidth}{!}{%
\begin{circuitikz}
\tikzstyle{every node}=[font=\normalsize]
\draw [->, >=Stealth] (-1.75,8.75) -- (3.5,8.75);
\draw [->, >=Stealth] (0.75,6.5) -- (0.75,11.5);
\node [font=\normalsize] at (0.25,11.5) {I};
\node [font=\normalsize] at (3.75,8.25) {V};
\draw [short] (0.75,8.75) .. controls (1.25,10.25) and (2,10.5) .. (2.75,10.5);
\draw [short] (0.75,8.75) .. controls (0,7.75) and (-0.25,7.75) .. (-1.5,7.75);
\end{circuitikz}
}%

\label{fig:my_label}
\end{figure}
    \end{enumerate}
    \item Consider a tiny current loop driven by a sinusoidal alternating current. If the surface integral of its time-averaged Poynting vector is constant, then the magnitude of the time-averaged magnetic field intensity, at any arbitrary position, $\overrightarrow{r}$, is proportional to
    \begin{enumerate}
        \item $\frac{1}{r^3}$
        \item $\frac{1}{r^2}$
        \item $\frac{1}{r}$
        \item $r$
    \end{enumerate}

    \item Consider a solenoid of length $L$ and radius $R$, where $R << L$. A steady-current flows through the solenoid. The magnetic field is uniform inside the solenoid and zero outside.
    Among the given options, choose the one that best represents the variation in the magnitude of the vector potential, $\brak{0,A_\varphi,0}$ at $z = \frac{L}{2}$, as a function of the radial distance (r) in cylindrical coordinates.\\
    Useful information: The curl of a vector $\overrightarrow{F}$, in cylindrical coordinates is
    \begin{align*}
        \overrightarrow{\nabla} \times \overrightarrow{F}(r, \varphi, z) = \hat{r} \sbrak{ \frac{1}{r} \frac{\partial F_z}{\partial \varphi} - \frac{\partial F_\varphi}{\partial z} } + \hat{\varphi} \sbrak{ \frac{\partial F_r}{\partial z} - \frac{\partial F_z}{\partial r}} + \hat{z} \frac{1}{r} \sbrak{ \frac{\partial (r F_\varphi)}{\partial r} - \frac{\partial F_r}{\partial \varphi} }
    \end{align*}
    \begin{enumerate}
        \item \begin{figure}[H]

\resizebox{0.4\textwidth}{!}{%
\begin{circuitikz}
\tikzstyle{every node}=[font=\normalsize]
\draw [->, >=Stealth] (-1.25,7.75) -- (2.75,7.75);
\draw [->, >=Stealth] (-1.25,7.75) -- (-1.25,10.25);
\draw [dashed] (0.5,9.5) -- (0.5,7.75);
\node [font=\normalsize] at (-1.5,10) {$A_\phi$};
\node [font=\normalsize] at (2.5,8) {r};
\node [font=\normalsize] at (0.5,7.5) {R};
\draw [line width=0.8pt, short] (-1.25,9.5) -- (0.5,9.5);
\draw [line width=0.8pt, short] (0.5,9.5) .. controls (1,8.25) and (1.25,8.25) .. (2,8);
\end{circuitikz}
}%

\label{fig:my_label}
\end{figure}
        \item \begin{figure}[H]
\
\resizebox{0.4\textwidth}{!}{%
\begin{circuitikz}
\tikzstyle{every node}=[font=\normalsize]
\draw [->, >=Stealth] (-1.25,7.75) -- (2.75,7.75);
\draw [->, >=Stealth] (-1.25,7.75) -- (-1.25,10.25);
\draw [dashed] (0.5,9.5) -- (0.5,7.75);
\node [font=\normalsize] at (-1.5,10) {$A_\phi$};
\node [font=\normalsize] at (2.5,8) {r};
\node [font=\normalsize] at (0.5,7.5) {R};
\draw [line width=0.8pt, short] (0.5,9.5) -- (2.25,9.5);
\draw [line width=0.8pt, short] (-1.25,7.75) -- (0.5,9.5);
\end{circuitikz}
}%

\label{fig:my_label}
\end{figure}
        \item \begin{figure}[H]
\resizebox{0.4\textwidth}{!}{%
\begin{circuitikz}
\tikzstyle{every node}=[font=\normalsize]
\draw [->, >=Stealth] (-1.25,7.75) -- (2.75,7.75);
\draw [->, >=Stealth] (-1.25,7.75) -- (-1.25,10.25);
\draw [dashed] (0.5,9.5) -- (0.5,7.75);
\node [font=\normalsize] at (-1.5,10) {$A_\phi$};
\node [font=\normalsize] at (2.5,8) {r};
\node [font=\normalsize] at (0.5,7.5) {R};
\draw [line width=0.8pt, short] (0.5,9.5) .. controls (1,8.25) and (1.25,8.25) .. (2,8);
\draw [line width=0.8pt, short] (-1.25,7.75) -- (0.5,9.5);
\end{circuitikz}
}%

\label{fig:my_label}
\end{figure}
        \item \begin{figure}[H]
\resizebox{0.4\textwidth}{!}{%
\begin{circuitikz}
\tikzstyle{every node}=[font=\normalsize]
\draw [->, >=Stealth] (-1.25,7.75) -- (2.75,7.75);
\draw [->, >=Stealth] (-1.25,7.75) -- (-1.25,10.25);
\draw [dashed] (0.5,9.5) -- (0.5,7.75);
\node [font=\normalsize] at (-1.5,10) {$A_\phi$};
\node [font=\normalsize] at (2.5,8) {r};
\node [font=\normalsize] at (0.5,7.5) {R};
\draw [line width=0.8pt, short] (-1.25,9.5) -- (2.25,9.5);
\end{circuitikz}
}%

\label{fig:my_label}
\end{figure}
    \end{enumerate}

    \item Assume that  $^{13}\text{N}$ $\brak{z=7}$ undergoes first forbidden $\beta^{+}$ decay from its ground state with spin-parity $J^{\pi}_i$, to a final state with spin $J^{\pi}_f$. The possible values for $J^{\pi}_i$ and $J^{\pi}_f$, respectively, are
    \begin{enumerate}
        \item $\frac{1^{-}}{2} , \frac{5^{+}}{2}$
        \item $\frac{1^{+}}{2} , \frac{5^{+}}{2}$
        \item $\frac{1^{-}}{2} , \frac{1^{-}}{2}$
        \item $\frac{1^{+}}{2} , \frac{1^{-}}{2}$
    \end{enumerate}

    \item In an experiment, it is seen that an electric-dipole $\brak{E1}$ transition can connect an initial nuclear state of spin-parity $J^{\pi}_i = 2^{+}$ to a final state $J^{\pi}_f$ . All possible values of $J^{\pi}_f$  are
    \begin{enumerate}
        \item $1^{+} , 2^{+}$
        \item $1^{+} , 2^{+} , 3^{+}$
        \item $1^{-} , 2^{-}$
        \item $1^{-} , 2^{-} , 3^{-}$
    \end{enumerate}

    \item Choose the correct statement of the following
    \begin{enumerate}
        \item Silicon is a direct band gap semiconductor.
        \item Conductivity of metals decreases with increase in temperature.
        \item Conductivity of semiconductors decreases with increase in temperature.
        \item Gallium Arsenide is an indirect band gap semiconductor.
    \end{enumerate}

    \item A two-dimensional square lattice has lattice constant $a$. $k$ represents the wavevector in reciprocal space. The coordinates $\brak{k_x, k_y}$ of reciprocal space where band gap(s) can occur, are
    \begin{enumerate}
        \item $\brak{0,0}$
        \item $\brak{\pm \frac{\pi}{a} , \pm \frac{\pi}{a}}$
        \item $\brak{\pm \frac{\pi}{a} , \pm \frac{\pi}{1.3a}}$
        \item $\brak{\pm \frac{\pi}{3a} , \pm \frac{\pi}{a}}$
    \end{enumerate}

    \item As shown in the figure, an electromagnetic wave with intensity $I_{I}$ is incident at the interface of two media having refractive indices $n_{1} = 1$ and $n_{2} = \sqrt{3}$ The wave is reflected with intensity $I_{R}$ and transmitted with intensity $I_{T}$ Permeability of each medium is the same. (Reflection coefficient $R = \frac{I_{R}}{I_{I}}$ and Transmission coefficient $T = \frac{I_{T}}{I_{I}} $
    \begin{figure}[H]
\centering
\resizebox{0.5\textwidth}{!}{%
\begin{circuitikz}
\tikzstyle{every node}=[font=\normalsize]
\draw (-1.5,9.75) to[short] (0,8.25);
\draw (0,8.25) to[short] (1.5,9.75);
\draw (0,8.25) to[short] (1.5,6.75);
\draw [dashed] (0,10.25) -- (0,6.5);
\draw [short] (-2.25,8.25) -- (2.75,8.25);
\draw [->, >=Stealth] (-1,9.25) -- (-0.75,9);
\draw [->, >=Stealth] (0.75,9) -- (1,9.25);
\draw [->, >=Stealth] (1,7.25) -- (1.25,7);
\draw [->, >=Stealth] (0,9) .. controls (-0.25,9) and (-0.25,8.75) .. (-0.5,8.75) ;
\draw [->, >=Stealth] (0,9) .. controls (0.25,8.5) and (0.25,8.75) .. (0.25,8.5) ;
\node [font=\normalsize] at (2.5,8.5) {$n_1$};
\node [font=\normalsize] at (2.5,8) {$n_2$};
\node [font=\normalsize] at (-0.25,9.25) {$\theta_I$};
\node [font=\normalsize] at (0.25,9) {$\theta_R$};
\node [font=\normalsize] at (0.75,7.25) {$\theta_T$};
\draw [->, >=Stealth] (0,7.5) .. controls (0.5,7.25) and (0.5,7.5) .. (0.75,7.5) ;
\end{circuitikz}
}%

\label{fig:my_label}
\end{figure}
    Choose the correct statement(s)
    \begin{enumerate}
        \item $R = 0$ if $\theta_1 = 0^{\circ}$ and polarization of incident light is parallel to the plane of incidence.
         \item $T = 1$ if $\theta_1 = 60^{\circ}$ and polarization of incident light is parallel to the plane of incidence.
         \item $R = 0$ if $\theta_1 = 60^{\circ}$ and polarization of incident light is perpendicular to the plane of incidence.
          \item $T = 1$ if $\theta_1 = 60^{\circ}$ and polarization of incident light is perpendicular
          to the plane of incidence.
    \end{enumerate}

    \item A material is placed in a magnetic field intensity $H$. As a result, bound current density $J_{b}$ is induced and magnetization of the material is $M$. The magnetic flux density is $B$. Choose the correct option(s) valid at the surface of the material.
    \begin{enumerate}
        \item $\nabla . M = 0$
        \item $\nabla . B = 0$
        \item $\nabla . h = 0$
        \item $\nabla . J_b = 0$
    \end{enumerate}

    \item For a finite system of Fermions where the density of states increases with energy, the chemical potential
    \begin{enumerate}
        \item decreases with temperature
        \item increases with temperature
        \item does not vary with temperature
        \item corresponds to the energy where the occupation probability is $0.5$
    \end{enumerate}

    \item Among the term symbols $^4 S_1 , ^2 D_{\frac{7}{2}} , ^3 S_1 \text{ and } ^2 D_{\frac{5}{2}}$ choose the option(s) possible in the LS coupling notation.
    \begin{enumerate}
        \item $^4 S_1$
        \item $^2 D_{\frac{7}{2}}$
        \item $^3 S_1$
        \item $^2 D_{\frac{5}{2}}$
    \end{enumerate}

    \item To sustain lasing action in a three-level laser as shown in the figure, necessary condition(s) is (are)
    \begin{figure}[!ht]
\centering
\resizebox{0.3\textwidth}{!}{%
\begin{circuitikz}
\tikzstyle{every node}=[font=\Large]
\draw [ color={rgb,255:red,69; green,23; blue,238}, line width=1pt, short] (-1.25,9.25) -- (1.25,9.25);
\draw [ color={rgb,255:red,51; green,30; blue,210}, line width=1pt, short] (-2.75,8.5) -- (-0.5,8.5);
\draw [ color={rgb,255:red,111; green,45; blue,210}, line width=1pt, short] (-1.25,7) -- (1.25,7);
\draw [ color={rgb,255:red,228; green,17; blue,17}, line width=1pt, ->, >=Stealth] (-1,8.5) -- (-1,7);
\draw[domain=0.25:3.25,samples=100,smooth, line width=1pt] plot (\x,{0.2*sin(4.43*\x r +0.15000000000000002 r ) +7.75});
\draw [line width=1pt, ->, >=Stealth] (3.25,7.75) -- (3.75,7.75);
\node [font=\Large] at (-1.5,7) {0};
\node [font=\Large] at (-3,8.5) {1};
\node [font=\Large] at (-1.5,9.5) {2};
\end{circuitikz}
}%

\label{fig:my_label}
\end{figure}
    \begin{enumerate}
        \item lifetime of the energy level $1$ should be greater than that of energy level $2$
        \item population of the particles in level $1$ should be greater than that of level $0$
        \item lifetime of the energy level $2$ should be greater than that of energy level $0$
        \item population of the particles in level $2$ should be greater than that of level $1$
    \end{enumerate}

    \item If $y_n\brak{x}$ is a solution of the differential equation
    \begin{align*}
        y" - 2xy' + 2ny = 0
    \end{align*}
    where n is an integer and the prime $\brak{'}$ denotes differentiation with respect to x, then acceptable plot(s) of $\psi_n \brak{x} = e^{\frac{-x^2}{2}} y_n \brak{x}$, is(are) 
    \begin{enumerate}
        \item \begin{figure}[H]

\resizebox{0.4\textwidth}{!}{%
\begin{circuitikz}
\tikzstyle{every node}=[font=\normalsize]
\draw [short] (-3.25,10) -- (-3.25,5);
\draw [short] (-3.25,7.5) -- (2.75,7.5);
\draw [short] (-3.5,10) -- (-3.25,10);
\draw [short] (-3.5,8.75) -- (-3.25,8.75);
\draw [short] (-3.5,6.25) -- (-3.25,6.25);
\draw [short] (-3.5,5) -- (-3.25,5);
\draw [short] (-1.75,7.5) -- (-1.75,7.25);
\draw [short] (-0.25,7.5) -- (-0.25,7.25);
\draw [short] (1.25,7.5) -- (1.25,7.25);
\draw [short] (2.75,7.5) -- (2.75,7.25);
\node [font=\normalsize] at (-3.5,7.5) {0};
\node [font=\normalsize] at (-3.75,8.75) {1};
\node [font=\normalsize] at (-3.75,10) {2};
\node [font=\normalsize] at (-3.75,6.25) {-1};
\node [font=\normalsize] at (-3.75,5) {-2};
\node [font=\normalsize] at (-0.25,7) {0};
\node [font=\normalsize] at (1.25,7) {2};
\node [font=\normalsize] at (2.75,7) {4};
\node [font=\normalsize] at (-1.75,7) {-2};
\node [font=\normalsize] at (-3.25,7) {-4};
\draw [line width=0.9pt, short] (-3.25,8.75) -- (2.75,8.75);
\node [font=\normalsize] at (2,9.5) {n = 0};
\node [font=\normalsize] at (2.25,6.25) {x};
\node [font=\normalsize] at (-4.5,8.75) {$\psi_n (x)$};
\end{circuitikz}
}%

\label{fig:my_label}
\end{figure}
        \item \begin{figure}[H]

\resizebox{0.4\textwidth}{!}{%
\begin{circuitikz}
\tikzstyle{every node}=[font=\normalsize]
\draw [short] (-3.25,10) -- (-3.25,5);
\draw [short] (-3.25,7.5) -- (2.75,7.5);
\draw [short] (-3.5,10) -- (-3.25,10);
\draw [short] (-3.5,8.75) -- (-3.25,8.75);
\draw [short] (-3.5,6.25) -- (-3.25,6.25);
\draw [short] (-3.5,5) -- (-3.25,5);
\draw [short] (-1.75,7.5) -- (-1.75,7.25);
\draw [short] (-0.25,7.5) -- (-0.25,7.25);
\draw [short] (1.25,7.5) -- (1.25,7.25);
\draw [short] (2.75,7.5) -- (2.75,7.25);
\node [font=\normalsize] at (-3.5,7.5) {0};
\node [font=\normalsize] at (-3.75,8.75) {1};
\node [font=\normalsize] at (-3.75,10) {2};
\node [font=\normalsize] at (-3.75,6.25) {-1};
\node [font=\normalsize] at (-3.75,5) {-2};
\node [font=\normalsize] at (-0.25,7) {0};
\node [font=\normalsize] at (1.25,7) {2};
\node [font=\normalsize] at (2.75,7) {4};
\node [font=\normalsize] at (-1.75,7) {-2};
\node [font=\normalsize] at (-3.25,7) {-4};
\draw [line width=0.9pt, short] (-3.25,7.5) .. controls (1.25,9.5) and (-1.25,8.75) .. (2.75,7.5);
\node [font=\normalsize] at (2,9.5) {n = 0};
\node [font=\normalsize] at (2.25,6.25) {x};
\node [font=\normalsize] at (-4.5,8.75) {$\psi_n (x)$};
\end{circuitikz}
}%

\label{fig:my_label}
\end{figure}
        \item \begin{figure}[H]

\resizebox{0.4\textwidth}{!}{%
\begin{circuitikz}
\tikzstyle{every node}=[font=\normalsize]
\draw [short] (-3.25,10) -- (-3.25,5);
\draw [short] (-3.25,7.5) -- (2.75,7.5);
\draw [short] (-3.5,10) -- (-3.25,10);
\draw [short] (-3.5,8.75) -- (-3.25,8.75);
\draw [short] (-3.5,6.25) -- (-3.25,6.25);
\draw [short] (-3.5,5) -- (-3.25,5);
\draw [short] (-1.75,7.5) -- (-1.75,7.25);
\draw [short] (-0.25,7.5) -- (-0.25,7.25);
\draw [short] (1.25,7.5) -- (1.25,7.25);
\draw [short] (2.75,7.5) -- (2.75,7.25);
\node [font=\normalsize] at (-3.5,7.5) {0};
\node [font=\normalsize] at (-3.75,8.75) {1};
\node [font=\normalsize] at (-3.75,10) {2};
\node [font=\normalsize] at (-3.75,6.25) {-1};
\node [font=\normalsize] at (-3.75,5) {-2};
\node [font=\normalsize] at (-0.25,7) {0};
\node [font=\normalsize] at (1.25,7) {2};
\node [font=\normalsize] at (2.75,7) {4};
\node [font=\normalsize] at (-1.75,7) {-2};
\node [font=\normalsize] at (-3.25,7) {-4};
\draw [line width=0.9pt, short] (-3.25,7.5) .. controls (-1,3.5) and (0.5,11.5) .. (2.75,7.5);
\node [font=\normalsize] at (2,9.5) {n = 1};
\node [font=\normalsize] at (2.25,6.25) {x};
\node [font=\normalsize] at (-4.5,8.75) {$\psi_n (x)$};
\end{circuitikz}
}%

\label{fig:my_label}
\end{figure}
        \item \begin{figure}[H]

\resizebox{0.4\textwidth}{!}{%
\begin{circuitikz}
\tikzstyle{every node}=[font=\normalsize]
\draw [short] (-3.25,10) -- (-3.25,5);
\draw [short] (-3.25,7.5) -- (2.75,7.5);
\draw [short] (-3.5,10) -- (-3.25,10);
\draw [short] (-3.5,8.75) -- (-3.25,8.75);
\draw [short] (-3.5,6.25) -- (-3.25,6.25);
\draw [short] (-3.5,5) -- (-3.25,5);
\draw [short] (-1.75,7.5) -- (-1.75,7.25);
\draw [short] (-0.25,7.5) -- (-0.25,7.25);
\draw [short] (1.25,7.5) -- (1.25,7.25);
\draw [short] (2.75,7.5) -- (2.75,7.25);
\node [font=\normalsize] at (-3.5,7.5) {0};
\node [font=\normalsize] at (-3.75,8.75) {1};
\node [font=\normalsize] at (-3.75,10) {2};
\node [font=\normalsize] at (-3.75,6.25) {-1};
\node [font=\normalsize] at (-3.75,5) {-2};
\node [font=\normalsize] at (-0.25,7) {0};
\node [font=\normalsize] at (1.25,7) {2};
\node [font=\normalsize] at (2.75,7) {4};
\node [font=\normalsize] at (-1.75,7) {-2};
\node [font=\normalsize] at (-3.25,7) {-4};
\node [font=\normalsize] at (2,9.5) {n = 1};
\node [font=\normalsize] at (2.25,6.25) {x};
\node [font=\normalsize] at (-4.5,8.75) {$\psi_n (x)$};
\draw [line width=0.8pt, short] (0.75,10.25) -- (-1.25,5);
\end{circuitikz}
}%

\label{fig:my_label}
\end{figure}
    \end{enumerate}



\iffalse
    \title{Assignment}
    \author{EE24BTECH11034}
    \section{ph}
    \chapter{2015}
  \fi
\item Let $\vec{L}$ and $\vec{p}$ be the angular and linear momentum operators, respectively, for a particle. The commutator $\sbrak{L_z, p_y}$ gives

\begin{enumerate}
    \item $-i\hbar p_x$
    \item $0$
    \item $i\hbar p_x$
    \item $i\hbar p_y$
\end{enumerate}

\item The decay $\mu^+ \rightarrow e^+ + \gamma$ is forbidden, because it violates

\begin{enumerate}
    \item momentum and lepton number conservations
    \item baryon and lepton number conservations
    \item angular momentum conservation
    \item lepton number conservation
\end{enumerate}

\item An operator for a spin-$\frac{1}{2}$ particle is given by $\hat{A}=\lambda \vec{\sigma} \cdot \vec{B}$, where $\vec{B}=\frac{B}{\sqrt{2}}\hat{x}+i\frac{B}{\sqrt{2}}\hat{y}$, $\vec{\sigma}$ denotes Pauli matrices and $\lambda$ is a constant. The eigenvalues of $\hat{A}$ are

\begin{enumerate}
    \item $\pm \lambda B \sqrt{2}$
    \item $\pm \lambda B$
    \item $0, \lambda B$
    \item $0, -\lambda B$
\end{enumerate}

\item In an inertial frame S, two events A and B take place at $\brak{ct_A = 0, \vec{r}_A = 0}$ and $\brak{ct_B = 0, \vec{r}_B = 2\hat{y}}$, respectively. The times at which these events take place in a frame $S'$ moving with a velocity $0.6cy$ with respect to S are given by

\begin{enumerate}
    \item $ct_A' = 0; ct_B' = -1.5$
    \item $ct_A' = 0; ct_B' = 0$
    \item $ct_A' = 0; ct_B' = 1.5$
    \item $ct_A' = 0; ct_B' = 0.5$
\end{enumerate}

\item Given that the magnetic flux through the closed loop PQRSP is $\Phi$. If $\int_P^R \vec{A} \cdot \vec{dl} = \phi_1$ along PQR, the value of $\int_P^R \vec{A} \cdot \vec{dl}$ along PSR is

\begin{enumerate}
    \item $\Phi - \phi_1$
    \item $\phi_1 - \Phi$
    \item $-\phi_1$
    \item $\phi_1$
\end{enumerate}

\item If $f\brak{x} = e^{-x^2}$ and $g\brak{x} = |x|e^{-x^2}$, then

\begin{enumerate}
    \item $f$ and $g$ are differentiable everywhere
    \item $f$ is differentiable everywhere but $g$ is not
    \item $g$ is differentiable everywhere but $f$ is not
    \item $g$ is discontinuous at $x=0$
\end{enumerate}

\item In Bose-Einstein condensates, the particles

\begin{enumerate}
    \item have strong interparticle attraction
    \item condense in real space
    \item have overlapping wavefunctions
    \item have large and positive chemical potential
\end{enumerate}


\item Consider a system of $N$ non-interacting spin-$\frac{1}{2}$ particles, each having a magnetic moment $\mu$, is in a magnetic field $\vec{B} = B\hat{z}$. If $E$ is the total energy of the system, the number of accessible microstates $\Omega$ is given  
 by

\begin{enumerate}
    \item $\Omega = \dfrac{N!}{\frac{1}{2}\brak{N - \frac{E}{\mu B}}!\frac{1}{2}\brak{N + \frac{E}{\mu B}}!}$
    \item $\Omega = \dfrac{\brak{N - \frac{E}{\mu B}}!}{\brak{N + \frac{E}{\mu B}}!}$
    \item $\Omega = \frac{1}{2}\brak{N - \frac{E}{\mu B}}!\frac{1}{2}\brak{N + \frac{E}{\mu B}}!$
    \item $\Omega = \dfrac{N!}{\brak{N + \frac{E}{\mu B}}!}$
\end{enumerate}

\item For a black body radiation in a cavity, photons are created and annihilated freely as a result of emission and absorption by the walls of the cavity. This is because  


\begin{enumerate}
    \item the chemical potential of the photons is zero
    \item photons obey Pauli exclusion principle
    \item photons are spin-1 particles
    \item the entropy of the photons is very large
\end{enumerate}

\item Consider $w = f\brak{z} = u\brak{x, y} + iv\brak{x, y}$ to be an analytic function in a domain $D$. Which one of the following options is NOT correct?

\begin{enumerate}
    \item $u\brak{x, y}$ satisfies Laplace equation in $D$
    \item $v\brak{x, y}$ satisfies Laplace equation in $D$
    \item $\int_{z_1}^{z_2} f\brak{z}dz$ is dependent on the choice of the contour between $z_1$ and $z_2$ in $D$
    \item $f\brak{z}$ can be Taylor expanded in $D$
\end{enumerate}

\item The value of $\int_{0}^{3}t^{3} \delta\brak{3t-6}dt$ is 
\item Which one of the following DOES NOT represent an exclusive OR operation for inputs $A$ and $B$?

\begin{enumerate}
    \item $\brak{A + \bar{B}}\bar{A}B$
    \item $\bar{A}B + \bar{B}A$
    \item $\brak{A + B}\brak{\bar{A} + \bar{B}}$
    \item $\brak{A + B}\bar{A}B$
\end{enumerate}

\item Consider a complex function $f\brak{z}=\frac{1}{z\brak{z+\frac{1}{z}}\cos\brak{z\pi}}$. Which one of the following statements is correct?

\begin{enumerate}
    \item $f\brak{z}$ has simple poles at $z=0$ and $z=-\frac{1}{2}$
    \item $f\brak{z}$ has a second order pole at $z=-\frac{1}{2}$
    \item $f\brak{z}$ has infinite number of second order poles
    \item $f\brak{z}$ has all simple poles
\end{enumerate}





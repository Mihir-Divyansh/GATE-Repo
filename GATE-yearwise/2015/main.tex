 \iffalse
    \title{Assignment}
    \author{EE24BTECH11035}
    \section{xe}
    \chapter{2015}
  \fi
  \item The velocity profile in turbulent flow through a pipe is approximated as
    \begin{equation*}
    \frac{u}{u_{\text{max}}} = \left(\frac{y}{R}\right)^{1/7}
    \end{equation*}
    where $u_{\text{max}}$ is the maximum velocity, $R$ is the radius, and $y$ is the distance measured normal to the pipe wall towards the centerline. If $u_{\text{av}}$ denotes the average velocity, the ratio $\frac{u_{\text{av}}}{u_{\text{max}}}$ is
    \begin{enumerate}
        \item $\frac{2}{15}$\\
        \item $\frac{1}{5}$\\
        \item $\frac{1}{3}$\\
        \item $\frac{49}{60}$
    \end{enumerate}

\item A steel sphere (density $\rho_s = 7900 \, \text{kg/m}^3$) of diameter $0.1 \, \text{m}$ is dropped from rest in water (density $\rho_w = 1000 \, \text{kg/m}^3$). The gravitational acceleration is $9.81 \, \text{m/s}^2$. Assuming that the drag coefficient is constant and equal to $1.33$, the terminal velocity attained by the sphere in $\text{m/s}$ is \underline{\hspace{2cm}}
    
\item An inclined venturimeter connected to an inverted manometer is shown in the figure. The cross-sectional areas at the inlet and the throat are $2 \times 10^{-3} \, \text{m}^2$ and $2 \times 10^{-4} \, \text{m}^2$, respectively. The densities of water and oil are $1000 \, \text{kg/m}^3$ and $800 \, \text{kg/m}^3$, respectively. The gravitational acceleration is $9.81 \, \text{m/s}^2$. If the discharge of water through the venturimeter is $5 \times 10^{-4} \, \text{m}^3/\text{s}$, neglecting viscous effects and assuming uniform velocities across the inlet and the throat, the manometer reading $h$, in meters, will be \underline{\hspace{2cm}}
\begin{figure}[!ht]
\centering
\resizebox{0.3\textwidth}{!}{%
\begin{circuitikz}
\tikzstyle{every node}=[font=\LARGE]
\draw  (5.25,15.75) rectangle (5.5,8.25);
\draw  (5.25,15.75) rectangle (7.75,15.5);
\draw  (7.5,15.75) rectangle (7.75,9.5);
\draw [short] (8,10.25) -- (6.75,9.25);
\draw [short] (8,10.25) -- (10,12.5);
\draw [short] (6.75,9.25) -- (5.75,9);
\draw [short] (5.75,9) -- (4.25,7.75);
\draw [short] (10,12.5) -- (11,13.75);
\draw [short] (10,12.5) -- (10.5,12);
\draw [short] (8,10.25) -- (9,9.25);
\draw [short] (5.75,9) -- (7.25,7.25);
\draw [short] (9,9.25) -- (11.25,11.25);
\draw [short] (11.25,11.25) -- (12.75,12);
\draw [short] (11.25,11.25) -- (10.5,12);
\draw [short] (9,9.25) -- (8,8.5);
\draw [short] (8,8.5) -- (7.25,7.25);
\draw [short] (7.25,7.25) -- (6,6.25);
\draw [dashed] (12,13) -- (4.75,6.5);
\draw [->, >=Stealth] (5.5,7) -- (6.25,7.75);
\draw [->, >=Stealth] (10,11.25) -- (11.25,12.25);
\draw [<->, >=Stealth] (8,12.75) -- (8,9.5);
\draw [->, >=Stealth] (8,16) -- (8,12.75);
\draw [short] (8,16) -- (5,16);
\draw [->, >=Stealth] (5,16) -- (5,14.5);
\draw [<->, >=Stealth] (5,14.5) -- (5,8.5);
\draw [short] (7.5,12.75) -- (6,12.75);
\draw [dashed] (9,12.75) -- (7.75,12.75);
\draw [<->, >=Stealth] (6,14.5) -- (6,12.75);
\draw [dashed] (5,14.5) -- (6.25,14.5);
\node [font=\Huge] at (5.5,8.5) {.};
\node [font=\LARGE] at (4,11.75) {water};
\node [font=\LARGE] at (9,11.75) {water};
\node [font=\LARGE] at (5,6.25) {water};
\node [font=\LARGE] at (6,13.75) {h};
\node [font=\LARGE] at (6.5,16.5) {oil};
\draw [dashed] (5.25,15.75) -- (7.25,15.75);
\draw [dashed] (5.25,15.75) -- (7.5,15.75);
\draw [dashed] (7.5,15.75) -- (7.75,12.75);
\draw [dashed] (7.75,15.5) -- (7.5,13);
\draw [dashed] (7.5,15.5) -- (5,15.75);
\draw [dashed] (7.75,15.5) -- (6.5,15.75);
\draw [dashed] (5.5,15.75) -- (5.5,14.75);
\draw [dashed] (5.25,15.5) -- (5.5,14.5);
\draw [dashed] (5.5,15.75) -- (5.25,14.75);
\end{circuitikz}
}%

\label{fig:my_label}
\end{figure}

\item A plane jet of water with volumetric flow rate $0.012 \, \text{m}^3/\text{s}$ and cross-sectional area $6 \times 10^{-4} \, \text{m}^2$ strikes a stationary plate inclined at angle $\theta$ and leaves as two streams, as shown in the figure. The ratio of the discharge through section 2 to that through section 3 is $3:1$. The velocities may be considered uniform across the cross-sections, and the effects of friction may be neglected. The density of water is $1000 \, \text{kg/m}^3$. Ignoring the effects of gravity, the magnitude of the normal force exerted on the plate, in N, is
\begin{figure}[!ht]
\centering
\resizebox{0.4\textwidth}{!}{%
\begin{circuitikz}
\tikzstyle{every node}=[font=\LARGE]
\draw [short] (12.25,14.5) .. controls (10.25,11.25) and (10.5,11.75) .. (7.25,11.5);
\draw [short] (7.25,11) .. controls (11.75,11.75) and (9,10) .. (8.5,9.25);
\draw [short] (13.25,14.25) -- (9,8.75);
\draw [ rotate around={-37:(11.75, 11.875)}] (12,14.25) rectangle (11.5,9.5);
\draw [short] (12,11.75) -- (14,11.75);
\draw [short] (12.5,12.5) .. controls (12.75,12.25) and (12.75,12.25) .. (12.75,11.75);
\draw [->, >=Stealth] (12.5,14) -- (13.25,15.25);
\draw [->, >=Stealth] (9,9.25) -- (8,8.25);
\draw [short] (9,9.75) -- (9.5,9.5);
\draw [short] (12,14) -- (12.75,13.5);
\draw [dashed] (7.75,12) -- (7.75,10.75);
\draw [->, >=Stealth] (6,11.25) -- (7.5,11.25);
\node [font=\LARGE] at (8,12.25) {1};
\node [font=\LARGE] at (11.5,14.5) {2};
\node [font=\LARGE] at (9.75,9) {3};
\node [font=\LARGE] at (12.5,11) {plate};
\node [font=\LARGE] at (13.25,12.5) {$\theta$};
\end{circuitikz}
}%

\label{fig:my_label}
\end{figure}
\item Arrange the following elements in order of increasing melting point:\\
(P) Gallium\\
(Q) Tungsten\\
(R) Aluminium\\
(S) Gold

\begin{enumerate}
    \item (A) $P < R < Q < S$
    \item (B) $S < P < R < Q$
    \item (C) $P < R < S < Q$
    \item (D) $R < S < Q < P$
\end{enumerate}
\item When the atoms in a solid are separated by their equilibrium distance,
\begin{enumerate}
    \item (A) the potential energy of the solid is lowest
    \item (B) the force of attraction between the atoms is maximum
    \item (C) the force of repulsion between the atoms is zero
    \item (D) the potential energy of the solid is zero
\end{enumerate}
\item To which of the following categories of materials does Teflon (PTFE) belong?
\begin{enumerate}
    \item (A) Thermosets
    \item (B) Thermoplastics
    \item (C) Elastomers
    \item (D) Block copolymers
\end{enumerate}

\item Which of the following statements is \textbf{TRUE} about the glass transition temperature ($T_g$)?
    \begin{enumerate}
        \item $T_g$ appears below the melting temperature in a perfectly crystalline material
        \item Upon heating through $T_g$, heat capacity remains constant but the thermal expansion coefficient changes
        \item Upon heating through $T_g$, heat capacity changes but the thermal expansion coefficient remains the same
        \item Upon heating through $T_g$, both the heat capacity and thermal expansion coefficient change
    \end{enumerate}
   
\item The slope of a graph of $\log_e(\text{conductivity})$ versus $1/T$ (where $T$ is the temperature) for an intrinsic semiconductor with energy gap $E_g$ is:
    \begin{enumerate}
        \item $E_g / 2k$
        \item $-E_g / 2k$
        \item $E_g / k$
        \item $-E_g / k$
    \end{enumerate}
   
\item    Which is \textbf{NOT} a ceramic forming process?
    \begin{enumerate}
        \item extrusion
        \item slip casting
        \item forging
        \item tape casting
    \end{enumerate}
   
\item    Which of the following is \textbf{NOT} a soft magnetic material?
    \begin{enumerate}
        \item Iron-silicon steel
        \item Nickel zinc ferrite
        \item Nickel iron alloy
        \item Alnico
    \end{enumerate}
    
 \item The eutectic reaction is \\
    \textbf{[Note: S = solid; L = liquid]}
    \begin{enumerate}
        \item $L \rightleftharpoons S_1 + S_2 + S_3$
        \item $L \rightleftharpoons S_1 + S_2$
        \item $L_1 + S_1 \rightleftharpoons L_2 + S_2$
        \item $L_1 + S_1 \rightleftharpoons S_2 + S_3$
    \end{enumerate}
  
  \item  Vacancies play an important role in:
    \begin{enumerate}
        \item deformation twinning
        \item self diffusion
        \item strain hardening
        \item cross-slip
    \end{enumerate}



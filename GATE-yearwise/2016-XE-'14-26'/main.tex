\iffalse
\author{EE24BTECH11050}
\chapter{2016}
\section{xe}
\fi
\item %14
Which of the following is a quasi-linear partial differential equation?
\begin{enumerate}
\item $ \frac{\partial ^2 u}{\partial t^2} + u^2 = 0$
\item $ \brak{\frac{\partial u}{\partial t}} ^2 + \frac{\partial u}{\partial x} =0$
\item $ \brak{\frac{\partial u}{\partial t}} ^2 - \brak{\frac{\partial u}{\partial x}} ^2 = 0$
\item $ \brak{\frac{\partial u}{\partial t}} ^4 - \brak{\frac{\partial u}{\partial x}} ^3 = 0$
\end{enumerate}
\item % 15
Let $P\brak{x}$ and $Q\brak{x}$ be the polynomials of degree 5, generated by Lagrange and Newton interpolation methods respectively, both passing through given six distinct points on the $xy-plane$. Which of the following is correct?
\begin{enumerate}
\item $P\brak{x}\equiv Q\brak{x}$
\item $P\brak{x} - Q\brak{x}$ is a polynomial of degree 1
\item $P\brak{x} - Q\brak{x}$ is a polynomial of degree 2
\item $P\brak{x} - Q\brak{x}$ is a polynomial of degree 3
\end{enumerate}
\item %16
The Laurent series of $ f\brak{x}=1/\brak{z^3 - z^4} $ with center at $z=0$ in the region $\abs{z}>1$ is 
\begin{enumerate}
\begin{multicols}{4}
\item $ \sum\limits_{n=0}^{\infty} z^{n-3} $
\item $ - \sum\limits_{n=0}^{\infty} \frac{1}{z^{n+4}} $
\item $ \sum\limits_{n=0}^{\infty} z^{n} $
\item $ \sum\limits_{n=0}^{\infty} \frac{1}{z^n} $
\end{multicols}
\end{enumerate}
\item %17
The value  of the surface integral $\iint \vec{F} . n ds$ over the sphere $\Gamma$ given by $x^2+y^2+z^2=1$ where $\vec{F}=4x\hat{i}-z\hat{k} $, and $n$ denotes the outward unit normal, is 
\begin{enumerate}
\begin{multicols}{4}
\item $\pi$
\item $2\pi$
\item $3\pi$
\item $4\pi$
\end{multicols}
\end{enumerate}
Q.8 - Q.11 carry two marks each
\item %18 
A diagnostic test for a certain disease is $90 \%$ accurate. That is, the probability of a person having (respectively, not having) the disease tested positive (respectively,  negative) is 0.9. Fifty percent of the population has the disease. What is the probability that a randomly chosen person has the disease given that the person tested negative?
\item %19
Let $M = \myvec{
1 & 1 \\ 0 & 1  
} $. Which of the following is correct?
\begin{enumerate}
\item Rank of M is 1 and M is not diagonalizable
\item Rank of M is 2 and M is diagonalizable
\item 1 is the only eigenvalue and M is not diagonalizable
\item 1 is the only eigenvalue and M is diagonalizable
\end{enumerate}
\item %20
Let $f\brak{x}=2x^3-3x^2+69, -5\leq x \leq 5$. Find the point at which $f$ attains the global maximum.
\item %21
Calculate $\int\limits_{c_{1}} \vec{F}.dr - \int\limits_{c_{2}}\vec{F}.dr$, where $c_1 : \vec{r}\brak{t,t^2} and c_1 : \vec{r}\brak{t,\sqrt{t}}$, t varying from 0 to 1 and $\vec{F}=xy\hat{j}$.
\end{enumerate}
\begin{center}
   B.Fluid Mechanics
\end{center}
Q.1-Q.9 carry one mark each.
\begin{enumerate}[start=1]
\item %22
In the parallel-plate configuration shown, steady-flow of an incompressible Newtonian fluid is established by moving the top plate with a constant speed, $U_0=1m/s$. If the force required on the top plate to support this motion is 0.5 per unit area (in $m_2$) of the plate then the viscosity of the fluid between the plates is \underline{\hspace{2cm}} $N-s/m^2$ 
\begin{figure}[!ht]
\centering
\resizebox{0.4\textwidth}{!}{%
\begin{circuitikz}
\tikzstyle{every node}=[font=\normalsize]
\draw [line width=1.2pt, short] (7,11) -- (14.25,11);
\draw [line width=1.2pt, short] (6.75,15) -- (14.5,15);
\draw [line width=1.2pt, dashed] (9.5,15) -- (9.5,11.25);
\draw [line width=1.2pt, dashed] (9.5,11) -- (13.5,15);
\draw [line width=0.8pt, ->, >=Stealth] (9.5,14.75) -- (13,14.75);
\draw [line width=0.8pt, ->, >=Stealth] (9.5,14) -- (12.25,14);
\draw [line width=0.8pt, ->, >=Stealth] (9.5,13) -- (11.25,13);
\draw [line width=0.8pt, ->, >=Stealth] (9.5,12.25) -- (10.5,12.25);
\draw [line width=0.8pt, ->, >=Stealth] (9.75,15) -- (13.5,15);
\draw [line width=0.7pt, <->, >=Stealth] (7.25,14.75) -- (7.25,11.25);
\node [font=\normalsize] at (8,12.75) {10 mm};
\node [font=\normalsize] at (10.25,15.5) {$U_0$};
\end{circuitikz}
}%
\label{fig:my_label}
\end{figure}
\item %23
For a newly designed vehicle by some students, volume of fuel consumed per unit distance travelled $(q_f in m^3/m)$ depends upon the vicosity $\brak{\mu}$ and density $\brak{\rho}$ of the fuel and, speed \brak{U} and size \brak{L} of the vehicle as \\
$q_f = c\frac{\rho U^2 L}{\mu ^3}$ \\
where C is a constant. The dimensions of the constant C are
\begin{enumerate}
\begin{multicols}{4}
\item $M^0L^0T^0$
\item $M^2L^{-1}T^{-1}$
\item $M^2L^{-5}T^{-1}$
\item $M^{-2}L^1T^1$
\end{multicols}
\end{enumerate}
\item %24
A semicircular gate of radius $1m$ is placed at the bottom of a water reservoir as shown in the figure below. The hydrostatic force per unit width of the cylindrical gate in y-direction is \underline{\hspace{2cm}} $kN$. The gravitational acceleration, $g=9.8m/s^2$ and density of the water = $1000kg/m^3$ . 
\begin{figure}[!ht]
\centering
\resizebox{0.4\textwidth}{!}{%
\begin{circuitikz}
\tikzstyle{every node}=[font=\small]
\draw [line width=1pt, short] (5.75,15) -- (5.75,8.25);
\draw [line width=1pt, short] (5.75,8.25) -- (11.25,8.25);
\draw [line width=1pt, short] (11.25,10.5) -- (11.25,15);
\draw [ line width=1pt ] (11.25,10.5) ellipse (0cm and 0.25cm);
\draw [line width=0.6pt, dashed] (5.75,14) -- (11.25,14);
\draw [line width=1.8pt, short] (11.25,10.25) .. controls (9.5,9.75) and (9.5,8.75) .. (11.25,8.25);
\draw [line width=0.5pt, ->, >=Stealth] (7.75,10.25) -- (7.75,11.75);
\draw [line width=0.5pt, ->, >=Stealth] (7.75,10.25) -- (9.25,10.25);
\draw [line width=0.8pt, short] (11.25,14) -- (12.25,14);
\draw [line width=0.8pt, short] (11.25,10.25) -- (12,10.25);
\draw [line width=0.8pt, <->, >=Stealth] (11.75,13.75) -- (11.75,10.5);
\node [font=\normalsize] at (12.5,12) {2cm};
\node [font=\Large] at (11.25,9.25) {+};
\node [font=\large] at (12.5,8.75) {gate};
\draw [line width=0.8pt, ->, >=Stealth] (11.25,9.25) -- (10.5,9.75);
\draw [line width=0.8pt, short] (8.25,14) -- (8,13.5);
\draw [line width=0.8pt, short] (8.25,14) -- (8.5,13.5);
\draw [line width=0.8pt, short] (8,13.5) -- (8.5,13.5);
\node [font=\small] at (9.25,9.75) {x};
\node [font=\small] at (7.25,11.5) {y};
\end{circuitikz}
}
\label{fig:my_label}
\end{figure}
\item %25
Velocity vector in $m/s$ for a 2-D flow is given in Cartesian coordinate \brak{x,y} as $ \bar{V} = \brak{\frac{x^2}{4}\hat{i}-\frac{xy}{2}\hat{j}} $. Symbols bear usual meaning. At a point in the flow, the x-component and y-component of the acceleration vector are given as $1m/s^2$ and $-0.5m/s^2$, respectively. The velocity magnitude at that points is \underline{\hspace{2cm}} $m/s$.
\item %26
If $\phi\brak{x,y}$ is velocity potential and $\psi\brak{x,y}$ is stream function for a 2-D, steady, incompressible and irrotational flow, which of the followings is correct?
\begin{enumerate}
\begin{multicols}{2}
\item $\brak{\frac{dy}{dx}}_{\phi=const}=-\frac{1}{\brak{\frac{dy}{dx}_{\psi=const}}}$
\item $\frac{\partial^2\psi}{\partial x^2}+\frac{\partial^2\psi}{\partial y^2}=0$
\end{multicols}
\begin{multicols}{2}
\item $\brak{\frac{dy}{dx}}_{\phi=const}=\frac{1}{\brak{\frac{dy}{dx}_{\psi=const}}}$
\item $\frac{\partial^2\phi}{\partial x^2}+\frac{\partial^2\phi}{\partial y^2}=0$
\end{multicols}
\end{enumerate}


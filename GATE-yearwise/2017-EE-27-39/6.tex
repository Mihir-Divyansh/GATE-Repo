\iffalse         
\title{2017-EE}
\author{EE24BTECH11020 -  Ellanti Rohith}
\section{ee}
\chapter{2017}
\fi

\item Consider the differential equation 

\begin{align*}
\brak{t^2 - 81} \frac{dy}{dt} + 5t \, y = \sin\brak{t} \quad \text{with } y\brak{1} = 2\pi.
\end{align*}


There exists a unique solution for this differential equation when $t$  belongs to the interval

\hfill{[GATE 2017]} \begin{enumerate}
    \begin{multicols}{2}
        \item $\brak{-2, 2}$
        \item $\brak{-10, 10}$
        \item $\brak{-10, 2}$
        \item $\brak{0, 10}$
    \end{multicols}
\end{enumerate}
\item Consider the line integral 
\begin{align*}
I = \int_C \brak{x^2 + iy} \, dz,
\end{align*}
where $ z = x + iy $. The line $ C $ is shown in the figure below.

\begin{center}
\begin{figure}[!ht]
\centering
\resizebox{0.3\textwidth}{!}{%
\begin{circuitikz}
\tikzstyle{every node}=[font=\large]
\draw [line width=0.5pt, ->, >=Stealth] (11.25,5.25) -- (15.5,5.25);
\draw [line width=0.5pt, ->, >=Stealth] (11.25,5.25) -- (11.25,9.5);
\draw [line width=0.5pt, dashed] (11.25,7.75) -- (13.75,7.75);
\draw [line width=0.5pt, dashed] (13.75,7.75) -- (13.75,5.5);
\draw [line width=0.5pt, dashed] (13.75,5.5) -- (13.75,5.25);
\draw [line width=1.2pt, dashed] (17.5,8.25) -- (17.5,8.5);
\draw [line width=1.8pt, ->, >=Stealth] (11.25,5.25) -- (12.5,6.5);
\draw [ line width=1.2pt](13.75,7.75) to[short] (12.5,6.5);
\node [font=\large] at (10.75,9) {$Y$};
\node [font=\large] at (15.25,4.75) {$X$};
\node [font=\large] at (11,4.85) {$(0,0)$};
\node [font=\large] at (14,8.25) {$(1,i)$};
\node [font=\large] at (12.25,7) {$c$};
\node [font=\large] at (11,8) {$i$};
\node [font=\large] at (13.75,4.75) {$1$};
\end{circuitikz}
}%

\end{figure}
\end{center}

The value of $ I $ is

\hfill{[GATE 2017]} \begin{enumerate}
    \begin{multicols}{2}
        \item $\dfrac{1}{2} i$\\
        \item  $\dfrac{2}{3} i$
        \item $\dfrac{3}{4} i$\\
        \item $\dfrac{4}{5} i$
    \end{multicols}
\end{enumerate}
\item Two passive two-port networks are connected in cascade as shown in the figure. A voltage source is connected at port 1.

\begin{center}
\begin{figure}[!ht]
\centering
\resizebox{0.7\textwidth}{!}{%
\begin{circuitikz}
\tikzstyle{every node}=[font=\large]
\draw [ line width=1.3pt](6.25,11.75) to[sinusoidal voltage source, sources/symbol/rotate=auto] (6.25,8.75);
\draw [ line width=1.3pt](6.25,8.75) to[short] (7.75,8.75);
\draw [ line width=1.3pt](6.25,11.75) to[short] (7.75,11.75);
\draw [ line width=1.3pt](7.75,8.5) to[short] (7.75,12.5);
\draw [ line width=1.3pt](7.75,8.25) to[short] (7.75,9);
\draw [ line width=1.3pt ] (7.75,12.5) rectangle (13.75,8.25);
\draw [ line width=1.3pt ] (13.75,12) rectangle (17.5,8.75);
\draw [ line width=1.3pt ] (17.5,12.5) rectangle (24,8.5);
\draw [line width=1.3pt, short] (24,12) -- (25.5,12);
\draw [line width=1.3pt, short] (24,8.75) -- (25.5,8.75);
\draw [ line width=1.3pt](14,12.75) to[short] (15,12.75);
\draw [ line width=1.3pt](15,12.75) to[short] (14.5,13);
\draw [ line width=1.3pt](15,12.75) to[short] (14.5,12.5);
\node [font=\large] at (16,12.75) {$I_2$};
\draw [ line width=1.3pt](24,12.75) to[short] (25,12.75);
\draw [ line width=1.3pt](25,12.75) to[short] (24.5,12.5);
\draw [ line width=1.3pt](25,12.75) to[short] (24.5,13);
\node [font=\large] at(26,12.75)  {$I_3$};

\draw [ line width=1.3pt](5.25,12.5) to[short] (6.25,12.5);
\draw [ line width=1.3pt](6.25,12.5) to[short] (6,12.25);
\draw [ line width=1.3pt](6.25,12.5) to[short] (6,12.75);
\node [font=\large] at (7,12.5) {$I_1$};

\node [font=\large] at (10.75,10.25) {Two-Port Network 1};
\node [font=\large] at (21,10.25) {Two-Port Network 2};

\node [font=\large] at(25,10.5)  {$V_2$};
\node [font=\large] at (5.5,10.5) {$V_1$};
\node [font=\large] at (15.5,10.5) {$V_3$};

\node [font=\large] at (7,8) {Port 1};
\node [font=\large] at (15.5,8) {Port 2};
\node [font=\large] at (25,8) {Port 3};
\node [font=\large] at (5.5,11.2) {+};
\node [font=\large] at (15.5,11.5) {+};
\node [font=\large] at (24.75,11.5) {+};
\node [font=\large] at (24.75,9.5) {-};
\node [font=\large] at (15.5,9.5) {-};
\node [font=\large] at (5.5,9.7) {-};
\end{circuitikz}
}%


\end{figure}
\end{center}

Given
\begin{align*}
    V_1 &= A_1 V_2 + B_1 I_2\\ I_1 &= C_1 V_2 + D_1 I_2\\
    V_2 &= A_2 V_3 + B_2 I_3 \\ I_2 &= C_2 V_3 + D_2 I_3
\end{align*}
    



where  $A_1, B_1, C_1, D_1, A_2, B_2, C_2$ and $ D_2$ are the generalized circuit constants. If the Thevenin equivalent circuit at port 3 consists of a voltage source  $V_T$ and an impedance $ Z_T$, connected in series, then

\hfill{[GATE 2017]} \begin{enumerate}
    \begin{multicols}{2}
        \item  $V_T = \dfrac{V_1}{A_1 + A_2}, \quad Z_T = \dfrac{A_1 B_2 + B_1 D_2}{A_1 A_2}$\\
        \item  $V_T = \dfrac{V_1}{A_1 A_2 + B_1 C_2}, \quad Z_T = \dfrac{A_1 B_2 + B_1 D_2}{A_1 A_2 + B_1 C_2} $
        \item$ V_T = \dfrac{V_1}{A_1 + A_2}, \quad Z_T = \dfrac{A_1 A_2 + B_1 C_2}{A_1 A_2 + B_1 D_2}$\\
        \item $ V_T = \dfrac{V_1}{A_1 A_2 + B_1 C_2}, \quad Z_T = \dfrac{A_1 A_2 + B_1 C_2}{A_1 A_2 + B_1 C_2}$
    \end{multicols}
\end{enumerate}

\item Let a causal LTI system be characterized by the following differential equation, with initial rest condition
\begin{align*}
   \frac{d^2 y}{dt^2} + 7 \frac{dy}{dt} + 10 y\brak{t} = 4x\brak{t} + 5 \frac{dx\brak{t}}{dt} 
\end{align*}

where $x\brak{t}$ and  $y\brak{t}$ are the input and output respectively. The impulse response of the system is $ u\brak{t}$ is the unit step function

\hfill{[GATE 2017]} \begin{enumerate}
    \begin{multicols}{2}
        \item  $2e^{-2t} u\brak{t} - 7e^{-5t} u\brak{t} $
        \item  $-2e^{-2t} u\brak{t} + 7e^{-5t} u\brak{t} $
        \item $ 7e^{-2t} u\brak{t} - 2e^{-5t} u\brak{t} $
        \item $-7e^{-2t} u\brak{t} + 2e^{-5t} u\brak{t} $
    \end{multicols}
\end{enumerate}
\item Let the signal
\begin{align*}
  x\brak{t} = \sum_{k=-\infty}^{+\infty} \brak{-1}^k \delta \brak{ t - \frac{k}{2000} }
\end{align*}

be passed through an LTI system with frequency response $H\brak{\omega}$, as given in the figure below.
\begin{centering}
    \resizebox{0.3\textwidth}{!}{%
\begin{circuitikz}
\tikzstyle{every node}=[font=\large]
\draw [line width=1.3pt, short] (6.75,10.25) -- (13.25,10.25);
\draw [line width=1.3pt, short] (10,10.25) -- (10,14);
\draw [line width=1.3pt, short] (7.5,12.5) .. controls (9.75,12.5) and (10,12.5) .. (12.5,12.5);
\draw [line width=1.3pt, short] (12.5,12.5) -- (12.5,10.25);
\draw [line width=1.3pt, short] (7.5,12.5) -- (7.5,10.25);
\node [font=\large] at (7.75,9.75) {-5000$\pi$};
\node [font=\large] at (12.75,9.75) {5000$\pi$};
\node [font=\large] at (13.5,10.5) {$\omega$};
\node [font=\large] at (10.25,12.75) {1};
\node [font=\large] at (10,14.2) {$H\brak{\omega}$};
\end{circuitikz}
}%

\end{centering}


The Fourier series representation of the output is given as\\

\hfill{[GATE 2017]} \begin{enumerate}
 
        \item $ 4000 + 4000 \cos\brak{2000 \pi t} + 4000 \cos\brak{4000 \pi t} $   \item$ 2000 + 2000 \cos\brak{2000 \pi t} + 2000 \cos\brak{4000 \pi t}$
        \item $ 2000 \cos\brak{2000 \pi t} $
       \item $ 4000 \cos\brak{2000 \pi t}$\\
    
\end{enumerate}

\item In the system whose signal flow graph is shown in the figure,$ U_1\brak{s}$ and  $U_2\brak{s}$ are inputs. The transfer function $\dfrac{Y\brak{s}}{U_1\brak{s}}$  is
\hfill{[GATE 2017]} 

\begin{centering}
\resizebox{0.4\textwidth}{!}{%
\begin{circuitikz}
\tikzstyle{every node}=[font=\Large]
\draw [ line width=2pt](1.5,9) to[short] (3.75,9);
\draw [ line width=2pt](2.5,8.75) to[short] (3,9);
\draw [ line width=2pt](3,9) to[short] (2.5,9.25);
\node at (1.5,9) [circ,scale=2] {};
\draw [ line width=2pt](3.75,9) to[short] (6,9);
\draw [ line width=2pt](4.75,8.75) to[short] (5.25,9);
\draw [ line width=2pt](5.25,9) to[short] (4.75,9.25);
\node at (3.75,9) [circ,scale=2] {};
\draw [ line width=2pt](6,9) to[short] (8.25,9);
\draw [ line width=2pt](7,8.75) to[short] (7.5,9);
\draw [ line width=2pt](7.5,9) to[short] (7,9.25);
\draw [ line width=2pt](8,9) to[short] (10.25,9);
\node at (6.25,9) [circ,scale=2] {};
\node at (8.75,9) [circ,scale=2] {};
\draw [ line width=2pt](10.25,9) to[short] (12.5,9);
\draw [ line width=2pt](11.25,8.75) to[short] (11.75,9);
\draw [ line width=2pt](11.75,9) to[short] (11.25,9.25);
\node at (10.25,9) [circ,scale=2] {};
\draw [ line width=2pt](12.5,9) to[short] (14.75,9);
\draw [ line width=2pt](13.5,8.75) to[short] (14,9);
\draw [ line width=2pt](14,9) to[short] (13.5,9.25);
\node at (12.5,9) [circ,scale=2] {};
\draw [ line width=2pt](14.75,9) to[short] (17,9);
\draw [ line width=2pt](15.75,8.75) to[short] (16.25,9);
\draw [ line width=2pt](16.25,9) to[short] (15.75,9.25);
\node at (15,9) [circ,scale=2] {};
\node at (17,9) [circ,scale=2] {};
\draw [ line width=2pt](9,8.75) to[short] (9.5,9);
\draw [ line width=2pt](9.5,9) to[short] (9,9.25);
\draw [line width=2pt, short] (3.75,9) .. controls (5.75,2.75) and (13.5,4) .. (15,9);

\draw [ line width=2pt](3.75,9) to[short] (8.75,9);
\draw [line width=2pt, short] (3.75,9) .. controls (4,12.25) and (8.25,12.75) .. (8.75,9);
\draw [ line width=2pt](9,4.75) to[short] (9.5,5);
\draw [ line width=2pt](9.5,4.5) to[short] (9,4.75);
\draw [ line width=2pt](6.25,11.58) to[short] (6.75,12);
\draw [ line width=2pt](6.75,11.29) to[short] (6.25,11.58);
\draw [ line width=2pt](12.5,11) to[short] (12.5,9);
\draw [ line width=2pt](12.5,10) to[short] (12.75,10.5);
\draw [ line width=2pt](12.5,10) to[short] (12.25,10.5);
\node [font=\Large] at (1.5,9.5) {$U_1$};
\node [font=\Large] at (3,10) {1};
\node [font=\Large] at (5,8) {1/L};
\node [font=\Large] at (6.5,12.5) {-R};
\node [font=\Large] at (7,8) {1/S};
\node [font=\Large] at (9,8.5) {$k_1$};
\node [font=\Large] at (11.5,8) {1/J};
\node [font=\Large] at (12.5,11.5) {$U_2$};
\node [font=\Large] at (13,11) {-1};
\node [font=\Large] at (13.5,8) {1/S};
\node [font=\Large] at (16,10) {1};
\node [font=\Large] at (17,9.5) {Y};
\end{circuitikz}
}%

\end{centering}\begin{enumerate}
    \begin{multicols}{2}
        \item $\dfrac{k_1}{JL s^2 + JR s + k_1 k_2}$
        \item  $\dfrac{k_1}{JL s^2 - JR s - k_1 k_2}$
        \item  $\dfrac{k_1 - U_2 \brak{R + sL}}{JL s^2 + \brak{JR - U_2 L} s + k_1 k_2 - U_2 R}$
        \item  $\dfrac{k_1 - U_2 \brak{sL - R}}{JL s^2 - \brak{JR + U_2 L} s - k_1 k_2 + U_2 R}$
    \end{multicols}
\end{enumerate}
\item The transfer function of the system is given by:

\begin{align*}
x\brak{t} &= \myvec{ 1 & 2 \\ 2 & 0} + \myvec {1 \\ 2 } u\brak{t}  \\
y\brak{t} &= \myvec{ 1 & 0} x\brak{t}
\end{align*}
The options for the transfer function are:\hfill{[GATE 2017]} 
\begin{multicols}{2}
\begin{enumerate}
    \item $\dfrac{\brak{s + 2}}{s^2 - 2s - 2}$\\
    \item  $\dfrac{\brak{s - 2}}{s^2 + s - 4}$
    \item  $\dfrac{\brak{s + 4}}{s^2 + s - 4}$\\
    \item  $\dfrac{\brak{s + 4}}{s^2 - s - 4}$
\end{enumerate}     
\end{multicols}

\item The load shown in the figure is supplied by a 400 $V$ (line-to-line), 3-phase source RYB sequence. The load is balanced and inductive, drawing 3464 $VA$. When the switch $S$ is in position $ N $, the three wattmeters $W_1 $,  $W_2$, and $ W_3 $ read 577.35 $W$ each. If the switch is moved to position $Y$, the readings of the wattmeters in watts will be:
\hfill{[GATE 2017]} 

\begin{figure}[!ht]
\centering
\resizebox{0.6\textwidth}{!}{%
\begin{circuitikz}
\tikzstyle{every node}=[font=\LARGE]
\draw [line width=1.1pt, short] (16.5,-2.25) .. controls (18,-3.25) and (15.5,-3.25) .. (17,-2.25);
\draw [line width=0.9pt, short] (17,-2.25) .. controls (21.5,-2.25) and (21.75,-2.25) .. (26.25,-2.25);
\draw [line width=1.3pt, short] (16.5,-2.25) -- (11.25,-2.25);
\draw [line width=1.2pt, short] (21.75,-2.25) -- (23.75,-2.25);
\draw [line width=1.1pt, short] (16.5,2.75) .. controls (18,1.75) and (15.5,1.75) .. (17,2.75);
\draw [line width=0.9pt, short] (17,2.75) .. controls (21.5,2.75) and (21.75,2.75) .. (26.25,2.75);
\draw [line width=1.3pt, short] (16.5,2.75) -- (11.25,2.75);
\draw [line width=1.2pt, short] (21.75,2.75) -- (23.75,2.75);
\draw [line width=1.1pt, short] (16.5,-7.25) .. controls (18,-8.25) and (15.5,-8.25) .. (17,-7.25);
\draw [line width=0.9pt, short] (17,-7.25) .. controls (21.5,-7.25) and (21.75,-7.25) .. (26.25,-7.25);
\draw [line width=1.3pt, short] (16.5,-7.25) -- (11.25,-7.25);
\draw [line width=1.2pt, short] (21.75,-7.25) -- (23.75,-7.25);
\draw [ line width=0.5pt ] (7.5,5.25) rectangle (11.25,-12.25);
\draw [ line width=0.5pt ] (26.25,5.25) rectangle (30,-12.25);
\draw [ line width=0.5pt , dashed] (15,4) rectangle  (19.75,0.25);
\draw [ line width=1.1pt](15.5,2.75) to[short] (15.5,1.5);
\draw [line width=1pt](15.5,1.5) to[L ] (18.5,1.5);
\draw [ line width=1pt](18.5,1.5) to[R] (18.5,-0.25);
\draw [ line width=1pt](18.5,-0.25) to[short] (22,-0.25);
\draw [line width=1pt, short] (22.25,-2) .. controls (23,-1.75) and (23,-2.5) .. (22.25,-2.5);
\draw [ line width=1pt](22.25,-2) to[short] (22.25,-0.25);
\draw [ line width=1pt](22.25,-0.25) to[short] (21.75,-0.25);
\draw [ line width=1pt](22.25,-2.5) to[short] (22.25,-7);
\draw [line width=1pt, short] (22.25,-7) .. controls (23,-6.75) and (23,-7.5) .. (22.25,-7.5);
\draw [line width=1.1pt, short] (16.5,-2.25) .. controls (18,-3.25) and (15.5,-3.25) .. (17,-2.25);
\draw [line width=0.9pt, short] (17,-2.25) .. controls (21.5,-2.25) and (21.75,-2.25) .. (26.25,-2.25);
\draw [line width=1.3pt, short] (16.5,-2.25) -- (11.25,-2.25);
\draw [ line width=0.5pt , dashed] (15,-1) rectangle  (19.75,-4.75);
\draw [ line width=1.1pt](15.5,-2.25) to[short] (15.5,-3.5);
\draw [line width=1pt](15.5,-3.5) to[L ] (18.5,-3.5);
\draw [ line width=1pt](18.5,-3.5) to[R] (18.5,-5.25);
\draw [line width=1.1pt, short] (16.5,-7.25) .. controls (18,-8.25) and (15.5,-8.25) .. (17,-7.25);
\draw [line width=0.9pt, short] (17,-7.25) .. controls (21.5,-7.25) and (21.75,-7.25) .. (26.25,-7.25);
\draw [line width=1.3pt, short] (16.5,-7.25) -- (11.25,-7.25);
\draw [ line width=0.5pt , dashed] (15,-6) rectangle  (19.75,-9.75);
\draw [ line width=1.1pt](15.5,-7.25) to[short] (15.5,-8.5);
\draw [line width=1pt](15.5,-8.5) to[L ] (18.5,-8.5);
\draw [ line width=1pt](18.5,-8.5) to[R] (18.5,-10.25);
\draw [line width=1pt, short] (18.5,-5.25) -- (22.25,-5.25);
\draw [line width=1pt, short] (18.5,-10.25) -- (22.25,-10.25);
\draw [line width=1pt, short] (22.25,-7.5) -- (22.25,-10.25);
\draw [ line width=1pt](11.25,-11) to[short] (24.75,-11);
\draw [ line width=1pt](24.75,-9.75) to[short] (24.75,-11);
\draw [ line width=1pt](24.75,-9.75) to[short, -o] (23.75,-9.75) ;
\draw [ line width=1pt](22.25,-9.25) to[short, -o] (23,-9.25) ;
\draw [ line width=1pt](23,-9.25) to[short, -o] (23.75,-8.5) ;
\draw [ line width=1pt](23.75,-8.5) to[short] (25,-8.5);
\draw [ line width=1pt](25,-6.75) to[short] (25,-2.25);
\draw [line width=1pt, short] (25,-7) .. controls (25.75,-6.75) and (25.75,-7.5) .. (25,-7.5);
\draw [ line width=1pt](25,-8.5) to[short] (25,-7.75);
\draw [ line width=1pt](25,-7.5) to[short] (25,-8);
\draw [ line width=1pt](25,-7) to[short] (25,-5.5);
\node [font=\LARGE, rotate around={90:(0,0)}] at (9,-5) {400V, 3-Phase Source};
\node [font=\LARGE, rotate around={90:(0,0)}] at (27.75,-3.5) {Load};
\node [font=\LARGE] at (10.5,2.75) {R};
\node [font=\LARGE] at (10.5,-2) {Y};
\node [font=\LARGE] at (11,-6.5) {B};
\node [font=\LARGE] at (10.75,-10.75) {N};
\node [font=\LARGE] at (14,4.25) {$W_1$};
\node [font=\LARGE] at (13.75,-0.75) {$W_2$};
\node [font=\LARGE] at (13.75,-5.5) {$W_3$};
\node [font=\LARGE] at (22.75,-9.75) {$S$};
\node [font=\LARGE] at (24,-8) {$Y$};
\node [font=\LARGE] at (23.75,-9.5) {$N$};
\draw [ line width=1pt , dashed] (22.5,-8) rectangle  (24.25,-10.25);
\node at (15.5,-7.25) [circ] {};
\node at (22.25,-5.25) [circ] {};
\node at (22.25,-9.25) [circ] {};
\node at (25,-2.25) [circ] {};
\node at (15.5,-2.25) [circ] {};
\node at (15.5,2.75) [circ] {};
\end{circuitikz}
}%

\end{figure}
\item The approximate transfer characteristic for the circuit shown below with an ideal operational amplifier and diode is as follows:

\begin{figure}[!ht]
\centering
\resizebox{0.35\textwidth}{!}{%
\begin{circuitikz}
\draw [ line width=0.9pt](7.25,5.25) node[op amp,scale=1, yscale=-1 ] (opamp2) {};
\draw [ line width=0.9pt](opamp2.+) to[short] (5.75,5.75);
\draw [ line width=0.9pt] (opamp2.-) to[short] (5.75,4.75);
\draw [ line width=0.9pt](8.45,5.25) to[short](8.75,5.25);
\draw [ line width=0.9pt](8.75,5.25) to[D] (8.75,3);
\draw [ line width=0.9pt](5.75,4.75) to[short] (5.75,3.5);
\draw [ line width=0.9pt](5.75,3) to[short] (10,3);
\draw [ line width=0.9pt](5.75,3) to[short] (5.75,4);
\draw [ line width=0.9pt](9.5,3) to[R] (9.5,1.25);
\draw [line width=0.9pt](9.5,1.5) to (9.5,1.25) node[ground]{};
\draw [ line width=0.9pt](5.75,5.75) to[short] (5.2,5.75);
\node [font=\normalsize] at (4.85,5.75) {$V_{in}$};
\draw [ line width=0.9pt](7,4.25) to[short] (7,4.5);
\draw [ line width=0.9pt](7,5.85) to[short] (7,6.25);
\draw [ line width=0.9pt](7,4.25) to[short] (7,4.67);
\node [font=\normalsize] at (7,6.5) {$V_{SS}$};
\node [font=\normalsize] at (7,4.1) {$V_{SS}$};
\node [font=\normalsize] at (9.5,4.25) {$D$};
\node [font=\normalsize] at (10.5,3) {$V_{o}$};
\node [font=\normalsize] at (10,2.25) {$R$};
\end{circuitikz}
}%

\end{figure}
\begin{multicols}{2}
     \begin{enumerate}
\item
\resizebox{0.2\textwidth}{!}{%
\begin{circuitikz}
\tikzstyle{every node}=[font=\normalsize]
\draw [ line width=0.9pt](1.25,9) to[short] (3.75,9);
\draw [ line width=0.9pt](3.75,9) to[short] (5,9);
\draw [ line width=0.9pt](3,9) to[short] (3,10.75);
\draw [ line width=0.9pt](3,9) to[short] (4.5,10.5);
\node [font=\normalsize] at (2.5,10.75) {$V_o$};
\node [font=\normalsize] at (5.25,8.75) {$V_{in}$};

\end{circuitikz}
}%
\item
\resizebox{0.2\textwidth}{!}{%
\begin{circuitikz}
\tikzstyle{every node}=[font=\normalsize]
\draw [ line width=0.8pt](1.25,9) to[short] (5,9);
\draw [ line width=0.8pt](3,9) to[short] (3,10.75);
\draw [ line width=0.8pt](3,9) to[short] (1.5,10.5);
\node [font=\normalsize] at (3,11) {$V_o$};
\node [font=\normalsize] at (5.25,8.75) {$V_{in}$};
\end{circuitikz}
}%
\item 
\resizebox{0.2\textwidth}{!}{%
\begin{circuitikz}
\tikzstyle{every node}=[font=\normalsize]
\draw [ line width=0.8pt](1.25,9) to[short] (5,9);
\draw [ line width=0.8pt](3,9) to[short] (3,10.75);
\draw [ line width=0.8pt](3,9) to[short] (1.5,10.5);
\node [font=\normalsize] at (3,11) {$V_o$};
\node [font=\normalsize] at (5.25,8.75) {$V_{in}$};
\draw [ line width=0.8pt](3,9) to[short] (4.5,10.5);
\end{circuitikz}
}%
\item 
\resizebox{0.2\textwidth}{!}{%
\begin{circuitikz}
\tikzstyle{every node}=[font=\normalsize]
\draw [ line width=0.8pt](1.25,9) to[short] (5,9);
\draw [ line width=0.8pt](3,9) to[short] (3,10.75);
\node [font=\normalsize] at (3,11) {$V_o$};
\node [font=\normalsize] at (5.25,8.75) {$V_{in}$};
\draw [ line width=0.8pt](1.75,10.25) to[short] (4.25,10.25);
\end{circuitikz}
}%
\end{enumerate}
\end{multicols}

   

    

\item The output expression for the Karnaugh map shown below is
\begin{align*}
  \begin{array}{|c|c|c|c|c|}\hline
    AB \backslash CD & 00 & 01 & 11 & 10 \\
    \hline
    00 & 0 & 0 & 0 & 0 \\\hline
    01 & 1 & 0 & 1 & 1 \\ \hline
    11 & 1 & 0 & 1 & 1 \\ \hline
    10 & 0 & 0 & 0 & 0 \\ \hline
\end{array}  
\end{align*}

\hfill{[GATE 2017]}
\begin{multicols}{4}
\begin{enumerate}
    \item $ \overline{B} \overline{D} + BCD $
    \item $ \overline{B} D + AB $
    \item $ \overline{B} D + ABC $
    \item $ B \overline{D} + ABC $
\end{enumerate}
\end{multicols}
\item A load is supplied by a 230 $V$, 50 $Hz$ source. The active power $ P $ and the reactive power $ Q $ consumed by the load are such that $1 \text{ kW} \leq P \leq 2 \text{ kW}$ and $ 1 \text{ kVAR} \leq Q \leq 2 \text{ kVAR}$. A capacitor connected across the load for power factor correction generates 1 $kVAR$ reactive power. The worst case power factor after power factor correction is
\hfill{[GATE 2017]} 
\begin{multicols}{4}
 \begin{enumerate}
    \item 0.447 lag
    \item 0.707 lag
    \item 0.894 lag
    \item 1
\end{enumerate}
\end{multicols}

\item The logical gate implemented using the circuit shown below where $ V_1$ and $ V_2 $ are inputs with (0 $V$ as digital 0 and 5 $V$ as digital 1) and $ V_{\text{OUT}} $ is the output, is \\ \newpage
\begin{figure}[!ht]
\centering
\begin{circuitikz}[scale=0.85, transform shape] 
\tikzstyle{every node}=[font=\normalsize]
\draw [ line width=0.9pt](6.25,10.5) to[R] (6.25,8.5);
\draw [ line width=0.9pt](6.25,8.5) to[short] (11.25,8.5);
\draw [line width=0.9pt](10.5,6) to[Tnpn, transistors/scale=1.19] (10.5,8.5);
\draw [ line width=0.9pt](9.75,7.25) to[R] (8,7.25);
\draw [line width=0.9pt](10.5,6.25) to (10.5,6) node[ground]{};
\draw [line width=0.9pt](6.25,6.25) to[Tnpn, transistors/scale=1.19] (6.25,8.75);
\draw [ line width=0.9pt](5.5,7.5) to[R] (3.75,7.5);
\draw [line width=0.9pt](6.25,6.75) to (6.25,6.5) node[ground]{};
\draw [ line width=0.9pt](5.75,10.5) to[short] (6.75,10.5);
\node [font=\normalsize] at (8.85,7.75) {1k$\Omega$};
\node [font=\normalsize] at (7,9.75) {1K$\Omega$};
\node [font=\normalsize] at (4.5,8) {1k$\Omega$};
\node [font=\normalsize] at (6.5,10.75) {5 $V$};
\node [font=\normalsize] at (3.5,7.75) {$V_1$};
\node [font=\normalsize] at (10.5,8.75) {$V_{o}$};
\node [font=\normalsize] at (7.75,7.5) {$V_2$};
\node [font=\normalsize] at (6.25,7.5) {$Q_1$};
\node [font=\normalsize] at (10.5,7.25) {$Q_2$};
\end{circuitikz}

\end{figure}

\hfill{[GATE 2017]}
\item The input voltage $V_{DC}$ of the buck-boost converter shown below varies from 32 $V$ to 72 $V$. Assume that all components are ideal, inductor current is continuous, and output voltage is ripple-free. The range of duty ratio $D$ of the converter for which the magnitude of the steady-state output voltage remains constant at 48 $V$ is
\hfill{[GATE 2017]}
\begin{figure}[!ht]
\centering
\begin{circuitikz}[scale=0.7]

\draw [ line width=0.8pt](4.5,12) to[american voltage source] (4.5,8);
\draw [ line width=0.9pt](4.5,8) to[short] (10.75,8);
\draw [line width=0.9pt](7,12) to[L ] (7,8);
\draw [ line width=0.9pt](4.5,12) to[short] (5.5,12);
\draw [line width=0.9pt, ->, >=Stealth] (5.5,12) -- (6,12.75);
\draw [ line width=0.9pt](6,12) to[short] (7,12);
\draw [ line width=0.9pt](10.5,12) to[D] (6.25,12);
\draw [line width=0.9pt](10,12) to[C] (10,8);
\draw [ line width=0.9pt](12.5,12) to[R] (12.5,8);
\draw [ line width=0.9pt](10.5,12) to[short] (12.5,12);
\draw [ line width=0.9pt](12.5,8) to[short] (10.5,8);
\node at (3.2,10) {$V_{DC}$};
\node at (5.3,12.5) {$S_1$};
\node at (6.3,10) {$L$};
\node at (8.85,10) {$C$};
\node at (11.6,10) {$R$};
\node at (13.2,10) {$V_{o}$};
\node at (13.2,9) {$+$};
\node at (13.2,10.75) {$-$};

\end{circuitikz}


\end{figure}
\begin{multicols}{4}
\begin{enumerate}
    \item $ \dfrac{2}{5} \leq D \leq \dfrac{3}{5} $
    \item $ \dfrac{2}{3} \leq D \leq \dfrac{3}{4} $
    \item $ 0 \leq D \leq 1 $
    \item $ \dfrac{1}{3} \leq D \leq \dfrac{2}{3} $
\end{enumerate}

\end{multicols}





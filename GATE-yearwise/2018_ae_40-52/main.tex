\iffalse
\chapter{2018}
\author{EE24BTECH11005}
\section{ae}
\fi
        \item A clamped-clamped beam, subjected to a point load $P$ at the midspan, is shown in the figure below. The magnitude of the moment reaction at the two fixed ends of the beam is
            \begin{figure}[H]
                \centering
                \begin{tikzpicture}
                    \tikzstyle{every node}=[font=\large]
                    \draw  (5,14.75) -- (12.5,14.75);
                    \draw  (5,13.5) -- (12.5,13.5);
                    \draw  (5,15.25) -- (5,13);
                    \draw  (4.75,14.75) -- (5,15);
                    \draw  (4.75,14.5) -- (5,14.75);
                    \draw  (4.75,14.25) -- (5,14.5);
                    \draw  (4.75,14) -- (5,14.25);
                    \draw  (4.75,13.75) -- (5,14);
                    \draw  (4.75,13.5) -- (5,13.75);
                    \draw  (4.75,13.25) -- (5,13.5);
                    \node [font=\large] at (5.5,12.75) {};
                    \draw  (4.75,13) -- (5,13.25);
                    \draw  (12.5,15.25) -- (12.5,13);
                    \draw  (12.5,15) -- (12.75,14.75);
                    \node [font=\large] at (13.25,14.25) {};
                    \node [font=\large] at (13.25,14.25) {};
                    \node [font=\large] at (13.25,14.25) {};
                    \draw  (12.5,14.75) -- (12.75,14.5);
                    \draw  (12.5,14.5) -- (12.75,14.25);
                    \draw  (12.5,14.25) -- (12.75,14);
                    \draw  (12.5,14) -- (12.75,13.75);
                    \draw  (12.5,13.75) -- (12.75,13.5);
                    \draw  (12.5,13.5) -- (12.75,13.25);
                    \draw  (12.5,13.25) -- (12.75,13);
                    \draw [<->] (5,13.25) -- (12.5,13.25);
                    \node [font=\large] at (8.75,13) {L};
                    \draw [->] (8.75,16.5) -- (8.75,14.75);
                    \node [font=\large] at (9,15.5) {P};
                \end{tikzpicture}
            \end{figure}
            \begin{enumerate}
                    \begin{multicols}{2}
                    \item $\frac{PL}{2}$
                        \columnbreak
                    \item $\frac{PL}{4}$
                    \end{multicols}
                    \begin{multicols}{2}
                    \item $\frac{PL}{8}$
                        \columnbreak
                    \item $\frac{PL}{16}$
                    \end{multicols}
            \end{enumerate}
        \item Which of the following statement(s) is/are true about the state of a body in plane strain condition?\\
            $P:$ All the points in the body undergo displacements in one plane only, for example the $x-y$ plane, leading to $\varepsilon_{zz} = \gamma_{xz}=\gamma_{yz}=0$.\\
            $Q:$ All the components of stress perpendicular to the plane of deformation, for example the $x-y$ plane, of the body are equal to zero, i.e. $\sigma_{zz}= \tau{xz}=\tau{yz}=0$\\ 
            $R:$ Except the normal component, all the other components of stress perpendicular to the plane of deformation of the body, for example the $x-y$ plane, are equal to zero, i.e. $\sigma_{zz} \ne 0, \tau_{sz}=\tau_{yz}=0$. 
            \begin{enumerate}
                    \begin{multicols}{2}
                    \item $P$ only
                        \columnbreak
                    \item $Q$ only
                    \end{multicols}
                    \begin{multicols}{2}
                    \item $P$ and $Q$
                        \columnbreak
                    \item $P$ and $R$ 
                    \end{multicols}
            \end{enumerate}
        \item An aircraft with a turbojet engine flies at a velocity of $100 m/s$. If the jet exhaust velocity is $300 m/s$, the propulsive efficiency of the engine, assuming a negligible fuel-air ratio, is
            \begin{enumerate}
                    \begin{multicols}{2}
                    \item  $0.33$
                        \columnbreak
                    \item  $0.50$
                    \end{multicols}
                    \begin{multicols}{2}
                    \item $0.67$ 
                        \columnbreak
                    \item $0.80$
                    \end{multicols}
            \end{enumerate}
        \item An axial compressor that generates a stagnation pressure ratio of $4.0$, operates with inlet and exit stagnation temperatures of $300 K$ and $480 K$, respectively. If the ratio of specific heats $\gamma$ is $1.4$, the isentropic effeciency of the compressor is,
            \begin{enumerate}
                    \begin{multicols}{2}
                    \item $0.94$
                        \columnbreak
                    \item $0.81$
                    \end{multicols}
                    \begin{multicols}{2}
                    \item $0.72$
                        \columnbreak
                    \item $0.63$
                    \end{multicols}
            \end{enumerate}
        \item An aircraft with a turboprop engine produces a thrust of $500 N$ and flies at $100 m/s$. If the propeller efficiency is $0.5$, the shaft power produced by the engine is 
            \begin{enumerate}
                    \begin{multicols}{2}
                    \item $50kW$
                        \columnbreak
                    \item $100kW$ 
                    \end{multicols}
                    \begin{multicols}{2}
                    \item $125kW$
                        \columnbreak
                    \item $500kW$
                    \end{multicols}
            \end{enumerate}
        \item A rocket has an initial mass of $150 kg$. After operating for a duration of $10 s$, its final mass is $50 kg$. If the acceleration due to gravity is $9.81 m/s^2$ and the thrust produced by the rocket is $19.62 kN$, the specific impulse of the rocket is
            \begin{enumerate}
                    \begin{multicols}{2}
                    \item $400s$
                        \columnbreak
                    \item $300s$
                    \end{multicols}
                    \begin{multicols}{2}
                    \item $200s$
                        \columnbreak
                    \item $100s$
                    \end{multicols}
            \end{enumerate}
        \item Consider the vector field $\vec{v}=\myvec{-\frac{y}{r^2} \\ \frac{x}{r^2}}$ where $r= \sqrt{x^2+y^2}$. The contour integral $ \oint \vec{v} \cdot \vec{ds}$ where $\vec{ds}$ is tangent to the contour that encloses the origin is, \rule{2cm}{0.2pt} \brak{ \text{accurate to two decimal places}}
        \item The magnitude of the $x$-component of the unit vector at the point $\myvec{1\\1}$ that is normal to equipotential lines of the potential function $\phi\brak{r}=\frac{1}{r^2+4}$, where $r=\sqrt{x^2+y^2}$ is,\brak{	\text{accurate to two decimal places}}
        \item Assuming $ISA$ standard sea level conditions , the density of air $\brak{\text{in } kg/m^3}$ at Leh, which has an altitude of $3500m$ above mean sea level is \rule{2cm}{0.2pt} \brak{\rule{2cm}{0.2pt}\text{accurate to two decimal places}}
        \item Consider a cubical tank of side $2 m$ with its top open. It is filled with water up to a height of $1 m$. Assuming the density of water to be $1000 kg/m^3, g = 9.81 m/s^2$ and the atmospheric pressure to be $100 kPa$, the net hydrostatic force $\brak{\text{in } kN}$ on the side face of the tank due to the air and water is \rule{2cm}{0.2pt} \brak{	\text{accurate to two decimal places}}
        \item An aircraft with mass of $400,000 kg$ cruises at $240 m/s$ at an altitude of $10 km$. Its lift to drag ratio at cruise is $15$. Assuming $g$ as $9.81 m/s^2$ , the power $\brak{\text{in } MW}$ needed for it to cruise is\rule{2cm}{0.2pt} \brak{	\text{accurate to two decimal places}}
        \item A statically-stable aircraft has a 
            \begin{align*}
                C_{L_{\alpha}}=5 \brak{\text{where the angle of attack,} \alpha \text{, is measured in radians}}
            \end{align*}. The coefficient of moment of the aircraft about the center of gravity is given as $C_{M,C,g}=0.05-4\alpha$. The mean aerodynamic chord of the aircraft wing is $1 m$. The location \brak{\text{positive towards the nose}} of the neutral point of the aircraft from the center of gravity is \rule{2cm}{0.2pt} \brak{	\text{accurate to two decimal places}}

        \item An aircraft with a gross weight of $2000kg$ has a speed of $130m/s$ at sea level, where the conditions are: $1$ atmosphere \brak{pressure}, $288K$ \brak{temperature}, and $1.23 kg/m^3$ \brak{density}. The speed $\brak{\text{in } m/s}$ required by the aircraft at an altitude of $9000m$, where the conditions are: $0.31$ atmosphere, $230K$, and $0.47 kg/m^3$, to maintain a steady, level flight is \rule{2cm}{0.2pt} \brak{	\text{accurate to two decimal places}} 


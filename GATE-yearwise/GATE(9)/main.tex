\iffalse
\chapter{2019}
\section{ma}
\author{EE24BTECH11030}
\fi
\item  Let
$
u_n = \frac{n!}{1.3.5\ldots (2n-1)}, \quad n \in \mathbb{N} \text{ (the set of all natural numbers)}.
$
Then $\lim_{n \to \infty} u_n$ is equal to \underline{\hspace{2cm}}.

\bigskip

\item  If the differential equation
\[
\frac{dy}{dx} = \sqrt{x^2 + y^2}, \quad y(1) = 2
\]
is solved using Euler's method with step-size $h = 0.1$, then $y(1.2)$ is equal to \underline{\hspace{2cm}} (round off to 2 places of decimal).

\bigskip

\item  Let $f$ be any polynomial function of degree at most 2 over $\mathbb{R}$ (the set of all real numbers).\\
If the constants $a$ and $b$ are such that
\[
\frac{df}{dx} = a \, f(x) + 2 \, f(x+1) + b \, f(x+2), \quad \text{for all } x \in \mathbb{R},
\]
then $4a + 3b$ is equal to \underline{\hspace{2cm}} (round off to 2 places of decimal).

\bigskip

\item  Let $L$ denote the value of the line integral
\[
\oint_C \left(3x - 4x^2y\right) dx + \left(4xy^2 + 2y\right) dy,
\]
where $C$, a circle of radius 2 with center at origin of the $xy$-plane, is traversed once in the anti-clockwise direction. Then $\frac{L}{\pi}$ is equal to \underline{\hspace{2cm}}.

\bigskip

\item  The temperature $T : \mathbb{R}^3 \setminus \{(0, 0, 0)\} \rightarrow \mathbb{R}$ at any point $P(x, y, z)$ is inversely proportional to the square of the distance of $P$ from the origin. If the value of the temperature $T$ at the point $R(0, 0, 1)$ is $\sqrt{3}$, then the rate of change of $T$ at the point $Q(1, 1, 2)$ in the direction of $\overrightarrow{QR}$ is equal to \underline{\hspace{2cm}} (round off to 2 places of decimal).

\bigskip


\item Let $f$ be a continuous function defined on $[0,2]$ such that $f(x) \geq 0$ for all $x \in [0,2]$. If the area bounded by $y = f(x), x = 0, y = 0$ and $x = b$ is $\sqrt{3 + b^2} - \sqrt{3}$, where $b \in (0,2]$, then $f(1)$ is equal to \underline{\hspace{1cm}} (round off to 1 place of decimal).
\bigskip

\item If the characteristic polynomial and minimal polynomial of a square matrix $A$ are $(\lambda - 1)(\lambda + 1)^4(\lambda - 2)^5$ and $(\lambda - 1)(\lambda + 1)(\lambda - 2)$, respectively, then the rank of the matrix $A + I$ is \underline{\hspace{1cm}}, where $I$ is the identity matrix of appropriate order.
\bigskip

\item Let $\omega$ be a primitive complex cube root of unity and $i = \sqrt{-1}$. Then the degree of the field extension $\mathbb{Q}\left(i, \sqrt{3}, \omega\right)$ over $\mathbb{Q}$ (the field of rational numbers) is \underline{\hspace{1cm}}.
\bigskip

\item Let
\[
\alpha = \int_C \frac{e^{i \pi z}}{2z^2 - 5z + 2} \, dz, \quad C : \cos t + i \sin t, \, 0 \leq t \leq 2\pi, \, i = \sqrt{-1}.
\]
Then the greatest integer less than or equal to $|\alpha|$ is \underline{\hspace{1cm}}.
\bigskip

\item Consider the system:
$
\begin{cases}
3x_1 + x_2 + 2x_3 - x_4 = a, \\
x_1 + x_2 + x_3 - 2x_4 = 3, \\
x_1, x_2, x_3, x_4 \geq 0.
\end{cases}
$
If $x_1 = 1, x_2 = b, x_3 = 0, x_4 = c$ is a basic feasible solution of the above system (where $a, b,$ and $c$ are real constants), then $a + b + c$ is equal to \underline{\hspace{1cm}}.
\bigskip

\item Let $f : \mathbb{C} \to \mathbb{C}$ be a function defined by $f(z) = z^6 - 5z^4 + 10$. Then the number of zeros of $f$ in $\{z \in \mathbb{C} : |z| < 2\}$ is \underline{\hspace{1cm}}. \\
($\mathbb{C}$ is the set of all complex numbers)
\bigskip

\item Let
$
\ell^2 = \{ x = (x_1, x_2, \ldots) : x_i \in \mathbb{C}, \sum_{i=1}^{\infty} |x_i|^2 < \infty \}
$
be a normed linear space with the norm
$
\| x \|_2 = \left( \sum_{i=1}^{\infty} |x_i|^2 \right)^{\frac{1}{2}}.
$
Let $g : \ell^2 \to \mathbb{C}$ be the bounded linear functional defined by
\[
g(x) = \sum_{n=1}^{\infty} \frac{x_n}{3^n} \quad \text{for all } x = (x_1, x_2, \dots) \in \ell^2.
\]
Then
\[
\left( \sup \{ |g(x)| : \| x \|_2 \leq 1 \} \right)^2
\]
is equal to \underline{\hspace{1cm}} (round off to 3 places of decimal). \\
($\mathbb{C}$ is the set of all complex numbers)
\bigskip

\item For the linear programming problem (LPP):
\[
\text{Maximize } Z = 2x_1 + 4x_2
\]
subject to
$
\begin{cases}
-x_1 + 2x_2 \leq 4, \\
3x_1 + \beta x_2 \leq 6, \\
x_1, x_2 \geq 0, \quad \beta \in \mathbb{R},
\end{cases}
$
($\mathbb{R}$ is the set of all real numbers)

consider the following statements:
\begin{enumerate}
    \item[I.] The LPP always has a finite optimal value for any $\beta \geq 0$.
    \item[II.] The dual of the LPP may be infeasible for some $\beta \geq 0$.
    \item[III.] If for some $\beta$, the point $(1,2)$ is feasible to the dual of the LPP, then $Z \leq 16$, for any feasible solution $(x_1, x_2)$ of the LPP.
    \item[IV.] If for some $\beta$, $x_1$ and $x_2$ are the basic variables in the optimal table of the LPP with $x_1 = \frac{1}{2}$, then the optimal value of dual of the LPP is 10.
\end{enumerate}

Then which of the above statements are TRUE?

\begin{enumerate}
    \item (A) I and III only
    \item (B) I, III and IV only
    \item (C) III and IV only
    \item (D) II and IV only
\end{enumerate}
\bigskip


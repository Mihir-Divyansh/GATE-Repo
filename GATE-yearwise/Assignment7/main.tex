\iffalse
  \author{EE24BTECH11007}
  \section{ce}
  \chapter{2023}
\fi
\item Identify the waterborne diseases caused by viral pathogens:
\begin{enumerate}
\item Acute anterior poliomyelitis
\item Cholera
\item Infectious hepatitis
\item Typhoid fever
\end{enumerate}
\item Which of the following statements is/are TRUE for the Refuse-Derived Fuel (RDF) in the context of Municipal Solid Waste (MSW) management?
\begin{enumerate}
\item Higher Heating Value (HHV) of the unprocessed MSW is higher than the HHV of RDF processed from the same MSW
\item RDF can be made in the powdered form
\item Inorganic fraction of MSW is mostly converted to RDF
\item RDF cannot be used in conjunction with oil
\end{enumerate}
\item  The probabilities of occurrences of two independent events A and B are 0.5 and 0.8, respectively. What is the probability of occurrence of at least A or B (rounded off to one decimal place)?\rule{2cm}{0.15mm}
\item In the differential equation $\frac{dy}{dx}+\alpha xy=0$, $\alpha$ is a positive constant. If $y=1.0$ at $x=0.0$, and $y=0.8$ at $x=1.0$, the value of $\alpha$ is \rule{2cm}{0.15mm}(rounded off to three decimal places).
\item Consider the fillet-welded lap joint shown in the figure (not to scale). The length of the weld shown is the effective length. The welded surfaces meet at right angle. The weld size is 8$mm$, and the permissible stress in the weld is 120$MPa$. What is the safe load P (in $kN$, rounded off to one decimal place) that can be transmitted by this welded joint?
\begin{figure}[H]
    \centering
    \begin{circuitikz}
\tikzstyle{every node}=[font=\Large]
\draw  (6.25,14.75) rectangle (16.25,9.75);
\draw [dashed] (6.25,12.25) -- (18.75,12.25);
\draw [dashed] (11.25,14.75) -- (11.25,9.75);
\draw [short] (11.25,14.75) -- (11.25,16);
\draw [short] (11.25,16) -- (18.75,16);
\draw [short] (18.75,16) -- (18.75,8.5);
\draw [short] (11.25,9.75) -- (11.25,8.5);
\draw [short] (11.25,8.5) -- (18.75,8.5);
\draw [line width=2pt, short] (11.25,15) -- (16.5,15);
\draw [line width=2pt, short] (16.5,15) -- (16.5,9.5);
\draw [line width=2pt, short] (16.5,9.5) -- (11.25,9.5);
\draw [line width=1.2pt, ->, >=Stealth] (6.25,12.25) -- (5,12.25);
\draw [line width=1.2pt, ->, >=Stealth] (18.75,12.25) -- (20,12.25);
\draw [line width=1.2pt, <->, >=Stealth] (11.25,15.25) -- (16.5,15.25);
\draw [line width=1.2pt, <->, >=Stealth] (16.75,15) -- (16.75,9.5);
\node [font=\large] at (13.5,15.5) {$75mm$};
\node [font=\large] at (17.5,12.5) {$50mm$};
\node [font=\Large] at (20.25,12.25) {$P$};
\node [font=\Large] at (4.75,12.25) {$P$};
\end{circuitikz}
 \end{figure}
\item A drained direct shear test was carried out on a sandy soil. Under a normal stress of
50$kPa$, the test specimen failed at a shear stress of 35$kPa$. The angle of internal
friction of the sample is \rule{2cm}{0.15mm} degree (round off to the nearest integer).
\item A canal supplies water to an area growing wheat over 100 hectares. The duration between the first and last watering is 120 days, and the total depth of water required by the crop is 35$cm$. The most intense watering is required over a period of 30 days and requires a total depth of water equal to 12$cm$. Assuming precipitation to be negligible and neglecting all losses, the minimum discharge (in $m^3/s$, rounded off to three decimal places) in the canal to satisfy the crop requirement is \rule{2cm}{0.15mm}.
\item The ordinates of a one-hour unit hydrograph for a catchment are given below:
\begin{table}[h]
\centering
\begin{tabular}{|c|c|c|c|c|c|c|c|c|}
\hline
$t$ (hour) & 0 & 1 & 2 & 3 & 4 & 5 & 6 & 7 \\ \hline
$Q \brak{m^3/s}$ & 0 & 9 & 21 & 18 & 12 & 5 & 2 & 0 \\ \hline
\end{tabular}
\end{table}\\
Using the principle of superposition, a $D$-hour unit hydrograph for the catchment was derived from this one-hour unit hydrograph. The ordinates of the $D$-hour unit hydrograph were obtained as 3$m^3/s$ at $t$=1 hour and 10$m^3/s$ at $t$=2 hour. The value of $D$ (in integer) is \rule{2cm}{0.15mm}.
\item For a horizontal curve, the radius of a circular curve is obtained as 300$m$ with the design speed as 15$m/s$. If the allowable jerk is 0.75$m/s^3$, what is the minimum length (in $m$, in integer) of the transition curve?\rule{2cm}{0.15mm}
\item A function $f\brak{x}$, that is smooth and convex-shaped between interval \brak{x_l,x_u} is shown in the figure. This function is observed at odd number of regularly spaced points. If the area under the function is computed numerically, then \rule{2cm}{0.15mm}.
\begin{figure}[H]
    \centering
    \begin{circuitikz}
\tikzstyle{every node}=[font=\large]
\draw [line width=1.2pt, ->, >=Stealth] (5,8.5) -- (5,14.75);
\draw [line width=1.2pt, ->, >=Stealth] (5,8.5) -- (15,8.5);
\draw [line width=0.8pt, short] (6.25,9.75) -- (6.25,8);
\draw [line width=0.8pt, short] (6.75,8.5) -- (6.75,8);
\draw [line width=0.8pt, short] (7.25,8.5) -- (7.25,8);
\draw [line width=0.8pt, short] (7.75,8.5) -- (9.75,8.5);
\draw [line width=0.8pt, short] (7.75,8.5) -- (7.75,8);
\draw [line width=0.8pt, short] (8.25,8.5) -- (8.25,8);
\draw [line width=0.8pt, short] (8.75,8.5) -- (8.75,8);
\draw [line width=0.8pt, short] (9.25,8.5) -- (9.25,8);
\draw [line width=0.8pt, short] (9.75,8.5) -- (9.75,8);
\draw [line width=0.8pt, short] (10.25,8.5) -- (10.25,8);
\draw [line width=0.8pt, short] (10.75,8.5) -- (10.75,8);
\draw [line width=0.8pt, short] (11.25,9.75) -- (11.25,8);
\draw [line width=0.8pt, short] (6,9) .. controls (8.5,14.75) and (9.25,14.75) .. (11.25,9);
\draw [ color={rgb,255:red,255; green,0; blue,0}, line width=0.8pt, dashed] (6.25,9.5) .. controls (8.75,15.75) and (9.75,13) .. (11,9.5);
\node [font=\Large] at (4.25,15) {$f\brak{x}$};
\node [font=\Large] at (4.25,8) {$\brak{0,0}$};
\node [font=\Large] at (6.25,7.5) {$x_l$};
\node [font=\Large] at (11.25,7.5) {$x_u$};
\node [font=\Large] at (15,7.5) {$x$};
\draw [line width=1.4pt, short] (11.25,14) -- (12.25,14);
\draw [ color={rgb,255:red,254; green,6; blue,6}, line width=0.7pt, dashed] (11.25,13.5) -- (12.25,13.5);
\node [font=\normalsize, color={rgb,255:red,5; green,5; blue,5}] at (13.75,14) {Actual function};
\node [font=\normalsize, color={rgb,255:red,5; green,5; blue,5}] at (14.5,13.5) {Approximated function};
\end{circuitikz}
 \end{figure}
\begin{enumerate}
\item the numerical value of the area obtained using the trapezoidal rule will be less than the actual
\item the numerical value of the area obtained using the trapezoidal rule will be more than the actual
\item the numerical value of the area obtained using the trapezoidal rule will be exactly equal to the actual
\item with the given details, the numerical value of area cannot be obtained using trapezoidal rule
\end{enumerate}
\item Consider a doubly reinforced RCC beam with the option of using either Fe250 plain bars or Fe500 deformed bars in the compression zone. The modulus of elasticity of steel is $2 \times 10^5 N/mm^2$. As per IS456:2000, in which type(s) of the bars, the stress in the compression steel \brak{f_{sc}}can reach the design strength \brak{0.87f_y} at the limit
state of collapse?
\begin{enumerate}
\item Fe250 plain bars only
\item Fe500 deformed bars only
\item Both Fe250 plain bars and Fe500 deformed bars
\item Neither Fe250 plain bars nor Fe500 deformed bars
\end{enumerate}
\item Consider the horizontal axis passing through the centroid of the steel beam cross-section shown in the figure. What is the shape factor (rounded off to one decimal place) for the cross-section?
\begin{figure}[H]
    \centering
    \begin{circuitikz}
\tikzstyle{every node}=[font=\normalsize]
\draw [line width=0.7pt, short] (8.75,14.75) -- (8.75,12.25);
\draw [line width=0.7pt, short] (8.75,12.25) -- (6.25,12.25);
\draw [line width=0.7pt, short] (6.25,12.25) -- (6.25,9.75);
\draw [line width=0.7pt, short] (6.25,9.75) -- (8.75,9.75);
\draw [line width=0.7pt, short] (8.75,9.75) -- (8.75,7.25);
\draw [line width=0.7pt, short] (8.75,7.25) -- (11.25,7.25);
\draw [line width=0.7pt, short] (11.25,7.25) -- (11.25,9.75);
\draw [line width=0.7pt, short] (11.25,9.75) -- (13.75,9.75);
\draw [line width=0.7pt, short] (13.75,9.75) -- (13.75,12.25);
\draw [line width=0.7pt, short] (13.75,12.25) -- (11.25,12.25);
\draw [line width=0.7pt, short] (11.25,12.25) -- (11.25,14.75);
\draw [line width=0.7pt, short] (11.25,14.75) -- (8.75,14.75);
\draw [line width=0.7pt, <->, >=Stealth] (14.25,14.75) -- (14.25,12.25);
\draw [line width=0.7pt, <->, >=Stealth] (14.25,12.25) -- (14.25,9.75);
\draw [line width=0.7pt, <->, >=Stealth] (14.25,9.75) -- (14.25,7.25);
\draw [line width=0.7pt, <->, >=Stealth] (13.75,7) -- (11.25,7);
\draw [line width=0.7pt, <->, >=Stealth] (11.25,7) -- (8.75,7);
\draw [line width=0.7pt, <->, >=Stealth] (8.75,7) -- (6.25,7);
\node [font=\normalsize] at (7.5,6.75) {$b$};
\node [font=\normalsize] at (10,6.75) {$b$};
\node [font=\normalsize] at (12.5,6.75) {$b$};
\node [font=\normalsize] at (14.5,8.5) {$b$};
\node [font=\normalsize] at (14.5,11) {$b$};
\node [font=\normalsize] at (14.5,13.5) {$b$};
\end{circuitikz}
 \end{figure}
\begin{enumerate}
\begin{multicols}{4}
\item 1.5
\item 1.7
\item 1.3
\item 2.0
\end{multicols}
\end{enumerate}
\item Consider the pin-jointed truss shown in the figure (not to scale). All members have the same axial rigidity, $AE$. Members $QR$, $RS$, and $ST$ have the same length $L$. Angles $QBT$, $RCT$, $SDT$ are all 90$\degree$. Angles $BQT$, $CRT$, $DST$ are all 30$\degree$. The joint $T$ carries a vertical load $P$. The vertical deflection of joint $T$ is $k\frac{PL}{AE}$ . What is the value of $k$
\begin{figure}[H]
    \centering
    \begin{circuitikz}
\tikzstyle{every node}=[font=\normalsize]
\draw [line width=0.7pt, short] (7.5,16) -- (7.5,6);
\draw [line width=0.7pt, short] (7.5,6) -- (13.75,6);
\draw [line width=0.7pt, short] (7.5,16) -- (13.75,6);
\draw [line width=0.7pt, short] (7.5,14.75) .. controls (8,14.5) and (8,14.5) .. (8.25,14.75);
\draw [line width=0.7pt, short] (13,6) .. controls (13,6.5) and (13,6.5) .. (13.25,6.75);
\draw [line width=0.7pt, short] (7.5,16) -- (6.75,16.5);
\draw [line width=0.7pt, short] (6.75,16.5) -- (6.75,15.5);
\draw [line width=0.7pt, short] (6.75,15.5) -- (7.5,16);
\draw [ fill={rgb,255:red,163; green,163; blue,163} , line width=0.7pt , dashed] (6.25,16.5) rectangle  (6.75,15.5);
\draw [line width=0.7pt, short] (7.5,6) -- (6.75,6.75);
\draw [line width=0.7pt, short] (6.75,6.75) -- (6.75,5.5);
\draw [line width=0.7pt, short] (6.75,5.5) -- (7.5,6);
\draw [ fill={rgb,255:red,163; green,163; blue,163} , line width=0.7pt ] (6.5,6.5) circle (0.25cm);
\draw [ fill={rgb,255:red,163; green,163; blue,163} , line width=0.7pt ] (6.5,5.75) circle (0.25cm);
\draw [ fill={rgb,255:red,163; green,163; blue,163} , line width=0.7pt , dashed] (5.75,6.75) rectangle  (6.25,5.5);
\draw [line width=0.7pt, short] (7.5,6) -- (9.5,12.75);
\draw [line width=0.7pt, short] (9.5,12.75) -- (9.5,6);
\draw [line width=0.7pt, short] (9.5,6) -- (11.25,10);
\draw [line width=0.7pt, short] (11.25,10) -- (11.25,6);
\draw [line width=0.7pt, <->, >=Stealth] (7.75,16.25) -- (9.75,13);
\draw [line width=0.7pt, <->, >=Stealth] (9.75,13) -- (11.5,10.25);
\draw [line width=0.7pt, <->, >=Stealth] (11.5,10.25) -- (14,6.25);
\node [font=\normalsize] at (7.5,16.5) {$Q$};
\node [font=\normalsize] at (8.75,15) {$L$};
\node [font=\normalsize] at (10.75,11.75) {$L$};
\node [font=\normalsize] at (12.75,8.75) {$L$};
\node [font=\normalsize] at (7.75,14.25) {$30\degree$};
\node [font=\normalsize] at (12.5,6.75) {$60\degree$};
\node [font=\normalsize] at (7.5,5.5) {$B$};
\node [font=\normalsize] at (9.5,5.5) {$C$};
\node [font=\normalsize] at (11.25,5.5) {$D$};
\node [font=\normalsize] at (14,6) {$T
T$};
\draw [line width=0.7pt, ->, >=Stealth] (13.75,6) -- (13.75,5);
\node [font=\normalsize] at (14,5.25) {$P$};
\draw [ fill={rgb,255:red,163; green,163; blue,163} , line width=0.7pt ] (7.5,16) circle (0.25cm);
\draw [ fill={rgb,255:red,163; green,163; blue,163} , line width=0.7pt ] (9.5,12.75) circle (0.25cm);
\draw [ fill={rgb,255:red,163; green,163; blue,163} , line width=0.7pt ] (11.25,10) circle (0.25cm);
\draw [ fill={rgb,255:red,163; green,163; blue,163} , line width=0.7pt ] (13.75,6) circle (0.25cm);
\draw [ fill={rgb,255:red,163; green,163; blue,163} , line width=0.7pt ] (11.25,6) circle (0.25cm);
\draw [ fill={rgb,255:red,163; green,163; blue,163} , line width=0.7pt ] (9.5,6) circle (0.25cm);
\draw [ fill={rgb,255:red,163; green,163; blue,163} , line width=0.7pt ] (7.5,6) circle (0.25cm);
\node [font=\normalsize] at (10,13) {$R$};
\node [font=\normalsize] at (11.75,10.25) {$S$};
\end{circuitikz}
 \end{figure}
 \begin{enumerate}
\begin{multicols}{4}
\item 1.5
\item 4.5
\item 3.0
\item 9.0
\end{multicols}
\end{enumerate}

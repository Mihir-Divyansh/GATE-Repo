\iffalse
\title{Assignment2}
\author{ee24btech110664}
\chapter{2007}
\section{ma}
\fi

%\begin{enumerate}
\item Consider $\mathbb{R}^2$ with the usual topology. Let $S=\{(x,y)\in \mathbb{R}^2:\text{x is an integer}\}$. Then S is 
\begin{enumerate}
    \item open but NOT closed 
    \item  both open and closed
    \item  neither open nor closed
    \item  closed but NOT open
\end{enumerate}
\item Suppose $X=\{\alpha,\beta,\gamma\}$. Let
\begin{align*}
    S_1=\{\phi,X,\{\alpha\},\{\alpha,\beta\}\} 
\end{align*}
\begin{align*}
    S_2=\{\phi,X,\{\alpha\},\{\beta,\gamma\}\} 
\end{align*}
Then,
\begin{enumerate}
    \item both $S_1 \cap S_2$ and $S_1\cup S_2$ are topologies
    \item neither$S_1 \cap S_2$ nor $S_1\cup S_2$ is a topology
    \item $S_1\cup S_2$ is a topology but $S_1 \cap S_2$ is NOT a topology
    \item $S_1 \cap S_2$ is a topology but $S_1\cup S_2$ is NOT a topology
\end{enumerate}
\item For a positive integer n, let $f_n:R\rightarrow R$ be defined by
\[
f_n(x) = \begin{cases}
\frac{1}{4n+5}, & \text{if } 0 \leq x \leq n, \\
0, & \text{otherwise.}
\end{cases}
\]
Then $\{f_n\brak{x}\}$ converges to zero
\begin{enumerate}
       \item uniformly but NOT in $L'$ norm
       \item uniformly and also in $L'$ norm
       \item pointwise but NOT uniformly
       \item in $L'$ norm but NOT pointwise 
\end{enumerate}
\item Let $P_1$ and $P_2$ be two projection operators on a vector space. Then 
\begin{enumerate}
    \item $P_1+P_2$ is a projection if $P_1P_2=P_2P_1=0$
    \item $P_1-P_2$ is a projection if $P_1P_2=P_2P_1=0$
    \item $P_1+P_2$ is a projection 
    \item $P_1-P_2$ is a projection 
\end{enumerate}
\item Consider the system of linear equations 
\begin{align*}
    x+y+z=3
\end{align*}
\begin{align*}
    x-y-z=4
\end{align*}
\begin{align*}
    x-5y+kz=6
\end{align*}
Then the value of k for which this systen had an infinite number of solutions is
\begin{multicols}{2}
    \begin{enumerate}
    \item $k=-5$
    \item $k=0$
    \item $k=1$
    \item $k=3$
\end{enumerate}
\end{multicols}
\item Let 
\begin{align*}
    A=\myvec{1 & 1 & 1 \\ 2 &2 &3\\x& y & z }
\end{align*}
and let $V=\{\brak{x,y,z}\in R^3:\text{det}\brak{A}=0\}$. Then the dimension of V equals
\begin{multicols}{2}
\begin{enumerate}
    \item $0$
    \item $1$
    \item $2$
    \item $3$
\end{enumerate}    
\end{multicols}
\item Let $S=\{0\}\cup \{\frac{1}{4n+7}:n=1,2,\cdots\}$. Then the number of analytic functions which vanish only on S is
\begin{multicols}{2}
\begin{enumerate}
    \item infinite
    \item $0$
    \item $1$
    \item $2$
\end{enumerate}    
\end{multicols}
\item It is given that $\sum_{n=0}^{\infty}a_n z^n$ converges at $z=3+i4$. Then the radius of convergance of the power series $\sum_{n=0}^{\infty}a_n z^n$ is 
\begin{multicols}{2}
\begin{enumerate}
    \item $\leq 5$
    \item $\geq 5$
    \item $<5$
    \item $>5$
\end{enumerate}    
\end{multicols}
\item The value of $\alpha$ for which $G=\{\alpha,1,3,9,19,27\}$ is a cyclic group under multiplication modulo 56 is 
\begin{multicols}{2}
\begin{enumerate}
    \item $5$
    \item $15$
    \item $25$
    \item $35$
\end{enumerate}    
\end{multicols}
\item Consider $\mathbb{Z}_{24}$ as the additive group modulo 24. Then the number of elements of order 8 in the group $\mathbb{Z}_{24}$ is
\begin{multicols}{2}
\begin{enumerate}
   \item $1$
    \item $2$
    \item $3$
    \item $4$
\end{enumerate}    
\end{multicols}
\item Define $f:\mathbb{R}^2\rightarrow \mathbb{R}$ by
\[
f(x,y) = \begin{cases}
1, & \text{if } xy = 0, \\
2, & \text{otherwise.}
\end{cases}
\]
\text{If } S = \{(x,y): f \text{ is continuous at the point } (x,y)\}, \text{ then}
\begin{multicols}{2}
\begin{enumerate}
    \item S is open
    \item S is connected 
    \item S=$\phi$
    \item S is closed 
\end{enumerate}    
\end{multicols}
\item Consider the linear programming problem,\\Max:$z=c_1x_1+c_2x_2,c_1,c_2>0$ subject to
\begin{align*}
    x_1+x_2 \leq 3
\end{align*}
\begin{align*}
    2x_1+3x_2 \leq 4
\end{align*}
\begin{align*}
    x_1,x_2\geq 0
\end{align*}
Then,
\begin{enumerate}
    \item the primal has an optimal solution but the dual does not have an optimal solution
    \item both the primal and the dual have optimal solutions 
    \item the dual has an optimal solution bu the primal does not have an optimal solution
    \item neither the primal nor the dual have optimal solutions
\end{enumerate}    
\item Let 
\begin{align*}
    f\brak{x}=x^{10}+x-1,x \in R
\end{align*}
and let $x_k=k, k=0,1,2,\cdots,10$.Then the value of the divided difference \\$f[x_0,x_1,x_2,x_3,x_4,x_5,x_6,x_7,x_8,x_9,x_{10}]$ is 
\begin{multicols}{2}
\begin{enumerate}
  \item $-1$
    \item $0$
    \item $1$
    \item $10$
\end{enumerate}    
\end{multicols}
\item Let $X$ and $Y$ be jointly distributed random variables having the joint probability density function 
\[
f(x,y) = \begin{cases}
\frac{1}{\pi}, & \text{if } x^2 + y^2 \leq 1, \\
0, & \text{otherwise.}
\end{cases}
\]
Then $P\brak{Y>max\brak{X,-X}}=$
\begin{multicols}{2}
    \begin{enumerate}
    \item $\frac{1}{2}$
    \item $\frac{1}{3}$
    \item $\frac{1}{4}$
    \item $\frac{1}{6}$
\end{enumerate}
\end{multicols}
\item Let $X_1,X_2,\cdots$ be a sequence of independent and identity distributed chi-square random variables, each having 4 degrees of freedom. Define $S_n=\sum_{i=1}^{n}X_i^{2},n=1,2,\cdots$. If $\frac{S_n}{n}\rightarrow \micro$ as $n\rightarrow \infty$, then $\micro=$
\begin{multicols}{2}
    \begin{enumerate}
  \item $8$
  \item $16$
  \item $24$
  \item $32$
\end{enumerate}
\end{multicols}
\item Let $\{E_n:n=1,2,\cdots\}$ be a deceasing sequence of Lebesgue measurable sets on $\mathbb{R}$ and let F be a Lebesgue measurable set on $\mathbb{R}$ such that $E_1 \cap F=\phi$. Suppose that F has Lebesgue measure 2 and the Lebesgue measure of $E_n$ equals $\frac{2n+2}{3n+1},n=1,2,\cdots$ Then the Lebesgue measure of the set ($\cap_{n=1}^{\infty}E_n)\cup F$ equals
\begin{multicols}{2}
\begin{enumerate}
  \item $\frac{5}{3}$
    \item $2$
    \item $\frac{7}{3}$
    \item $\frac{8}{3}$
\end{enumerate}    
\end{multicols}
\item The extremum for the variational problem
\begin{align*}
    \int_{0}^{\frac{\pi}{8}} \left[ (y')^2 + 2yy' - 16y^2 \right] dx, \quad 
\end{align*}
\begin{align*}
    y(0)=0,\quad y\left(\frac{\pi}{8}\right)=1
\end{align*}
occurs for the curve
\begin{multicols}{2}
    \begin{enumerate}
        \item $y=\sin{4x}$
        \item $y=\sqrt{2}\sin{2x}$
        \item $y=1-\cos{4x}$
        \item $y=\frac{1-\cos{8x}}{2}$
    \end{enumerate}
\end{multicols}
%\end{enumerate}

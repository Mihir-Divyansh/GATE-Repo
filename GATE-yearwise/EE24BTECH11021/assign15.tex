 \iffalse
\chapter{2020}
\author{EE24BTECH11021 - Eshan Ray}
\section{me}
\fi
        \item The members carrying zero force \brak{\text{$i\cdot e\cdot$ zero-force members}} in the truss shown in the figure, for any load $P\textgreater 0$ with no appreciable deformation of the truss \brak{\text{$i\cdot e\cdot$ with no appreciable change in angles between the members}}, are
    \begin{circuitikz} [scale=0.6]
\tikzstyle{every node}=[font=\large]
\draw [short] (6.5,10) -- (17.25,10);
\draw [short] (6.5,10) -- (9.75,12.75);
\draw [short] (9.75,12.75) -- (12.75,10);
\draw [short] (12.75,10) -- (15.75,12.75);
\draw [short] (15.75,12.75) -- (19.5,10);
\draw [short] (17.25,10) -- (19.5,10);
\draw [short] (9.75,12.75) -- (15.75,12.75);
\draw [short] (12.75,10) -- (12.75,12.75);
\draw [->, >=Stealth] (12.75,10) -- (12.75,7.25);
\draw [short] (9.75,12.75) -- (9.75,10);
\draw [short] (15.75,12.75) -- (15.75,10);
\draw [short] (9,12.75) -- (5.5,12.75);
\draw [short] (6,10) -- (5.5,10);
\draw [<->, >=Stealth] (5.75,12.75) -- (5.75,10);
\draw [short] (6.5,10.25) -- (6,9.75);
\draw [short] (6.5,10.25) -- (7,9.75);
\draw [short] (7,9.75) -- (6,9.75);
\draw [short] (19,10.25) -- (19.75,9.5);
\draw [short] (19,10.25) -- (20,10.25);
\draw [short] (19.75,9.5) -- (20.25,10.25);
\draw [short] (20.25,10.25) -- (19.75,10.25);
\draw [short] (19.75,9) -- (20.75,10.5);
\draw [dashed] (20.75,10.5) -- (21.5,11.5);
\draw [dashed] (21,10.75) -- (22.75,10.75);
\draw [short] (15.75,9.75) -- (15.75,9);
\draw [short] (19.5,9.75) -- (19.5,9);
\draw [short] (9.75,9.75) -- (9.75,9);
\draw [short] (6.5,9.75) -- (6.5,9);
\draw [<->, >=Stealth] (6.5,9.25) -- (9.75,9.25);
\draw [<->, >=Stealth] (9.75,9.25) -- (12.75,9.25);
\draw [<->, >=Stealth] (12.75,9.25) -- (15.75,9.25);
\draw [<->, >=Stealth] (15.75,9.25) -- (19.5,9.25);
\node [font=\normalsize] at (7.75,9) {$1$ m};
\node [font=\normalsize] at (11.25,9) {$1$ m};
\node [font=\normalsize] at (14,9) {$1$ m};
\node [font=\normalsize] at (17.5,9) {$1$ m};
\node [font=\normalsize] at (5.25,11.25) {$1$ m};
\node [font=\large] at (6.5,10.75) {A};
\node [font=\large] at (9.5,10.5) {B};
\node [font=\large] at (12.5,9.75) {C};
\node [font=\large] at (15.5,10.5) {D};
\node [font=\large] at (19.25,10.75) {E};
\node [font=\large] at (9.75,13.25) {F};
\node [font=\large] at (12.75,13) {G};
\node [font=\large] at (15.75,13) {H};
\node [font=\large] at (12.75,6.75) {P};
\node [font=\large] at (22.75,11.5) {$45\degree$};
\draw [<->, thick] (22,10.75) to[bend right=45] (21.25,11.25);
\draw (20.3,10.1) circle (0.135 cm);
\draw (20,9.7) circle (0.135 cm);
\end{circuitikz}
    \begin{enumerate}
        \item $BF$ and $DH$ only
        \item $BF,DH$ and $GC$ only
        \item $BF,DH,GC,CD$ and $DE$ only
        \item $BF,DH,GC,FG$ and $GH$ only
    \end{enumerate}
    \item Which of the following function $f\brak{z},$ of the complex variable $z$, is NOT analytic at all points on the complex plane?
    \begin{enumerate}
        \item $f\brak{z}=z^2$
        \item $f\brak{z}=e^z$
        \item $f\brak{z}=\sin{z}$
        \item $f\brak{z}=\log z$
    \end{enumerate}
    \item A single-degree-of-freedom oscillator is subjected to harmonic excitation $F\brak{t}=F_0\cos\brak{\omega t}$ as shown in the figure.\\
    
    \begin{circuitikz}
\tikzstyle{every node}=[font=\large]
\draw (12,11.75) to[R] (12,10.25);
\draw (12,10.25) to (12,10) node[cground]{};
\draw (12.5,10.5) to (12.5,10) node[cground]{};
\draw [short] (12.25,10.5) -- (12.75,10.5);
\draw [short] (12.75,10.5) -- (12.75,11);
\draw [short] (12.25,10.5) -- (12.25,11);
\draw [short] (12.25,10.75) -- (12.75,10.75);
\draw [short] (12.5,10.75) -- (12.5,11.75);
\draw [short] (11.5,11.75) -- (13,11.75);
\draw [short] (13,11.75) -- (13,13);
\draw [short] (11.5,11.75) -- (11.5,13);
\draw [short] (11.5,13) -- (13,13);
\draw [->, >=Stealth] (12.25,13) -- (12.25,14.25);
\node [font=\Large] at (12.25,12.5) {m};
\node [font=\large] at (12.75,14.5) {F(t)};
\node [font=\large] at (11.5,11) {k};
\node [font=\large] at (13,10.75) {c};
\end{circuitikz}\\
The non-zero value of $\omega$, for which  the amplitude of the force transmitted to the ground will be $F_0$, is
    \begin{enumerate}
        \item $\sqrt{\frac{k}{2m}}$
        \item $\sqrt{\frac{k}{m}}$
        \item $\sqrt{\frac{2k}{m}}$
        \item 2$\sqrt{\frac{k}{m}}$
    \end{enumerate}
    \item The stress state at a point in a material under plane stress condition is equi-biaxial tension with a magnitude of $10\,MPa$. If one unit on the $\sigma-\tau$ plane is $1\,MPa$, the Mohr's circle representation of the state-of-stress is given by 
    \begin{enumerate}
        \item a circle with a radius equal to principal stress and its center at the origin of the $\sigma-\tau$ plane
        \item a point on the $\sigma$ axis at a distance of $10$ units from the origin 
        \item a circle with a radius of $10$ units on the $\sigma-\tau$ plane
        \item a point on the $\tau$ axis at a distance of $10$ units from the origin
    \end{enumerate}
    \item A four bar mechanism is shown below.\\
\begin{circuitikz}
\tikzstyle{every node}=[font=\normalsize]
\draw [short] (10,9.25) -- (8.75,10.75);
\draw [short] (8.75,10.75) -- (14,12.25);
\draw [short] (14,12.25) -- (14,9.75);
\draw [short] (14,9.5) -- (14,9.75);
\draw (10,9.25) to (10,9) node[cground]{};
\node [font=\normalsize] at (11,11.75) {600 mm};
\node [font=\normalsize] at (14.75,11) {300 mm};
\node [font=\normalsize] at (11.75,9) {400 mm};
\node [font=\normalsize] at (9.5,9.25) {P};
\node [font=\normalsize] at (8.5,10.75) {Q};
\node [font=\normalsize] at (14.25,12.5) {R};
\node [font=\normalsize] at (14.25,9.5) {S};
\draw (10,9.25) to[short] (14,9.25);
\draw (14,9.5) to (14,9) node[cground]{};
\draw [short] (10,9.25) -- (9.75,8.5);
\draw [short] (10,9.25) -- (10.25,8.5);
\draw [short] (14,9.25) -- (13.75,8.5);
\draw [short] (14,9.25) -- (14.25,8.5);
\end{circuitikz}\\
For the mechanism to be a crank-rocker mechanism, the length of the link $PQ$ can be
    \begin{enumerate}
        \item $80\, mm$
        \item $200\, mm$
        \item $300\, mm$
        \item $350\, mm$
    \end{enumerate}
    \item A helical gear with $20\degree$ pressure angle and $30\degree$ helix angle mounted at the mid-span of a shaft that is supported between two bearings at the ends. The nature of the stresses induced in the shaft is 
    \begin{enumerate}
        \item normal stress due to bending only
        \item normal stress due to bending in one plane and axial loading; shear stress due to torsion
        \item normal stress due to bending in two planes and axial loading; shear stress due to torsion
        \item normal stress due to bending in one plane; shear stress due to torsion
    \end{enumerate}
    \item The crystal structure of $\gamma$ ion \brak{\text{austenite phase}} is
    \begin{enumerate}
        \item BCC
        \item FCC 
        \item HCP
        \item BCT
    \end{enumerate}
    \item Match the following.
	\begin{table}[H]    
  \centering
  \begin{tabular}[12pt]{ |c| c|}
    \hline
	\textbf{Heat treatment process}  & \textbf{Effect} \\
    \hline
	$P\colon\, Tempering$ &  $1.\quad Strengthening$  \\
    \hline 
	$Q\colon\,Quenching$ &  $2.\quad Toughening$ \\
    \hline
	$R\colon\, Annealing$ &  $3.\quad Hardening$  \\  
    \hline
    	$S\colon\, Normalizing$ &  $4.\quad Softening$  \\
    \hline         
\end{tabular}

  \label{tab1.1.9.2}
\end{table}
    
    \begin{enumerate}
        \item $P-2,Q-3,R-4,S-1$
        \item $P-1,Q-1,R-3,S-2$
        \item $P-3,Q-3,R-1,S-3$
        \item $P-4,Q-3,R-2,S-1$
    \end{enumerate}
    \item The base of a brass bracket needs rough grinding. For this purpose, the most suitable grinding wheel grade specification is
    \begin{enumerate}
        \item $C30Q12V$
        \item $A50G8V$
        \item $C90J4B$
        \item $A30D12V$
    \end{enumerate}
    \item In the critical Path Method \brak{CPM}, the cost-time slope of an activity is given by 
    \begin{enumerate}
        \item $\frac{\text{Crash Cost - Normal Cost}}{\text{Crash Time}}$
        \item $\frac{\text{Normal Cost}}{\text{Crash Time - Normal Time}}$
        \item $\frac{\text{Crash Cost}}{Crash Time - Normal Time}$
        \item $\frac{\text{Crash Cost - Normal Cost}}{\text{Normal Time - Crash Time}}$
    \end{enumerate}
    \item Froude number is the ratio of 
    \begin{enumerate}
        \item buoyancy forces to viscous forces 
        \item inertia forces to viscous forces
        \item buoyancy forces to inertia forces
        \item inertia forces to gravity forces
    \end{enumerate}
    \item Match the following non-dimensional numbers with the corresponding definitions $\colon$
	\begin{table}[H]    
  \centering
      \begin{tabular}{|c|c|}
        \hline
        \textbf{Non-dimensional number} &  \textbf{Definition} \\
        \hline
        $P\colon$  Reynolds number & $1.\frac{\text{Inertia force}}{\text{Viscous force}}$ \\
        \hline
        $Q\colon$ Grashof number & $2.\frac{\text{Buoyancy force}}{\text{Viscous force}}$ \\
        \hline
        $R\colon$ Nusselt number & $3.\frac{\text{Convective heat transfer}}{\text{Conduction heat transfer}}$ \\
        \hline
        $S\colon$ Prandtl number & $4.\frac{\text{Momentum diffusivity}}{\text{Thermal diffusivity}}$ \\
        \hline
    \end{tabular}


  \label{tab1.1.9.2}
\end{table}

    \begin{enumerate}
        \item $P-1,Q-3,R-2,S-4$
        \item $P-3,Q-1,R-2,S-4$
        \item $P-4,Q-3,R-1,S-2$
        \item $P-3,Q-1,R-4,S-2$
    \end{enumerate}
    \item The velocity field of an incompressible flow in a Cartesian system is repressented by 
    $$\overrightarrow{V}=2\brak{x^2-y^2}\hat{i}+v\hat{j}+3\hat{k}$$
    Which one of the following expressions for $v$ is valid?
    \begin{enumerate}
        \item $-4xz+6xy$
        \item $-4xy-4xz$
        \item $4xz-6xy$
        \item $4xy+4xz$
    \end{enumerate}

 \iffalse
\chapter{2022}
\author{EE24BTECH11021 - Eshan Ray}
\section{ce}
\fi
    \item The correct match between the physical states of the soils given in $\vec{Group\, I}$ and
the governing conditions given in $\vec{Group\, II}$ is 
\begin{table}[H]    
  \centering
  \begin{tabular}[12pt]{ |c| c| }
    \hline
	\textbf{Group I}  &  \textbf{Group II} \\
    \hline
$1$. normally consolidated soil &  $P.$ sensitivity $\textgreater 16$\\ \hline
$2$. quick clay & $Q$. $dilation\, angle\, = 0$\\ \hline
$3$. sand in critical state & $R$. liquid limit $\textgreater 50$ \\ \hline
$4$. clay of high plasticity & $S$. over consolidation $ratio = 1$ \\  
   \hline	
\end{tabular}

 \end{table}

\begin{enumerate}
    \item $1-S, 2-P, 3-Q, 4-R$
    \item $1-Q, 2-S, 3-P, 4-R$
    \item $1-Q, 2-P, 3-R, 4-S$
    \item $1-S, 2-Q, 3-P, 4-R$
\end{enumerate}
    \item As per Rankine's theory of earth pressure, the inclination of failure planes is $\brak{45+\frac{\phi}{2}}^{\degree}$ with respect to the direction of the minor principal stress.\\ \\
    The above statement is correct for which one of the following options? 
    \begin{enumerate}
        \item Only the active state and not the passive state
        \item Only the passive state and not the active state
        \item Both active as well as passive states
        \item Neither active nor passive state 
    \end{enumerate}
    \item Henry's law constant for transferring $O_2$ from air into water, at room temperature, is $1.3\,\frac{mmol}{liter-atm}$ . Given that the partial pressure of $O_2$ in the atmosphere is $0.21\, atm$, the concentration of dissolved oxygen $\brak{\frac{mg}{liter}}$ in water in equilibrium with the atmosphere at room temperature is \\ \\
    \brak{\text{Consider the molecular weight of}\, O_2\, \text{as}\, 32 \frac{g}{mol} }
    \begin{enumerate}
        \item $8.7$
        \item $0.8$
        \item $198.1$
        \item $0.2$
    \end{enumerate}
    \item In a water sample, the concentrations of $Ca^{2+}, Mg^{2+}$ and $HCO3^-$ are $\vec{100\,\frac{mg}{L}},\vec{36\,\frac{mg}{L}}$ and $\vec{122\,\frac{mg}{L}}$, respectively. The atomic masses of various elements are $\colon$ \\
    Ca = $\vec{40}$, Mg = $\vec{24}$, H = $\vec{1}$, C = $\vec{12}$, O = $\vec{16}$. \\ \\
    The total hardness and the temporary hardness in the water sample \brak{\text{in}\,\frac{mg}{L}\, \text{as}\, CaCO_3} will be 
    \begin{enumerate}
        \item $400$ and $100$, respectively.
        \item $400$ and $300$, respectively.
        \item $500$ and $100$, respectively.
        \item $800$ and $200$, respectively. 
    \end{enumerate}
    \item Consider the four points P, Q, R, and S shown in the Greenshields fundamental
speed-flow diagram. Denote their corresponding traffic densities by $k_P,k_Q,k_R$ and $k_S$, respectively. The correct order of these densities is 

  \begin{circuitikz}
\tikzstyle{every node}=[font=\large]
\draw [line width=0.7pt, ->, >=Stealth] (9.75,9.25) -- (16.75,9.25);
\draw [line width=0.7pt, ->, >=Stealth] (9.75,9.25) -- (9.75,14);
\draw [line width=0.7pt, short] (9.75,13) .. controls (15.75,11.5) and (12.75,10.5) .. (9.75,9.25);
\node [font=\large] at (9.25,13) {S};
\node [font=\large] at (12,12.75) {R};
\node [font=\large] at (14,11.5) {Q};
\node [font=\large] at (9.5,9) {P};
\node [font=\large] at (15,8.75) {Flow};
\node [font=\large] at (8.75,11.25) {Speed};
\fill (9.75,9.25) circle (2pt); % Small filled circle
\fill (9.75,13) circle (2pt); % Small filled circle
\fill (11.5,12.5) circle (2pt); % Small filled circle
\fill (13.15,11.5) circle (2pt); % Small filled circle
\end{circuitikz}
    \begin{enumerate}
        \item $k_P \textgreater k_Q \textgreater k_R \textgreater k_S$
        \item $k_S\textgreater k_R\textgreater k_Q\textgreater k_P$
        \item $k_Q\textgreater k_R\textgreater k_S\textgreater k_P$
        \item $k_Q\textgreater k_R\textgreater k_P\textgreater k_S$
    \end{enumerate}
    \item Let max $\cbrak{a,b}$ denote the maximum of two real numbers $a$ and $b$.\\
    Which of the following statement\brak{s} is/are TRUE about the function\\
    $f\brak{x}=max\cbrak{3-x,x-1}$ ? 
    \begin{enumerate}
        \item It is continuous on its domain.
        \item It has a local minimum at $x=2$.
        \item It has a local maximum at $x=2$.
        \item It is differentiable on its domain. 
    \end{enumerate}
    \item A horizontal force of $P\, kN$ is applied to a homogeneous body of weight $25\, kN$,
    as shown in the figure. The coefficient of friction between the body and the
floor is $0.3$. Which of the following statement\brak{s} is/are correct? 
    
  \begin{circuitikz}
\tikzstyle{every node}=[font=\large]
\draw  (9.5,14.75) rectangle (13.5,9.25);
\draw [->, >=Stealth] (15.75,14.75) -- (13.5,14.75);
\draw [short] (13.5,15) -- (13.5,15.75);
\draw [short] (9.5,15) -- (9.5,15.75);
\draw [short] (9.25,14.75) -- (8.5,14.75);
\draw [short] (9.25,9.25) -- (8.5,9.25);
\draw [<->, >=Stealth] (9.5,15.25) -- (13.5,15.25);
\draw [<->, >=Stealth] (9,14.75) -- (9,9.25);
\node [font=\large] at (11.5,15.75) {1 m};
\node [font=\large] at (8.5,12.25) {2 m};
\node [font=\large] at (16.25,14.75) {P};
\end{circuitikz}
    \begin{enumerate}
        \item The motion of the body will occur by overturning.
        \item Sliding of the body never occurs.
        \item No motion occurs for $P \leq6\, kN$.
        \item The motion of the body will occur by sliding only. 
    \end{enumerate}
    \item In the context of cross-drainage structures, the correct statement(s) regarding
    the relative positions of a natural drain $\brak{\frac{\text{stream}}{\text{river}}}$ and an irrigation canal, is/are 
    \begin{enumerate}
        \item In an aqueduct, natural drain water goes under the irrigation canal, whereas in a super-passage, natural drain water goes over the irrigation canal.
        \item In a level crossing, natural drain water goes through the irrigation canal.
        \item In an aqueduct, natural drain water goes over the irrigation canal, whereas in a super-passage, natural drain water goes under the irrigation canal.
        \item In a canal syphon, natural drain water goes through the irrigation canal. 
    \end{enumerate} 
    \item Consider the differential equation 
        $$\frac{dy}{dx}=4\brak{x+2}-y$$
        For the initial condition $y=3$ at $x=1$, the value of $y$ at $x=1.4$ obtained using Euler's method with a step-size of $0.2$ is \dots . \brak{round\, off \, to\, one\, decimal\,place}
    \item A set of observations of independent variable $\brak{x}$ and the corresponding
    dependent variable $\brak{y}$ is given below. 
	\begin{table}[H]    
  \centering
  \begin{tabular}[12pt]{ |c| c| c| c| c| }
    \hline
	$x$ & $5$ & $2$ & $4$ & $3$ \\
   \hline	
   	$y$ & $16$ & $10$ & $13$ & $12$ \\
   	\hline
\end{tabular}

 \end{table}
 
    Based on the data, the coefficient $a$ of the linear regression model 
        $$y=a+bx$$
        is estimated as $6.1$\\
        The coefficient $b$ is \dots 

    \item The plane truss shown in the figure is subjected to an external force $P$. It is given
that $P = 70\, kN$, $a = 2\, m$, and $b = 3\, m$. 
    
  \begin{circuitikz}[scale=0.8]
\tikzstyle{every node}=[font=\Large]
\draw [ line width=0.8pt](6.75,9) to[short, -o] (7.25,9.5) ;
\draw [ line width=0.8pt](7.25,9.5) to[short, -o] (7.25,12) ;
\draw [ line width=0.8pt](7.25,12) to[short, -o] (7.25,14.5) ;
\draw [ line width=0.8pt](7.25,14.5) to[short, -o] (9.75,12) ;
\draw [ line width=0.8pt](7.25,12) to[short, -o] (9.75,12) ;
\draw [ line width=0.8pt](9.75,12) to[short, -o] (9.75,14.5) ;
\draw [ line width=0.8pt](9.75,14.5) to[short, -o] (7.25,14.5) ;
\draw [ line width=0.8pt](7.25,9.5) to[short, -o] (9.75,12) ;
\draw [ line width=0.8pt](9.75,12) to[short, -o] (12.25,14.5) ;
\draw [ line width=0.8pt](9.75,14.5) to[short, -o] (12.25,14.5) ;
\draw [ line width=0.8pt](7.25,9.5) to[short] (7.75,9);
\draw [ fill={rgb,255:red,88; green,81; blue,81} , line width=0.8pt ] (6.5,9) rectangle (8,8.75);
\draw [ line width=0.8pt](12.25,14.5) to[short, -o] (12.25,11) ;
\draw [ line width=0.8pt](12.25,11) to[short, -o] (15.75,14.5) ;
\draw [ line width=0.8pt](15.75,14.5) to[short, -o] (19.25,11) ;
\draw [ line width=0.8pt](12.25,11) to[short, -o] (19.25,11) ;
\draw [ line width=0.8pt](12.25,14.5) to[short, -o] (15.75,14.5) ;
\draw [ line width=0.8pt](19.25,11) to[short, -o] (19.25,14.5) ;
\draw [ line width=0.8pt](15.75,14.5) to[short, -o] (19.25,14.5) ;
\draw [ line width=0.8pt](15.75,14.5) to[short, -o] (15.75,11) ;
\draw [ line width=0.8pt](19.25,11) to[short] (18.5,10.25);
\draw [ line width=0.8pt](19.25,11) to[short] (20,10.25);
\draw [ fill={rgb,255:red,88; green,81; blue,81} , line width=0.8pt ] (18,10.25) rectangle (20.5,10);
\node [font=\large] at (6.75,11) {a};
\node [font=\large] at (6.75,13.25) {a};
\node [font=\large] at (8.25,15) {a};
\node [font=\large] at (10.75,15) {a};
\node [font=\large] at (14,15) {a};
\node [font=\large] at (17.5,15) {a};
\node [font=\large] at (19.75,13) {b};
\node [font=\Large] at (6.75,12) {B};
\node [font=\Large] at (6.75,9.75) {A};
\node [font=\Large] at (6.75,14.75) {C};
\node [font=\Large] at (9.75,15) {D};
\node [font=\Large] at (12,10.5) {F};
\node [font=\Large] at (15.75,15) {G};
\node [font=\Large] at (15.75,10.5) {H};
\node [font=\Large] at (19.25,15) {I};
\node [font=\Large] at (19.75,11.25) {J};
\draw [ color={rgb,255:red,247; green,2; blue,2}, line width=1.1pt, ->, >=Stealth] (12.25,15.75) -- (12.25,14.75);
\node [font=\LARGE, color={rgb,255:red,247; green,2; blue,2}] at (12.5,16) {P};
\node [font=\Large] at (12.75,15) {E};
\end{circuitikz}
    The magnitude \brak{\text{absolute value}} of force \brak{\text{in kN}} in member EF is \dots.
    \brak{round\, off \, to\, one\, decimal\,place}
    \item Consider the linearly elastic plane frame shown in the figure. Members $HF, FK$
and $FG$ are welded together at joint $F$. Joints $K, G$ and $H$ are fixed supports. A
counter-clockwise moment $M$ is applied at joint $F$. Consider flexural rigidity $EI = 10^5\, kN-m^2$ for each member and neglect axial deformations. 
    

 \begin{circuitikz}
\tikzstyle{every node}=[font=\large]
\draw [ fill={rgb,255:red,93; green,86; blue,86} ] (10,14.75) rectangle (11.25,14.5);
\draw [line width=1pt, short] (10.5,14.5) -- (10.75,14.5);
\draw [line width=1.5pt, short] (10.5,14.5) -- (10.5,11.25);
\draw [line width=1.5pt, short] (10.5,11.25) -- (8.25,8.25);
\draw [line width=1.5pt, short] (10.5,11.25) -- (12.75,8.25);
\draw [ fill={rgb,255:red,93; green,86; blue,86} , line width=1.5pt ] (7.5,8.25) rectangle (9,8);
\draw [ fill={rgb,255:red,93; green,86; blue,86} , line width=1.5pt ] (12.25,8.25) rectangle (13.25,8);
\draw [line width=0.5pt, dashed] (10.5,11.25) -- (10.5,8.25);
\draw [line width=0.5pt, short] (10.5,8.25) -- (10.5,6.75);
\draw [line width=0.5pt, short] (12.75,7.75) -- (12.75,6.75);
\draw [line width=0.5pt, short] (8.25,7.75) -- (8.25,6.75);
\draw [line width=0.5pt, <->, >=Stealth] (8.25,7.25) -- (10.5,7.25);
\draw [line width=0.5pt, <->, >=Stealth] (10.5,7.25) -- (12.75,7.25);
\draw [line width=0.5pt, short] (13.25,8.25) -- (14,8.25);
\draw [line width=0.5pt, <->, >=Stealth] (13.5,8.25) -- (13.5,11.25);
\draw [line width=0.5pt, <->, >=Stealth] (13.5,11.25) -- (13.5,14.5);
\draw [line width=0.5pt, short] (13.25,11.25) -- (13.75,11.25);
\draw [line width=0.5pt, short] (13.25,14.5) -- (14,14.5);
\node [font=\large] at (13,13) {$5$ m};
\node [font=\large] at (13,9.75) {$4$ m};
\node [font=\large] at (11.5,7.75) {$3$ m};
\node [font=\large] at (9.5,7.75) {$3$ m};
\node [font=\large] at (10.25,11.5) {F};
\node [font=\large] at (13,8.75) {G};
\node [font=\large] at (9.75,14.75) {H};
\node [font=\large] at (8,8.75) {K};
\node [font=\large] at (10,13) {EI};
\node [font=\large] at (9.75,9.5) {EI};
\node [font=\large] at (11.25,9.5) {EI};
 \draw[->,thick, color= red] (9.5,11) to[bend right=60] (11.5,11.75);
 \node [font=\large, color = red] at (11.75,11.25) {M};
\end{circuitikz}
    If the magnitude \brak{\text{absolute value}} of the support moment at $H$ is $10\, kN-m$, the
    magnitude \brak{\text{absolute value}} of the applied moment $M$ \brak{\text{in kN-m}} to maintain static equilibrium is \dots. \brak{round\, off \, to\, one\, decimal\,place}
    \item Consider a simply supported beam $PQ$ as shown in the figure. A truck having
    $100\, kN$ on the front axle and $200\, kN$ on the rear axle, moves from left to right.
    The spacing between the axles is $3\, m$. The maximum bending moment at point $R$ is \dots kNm. \brak{in\, integer} 
        

\begin{circuitikz}
\tikzstyle{every node}=[font=\Large]
\draw [short] (8,11.75) -- (7.5,11);
\draw [short] (8,11.75) -- (8.5,11);
\draw [short] (7.5,11) -- (8.5,11);
\draw [short] (8,11.75) -- (16.25,11.75);
\draw [short] (16.25,11.75) -- (15.75,11);
\draw [short] (16.25,11.75) -- (16.75,11);
\draw [short] (16.75,11) -- (15.75,11);
\draw  (16,10.875) circle (0.125 cm);
\draw  (16.5,10.875) circle (0.125cm);
\draw [short] (7,11) -- (9,11);
\draw [short] (15.75,10.75) -- (17,10.75);
\draw [short] (16,10.75) -- (15.25,10.75);
\draw [short] (7,11) -- (7.25,10.75);
\draw [short] (7.5,11) -- (7.75,10.75);
\draw [short] (8,11) -- (8.25,10.75);
\draw [short] (8.5,11) -- (8.75,10.75);
\draw [short] (8.75,11) -- (9,10.75);
\draw [short] (15.25,10.75) -- (15.5,10.5);
\draw [short] (15.75,10.75) -- (16,10.5);
\draw [short] (16.25,10.75) -- (16.5,10.5);
\draw [short] (16.75,10.75) -- (17,10.5);
\draw [short] (17,10.75) -- (17.25,10.5);
\draw [short] (16.25,10) -- (16.25,9);
\draw [short] (8,9.75) -- (8,10);
\draw [short] (8,10) -- (8,9);
\draw [short] (10.25,9) -- (10.25,10);
\draw [dashed] (10.25,10) -- (10.25,13);
\draw [<->, >=Stealth] (8,9.5) -- (10.25,9.5);
\draw [<->, >=Stealth] (10.25,9.5) -- (16.25,9.5);
\node [font=\Large] at (8,12.25) {P};
\node [font=\Large] at (16.75,12) {Q};
\node [font=\Large] at (10,12.25) {R};
\node [font=\Large] at (9,9.75) {$1$ m};
\node [font=\Large] at (13,9.75) {$4$ m};
\end{circuitikz}

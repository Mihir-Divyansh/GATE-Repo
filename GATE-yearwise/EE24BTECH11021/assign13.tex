 \iffalse
\chapter{2017}
\author{EE24BTECH11021 - Eshan Ray}
\section{ma}
\fi

    \item For $n\in Z$, define $c_n=\frac{1}{\sqrt{2\pi}}\int_{-\pi}^{\pi}e^{i\brak{n-i}x}\,dx,$ where $i^2=-1$. Then $\sum_{n\in Z}\abs{c_n}^2$ equals 
    \begin{enumerate}
        \item  $\cosh\brak{\pi}$
        \item $\sinh\brak{\pi}$
        \item $\cosh\brak{2\pi}$
        \item $\sinh\brak{2\pi}$
    \end{enumerate}
    \item If the fourth order divided difference of $f\brak{x}=\alpha x^4+5x^3+3x+2,\alpha\in R,$ at the points $0.1,0.2,0.3,0.4,0.5$ is $5,$ then $\alpha$ equals \dots
    \item If the quadrature rule $\int_{0}^{2}f\brak{x}\,dx\approx c_1 f\brak{0}+3f\brak{c_2},$ where $c_1,c_2\in R,$ is exact for all polynomials of degree $\leq 1,$ then $c_1+3c_2$ equals \dots
    \item If $u\brak{x,y}=1+x+y+f\brak{xy},$ where $f\colon R^2\to R$ is a differential function, then $u$ satisfies
    \begin{enumerate}
        \item $x\frac{\partial u}{\partial x}-y\frac{\partial u}{\partial y}=x^2-y^2$
        \item $x\frac{\partial u}{\partial x}-y\frac{\partial u}{\partial y}=0$
        \item $x\frac{\partial u}{\partial x}-y\frac{\partial u}{\partial y}=x-y$
        \item $y\frac{\partial u}{\partial x}-x\frac{\partial u}{\partial y}=x-y$
    \end{enumerate}
    \item The partial differential equation $x\frac{\partial^2 u}{\partial x^2}+\brak{x-y}\frac{\partial^2 u}{\partial x\partial y}-y\frac{\partial^2 u}{\partial y^2}+\frac{1}{4}\brak{\frac{\partial u}{\partial y}-\frac{\partial u}{\partial x}}=0$ is
    \begin{enumerate}
        \item hyperbolic along the line $x+y=0$
        \item elliptic along the line $x-y=0$
        \item elliptic along the line $x+y=0$
        \item parabolic along the line $x+y=0$
    \end{enumerate}
    \item Let $X$ and $Y$ be topological spaces and let $f\colon X\to Y$ be a continuous surjective function. Which one of the following statements is TRUE?\
    \begin{enumerate}
        \item If $X$ is separable, then $Y$ is separable 
        \item If $X$ is first countable, then $Y$ is first countable 
        \item If $X$ is Hausdorff, then $Y$ is Hausdorff
        \item If $X$ is regular, then $Y$ is regular
    \end{enumerate}
    \item Consider the topology $\mathcal{T}=\cbrak{U\subseteq Z\colon Z\backslash U\,is\,finite\,\,or\,0\neq U}$ on $Z$. Then, the topological space $\brak{Z,\mathcal{T}}$ is
    \begin{enumerate}
        \item compact but NOT connected
        \item connected but NOT compact
        \item both compact and connected
        \item neither compact nor connected
    \end{enumerate}
    \item Let $F\brak{x}$ be the distribution function of a random variable $X$. Consider the functions $\colon$\\
    $G_1\brak{x}=\brak{F\brak{x}}^3,x\in R$\\
    $G_2\brak{x}=1-\brak{1-F\brak{x}}^5,x\in R$\\ \\
    Which of the above functions are distribution functions?
    \begin{enumerate}
        \item Neither $G_1$ nor $G_2$
        \item Only $G_1$ 
        \item Only $G_2$
        \item Both $G_1$ and $G_2$
    \end{enumerate}
    \item Let $X_1,X_2,\dots,X_n\,\brak{n\geq 2}$ be independent and identically distributed random variables with finite variance $\sigma^2$ and let $\Bar{X}=\frac{1}{n}\sum_{i=1}^{n}X_i$. Then the covariance between $\Bar{X}$ and $X_1-\Bar{X}$ IS
    \begin{enumerate}
        \item $0$
        \item $-\sigma^2$
        \item $\frac{-\sigma^2}{n}$
        \item $\frac{\sigma^2}{n}$
    \end{enumerate}
    \item Let $X_1,X_2,\dots,X_n\,\brak{n\geq2}$ be a random sample from a $N\brak{\mu,\sigma^2}$ population, where $\sigma^2=144.$ The smallest $n$ such that the length of the shortest $95\%$ confidence interval for $\mu$ will not exceed $10$ is \dots
    \item Consider the linear programming problem \brak{LPP}$\colon$
    $  \begin{array}{lc}
\text{Maximize} & 4x_1+6x_2 \\
\text{subject to} & x_1+x_2\leq 8, \\
& 2x_1+3x_2\geq 18, \\
& x_1\geq6,x_2 \,is\,unrestricted\,in\,sign.
\end{array} $ \\
Then the LPP has
    \begin{enumerate}
        \item no optimal solution 
        \item only one basic feasible solution and that is optimal 
        \item more than one basic feasible solution and a unique optimal solution
        \item infinitely many optimal solutions
    \end{enumerate}
    \item For a linear programming problem \brak{LPP} and its dual, which one of the following is NOT TRUE?
    \begin{enumerate}
        \item The dual of the dual is primal
        \item If the primal LPP has an unbounded objective function, then the dual LPP is infeasible
        \item If the primal LPP is infeasible, then the dual LPP has an unbounded objective function
        \item If the primal LPP has a finite optimal solution, then the dual LPP also has a finite optimal solution
    \end{enumerate}
    \item If $U$ and $V$ are the null spaces of $\myvec{1&1&0&0\\0&0&1&1}$ and $\myvec{1&2&3&2\\0&1&2&1}$, respectively, then the dimension of the subspace $U+V$ equals \dots

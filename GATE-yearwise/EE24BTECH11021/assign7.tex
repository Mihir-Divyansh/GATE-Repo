 \iffalse
\chapter{2012}
\author{EE24BTECH11021 - Eshan Ray}
\section{ae}
\fi
    \item The constraint $A^2=A$ on any square matrix $A$ is satisfied for
    \begin{enumerate}
        \item The identity matrix only.
        \item the null matrix only.
        \item both the identity matrix and null matrix.
        \item no square matrix $A$
    \end{enumerate}
    \item The general solution of the differential equation $\frac{d^2y}{dt^2}+\frac{dy}{dt}-2y=0$ is  
    \begin{enumerate}
        \item $Ae^{-t}+Be^{2t}$
        \item $Ae^{-2t}+Be^{-t}$
        \item $Ae^{-2t}+Be^{t}$
        \item $Ae^{t}+Be^{2t}$
    \end{enumerate}
    \item An aircraft in trimmed condition has zero pitching moment at 
    \begin{enumerate}
        \item its aerodynamic centre.
        \item its centre of gravity.
        \item $25\%$ of its mean aerodynamic chord.
        \item $50\%$ od its wing root chord.
    \end{enumerate}
    \item In an aircraft , constant roll rate can be produced using ailerons by applying
    \begin{enumerate}
        \item a step input
        \item a ramp input
        \item a sinusoidal input
        \item an impulse input
    \end{enumerate}
    \item For a symmetric airfoil,the lift coefficient for zero degree angle of attack is
    \begin{enumerate}
        \item $-1.0$
        \item $0.0$
        \item $0.5$
        \item $1.0$
    \end{enumerate}
    \item The critical Mach number of an airfoil is attained when
    \begin{enumerate}
        \item the freestream Mach number is sonic.
        \item the freestream Mach number is supersonic.
        \item the Mach number somewhere on the airfoil is unity.
        \item the Mach number everywhere on the airfoil is supersonic.
    \end{enumerate}
    \item The shadowgraph flow visualization technique depends on
        \begin{enumerate}
            \item the variation of the value of density in the flow
            \item the first derivative of density with respect to spacial coordinate
            \item the second derivative of density with respect to spacial coordinate
            \item the third derivative of density with respect to spacial coordinate
        \end{enumerate}
    \item The Hohmann ellipse used as earth-Mars transfer orbit has
    \begin{enumerate}
        \item apogee at earth and perigee at Mars
        \item both apogee and perigee at earth
        \item apogee at Mars and perigee at earth
        \item both apogee and perigee at Mars
    \end{enumerate}
    \item The governing equation for the static transverse deflection of beam under an uniformly distributed load, according to Euler-Bernoulli \brak{engineering} beam theory, is a 
    \begin{enumerate}
        \item $2^{nd}$ order linear homogeneous partial differential equation
        \item $4^{th}$ order linear non-homogeneous ordinary differential equation
        \item $2^{nd}$ order linear non-homogeneous ordinary differential equation
        \item $4^{th}$ order nonlinear homogeneous ordinary differential equation
    \end{enumerate}
    \item The Poisson's ratio, $\nu$ of most aircraft grade metallic alloys has values in the range $\colon$
    \begin{enumerate}
        \item $-1\leq\nu\leq0$
        \item $0\leq\nu0.2$
        \item $0.2\leq\nu0.4$
        \item $0.4\leq\nu0.5$
    \end{enumerate}
    \item The value of $k$ for which the system of equations $x+2y+kz=1;\,2x++ky+8z=3$ hsa no solution is
        \begin{enumerate}
            \item $0$
            \item $2$
            \item $4$
            \item $8$
        \end{enumerate}
    \item If $u\brak{t}$ is a unit step function, the solution of the differential equation $m\frac{d^2y}{dt^2}+kx=u\brak{t}$ in Laplace domain is
            \begin{enumerate}
                \item $\frac{1}{s\brak{ms^2+k}}$
                \item $\frac{1}{ms^2+k}$
                \item $\frac{s}{ms^2+k}$
                \item $\frac{1}{s^2\brak{ms^2+k}}$
            \end{enumerate}
    \item The general solution of the differential equation $\frac{dy}{dx}-2\sqrt{y}=0$ is
    \begin{enumerate}
        \item $y-\sqrt{x}+C=0$
        \item $y-x+C=0$
        \item $\sqrt{y}-\sqrt{x}+C=0$
        \item $\sqrt{y}-x+C=0$
    \end{enumerate}

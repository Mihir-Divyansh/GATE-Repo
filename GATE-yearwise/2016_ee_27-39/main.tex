\iffalse
\chapter{2016}
\author{EE24BTECH11005}
\section{ee}
\fi
	\item In the portion fo a circuit shown, if the ehat generated in $5\Omega$ resistance is $10$ calories per second, the heat generated by the $4\Omega$ resistance, in calories per second is, \rule{2cm}{0.2pt}
		\begin{figure}[H]
			\centering
			\begin{circuitikz}
				\tikzstyle{every node}=[font=\normalsize]
				\draw (5,16) to[R,l={ \normalsize 4$\Omega$}] (7.5,16);
				\draw (7.5,16) to[R,l={ \normalsize 6$\Omega$ }] (10,16);
				\draw (5,14.75) to[R,l={ \normalsize 5$\Omega$}] (10,14.75);
				\draw (10,14.75) to[short] (10,16);
				\draw (5,14.75) to[short] (5,16);
				\draw (5,15.5) to[short] (2.75,15.5);
				\draw (10,15.5) to[short] (12,15.5);
			\end{circuitikz}
		\end{figure}

	\item In the given circuit, the current supplied by the battery, in ampere is,\rule{2cm}{0.2pt}
		\begin{figure}[H]
			\centering
			\begin{circuitikz}
				\tikzstyle{every node}=[font=\normalsize]
				\draw (5,14.75) to[R,l={ \normalsize 1$\Omega$}] (7.5,14.75);
				\draw (7.5,14.75) to[american controlled current source,l={ \normalsize $I_2$}] (7.5,12.25);
				\draw (7.5,14.75) to[R,l={ \normalsize 1$\Omega$}] (10,14.75);
				\draw (10,14.75) to[R,l={ \normalsize 1$\Omega$}] (10,12.25);
				\draw (5,14.75) to[battery2,l=$1V$] (5,12.25);
				\draw (5,12.25) to[short] (10,12.25);
				\node [font=\normalsize] at (9.75,15) {$I_2$};
				\draw [->, >=Stealth] (9.25,14.75) -- (9.75,14.75);
				\draw [->, >=Stealth] (5,14.75) -- (5.5,14.75);
				\node [font=\normalsize] at (5.5,15) {$I_1$};
			\end{circuitikz}
		\end{figure}
	\item In a 100 bus power system, there are $10$ generators. In a particular iteration of Newton Raphson load flow technique $\brak{\text{in polar coordinates}}$, two of the $PV$ buses are converted to $PQ$ type. In this iteration,
		\begin{enumerate}
			\item  the number of unknown voltage angles increases by two and the number of unknown voltage magnitudes increases by two.
			\item  the number of unknown voltage angles remains unchanged and the number of unknown voltage magnitudes increases by two.
			\item the number of unknown voltage angles increases by two and the number of unknown voltage magnitudes decreases by two.
			\item the number of unknown voltage angles remains unchanged and the number of unknown voltage magnitudes decreases by two.
		\end{enumerate}

	\item The magnitude of three-phase fault currents at buses $A$ and $B$ of a power system are $10 pu$ and $8 pu$, respectively. Neglect all resistances in the system and consider the pre-fault system to be unloaded. The pre-fault voltage at all buses in the system is $1.0 pu$. The voltage magnitude at bus B during a three-phase fault at bus $A$ is $0.8 pu$. The voltage magnitude at bus $A$ during a three-phase fault at bus $B$, in $pu$, is \rule{2cm}{0.2pt}
	\item Consider a system consisting of a synchronous generator working at a lagging power factor, a synchronous motor working at an overexcited condition and a directly grid-connected induction generator. Consider capacitive VAr to be a source and inductive VAr to be a sink of reactive power. Which one of the following statements is TRUE? 

		\begin{enumerate}
			\item Synchronous motor and synchronous generator are sources and induction generator is a sink of reactive power.
			\item Synchronous motor and induction generator are sources and synchronous generator is a sink of reactive power.
			\item Synchronous motor is a source and induction generator and synchronous generator are sinks of reactive power.
			\item A $4$-pole, lap-connected, separately excited dc motor is drawing a steady current of $40$ A while running at $600 rpm$. A good approximation for the waveshape of the current in an armature conductor of the motor is given by
		\end{enumerate}

	\item A steady dc current of $100 A$ is flowing through a power module $\brak{S, D}$ as shown in Figure $\brak{a}$. The V-I characteristics of the IGBT $\brak{S}$ and the diode $\brak{D}$ are shown in Figures $\brak{b}$ and $\brak{c}$, respectively. The conduction power loss in the power module $\brak{C,D}$, in watts, is \rule{2cm}{0.2pt}
		\begin{multicols}{2}
			\begin{figure}[H]
				\centering
				\begin{circuitikz}
					\tikzstyle{every node}=[font=\normalsize]
					%\draw (6.25,15.5) to[short] (6.5,15.5);
					%\draw (6.625,15) to[short] (6.625,16);
					\draw (6.375,16) to[short] (3.75,16);
					\draw (3.75,16) to[american voltage source,l={ \normalsize $V_g$}] (3.75,9.75);
					\draw (11.25,9.75) to[D,l={ \normalsize $D$}] (11.25,16);
					\draw (11.25,16) to[L,l={ \normalsize $V_l$} ] (15,16);
					\draw (15,16) to[C,l={ \normalsize $C$}] (15,9.75);
					\draw (17.5,16) to[R,l={ \normalsize $R$}] (17.5,9.75);
					\draw (15,16) to[short] (17.5,16);
					\draw (3.75,9.75) to[short] (17.5,9.75);
					\draw [short] (6.25,15) -- (6.5,15);
					\draw [short] (6.75,15) -- (7,15);
					\draw [short] (7.25,15) -- (7.5,15);
					\draw [short] (6.25,14.75) -- (7.5,14.75);
					\draw [short] (7.5,14.75) -- (7.5,14.5);
					\draw [short] (6.375,15) -- (6.375,16);
					\draw [short] (7.375,15) -- (7.375,16);
					\draw [short] (7.25,16) -- (11.25,16);
					\draw [->, >=Stealth] (6.875,16) -- (6.875,15);
					\draw [short] (6.875,16) -- (7.5,16);
					\node [font=\normalsize] at (16.75,12.75) {$V_0$};
					\draw [->, >=Stealth] (16.75,13.25) -- (16.75,15.5);
					\draw [->, >=Stealth] (16.75,12.5) -- (16.75,10);
					\draw [->, >=Stealth] (13.5,16.75) -- (15,16.75);
					\draw [->, >=Stealth] (12.75,16.75) -- (11.25,16.75);
				\end{circuitikz}
			\end{figure}
			\columnbreak
			\begin{figure}[H]
				\centering
				\caption{}
				\begin{circuitikz}
					\tikzstyle{every node}=[font=\large]
					\draw [->, >=Stealth] (3.75,11) -- (11.75,11);
					\draw [->, >=Stealth] (5,6) -- (5,16);
					\draw [line width=1pt, short] (5,14.75) -- (7.5,14.75);
					\draw [line width=1pt, short] (7.5,14.75) -- (7.5,8.5);
					\draw [line width=1pt, short] (7.5,8.5) -- (10,8.5);
					\draw [line width=1pt, short] (10,8.5) -- (10,14.75);
					\draw [line width=1pt, short] (10,14.75) -- (11.25,14.75);
					\node [font=\normalsize] at (4.75,11.25) {0};
					\node [font=\normalsize] at (12,11.25) {t};
					\node [font=\normalsize] at (4.75,16.25) {$V_L$};
					\node [font=\large] at (4.25,14.75) {$30V$};
					\node [font=\large] at (4.25,8.5) {$-20V$};
					\node [font=\large] at (7.5,7.25) {$T_S$};
					\node [font=\large] at (6,12) {$T_{ON}$};
					\node [font=\large] at (8.5,12) {$T_{OFF}$};
					\draw [<->, >=Stealth] (5.25,11.5) -- (7.25,11.5);
					\draw [<->, >=Stealth] (7.75,11.5) -- (9.75,11.5);
					\draw [<->, >=Stealth] (5,7.75) -- (10,7.75);
					\draw [dashed] (10,7.25) -- (10,8.75);
					\draw [dashed] (7.5,8.5) -- (5,8.5);
				\end{circuitikz}
				\label{figure(b)}
			\end{figure}
		\end{multicols}
	\item A steady dc current of $100A$ is flowing through a power module $\brak{\text{S,D}}$ as shown in $\brak{1}$. The $V-I$ characteristics of the IGBT $\brak{\text{S}}$ and the diode $\brak{\text{D}}$ are shown in Figures $\brak{2}$, $\brak{3}$ respectively. The conduction power loss in the powewr module $\brak{\text{S,D}}$ in watts is \rule{2cm}{0.2pt}
		\begin{multicols}{3}
			\begin{figure}[H]
				\centering
				\caption{}
				\begin{circuitikz}
					\tikzstyle{every node}=[font=\large]
					\draw (6.25,12.25) to[short] (6.25,13.25);
					\draw [ line width=2pt](6.5,12.25) to[short] (6.5,13.25);
					\draw (6.5,12.75) to[short] (7,13.25);
					\draw (7,13.25) to[short] (7,15.25);
					\draw [->, >=Stealth] (6.5,12.5) -- (7,12);
					\draw [short] (7,12) -- (7,9.75);
					\draw [short] (7,14) -- (7.5,14);
					\draw (7.5,11) to[D] (7.5,14);
					\draw (7,11) to[short] (7.5,11);
					\draw  (7,12.5) ellipse (1.5cm and 2.5cm);
					\draw [->, >=Stealth] (7.5,8.5) -- (7.5,9.75);
					\node [font=\large] at (8,9) {100A};
				\end{circuitikz}
			\end{figure}
			\columnbreak
			\begin{figure}[H]
				\centering
				\caption{}
				\begin{circuitikz}
					\tikzstyle{every node}=[font=\large]
					\draw [->, >=Stealth] (3.75,14.75) -- (11.5,14.75);
					\draw [->, >=Stealth] (5,13.5) -- (5,19.75);
					\draw [line width=2pt, short] (5,14.75) -- (6.25,14.75);
					\draw [line width=2pt, short] (6.25,14.75) -- (7.5,19.5);
					\draw [dashed] (6.25,14.25) -- (6.25,15.5);
					\node [font=\large] at (11,14.25) {$V_S(Volt)$};
					\node [font=\large] at (4.75,20.25) {$I_S(A)$};
					\node [font=\large] at (8.5,18) {$\frac{dV}{dI}=0.02\Omega$};
					\node [font=\large] at (5.75,16.25) {$V_0=1V$};
				\end{circuitikz}
			\end{figure}
			\columnbreak
			\begin{figure}[H]
				\centering
				\begin{circuitikz}
					\tikzstyle{every node}=[font=\large]
					\draw [->, >=Stealth] (3.75,14.75) -- (11.5,14.75);
					\draw [->, >=Stealth] (5,13.5) -- (5,19.75);
					\draw [line width=2pt, short] (5,14.75) -- (6.25,14.75);
					\draw [line width=2pt, short] (6.25,14.75) -- (7.5,19.5);
					\draw [dashed] (6.25,14.25) -- (6.25,15.5);
					\node [font=\large] at (11,14.25) {$V_S(Volt)$};
					\node [font=\large] at (4.75,20.25) {$I_D(A)$};
					\node [font=\large] at (8.5,18) {$\frac{dV}{dI}=0.01\Omega$};
					\node [font=\large] at (5.75,16.25) {$V_0=0.7V$};
				\end{circuitikz}
				\caption{}
			\end{figure}
		\end{multicols}
	\item A $4$-pole, lap-connected, separately excited dc motor is drawing a steady current of $40$ A while running at $600 rpm$. A good approximation for the waveshape of the current in an armature conductor of the motor is given by
		\begin{enumerate}
				\begin{multicols}{2}

				\item . \begin{figure}[H]
						\centering
						\begin{circuitikz}
							\tikzstyle{every node}=[font=\large]
							\draw [->, >=Stealth] (3.75,13.5) -- (11.25,13.5);
							\draw [->, >=Stealth] (5,12.25) -- (5,18.5);
							\draw [->, >=Stealth] (3.75,14.75) -- (3.75,16);
							\node [font=\normalsize] at (3.75,16.5) {i};
							\node [font=\normalsize] at (11.5,13.5) {t};
							\draw [short] (5,16) -- (10.75,16);
							\node [font=\large] at (11.5,16) {10A};
						\end{circuitikz}
				\end{figure}

				\columnbreak

			\item . \begin{figure}[H]
					\centering
					\begin{circuitikz}
						\tikzstyle{every node}=[font=\large]
						\draw [->, >=Stealth] (3.75,13.5) -- (11.25,13.5);
						\draw [->, >=Stealth] (5,12.25) -- (5,18.5);
						\draw [->, >=Stealth] (3.75,14.75) -- (3.75,16);
						\node [font=\normalsize] at (3.75,16.5) {i};
						\node [font=\normalsize] at (11.5,13.5) {t};
						\draw [short] (5,18) -- (10.75,18);
						\node [font=\large] at (11.5,18) {40A};
					\end{circuitikz}

			\end{figure}

				\end{multicols}
				\begin{multicols}{2}

				\item . \begin{figure}[H]
						\centering
						\begin{circuitikz}
							\tikzstyle{every node}=[font=\normalsize]
							\draw [->, >=Stealth] (3.75,13.5) -- (12.5,13.5);
							\draw [->, >=Stealth] (5,12.25) -- (5,18.5);
							\draw [->, >=Stealth] (3.75,14.75) -- (3.75,16);
							\node [font=\normalsize] at (3.75,16.5) {i};
							\node [font=\normalsize] at (13,13.5) {t};
							\draw [short] (5,13.5) -- (5.75,16);
							\draw [short] (5.75,16) -- (7.5,16);
							\draw [short] (7.5,16) -- (8.75,12.25);
							\draw [short] (8.75,12.25) -- (10.75,12.25);
							\draw [short] (10.75,12.25) -- (12,16);
							\node [font=\normalsize] at (6.5,16.5) {10A};
							\node [font=\normalsize] at (9.75,11.75) {-10A};
							\node [font=\normalsize] at (9.5,14.75) {T=25ms};
							\draw [<->, >=Stealth] (8.5,14.25) -- (11,14.25);
							\node [font=\normalsize] at (6.5,12.25) {T=25ms};
							\draw [<->, >=Stealth] (5.25,12.75) -- (8.25,12.75);
						\end{circuitikz}
				\end{figure}

				\columnbreak

			\item . \begin{figure}[H]
					\centering
					\begin{circuitikz}
						\tikzstyle{every node}=[font=\normalsize]
						\draw [->, >=Stealth] (3.75,13.5) -- (12.5,13.5);
						\draw [->, >=Stealth] (5,12.25) -- (5,18.5);
						\draw [->, >=Stealth] (3.75,14.75) -- (3.75,16);
						\node [font=\normalsize] at (3.75,16.5) {i};
						\node [font=\normalsize] at (13,13.5) {t};
						\node [font=\normalsize] at (6.5,16.5) {10A};
						\node [font=\normalsize] at (9.75,11.75) {-10A};
						\node [font=\normalsize] at (9.5,14.75) {T=25ms};
						\draw [<->, >=Stealth] (8.5,14.25) -- (11.25,14.25);
						\node [font=\normalsize] at (6.5,12.25) {T=25ms};
						\draw [<->, >=Stealth] (5.25,12.75) -- (8.25,12.75);
						\draw [short] (5,13.5) .. controls (6,16.25) and (7,17.25) .. (8.25,13.5);
						\draw [short] (8.25,13.5) .. controls (9.75,9.25) and (11,11.25) .. (11.5,13.5);
					\end{circuitikz}
			\end{figure}

				\end{multicols}
		\end{enumerate}


	\item If an ideal transformer has an inductive load element at port $2$ shown in the ifgure below, the equivalent inductance at port $1$ is
		\begin{figure}[H]
			\centering
			\begin{circuitikz}
				\tikzstyle{every node}=[font=\normalsize]
				\draw (5,18.5) to[L ] (5,13.5);
				\draw (7.5,18.5) to[L ] (7.5,13.5);
				\draw (13.75,18.5) to[L,l={ \normalsize L} ] (13.75,13.5);
				\draw (7.5,18.5) to[short] (13.75,18.5);
				\draw (7.5,13.5) to[short] (13.75,13.5);
				\draw (2.5,18.5) to[short] (5,18.5);
				\draw (2.5,13.5) to[short] (5,13.5);
				\draw [ dashed] (4,19) rectangle  (8.5,13);
				\draw [->, >=Stealth] (1.25,16.75) -- (2.5,16.75);
				\draw [->, >=Stealth] (11.25,16.5) -- (10.5,16.5);
				\draw [short] (1.25,16.75) -- (1.25,12.25);
				\draw [short] (11.25,16.5) -- (11.25,12.25);
				\node [font=\normalsize] at (1,11.75) {Port 1};
				\node [font=\normalsize] at (11.25,11.75) {Port 2};
				\node [font=\normalsize] at (6.25,17.75) {n:1};
			\end{circuitikz}
		\end{figure}
		\begin{enumerate}
			\item $nL$ 
			\item $n^2L$ 
			\item $\frac{n}{L}$ 
			\item $\frac{n^2}{L}$
		\end{enumerate}
	\item Candidates were asked to come to an interview with $3$ pens each. Black, blue green and red were the permitted pen colors that the cnadidate could bring. The probability that a condidate comes with all $3$ pens having the same color is \rule{2cm}{0.2pt}

	\item Let $S=\sum_{n=0}^{\infty}n\alpha^n$ where $\abs{\alpha}<1$. The value of $\alpha$ in the range $0<\alpha<1$, such that $S=2\alpha$ is \rule{2cm}{0.2pt} 
	\item let the eigenvalues of a $2\times2$ matrix $A$ be $1,-2$ with eigenvectors $x_1,x_2$ respectively. Then the eigenvalues and eigenvectors of the matrix $A^2-3A+4I$ would, respectively be 
		\begin{enumerate}
			\item $2,14;x_1,x_2$
			\item $2,14;x_1+x_2;x_1-x_2$
			\item $2,0;x_1,x_2$ 
			\item $2,0;x_1+x_2;x_1-x_2$
		\end{enumerate}
	\item Let $A$ be a $4\times3$ real matrix with rank $2$. Which  of the following statement is TRUE? 
		\begin{enumerate}
			\item Rank of $A^{\top}A$ is less than $2$ 
			\item Rank of $A^{\top}A$ is equal to $2$
			\item Rank of $A^{\top}A$ is greater than $2$ 
			\item Rank of $A^{\top}A$ is between $1$ and $3$
		\end{enumerate}



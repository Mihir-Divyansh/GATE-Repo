\iffalse
\title{2021-ME-53-65}
\author{EE24BTECH11010 - Balaji B}
\section{ee}
\chapter{2021}
\fi

\item Consider a single machine workstation to which jobs arrive according to a
Poisson distribution with a mean arrival rate of 12 jobs/hour. The process
time of the workstation is exponentially distributed with a mean of 4
minutes. The expected number of jobs at the workstation at any given
point of time is \rule{2cm}{0.4pt} (round off to nearest integer).

\hfill [2021 ME]
    \item An uninsulated cylindrical wire of radius $1.0 mm$ produces electric heating
at the rate of $5.0 W/m$. The temperature of the surface of the wire is $75 ^{\degree} C$
when placed in air at $25 ^{\degree} C$. When the wire is coated with PVC of thickness
$1.0 mm$, the temperature of the surface of the wire reduces to $55 ^{\degree} C$.
Assume that the heat generation rate from the wire and the convective heat
transfer coefficient are same for both uninsulated wire and the coated wire.
The thermal conductivity of PVC is \rule{2cm}{0.4pt} $W/m\cdot K$(round off to two decimal places). 

\hfill [2021 ME]
\item A solid sphere of radius $10 mm$ is placed at the centroid of a hollow cubical
enclosure of side length $30 mm$. The outer surface of the sphere is denoted
by 1 and the inner surface of the cube is denoted by 2. The view factor $F_{22}$
for radiation heat transfer is \rule{2cm}{0.4pt} (round off to two decimal places).

\hfill [2021 ME]
\item Consider a steam power plant operating on an ideal reheat Rankine cycle.
The work input to the pump is $20 kJ/kg$. The work output from the high
pressure turbine is $750 kJ/kg$. The work output from the low pressure
turbine is $1500 kJ/kg$. The thermal efficiency of the cycle is $50 \% $. The
enthalpy of saturated liquid and saturated vapour at condenser pressure
are $200 kJ/kg$ and $2600 kJ/kg$, respectively. The quality of steam at the exit
of the low pressure turbine is \rule{2cm}{0.4pt} $\%$ (round off to the nearest integer). 

\hfill [2021 ME] 
\item In the vicinity of the triple point, the equation of liquid-vapour boundary
in the $P - T$ phase diagram for ammonia is $\log P = 24.38 - 3063/T$,
where $P$ is pressure (in $Pa$) and $T$ is temperature (in $K$). Similarly, the
solid-vapour boundary is given by $\log P = 27.92 - 3754/T$. The
temperature at the triple point is \rule{2cm}{0.4pt}$K$ (round off to the nearest integer).

\hfill [2021 ME]
\item A cylindrical jet of water (density = $1000 kg/m^3$) impinges at the center of a
flat, circular plate and spreads radially outwards, as shown in the figure.
The plate is resting on a linear spring with a spring constant $k = 1 kN/m.$
The incoming jet diameter is $D = 1 cm.$
\begin{figure}[H]
\centering
\resizebox{5cm}{!}{%
\begin{circuitikz}
\tikzstyle{every node}=[font=\Large]
\draw [short] (1.25,12.75) .. controls (2,13) and (2,13) .. (2,13.75);
\draw [short] (2.5,13.75) .. controls (2.5,12.5) and (2.75,12.75) .. (3.5,12.75);
\draw [short] (2,13.75) -- (2,16.5);
\draw [short] (2.5,13.75) -- (2.5,16.5);
\draw [short] (3.5,12.75) -- (4.5,12.75);
\draw [line width=1.1pt, short] (0,12.25) -- (4.5,12.25);
\draw [line width=1pt, short] (0,12) -- (2,12);
\draw [short] (0,12.25) -- (0,12);
\draw [line width=1.1pt, short] (2,12) -- (4.5,12);
\draw [short] (4.5,12.25) -- (4.5,12);
\draw (2.25,12) to[R] (2.25,10.25);
\draw [line width=1.1pt, short] (0,10.25) -- (4.75,10.25);
\draw [line width=0.8pt, ->, >=Stealth] (3.25,15.25) -- (2.5,15.25);
\draw [line width=0.8pt, ->, >=Stealth] (1.25,15.25) -- (2,15.25);
\draw [line width=0.8pt, ->, >=Stealth] (2.25,14) -- (2.25,13);
\draw [line width=0.8pt, ->, >=Stealth] (3.5,12.5) -- (4.5,12.5);
\draw [line width=0.8pt, ->, >=Stealth] (0.5,12.5) -- (-0.5,12.5);
\node [font=\LARGE] at (3.75,15.25) {$D$};
\node [font=\Large] at (4.75,11.25) {spring constant, $k$};
\draw [line width=0.8pt, short] (0,10.25) -- (0.25,10);
\draw [line width=0.8pt, short] (0.25,10.25) -- (0.5,10);
\draw [line width=0.8pt, short] (0.75,10.25) -- (1,10);
\draw [line width=0.8pt, short] (0.5,10.25) -- (0.75,10);
\draw [line width=0.8pt, short] (1.25,10.25) -- (1.5,10);
\draw [line width=0.8pt, short] (1,10.25) -- (1.25,10);
\draw [line width=0.8pt, short] (1.5,10.25) -- (1.75,10);
\draw [line width=0.8pt, short] (1.75,10.25) -- (2,10);
\draw [line width=0.8pt, short] (2.25,10.25) -- (2.5,10);
\draw [line width=0.8pt, short] (2,10.25) -- (2.25,10);
\draw [line width=0.8pt, short] (2.5,10.25) -- (2.75,10);
\draw [line width=0.8pt, short] (2.75,10.25) -- (3,10);
\draw [line width=0.8pt, short] (3.25,10.25) -- (3.5,10);
\draw [line width=0.8pt, short] (3,10.25) -- (3.25,10);
\draw [line width=0.8pt, short] (3.5,10.25) -- (3.75,10);
\draw [line width=0.8pt, short] (3.75,10.25) -- (4,10);
\draw [line width=0.8pt, short] (4,10.25) -- (4.25,10);
\draw [line width=0.8pt, short] (4.25,10.25) -- (4.5,10);
\draw [line width=0.8pt, short] (4.5,10.25) -- (4.75,10);
\end{circuitikz}
}%

\label{fig:my_label}
\end{figure}
If the spring shows a steady deflection of $1 cm$ upon impingement of jet,
then the velocity of the incoming jet is \rule{2cm}{0.4pt}$m/s$ (round off to one decimal places).

\hfill [2021 ME]
\item A single jet Pelton wheel operates at 300 rpm. The mean diameter of the
wheel is $2 m$. Operating head and dimensions of jet are such that water
comes out of the jet with a velocity of $40 m/s$ and flow rate of $5 m^3
/s$. The jet
is deflected by the bucket at an angle of $165^{\degree}.$ Neglecting all losses, the
power developed by the Pelton wheel is \rule{2cm}{0.4pt} $MW$ (round off to two decimal places)

\hfill [2021 ME]
\item An air-conditioning system provides a continuous flow of air to a room using
an intake duct and an exit duct, as shown in the figure. To maintain the
quality of the indoor air, the intake duct supplies a mixture of fresh air with
a cold air stream. The two streams are mixed in an insulated mixing chamber
located upstream of the intake duct. Cold air enters the mixing chamber at
$5 ^{\degree} C$, $105 kPa$ with a volume flow rate of $1.25 m^3/s$ during steady state
operation. Fresh air enters the mixing chamber at $34 ^{\degree} C$ and $105 kPa$. The
mass flow rate of the fresh air is 1.6 times of the cold air stream. Air leaves
the room through the exit duct at $24 ^{\degree} C.$
\begin{figure}[H]
\centering
\resizebox{5cm}{!}{%
\begin{circuitikz}
\tikzstyle{every node}=[font=\normalsize]
\draw [short] (3.5,14.5) -- (3.5,11.5);
\draw [short] (3.5,14.5) -- (6.5,14.5);
\draw [short] (6.5,14.5) -- (6.5,11.5);
\draw [short] (3.5,11.5) -- (6.5,11.5);
\draw [short] (3.5,13.25) -- (2.5,13.25);
\draw [short] (3.5,12.75) -- (2.5,12.75);
\draw [short] (2.5,14) -- (2.5,12);
\draw [short] (2.5,12) -- (1.75,12);
\draw [short] (1.75,12) -- (1.75,14);
\draw [short] (1.75,14) -- (2.5,14);
\draw [short] (1.75,13.75) -- (0.5,14.5);
\draw [short] (0.5,14.5) -- (0.5,13.75);
\draw [short] (0.5,13.75) -- (1.75,13);
\draw [short] (1.75,13) -- (0.5,12);
\draw [short] (1.75,12.25) -- (0.5,11.25);
\draw [short] (0.5,11.25) -- (0.5,12);
\draw [short] (6.5,13.25) -- (7,13.25);
\draw [short] (7,13.25) -- (7,12.75);
\draw [short] (6.5,12.75) -- (7,12.75);
\node [font=\normalsize] at (0.75,14.75) {Cold Air};
\node [font=\normalsize] at (0.75,10.75) {Fresh Air};
\node [font=\normalsize] at (3,14) {Intake};
\node [font=\normalsize] at (3,13.5) {Duct};
\node [font=\normalsize] at (2,11.75) {Mixing };
\node [font=\normalsize] at (2,11.5) {Chamber};
\node [font=\normalsize] at (4.75,13) {Room};
\node [font=\normalsize] at (7.25,12.5) {Exit Duct};
\draw [->, >=Stealth] (0.25,14.25) -- (1,13.75);
\draw [->, >=Stealth] (0,11.5) -- (1,12);
\draw [->, >=Stealth] (3.25,13) -- (4,13);
\draw [->, >=Stealth] (6.75,13) -- (7.75,13);
\end{circuitikz}
}%

\label{fig:my_label}
\end{figure}
Assuming the air behaves as an ideal gas with $C_p = 1.005 kJ/kg\cdot K$ and $R = 0.287 kJ/kg.K$, the rate of heat gain by the air from the room is \rule{2cm}{0.4pt} (round off to two decimal places).

\hfill [2021 ME]
\item Two smooth identical spheres each of radius $125 mm$ and weight $100 N$ rest
in a horizontal channel having vertical walls. The distance between vertical
walls of the channel is $400 mm.$ 
\begin{figure}[H]
\centering
\resizebox{3cm}{!}{%
\begin{circuitikz}
\tikzstyle{every node}=[font=\Large]
\draw [line width=1pt, short] (-2,16.25) -- (-2,9.75);
\draw [line width=1pt, short] (-2,9.75) -- (2.25,9.75);
\draw [line width=1pt, short] (2.25,9.75) -- (2.25,16.25);
\draw [ line width=1pt ] (1,12.75) circle (1.25cm);
\draw [ line width=1pt ] (-0.75,11) circle (1.25cm);
\draw [line width=1pt, <->, >=Stealth] (-2,15.5) -- (2.25,15.5)node[pos=0.5, fill=white]{400};
\draw [line width=1pt, ->, >=Stealth] (1,12.75) -- (2,13.5);
\draw [line width=1pt, ->, >=Stealth] (-0.75,11) -- (0.25,10.25);
\node at (-0.75,11) [circ] {};
\node at (1,12.75) [circ] {};
\node at (-0.75,11) [circ] {};
\node [font=\Large] at (1,13.5) {125};
\node [font=\Large] at (0,11) {125};
\draw [line width=1.5pt, short] (2.25,16.25) -- (2.5,16.25);
\draw [line width=1.5pt, short] (2.5,16.25) -- (2.25,16.25);
\draw [line width=1.5pt, short] (-2,16.25) -- (-2.25,16.25);
\draw [line width=0.9pt, short] (2.5,16.25) -- (2.5,9.5);
\draw [line width=0.9pt, short] (2.5,9.5) -- (-2.25,9.5);
\draw [line width=0.9pt, short] (-2.25,16.25) -- (-2.25,9.5);
\node [font=\Large] at (0,8.5) {All dimensions are in $mm$};
\end{circuitikz}
}%

\label{fig:my_label}
\end{figure}
The reaction at the point of contact between two spheres is \rule{2cm}{0.4pt} $N$ (round off to one decimal place).

\hfill [2021 ME]
\item An overhanging beam $PQR$ is subjected to uniformly distributed load 20
$kN/m$ as shown in the figure. 
\begin{figure}[H]
\centering
\resizebox{8cm}{!}{%
\begin{circuitikz}
\tikzstyle{every node}=[font=\Large]
\draw [line width=0.6pt, short] (-4,14.75) -- (4.75,14.75);
\draw [line width=1pt, short] (-4,13.25) -- (4.75,13.25);
\draw [line width=1pt, short] (4.75,13.25) -- (4.75,12.25);
\draw [line width=1pt, short] (-4,13.25) -- (-4,12.25);
\draw [line width=1pt, short] (-4,12.25) -- (4.75,12.25);
\draw [line width=1pt, short] (-4,12.25) -- (-4.25,12);
\draw [line width=1pt, short] (-4,12.25) -- (-3.75,12);
\draw [line width=1pt, short] (-4.25,12) -- (-3.75,12);
\draw [line width=1pt, short] (-4.5,12) -- (-3.5,12);
\draw [line width=1pt, short] (2.5,12.25) -- (2.25,12);
\draw [line width=1pt, short] (2.5,12.25) -- (2.75,12);
\draw [line width=1pt, short] (2.25,12) -- (2.75,12);
\draw [line width=1pt, ->, >=Stealth] (-4,14.75) -- (-4,13.25);
\draw [line width=1pt, ->, >=Stealth] (-3.25,14.75) -- (-3.25,13.25);
\draw [line width=1pt, ->, >=Stealth] (-2.5,14.75) -- (-2.5,13.25);
\draw [line width=1pt, ->, >=Stealth] (-1.75,14.75) -- (-1.75,13.25);
\draw [line width=1pt, ->, >=Stealth] (-1,14.75) -- (-1,13.25);
\draw [line width=1pt, ->, >=Stealth] (-0.25,14.75) -- (-0.25,13.25);
\draw [line width=1pt, ->, >=Stealth] (0.5,14.75) -- (0.5,13.25);
\draw [line width=1pt, ->, >=Stealth] (1.25,14.75) -- (1.25,13.25);
\draw [line width=1pt, ->, >=Stealth] (2,14.75) -- (2,13.25);
\draw [line width=1pt, ->, >=Stealth] (2.75,14.75) -- (2.75,13.25);
\draw [line width=1pt, ->, >=Stealth] (3.5,14.75) -- (3.5,13.25);
\draw [line width=1pt, ->, >=Stealth] (4.25,14.75) -- (4.25,13.25);
\draw [line width=1pt, ->, >=Stealth] (4.75,14.75) -- (4.75,13.25);
\draw [line width=0.5pt, dashed] (-4.25,12.75) -- (5.25,12.75);
\draw [line width=1.1pt, short] (2,12) -- (3,12);
\node [font=\Large] at (2.5,12.75) {$Q$};
\node [font=\Large] at (-3.75,12.75) {$P$};
\node [font=\Large] at (5.25,12.75) {$R$};
\node [font=\Large] at (6.25,12.75) {$z$};
\node [font=\Large] at (-4.75,12.75) {$z$};
\node [font=\Large] at (1.25,16) {$20kN/m$};
\draw [line width=0.5pt, short] (2.25,12) -- (2,11.75);
\draw [line width=0.5pt, short] (2.5,12) -- (2.25,11.75);
\draw [line width=0.5pt, short] (2.75,12) -- (2.5,11.75);
\draw [line width=0.5pt, short] (3,12) -- (2.75,11.75);
\draw [line width=0.5pt, short] (-3.5,12) -- (-3.75,11.75);
\draw [line width=0.5pt, short] (-3.75,12) -- (-4,11.75);
\draw [line width=0.5pt, short] (-4,12) -- (-4.25,11.75);
\draw [line width=0.5pt, short] (-4.5,12) -- (-4.75,11.75);
\draw [line width=0.5pt, short] (-4.25,12) -- (-4.5,11.75);
\draw [line width=0.5pt, <->, >=Stealth] (-4,11.5) -- (2.5,11.5)node[pos=0.5, fill=white]{2000};
\draw [line width=0.5pt, <->, >=Stealth] (2.5,11.5) -- (4.75,11.5)node[pos=0.5, fill=white]{1000};
\draw [line width=0.5pt, ->, >=Stealth] (0.25,16) -- (-0.75,15);
\draw [line width=0.5pt, short] (-4,11.75) -- (-4,11.25);
\draw [line width=0.5pt, short] (2.5,11.75) -- (2.5,11.25);
\draw [line width=0.5pt, short] (4.75,11.75) -- (4.75,11.25);
\draw [line width=1pt, short] (8,15) -- (8,12);
\draw [line width=1pt, short] (8,15) -- (9.5,15);
\draw [line width=1pt, short] (9.5,15) -- (9.5,12);
\draw [line width=1pt, short] (8,12) -- (9.25,12);
\draw [line width=1pt, short] (9.25,12) -- (9.5,12);
\draw [line width=0.7pt, ->, >=Stealth] (8.75,13.75) -- (8.75,16.25);
\draw [line width=0.7pt, ->, >=Stealth] (8.75,13.75) -- (11,13.75);
\node [font=\Large] at (9.25,16.25) {$y$};
\node [font=\Large] at (11,13.25) {$x$};
\draw [line width=0.7pt, <->, >=Stealth] (7.5,15) -- (7.5,12)node[pos=0.5, fill=white]{100};
\draw [line width=0.7pt, <->, >=Stealth] (8,11.5) -- (9.5,11.5);
\node [font=\Large] at (8.75,11) {24};
\draw [line width=0.7pt, short] (7.25,15) -- (7.75,15);
\draw [line width=0.7pt, short] (7.25,12) -- (7.75,12);
\draw [line width=0.7pt, short] (8,11.75) -- (8,11.25);
\draw [line width=0.7pt, short] (9.5,11.75) -- (9.5,11.25);
\node [font=\Large] at (0.5,10.5) {All Dimensions are in $mm$};
\end{circuitikz}
}%

\label{fig:my_label}
\end{figure}
The maximum bending stress developed in the beam is \rule{2cm}{0.4pt} $MPa$ (round off to one decimal place).

\hfill [2021 ME]

\item A short shoe drum (radius $260 mm$) brake is shown in the figure. A force of
$1 kN$ is applied to the lever. The coefficient of friction is 0.4. 
\begin{figure}[H]
\centering
\resizebox{4cm}{!}{%
\begin{circuitikz}
\tikzstyle{every node}=[font=\Huge]
\draw [ line width=1pt ] (-0.75,-1.25) circle (5cm);
\draw [line width=0.7pt, short] (-1.75,3.75) -- (-2.25,4.5);
\draw [line width=0.7pt, short] (-0.5,3.75) -- (0,4.5);
\draw [line width=0.7pt, short] (-2.25,4.5) -- (-2.25,6);
\draw [line width=0.7pt, short] (0,4.5) -- (0,6);
\draw [line width=0.7pt, short] (-2.25,6) -- (0,6);
\draw [line width=0.7pt, short] (-1.25,6) -- (-2.25,4.5);
\draw [line width=0.7pt, short] (-0.5,6) -- (-2,4);
\draw [line width=0.7pt, short] (0,5.25) -- (-1.25,3.75);
\draw [line width=0.7pt, short] (0,6) -- (-1.75,3.75);
\draw [line width=0.7pt, short] (-1.75,6) -- (-2.25,5.5);
\draw [line width=1pt, short] (-10,6.25) -- (6.75,6);
\draw [line width=1pt, short] (-10,6.25) -- (-10,7);
\draw [line width=1pt, short] (-10,7) -- (6.75,6.75);
\draw [line width=1pt, short] (6.75,6.75) -- (6.75,6);
\draw [line width=1.3pt, ->, >=Stealth] (6.75,9.75) -- (6.75,6.75);
\draw [line width=1.3pt, ->, >=Stealth] (-0.75,-1.25) -- (2.25,-5)node[pos=0.5, fill=white]{260};
\draw [line width=1.3pt, <->, >=Stealth] (-1.25,9) -- (6.75,9)node[pos=0.5, fill=white]{500};
\draw [line width=1.3pt, dashed] (-8,-1.5) -- (5.75,-1.5);
\draw [line width=1.3pt, dashed] (-1.25,10.5) -- (-0.75,-4);
\draw [line width=1.3pt, ->, >=Stealth] (-2.5,2.5) .. controls (-1.5,3.25) and (-0.75,3.5) .. (0.75,2.5) ;
\draw [line width=1.3pt, <->, >=Stealth] (-7.5,6) -- (-7.5,-1.5)node[pos=0.5,
 fill=white]{310};
\draw [ line width=1.3pt ] (-10.75,6.75) circle (0.75cm);
\draw [line width=1.3pt, short] (-10.75,9.25) -- (-10.75,7.5);
\draw [line width=1.3pt, <->, >=Stealth] (-10.75,9.25) -- (-1,9)node[pos=0.5, fill=white]{500};
\node [font=\Huge] at (6.5,10.75) {$1kN$};
\node [font=\Huge] at (-1.25,1.25) {Drum rotation};
\node [font=\Huge] at (-10.75,6.75) {$+$};
\draw [line width=1.3pt, short] (-1.5,-2) -- (-0.25,-2);
\draw [line width=1.3pt, short] (-1.5,-2) .. controls (-1.75,-0.5) and (-0.25,0) .. (-0.25,-2);
\node [font=\Huge] at (-1.5,-8.25) {All dimension are in $mm$};
\end{circuitikz}
}%

\label{fig:my_label}
\end{figure}
The magnitude of the torque applied by the brake is \rule{2cm}{0.4pt}$N \cdot m$ (round off to one decimal place). 

\hfill [2021 ME]
\item A machine part in the form of cantilever beam is subjected to fluctuating
load as shown in the figure. The load varies from $800 N$ to $1600 N$. The
modified endurance, yield and ultimate strengths of the material are $200
MPa$, $500 MPa$ and $600 MPa$, respectively. 
\begin{figure}[H]
\centering
\resizebox{6cm}{!}{%
\begin{circuitikz}
\tikzstyle{every node}=[font=\Huge]
\draw [line width=0.9pt, short] (-8.5,4.5) -- (-8.5,-1.5);
\draw [line width=1.3pt, short] (-8.5,2) -- (3,2);
\draw [line width=1.3pt, short] (3,2) -- (3,1);
\draw [line width=1.3pt, short] (-8.5,1.25) -- (3,1.25);
\draw [line width=1.3pt, short] (-8.5,4.5) -- (-9.5,4.5);
\draw [line width=1.3pt, short] (-9.5,4.5) -- (-9.5,-1.5);
\draw [line width=1.3pt, short] (-9.5,-1.5) -- (-8.75,-1.5);
\draw [line width=1.3pt, short] (-8.5,-1.5) -- (-9,-1.5);
\draw [line width=1.3pt, short] (5.5,4.5) -- (5.5,-1.5);
\draw [line width=1.3pt, short] (5.5,4.5) -- (8.25,4.5);
\draw [line width=1.3pt, short] (8.25,4.5) -- (8.25,-1.5);
\draw [line width=1.3pt, short] (5.5,-1.5) -- (8.25,-1.5);
\draw [line width=1.3pt, ->, >=Stealth] (2.25,4.25) -- (2.25,2);
\node [font=\Huge] at (1.5,5.25) {800$N$ to $1600N$};
\draw [line width=1.3pt, <->, >=Stealth] (-8.25,0.25) -- (3,0.25)node[pos=0.5, fill=white]{100};
\draw [line width=1.3pt, <->, >=Stealth] (6.75,7) -- (8.75,7)node[pos=0.5, fill=white]{Beam };
\draw [line width=1.3pt, <->, >=Stealth] (6.75,5.5) -- (8.75,5.5)node[pos=0.5, fill=white]{cross-section};
\draw [line width=1.3pt, <->, >=Stealth] (8.75,4.25) -- (8.75,-1.5)node[pos=0.5, fill=white]{20};
\draw [line width=1.3pt, <->, >=Stealth] (5.5,-2.25) -- (8.5,-2.25)node[pos=0.5, fill=white]{12};
\node [font=\Huge] at (-2.5,-4) {All dimension are in $mm$};
\end{circuitikz}
}%

\label{fig:my_label}
\end{figure}
The factor of safety of the beam using modified Goodman criterion is \rule{2cm}{0.4pt} (round off to one decimal place).
\hfill [2021 ME]
\item A cantilever beam of rectangular cross-section is welded to a support by
means of two fillet welds as shown in figure. A vertical load of 2 $kN$ acts at
free end of the beam. 
\begin{figure}[H]
\centering
\resizebox{7cm}{!}{%
\begin{circuitikz}
\tikzstyle{every node}=[font=\Huge]
\draw [line width=1pt, short] (-5.5,13) -- (-5.5,5.25);
\draw [line width=1pt, short] (-5.5,13) -- (2,13);
\draw [line width=1pt, short] (2,13) -- (2,5.25);
\draw [line width=1pt, short] (-5.5,5.25) -- (2,5.25);
\draw [line width=1pt, short] (-3.75,11) -- (-3.75,7.25);
\draw [line width=1pt, short] (-3.75,11) -- (0,11);
\draw [line width=1pt, short] (0,11) -- (0,7.25);
\draw [line width=1pt, short] (-3.75,7.25) -- (0,7.25);
\draw [line width=1pt, short] (0,11) -- (0.5,11);
\draw [line width=1pt, short] (0.5,11) -- (0.5,7.25);
\draw [line width=1pt, short] (0,7.25) -- (0.5,7.25);
\draw [line width=1pt, short] (-3.75,11) -- (-4,11);
\draw [line width=1pt, short] (-4,11) -- (-4.25,11);
\draw [line width=1pt, short] (-4.25,11) -- (-4.25,7.25);
\draw [line width=1pt, short] (-4.25,7.25) -- (-3.75,7.25);
\draw [line width=1pt, short] (6.5,13) -- (6.5,5);
\draw [line width=1pt, short] (6.5,12.5) -- (6.5,13);
\draw [line width=1pt, short] (6.5,13) -- (8,13);
\draw [line width=1pt, short] (8,13) -- (8,5);
\draw [line width=1pt, short] (6.5,5) -- (8,5);
\draw [line width=1pt, short] (8,11) -- (17.25,11);
\draw [line width=1pt, short] (17.25,11) -- (19.5,11);
\draw [line width=1pt, short] (19.5,11) -- (19.5,7.25);
\draw [line width=1pt, short] (8,7.25) -- (19.5,7.25);
\draw [line width=1pt, <->, >=Stealth] (5.25,11) -- (5.25,7.25)node[pos=0.5, fill=white]{$40mm$};
\draw [line width=1pt, <->, >=Stealth] (8,12.25) -- (19.25,12.25)node[pos=0.5, fill=white]{$150mm$};
\draw [line width=1pt, <->, >=Stealth] (-3.75,11.75) -- (-0.25,11.75)node[pos=0.5, fill=white]{50mm};
\draw [line width=1pt, short] (8.5,11) -- (8.5,7.25);
\draw [line width=0.5pt, dashed] (-7.25,9.25) -- (4,9);
\draw [line width=0.5pt, dashed] (-1.75,15.25) -- (-1.75,3.75);
\draw [line width=0.5pt, dashed] (6,9) -- (21.75,9);
\draw [line width=0.5pt, ->, >=Stealth] (19.25,13.75) -- (19.25,11.25);
\node [font=\Huge] at (19,14.5) {$2kN$};
\node [font=\Huge] at (3.75,3.25) {Support};
\draw [line width=0.5pt, ->, >=Stealth] (5,3.25) -- (6.75,4.5);
\draw [line width=0.5pt, ->, >=Stealth] (2.25,3.25) .. controls (1.25,4) and (1.5,4) .. (0.25,5) ;
\end{circuitikz}
}%

\label{fig:my_label}
\end{figure}
Considering that the allowable shear stress in weld is $60 N/mm^2$
, the
minimum size (leg) of the weld required is \rule{2cm}{0.4pt} $mm$ (round off to one
decimal place).

\hfill [2021 ME]

\iffalse
    \author{EE24BTECH11029}
    \section{ma}
    \chapter{2014}
\fi

\item If $a\in \abs{a} \textless 1$, then the value of\\
    $\frac{\brak{1-\abs{a^2}}}{\pi}\int_{\Gamma}\frac{\abs{dz}}{{\abs{z+\alpha}}^2}$,\\
    where $\Gamma$ is simple closed curve $\abs{z}=1$ taken with the positive orientation is\\
    \item Cosider $C\sbrak{-1,1}$ equipped with the supremum norm given by $\abs{\abs{f}}_{\infty}=sup\{\abs{f\brak{t}}:t\in \sbrak{-1,1}\}$ for 
    $f\in C\sbrak{-1,1}.$Define a linear functional $T$ on $C\sbrak{-1,1}$ by $T\brak{f}=\int_{0}^{1}f\brak{t}\,dt-\int_{0}^{1}f\brak{t}\,dt$ for all $f\in C\sbrak{-1,1}.$Then the value of $\abs{\abs{T}}$ is\\
    \item Consider the vector space $C\sbrak{0,1}$ over $R$. Cosider the following statements:\\
    $P:$ If the set $\{t,f_1,t^2f_2,t^3f_3\}$ is linearly independent, then the set $\{f_1,f_2,f_3\}$ is linearly independent, where $f_1,f_2,f_3\in C\sbrak{0,1}$ and $t^n$ represents the polynomial function $t\rightarrow t^n,n\in N$\\
    $Q:$ If $F:C\sbrak{0,1}\rightarrow R$ is given by $F\brak{x}=\int_{0}^{1}x\brak{t^2}\,dt$ for each $x\in C\sbrak{0,1},$ then $F$ is a linear map.\\
    Which of the above statements hold TRUE?
    \begin{enumerate}
        \item Only $P$
        \item Only $Q$
        \item Both $P$ and $Q$
        \item Neither $P$ nor $Q$\\
    \end{enumerate}
    \item Using the Newton-Raphson method with the initial guess $x^{\brak{0}}=6,$ the apporximate value of the real root of $x\log_{10}x=4.77,$ after the second iteration, is\\
    \item Let the following dicrete data be obtained from a curve $y=y\brak{x};$\\
    $x: \quad 0 \quad 0.25  \quad 0.5  \quad  0.75  \quad 1.0\\
    y: \quad 1 \quad 0.9896 \quad 0.9589\quad 0.9089 \quad 0.8415$\\
    Let $S$ be the solid of revolution obtained by rotating the above curve about the $x-axis$ between $x=0$ and $x=1$ and let $V$ denote its volume. The approximate value of $V$, obtained using Simpsons $\frac{1}{3}$ rule, is\\
    \item The integral surface of the first order partial differential equation\\
    $2y\brak{z-3}\frac{dz}{dx}+\brak{2x-z}\frac{dz}{dy}=y\brak{2x-3}$\\
    passing through the curve $x^2+y^2=2x,z=0$ is
    \begin{enumerate}
        \item $x^2+y^2-z^2-2x+4z=0$
        \item $x^2+y^2-z^2-2x+8z=0$
        \item $x^2+y^2+z^2-2x+16z=0$
        \item $x^2+y^2+z^2-2x+8z=0$\\
    \end{enumerate}
    \item The boundary value problem $\frac{d^2\phi}{dx^2}+\lambda\phi=x, \phi\brak{0}=0$ and $\frac{d\phi}{dx}\brak{1}=0,$ is converted into the integral equation $\phi\brak{x}=g\brak{x}+\lambda\int_{0}^{1}k\brak{x,\xi}\phi\brak{\xi}d\xi,$ where the kernel $k\brak{x,\xi }=\begin{cases}
    \xi,0\textless\xi\textless x\\x,x\textless \xi \textless 1
    \end{cases} $ then $g\brak{\frac{2}{3}}$ is\\
    \item If $y_1\brak{x}=x$ is a solution to the differential equation $\brak{1-x^2}\frac{d^2y}{dx^2}-2x\frac{dy}{dx}+2y=0,$ then its general solution is 
    \begin{enumerate}
        \item $y\brak{x}=c_1x+c_2\brak{x\ln\abs{1+x^2}-1}$
        \item $y\brak{x}=c_1x+c_2\brak{\ln\frac{\abs{1-x}}{\abs{1+x}}+1}$
        \item $y\brak{x}=c_1x+c_2\brak{\frac{x}{2}\ln\frac{\abs{1+x}}{\abs{1-x}}-1}$
        \item $y\brak{x}=c_1x+c_2\brak{x\ln\abs{1-x^2}-1}$\\
    \end{enumerate}
    \item The solution to the initial value problem $\frac{d^2y}{dt^2}+2\frac{dy}{dt}+5y=3e^{-t}\sin{t},y\brak{0}=0$ and $\frac{dy}{dt}\brak{0}=3,$ is
    \begin{enumerate}
        \item $y\brak{t}=e^{t}\brak{\sin{t}+\sin{2t}}$
        \item $y\brak{t}=e^{-t}\brak{\sin{t}+\sin{2t}}$
        \item $y\brak{t}=3e^{t}\brak{\sin{t}}$
        \item $y\brak{t}=3e^{-t}\brak{\sin{t}}$ \\
    \end{enumerate}
    \item The time to failure, in months, of light bulbs manufactured at two plants $A$ and $B $obey the exponential distribution with means $6$ and $2$ months respectively. Plant $B$ produces four times as many bulbs as plant $A$ does. Bulbs from these plants are indistinguishable. They are mixed and sold together. Given that a bulb purchased at random is working after $12$ months, the probability that it was manufactured at plant $A$ is\\
    \item Let $X,Y$ be continuous random variables with joint density function\\
    $f_{x,y}\brak{x,y}=\begin{cases}
        e^{-y}\brak{1-e^{-x}}  if  0\textless x\textless y \textless\infty\\ e^{-x}\brak{1-e^{-y}}  if  0\textless y\textless x \textless\infty
    \end{cases}$\\
    The value of  $E\sbrak{X+Y}$ is \\
    \item Let $X\brak{0,1} \brak{1,2}$ be the subspace of $R$, where $R$ is equipped with the usual topology. Which of the following is FALSE?
    \begin{enumerate}
        \item There exists a non-constant continitous function $f:X\rightarrow Q$
        \item  $X$ is homeomorphic to $\brak{-\infty,-3}\cup[ 0,\infty ) $ 
        \item There exists an onto continuous function : $\sbrak{0,1}\rightarrow \vec{X},$ where $\Vec{X}$ is closure of $X$ in $\vec{R}$
        \item There exists an onto continuous function $f:\sbrak{0,1}\rightarrow X$\\
    \end{enumerate}
    \item Let $X=\myvec{2&0&-3&\\
                        3&-1&-3\\
                        0&0&-1}$.A matrix $P$ such that $P^{-1}XP$ is a diagonal matrix, is
    \begin{enumerate}
        \item $\myvec{1&1&1&\\
                        0&1&1\\
                        1&1&0}$
        \item $\myvec{-1&1&1&\\
                        0&1&1\\
                        1&1&0}$
        \item $\myvec{1&-1&1&\\
                        0&1&1\\
                        1&1&0}$
        \item $\myvec{-1&-1&1&\\
                        0&-1&1\\
                        1&1&0}$
    \end{enumerate}


    


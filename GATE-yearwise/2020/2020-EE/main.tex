\iffalse
\chapter{2020}
\author{AI24BTECH11009}
\section{ee}
\fi

\item A three-phase, 50 Hz, 4-pole induction motor runs at no-load with a slip of 1\%. With full load, the slip increases to 5\%. The \% speed regulation of the motor (rounded off to 2 decimal places) is $\_\_\_\_$. \\
\item Currents through ammeters A2 and A3 in the figure are $1\angle{10\degree}$ and $1\angle{70\degree}$ respectively. The reading of
the ammeter A1 (rounded off to 3 decimal places) is $\_\_\_\_$A.
\begin{figure}[!ht]
\centering
\resizebox{0.5\textwidth}{!}{%
\begin{circuitikz}
\tikzstyle{every node}=[font=\normalsize]
\draw  (5.75,9.75) circle (0.25cm);
\draw  (9.25,8.5) circle (0.25cm);
\draw  (9.25,10.75) circle (0.25cm);
\draw [short] (9.5,10.75) -- (11.25,10.75);
\draw [short] (9.5,8.5) -- (11.25,8.5);
\draw [short] (9,10.75) -- (7.5,10.75);
\draw [short] (7.5,10.75) -- (7.5,8.5);
\draw [short] (7.5,8.5) -- (9,8.5);
\draw [short] (6,9.75) -- (7.5,9.75);
\draw [short] (4.25,9.75) -- (5.5,9.75);
\draw [->, >=Stealth] (7.5,11) -- (8.75,11);
\draw [->, >=Stealth] (7.75,8.75) -- (8.75,8.75);
\draw [->, >=Stealth] (4.25,10) -- (5.25,10);
\node [font=\normalsize] at (9.25,10.75) {A2};
\node [font=\normalsize] at (9.25,8.5) {A3};
\node [font=\normalsize] at (5.75,9.75) {A1};
\node [font=\normalsize] at (4.75,10.25) {$I_1$};
\node [font=\normalsize] at (8.25,11.25) {$I_2$};
\node [font=\normalsize] at (8.25,8.25) {$I_3$};
\end{circuitikz}

}%
\end{figure}\\
\item The Thevenin equivalent voltage, $V_{TH}$, in V (rounded off two 2 decimal places) of the network shown below, is $\_\_\_\_$.
\begin{figure}[!ht]
\centering
\resizebox{0.5\textwidth}{!}{%
\begin{circuitikz}
\tikzstyle{every node}=[font=\normalsize]
\draw (5.5,11) to[american voltage source] (5.5,8);
\draw (7,11) to[R] (7,8);
\draw (7,11) to[R] (9.5,11);
\draw (9.5,11) to[R] (11.75,11);
\draw [short] (5.5,11) -- (7.25,11);
\draw [short] (5.5,8) -- (11.75,8);
\draw (9.5,8) to[american current source] (9.5,11);
\node at (11.75,11) [circ] {};
\node at (11.75,8) [circ] {};
\node [font=\normalsize] at (4.75,9.5) {4 V};
\node [font=\normalsize] at (8.25,11.5) {2 $\Omega$};
\node [font=\normalsize] at (10.75,11.5) {3 $\Omega$};
\node [font=\normalsize] at (7.5,9.5) {3 $\Omega$};
\node [font=\normalsize] at (10.25,9.5) {5 A};
\node [font=\normalsize] at (11.75,10.75) {+};
\node [font=\normalsize] at (11.75,8.25) {-};
\node [font=\normalsize] at (12,9.5) {$V_{TH}$};
\end{circuitikz}

}%
\end{figure}\\
\item A double pulse measurement for an inductivity loaded circuit controlled by the IGBT switch is carried out to evaluate the reverse recovery characteristics of the diode, D, represented approximately as a piecewise linear plot of current vs time at diode turn-off. $L_{par}$ is a parasitic inductance due to the wiring of the circuit, and is in series with the diode. The point on the plot (indicate your choice by entering 1, 2, 3 or 4) at which the IGBT experiences the highest current stress is $\_\_\_\_$.
\begin{figure}[!ht]
\centering
\resizebox{0.7\textwidth}{!}{%
\begin{circuitikz}
\tikzstyle{every node}=[font=\normalsize]
\draw [->, >=Stealth] (5.75,4) -- (5.75,12);
\draw [->, >=Stealth] (4.5,7) -- (13.25,7);
\draw [short] (3.75,10) -- (5.75,10);
\draw [short] (5.75,10) -- (9,5);
\draw [short] (9,5) -- (11.25,7);
\draw (8.75,11.75) to[american voltage source] (8.75,7.5);
\draw [short] (8.75,11.75) -- (10,11.75);
\draw [short] (10,11.75) -- (10,11);
\draw [short] (8.75,7.5) -- (10,7.5);
\draw [short] (10,11) -- (9.75,10.75);
\draw [short] (9.75,11) -- (9.75,10.5);
\draw [short] (9.5,11) -- (9.5,10.5);
\draw [short] (9.25,10.75) -- (9.5,10.75);
\draw [->, >=Stealth] (9.75,10.75) -- (10,10.25);
\draw [short] (10,10.25) -- (10,10);
\draw [short] (10,10) -- (11.5,10);
\draw (10,10) to[L ] (10,8.25);
\draw (10,7.5) to[D] (10,8.5);
\draw (11.5,10) to[R] (11.5,8.5);
\draw (11.5,8.5) to[L ] (11.5,7.5);
\draw [short] (10,7.5) -- (11.5,7.5);
\node [font=\normalsize] at (12.75,6.75) {Time};
\node [font=\normalsize] at (5.25,8.25) {Diode};
\node [font=\normalsize] at (5.25,8) {current};
\node [font=\normalsize] at (6,10) {1};
\node [font=\normalsize] at (7.5,6.75) {2};
\node [font=\normalsize] at (9,4.75) {3};
\node [font=\normalsize] at (11.25,6.75) {4};
\node [font=\normalsize] at (8,10) {$V_{source}$};
\node [font=\normalsize] at (10.5,11.25) {IGBT};
\node [font=\normalsize] at (10.75,9) {$L_{par}$};
\node [font=\normalsize] at (12.25,9.25) {$R_{load}$};
\node [font=\normalsize] at (12.25,8) {$L_{load}$};
\node [font=\normalsize] at (10.5,8) {D};
\end{circuitikz}

}%
\end{figure}\\
\item A single-phase, 4 kVA, 200 V/100 V, 50 Hz transformer with laminated CRGO steel core has rated no-load loss of 450 W. When the high-voltage winding is excited with 160 V, 40 Hz sinusoidal ac supply, the no-load losses are found to be 320 W. When the high-voltage winding of the same transformer is supplied from a 100 V, 25 Hz sinusoidal ac source, the no-load losses will be $\_\_\_\_$ W (rounded off two 2 decimal places). \\
\item A single-phase inverter is fed from a 100 V dc source and is controlled using a quasi-square wave modulation scheme to produce an output waveform, $v\brak{t}$, as shown. The angle $\sigma$ is adjusted to entirely eliminate the $3^{\text{rd}}$ harmonic component from the output voltage. Under this condition, for $v\brak{t}$, the magnitude of the $5^{\text{th}}$ harmonic component as a percentage of the magnitude of the fundamental component is $\_\_\_\_$ (rounded off to 2 decimal places).
\begin{figure}[!ht]
\centering
\resizebox{0.7\textwidth}{!}{%
\begin{circuitikz}
\tikzstyle{every node}=[font=\normalsize]
\draw [->, >=Stealth] (4.25,5.25) -- (4.25,11.75);
\draw [->, >=Stealth] (4.25,8.5) -- (15.5,8.5);
\draw  (5,10.25) rectangle (6.5,8.5);
\draw  (7.5,8.5) rectangle (9,6.75);
\draw  (10,8.5) rectangle (11.5,10.25);
\draw  (12.75,8.5) rectangle (14.25,6.75);
\draw [dashed] (5,8.5) -- (5,6.75);
\draw [dashed] (7,9.75) -- (7,8.5);
\draw [dashed] (7.5,7.75) -- (7.5,6.75);
\draw [dashed] (7,7.75) -- (7,6.75);
\draw [dashed] (9.5,9.75) -- (9.5,8.75);
\draw [dashed] (9.5,8) -- (9.5,6.75);
\draw [dashed] (12,10) -- (12,8.75);
\draw [dashed] (12,8) -- (12,6.75);
\draw [dashed] (15,8.5) -- (15,7);
\draw [short] (4,10.25) -- (4.5,10.25);
\draw [short] (4.5,6.75) -- (4,6.75);
\draw [short] (5.75,8.75) -- (5.75,8.25);
\draw [short] (7,8.75) -- (7,8.25);
\draw [short] (8.25,8.75) -- (8.25,8.25);
\draw [short] (9.5,8.75) -- (9.5,8.25);
\draw [short] (10.75,8.75) -- (10.75,8.25);
\draw [short] (12,8.75) -- (12,8.25);
\draw [short] (13.5,8.75) -- (13.5,8.25);
\draw [short] (15,8.75) -- (15,8.25);
\draw [->, >=Stealth] (3.5,9.5) -- (4.25,9.5);
\draw [->, >=Stealth] (5.75,9.5) -- (5,9.5);
\draw [->, >=Stealth] (6,9.25) -- (6.5,9.25);
\draw [->, >=Stealth] (7.5,9.25) -- (7,9.25);
\draw [->, >=Stealth] (3.5,7.5) -- (4.25,7.5);
\draw [->, >=Stealth] (5.5,7.5) -- (5,7.5);
\draw [->, >=Stealth] (6.5,7.25) -- (7,7.25);
\draw [->, >=Stealth] (8,7.25) -- (7.5,7.25);
\draw [->, >=Stealth] (8.5,7.25) -- (9,7.25);
\draw [->, >=Stealth] (10,7.25) -- (9.5,7.25);
\draw [->, >=Stealth] (9,9.5) -- (9.5,9.5);
\draw [->, >=Stealth] (10.5,9.5) -- (10,9.5);
\draw [->, >=Stealth] (11,9.5) -- (11.5,9.5);
\draw [->, >=Stealth] (12.5,9.5) -- (12,9.5);
\draw [->, >=Stealth] (11.25,7.5) -- (12,7.5);
\draw [->, >=Stealth] (13.5,7.5) -- (12.75,7.5);
\draw [->, >=Stealth] (13.75,7.5) -- (14.25,7.5);
\draw [->, >=Stealth] (15.5,7.5) -- (15,7.5);
\node [font=\normalsize] at (3.5,10.25) {100 V};
\node [font=\normalsize] at (3.5,6.75) {-100 V};
\node [font=\normalsize] at (15.5,8.25) {$\omega t$};
\node [font=\normalsize] at (4,11.75) {$v(t)$};
\node [font=\normalsize] at (5.75,8) {$\pi$/2};
\node [font=\normalsize] at (4,8.5) {0};
\node [font=\normalsize] at (7,8) {$\pi$};
\node [font=\normalsize] at (8.25,9) {3$\pi$/2};
\node [font=\normalsize] at (9.5,9) {2$\pi$};
\node [font=\normalsize] at (10.75,8) {5$\pi$/2};
\node [font=\normalsize] at (12,8) {3$\pi$};
\node [font=\normalsize] at (13.5,9) {7$\pi$/2};
\node [font=\normalsize] at (15,9) {4$\pi$};
\node [font=\normalsize] at (4.5,9.5) {$\sigma$};
\node [font=\normalsize] at (4.5,7.5) {$\sigma$};
\node [font=\normalsize] at (6.75,9.25) {$\sigma$};
\node [font=\normalsize] at (7.25,7.25) {$\sigma$};
\node [font=\normalsize] at (9.75,9.5) {$\sigma$};
\node [font=\normalsize] at (9.25,7.25) {$\sigma$};
\node [font=\normalsize] at (11.75,9.5) {$\sigma$};
\node [font=\normalsize] at (12.25,7.5) {$\sigma$};
\node [font=\normalsize] at (14.5,7.5) {$\sigma$};
\end{circuitikz}

}%
\end{figure}\\
\item A single 50 Hz synchronous generator on droop control was delivering 100 MW power to a system. Due to increase in load, generator power had to be increased by 10 MW, as a result of which, system frequency dropped to 49.75 Hz. Further increase in load in the system resulted in frequency of 49.25 Hz.
At this condition, the power in MW supplied by the generator is $\_\_\_\_$( rounded off to 2 decimal places). \\
\item Consider a negative unity feedback system with forward path transfer function 
\begin{align*}
    G\brak{s} = \frac{K}{\brak{s+a}\brak{s-b}\brak{s+c}},
\end{align*}
where $K, a, b, c$ are positive real numbers. For a Nyquist path enclosing the
entire imaginary axis and right half of the $s$-plane in the clockwise direction, the Nyquist plot of $\brak{1 + G\brak{s}}$, encircles the origin of $\brak{1 + G\brak{s}}$-plane once in the clockwise direction and never passes
through this origin for a certain value of $K$. Then, the number of poles of $\frac{G\brak{s}}{1 + G\brak{s}}$ lying in the open right half of the $s$-plane is $\_\_\_\_$. \\
\item The cross-section of a metal-oxide-semiconductor structure is shown schematically. Starting from an uncharged condition, a bias +3 V is applied to the gate contact with respect to the body contact. The charge inside the silicon dioxide layer is then measured to be $+Q$. The total charge contained within the dashed box shown, upon application of bias, expressed as a multiple of $Q$ (absolute value in Coulombs, rounded off to the nearest integer) is $\_\_\_\_$.
\begin{figure}[!ht]
\centering
\resizebox{0.7\textwidth}{!}{%
\begin{circuitikz}
\tikzstyle{every node}=[font=\normalsize]
\draw  (3.25,10.75) rectangle (12,10.25);
\draw  (3.25,10.25) rectangle (12,9.75);
\draw  (3.25,9.75) rectangle (12,7.5);
\draw  (3.25,7.5) rectangle (12,7);
\draw [ dashed] (2.5,11.75) rectangle  (13,6);
\fill[gray!50] (3.25, 9.75) rectangle (12, 7.5);
\fill[black] (3.25, 10.75) rectangle (12, 10.25);
\fill[black] (3.25, 7.5) rectangle (12, 7);
\node [font=\normalsize] at (7,5.75) {DASHED BOX};
\node [font=\normalsize, white] at (7.25,10.5) {GATE};
\node [font=\normalsize] at (7.25,10) {Silicon Dioxide};
\node [font=\normalsize] at (7.25,8.5) {Si};
\node [font=\normalsize, white] at (7.25,7.25) {Body};
\end{circuitikz}

}%
\end{figure}\\
\item For real numbers, $x$ and $y$, with $y = 3x^2 + 3x + 1$ the maximum and minimum value of $y$ for$x \in \sbrak{-2, 0}$ are respectively, $\_\_\_\_$.
\begin{enumerate}
    \item 7 and $\frac{1}{4}$.
    \item 7 and 1.
    \item -2 and $-\frac{1}{2}$.
    \item 1 and $\frac{1}{4}$. \\ 
\end{enumerate}
\item The vector function expressed by
\begin{align*}
    \vec{F} = \vec{a}_x\brak{5y - k_1z} + \vec{a}_y\brak{3z + k_2x}  +\vec{a}_z\brak{k_3y - 4x}
\end{align*}
represents a conservative field, where $\vec{a}_x$, $\vec{a}_y$, $\vec{a}_z$ are unit vectors along $x$, $y$ and $z$ directions, respectively. The values of constants $k_1$, $k_2$, $k_3$ are given by:
\begin{enumerate}
    \item $k_1 = 3$, $k_2 = 3$, $k_3 = 7$
    \item $k_1 = 3$, $k_2 = 8$, $k_3 = 5$
    \item $k_1 = 4$, $k_2 = 5$, $k_3 = 3$
    \item $k_1 = 0$, $k_2 = 0$, $k_3 = 0$ \\
\end{enumerate}
\item A 250 V dc shunt motor has an armature resistance of 0.2 $\Omega$ and a field resistance of 100 $\Omega$. When the motor is operated no-load at rated voltage, it draws an armature current of 5 A and runs at 1200 rpm. When a load is coupled to the motor, it draws total line current of 50 A at rated voltage, with a 5\% reduction in the air-gap flux due to armature reaction. Voltage drop across the brushes can be taken as 1 V per brush under all operating conditions. The speed of the motor, in rpm, under this loaded condition,
is closet to :
\begin{enumerate}
    \item 1200
    \item 1000
    \item 1220
    \item 900 \\
\end{enumerate}
\item Two buses, $i$ and $j$, are connected with a transmission line of admittance $Y$, at the two ends of which there are ideal transformers with turns ratio as shown. Bus admittance matrix for the system is :
\begin{figure}[!ht]
\centering
\resizebox{0.7\textwidth}{!}{%
\begin{circuitikz}
\tikzstyle{every node}=[font=\normalsize]
\draw [short] (3,9.75) -- (3,7.25);
\draw [short] (13,9.75) -- (13,7.25);
\draw [short] (3,8.5) -- (4.75,8.5);
\draw [short] (4.75,8.5) .. controls (5,8.75) and (5,8.75) .. (4.75,8.75);
\draw [short] (4.75,8.5) .. controls (5,8.5) and (5,8.5) .. (4.75,8.25);
\draw [short] (4.75,8.75) .. controls (5,9) and (5,9) .. (4.5,9);
\draw [short] (4.75,8.25) .. controls (5,8.25) and (5.25,8) .. (4.5,8);
\draw [short] (5.25,8.75) .. controls (5,9) and (5,9) .. (5.5,9);
\draw [short] (5.25,8.25) .. controls (5,8.25) and (5.25,8) .. (5.5,8);
\draw [short] (5.25,8.5) -- (9.75,8.5);
\draw [short] (9.75,8.5) .. controls (10,8.75) and (10,8.75) .. (9.75,8.75);
\draw [short] (9.75,8.5) .. controls (10,8.5) and (10,8.5) .. (9.75,8.25);
\draw [short] (9.75,8.75) .. controls (10,9) and (10,9) .. (9.5,9);
\draw [short] (9.75,8.25) .. controls (10,8.25) and (10,8) .. (9.5,8);
\draw [short] (10.25,8.75) .. controls (10,9) and (10,9) .. (10.5,9);
\draw [short] (10.25,8.25) .. controls (10,8) and (10.25,8) .. (10.5,8);
\draw [short] (10.25,8.5) -- (13,8.5);
\draw [short] (5.25,8.75) .. controls (5,8.75) and (5,8.75) .. (5.25,8.5);
\draw [short] (5.25,8.5) .. controls (5,8.5) and (5,8.5) .. (5.25,8.25);
\draw [short] (10.25,8.75) .. controls (10,8.75) and (10,8.75) .. (10.25,8.5);
\draw [short] (10.25,8.5) .. controls (10,8.5) and (10,8.5) .. (10.25,8.25);
\node [font=\normalsize] at (7.5,8.75) {$Y$};
\node [font=\normalsize] at (2.75,9.75) {$V_i$};
\node [font=\normalsize] at (13.25,9.75) {$V_j$};
\node [font=\normalsize] at (3,7) {Bus $i$};
\node [font=\normalsize] at (13,7) {Bus $j$};
\node [font=\normalsize] at (5,7.5) {$1:t_1$};
\node [font=\normalsize] at (10,7.5) {$t_j:1$};
\end{circuitikz}

}%
\end{figure}
\begin{enumerate}
    \item $\sbrak{\begin{matrix}
        -t_it_jY & t_j^2Y \\ t_i^2 & -t_it_jY
    \end{matrix}}$
     \item $\sbrak{\begin{matrix}
        t_it_jY & -t_j^2Y \\ -t_i^2 & t_it_jY
    \end{matrix}}$
     \item $\sbrak{\begin{matrix}
        t_i^2 & -t_it_jY \\ -t_it_jY & t_j^2
    \end{matrix}}$
     \item $\sbrak{\begin{matrix}
        t_it_jY & -\brak{t_i - t_j}^2Y \\ -\brak{t_i - t_j}^2Y & t_it_jY
    \end{matrix}}$
     \end{enumerate}

\iffalse
\section{ma}
\chapter{2022}
\author{ai24btech11036}
\fi

%\section{Questions with one mark each}

%\begin{enumerate}

\item Suppose that the characteristic equation of $M \in \mathbb{C}^{3 \times 3}$ is 
	\begin{align*}
		\lambda^{3}+\alpha \lambda^{2} + \beta \lambda -1=0
	\end{align*}
where $\alpha, \beta \in \mathbb{C}$ with $\alpha+\beta \neq 0$. Which of the following statements is TRUE?
		
		\hfill{2022-MA}
	\begin{enumerate}
		\item $M\brak{I-\beta M}=M^{-1} \brak{M+ \alpha I}$
		\item $M\brak{I+\beta M}=M^{-1} \brak{M- \alpha I}$
		\item $M^{-1}\brak{M^{-1}+\beta M}=M-\alpha I$
		\item $M^{-1}\brak{M^{-1}-\beta M}=M+\alpha I$
	\end{enumerate}

\item Consider \\
\textbf{P}: Let $M \in \mathbb{R}^{m \times n}$ with $m>n\geq 2$. If rank$\brak{M}=n$, then the system of linear equations $Mx=0$ has $x=0$ as the only solution. \\
\textbf{Q}: Let $E \in \mathbb{R}^{m \times n}$, $n\geq 2$ be a non-zero matrix such that $E^{3}=0$. Then $I+E^2$ is a singular matrix. \\
Which of the following statements is TRUE?
\hfill{2022-MA}

	\begin{enumerate}
		\item Both \textbf{P} and \textbf{Q} are TRUE
		\item Both \textbf{P} and \textbf{Q} are FALSE
		\item \textbf{P} is TRUE and \textbf{Q} is FALSE
		\item \textbf{P} is FALSE and \textbf{Q} is TRUE
	\end{enumerate}

\item Consider the real function of two real variables given by
	\begin{align*}
		u\brak{x,y}=e^{2x}\cbrak{\sin 3x \cos 2y \cosh 3y - \cos 3x \sin 2y \sinh 3y}
	\end{align*}
Let $v\brak{x,y}$ be the harmonic conjugate of $u\brak{x,y}$ such that $v\brak{0,0}=2.$ Let $z=x+iy$ and $f\brak{z}=u\brak{x,y}+iv\brak{x,y}$, then the value of $4+2if\brak{i \pi}$ is

		\hfill{2022-MA}

	\begin{enumerate}
		\item $e^{3 \pi}+e^{-3 \pi}$
		\item $e^{3 \pi}-e^{-3 \pi}$
		\item $-e^{3 \pi}+e^{-3 \pi}$
		\item $-e^{3 \pi}-e^{-3 \pi}$
	\end{enumerate}

\item The value of the integral 
	\begin{align*}
		\int_{C} \frac{z^{100}}{z^{101}+1} dz
	\end{align*}
	where $C$ is the circle of radius 2 centred at the origin taken in the anti-clockwise
direction is

		\hfill{2022-MA}

	\begin{enumerate}
		\item $-2 \pi i$
		\item $2 \pi$
		\item 0
		\item $2 \pi i$
	\end{enumerate}

\item Let $X$ be a real normed linear space. Let $X_0 = \cbrak{ x \in X: \norm{x}=1}$. If $X_0$ contains two distinct points $x$ and $y$ nd the line segment joining them, then, which of the following statements is TRUE?
	
	\hfill{2022-MA}

	\begin{enumerate}
		\item $\norm{x+y}=\norm{x}+\norm{y}$ and $x$, $y$ are linearly independent
		\item $\norm{x+y}=\norm{x}+\norm{y}$ and $x$, $y$ are linearly dependent
		\item $\norm{x+y}^{2}=\norm{x}^{2}+\norm{y}^{2}$ and $x$, $y$ are linearly independent
		\item $\norm{x+y}=2\norm{x} \norm{y}$ and $x$, $y$ are linearly dependent
	\end{enumerate}

\item Let $\cbrak{e_k : k \in \mathbb{N}}$ be an orthonormal basis for a Hilbert space $H$. Define $f_k = e_k +e_{k+1}, k\in \mathbb{N}$ and $g_j = \sum^{j} _{n=1} \brak{-1}^{n+1} e_n, j \in \mathbb{N}$. Then $\sum _{k=1} ^{\infty} \abs{\langle g_j, f_k \rangle}^{2}$=
	
	\hfill{2022-MA}

	 \begin{enumerate}
		\item 0
		\item $j^2$
		\item 4$j^2$
		\item 1
	 \end{enumerate}

 \item Consider $\mathbb{R}^2$ with the usual metric. Let $A=\cbrak{\brak{x,y} \in \mathbb{R}^2 : x^2+y^2 \leq 1}$ and $B=\cbrak{\brak{x,y} \in \mathbb{R}^2 : (x-2)^2+y^2 \leq 1}$. Let $M=A \cup B$ and $N=\text{interior}\brak{A} \cup \text{interior}\brak{B}$. Then, which of the following statements is TRUE?
	 
	 \hfill{2022-MA}

	 \begin{enumerate}
		\item $M$ and $N$ are connected
		\item Neither $M$ nor $N$ is connected
		\item $M$ is connected and $N$ is not connected
		\item $M$ is not connected and $N$ is connected
	\end{enumerate}

\item The real sequence generated by the iterative scheme
	\begin{align*}
		x_n=\frac{x_{n-1}}{2}+\frac{1}{x_{n-1}}, \, n\geq 1
	\end{align*}
	
	\hfill{2022-MA}

 	
	\begin{enumerate}
		\item converges to $\sqrt{2}$, for all $x_0 \in \mathbb{R}$ \ $\cbrak{0}$
		\item converges to $\sqrt{2}$, whenever $x_0 > \sqrt{\frac{2}{3}}$
		\item converges to $\sqrt{2}$, whenever $x_0 \in \brak{-1,1}$ \ $\cbrak{0}$
		\item diverges for any $x_0 \neq 0$
	\end{enumerate}

\item The initial value problem 
	\begin{align*}
		\frac{dy}{dx}= \cos \brak{xy}, \, x \in \mathbb{R}, \, y\brak{0} =y_0,
	\end{align*}
 where $y_0$ is a real constant, has

\hfill{2022-MA}


 	\begin{enumerate}
		\item a unique solution
		\item exactly two solutions
		\item infinitely many solutions
		\item no solution
	\end{enumerate}

\item If eigenfunctions corresponding to distinct eigenvalues $\lambda$ of the Sturm-Liouville problem 
	\begin{align*}
		\frac{d^2 y}{dx^2}-3\frac{dy}{dx}= \lambda y, \, 0<x<\pi, \\
		y\brak{0}=y\brak{\pi}=0
	\end{align*}
are orthogonal with respect to the weight function $w\brak{x}$, then $w\brak{x}$ is

\hfill{2022-MA}

	\begin{enumerate}
		\item $e^{-3x}$
		\item $e^{-2x}$
		\item $e^{2x}$
		\item $e^{3x}$
	\end{enumerate}

\item The steady state solution for the heat equation
	\begin{align*}
		\frac{\partial u}{\partial t}-\frac{\partial^2 u}{\partial x^2}=0, \, 0<x<2, \, t>0,
	\end{align*}
	with the initial condition $u\brak{x,0}=0, 0<x<2$ and the boundary conditions $u\brak{0,t}=1$ and $u\brak{2,t}=3$, $t>0$, at $x=1$ is
	
	\hfill{2022-MA}


	\begin{enumerate}
		\item 1
		\item 2
		\item 3
		\item 4
	\end{enumerate}

\item Consider $\brak{\cbrak{0,1}, T_1}$, where $T_1$ is the subspace topology induced by the Euclidean topology on $\mathbb{R}$, and let $T_2$ be any topology on $\cbrak{0,1}$. Consider the following statements: \\
	\textbf{P}: If $T_1$ is a proper subset of $T_2$, then $\brak{\cbrak{0,1}, T_2}$ is not compact. \\
	\textbf{Q}: If $T_2$ is a proper subset of $T_1$, then $\brak{\cbrak{0,1}, T_2}$ is not Hausdorff. \\
	Then
	
	\hfill{2022-MA}

	\begin{enumerate}
		\item Both \textbf{P} and \textbf{Q} are TRUE
		\item Both \textbf{P} and \textbf{Q} are FALSE
		\item \textbf{P} is TRUE and \textbf{Q} is FALSE
		\item \textbf{P} is FALSE and \textbf{Q} is TRUE
	\end{enumerate}

\item Let $p:\brak{\cbrak{0,1}, T_1} \to \brak{\cbrak{0,1}, T_2}$ be the quotient map, arising from the characteristic function on $\cbrak{ \frac{1}{2}, 1}$, where $T_1$ is the subspace topology induced by the Euclidean topology on $\mathbb{R}$. Which of the following statements is TRUE?
	
	\hfill{2022-MA}

	\begin{enumerate}
		\item $p$ is an open map but not a closed map
		\item $p$ is a closed map but not an open map
		\item $p$ is a closed map as well as an open map
		\item $p$ is neither an open map nor a closed map
	\end{enumerate}



%\end{enumerate}
%\end{document}

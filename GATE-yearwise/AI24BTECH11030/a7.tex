\iffalse
\chapter{2008}
\author{AI24BTECH11030}
\section{ma}
\fi
    

\item {Consider the subspace $W=\{[a_{ij}]: a_{ij}=0 \text{ if } i \text{ is even}\}$ of all 10x10 real matrices. Then the dimension of $W$ is \hfill (GATE 2008)

\begin{multicols}{4}
    \begin{enumerate}
        \item $25$
        \item $50$
        \item $75$
        \item $100$
    \end{enumerate}
\end{multicols}}

\item { Let $S$ be the open unit disk and $f: S\rightarrow C$ be a real-valued analytic function with $f(0) = 1$. Then the set $\{z \in S: f(z) \ne 1\}$ is \hfill (GATE 2008)

\begin{multicols}{2}
    \begin{enumerate}
        \item{empty}
        \item{nonempty finite }
        \item{countably infinite}
        \item{uncountable}
    \end{enumerate}
\end{multicols}
}

\item{ Let $E=\{(x,y)\in R^2:0\leq x\leq1,0\leq y\leq x\}$. Then $\int_E\int(x+y) \: dx \: dy$ is equal to \hfill (GATE 2008)
\begin{multicols}{4}
\begin{enumerate}
    \item $-1$
    \item $0$
    \item $112$
    \item $1$
\end{enumerate}
\end{multicols}
}

\item {
For $(x, y) \in R^2$, let
$f(x,y) =$ $\begin{cases}\frac{2xy}{x^2+y^2} & \text{if } (x,y)=(0,0)\\
    0 & \text{if } (x, y)=(0,0)
\end{cases}$
Then \hfill (GATE 2008)

\begin{enumerate}
    \item $f_x$ and $f_y$ exist at (0,0), and $f$ is continuous at (0,0) 
    \item $f_x$ and $f_y$ exist at (0,0), and $f$ is discontinuous at (0,0)
    \item $f_x$ and $f_y$ do not exist at (0,0), and $f$ is continuous at (0,0)
    \item $f_x$ and $f_y$ do not exist at (0,0), and $f$ is discontinuous at (0,0)
\end{enumerate}
}

\item { Let $y$ be a solution of $y^\prime=e^{-y^2}-1$ on \sbrak{0,1} which satisfies $y(0) = 0$. Then \hfill (GATE 2008)
\begin{multicols}{2}
\begin{enumerate}
    \item $y(x)>0$ for $x>0$
    \item $y$ changes sign in \sbrak{0,1}
    \item $y(x) <0$ for $x>0 $
    \item $y=0$ for $x>0$
\end{enumerate}
\end{multicols}
}

\item{ For the equation $x(x-1)y^{\prime\prime}+sin(x)y^\prime+2x(x-1)y=0$ , consider the following statements  \hfill (GATE 2008)
    \begin{itemize}
        \item[P:] $x=0$ is a regular singular point. 
        \item[Q:] $x=1$ is a regular singular point.
    \end{itemize}
    
Then
\begin{multicols}{2}
\begin{enumerate}
    \item both P and Q are true
    \item P is true but Q is false
    \item P is false but Q is true
    \item both P and Q are false
\end{enumerate}
\end{multicols}
}

\item{ Let $G = R \backslash \{0\}$ and $H=\{-1, 1\}$ be groups under multiplication. Then the map $\phi: G \rightarrow H$ defined by $\phi(x) = \frac{x}{\abs{x}}$ is \hfill (GATE 2008)
\begin{enumerate}
    \item not a homomorphism
    \item a one-one homomorphism, which is not onto 
    \item an onto homomorphism, which is not one-one 
    \item an isomorphism
\end{enumerate}
}



\item {The number of maximal ideals in $\mathbb{Z}_{27}$ is   \hfill (GATE 2008)
\begin{multicols}{4}
    \begin{enumerate}
        \item $0$
        \item $1$
        \item $2$
        \item $3$
    \end{enumerate}
\end{multicols}
}

\item {For $1\leq p \leq\infty$, let $\Vert\:\Vert_p$ denote the $p$-norm on $\mathbb{R}^2$. If $\Vert\:\Vert_p$ satisfies the parallelogram law, then $p$ is equal to \hfill (GATE 2008)
\begin{multicols}{4}
    \begin{enumerate}
        \item $1$
        \item $2$
        \item $3$
        \item $\infty$
    \end{enumerate}
\end{multicols}
}

\item { Consider the initial value problem $\frac{dy}{dx} = f(x, y), y(x_0) = y_0$. The aim is to compute the value of $y_1 = y(x_1)$, where $x_1 = x_0 + h (h>0)$. At $x=x_1$, if the value of $y_1$ is equated to the corresponding value of the straight line passing through $(x_0, y_0)$ and having the slope equal to the slope of the curve $y(x)$ at $x = x_0$, then the method is called  \hfill (GATE 2008)

\begin{multicols}{2}
\begin{enumerate}
    \item Euler's method
    \item Improved Euler's method
    \item Backward Euler's method
    \item Taylor series method of order 2
\end{enumerate}
\end{multicols}
}

\item{The solution of $xu_x + yu_y =0$ is of the form \hfill (GATE 2008)
\begin{multicols}{4}
    \begin{enumerate}
        \item $f(y/x)$
        \item $f(x+y) $
        \item $f(x-y)$
        \item $f(x)$
    \end{enumerate}
\end{multicols}}

\item{ If the partial differential equation $(x-1)^2 u_{xx}-(y-2)^2u_{yy}, +2xu_x + 2yu_y +2xyu=0$ is parabolic in $S \subseteq \mathbb{R}^2$ but not in $\mathbb{R}^2\backslash S$, then $S$ is \hfill (GATE 2008)
\begin{multicols}{2}
    \begin{enumerate}
        \item $\{(x, y) \in \mathbb{R}:x=1 \text{ or } y=2\}$
        \item $\{(x,y) \in \mathbb{R}:x=1\}$
        \item $\{(x, y) \in \mathbb{R}^2:x=1 \text{ and } y = 2\}$
        \item $\{(x, y) \in \mathbb{R}^2: y=2\}$
    \end{enumerate}
\end{multicols}




}

\item{ Let $E$ be a connected subset of $\mathbb{R}$ with at least two elements. Then the number of elements in $E$ is \hfill (GATE 2008)
\begin{multicols}{2}
    \begin{enumerate}
        \item exactly two
        \item countably infinite
        \item more than two but finite 
        \item uncountable
    \end{enumerate}
\end{multicols}
}

\item{ Let $X$ be a non-empty set. Let $\epsilon_1$, and $\epsilon_2$, be two topologies on $X$ such that $\epsilon_1$, is strictly contained in $\epsilon_2$. If $I: (X,\epsilon_1)\rightarrow (X,\epsilon_2)$ is the identity map, then \hfill (GATE 2008)
    \begin{enumerate}
        \item both $I$ and $I^{-1}$ are continuous
        \item $I$ is continuous but $I^{-1}$ is not continuous
        \item both $I$ and $I^{-1}$ are not continuous
        \item $I$ is not continuous but $I^{-1}$ is continuous
    \end{enumerate}
}

\item{ Let $X_1, X_2, \ldots X_{10}$ be a random sample from $N(80,3^2)$ distribution. Define 
$$S=\sum_{i=1}^{10}U_i \text{ and } T=\sum_{i=1}^{10}\brak{U_i - \frac{S}{10}}^2$$

where $U_i = \frac{X_i - 80}{3}, i=1,2,\ldots,10$. Then the value of $E(ST)$ is equal to \hfill (GATE 2008)
\begin{multicols}{4}
    \begin{enumerate}
        \item $0$
        \item $1$
        \item $10$
        \item $\frac{80}{3}$
    \end{enumerate}
\end{multicols}
}

\item{ Two (distinguishable) fair coins are tossed simultaneously. Given that ONE of them lands up head, the probability of the OTHER to land up tail is equal to \hfill (GATE 2008)
\begin{multicols}{4}
    \begin{enumerate}
        \item $\frac{1}{3}$
        \item $\frac{1}{2}$
        \item $\frac{2}{3}$
        \item $\frac{3}{4}$
    \end{enumerate}
\end{multicols}
}

\item{ Let $c_{ij}\geq2$ be the cost of the $(i, j)^{th}$ cell of an assignment problem. If a new cost matrix is generated by the elements $c_{ij}^*= \frac{1}{2} c_{ij} +1$, then \hfill (GATE 2008)

\begin{enumerate}
    \item optimal assignment plan remains unchanged and cost of assignment decreases 
    \item optimal assignment plan changes and cost of assignment decreases
    \item optimal assignment plan remains unchanged and cost of assignment increases 
    \item optimal assignment plan changes and cost of assignment increases
\end{enumerate}
}

\iffalse
\chapter{2009}
\author{AI24BTECH11030}
\section{xe}
\fi

    \item Correlate the material properties given in Column I with the units given in Column II. \hfill (GATE XE 2009)

    \begin{table}[h!]
    \centering
    \begin{tabular}{l | l}
    \textbf{Column I}           & \textbf{Column II} \\
    \hline
    P. Magnetic moment          & 1. MN m$^{3/2}$ \\
    Q. Thermal conductivity      & 2. H m$^{-1}$ \\
    R. Fracture toughness        & 3. A m$^{2}$ \\
    S. Electron mobility         & 4. m$^{2}$ V$^{-1}$ s$^{-1}$ \\
                                & 5. J s$^{-1}$ m$^{-1}$ K$^{-1}$ \\
    \end{tabular}
    \end{table}

    \begin{multicols}{2}
        \begin{enumerate}
            \item P-2, Q-5, R-1, S-4
            \item P-4, Q-5, R-1, S-3
            \item P-3, Q-5, R-1, S-4
            \item P-3, Q-2, R-4, S-1
        \end{enumerate}
    \end{multicols}

    \item If the spacing between two consecutive (110) planes in a BCC material is 0.203 nm, the lattice parameter and radius of the atom of the said material will be  \hfill (GATE XE 2009)

    \begin{multicols}{2}
        \begin{enumerate}
            \item 0.242 nm and 0.110 nm
            \item 0.242 nm and 0.120 nm
            \item 0.287 nm and 0.134 nm
            \item 0.287 nm and 0.124 nm
        \end{enumerate}
    \end{multicols}

    \item A continuous and aligned carbon fibre reinforced composite is made up of 30 vol\% carbon fibre having a modulus of elasticity of 300 GPa dispersed in a polymer matrix which on hardening has a modulus of elasticity of 4 GPa. What will be the modulus of elasticity of the composite in longitudinal and transverse directions of the carbon fibres respectively? \hfill (GATE XE 2009)

    \begin{multicols}{2}
        \begin{enumerate}
            \item 92.8 GPa and 5.7 GPa
            \item 211.0 GPa and 9.3 GPa
            \item 304.0 GPa and 7.5 GPa
            \item 92.8 GPa and 6.7 GPa
        \end{enumerate}
    \end{multicols}

    \item A potential of 10 volts is applied across a parallel plate capacitor which has a plate area of $10^{-4} \ \text{m}^2$ and a plate separation of $2 \times 10^{-3} \ \text{m}$. If dielectric constant of the material placed between parallel plates is 10, the capacitance and the magnitude of the charge stored between the plates will be \hfill (GATE XE 2009)

    \begin{enumerate}
        \item $4.425 \times 10^{-13} \text{F}$ and $4.425 \times 10^{-12} \text{C}$
        \item $8.850 \times 10^{-13} \ \text{F}$ and $8.850 \times 10^{-12} \ \text{C}$
        \item $4.425 \times 10^{-12} \ \text{F}$ and $4.425 \times 10^{-11} \ \text{C}$
        \item $8.850 \times 10^{-12} \ \text{F}$ and $8.850 \times 10^{-11} \ \text{C}$
    \end{enumerate}

    \item Conductivity of Silicon at 300 K is $3.16 \times 10^{-4} \ \text{ohm}^{-1} \ \text{m}^{-1}$ and that of Germanium is $2.12 \times 10^{-2} \ \text{ohm}^{-1} \ \text{m}^{-1}$ at 300 K. At what temperature would the conductivity of intrinsic Silicon be the same as the conductivity of intrinsic Germanium at 300 K? (Given: $E_g$ of Silicon at 300 K $= 1.12 \ \text{eV}$, $E_g$ of Germanium at 300 K $= 0.72 \ \text{eV}$) \hfill (GATE XE 2009)

    \begin{multicols}{4}
        \begin{enumerate}
            \item $\sim 506 \ \text{K}$
            \item $\sim 606 \ \text{K}$
            \item $\sim 726 \ \text{K}$
            \item $\sim 816 \ \text{K}$
        \end{enumerate}
    \end{multicols}

    \item Molecular weight distribution of a polystyrene polymer and the number fraction of polymer chains in the molecular weight range are given below.

    \begin{table}[h!]
    \centering
    \begin{tabular}{|c|c|}
    \hline
    \textbf{Range of Molecular weight (kg/mol)} & \textbf{Number fraction of polymer chain} \\ \hline
    5 - 10   & 0.05 \\ \hline
    10 - 15  & 0.15 \\ \hline
    15 - 20  & 0.20 \\ \hline
    20 - 25  & 0.30 \\ \hline
    25 - 30  & 0.20 \\ \hline
    30 - 35  & 0.08 \\ \hline
    35 - 40  & 0.02 \\ \hline
    \end{tabular}
    \end{table}

    The number average molecular weight and the number average degree of polymerization will be \hfill (GATE XE 2009)

    \begin{multicols}{2}
        \begin{enumerate}
            \item 15.750 kg/mol and 151
            \item 21.350 kg/mol and 203
            \item 15.750 kg/mol and 302
            \item 21.350 kg/mol and 205
        \end{enumerate}
    \end{multicols}


    \item[] \textbf{Common Data for Questions 19 and 20:}

    Nickel has FCC structure and its lattice parameter is 0.353 nm. Weight of one mole of Nickel is 0.05871 kg.

    \item The Ni-Ni nearest neighbour distance (in nm) is \hfill (GATE XE 2009)

    \begin{multicols}{4}
        \begin{enumerate}
            \item 0.173
            \item 0.223
            \item 0.250
            \item 0.273
        \end{enumerate}
    \end{multicols}

    \item Theoretical density of Nickel (in kg m$^{-3}$) is closer to \hfill (GATE XE 2009)

    \begin{multicols}{4}
        \begin{enumerate}
            \item 8700
            \item 8900
            \item 9100
            \item 9300
        \end{enumerate}
    \end{multicols}

    \item[] \textbf{Common Data for Questions 21 and 22:}
    
    The diffusivity of lithium in Silicon is $10^{-9}$ m$^2$ s$^{-1}$ at 1400 K and $10^{-10}$ m$^2$ s$^{-1}$ at 1000 K.
    
    \item The value of activation energy (J mol$^{-1}$) of lithium diffusion in silicon is \hfill (GATE XE 2009)
    \begin{multicols}{4}
        \begin{enumerate}
            \item 66086
            \item 66986
            \item 67086
            \item 67986
        \end{enumerate}
    \end{multicols}

    \item The value of jump frequency factor of lithium in silicon in m$^2$ s$^{-1}$ is \hfill (GATE XE 2009)
    \begin{multicols}{4}
        \begin{enumerate}
            \item 2.15 $\times$ 10$^{-7}$
            \item 3.15 $\times$ 10$^{7}$
            \item 3.15 $\times$ 10$^{-8}$
            \item 2.15 $\times$ 10$^{-8}$
        \end{enumerate}
    \end{multicols}


    \item[] \textbf{Statement for Linked Answer Questions 23 and 24:}

    Aluminum has a density of 2710 kg m$^{-3}$ and weight of one mole of aluminum is 0.02698 kg. The collision time, $\tau$, for electron scattering in Aluminum is 2 $\times$ 10$^{-14}$ s at 300 K.

    \item The number of free electrons per m$^3$ of Aluminum at 300 K is \hfill (GATE XE 2009)
    \begin{multicols}{4}
        \begin{enumerate}
            \item 6.05 $\times$ 10$^{28}$
            \item 7.05 $\times$ 10$^{28}$
            \item 6.05 $\times$ 10$^{27}$
            \item 7.05 $\times$ 10$^{27}$
        \end{enumerate}
    \end{multicols}

    \item The conductivity of aluminum (ohm$^{-1}$ m$^{-1}$) at 300 K is \hfill (GATE XE 2009)
    \begin{multicols}{4}
        \begin{enumerate}
            \item 3.40 $\times$ 10$^{6}$
            \item 4.40 $\times$ 10$^{6}$
            \item 3.40 $\times$ 10$^{7}$
            \item 4.40 $\times$ 10$^{7}$
        \end{enumerate}
    \end{multicols}

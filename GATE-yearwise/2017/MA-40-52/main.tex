\iffalse
\title{GATE Questions 12}
\author{EE24BTECH11012 - Bhavanisankar G S}
\section{ma}
\chapter{2017}
\fi
%\begin{enumerate}
	\item Connsider the subspace $V = \cbrak{ \brak{x_n} \in l^2 : \sum_{n=1}^{infty} \abs{x_n} \leq \infty}$ of the Hilbert space $l^2$ of all square summable real sequences. For $n \in \mathbb{N}$, define $T_n = V \to \mathbb{R} $ by $T_n\brak{x_n} = \sum_{i=1}^{n} x_l $. Consider the following statements. \\
		P: $\cbrak{T_n : n \in \mathbb{N}}$ is pointwise bounded on V \\
		Q: $\cbrak{T_n : n \in \mathbb{N}}$ is uniformly bounded on $\cbrak{x \in V : \norm{x}_2 = 1}$ \\
	Which of the following statements holds good ?
		\begin{enumerate}
				\begin{multicols}{2}
			\item Both P and Q
			\item Only P
			\item Only Q
			\item Neither P nor Q
				\end{multicols}
		\end{enumerate}
	\item Let p(x) be the polynomial of degree at most 2 that interpolates the data (-1,2), (0,1) and (1,2). If q(x) is a polynomial of degree at most 3 such that p(x) + q(x) interpolates the data (-1,2), (0,1), (1,2) and (2,11), then q(3) equals
	\item Let J be the Jacobi iteration matrix of the linear system 
		$$ \myvec{1 && 2 && 1 \\ 2 && 1 && 2 \\ -4 && 2 && 1} \myvec{x \\ y \\ z} = \myvec{1 \\ 2 \\ 3} $$
		Consider the following statements : \\
		P: One of the eigen values of J lies in the interval [2,3] \\
		Q: The jacobi iteration converges for the above system. \\
		Which of the above statements hold good ?
		\begin{enumerate}
				\begin{multicols}{2}
				\item Both P and Q
				\item Only P
				\item Only Q
				\item Neither P nor Q
				\end{multicols}
		\end{enumerate}
	\item Let $u(x,y)$ be the solution of $x \frac{\partial u}{\partial x} + y \frac{\partial u}{\partial y} = 4u$ satisfying the consition $u(x,y)=1$ on the circle $x^2 + y^2 = 1$. Then $u(2,2)$ equals $\underline{   }$
	\item Let $u(r, \theta)$ be the bounded solution of the following boundary value problem in polar coordinates :
		$$ r^2 \frac{\partial^2 u}{\partial r^2} + r \frac{\partial u}{\partial r} + \frac{\partial ^2 u}{\partial \theta ^2} = 0, 0 \leq r \leq 2, 0 \leq \theta \leq 2 \pi$$
		$$ u(2,\theta) = \cos^2{\theta}, 0 \leq \theta \leq 2 \pi$$
		Then $u(1, \frac{\pi}{2}) + u(1, \frac{\pi}{4})$ equals
		\begin{enumerate}
				\begin{multicols}{4}
				\item 1
				\item $\frac{9}{8}$
				\item $\frac{7}{8}$
				\item $\frac{3}{8}$
				\end{multicols}
		\end{enumerate}
	\item Let $T_u$ and $T_d$ denote the usual topology and the discrete topology on $\mathbb{R}$ respectively. Consider the following topologies : \\
		$T_1$: Usual topology on $\mathbb{R}^2 = \mathbb{R} \times \mathbb{R} $ \\
		$T_2$: Topology generated by  the basis $\cbrak{U \times V : U \in T_d, V \in T_u}$ on $\mathbb{R} \times \mathbb{R}$ \\
		$T_3$: Dictionary order topology on $\mathbb{R} \times \mathbb{R}$ . Then
		\begin{enumerate}
				\begin{multicols}{2}
				\item $T_3 \subset T_1 \subseteq T_2 $
				\item $T_1 \subset T_2 \subseteq T_3 $
				\item $T_3 \subset T_2 \subseteq T_1 $
				\item $T_1 \subset T_2 \subseteq T_3 $
				\end{multicols}
		\end{enumerate}
	\item Let X be a random variable with probability mass function 
		$$ p(n) = \brak{\frac{3}{4}}^{n-1} \brak{\frac{1}{4}} $$ for $n = 1, 2, \dots $. Then E(X-3 | X $\geq$ 3) equals
	\item Let X and Y be independent and identically distributed random variables with probability mass function $p(n) = 2^{-n}, n=1,2,\dots$. Then P(X $\geq$ 2Y) equals (rounded off to two decimals)
	\item Let $X_1, X_2, \dots$ be a sequence of independent and identically distributed Poisson random variables with mean 4. Then 
		$$ \lim_{n \to \infty} P \brak{4 - \frac{2}{\sqrt{n}} \leq \frac{1}{n} \sum_{i=1}^{n} X_i \leq 4 + \frac{2}{\sqrt{n}}} $$ equals
	\item Let X and Y be independent and identically distributed exponential random variables with probability density function 
		$$ f(x) = e^{-x} $$ for all positive x. Then $P \brak{max(X,Y) \leq 2} $ equals (rounded to 2 decimals)
	\item Let E and F be any two events with $P(E) = 0.4$, $P(F) = 0.3$ and $P(F|E) = 3P(F|E^C)$. Then $P(E|F)$ equals (rounded to 2 decimals)
	\item Let $X_1, X_2, \dots, X_m$ be a random sample from a binomial distribution with parameters $n=1$ and $p$, $p \in \brak{0,1}$, and let $\overline{X} = \frac{1}{m} \sum_{i=1}^{m} X_i$. Then a uniformly minimum variance unbiased estimator for $p(1-p)$ is
		\begin{enumerate}
				\begin{multicols}{2}
				\item $\frac{m}{m-1} \overline{X} \brak{1 - \overline{X}}$
				\item $\overline{X} \brak{1 - \overline{X}}$
				\item $\frac{m-1}{m} \overline{X} \brak{1 - \overline{X}}$
				\item $\frac{1}{m} \overline{X} \brak{1 - m \overline{X}}$
				\end{multicols}
		\end{enumerate}
	\item Let $X_1, X_2, \dots, X_9$ be a random variable from a $N \brak{0, \sigma ^2}$ population. For testing $H_0 : \sigma ^2 = 2$ against $H_1 : \sigma ^2 = 1$, the most  powerful test rejects $H_0$ if $\sum_{i=1}^{9} X_i \leq c $ where $c$ is to be chosen such that the level of significace is 0.1 Then the power of this test equals


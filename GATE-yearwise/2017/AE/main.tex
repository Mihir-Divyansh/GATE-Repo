
\iffalse
\chapter{2017}
\author{AI24BTECH11008}
\section{ae}
\fi

    \item The period of revolution of earth about the sun is 365.256 days, approximately. The semi major axis of the earth's orbit is close to $1.4953\times 10^{11}$m. The smi major axis of the orbit of Mars is $2.2783\times 10^{11}$m. The period of revolution of Mars, about the sun, is x Earth days. What is the value of x(in three decimal place).\hfill (2017)
    \item Consider a system consisting of certain amount of perfect gas enclosed in a cylinder fitted with a frictionless piston. This system can undergo following process:\hfill (2017)
    \begin{enumerate}
        \item Expansion with finite pressure difference with the surroundings.
        \item Compression with infinitesimal pressure difference with the surroundings.
        \item Heat transfer with finite temperature difference with the reservoir.
        \item Heat transfer with infinitesimal temperature difference with the reservoir.
    \end{enumerate}
    Out of these which processes are reversible?
    \begin{enumerate}[label=(\Alph*)]
        \item a and c
        \item a and d 
        \item b and c 
        \item b and d 
    \end{enumerate}
    
    \item Among the following engines, which one is expected to have the maximum Specific Impulse?\hfill (2017)
    \begin{enumerate}[label=(\Alph*)]
        \item Cryogenic Rocket
        \item Solid Propellant Rocket
        \item Liquid propellant Rocket
        \item SCRAM jet
    \end{enumerate}
    \item The maximum gas flow rate that can be handled by a multistage axial compressor at a given rotational speed is dictated by \hfill (2017)
    \begin{enumerate}[label=(\Alph*)]
        \item Compressor Surge
        \item Rotating Stall
        \item Choking
        \item Optimum Design Pressure Ratio 
    \end{enumerate}
    \item For a turbine stage, which one of the following losses occurs due to the turning of the wall boundary layer through an angle due to curved surface? \hfill (2017)
    \begin{enumerate}[label=(\Alph*)]
        \item Profile Loss 
        \item Annulus Loss 
        \item Tip clearance loss 
        \item Secondary flow loss 
    \end{enumerate}
    \item In the vane-less space between the impeller and the diffuser vanes in a Centrifugal Compressor, the angular momentum varies in the following manner in the radial direction. \hfill (2017)
    \begin{enumerate}[label=(\Alph*)]
        \item Increases
        \item Remains constant
        \item Decreases
        \item First Increases and then Decreases
    \end{enumerate}
    \item Which of the following statements about the neutral axis of a beam with unsymmetrical cross section is true: \hfill (2017)
    \begin{enumerate}[label=(\Alph*)]
        \item The product of second moment of area about the neutral axis is always zero.
        \item The normal stress along the neutral axis is always zero.
        \item The shear stress along the neutral axis is always zero.
        \item The product of second momentum of area about the neutral axis and the normal stress about the neutral axis are always zero.
    \end{enumerate}
    \item Assuming that the aircraft is flying straight, the top spar cap/ flange of wing is most likely to fail in: \hfill (2017)
    \begin{figure}[!ht]
        \centering
        \resizebox{0.4\textwidth}{!}{%
        \begin{circuitikz}
        \tikzstyle{every node}=[font=\normalsize]
        \draw [short] (3.75,10) -- (10.75,9);
        \draw [short] (10.75,9) -- (3.75,8);
        \draw [short] (3.75,10) .. controls (2.5,9.5) and (2.25,8.75) .. (3.75,8);
        \draw [short] (5.25,9.75) -- (5.25,8.25);
        \draw [short] (7.25,9.5) -- (7.25,8.5);
        \draw [->, >=Stealth] (4.75,9.25) -- (4.75,9.75);
        \draw [short] (4.75,9.25) -- (4.75,7.75);
        \draw [short] (4.75,7.75) -- (5.75,7.75);
        \node [font=\normalsize] at (7.25,7.75) {Top spar cap/flange};
        \end{circuitikz}
        }% % Specify the path to your TikZ file
        \caption{1}
        %\label{fig2}
    \end{figure}
    \begin{enumerate}[label=(\Alph*)]
        \item  Yielding
        \item  Buckling 
        \item  Crushing
        \item  Creep
    \end{enumerate} 
    
    \item A 2-DOF undamped spring mass sysytem with two masses and two springs has natural frequencies $\omega_1 = 0.79\frac{rad}{s}$ and $\omega_2 = 1.538\frac{rad}{s}$. The mode shapes for the system are given by $\phi_1 = [0.732]^T$ and $\phi_2 = [-2.731]^T$. If the first mass is displaced by 1cm, the mimimum displacement in cms to be given to second mass to make the system vibrate in first mode alone is  (in three decimal places)\hfill (2017)
    
    \item An aircraft landing gear can be idealized as a single degree of freedom spring-mass-damper system. The desirable damping characteristics of such a system is  \hfill (2017)
     \begin{enumerate}[label=(\Alph*)]
        \item Under damped
        \item Over damped
        \item Critically damped
        \item Undamped
     \end{enumerate}
    \item A single degree of freedom spring-mass system of natural frequency 5Hz is modified in the following manners:
    \\ Case 1: Viscous damping with damping ratio $\zeta = 0.2$ is introduced in parallel to the spring
    \\ Case 2: The original undamped spring-mass system is moved to a surface with coefficient of fraction, $\mu = 0.01$\\
    The ratio of damped natural frequency for the cases 1 and 2 is given by (in three decimal places).\hfill (2017)
    \item  Which of the following statements about the compatibility equations are true:\hfill (2017)
    \begin{enumerate}
        \item Strain compatibility equations must be satisfied in the solution of three-dimensional problems in elasticity.
        \item Six strains are defined in terms of three displacement functions and can have arbitrary values.
        \item Compatibility equations are an expression of the continuity of displacements.
    \end{enumerate}
    \begin{enumerate}[label=(\Alph*)]
        \item a and b 
        \item b and c 
        \item a and c 
        \item a, b and c 
    \end{enumerate}
    
    \item Matrix [A] = $\begin{bmatrix} 2&0&2\\3&2&7\\3&1&5\end{bmatrix}$ and vector \{b\} = $\begin{bmatrix}4\\4\\5\end{bmatrix}$ are given. If vector \{x\} is the solution to the system of equations [A]\{x\}=\{b\}, which of the following is true for \{x\}: \hfill (2017)
    \begin{enumerate}[label=(\Alph*)]
        \item Solution does not exist
        \item Infinite solutions exist
        \item Unique solution exists 
        \item Five possible solutions exist 
    \end{enumerate} 


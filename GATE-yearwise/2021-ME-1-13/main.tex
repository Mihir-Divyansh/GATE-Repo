\iffalse
\title{2021-ME-1-13}
\author{EE24BTECH11001 -  ADITYA TRIPATHY}
\section{me}
\chapter{2021}
\fi
    \item 
        If $y\brak{x}$ satisfies the differential equation $\brak{\sin x}\frac{dy}{dx} + y\cos x = 1$,
        subject to the domain y$\brak{\frac{\pi}{2}} = \frac{\pi}{2}$, then y$\brak{\frac{\pi}{2}}$
        is 
        \hfill{\brak{2021-ME}}
        \begin{multicols}{4}
            \begin{enumerate}
                \item 0
                    \columnbreak
                \item $\frac{\pi}{6}$
                    \columnbreak
                \item $\frac{\pi}{3}$
                    \columnbreak
                \item $\frac{\pi}{2}$
            \end{enumerate}
        \end{multicols}
    \item The value of $\lim_{x \rightarrow 0} \frac{1 - \cos x}{x^2}$ is
        \hfill{\brak{2021-ME}}
        \begin{multicols}{4}
            \begin{enumerate}
                \item $\frac{1}{4}$
                    \columnbreak
                \item $\frac{1}{3}$
                    \columnbreak
                \item $\frac{1}{2}$
                    \columnbreak
                \item $1$
            \end{enumerate}
        \end{multicols}

    \item The Dirac-Delta function $\delta\brak{t -t_0}$ for $t, t_0, \in \mathbb{R}$, has the following
        property 
        \begin{align}
            \int_a^b \phi\brak{t}\delta{t - t_0} \, dt = \begin{cases}
                \phi\brak{t_0} & a < t_0 < b\\
                0 & otherwise
            \end{cases} 
        \end{align}
        The laplace transform of the Dirac-Delta function $\delta\brak{t - a}$ for $a > 0$, 
        $\mathcal{L}\brak{\delta\brak{t - a}} = F\brak{s}$ is 
        \hfill{\brak{2021-ME}}
        \begin{multicols}{4}
            \begin{enumerate}
                \item $0$
                    \columnbreak
                \item $\infty$
                    \columnbreak
                \item $e^{sa}$
                    \columnbreak
                \item $e^{-sa}$
            \end{enumerate}
        \end{multicols}
    \item The ordinary differential equation $\frac{dy}{dx} = -\pi y$ subject to an initial condition
        $y\brak{0} = 1$ is solved numerically using the following scheme:
        \begin{align}
            \frac{y\brak{t_{n+1}} - y\brak{t_n}}{h} = -\pi y\brak{t_n} 
        \end{align}
        where $h$ is the time step, $t_n = nh$, and $n = 0, 1, 2 \cdots$ . This numerical scheme
        is stable for all values of $h$ in the interval
        \hfill{\brak{2021-ME}}
        \begin{multicols}{4}
            \begin{enumerate}
                \item $0 < h < \frac{2}{\pi}$
                    \columnbreak
                \item $0 < h < 1$
                    \columnbreak
                \item $0 < h < \frac{\pi}{2}$
                    \columnbreak
                \item for all $h > 0$
            \end{enumerate}
        \end{multicols}

\item Consider a binomial random variable $X$. If $X_1, X_2, \cdots , X_n$ are independent and 
    identically distributed samples from the distribution of $X$ with sum $Y = \sum_{i = 1}^{n}X_i$,
    then distribution of $Y$ as $n \rightarrow \infty$ can be approximated as
        \hfill{\brak{2021-ME}}
        \begin{multicols}{4}
            \begin{enumerate}
                \item Exponential
                    \columnbreak
                \item Bernoulli
                    \columnbreak
                \item Binomial
                    \columnbreak
                \item Normal
            \end{enumerate}
        \end{multicols}
    \item The loading and unloading response of a metal is shown in the figure. The elastic and
        plastic strains corresponding to $200 MPa$ stress, respectively, are
        \begin{center}
            \resizebox{0.5\textwidth}{!}{
                \begin{circuitikz}
                    \tikzstyle{every node}=[font=\normalsize]
                    \draw [->, >=Stealth] (2.75,11) -- (2.75,17.5);
                    \draw [->, >=Stealth] (2.75,11) -- (9.75,11);
                    \draw [short] (2.75,11) .. controls (3.75,14.25) and (4.75,14.5) .. (7,15.75);
                    \draw [dashed] (7,15.75) -- (2.75,15.75);
                    \draw [dashed] (7,15.75) -- (7,11);
                    \draw [short] (4.25,11) -- (7,15.75);
                    \draw [short] (4.25,14) -- (4,14);
                    \draw [short] (4.25,14) -- (4.25,13.75);
                    \draw [short] (5.25,12.75) -- (5.25,13);
                    \draw [short] (5.25,12.75) -- (5.5,12.75);
                    \node [font=\normalsize] at (4,10.5) {0.01};
                    \node [font=\normalsize] at (7,10.5) {0.03};
                    \node [font=\normalsize, rotate around={90:(0,0)}] at (2,13.25) {stress (MPa)};
                    \node [font=\normalsize] at (2,15.5) {200};
                    \node [font=\normalsize] at (9.75,10.5) {strain};
                \end{circuitikz}
            } 
        \end{center}
        \hfill{\brak{2021-ME}}

        \begin{multicols}{4}
            \begin{enumerate}
                \item 0.01 and 0.01
                    \columnbreak
                \item 0.02 and 0.01
                    \columnbreak
                \item 0.01 and 0.02
                    \columnbreak
                \item 0.02 and 0.02
            \end{enumerate}
        \end{multicols}

    \item In a machining operation, if a cutting tool traces the workpiece such that the directrix is
        perpendicular to the plane of the generatrix as shown in figure, the surface generated is 
        \begin{center}
            \resizebox{0.4\textwidth}{!}{
                \begin{circuitikz}
                    \tikzstyle{every node}=[font=\normalsize]
                    \draw  (4,17) ellipse (1.5cm and 3.25cm);
                    \draw [short] (2.5,17.5) -- (8.25,17.5);
                    \draw [->, >=Stealth] (2.5,16.5) -- (4.75,16.5);
                    \draw [->, >=Stealth] (8,19.5) -- (7,17.75);
                    \draw [->, >=Stealth] (1.5,14.5) -- (2.5,15.75);
                    \node [font=\normalsize] at (1.5,14.25) {Generatrix};
                    \node [font=\normalsize] at (8,19.75) {Directrix};
                \end{circuitikz}
            } 
        \end{center}
        \hfill{\brak{2021-ME}}
        \begin{multicols}{4}
            \begin{enumerate}
                \item plane
                    \columnbreak
                \item cylindrical
                    \columnbreak
                \item spherical
                    \columnbreak
                \item a surface of revolution
            \end{enumerate}
        \end{multicols}


    \item The correct sequence of machining operations to be performed to finish a large diameter
        through hole is
        \hfill{\brak{2021-ME}}
            \begin{enumerate}
        \begin{multicols}{2}
                \item drilling, boring, reaming
                    \columnbreak
                \item boring, drilling, reaming
        \end{multicols}
        \begin{multicols}{2}
                \item drilling, reaming, boring
                    \columnbreak
                \item boring, reaming, drilling
        \end{multicols}
            \end{enumerate}

    \item In modern CNC machine tools, the backlash has been eliminated by
        \hfill{\brak{2021-ME}}
            \begin{enumerate}
        \begin{multicols}{2}
                \item preloaded ballscrews
                    \columnbreak
                \item rack and pinion
        \end{multicols}
        \begin{multicols}{2}
                \item ratchet and pinion
                    \columnbreak
                \item slider crank mechanism
        \end{multicols}
            \end{enumerate}

        \item Consider the surface roughness profile as shown in the figure. The center line average 
            roughness ($R_a$, in $\mu$m)of the measured length (L) is
            \begin{center}
                \resizebox{0.5\textwidth}{!}{
                    \begin{circuitikz}
                        \tikzstyle{every node}=[font=\normalsize]
                        \draw [->, >=Stealth] (3.25,7.25) -- (3.25,19);
                        \draw [->, >=Stealth] (3.25,13.25) -- (14.75,13.25);
                        \draw [short] (3.25,17) -- (6.25,17);
                        \draw [short] (6.25,17) -- (6.25,10);
                        \draw [short] (6.25,10) -- (9,10);
                        \draw [short] (9,10) -- (9,17);
                        \draw [short] (9,17) -- (11.5,17);
                        \draw [short] (11.5,17) -- (11.5,10);
                        \draw [short] (11.5,10) -- (13.5,10);
                        \draw [short] (13.5,10) -- (13.5,13.25);
                        \draw [<->, >=Stealth] (7.5,13.25) -- (7.5,10);
                        \draw [<->, >=Stealth] (4.75,13.25) -- (4.75,17);
                        \draw [<->, >=Stealth] (10.25,13.25) -- (10.25,17);
                        \draw [<->, >=Stealth] (12.5,13.25) -- (12.5,10);
                        \node [font=\Large, rotate around={90:(0,0)}] at (1.25,14.25) {Roughness height
                        in $\mu m$};
                        \node [font=\normalsize] at (2.75,17) {1};
                        \node [font=\normalsize] at (2.5,13) {0};
                        \node [font=\normalsize] at (2.5,10) {-1};
                        \draw [dashed] (6.25,10) -- (3.25,10);
                        \node [font=\normalsize] at (4,15) {$y_1$};
                        \node [font=\normalsize] at (7,11.5) {$y_2$};
                        \node [font=\normalsize] at (9.5,15.5) {$y_3$};
                        \node [font=\normalsize] at (12,11.5) {$y_4$};
                        \draw [dashed] (9,10) -- (11.5,10);
                        \draw [dashed] (9,17) -- (6.25,17);
                        \draw [dashed] (6.25,10) -- (6.25,7.5);
                        \draw [dashed] (11.5,10) -- (11.5,7.5);
                        \draw [dashed] (13.5,13.25) -- (13.5,17);
                        \draw [dashed] (13.5,17) -- (11.25,17);
                        \draw [<->, >=Stealth] (3.25,9) -- (6,9);
                        \draw [<->, >=Stealth] (6.25,9) -- (8.75,9);
                        \draw [<->, >=Stealth] (9,9) -- (11.25,9);
                        \draw [<->, >=Stealth] (11.5,9) -- (13.5,9);
                        \draw [dashed] (9,10) -- (9,7.75);
                        \draw [dashed] (13.5,10) -- (13.5,7.25);
                        \node [font=\normalsize] at (4.5,8.5) {$\frac{L}{4}$};
                        \node [font=\normalsize] at (10.25,8.5) {$\frac{L}{4}$};
                        \node [font=\normalsize] at (12.5,8.5) {$\frac{L}{4}$};
                        \node [font=\normalsize] at (7.5,8.5) {$\frac{L}{4}$};
                    \end{circuitikz}
                } 
            \end{center}
            \hfill{\brak{2021-ME}}
            \begin{multicols}{4}
                \begin{enumerate}
                    \item 0
                        \columnbreak
                    \item 1
                        \columnbreak
                \item 2
                    \columnbreak
                \item 4
            \end{enumerate}
        \end{multicols}

    \item In which of the following pairs of cycles, both cycles have at least one isothermal
        process?
        \hfill{\brak{2021-ME}}
            \begin{enumerate}
        \begin{multicols}{2}
                \item  Diesel cycle and Otto cycle
                    \columnbreak
                \item Carnot cycle and Stirling cycle
        \end{multicols}
        \begin{multicols}{2}
                \item Brayton cycle and Rankine cycle
                    \columnbreak
                \item Bell-Coleman cycle and Vapour compression refrigeration cycle
        \end{multicols}
            \end{enumerate}

    \item Supeheated steam at 1500$kPa$, has a specific volume of $2.75 m^3/kmol$ and compressibility
        factor $\brak{Z}$ of 0.95 . The temperature of steam is (in $\quad^{\degree}C$)\\
        (round off to the nearest integer).
        \hfill{\brak{2021-ME}}
        \begin{multicols}{4}
            \begin{enumerate}
                \item 522
                    \columnbreak
                \item 471
                    \columnbreak
                \item 249
                    \columnbreak
                \item 198
            \end{enumerate}
        \end{multicols}

    \item A hot steel spherical ball is suddenly dipped into a low temperature oil bath. Which of 
        the following dimensionless parameters are required to determine instantaneous center temperature
        of the ball using a Heisler chart?
        \hfill{\brak{2021-ME}}
            \begin{enumerate}
        \begin{multicols}{2}
                \item Biot number and Fourier number
                    \columnbreak
                \item Reynolds Number and Prandtl number
        \end{multicols}
        \begin{multicols}{2}
                \item Biot number and Froude number
                    \columnbreak
                \item Nusselt number and Grashoff number
        \end{multicols}
            \end{enumerate}
    

\iffalse
\chapter{2019}
\author{EE24BTECH11019 - Dwarak A}
\section{me}
\fi

    %17
    \item The table presents the demand of a product. By simple three-months moving average method, the demand-forecast of the product for the month of September is
        \begin{table}[!ht]
            \centering
            \begin{tabular}{|l|c|}
\hline
Month    & Demand \\
\hline
January  & 450    \\
\hline
February & 440    \\
\hline
March    & 460    \\
\hline
April    & 510    \\
\hline
May      & 520    \\
\hline
June     & 495    \\
\hline
July     & 475    \\
\hline
August   & 560    \\
\hline

\end{tabular}
        \end{table}
        \begin{enumerate}
            \item $490$
            \item $510$
            \item $530$
            \item $536.67$
        \end{enumerate}

    %18
    \item Evaluation of $\int\limits_2^4x^3dx$ using a 2-equal-segment trapezoidal rule gives a value of \underline{\hspace{1cm}}
    
    %19
    \item A block of mass $10 kg$ rests on a horizontal floor. The acceleration due to gravity is $9.81 m/s^2$. The coefficient of static friction between the floor and the block is $0.2$. A horizontal force of $10 N$ is applied on the block as shown in the figure. The magnitude of force of friction (in $N$) on the block is \underline{\hspace{1cm}}
        \begin{figure}[!ht]
            \centering
            \resizebox{0.7\textwidth}{!}{
\begin{circuitikz}
\tikzstyle{every node}=[font=\LARGE]
\draw [ line width=1pt ] (5,13.75) rectangle (13.75,12.5);
\draw [line width=0.8pt, short] (3.75,12.5) -- (15,12.5);
\draw [line width=0.8pt, short] (3.75,12.5) -- (3.25,12);
\draw [line width=0.8pt, short] (5,12.5) -- (4.5,12);
\draw [line width=0.8pt, short] (6.25,12.5) -- (5.75,12);
\draw [line width=0.8pt, short] (7.5,12.5) -- (7,12);
\draw [line width=0.8pt, short] (8.75,12.5) -- (8.25,12);
\draw [line width=0.8pt, short] (10,12.5) -- (9.5,12);
\draw [line width=0.8pt, short] (11.25,12.5) -- (10.75,12);
\draw [line width=0.8pt, short] (12.5,12.5) -- (12,12);
\draw [line width=0.8pt, short] (13.75,12.5) -- (13.25,12);
\draw [line width=0.8pt, short] (15,12.5) -- (14.5,12);
\node [font=\LARGE] at (9.25,13.25) {$10 kg$};
\draw [line width=0.8pt, ->, >=Stealth] (13.75,13.25) -- (18.75,13.25)node[pos=0.5,above, fill=white]{10 N};
\end{circuitikz}
}
        \end{figure}

    %20
    \item  A cylindrical rod of diameter $10 mm$ and length $1.0 m$ is fixed at one end. The other end is twisted by an angle of $10\degree$ by applying a torque. If the maximum shear strain in the rod is $p\times10^{-3}$, then $p$ is equal to \underline{\hspace{1cm}} (round off to two decimal places).

    %21
    \item A solid cube of side $1 m$ is kept at a room temperature of $32 \degree C$. The coefficient of linear thermal expansion of the cube material is $1\times 10^{-5}/{}^{\degree} C$ and the bulk modulus is $200 GPa$. If the cube is constrained all around and heated uniformly to $42 \degree C$, then the magnitude of volumetric (mean) stress (in $MPa$) induced due to heating is \underline{\hspace{1cm}}

    %22
    \item During a high cycle fatigue test, a metallic specimen is subjected to cyclic loading with a mean stress of $+140 MPa$, and a minimum stress of $-70 MPa$. The $R$-ratio (minimum stress to maximum stress) for this cyclic loading is \underline{\hspace{1cm}} (round off to one decimal place).

    %23
    \item Water flows through a pipe with a velocity given by $\vec{V} = \brak{\frac{4}{t}+x+y}\hat{j}m/s$, where $\hat{j}$ is the unit vector in the $y$ direction, $t (> 0)$ is in seconds, and $x$ and $y$ are in meters. The magnitude of total acceleration at the point $(x, y) = (1, 1)$ a $t = 2 s$ is \underline{\hspace{1cm}} $m/s^2$

    %24
    \item Air of mass $1 kg$, initially at $300 K$ and $10\,bar$, is allowed to expand isothermally till it reaches a pressure of $1\,bar$. Assuming air as an ideal gas with gas constant of $0.287 kJ/kg K$, the change in entropy of air (in $kJ/kg.K$, round off to two decimal places) is \underline{\hspace{1cm}}

    %25
    \item Consider the stress-strain curve for an ideal elastic-plastic strain hardening metal as shown in the figure. The metal was loaded in uniaxial tension starting from $\vec{O}$. Upon loading, the stress-strain curve passes through initial yield point at $\vec{P}$, and then strain hardens to point $\vec{Q}$, where the loading was stopped. From point $\vec{Q}$, the specimen was unloaded to point $\vec{R}$, where the stress is zero. If the same specimen is reloaded in tension from point $\vec{R}$, the value of stress at which the material yields again is \underline{\hspace{1cm}} $MPa$
        \begin{figure}[!ht]
            \centering
            \resizebox{0.4\textwidth}{!}{%
\begin{circuitikz}
\tikzstyle{every node}=[font=\Large]
\draw [line width=0.8pt, ->, >=Stealth] (8.75,7.5) -- (8.75,15);
\draw [line width=0.8pt, ->, >=Stealth] (8.75,7.5) -- (16.25,7.5);
\draw [line width=1pt, short] (8.75,7.5) -- (10.75,12.5);
\draw [line width=1pt, short] (12.5,7.5) -- (15,13.75);
\draw [line width=1pt, short] (10.75,12.5) .. controls (11.75,13.5) and (12,13.6) .. (15,13.75);
\draw [line width=0.8pt, short] (8.75,13.75) -- (15,13.75);
\draw [line width=0.8pt, short] (8.75,12.5) -- (10.75,12.5);
\node [font=\Large] at (8,13.75) {$210$};
\node [font=\Large] at (8,12.5) {$180$};
\node [font=\Large] at (10.5,13) {$P$};
\node [font=\Large] at (15.5,13.75) {$Q$};
\node [font=\Large] at (8.5,7.25) {$O$};
\node [font=\Large] at (12.25,7) {$R$};
\node [font=\Large] at (14.75,7) {Strain};
\node [font=\Large, rotate around={90:(0,0)}] at (8.25,10) {Stress $(MPa)$};
\end{circuitikz}
}
        \end{figure}


    %26
    \item The set of equations
        $$x+y+z=1$$
        $$ax-ay+3z=5$$
        $$5x-3y+az=6$$
        has infinite solutions if $a=$
        \begin{enumerate}
            \item $-3$
            \item $3$
            \item $4$
            \item $-4$
        \end{enumerate}

    %27
    \item A harmonic function is analytic if it satisfies the Laplace equation.

        If $u(x,y)=2x^2-2y^2+4xy$ is a harmonic function, then its conjugate harmonic function $v(x,y)$ is
        \begin{enumerate}
            \item $4xy-2x^2+2y^2+$ constant
            \item $4y^2-4xy+$ constant
            \item $2x^2-2y^2+xy+$ constant
            \item $-4xy+2y^2-2x^2+$ constant
        \end{enumerate}

    %28
    \item The variable $x$ takes a value between $0$ and $10$ with uniform probability distribution. The variable $y$ takes a value between $0$ and $20$ with uniform probability distribution. The probability of the sum of variables $(x + y)$ being greater than $20$ is
        \begin{enumerate}
            \item $0$
            \item $0.25$
            \item $0.33$
            \item $0.50$
        \end{enumerate}

    %29
    \item A car having weight $W$ is moving in the direction as shown in the figure. The center of gravity ($CG$) of the car is located at height $h$ from the ground, midway between the front and rear wheels. The distance between the front and rear wheels is $l$. The acceleration of the car is $a$, and acceleration due to gravity is $g$. The reactions on the front wheels $(R_{f})$ and rear wheels $(R_{r})$ are given by
        \begin{figure}[!ht]
            \centering
            \resizebox{0.8\textwidth}{!}{%
\begin{circuitikz}
\tikzstyle{every node}=[font=\LARGE]
\draw [ line width=1.2pt ] (11.25,10) circle (1.25cm);
\draw [ line width=1.2pt ] (21.25,10) circle (1.25cm);
\draw [line width=1.2pt, short] (13,10) -- (19.5,10);
\draw [line width=1.2pt, short] (9.75,10) -- (7.5,10);
\draw [line width=1.2pt, short] (22.75,10) -- (25,10);
\draw [line width=1.2pt, short] (12.5,15) -- (20,15);
\draw [line width=1.2pt, short] (20,15) .. controls (25.25,13.25) and (25,12.25) .. (25,10);
\draw [line width=1.2pt, short] (12.5,15) .. controls (8,14.75) and (8,14.25) .. (7.5,10);
\draw [ line width=1pt ] (15.75,13.25) circle (0.25cm);
\draw [line width=1pt, short] (15.5,13.5) -- (16,13);
\draw [line width=1pt, short] (16,13.5) -- (15.5,13);
\node at (15,13.75) {CG};
\draw [line width=1pt, ->, >=Stealth] (15.75,13.25) -- (15.75,11.5)node[pos=0.5,right, fill=white]{$W$};
\draw [line width=1pt, short] (6.25,8.75) -- (27.5,8.75);
\draw [line width=1pt, ->, >=Stealth] (11.25,6.25) -- (11.25,8.75)node[pos=0.5,left, fill=white]{$R_{r}$};
\draw [line width=1pt, ->, >=Stealth] (21.25,6.25) -- (21.25,8.75)node[pos=0.5,right, fill=white]{$R_{f}$};
\draw [line width=1pt, <->, >=Stealth] (11.25,6.75) -- (21.25,6.75)node[pos=0.5,above, fill=white]{l};
\draw [line width=1pt, short] (7.5,13) -- (6.25,13);
\draw [line width=1pt, <->, >=Stealth] (6.75,13) -- (6.75,8.75)node[pos=0.5,left, fill=white]{h};
\draw [line width=1pt, ->, >=Stealth] (25,15) -- (28.75,15)node[pos=0.5,above, fill=white]{a};
\node at (27,14.5) {Direction of Motion};
\draw [line width=1pt, short] (7.5,8.75) -- (7,8.25);
\draw [line width=1pt, short] (8.75,8.75) -- (8.25,8.25);
\draw [line width=1pt, short] (10,8.75) -- (9.5,8.25);
\draw [line width=1pt, short] (11.25,8.75) -- (10.75,8.25);
\draw [line width=1pt, short] (12.5,8.75) -- (12,8.25);
\draw [line width=1pt, short] (13.75,8.75) -- (13.25,8.25);
\draw [line width=1pt, short] (15,8.75) -- (14.5,8.25);
\draw [line width=1pt, short] (16.25,8.75) -- (15.75,8.25);
\draw [line width=1pt, short] (17.5,8.75) -- (17,8.25);
\draw [line width=1pt, short] (18.75,8.75) -- (18.25,8.25);
\draw [line width=1pt, short] (20,8.75) -- (19.5,8.25);
\draw [line width=1pt, short] (21.25,8.75) -- (20.75,8.25);
\draw [line width=1pt, short] (22.5,8.75) -- (22,8.25);
\draw [line width=1pt, short] (23.75,8.75) -- (23.25,8.25);
\draw [line width=1pt, short] (25,8.75) -- (24.5,8.25);
\draw [line width=1pt, short] (26.25,8.75) -- (25.75,8.25);
\draw [line width=1pt, short] (27.5,8.75) -- (27,8.25);
\draw [line width=1pt, short] (9.75,10) .. controls (9.75,12.25) and (12.75,12) .. (13,10);
\draw [line width=1pt, short] (19.5,10) .. controls (19.5,12.25) and (22.75,12) .. (22.75,10);
\end{circuitikz}
}
        \end{figure}
        \begin{enumerate}
            \item $R_{f}=R_{r}=\frac{W}{2}-\frac{W}{g}\brak{\frac{h}{l}}a$
            \item $R_{f}=\frac{W}{2}+\frac{W}{g}\brak{\frac{h}{l}}a; R_{r}=\frac{W}{2}-\frac{W}{g}\brak{\frac{h}{l}}a$
            \item $R_{f}=\frac{W}{2}-\frac{W}{g}\brak{\frac{h}{l}}a; R_{r}=\frac{W}{2}+\frac{W}{g}\brak{\frac{h}{l}}a$
            \item $R_{f}=R_{r}=\frac{W}{2}+\frac{W}{g}\brak{\frac{h}{l}}a$
        \end{enumerate}


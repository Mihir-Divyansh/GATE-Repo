\iffalse
\chapter{2007}
\author{EE24BTECH11021 - Eshan Ray}
\section{xe}
\fi
    %1
    \item Let $M=\myvec{1&1&1 \\ 0&1&1 \\ 0&0&1}$. Then the maximum number of linearly independent eigenvectors of $M$ is
        \begin{enumerate}
            \item $0$
            \item $1$
            \item $2$
            \item $3$
        \end{enumerate}
    
    %2
    \item Let $L=\lim\limits_{x\to\frac{\pi}{2}}\frac{\sin^22x}{\brak{x-\frac{\pi}{2}}^2}$. Then $L$ is equal to
        \begin{enumerate}
            \item $-4$
            \item $0$
            \item $2$
            \item $4$
        \end{enumerate}
    
    %3
    \item Let $f(z)=\frac{1}{1-z^2}$. The coefficient $\frac{1}{z-1}$ in the Laurent expansion of $f(z)$ about $z=1$ is
        \begin{enumerate}
            \item $-1$
            \item $-\frac{1}{2}$
            \item $\frac{1}{2}$
            \item $1$
        \end{enumerate}

    %4
    \item Let $u(x,t)$ be the solution of the initial value problem

        $\frac{\partial^2u}{\partial t^2}=9\frac{\partial^2u}{\partial x^2},t>0,-\infty<x<\infty$,

        $u(x,0)=x+5$,

        $\frac{\partial u}{\partial t}(x,0)=0$.

        Then $u(2,2)$ is
        \begin{enumerate}
            \item $7$
            \item $13$
            \item $14$
            \item $26$
        \end{enumerate}

    %5
    \item Two students take a test consisting of five TRUE/FALSE questions. To pass the test the students have to answer at least three questions correctly. Both of them know the correct answers to two questions and guess the answers to the remaining three. The probability that only one student passes the test is equal to
        \begin{enumerate}
            \item $\frac{6}{32}$
            \item $\frac{7}{32}$
            \item $\frac{1}{4}$
            \item $\frac{3}{4}$
        \end{enumerate}

    %6
    \item The equation $g(x)=x$ is solved by Newton-Raphson iteration method, starting with an initial approximation $x_0$ near the simple root $\alpha$. If $x_{n+1}$ is the approximation to $\alpha$ at the $(n+1)^{th}$ iteration, then
        \begin{enumerate}
            \item $x_{n+1}=\frac{x_ng^\prime(x_n)-g(x_n)}{1-g^\prime(x_n)}$
            \item $x_{n+1}=\frac{x_ng^\prime(x_n)-g(x_n)}{g^\prime(x_n)-1}$
            \item $x_{n+1}=g(x_n)$
            \item $x_{n+1}=\frac{x_ng^\prime(x_n)-g(x_n)+2x_n}{g^\prime(x_n)+1}$
        \end{enumerate}

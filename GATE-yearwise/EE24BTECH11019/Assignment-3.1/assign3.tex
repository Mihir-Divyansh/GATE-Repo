\iffalse
\chapter{2014}
\author{EE24BTECH11019 - Dwarak A}
\section{xe}
\fi

    %14
    \item Ten chocolates are distributed randomly among three children standing in a row. The probability that the first child receives exactly three chocolates is 
        \begin{enumerate}
            \item $\frac{5\times2^{11}}{3^9}$
            \item $\frac{5\times2^{10}}{3^9}$
            \item $\frac{1}{3^9}$
            \item $\frac{1}{3}$
        \end{enumerate}
    
    %15
    \item Let the function $f:[5,0]\to\mathbb{R}$ be defined by

        $$f(x)=
        \begin{cases}
            2x + 5, & 0 \leq x < 1 \\ 
            2x^2 + 5, & 1 \leq x < 2 \\ 
            \frac{2}{3}x^3 + \frac{23}{3}, & 2 \leq x \leq 5 
        \end{cases}$$
        The number of points where $f$ is not differentiable in $(0,5)$, is \underline{\hspace{1.5cm}}.
    
    %16
    \item An integrating factor of the differential equation $\brak{3x^2y^3e^y+y^3+y^2}dx+\brak{x^3y^3e^y-xy}dy=0$ is
        \begin{enumerate}
            \item $\frac{1}{y}$
            \item $\frac{1}{y^2}$
            \item $\frac{1}{y^3}$
            \item $\ln{y}$
        \end{enumerate}

    %17
    \item If a cubic polynomial passes through the points $(0, 1)$, $(1, 0)$, $(2, 1)$ and $(3, 10)$, then it also passes through the point 
        \begin{enumerate}
            \item $(-2,-11)$
            \item $(-1,-2)$
            \item $(-1,-4)$
            \item $(-2,-23)$
        \end{enumerate}


    %18
    \item Let the function $f:[0,\infty)\to\mathbb{R}$ be such that $f^\prime(x)=\frac{8}{x^2+3x+4}$ for $x>0$ and $f(0)=1$. Then $f(1)$ lies in the interval
        \begin{enumerate}
            \item $[0,1]$
            \item $[2,3]$
            \item $[4,5]$
            \item $[6,7]$
        \end{enumerate}

    %19
    \item The perimeter of a rectangle having the largest area that can be inscribed in the ellipse $\frac{x^2}{8}+\frac{y^2}{32}=1$, is \underline{\hspace{1.5cm}}.

    %20
    \item If the work done in moving a particle once around a circle $x^2+y^2=4$ under the force field $\vec{F}(x,y)=(2x-ay)\hat{i}+(2y+ax)\hat{j}$ is $16\pi$, then $\abs{a}$ is equal to \underline{\hspace{1.5cm}}.

    %21
    \item Let $r$ and $s$ be real numbers. If $A = \myvec{1&2&0\\2&0&3\\r&s&0}$ and $b=\myvec{1\\1\\s-1}$, then the system of linear equations $AX=b$ has
        \begin{enumerate}
            \item no solutions for $s\neq2r$.
            \item infinitely many solutions for $s=2r\neq2$.
            \item a unique solution for $s=2r=2$.
            \item infinitely many solutions for $s=2r=2$.
        \end{enumerate}


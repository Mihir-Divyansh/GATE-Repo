\iffalse
\chapter{2018}
\author{EE24BTECH11019 - Dwarak A}
\section{me}
\fi

    %43
    \item A solid block of $2.0 kg$ mass slides steadily at a velocity $V$ along a vertical wall as shown in the figure below. A thin oil film of thickness $h=0.15mm$ provides lubrication between the block and the wall. The surface area of the face of the block in contact with the oil film is $0.04 m^2$ . The velocity distribution within the oil film gap is linear as shown in the figure. Take dynamic viscosity of oil as $7\times10^{-3} Pa-s$ and acceleration due to gravity as $10 m/s^2$ . Neglect weight of the oil. The terminal velocity $V$ (in $m/s$) of the block is \underline{\hspace{1cm}} (correct to one decimal place).
        \begin{figure}[!ht]
            \centering
            \resizebox{0.35\textwidth}{!}{
\begin{circuitikz}
\tikzstyle{every node}=[font=\LARGE]
\draw  (7.5,15) rectangle (8.5,8.75);
\draw [ line width=0.6pt ] (8.5,13.75) rectangle (10.75,10.25);
\draw [line width=0.6pt, short] (7.5,12.5) -- (8.5,12.5);
\draw [line width=0.6pt, short] (8.5,12.5) -- (8.5,11);
\draw [line width=0.6pt, short] (8.5,11) -- (7.5,12.5);
\draw [line width=0.6pt, ->, >=Stealth] (8,12.5) -- (8,11.75);
\draw [line width=0.6pt, ->, >=Stealth] (8.25,12.5) -- (8.25,11.5);
\draw [line width=0.6pt, short] (7.5,15) -- (7,14.5);
\draw [line width=0.6pt, short] (7.5,14.5) -- (7,14);
\draw [line width=0.6pt, short] (7.5,14) -- (7,13.5);
\draw [line width=0.6pt, short] (7.5,13.5) -- (7,13);
\draw [line width=0.6pt, short] (7.5,13) -- (7,12.5);
\draw [line width=0.6pt, short] (7.5,12.5) -- (7,12);
\draw [line width=0.6pt, short] (7.5,12) -- (7,11.5);
\draw [line width=0.6pt, short] (7.5,11.5) -- (7,11);
\draw [line width=0.6pt, short] (7.5,11) -- (7,10.5);
\draw [line width=0.6pt, short] (7.5,10.5) -- (7,10);
\draw [line width=0.6pt, short] (7.5,9.5) -- (7,9);
\draw [line width=0.6pt, short] (7.5,9) -- (7,8.5);
\draw [line width=0.6pt, short] (7.5,16.25) -- (7.5,15.25);
\draw [line width=0.6pt, short] (8.5,16.25) -- (8.5,15.25);
\draw [line width=0.6pt, ->, >=Stealth] (6.5,15.75) -- (7.5,15.75);
\draw [line width=0.6pt, ->, >=Stealth] (9.5,15.75) -- (8.5,15.75);
\draw [line width=0.6pt, ->, >=Stealth] (5.75,7.5) -- (7.5,11.25);
\node at (5.75,7) {Impenetrable wall};
\node at (10,9.5) {$A=0.04m^2$};
\node [font = \large] at (9.5,13) {$m = 2.0 kg$};
\draw [line width=0.6pt, ->, >=Stealth] (9.5,12.25) -- (9.5,11)node[pos=0.5,left]{V};
\node at (11,15.75) {$h = 0.15 mm$};
\end{circuitikz}
}
        \end{figure}

    %44
    \item A tank of volume $0.05 m^3$ contains a mixture of saturated water and saturated steam at $200\degree C$.The mass of the liquid present is $8 kg$. The entropy (in $kJ/kg K$) of the mixture is \underline{\hspace{1cm}} (correct to two decimal places).

        Property data for saturated steam and water are:

        At $200\degree C$,$p_{sat}= 1.5538 MPa$

        $v_{f} = 0.001157 m^3/kg$,$v_{g} = 0.12736 m^3/kg$

        $s_{fg} = 4.1014 kJ/kg K$, $s_{f} = 2.3309 kJ/kg K$
    
    %45
    \item Steam flows through a nozzle at a mass flow rate of $\dot{m}=0.1kg/s$ with a heat loss of 5 kW. The enthalpies at inlet and exit are 2500 kJ/kg and 2350 kJ/kg, respectively. Assuming
        negligible velocity at inlet $(C_1\approx0)$, the velocity $C_2$ of steam (in $m/s$) at the nozzle exit is \underline{\hspace{1cm}} (correct to two decimal places).
        \begin{figure}[!ht]
            \centering
            \resizebox{0.5\textwidth}{!}{
\begin{circuitikz}
\tikzstyle{every node}=[font=\LARGE]
\draw [line width=1.0pt, short] (7.5,8.75) .. controls (12,12.25) and (12,11.5) .. (16.25,12);
\draw [line width=1.0pt, short] (16.25,13.75) .. controls (12,14) and (12,13.25) .. (7.5,16.75);
\draw [line width=0.8pt, ->, >=Stealth] (9.25,13) -- (11.5,13);
\node at (13.25,13) {$\dot{m} = 0.1kg/s$};
\node at (17.75,13.25) {$h_2 = 2350 kJ/kg$};
\node at (16.25,12.5) {$C_2$};
\node at (6,13.25) {$h_1 = 2500 kJ/kg$};
\node at (5,12.5) {$C_1\approx0$};
\draw [line width=1pt, ->, >=Stealth] (10.25,15) -- (12,16);
\node at (12.5,16.5) {$\dot{Q} = 5 kW$};
\end{circuitikz}
}
        \end{figure}

    %46
    \item An engine working on air standard Otto cycle is supplied with air at $0.1 MPa$ and $35\degree C$. The compression ratio is $8$. The heat supplied is $500 kJ/kg$. Property data for air: $c_p = 1.005 kJ/kg K,\,c_v = 0.718 kJ/kg K,\,R = 0.287 kJ/kg K$. The maximum temperature (in $K$) of the cycle is \underline{\hspace{1cm}} (correct to one decimal place).

    %47
    \item A plane slab of thickness $L$ and thermal conductivity $k$ is heated with a fluid on one side ($P$),and the other side ($Q$) is maintained at a constant temperature, $T_{Q}$ of $25\degree C$, as shown in the figure. The fluid is at $45\degree C$ and the surface heat transfer coefficient, $h$, is $10 W/m^2K$. The steady state temperature, $T_{P}$, (in $\degree C$) of the side which is exposed to the fluid is \underline{\hspace{1cm}} (correct to two decimal places).
        \begin{figure}[!ht]
            \centering
            \resizebox{0.4\textwidth}{!}{
\begin{circuitikz}
\tikzstyle{every node}=[font=\Large]
\draw [line width=1pt, short] (7.5,16.25) -- (7.5,10);
\draw [line width=1pt, short] (11.25,16.25) -- (11.25,10);
\draw [ line width=0.8pt ] (3,14.5) rectangle (6.5,12);
\draw[domain=7.5:11.25,samples=100,smooth, line width=0.8pt] plot (\x,{0.2*sin(7.61*\x r -7.5 r ) +16.25});
\draw[domain=7.5:11.25,samples=100,smooth, line width=0.8pt] plot (\x,{0.2*sin(7.61*\x r -10.4 r ) +10});
\draw [line width=0.8pt, <->, >=Stealth] (7.5,11.25) -- (11.25,11.25)node[pos=0.5, fill=white]{L = 20 cm};
\node [font=\Large] at (9.5,13.25) {$k = 2.5 W/mK$};
\node [font=\Large] at (12.5,14.5) {$T_Q = 25^{\degree} C$};
\node [font=\Large] at (7,15.75) {$T_P$};
\node [font=\Large] at (4.75,13.75) {$h=10 W/m^2K$};
\node [font=\Large] at (4.75,12.75) {$T_\infty = 45^{\degree} C$};
\end{circuitikz}
}
        \end{figure}

    %48
    \item The true stress $(\sigma)$ - true strain $(\varepsilon)$ diagram of a strain hardening material is shown in figure. First, there is loading up to point $\vec{A}$, i.e., up to stress of $500 MPa$ and strain of $0.5$. Then from point $\vec{A}$, there is unloading up to point $\vec{B}$, i.e., to stress of $100 MPa$. Given that the Young's modulus $E = 200 GPa$, the natural strain at point $\vec{B} (\varepsilon_{B})$ is \underline{\hspace{1cm}} (correct to two decimal places).
        \begin{figure}[!ht]
            \centering
            \resizebox{0.4\textwidth}{!}{%
\begin{circuitikz}
\tikzstyle{every node}=[font=\LARGE]
\draw [line width=1pt, ->, >=Stealth] (5,7.5) -- (5,16.25);
\draw [line width=1pt, ->, >=Stealth] (5,7.5) -- (15,7.5);
\draw [line width=1pt, dashed] (5,15) -- (13,15);
\draw [line width=1pt, dashed] (13,15) -- (13,7.5);
\draw [line width=1pt, dashed] (5,10) -- (12.75,10);
\draw [line width=1pt, dashed] (11.25,10) -- (11.25,7.75);
\draw [line width=1pt, short] (13,15) -- (11.25,10);
\draw [line width=1pt, short] (5,7.5) .. controls (7.75,14.25) and (7.5,13.75) .. (14.5,15.25);
\draw [line width=1pt, short] (8.25,13.75) -- (8.75,13.75);
\draw [line width=1pt, short] (8.75,13.75) -- (8.5,13.25);
\draw [line width=1pt, short] (11.5,11.5) -- (11.5,11);
\draw [line width=1pt, short] (11.5,11) -- (12,11.25);
\node [font=\LARGE] at (11,10.5) {B};
\node [font=\LARGE] at (13,15.5) {A};
\node [font=\LARGE] at (4.25,10) {$100$};
\node [font=\LARGE] at (4.25,15) {$500$};
\node [font=\LARGE] at (5,17.25) {$\sigma$};
\node [font=\LARGE] at (5,16.75) {$(MPa)$};
\node [font=\LARGE] at (13,7) {$0.5$};
\node [font=\LARGE] at (15,7) {$\varepsilon$};
\draw [line width=1pt, short] (5,7) -- (5,6.25);
\draw [line width=1pt, short] (11.25,7) -- (11.25,6.25);
\draw [line width=1pt, <->, >=Stealth] (5,6.75) -- (11.25,6.75)node[pos=0.5, fill=white]{$\varepsilon_{B}$};
\end{circuitikz}
}
        \end{figure}

    %49
    \item An orthogonal cutting operation is being carried out in which uncut thickness is $0.010 mm$, cutting speed is $130 m/min$, rake angle is $15\degree$ and width of cut is $6 mm$. It is observed that the chip thickness is $0.015 mm$, the cutting force is $60 N$ and the thrust force is $25 N$. The ratio of friction energy to total energy is \underline{\hspace{1cm}} (correct to two decimal places). 

    %50
    \item A bar is compressed to half of its original length. The magnitude of true strain produced in the deformed bar is \underline{\hspace{1cm}} (correct to two decimal places).

    %51
    \item The minimum value of $3x+5y$ such that:
        $$3x+5y\leq15$$
        $$4x+9y\leq8$$
        $$13x+2y\leq2$$
        $$x\geq0,y\geq0$$
        is \underline{\hspace{1cm}}.

    %52
    \item Processing times (including setup times) and due dates for six jobs waiting to be processed at a work centre are given in the table. The average tardiness (in days) using shortest processing time rule is \underline{\hspace{1cm}} (correct to two decimal places).
        \begin{table}[!ht]
            \centering
            \begin{tabular}{|l|l|l|}
    \hline
    Job & Processing time (days) & Due date (days) \\
    \hline
    A   & 3                      & 8               \\
    \hline
    B   & 7                      & 16              \\
    \hline
    C   & 4                      & 4               \\
    \hline
    D   & 9                      & 18              \\
    \hline
    E   & 5                      & 17              \\
    \hline
    F   & 13                     & 19              \\
    \hline
\end{tabular}
        \end{table}

    %53
    \item The schematic of an external drum rotating clockwise engaging with a short shoe is shown in the figure. The shoe is mounted at point $\vec{Y}$ on a rigid lever $\vec{XYZ}$ hinged at point $\vec{X}$. A force $F = 100 N$ is applied at the free end of the lever as shown. Given that the coefficient offriction between the shoe and the drum is $0.3$, the braking torque (in $Nm$) applied on thedrum is \underline{\hspace{1cm}} (correct to two decimal places).
        \begin{figure}[!ht]
            \centering
            \resizebox{0.4\textwidth}{!}{%
\begin{circuitikz}
\tikzstyle{every node}=[font=\Huge]
\draw [line width=1pt, short] (13.75,18.75) -- (13.75,15);
\draw [line width=1pt, short] (13.75,18.75) -- (14.75,17.75);
\draw [line width=1pt, short] (13.75,17.5) -- (14.75,16.5);
\draw [line width=1pt, short] (13.75,16.25) -- (14.75,15.25);
\draw [line width=1pt, short] (13.75,15) -- (14.75,14);
\draw [line width=1pt, short] (13.75,18.25) .. controls (11.25,17.75) and (11.25,16) .. (13.75,15.5);
\draw [ line width=1pt ] (13,17) ellipse (0.25cm and 0.25cm);
\draw [line width=1pt, short] (13,17) .. controls (0.5,17.5) and (15.5,4.75) .. (0,6);
\draw [ line width=1pt ] (3,3.5) circle (2cm);
\draw [line width=1pt, short] (1.5,6) -- (2,5.25);
\draw [line width=1pt, short] (4.75,6.25) -- (4,5.25);
\draw [line width=1pt, short] (13,19) -- (13,23.75);
\draw [line width=1pt, short] (0,23.75) -- (0,10);
\draw [line width=1.4pt, ->, >=Stealth] (0,9.5) -- (0,6.25)node[pos=0.5,right, fill=white]{F};
\draw [line width=1.4pt, short] (3,20.75) -- (3,6.5);
\draw [line width=1.4pt, <->, >=Stealth] (0.25,22.5) -- (12.75,22.5)node[pos=0.5, fill=white]{300};
\draw [line width=1.4pt, <->, >=Stealth] (3.25,20) -- (12.75,20)node[pos=0.5, fill=white]{200};
\node [font=\Huge] at (12.5,14.75) {$X$};
\draw [line width=1.4pt, short] (5,5.25) -- (18.75,5.25);
\draw [line width=1.4pt, short] (15,17) -- (18.75,17);
\draw [line width=1.4pt, short] (2.25,3) -- (3.75,3);
\draw [line width=1.4pt, short] (2.25,3) -- (2,2.75);
\draw [line width=1.4pt, short] (2.75,3) -- (2.5,2.75);
\draw [line width=1.4pt, short] (3.25,3) -- (3,2.75);
\draw [line width=1.4pt, short] (3.75,3) -- (3.5,2.75);
\draw [line width=1.4pt, short] (2.25,3) .. controls (2.5,4.5) and (3.5,4.5) .. (3.75,3);
\draw [ line width=1.4pt ] (3,3.5) circle (0.25cm);
\draw [line width=1.4pt, ->, >=Stealth] (3,3.5) -- (4.75,4.25)node[pos=0.5,above]{100};
\draw [line width=1.4pt, ->, >=Stealth] (2.75,2.25) .. controls (1,2.25) and (1,4.5) .. (2.75,4.75) ;
\node [font=\Huge] at (0,5.5) {Z};
\node [font=\Huge] at (3.75,7) {Y};
\draw [line width=1.4pt, <->, >=Stealth] (17.5,16.75) -- (17.5,5.5)node[pos=0.5, fill=white]{300};
\end{circuitikz}
}
        \end{figure}

    %54
    \item Block $P$ of mass $2 kg$ slides down the surface and has a speed $20 m/s$ at the lowest point, $\vec{Q}$, where the local radius of curvature is $2 m$ as shown in the figure. Assuming $g = 10 m/s^2$  , the normal force (in $N$) at $\vec{Q}$ is \underline{\hspace{1cm}} (correct to two decimal places).
        \begin{figure}[!ht]
            \centering
            \resizebox{0.3\textwidth}{!}{%
\begin{circuitikz}
\tikzstyle{every node}=[font=\Huge]
\draw [line width=0.8pt, short] (1.25,16.25) -- (8.75,10);
\draw [line width=0.8pt, short] (1.25,13.75) -- (8.75,7.5);
\draw [line width=0.8pt, short] (8.75,10) .. controls (10.5,8.5) and (12.25,8.25) .. (13.75,10);
\draw [line width=0.8pt, short] (8.75,7.5) .. controls (10.5,6) and (12.25,5.75) .. (13.75,7.5);
\draw [line width=0.8pt, short] (3,15.25) -- (5,13.5);
\draw [line width=0.8pt, short] (5,13.5) -- (6.25,14.75);
\draw [line width=0.8pt, short] (6.25,14.75) -- (4.25,16.5);
\draw [line width=0.8pt, short] (4.25,16.5) -- (3,15.25);
\draw [line width=0.8pt, dashed] (11.25,16.25) -- (11.25,6.25);
\node at (2.75,16.5) {P};
\node at (11.25,5.5) {Q};
\end{circuitikz}
}
        \end{figure}

    %55
    \item An electrochemical machining (ECM) is to be used to cut a through hole into a $12 mm$ thick aluminum plate. The hole has a rectangular cross-section, $10 mm \times 30 mm$. The ECM operation will be accomplished in $2$ minutes, with efficiency of $90\%$. Assuming specific removal rate for aluminum as $3.44 \times 10^{-2} mm^3 /(A s)$, the current (in $A$) required is \underline{\hspace{1cm}} (correct to two decimal places).

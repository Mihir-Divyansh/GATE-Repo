\iffalse
\chapter{2014}
\author{EE24BTECH11019 - Dwarak A}
\section{xe}
\fi

    %22
    \item A dam with a curved shape is shown in the figure. The cross sectional area of the dam (shaded portion) is $100m^2$ and its centroid is at $\bar{x}=10m$. The vertical component of the hydrostatic force, $F_z$, is acting at a distance $x_p$. The value of $x_p$ is \underline{\hspace{1.5cm}}$m$.
        \begin{figure}[!ht]
            \centering
            \resizebox{0.5\textwidth}{!}{%
\begin{circuitikz}
\tikzstyle{every node}=[font=\Large]

    % Coordinate axes
    \draw [line width=0.8pt, ->, >=Stealth] (8.25,8) -- (8.25,16.5);
    \draw [line width=0.8pt, ->, >=Stealth] (8.25,8) -- (16.75,8);
    
    % Filled area between axes and curve
    \fill[gray!30] (8.25,8) -- (8.25,14.5) .. controls (10.5,9.25) and (10.5,9.25) .. (12.5,8) -- cycle;

    % Ground and lines
    \draw [line width=0.8pt, short] (8.25,14.5) -- (14.75,14.5);
    \draw [line width=0.6pt, short] (13.75,14.5) -- (13.5,15);
    \draw [line width=0.6pt, short] (13.5,15) -- (14,15);
    \draw [line width=0.6pt, short] (14,15) -- (13.75,14.5);
    \draw [line width=0.6pt, short] (13.5,14.25) -- (14,14.25);
    \draw [short] (13.5,14) -- (14,14);

    % Dimensions and markers
    \draw [<->, >=Stealth] (7.75,14.5) -- (7.75,8) node[pos=0.5,left, fill=white]{40m};
    \draw [short] (8.25,14.5) .. controls (10.5,9.25) and (10.5,9.25) .. (12.5,8);

    % Points and labels
    \node [font=\Large] at (8,7.75) {$O$};
    \node [font=\Large] at (10.5,9.5) {$G$};
    \node at (10.25,9.5) [circ] {};

    % Horizontal distances
    \draw [short] (8.25,7.75) -- (8.25,6.25);
    \draw [short] (10.25,9.5) -- (10.25,6.25);
    \draw [<->, >=Stealth] (8.25,6.75) -- (10.25,6.75) node[pos=0.5,below]{$\overline{X}$};

    % Vertical lines
    \draw [line width=0.6pt, short] (12.5,8) -- (12.5,4.5);
    \draw [line width=0.6pt, short] (8.25,5.5) -- (8.25,4.5);

    % Dimension labels
    \draw [line width=0.6pt, <->, >=Stealth] (8.25,5) -- (12.5,5) node[pos=0.5,below, fill=white]{15m};

    % Force vector Fz
    \draw [line width=0.8pt, ->, >=Stealth] (9.75,13.5) -- (9.75,11) node[pos=0.5,right, fill=white]{$F_z$};
    \draw [line width=0.6pt, <->, >=Stealth] (8.25,12.25) -- (9.75,12.25) node[pos=0.5,below]{$X_p$};

    % Axis labels
    \node [font=\Large] at (17.5,8) {X};
    \node [font=\Large] at (8.25,17) {Z};

\end{circuitikz}
}
        \end{figure}

    %23
    \item For an unsteady incompressible fluid flow, the velocity field is $\vec{V}=\brak{3x^2+3}t\,\hat{i}-6xyt\,\hat{j}$, where $x,y$ are in meters and $t$ is in seconds. Acceleration in $m/s^2$ at the point $x=10m$ and $y=0$, as measured by a stationary observer is
        \begin{enumerate}
            \item $303$
            \item $162$
            \item $43$
            \item $13$
        \end{enumerate}

    %24
    \item For an incompressible flow, the existence of components of acceleration for different types of flow is described in the table below.

        \begin{tabular}{l l}
            Type of Flow & Components of Acceleration \\
            P: Steady and uniform & 1: Local exists, convective does not exist \\
            Q: Steady and non-uniform & 2: Both exist \\
            R: Unsteady and uniform & 3: Both do not exist \\
            S: Unsteady and non-uniform & 4: Local does not exist, convective exists \\
        \end{tabular}

        Which one of the following options connecting the left column with the right column is correct?
        \begin{enumerate}
            \item P-$1$; Q-$4$; R-$3$; S-$2$
            \item P-$4$; Q-$1$; R-$2$; S-$3$
            \item P-$3$; Q-$2$; R-$1$; S-$4$
            \item P-$3$; Q-$4$; R-$1$; S-$2$
        \end{enumerate}

    %25
    \item Velocity in a two-dimensional flow field is specified as $u=x^2y;v=-y^2x$. The magnitude of the rate of angular deformation at a location $(x=2m \text{ and } y=1m)$ is \underline{\hspace{1cm}}$s^{-1}$.

    %26
    \item For a plane irrotational flow, equi-potential lines and streamlines are
        \begin{enumerate}
            \item parallel to each other.
            \item at an angle of $90\degree$ to each other.
            \item at an angle of $45\degree$ to each other.
            \item at an angle of $60\degree$ to each other.
        \end{enumerate}


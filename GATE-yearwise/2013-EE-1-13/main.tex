\iffalse
\title{2013-EE-1-13}
\author{EE24BTECH11010 - BALAJI B}
\section{ee}
\chapter{2013}
\fi
    \item In the circuit shown below what is the output voltage $\brak{V_{\text{out}}}$ in Volts if a silicon transistor $Q$ and an ideal op-amp are used? \hfill(2013-EE)
    \begin{figure}[H]
\centering
\resizebox{7cm}{!}{%
\begin{circuitikz}
\tikzstyle{every node}=[font=\normalsize]
\draw (1.75,11.75) to (1.75,11) node[ground]{};
\draw (1.75,13.5) to[american voltage source] (1.75,11.5);
\draw (1.75,13.5) to[short] (3,13.5);
\draw (3,13.5) to[R] (4.75,13.5);
\draw (4.75,13.5) to[short] (6,13.5);
\draw (5,13.5) to[short] (5,15.75);
\draw (5,15.75) to[short] (8,15.75);
\draw [short] (6,14) -- (6,12.25);
\draw [short] (6,14) -- (7.5,13);
\draw [short] (6,12.25) -- (7.5,13);
\draw [short] (6,12.75) -- (5,12.75);
\draw (5,12.75) to (5,12) node[ground]{};
\draw (6.75,13.5) to[short, -o] (6.75,14.25) ;
\draw (6.5,12.5) to[short, -o] (6.5,12) ;
\draw (7.5,13) to[short] (9.25,13);
\draw (9.25,13) to[short] (9.25,13.75);
\draw (8,15.75) to[short] (9.25,15.75);
\draw (9.25,15.75) to[short] (9.25,15);
\draw [short] (9.25,15) -- (8.75,14.75);
\draw [short] (8.75,15) -- (8.75,14.25);
\draw [short] (8.75,14.5) -- (8.25,14.5);
\draw [short] (8.75,14.25) -- (8.75,14);
\draw [->, >=Stealth] (8.75,14.25) -- (9.25,13.75);
\draw (8.25,14.5) to (8.25,14) node[ground]{};
\node [font=\normalsize] at (1,12.5) {$5V$};
\node [font=\normalsize] at (6.5,11.5) {$-15V$};
\node [font=\normalsize] at (6.5,14.75) {$15V$};
\node [font=\normalsize] at (9.25,14.5) {$Q$};
\node [font=\normalsize] at (10.25,12.5) {$V_{out}$};
\node [font=\normalsize] at (6,14.75) {$+$};
\draw (9.25,13) to[short, -o] (10,13) ;
\node [font=\normalsize] at (6.25,12.75) {$+$};
\node [font=\normalsize] at (6.25,13.5) {$-$};
\node [font=\normalsize] at (3.75,14) {$1k\ohm$};
\node at (5,13.5) [circ] {};
\node at (9.25,13) [circ] {};
\end{circuitikz}
}%

\label{fig:my_label}
\end{figure}

    \begin{enumerate}
        \begin{multicols}{2}
            \item $-15$
            \item $-0.7$
            \item $+0.7$
            \item $+15$
        \end{multicols}
    \end{enumerate}
    \item The transfer function $\frac{V_2\brak{s}}{V_1\brak{s}}$ of the circuit shown below is \hfill(2013-EE) 
    \begin{figure}[H]
\centering
\resizebox{6cm}{!}{%
\begin{circuitikz}
\tikzstyle{every node}=[font=\large]
\draw (0.25,17.75) to[short, -o] (-1,17.75) ;
\draw (-0.25,17.75) to[curved capacitor] (1.5,17.75);
\draw (1.5,17.75) to[short, -o] (4.75,17.75) ;
\draw (3.25,17.75) to[R] (3.25,15.5);
\draw (3.25,15.5) to[curved capacitor] (3.25,13.75);
\draw (3.25,13.75) to[short, -o] (4.5,13.75) ;
\draw (3.25,13.75) to[short, -o] (-1.25,13.75) ;
\node [font=\large] at (-1.25,15.75) {$V_1(s)$};
\node [font=\large] at (4,16.75) {$10k\ohm$};
\node [font=\large] at (4.5,14.75) {$100\mu F$};
\node [font=\large] at (0.5,18.5) {$100\mu F$};
\node [font=\large] at (-1,18) {$+$};
\node [font=\large] at (4.75,18.25) {$+$};
\node [font=\large] at (4.5,14) {$-$};
\node [font=\large] at (-1.25,14.25) {$-$};
\node [font=\large] at (5,15.75) {$V_2(s)$};
\end{circuitikz}
}%

\label{fig:my_label}
\end{figure}

    \begin{enumerate}
        \begin{multicols}{2}
            \item $\frac{0.5s+1}{s+1}$
            \item $\frac{3s+6}{s+2}$
            \item $\frac{s+2}{s+1}$
            \item $\frac{s+1}{s+2}$
        \end{multicols}
    \end{enumerate}
    \item Assuming zero initial condition, the response $y(t)$ of the system given below to a unit step input $u(t)$ is 
    \begin{figure}[H]
\centering
\resizebox{5cm}{!}{%
\begin{circuitikz}
\tikzstyle{every node}=[font=\large]
\draw [short] (1.25,18.75) -- (1.25,16.75);
\draw [short] (1.25,18.75) -- (4.5,18.75);
\draw [short] (4.5,18.75) -- (4.5,16.75);
\draw [short] (1.25,16.75) -- (4.5,16.75);

\draw [->, >=Stealth] (4.5,17.75) -- (6.25,17.75);
\draw [->, >=Stealth] (-0.5,17.75) -- (1.25,17.75);
\node [font=\Large] at (2.75,17.75) {$\frac{1}{s}$};

\node [font=\large] at (5.5,18.25) {$Y(s)$};
\node [font=\large] at (0.25,18.25) {$U(s)$};
\end{circuitikz}
}%

\label{fig:my_label}
\end{figure}
    \begin{enumerate}
        \begin{multicols}{2}
            \item $u(t)$
            \item $t u(t)$
            \item $\frac{t^2}{2} u(t)$ 
            \item $e^{-t}u(t)$
        \end{multicols}
    \end{enumerate}
    \item The impulse response of the system is $h(t) = t u(t)$. For an input $u(t-1)$, the output is \hfill(2013-EE)
    \begin{enumerate}
        \begin{multicols}{4}
            \item $\frac{t^2}{2}u(t)$
            \item $\frac{t(t-1)}{2}u(t-1)$
             \item $\frac{(t-1)^2}{2}u(t-1)$
              \item $\frac{t^2-1}{2}u(t-1)$
        \end{multicols}
    \end{enumerate}
    \item Which one of the following statements is \textbf{NOT TRUE} for a continuous time casual and stable LTI system? \hfill(2013-EE)
    \begin{enumerate}
        \item All the poles of the system must lie on the left side of the $j\omega$ axis
        \item Zeros of the system can lie anywhere in the $s-$ plane.
        \item All the poles must lie within $\abs{s} = 1$
        \item All the roots of the characteristic equation must be located on the left side of the $j\omega$ axis.
    \end{enumerate}
    \item Two systems with impulse response $h_1(t)$ and $h_2(t)$ are connected is cascade. Then the overall impulse response of the cascaded system is given by \hfill(2013-EE)
    \begin{enumerate}
        \item A product of $h_1(t)$ and $h_2(t)$
        \item Sum of $h_1(t)$ and $h_{2}(t)$
        \item Convolution of $h_1(t)$ and $h_2(t)$
        \item substraction of $h_2(t)$ from $h_1(t)$
    \end{enumerate}
    \item A source $v_s(t) = V\cos{100\pi t}$ has an internal impedance of $\brak{4 + j3} \ohm$. If a purely resistive load connected to this source has to extract the maximum power out of the source, its value in $\ohm$ should be \hfill(2013-EE)
    \begin{enumerate}
        \begin{multicols}{4}
            \item 3
            \item 4
            \item 5
            \item 7
        \end{multicols}
    \end{enumerate}
    \item A single-phase load is supplied by a single-phase voltage source. If the current flowing from the load to the source is $10 \angle-150^{\degree}$ $A$ and if the voltage at the load terminals is $100 \angle 60^{\degree}$, then the \hfill(2013-EE)
    \begin{enumerate}
        \item load absorbs real power real power and delivers reactive power
        \item load absorbs real power real power and absorbs reactive power
        \item load delivers real power real power and delivers reactive power
        \item load delivers real power real power and absorbs reactive power
    \end{enumerate}
    \item A single-phase transformer has no-load loss of $64W$, as obtained from an open-circuit test. When a short-circuit test is performed on it with $90\%$ of the rated currents flowing in its both $LV$ and $HV$ windings, the measured load is $81W$. The transformer has maximum efficiency when operated at \hfill(2013-EE)
    \begin{enumerate}
        \begin{multicols}{2}
            \item $50.0\%$ of the rated current.
            \item $64.0\%$ of the rated current.
            \item $80.0\%$ of the rated current.
            \item $88.8\%$ of the rated current.
        \end{multicols}
    \end{enumerate}
    \item The flux density at a point in space is given by $\textbf{B} = 4x \textbf{a}_\textbf{x} + 2ky\textbf{a}_\textbf{y} + 8\textbf{a}_\textbf{z} Wb/m^2$. The value of constant $k$ must be equal to \hfill(2013-EE) 
    \begin{enumerate}
        \begin{multicols}{4}
            \item $-2$
            \item $-0.5$
            \item $+0.5$
            \item $+2$
        \end{multicols}
    \end{enumerate}
    \item A continuous random variable $X$ has a probability density function $f(x) = e^{-x}, 0 < x < \infty$. Then $P\cbrak{X > 1}$ \hfill(2013-EE)
    \begin{enumerate}
        \begin{multicols}{4}
            \item 0.368
            \item 0.5
            \item 0.632
            \item 1.0
        \end{multicols}
    \end{enumerate}
    \item The curl of the gradient of the scalar field defined by $V = 2x^2y + 3y^2z 
 4z^2x$ is \hfill(2013-EE)
 \begin{enumerate}
     \item $4xy\textbf{a}_\textbf{x} + 6yz\textbf{a}_\textbf{y} + 8zx \textbf{a}_ \textbf{z}$
     \item $4\textbf{a}_\textbf{x}+ 6\textbf{a}_\textbf{z}+ 8\textbf{a}_\textbf{z}$
     \item $\brak{4xy + 4z^2}\textbf{a}_\textbf{x} + \brak{2x^2 + 6yz}\textbf{a}_\textbf{y} + \brak{3y^2 + 8zx}\textbf{a}_\textbf{z}$
     \item 0
 \end{enumerate}
 \item In the feedback network shown below, if the feedback factor $k$ is increased, then the 

 \hfill(2013-EE)
 \begin{figure}[H]
\centering
\resizebox{5cm}{!}{%
\begin{circuitikz}
\tikzstyle{every node}=[font=\large]
\draw [short] (-0.5,17.5) -- (-0.5,15.25);
\draw [short] (-0.5,17.5) -- (3.5,17.5);
\draw [short] (3.5,17.5) -- (3.5,15.25);
\draw [short] (3.5,15.25) -- (-0.5,15.25);
\draw [short] (3.5,16.75) -- (4.75,16.75);
\draw [short] (3.5,15.75) -- (4.75,15.75);
\draw [short] (4.75,16.75) -- (4.75,13);
\draw [short] (4.75,13) -- (3.5,13);
\draw [short] (4.75,15.75) -- (5.5,15.75);
\draw [short] (5.5,15.75) -- (5.5,12.25);
\draw [short] (5.5,12.25) -- (3.5,12.25);
\draw [short] (3.5,13.5) -- (3.5,11.75);
\draw [short] (3.5,13.5) -- (-0.5,13.5);
\draw [short] (-0.5,13.5) -- (-0.5,11.75);
\draw [short] (-0.5,11.75) -- (3.5,11.75);
\draw [short] (-0.5,15.75) -- (-1.25,15.75);
\draw [short] (-1.25,15.75) -- (-1.25,13);
\draw [short] (-0.5,13) -- (-1.25,13);
\draw [short] (-0.5,16.75) -- (-0.5,17);
\draw (-0.5,16.75) to[short, -o] (-3.25,16.75) ;
\draw [short] (-0.5,12.25) -- (-2.25,12.25);
\draw [short] (-2.25,12.25) -- (-2.25,15.75);
\draw (-2.25,15.75) to[short, -o] (-3.25,15.75) ;
\draw (4.75,16.75) to[short, -o] (6.75,16.75) ;
\draw (5.5,15.75) to[short, -o] (6.75,15.75) ;
\node at (4.75,16.75) [circ] {};
\node at (5.5,15.75) [circ] {};
\node [font=\large] at (4.25,16.5) {$+$};
\node [font=\large] at (4.25,16) {$-$};
\node [font=\large] at (6.5,16.25) {$V_{out}$};
\node [font=\large] at (3.75,12.75) {$+$};
\node [font=\Large] at (3.75,12.5) {$-$};
\node [font=\Large] at (1.5,12.75) {$k$};
\node [font=\Large] at (1.5,16.5) {$A_0$};
\node [font=\Large] at (-1.25,16.25) {$v_1$};
\node [font=\Large] at (-0.75,16.5) {$+$};
\node [font=\Large] at (-0.75,16) {$-$};
\node [font=\Large] at (-2.5,16.25) {$v_{in}$};
\node [font=\Large] at (-3.5,16.5) {$+$};
\node [font=\Large] at (-3.5,15.75) {$-$};
\node [font=\normalsize] at (-1.5,12.75) {$v_f = kv_{out}$};
\node [font=\large] at (-0.75,12.75) {$+$};
\node [font=\large] at (-0.75,12.5) {$-$};
\end{circuitikz}
}%
\end{figure}

 \begin{enumerate}
     \item input impedance increases and output impedance decrease  
     \item input impedance increases and output impedance also increase 
     \item input impedance decrease  and output impedance also decrease  
     \item input impedance decreases and output impedance increases
 \end{enumerate}


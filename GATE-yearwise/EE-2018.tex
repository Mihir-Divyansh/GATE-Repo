\iffalse
\chapter{2018}
\author{AI24BTECH11033}
\section{me}
\fi

\item Consider a non-singular $2x2$ square matrix $A$. If trace$\brak{A} = 4$ and trace$\brak{A^2} = 5$, the determinant of the matrix $A$ is \underline{\hspace{1cm}} (up to 1 decimal place ).
\item Let $f$ be a real-valued function of a real variable defined as $f\brak{x} = x - \sbrak{x}, \text{where} \sbrak{x}$ denotes the largest integer less than or equal to $x$. The value of $\int_{0.25}^{1.25} f\brak{x} dx.$ is \underline{\hspace{1cm}} (up to 2 decimal place ).
\item In the two-port network shown, the $h_{11}$ parameter $\brak{\text{where}, h_{11} = \frac{V_1}{I_1}, \text{when} V_2 = 0}$ in ohms is \underline{\hspace{1cm}} (up to 2 decimal place ).
\begin{figure}[!ht]
    \centering
    \resizebox{0.5\textwidth}{!}{%
\begin{circuitikz}
\tikzstyle{every node}=[font=\LARGE]
\draw (-6.25,17.5) to[R] (-2.5,17.5);
\draw (-2.5,17.5) to[R] (-2.5,13.75);
\draw (-2.5,17.5) to[R] (1.25,17.5);
\draw (-6.25,13.75) to[short] (1.25,13.75);
\draw (-2.5,17.5) to[short] (-2.5,20);
\draw (1.25,20) to[short] (1.25,17.5);
\draw (1.25,17.5) to[short] (2.5,17.5);
\draw (1.25,13.75) to[short] (2.5,13.75);
\node at (2.5,17.5) [circ] {};
\node at (2.5,13.75) [circ] {};
\node at (-6.25,17.5) [circ] {};
\node at (-6.25,13.75) [circ] {};
\draw (-2.5,20) to[american controlled current source] (1.25,20);
\node [font=\LARGE] at (-1.75,15.5) {1 $\Omega$};
\node [font=\LARGE] at (-0.5,18) {1 $\Omega$};
\node [font=\LARGE] at (-4.5,18) {1 $\Omega$};
\node [font=\LARGE] at (-0.25,20.75) {$2 I_1$};
\node [font=\LARGE] at (3,17.5) {+};
\node [font=\LARGE] at (2.75,13.75) {-};
\node [font=\LARGE] at (-6.5,13.5) {-};
\node [font=\LARGE] at (-6.5,17.25) {+};
\node [font=\LARGE] at (-6.5,16) {$V_1$};
\node [font=\LARGE] at (2.5,16) {$V_2$};
\draw [->, >=Stealth] (-4.75,17) -- (-3.75,17);
\end{circuitikz}
}%
  % Specify the path to your TikZ file
    \label{fig:power system network}
    \end{figure}
\item The series impedance matrix of a short three-phase transmission line in phase coordinates is:
$\myvec{
Z_s & Z_m & Z_m \\
Z_m & Z_s & Z_m \\
Z_m & Z_m & Z_s}$
If the positive sequence impedance is $(1 + j10) \Omega$, and the zero sequence impedance is $(4 + j31) \Omega$, then the imaginary part of $Z_m$ is \underline{\hspace{1cm}} (up to 2 decimal place ).
\item The positive, negative and zero sequence impedances of a 125 MVA, three-phase, 15.5 kV, star-grounded, 50 Hz generator are $j0.1$ pu, $j0.05$ pu and $j0.01$ pu respectively, on the machine rating base. The machine is unloaded and working at the rated terminal voltage. If the grounding impedance of the generator is $j0.01$ pu, then the magnitude of fault current for a b-phase to ground fault (in kA) is \underline{\hspace{1cm}} (up to 2 decimal place ). 
\item A 1000 x 1000 bus admittance matrix for an electric power system has 8000 non-zero elements. The minimum number of branches (transmission lines and transformers) in this system are  \underline{\hspace{1cm}} (up to 2 decimal place ).
\item The waveform of the current drawn by a semi-converter from a sinusoidal AC voltage source is shown in the figure. If $I_0 = 20$ A, the rms value of the fundamental component of the current is \underline{\hspace{1cm}} A (up to 2 decimal
    \begin{figure}[!ht]
    \centering
    \resizebox{0.5\textwidth}{!}{%
\begin{circuitikz}
\tikzstyle{every node}=[font=\large]
\draw [short] (-6.25,13.75) .. controls (-5.25,16.75) and (-4.25,15.75) .. (-3.75,13.75);
\draw [short] (-3.75,13.75) .. controls (-3,10.5) and (-1.75,11.5) .. (-1.25,13.75);
\draw [short] (-1.25,13.75) .. controls (0,17.75) and (0.75,14.75) .. (1.25,13.75);
\draw [short] (1.25,13.75) .. controls (2.5,10) and (3.25,12.75) .. (3.75,13.75);
\draw [->, >=Stealth, dashed] (-6.25,12.5) -- (-6.25,17.5);
\draw [->, >=Stealth, dashed] (-6.25,13.75) -- (5,13.75);
\node [font=\LARGE] at (5,13.25) {$\omega t$};
\node [font=\large] at (-6.5,18.25) {Voltage and};
\node [font=\large] at (-3.75,15.75) {$V_{m} sin(\omega t)$};
\draw [dashed] (-5.75,15) -- (-5.75,13.25);
\draw [dashed] (-3.75,14.5) -- (-3.75,13);
\draw [dashed] (-3.25,14.25) -- (-3.25,11.5);
\draw [short] (-6.25,13.75) -- (-5.75,13.75);
\draw [short] (-5.75,13.75) -- (-5.75,14.5);
\draw [short] (-5.75,14.5) -- (-3.75,14.5);
\draw [short] (-3.75,14.5) -- (-3.75,13.75);
\draw [short] (-3.75,13.75) -- (-3.25,13.75);
\draw [short] (-3.25,13.75) -- (-3.25,13);
\draw [short] (-3.25,13) -- (-1.25,13);
\draw [short] (-1.25,13) -- (-1.25,13.75);
\draw [short] (-1.25,13.75) -- (-0.75,13.75);
\draw [short] (-0.75,13.75) -- (-0.75,14.5);
\draw [short] (-0.75,14.5) -- (1.25,14.5);
\draw [short] (1.25,14.5) -- (1.25,13.75);
\draw [short] (1.25,13.75) -- (1.75,13.75);
\draw [short] (1.75,13.75) -- (3.75,13.75);
\draw [short] (1.75,13.75) -- (1.75,13);
\draw [short] (1.75,13) -- (3.75,13);
\draw [short] (3.75,13) -- (3.75,13.75);
\draw [->, >=Stealth] (-5,15.25) -- (-5,14.5);
\draw [->, >=Stealth] (-5,12.75) -- (-5,13.75);
\draw [->, >=Stealth] (-2.5,14.5) -- (-2.5,13.75);
\draw [->, >=Stealth] (-2.5,12.25) -- (-2.5,13);
\node [font=\large] at (-6.5,13.5) {0};
\node [font=\large] at (-5.75,13) {$30\degree$};
\node [font=\large] at (-5,14) {$I_{0}$};
\node [font=\large] at (-4,12.75) {$180\degree$};
\node [font=\large] at (-3.25,11.25) {$210\degree$};
\node [font=\large] at (-2.5,13.25) {$I_{0}$};
\node [font=\large] at (-6.75,17.75) {current};
\end{circuitikz}
}%
  % Specify the path to your TikZ file
    \label{fig:power system network}
    \end{figure}
\item A separately excited DC motor has an armature resistance $R_a = 0.05 \, \Omega$. The field excitation is kept constant. At an armature voltage of 100 V, the motor produces a torque of 500 Nm at zero speed. Neglecting all mechanical losses, the no-load speed of the motor $\brak{\text{in} \ \frac{radian}{s}}$ for an armature voltage of 150 V is \underline{\hspace{1cm}} (up to 2 decimal places).
\item Consider a unity feedback system with forward transfer function given by:
\begin{align}
G(s) = \frac{1}{(s+1)(s+2)}
\end{align}
The steady-state error in the output of the system for a unit-step input is \underline{\hspace{1cm}} (up to 2 decimal places).
\item A transformer with toroidal core of permeability $\mu$ is shown in the figure. Assuming uniform flux density across the circular core cross-section of radius $r << R$, and neglecting any leakage flux, the best estimate for the mean radius $R$ is
\begin{figure}[!ht]
    \centering
    \resizebox{0.5\textwidth}{!}{%
\begin{circuitikz}
\tikzstyle{every node}=[font=\large]
\draw  (5,11.25) circle (6.25cm);
\draw  (5,11.25) circle (4.5cm);
\draw [ dashed] (5,11.25) circle (5.5cm);
\draw [short] (5,17.5) .. controls (5.75,17.5) and (6,16) .. (5,15.75);
\draw [short] (5,17.5) .. controls (4.25,17.5) and (4,16) .. (5,15.75);
\node at (5,16.75) [circ] {};
\node at (5,11.25) [circ] {};
\draw [->, >=Stealth] (5,11.25) -- (8.5,15.5);
\draw [->, >=Stealth] (5,16.75) -- (5.5,17);
\draw [short] (-0.75,13.75) -- (-6.25,13.75);
\draw [short] (-0.75,8.75) -- (-6.25,8.75);
\draw (-6.25,8.75) to[sinusoidal voltage source, sources/symbol/rotate=auto] (-6.25,13.75);
\draw (10.75,13.75) to[short] (16.25,13.75);
\draw (10.75,8.75) to[short] (16.25,8.75);
\draw [short] (-0.75,8.75) .. controls (0.5,9) and (1.5,9.25) .. (1,9.25);
\draw [short] (-0.75,13.75) .. controls (-0.5,13.25) and (0.25,13.25) .. (1,13);
\draw [short] (-1,13) .. controls (-0.5,12.5) and (0,12.5) .. (0.75,12.5);
\draw [short] (-1.25,11.75) .. controls (-0.5,11.75) and (-0.25,12) .. (0.5,11.75);
\draw [short] (-1.25,10.75) .. controls (-0.5,10.75) and (-0.25,10.5) .. (0.5,11);
\draw [short] (-1,9.5) .. controls (-0.25,9.5) and (1,9.75) .. (0.75,10);
\draw [short] (10.75,13.75) .. controls (10,14) and (9.25,13.75) .. (9.25,12.75);
\draw [short] (10.75,8.75) .. controls (9.75,9) and (9.25,8.75) .. (9,9.25);
\draw [short] (11,12.75) .. controls (10.5,13) and (9.75,13) .. (9.5,12.25);
\draw [short] (11.25,12) .. controls (10.75,12.25) and (10.25,12.25) .. (9.5,11.5);
\draw [short] (9.5,10.75) .. controls (10.25,11.5) and (11,11.5) .. (11.25,11);
\draw [short] (9.25,10) .. controls (10.25,9.75) and (10.75,9.75) .. (11.25,10);
\draw [short] (9.25,9.75) .. controls (9.75,9.5) and (10,9.5) .. (11,9.25);
\draw [->, >=Stealth] (12.5,14.25) -- (15,14.25);
\draw [->, >=Stealth] (-5,14.25) -- (-2.5,14.25);
\node [font=\large] at (16.5,14) {+};
\node [font=\large] at (16.5,8.5) {-};
\node [font=\large] at (15,11) {$v_{s}$};
\node [font=\large] at (12,11) {$N_{s}$};
\node [font=\large] at (-6.5,8.5) {-};
\node [font=\large] at (-6.5,14) {+};
\node [font=\large] at (-4.5,10.5) {$V_{p} = V cos(\omega t)$};
\node [font=\large] at (-2,11) {$N_{p}$};
\node [font=\large] at (-3.75,14.75) {$i_{p} = I sin(\omega t)$};
\node [font=\large] at (13.75,14.75) {$i_{s} = 0$};
\node [font=\large] at (7,12.75) {R};
\node [font=\large] at (5,17) {r};
\end{circuitikz}
}%
  % Specify the path to your TikZ file
    \label{fig:power system network}
    \end{figure}
\begin{enumerate}
    \item $\frac{\mu V r^2 N_p^2 \omega}{I}$
    \item $\frac{\mu I r^2 N_p N_s \omega}{V}$
    \item $\frac{\mu V r^2 N_p^2 \omega}{2I}$
    \item $\frac{\mu I r^2 N_p^2 \omega}{2V}$
\end{enumerate}
\item A $0-1$ Ampere moving iron ammeter has an internal resistance of $50 m\Omega$ and inductance of $0.1 mH$. A shunt coil is connected to extend its range to $0-10$ Ampere for all operating frequencies. The time constant in milliseconds and resistance in $m\Omega$ of the shunt coil respectively are
\begin{enumerate}
      \item 2, 5.55
      \item 2, 1
      \item 2.18, 0.55
      \item 11.1, 2
  \end{enumerate}
  \item The positive, negative and zero sequence impedances of a three phase generator are $Z_1$, $Z_2$ and $Z_0$ respectively. For a line-to-line fault with fault impedance $Z_f$, the fault current is $I_f = k I_f$, where $I_f$ is the fault current with zero fault impedance. The relation between $Z_f$ and $k$ is
\begin{enumerate}
    \item $Z_f = \frac{(Z_1 + Z_2)(1 - k)}{k}$
    \item $Z_f = \frac{(Z_1 + Z_2)(1 + k)}{k}$
    \item $Z_f = \frac{(Z_1 + Z_2) k}{1 - k}$
    \item $Z_f = \frac{(Z_1 + Z_2) k}{1 + k}$
\end{enumerate}
\item Consider the two bus power system network with given loads as shown in the figure. All the values shown in the figure are in per unit. The reactive power supplied by generator $G_1$ and $G_2$ are $Q_{G1}$ and $Q_{G2}$ respectively. The per unit values of $Q_{G1}$, $Q_{G2}$, and line reactive power loss $\brak{Q_{\text{loss}}}$ respectively are
 \begin{figure}[!ht]
    \centering
    \resizebox{0.5\textwidth}{!}{%
\begin{circuitikz}
\tikzstyle{every node}=[font=\LARGE]
\draw  (-3.75,17.5) circle (1.25cm);
\draw  (11.5,17.5) circle (1.25cm);
\draw (-2.5,17.5) to[L ] (10.25,17.5);
\draw [short] (-1.25,18.25) -- (-1.25,16.25);
\draw [short] (8.75,18.25) -- (8.75,16.25);
\draw [short] (-1.25,16.75) -- (-0.75,16.75);
\draw [short] (8.75,16.75) -- (8.25,16.75);
\draw [->, >=Stealth] (-0.75,16.75) -- (-0.75,15);
\draw [->, >=Stealth] (8.25,16.75) -- (8.25,15);
\node [font=\LARGE] at (-4,17.5) {};
\node [font=\Huge] at (11.5,17.5) {$G_2$};
\node [font=\LARGE] at (11.5,15.75) {$15 + jQ_{G2}$};
\node [font=\LARGE] at (-3.75,15.75) {$20 + jQ_{G1}$};
\node [font=\LARGE] at (-1,14.5) {15 + j5};
\node [font=\LARGE] at (8,14.5) {20 + j10};
\node [font=\LARGE] at (4,16.75) {$Q_{loss}$};
\node [font=\LARGE] at (4,18.5) {j0.1};
\node [font=\LARGE] at (9,18.5) {1.0 $\angle$ 0};
\node [font=\LARGE] at (-1.25,18.5) {1.0 $\angle$ $\delta$};
\node [font=\Huge] at (-3.75,17.5) {$G_1$};
\end{circuitikz}
}%
  % Specify the path to your TikZ file
    \label{fig:power system network}
    \end{figure}
\begin{enumerate}
    \item 5.00, 12.68, 2.68
    \item 6.34, 10.00, 1.34
    \item 6.34, 11.34, 2.68
    \item 5.00, 11.34, 1.34
\end{enumerate}


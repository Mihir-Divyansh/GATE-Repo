\iffalse
\chapter{2016}
\author{AI24BTECH11011}
\section{ae}
\fi


\item For a laminar incompressible flow past a flat plate at zero angle of attack, the variation of skin friction drag coefficient $\brak{C_f}$ with Reynolds number based on the chord length $\brak{Re_c}$ can be expressed as
\begin{enumerate}
    \item ${C_f} \propto \sqrt{Re_c}$
    \item $C_f \propto Re_c$
    \item $C_f \propto \frac{1}{\sqrt{Re_c}}$
    \item $C_f \propto \frac{1}{Re_c}$
\end{enumerate}

\item Which of the following is NOT TRUE across an oblique shock wave?
\begin{enumerate}
    \item Static temperature increases, total temperature remains constant.
    \item Static pressure increases, static temperature increases.
    \item Static temperature increases, total pressure decreases.
    \item Static pressure increases, total temperature decreases.
\end{enumerate}

\item For a completely subsonic isentropic flow through a convergent nozzle, which of the following statement is TRUE?
\begin{enumerate}
    \item Pressure at the nozzle exit $>$ back pressure.
    \item Pressure at the nozzle exit $>$ back pressure.
    \item Pressure at the nozzle exit $=$ back pressure.
    \item Pressure at the nozzle exit $=$ total pressure.
\end{enumerate}

\item Which of the following aircraft engines has the highest propulsive efficiency at cruising Mach number of less than 0.5?
\begin{enumerate}
    \item Turbofan engine
    \item Turbojet engine
    \item Turboprop engine
    \item Ramjet engine
\end{enumerate}

\item Air, with a Prandtl number of 0.7, flows over a flat plate at a high Reynolds number. Which of the following statement is TRUE?
\begin{enumerate}
    \item Thermal boundary layer is thicker than the velocity boundary layer.
    \item Thermal boundary layer is thinner than the velocity boundary layer.
    \item Thermal boundary layer is as thick as the velocity boundary layer.
    \item There is no relationship between the thickness of thermal and velocity boundary layers.
\end{enumerate}

\item Consider an eigenvalue problem given by $\textbf{Ax} = {\lambda_i}\textbf{x}$. If $\lambda_i$ represents the eigenvalues of the non-singular square matrix $\textbf{A}$, then what will be the eigenvalues of the matrix $\textbf{A}^2$?
\begin{enumerate}
   \item $\lambda_i^4$
    \item $\lambda_i^2$
    \item $\lambda_i^{\frac{1}{2}}$
    \item $\lambda_i^{\frac{1}{4}}$
\end{enumerate}

\item If $\textbf{A}$ and $\textbf{B}$ are both non-singular n$\times$n matrices, then which of the following statement is NOT TRUE. Note: det represents the determinant of the matrix.
\begin{enumerate}
    \item $det\brak{\textbf{AB}} = det\brak{\textbf{A}}det\brak{\textbf{B}}$
    \item $det\brak{\textbf{A}+\textbf{B}} = det \brak{\textbf{A}} + det \brak{\textbf{B}}$
    \item $det\brak{\textbf{AA}^{-\textbf{1}}}$ = 1
    \item $det\brak{\textbf{A}^{\textbf{T}}} = det\brak{\textbf{A}}$
\end{enumerate}

\item The total number of material constants that are necessary and sufficient to describe the three dimensional Hooke's law for an isotropic material is $\makebox[1cm][l]{\underline{\hspace{1cm}}}$.

\item Determine the correctness or otherwise of the following statements [a] and [r]:\\

[a]: In a plane stress problem, the shear strains along the thickness direction of a body are zero but the normal strain along the thickness is not zero.\\

[r]: In a plane stress problem, Poisson effect induces the normal strain along the thickness direction of the body.
\begin{enumerate}
    \item Both [a] and [r] are true and [r] is the correct reason for [a].
    \item Both [a] and [r] are true but [r] is not the correct reason for [a].
    \item Both [a] and [r] are false.
    \item
    
    [a] is true but [r] is false
\end{enumerate}

\item Consider four thin-walled beams of different open cross-sections, as shown in the cases \brak{i-iv}. A shear force of magnitude '\textbf{F}' acts vertically downward at the location '\textbf{P}' in all the beams. In which of the following case, does the shear force induce bending and twisting?  
\begin{figure}[!ht]
\centering
\resizebox{0.5\textwidth}{!}{%
\begin{circuitikz}
\tikzstyle{every node}=[font=\normalsize]
\draw [short] (1.5,10.5) -- (1.5,6.5);
\draw [short] (1.5,6.5) -- (5,6.5);
\draw [short] (8,10.75) -- (5.5,10.75);
\draw [short] (5.5,10.75) -- (5.5,6.5);
\draw [short] (5.5,6.5) -- (8.25,6.5);
\draw [short] (10.25,10.75) -- (10.25,6.5);
\draw [short] (8.5,10.75) -- (11.75,10.75);
\draw [short] (14,10.75) -- (14,6.5);
\draw [short] (12,8.5) -- (15.75,8.5);
\draw [line width=1.2pt, ->, >=Stealth] (1.5,8.75) -- (1.5,6.75);
\draw [line width=1.2pt, ->, >=Stealth] (5.5,8.75) -- (5.5,6.75);
\draw [line width=1.2pt, ->, >=Stealth] (10.25,10.75) -- (10.25,8.75);
\draw [line width=1.2pt, ->, >=Stealth] (14,8.5) -- (14,7);
\node at (1.5,6.5) [circ] {};
\node at (5.5,8.75) [circ] {};
\node at (10.25,10.75) [circ] {};
\node at (14,8.5) [circ] {};
\node [font=\normalsize] at (1.25,6.5) {P};
\node [font=\normalsize] at (2,8) {F};
\node [font=\normalsize] at (5.25,8.75) {P};
\node [font=\normalsize] at (5.75,6.75) {F};
\node [font=\normalsize] at (10,10.5) {P};
\node [font=\normalsize] at (10.5,8.75) {F};
\node [font=\normalsize] at (13.75,8.75) {P};
\node [font=\normalsize] at (14.25,7) {F};
\node [font=\normalsize] at (1.5,11) {Case(i)};
\node [font=\normalsize] at (5.75,11.25) {case(ii)};
\node [font=\normalsize] at (9,11.25) {case(iii)};
\node [font=\normalsize] at (12.75,11.25) {case(iv)};
\end{circuitikz}
}%
\end{figure}
\begin{enumerate}
    \item \brak{i}
    \item \brak{ii}
    \item \brak{iii}
    \item \brak{iv}
\end{enumerate}

\item The effective stiffness of the spring-mass system as shown in the figure below is $\makebox[1cm][l]{\underline{\hspace{1cm}}}$$\frac{N}{mm}$.   
\begin{figure}[!ht]
\centering
\resizebox{0.3\textwidth}{!}{%
\begin{circuitikz}
\tikzstyle{every node}=[font=\normalsize]
\draw [line width=0.2pt, short] (5.25,10) -- (10.5,10);
\draw [short] (5.25,10) -- (10.5,10);
\draw [short] (10.25,10) -- (10.25,9.25);
\draw [short] (10.25,10) -- (10.25,9.25);
\draw [short] (10.25,9.25) -- (10,9);
\draw [short] (10,9) -- (10.25,8.75);
\draw [short] (10.25,8.75) -- (10,8.5);
\draw [short] (10,8.5) -- (10.25,8.25);
\draw [short] (10.25,8.25) -- (10.25,7.75);
\draw [short] (10.25,7.75) -- (10,7.5);
\draw [short] (10,7.5) -- (10.25,7.25);
\draw [short] (10.25,7.25) -- (10,7);
\draw [short] (10,7) -- (10.25,6.75);
\draw [short] (10.25,6.75) -- (10.25,6.25);
\draw [short] (5.5,10) -- (5.5,8.5);
\draw [short] (5.5,8.5) -- (5.25,8.25);
\draw [short] (5.25,8.25) -- (5.5,8);
\draw [short] (5.5,8) -- (5.25,7.75);
\draw [short] (5.25,7.75) -- (5.5,7.5);
\draw [short] (5.5,7.5) -- (5.25,7.25);
\draw [short] (5.25,7.25) -- (5.5,7);
\draw [short] (5.5,7) -- (5.5,6.5);
\draw [short] (5.5,6.5) -- (5.5,6.25);
\draw [short] (5.5,6.25) -- (10.25,6.25);
\draw [short] (8,6.25) -- (8,6);
\draw  (7.25,6) rectangle (8.75,5.25);
\draw [short] (5.25,10) -- (7.25,10);
\draw [short] (5.25,10) -- (5.75,10.5);
\draw [short] (6,10) -- (6.5,10.5);
\draw [short] (6.75,10) -- (7.25,10.5);
\draw [short] (7.5,10) -- (8,10.5);
\draw [short] (8,10) -- (8.5,10.5);
\draw [short] (8.75,10) -- (10.75,10);
\draw [short] (8.75,10) -- (9.25,10.5);
\draw [short] (9.25,10) -- (9.75,10.5);
\draw [short] (10,10) -- (10.5,10.5);
\draw [short] (10.5,10) -- (11,10.5);
\node [font=\large] at (8,5.75) {m};
\node [font=\normalsize] at (11,8.75) {2N/mm};
\node [font=\normalsize] at (11,7.25) {2N/mm};
\node [font=\normalsize] at (4.5,7.75) {4N/mm};
\end{circuitikz}
}%
\end{figure}
\item A structural member supports loads, which produce at a particular point, a state of pure shear stress of 50 $\frac{N}{mm^2}$. At what angles are the principal planes oriented with respect to the plane of pure shear?
\begin{enumerate}
    \item $\frac{\pi}{6}$ and $\frac{2\pi}{3}$
    \item $\frac{\pi}{4}$ and $\frac{3\pi}{4}$
    \item $\frac{\pi}{4}$ and $\frac{\pi}{2}$
    \item $\frac{\pi}{2}$ and $\pi$
\end{enumerate}

\item Let x be a positive real number. The function $f\brak{x} = {x^2} + \frac{1}{x^2}$ has its minima at x= $\makebox[1cm][l]{\underline{\hspace{1cm}}}$.



 

\iffalse
\chapter{2024}
\author{AI24BTECH11011}
\section{ph}
\fi



\item If a thermodynamical system is adiabatically isolated and experiences a change in volume under an externally applied constant pressure, then the thermodynamical potential minimized at equilibrium is the
\begin{enumerate}
    \item enthalpy
    \item Helmholtz free energy
    \item Gibbs free energy
    \item grand potential
\end{enumerate}
\item The mean distance between the two atoms of HD molecules is $r$, where H and D denote hydrogen and deuterium, respectively. The mass of the hydrogen atom is $m_H$. The energy difference between two lowest lying rotational states of HD in multiples of $\frac{h^2}{m_Hr^2}$ is
\begin{enumerate}
    \item $\frac{3}{2}$
    \item $\frac{2}{3}$
    \item 6
    \item $\frac{4}{3}$
\end{enumerate}
\item Crystal structures of two metals A and B are two-dimensional square lattices with same lattice constant $a$. Electrons in metal behave as free electrons. The Fermi surfaces corresponding to A and B are shown by solid circles in figures
\begin{figure}[!ht]
\centering
\resizebox{0.5\textwidth}{!}{%
\begin{circuitikz}
\tikzstyle{every node}=[font=\normalsize]
\draw  (5.25,13.25) circle (2cm);
\draw  (11,13.25) circle (2cm);
\draw [ dashed] (3.25,15.25) rectangle  (7.25,11.25);
\draw [ dashed] (24,16.75) rectangle  (24,16.25);
\draw [ dashed] (9.5,14.5) rectangle  (12.5,12);
\draw [->, >=Stealth] (5.25,13) -- (5.25,16.25);
\draw [->, >=Stealth] (5.25,13) -- (8.5,13);
\draw [->, >=Stealth] (11,13.25) -- (11,16.25);
\draw [->, >=Stealth] (11,13.25) -- (13.75,13.25);
\draw [->, >=Stealth] (8,11.75) -- (7.5,12.75);
\draw [->, >=Stealth] (12.5,13.25) -- (12.75,13.25);
\draw [->, >=Stealth] (13.5,12.25) -- (12.5,13.25);
\node [font=\normalsize] at (14,13.25) {K};
\node [font=\normalsize] at (5,16.5) {K};
\node [font=\normalsize] at (10.5,16.25) {K};
\node [font=\normalsize] at (8.75,13) {K};
\node [font=\scriptsize] at (9,12.75) {X};
\node [font=\scriptsize] at (14.25,13) {X};
\node [font=\scriptsize] at (5.25,16.5) {y};
\node [font=\scriptsize] at (10.75,16.25) {y};
\node [font=\normalsize] at (5.25,12.75) {(0,0)};
\node [font=\normalsize] at (10.75,13) {(0,0)};
\node [font=\Large] at (13.75,12.25) {(};
\node [font=\small] at (14,12.5) {$\pi$};
\node [font=\small] at (14,12.25) {--};
\node [font=\small] at (13.5,8) {--};
\node [font=\small] at (14,12.25) {--};
\node [font=\small] at (14,12) {a};
\node [font=\normalsize] at (14.5,12.25) {,0};
\node [font=\Large] at (14.75,12.25) {)};
\node [font=\Large] at (8,11.25) {(};
\node [font=\Large] at (9,11.25) {)};
\node [font=\small] at (8.25,11.5) {$\pi$};
\node [font=\small] at (8.25,11.25) {--};
\node [font=\small] at (8.25,11) {a};
\node [font=\normalsize] at (8.75,11.25) {,0};
\node [font=\normalsize] at (5.5,10.75) {Metal A};
\node [font=\normalsize] at (11.25,10.75) {Metal B};
\end{circuitikz}
}%
\end{figure}
The electron concentrations in A and B are $n_A$ and $n_B$ respectively. The value of $\frac{n_B}{n_A}$ is
\begin{enumerate}
    \item 3
    \item 2
    \item $3\sqrt{3}$
    \item $\sqrt{2}$
\end{enumerate}
\item Consider the induced nuclear fission reaction 
$$^{235}_{92}U+n\rightarrow^{93}_{37}Rb+^{141}_{55}Cs+2n$$ where neutron momenta in both initial and final states are negligible. The ratio of the kinetic energies (KE) of the daughter nuclei,
$$\frac{KE\brak{^{93}_{37}Rb}}{KE\brak{^{141}_{55}Cs}}$$ is  $\makebox[3cm][l]{\underline{\hspace{1cm}}}$.
\begin{enumerate}
    \item $\frac{93}{141}$
    \item $\frac{141}{93}$
    \item 1
    \item 0
\end{enumerate}
\item The symbols $C,D,V_{in}$ and $V_{0}$ shown in the figure denote capacitor, ideal diode , input voltage and output voltage, respectively,
\begin{figure}[!ht]
\centering
\resizebox{0.5\textwidth}{!}{%
\begin{circuitikz}
\tikzstyle{every node}=[font=\scriptsize]
\draw [<->, >=Stealth] (3.5,12.75) -- (3.5,9.25);
\draw [dashed] (3.5,11) -- (5.5,11);
\draw [->, >=Stealth, dashed] (3.5,11) -- (8,11);
\draw [short] (3.25,9.75) -- (4.25,9.75);
\draw [short] (4.25,9.75) -- (4.25,12.25);
\draw [short] (4.25,12.25) -- (5,12.25);
\draw [short] (5,12.25) -- (5,9.75);
\draw [short] (5,9.75) -- (5.75,9.75);
\draw [short] (5.75,9.75) -- (5.75,12.25);
\draw [short] (5.75,12.25) -- (6.5,12.25);
\draw [short] (6.5,12.25) -- (6.5,9.75);
\draw [short] (6.5,9.75) -- (7.25,9.75);
\node [font=\normalsize] at (3,12.75) {V};
\node [font=\scriptsize] at (3.25,12.5) {in};
\node [font=\normalsize] at (3,12.25) {+3V};
\node [font=\normalsize] at (3,9.75) {-3V};
\draw [short] (3.25,12.25) -- (3.75,12.25);
\node [font=\normalsize] at (8,10.75) {t};
\draw (15.75,12.75) to[C] (10.75,12.75);
\draw (10.75,12.75) to[sinusoidal voltage source, sources/symbol/rotate=auto] (10.75,9.75);
\draw [short] (11,9.75) -- (10.75,9.75);
\draw [short] (10.75,9.75) -- (16,9.75);
\draw (14.5,9.5) to[D] (14.5,12.75);
\draw (14.5,10) to (14.5,9.75) node[ground]{};
\node [font=\normalsize] at (16,12.75) {+};
\node [font=\normalsize] at (16.25,9.75) {-};
\node [font=\normalsize] at (16,11.5) {V};
\node [font=\scriptsize] at (16.25,11.5) {0};
\node [font=\normalsize] at (14,11) {D};
\node [font=\normalsize] at (13,13.25) {C};
\node [font=\normalsize] at (10,11.25) {V};
\node [font=\scriptsize] at (10.25,11.25) {in};
\node at (14.5,12.75) [circ] {};
\node at (16,9.75) [circ] {};
\node at (15.75,12.75) [circ] {};
\end{circuitikz}
}%
\end{figure}
Which one of the following output waveforms $\brak{V_0}$ is correct for the given input waveform $\brak{V_{in}}$
\begin{enumerate}
\item
	\begin{figure}[!ht]
\centering
\resizebox{0.2\textwidth}{!}{%
\begin{circuitikz}
\tikzstyle{every node}=[font=\scriptsize]
\draw [<->, >=Stealth] (4.75,14.5) -- (4.75,9.5);
\draw [->, >=Stealth, dashed] (4.75,10.25) -- (9.75,10.25);
\draw [short] (5,10.25) -- (4.75,10.25);
\draw [short] (4.75,10.25) -- (5.5,10.25);
\draw [line width=0.6pt, short] (4.75,10.25) -- (5.5,10.25);
\draw [line width=0.6pt, short] (5.5,10.25) -- (5.5,13);
\draw [line width=0.6pt, short] (5.5,13) -- (6.5,13);
\draw [line width=0.6pt, short] (6.5,13) -- (6.5,10.25);
\draw [line width=0.6pt, short] (6.5,10.25) -- (7.5,10.25);
\draw [line width=0.6pt, short] (7.5,10.25) -- (7.5,13);
\draw [line width=0.6pt, short] (7.5,13) -- (8.5,13);
\draw [line width=0.6pt, short] (8.5,13) -- (8.5,10.25);
\draw [line width=0.6pt, short] (8.5,10.25) -- (9,10.25);
\node [font=\normalsize] at (4,14.25) {V};
\node [font=\scriptsize] at (4.25,14.25) {0};
\node [font=\normalsize] at (4,13) {+6V};
\node [font=\normalsize] at (4.25,10.25) {0V};
\node [font=\normalsize] at (9.75,9.75) {t};
\end{circuitikz}

}%
\end{figure}
\item 
\begin{figure}[!ht]
\centering
\resizebox{0.2\textwidth}{!}{%
\begin{circuitikz}
\tikzstyle{every node}=[font=\scriptsize]
\draw [<->, >=Stealth] (4.75,14.5) -- (4.75,9.5);
\draw [->, >=Stealth, dashed] (4.75,10.25) -- (9.75,10.25);
\draw [short] (5,10.25) -- (4.75,10.25);
\draw [short] (4.75,10.25) -- (5.5,10.25);
\draw [line width=0.6pt, short] (4.75,10.25) -- (5.5,10.25);
\draw [line width=0.6pt, short] (6.5,10.25) -- (7.5,10.25);
\draw [line width=0.6pt, short] (8.5,10.25) -- (9,10.25);
\node [font=\normalsize] at (4,14.25) {V};
\node [font=\scriptsize] at (4.25,14.25) {0};
\node [font=\normalsize] at (4,13) {+6V};
\node [font=\normalsize] at (4.25,10.25) {0V};
\node [font=\normalsize] at (9.75,9.75) {t};
\draw [line width=0.6pt, short] (4.5,13) -- (5,13);
\draw [line width=0.6pt, short] (4.5,10.25) -- (5,10.25);
\draw [line width=0.6pt, short] (5.5,10.25) -- (6,13);
\draw [line width=0.6pt, short] (6,13) -- (6.5,10.25);
\draw [line width=0.6pt, short] (7.5,10.25) -- (8,13);
\draw [line width=0.6pt, short] (8,13) -- (8.5,10.25);
\end{circuitikz}
}%
\end{figure}
\item 
	\begin{figure}[!ht]
\centering
\resizebox{0.2\textwidth}{!}{%
\begin{circuitikz}
\tikzstyle{every node}=[font=\normalsize]
\draw [<->, >=Stealth] (4.75,14.5) -- (4.75,9.5);
\draw [->, >=Stealth, dashed] (4.75,10.25) -- (9.75,10.25);
\draw [short] (5,10.25) -- (4.75,10.25);
\draw [short] (4.75,10.25) -- (5.5,10.25);
\node [font=\normalsize] at (4,14.25) {V};
\node [font=\scriptsize] at (4.25,14.25) {0};
\node [font=\normalsize] at (4,13) {+6V};
\node [font=\normalsize] at (4.25,10.25) {0V};
\node [font=\normalsize] at (9.75,9.75) {t};
\draw [line width=0.6pt, short] (4.5,13) -- (5,13);
\draw [line width=0.6pt, short] (4.5,10.25) -- (5,10.25);
\node [font=\normalsize] at (4,11.75) {+3V};
\draw [line width=0.6pt, short] (4.5,11.75) -- (5.25,11.75);
\draw [line width=0.6pt, short] (5.25,11.75) -- (5.25,13);
\draw [line width=0.6pt, short] (5.25,13) -- (6.25,13);
\draw [line width=0.6pt, short] (6.25,13) -- (6.25,11.75);
\draw [line width=0.6pt, short] (6.25,11.75) -- (7,11.75);
\draw [line width=0.6pt, short] (7,11.75) -- (7,13);
\draw [line width=0.6pt, short] (7,13) -- (8,13);
\draw [line width=0.6pt, short] (8,13) -- (8,11.75);
\draw [line width=0.6pt, short] (8,11.75) -- (8.75,11.75);
\end{circuitikz}
}%
\end{figure}
\item \begin{figure}[!ht]
\centering
\resizebox{0.2\textwidth}{!}{%
\begin{circuitikz}
\tikzstyle{every node}=[font=\normalsize]
\draw [<->, >=Stealth] (4.75,14.5) -- (4.75,9.5);
\draw [->, >=Stealth, dashed] (4.75,13) -- (9.75,13);
\draw [short] (5,10.5) -- (4.75,10.5);
\node [font=\normalsize] at (4,14.25) {V};
\node [font=\scriptsize] at (4.25,14.25) {0};
\node [font=\normalsize] at (4,13) {0V};
\node [font=\normalsize] at (10,12.75) {t};
\draw [line width=0.6pt, short] (4.5,13) -- (5,13);
\draw [line width=0.6pt, short] (4.5,10.5) -- (5,10.5);
\draw [line width=0.6pt, short] (4.5,11.75) -- (5.25,11.75);
\draw [line width=0.6pt, short] (5.25,10.5) -- (5.25,11.75);
\draw [line width=0.6pt, short] (5.25,10.5) -- (6.25,10.5);
\draw [line width=0.6pt, short] (6.25,11.75) -- (6.25,10.5);
\draw [line width=0.6pt, short] (6.25,11.75) -- (7,11.75);
\draw [line width=0.6pt, short] (7,10.5) -- (7,11.75);
\draw [line width=0.6pt, short] (7,10.5) -- (8,10.5);
\draw [line width=0.6pt, short] (8,11.75) -- (8,10.5);
\draw [line width=0.6pt, short] (8,11.75) -- (8.75,11.75);
\node [font=\normalsize] at (4.25,11.75) {-3V};
\node [font=\normalsize] at (4,10.5) {-6V};
\end{circuitikz}
}%
\end{figure}
\end{enumerate}
\item Let $N_0$ and $T_e$ respectively, denote number and kinetic energy of electrons produced in a nuclear beta decay. Which one of the following ditributions is correct?
\begin{enumerate}
\item \begin{figure}[!ht]
\centering
\resizebox{0.2\textwidth}{!}{%
\begin{circuitikz}
\tikzstyle{every node}=[font=\normalsize]
\draw [line width=0.6pt, ->, >=Stealth] (4,9.5) -- (4,13.75);
\draw [line width=0.6pt, ->, >=Stealth] (3.75,9.5) -- (8.25,9.5);
\node [font=\normalsize] at (3.5,13.5) {N};
\node [font=\scriptsize] at (3.75,13.5) {e};
\node [font=\normalsize] at (3.5,9.25) {(0,0)};
\node [font=\normalsize] at (8.25,9.25) {T};
\node [font=\scriptsize] at (8.5,9) {e};
\draw [line width=0.6pt, short] (7.5,12.5) -- (7.5,9.5);
\end{circuitikz}
}%
\end{figure}
\item \begin{figure}[!ht]
\centering
\resizebox{0.2\textwidth}{!}{%
\begin{circuitikz}
\tikzstyle{every node}=[font=\normalsize]
\draw [line width=0.6pt, ->, >=Stealth] (4,9.5) -- (4,13.75);
\draw [line width=0.6pt, ->, >=Stealth] (3.75,9.5) -- (8.25,9.5);
\node [font=\normalsize] at (3.5,13.5) {N};
\node [font=\scriptsize] at (3.75,13.5) {e};
\node [font=\normalsize] at (3.5,9.25) {(0,0)};
\node [font=\normalsize] at (8.25,9.25) {T};
\node [font=\scriptsize] at (8.5,9) {e};
\draw [line width=0.6pt, short] (4,9.5) .. controls (6.5,14.25) and (6,14.5) .. (8,9.5);
\end{circuitikz}
}%
\end{figure}
\item \begin{figure}[!ht]
\centering
\resizebox{0.2\textwidth}{!}{%
\begin{circuitikz}
\tikzstyle{every node}=[font=\normalsize]
\draw [line width=0.6pt, ->, >=Stealth] (4,9.5) -- (4,13.75);
\draw [line width=0.6pt, ->, >=Stealth] (3.75,9.5) -- (8.25,9.5);
\node [font=\normalsize] at (3.5,13.5) {N};
\node [font=\scriptsize] at (3.75,13.5) {e};
\node [font=\normalsize] at (3.5,9.25) {(0,0)};
\node [font=\normalsize] at (8.25,9.25) {T};
\node [font=\scriptsize] at (8.5,9) {e};
\draw [line width=0.6pt, short] (4,12.25) .. controls (5.5,15) and (5.25,10.25) .. (7.75,9.5);
\end{circuitikz}
}%
\end{figure}
\item \begin{figure}[!ht]
\centering
\resizebox{0.2\textwidth}{!}{%
\begin{circuitikz}
\tikzstyle{every node}=[font=\normalsize]
\draw [line width=0.6pt, ->, >=Stealth] (4,9.5) -- (4,13.75);
\draw [line width=0.6pt, ->, >=Stealth] (3.75,9.5) -- (8.25,9.5);
\node [font=\normalsize] at (3.5,13.5) {N};
\node [font=\scriptsize] at (3.75,13.5) {e};
\node [font=\normalsize] at (3.5,9.25) {(0,0)};
\node [font=\normalsize] at (8.25,9.25) {T};
\node [font=\scriptsize] at (8.5,9) {e};
\draw [line width=0.6pt, short] (4,9.5) .. controls (6.5,10.5) and (6.25,11.75) .. (7.25,13);
\draw [line width=0.6pt, short] (7.25,13) .. controls (7.5,13.5) and (7.75,13) .. (7.75,12.5);
\end{circuitikz}
}%
\end{figure}
\end{enumerate}
\item An infinitely long cylinder of radius R carries a frozen-in magnetization $\Bar{M}=ke^{-s}\hat{Z},$ where k is a constant and s is the distance from the axis of cylinder. The magnetic permeability of free space is $\mu_{0}$. There is no free current present anywhere. The magnetic flux density $\brak{\Bar{B}}$ inside the cylinder is
\begin{enumerate}
    \item 0
    \item $\mu_0ke^-R\hat{z}$
    \item $\mu_0ke^-R\hat{z}$
    \item $\mu_0ke^{-s}\brak{\frac{R}{S}}\hat{z}$
\end{enumerate}
\item Atomic numbers of V,Cr,Fe and Zn are 23,24,26 and 30, respectively. Which one of the following materials does NOT show an electron spin resonance $\brak{ESR}$ spectra?
\begin{enumerate}
    \item V
    \item Cr
    \item Fe
    \item Zn
\end{enumerate}
\item A particle is subjected to a potiential 
\[
V\brak{x}=
\begin{cases}
\infty & x\leq 0\\
V_0 & a\leq x \leq b\\
0 & ,elsewhere\\
\end{cases}
\]
Here, $a>0$ and $b>a$. If the energy of the particle $E<V_0$, which one of the following schematics is a valid quantum mechanical wavefunction $\brak{\psi}$ for the system?
\begin{enumerate}
    \item \begin{figure}[!ht]
\centering
\resizebox{0.3\textwidth}{!}{%
\begin{circuitikz}
\tikzstyle{every node}=[font=\scriptsize]
\draw [line width=0.6pt, ->, >=Stealth] (4,9.5) -- (4,13.75);
\draw [line width=0.6pt, ->, >=Stealth] (3.75,9.5) -- (9.5,9.5);
\node [font=\normalsize] at (3.5,13.5) {};
\node [font=\normalsize] at (3.5,9.25) {(0,0)};
\draw [short] (4,9.5) .. controls (5.25,12.25) and (5,13) .. (6.25,11.25);
\draw [short] (6.25,11.25) -- (6.5,12);
\draw [short] (6.5,12) .. controls (6.5,11.25) and (6.5,11.5) .. (7.5,11);
\draw (7.5,11) to[R] (8.5,11);
\draw [ dashed] (6.25,12.5) rectangle  (7.5,9.5);
\node [font=\normalsize] at (6.25,9.25) {a};
\node [font=\normalsize] at (7.5,9.25) {b};
\node [font=\normalsize] at (9.5,9.25) {x};
\node [font=\normalsize] at (5.25,12.5) {$\psi$};
\node [font=\normalsize] at (6.5,12.75) {V};
\node [font=\scriptsize] at (6.75,12.75) {0};
\end{circuitikz}
}%
\end{figure}
    \item \begin{figure}[!ht]
\centering
\resizebox{0.3\textwidth}{!}{%
\begin{circuitikz}
\tikzstyle{every node}=[font=\scriptsize]
\draw [line width=0.6pt, ->, >=Stealth] (4,9.5) -- (4,13.75);
\draw [line width=0.6pt, ->, >=Stealth] (3.75,9.5) -- (9.5,9.5);
\node [font=\normalsize] at (3.5,13.5) {};
\node [font=\normalsize] at (3.5,9.25) {(0,0)};
\draw [short] (4,9.5) .. controls (5.25,12.25) and (5.5,13.5) .. (6.25,11.25);
\draw (7.25,10.25) to[R] (8.25,10.25);
\node [font=\normalsize] at (6,9.25) {a};
\node [font=\normalsize] at (7.25,9.25) {b};
\node [font=\normalsize] at (9.5,9.25) {x};
\node [font=\normalsize] at (5.5,12.75) {$\psi$};
\node [font=\normalsize] at (6.5,12.75) {V};
\node [font=\scriptsize] at (6.75,12.5) {0};
\draw [short] (6.25,11.25) .. controls (6.5,10.75) and (6.75,10.25) .. (7.25,10.25);
\draw [ dashed] (6,12.25) rectangle  (7.25,9.5);
\end{circuitikz}
}%
\end{figure}
    \item \begin{figure}[!ht]
\centering
\resizebox{0.3\textwidth}{!}{%
\begin{circuitikz}
\tikzstyle{every node}=[font=\normalsize]
\draw [line width=0.6pt, ->, >=Stealth] (4,9.5) -- (4,13.75);
\draw [line width=0.6pt, ->, >=Stealth] (4,9.5) -- (8.75,9.5);
\node [font=\normalsize] at (3.5,13.5) {};
\node [font=\normalsize] at (3.5,9.25) {(0,0)};
\draw [short] (4,9.5) .. controls (5,12.25) and (5.25,13.5) .. (6,11.25);
\draw [ dashed] (6,12) rectangle  (7,9.5);
\end{circuitikz}
}%
\end{figure}
    \item \begin{figure}[!ht]
\centering
\resizebox{0.3\textwidth}{!}{%
\begin{circuitikz}
\tikzstyle{every node}=[font=\normalsize]
\draw [line width=0.6pt, ->, >=Stealth] (4,9.5) -- (4,13.75);
\draw [line width=0.6pt, ->, >=Stealth] (3.75,9.5) -- (9.5,9.5);
\node [font=\normalsize] at (3.5,13.5) {};
\node [font=\normalsize] at (3.5,9.25) {(0,0)};
\draw [short] (4,10.25) .. controls (5.25,12.25) and (5.5,14) .. (6.25,11.25);
\draw (7.25,10.25) to[R] (8.25,10.25);
\node [font=\normalsize] at (6,9.25) {a};
\node [font=\normalsize] at (7.25,9.25) {b};
\node [font=\normalsize] at (9.5,9.25) {x};
\node [font=\normalsize] at (5.25,13) {$\psi$};
\node [font=\normalsize] at (6.5,12.75) {V};
\node [font=\scriptsize] at (6.75,12.5) {0};
\draw [short] (6.25,11.25) .. controls (6.5,10.75) and (6.75,10.25) .. (7.25,10.25);
\draw [ dashed] (6,12.25) rectangle  (7.25,9.5);
\end{circuitikz}
}%
\end{figure}
\end{enumerate}
\item Let $\rho\brak{\vec{p},\vec{q},t}$ be the phase space density of an ensemble of a system. The Hamiltonian of the system is $H\brak{\vec{p},\vec{q}}$. If $\cbrak{A,B}$ denotes the poisson bracket A and B, the 
$$\frac{d\rho}{dt}=0$$ implies
\begin{enumerate}
    \item $\frac{d\rho}{dt}=0$
    \item $\frac{d\rho}{dt} \propto \cbrak{\rho,H}$
    \item $\frac{d\rho}{dt} \propto \cbrak{\rho,\frac{\vec{p}\cdot\vec{q}}{2}}$
    \item $\frac{d\rho}{dt} \propto \cbrak{\rho,\frac{\vec{q}\cdot\vec{q}}{2}}$
\end{enumerate}
\item Consider the following circuit:\\
\begin{figure}[!ht]
\centering
\resizebox{0.2\textwidth}{!}{%
\begin{circuitikz}
\tikzstyle{every node}=[font=\normalsize]
\draw (4.25,14.25) to[short] (4.75,14.25);
\draw (4.25,13.75) to[short] (4.75,13.75);
\draw (4.75,14.25) node[ieeestd nand port, anchor=in 1, scale=0.89](port){} (port.out) to[short] (6.75,14);
\draw [short] (4.25,14.25) -- (3,14.25);
\draw [short] (4.25,13.75) -- (4.25,14.25);
\draw [short] (6.75,14) -- (6.75,13);
\draw (4.25,11.5) to[short] (4.5,11.5);
\draw (4.25,11) to[short] (4.5,11);
\draw (4.5,11.5) node[ieeestd nand port, anchor=in 1, scale=0.89](port){} (port.out) to[short] (6.5,11.25);
\draw [short] (4.5,11) -- (3,11);
\draw [short] (4.25,11.5) -- (4.25,11);
\draw [short] (6.5,11.25) -- (6.75,11.25);
\draw [short] (6.75,11.25) -- (6.75,12.25);
\draw (6.75,12.75) to[short] (7,12.75);
\draw (6.75,12.25) to[short] (7,12.25);
\draw (7,12.75) node[ieeestd nand port, anchor=in 1, scale=0.89](port){} (port.out) to[short] (9,12.5);
\draw [short] (6.75,13) -- (6.75,12.75);
\draw [short] (9,12.5) -- (10,12.5);
\node [font=\normalsize] at (10,12.75) {Y};
\node [font=\normalsize] at (2.75,14.25) {P};
\node [font=\normalsize] at (3,10.75) {Q};
\end{circuitikz}
}%
\end{figure}
Suppose the input signal P is\\
\begin{figure}[!ht]
\centering
\resizebox{0.2\textwidth}{!}{%
\begin{circuitikz}
\tikzstyle{every node}=[font=\normalsize]
\draw [->, >=Stealth, dashed] (5,10.75) -- (5,12.75);
\draw [->, >=Stealth, dashed] (5,10.75) -- (12,10.75);
 (5,10.75) -- (5,10.75);
\draw [line width=0.5pt, short] (5,10.75) -- (5.75,10.75);
\draw [line width=0.5pt, short] (5.75,10.75) -- (5.75,11.75);
\draw [line width=0.5pt, short] (5.75,11.75) -- (6.5,11.75);
\draw [line width=0.5pt, short] (6.5,11.75) -- (6.5,10.75);
\draw [line width=0.5pt, short] (6.5,10.75) -- (7.25,10.75);
\draw [line width=0.5pt, short] (7.25,10.75) -- (7.25,11.75);
\draw [line width=0.5pt, short] (7.25,11.75) -- (8,11.75);
\draw [line width=0.5pt, short] (8,11.75) -- (8,10.75);
\draw [line width=0.5pt, short] (8,10.75) -- (11.25,10.75);
\node [font=\normalsize] at (12,10.5) {t};
\node [font=\normalsize] at (4.25,11.75) {+5V};
\node [font=\normalsize] at (4.5,10.75) {0V};
\draw [line width=0.5pt, short] (4.75,11.75) -- (5.25,11.75);
\draw [line width=0.5pt, short] (4.75,10.75) -- (5,10.75);
\end{circuitikz}
}%
\end{figure}
and the input signal Q is\\ 
\begin{figure}[!ht]
\centering
\resizebox{0.2\textwidth}{!}{%
\begin{circuitikz}
\tikzstyle{every node}=[font=\normalsize]
\draw [->, >=Stealth, dashed] (5,10.75) -- (5,12.75);
\draw [->, >=Stealth, dashed] (5,10.75) -- (10.75,10.75);
\draw [short] (5,10.75) -- (5,10.75);
\draw [line width=0.5pt, short] (5,10.75) -- (5.75,10.75);
\draw [line width=0.5pt, short] (6.5,10.75) -- (7.25,10.75);
\node [font=\normalsize] at (10.75,10.25) {t};
\node [font=\normalsize] at (4.25,11.75) {+5V};
\node [font=\normalsize] at (4.5,10.75) {0V};
\draw [line width=0.5pt, short] (4.75,11.75) -- (5.25,11.75);
\draw [line width=0.5pt, short] (4.75,10.75) -- (5,10.75);
\draw [line width=0.5pt, short] (8,10.75) -- (8.75,10.75);
\draw [line width=0.5pt, short] (8.75,10.75) -- (8.75,11.75);
\draw [line width=0.5pt, short] (8.75,11.75) -- (9.5,11.75);
\draw [line width=0.5pt, short] (9.5,11.75) -- (9.5,10.75);
\draw [line width=0.5pt, short] (9.5,10.75) -- (10.25,10.75);
\draw [line width=0.5pt, short] (5.75,10.75) -- (8.5,10.75);
\end{circuitikz}
}%
\end{figure}
Which one of the following output signals is correct?\\
\begin{enumerate}
\item \begin{figure}[!ht]
\centering
\resizebox{0.2\textwidth}{!}{%
\begin{circuitikz}
\tikzstyle{every node}=[font=\normalsize]
\draw [->, >=Stealth, dashed] (5,10.75) -- (5,12.75);
\draw [->, >=Stealth, dashed] (5,10.75) -- (10.75,10.75);
\draw [short] (5,10.75) -- (5,10.75);
\draw [line width=0.5pt, short] (5,10.75) -- (5.75,10.75);
\draw [line width=0.5pt, short] (5.75,10.75) -- (5.75,11.75);
\draw [line width=0.5pt, short] (5.75,11.75) -- (6.5,11.75);
\draw [line width=0.5pt, short] (6.5,11.75) -- (6.5,10.75);
\draw [line width=0.5pt, short] (6.5,10.75) -- (7.25,10.75);
\draw [line width=0.5pt, short] (7.25,10.75) -- (7.25,11.75);
\draw [line width=0.5pt, short] (7.25,11.75) -- (8,11.75);
\draw [line width=0.5pt, short] (8,11.75) -- (8,10.75);
\node [font=\normalsize] at (12,10.5) {t};
\node [font=\normalsize] at (4.25,11.75) {+5V};
\node [font=\normalsize] at (4.5,10.75) {0V};
\draw [line width=0.5pt, short] (4.75,11.75) -- (5.25,11.75);
\draw [line width=0.5pt, short] (4.75,10.75) -- (5,10.75);
\draw [line width=0.5pt, short] (8,10.75) -- (8.75,10.75);
\draw [line width=0.5pt, short] (8.75,10.75) -- (8.75,11.75);
\draw [line width=0.5pt, short] (8.75,11.75) -- (9.5,11.75);
\draw [line width=0.5pt, short] (9.5,11.75) -- (9.5,10.75);
\draw [line width=0.5pt, short] (9.5,10.75) -- (10.25,10.75);
\end{circuitikz}
}%
\end{figure}
\item \begin{figure}[!ht]
\centering
\resizebox{0.2\textwidth}{!}{%
\begin{circuitikz}
\tikzstyle{every node}=[font=\normalsize]
\draw [->, >=Stealth, dashed] (5,10.75) -- (5,12.75);
\draw [->, >=Stealth, dashed] (5,10.75) -- (12,10.75);
\draw [short] (5,10.75) -- (5,10.75);
\draw [line width=0.5pt, short] (5,10.75) -- (5.75,10.75);
\draw [line width=0.5pt, short] (5.75,10.75) -- (5.75,11.75);
\draw [line width=0.5pt, short] (5.75,11.75) -- (6.5,11.75);
\draw [line width=0.5pt, short] (6.5,11.75) -- (6.5,10.75);
\draw [line width=0.5pt, short] (6.5,10.75) -- (7.25,10.75);
\draw [line width=0.5pt, short] (7.25,10.75) -- (7.25,11.75);
\draw [line width=0.5pt, short] (7.25,11.75) -- (8,11.75);
\draw [line width=0.5pt, short] (8,11.75) -- (8,10.75);
\draw [line width=0.5pt, short] (8,10.75) -- (11.25,10.75);
\node [font=\normalsize] at (12,10.5) {t};
\node [font=\normalsize] at (4.25,11.75) {+5V};
\node [font=\normalsize] at (4.5,10.75) {0V};
\draw [line width=0.5pt, short] (4.75,11.75) -- (5.25,11.75);
\draw [line width=0.5pt, short] (4.75,10.75) -- (5,10.75);
\end{circuitikz}
}%
\end{figure}
\item \begin{figure}[!ht]
\centering
\resizebox{0.2\textwidth}{!}{%
\begin{circuitikz}
\tikzstyle{every node}=[font=\normalsize]
\draw [->, >=Stealth, dashed] (5,10.75) -- (5,12.75);
\draw [->, >=Stealth, dashed] (5,10.75) -- (10.75,10.75);
\draw [line width=0.5pt, short] (5,10.75) -- (5.75,10.75);
\draw [line width=0.5pt, short] (6.5,10.75) -- (7.25,10.75);
\node [font=\normalsize] at (10.75,10.25) {t};
\node [font=\normalsize] at (4.25,11.75) {+5V};
\node [font=\normalsize] at (4.5,10.75) {0V};
\draw [line width=0.5pt, short] (4.75,11.75) -- (5.25,11.75);
\draw [line width=0.5pt, short] (4.75,10.75) -- (5,10.75);
\draw [line width=0.5pt, short] (8,10.75) -- (8.75,10.75);
\draw [line width=0.5pt, short] (9.5,10.75) -- (10.25,10.75);
\draw [line width=0.5pt, short] (5.75,10.75) -- (8.5,10.75);
\draw [line width=0.5pt, short] (8.5,10.75) -- (9.75,10.75);
\end{circuitikz}
}%
\end{figure}
\item \begin{figure}[!ht]
\centering
\resizebox{0.2\textwidth}{!}{%
\begin{circuitikz}
\tikzstyle{every node}=[font=\normalsize]
\draw [->, >=Stealth, dashed] (5,10.75) -- (5,12.75);
\draw [->, >=Stealth, dashed] (5,10.75) -- (10.75,10.75);
\draw [short] (5,10.75) -- (5,10.75);
\draw [line width=0.5pt, short] (5,10.75) -- (5.75,10.75);
\node [font=\normalsize] at (10.75,10.25) {t};
\node [font=\normalsize] at (4.25,11.75) {+5V};
\node [font=\normalsize] at (4.5,10.75) {0V};
\draw [line width=0.5pt, short] (4.75,11.75) -- (5.25,11.75);
\draw [line width=0.5pt, short] (4.75,10.75) -- (5,10.75);
\draw [line width=0.5pt, short] (9.5,10.75) -- (10.25,10.75);
\draw [line width=0.5pt, short] (9.25,11.25) -- (9.25,11);
\draw [line width=0.5pt, short] (5.75,10.75) .. controls (5.75,11.5) and (5.75,11.25) .. (5.75,11.75);
\draw [line width=0.5pt, short] (5.75,11.75) -- (9.25,11.75);
\draw [line width=0.5pt, short] (9.25,11.75) -- (9.25,10.75);
\draw [line width=0.5pt, short] (9.25,10.75) -- (9.75,10.75);
\end{circuitikz}
}%
\end{figure}
\end{enumerate}
\item An inertial observer sees two spacecrafts S and T flying away from each other along x-axis with indivisual speed 0.5c, where c is the speed of light. The speed of T with respect to S is
\begin{enumerate}
    \item $\frac{4}{5}c$
    \item $\frac{4}{3}c$
    \item c
    \item $\frac{2}{3}c$
\end{enumerate}
\item Let P,Q and R be three different nuclei. Which one of the following nuclear processes is possible?
\begin{enumerate}
    \item $v_e+^{A}_{z}P \rightarrow ^{A}_{z+1}Q+e^-$
     \item $v_e+^{A}_{z}P \rightarrow ^{A}_{z-1}R+e^+$
      \item $v_e+^{A}_{z}P \rightarrow ^{A}_{z}P+e^++e^-$
    \item $v_e+^{A}_{z}P \rightarrow ^{A}_{z}P+\gamma$   
\end{enumerate}





\iffalse
\chapter{2021}
\author{AI24BTECH11009}
\section{st}
\fi

\item Let A be a $3 \times 3$ real matrix such that $I_3 + A$ is invertible and let $B = \brak{I_3 + A}^{-1}\brak{I_3 - A}$, where $I_3$ denotes the $3 \times 3$ identity matrix. Then which one of the following statements is true?
\begin{enumerate}
    \item If $B$ is orthogonal, then $A$ is invertible
    \item If $B$ is orthogonal, then all the eigenvalues of $A$ are real
    \item If $B$ is skew-symmetric, then $A$ is orthogonal
    \item If $B$ is skew-symmetric, then the determinant of $A$ equals -1 \\
\end{enumerate}
\item Let $X$ be a random variable having Poisson distribution such that $E\brak{X^2}$ = 110. Then which one of the following statements is NOT true?
\begin{enumerate}
    \item $E\brak{X^n} = 10 E\sbrak{\brak{X + 1}^{n-1}}$, for all $n = 1, 2, 3, \cdots$
    \item $P\brak{X\text{ is even}} = \frac{1}{4}\brak{1 + e^{-20}}$
    \item $P\brak{X = k} < P\brak{X = k + 1}$, for $k = 0, 1, \cdots, 8$
    \item $P\brak{X = k} > P\brak{X = k + 1}$, for $k = 10, 11, \cdots$ \\
\end{enumerate}
\item Let $X$ be a random variable having uniform distribution on $\sbrak{-\frac{\pi}{2}, \frac{\pi}{2}}$. Then which one of the following statements is NOT true?
\begin{enumerate}
    \item $Y = \cot\brak{X}$ follows standard Cauchy distribution
    \item $Y = \tan\brak{X}$ follows standard Cauchy distribution
    \item $Y = -\log_e\brak{\frac{1}{2} + \frac{X}{\pi}}$ has moment generating function $M\brak{t} = \frac{1}{1 - t},\ t < 1$
    \item $Y = -2\log_e\brak{\frac{1}{2} + \frac{X}{\pi}}$ follows central chi-square distribution with one degree of freedom \\
\end{enumerate}
\item Let $\Omega = \{1, 2, 3, \cdots \}$ represent the collection of all possible outcomes of a random experiment with probabilities $P\brak{\{n\}} = \alpha n$ for $n \in \Omega$. Then which one of the following statements is NOT true?
\begin{enumerate}
    \item $\lim\limits_{n\rightarrow\infty}\alpha_n = 0$
    \item $\sum_{n=1}^{\infty}\sqrt{\alpha_n}$ converges
    \item For any positive integer $k$, there exist $k$ disjoint events $A_1, A_2, \cdots , A_k$ such that $P\brak{\bigcup_{i=1}^{k}A_i} < 0.001$
    \item There exists a sequence $\{A_i\}_{i \geq 1}$ of strictly increasing events such that $P\brak{\bigcup_{i=1}^{k}A_i} < 0.001$ \\
\end{enumerate}
\item Let $\brak{X, Y}$ have the joint probability density function
\begin{align*}
    f_{X, Y}\brak{x, y} = \begin{cases}
        \frac{4}{\brak{x+y}^3}, & x > 1, y > 1, \\
        0, & \text{otherwise}.
    \end{cases}
\end{align*}
Then which one of the following statements is NOT true?
\begin{enumerate}
    \item The probability density function of $X + Y$ is
    \begin{align*}
        f_{X + Y}\brak{z} = \frac{4}{z^3}\brak{z - 2}, & z > 2, \\
        0, & \text{otherwise}.
    \end{align*}
    \item $P\brak{X + Y > 4} = \frac{3}{4}$
    \item $E\brak{X + Y} = 4\log_{e}2$
    \item $E\brak{Y | X = 2} = 4$ \\
\end{enumerate}
\item Let $X_1$, $X_2$ and $X_3$ be three uncorrelated random variables with common variance $\sigma^2 < \infty$. Let $Y_1 = 2X_1 + X_2 + X_3$, $Y_2 = X_1 + 2X_2 + X_3$ and $Y_3 = X_1 + X_2 + 2X_3$. Then which of the following statements is/are true? \\
P : The sum of eigenvalues of the variance covariance matrix of $\brak{Y_1, Y_2, Y_3}$ is $18\sigma^2$. \\
Q : The correlation coefficient between $Y_1$ and $Y_2$ equals that between
$Y_2$ and $Y_3$.
\begin{enumerate}
    \item P only
    \item Q only
    \item Both P and Q
    \item Neither P nor Q \\ 
\end{enumerate}
\item Let $\{X_n\}_{n \geq 0}$ be a time-homogeneous discrete time Markov chain with either finite or countable state space $S$. Then which one of the following statements is true?
\begin{enumerate}
    \item There is at least one recurrent state
    \item If there is an absorbing state, then there exists at least one stationary distribution
    \item If all the states are positive recurrent, then there exists a unique stationary distribution
    \item If $\{X_n\}_{n \geq 0}$ is irreducible, $S = \{1, 2\}$ and $\sbrak{\pi_1}, \sbrak{\pi_2}$ is a stationary distribution, then $\lim\limits_{n\rightarrow\infty}P\brak{X_n = i|X_0 =i} = \pi_i$ for $i = 1, 2$ \\
\end{enumerate}
\item Let customers arrive at a departmental store according to a Poisson process with rate 10. Further, suppose that each arriving customer is either a male or a female with probability $\frac{1}{2}$ each, independent of all other arrivals. Let $N\brak{t}$ denote the total number of customers who have arrived by time $t$. Then which one of the following statements is NOT true?
\begin{enumerate}
    \item If $S_2$ denotes the time of arrival of the second female customer, then $P\brak{S_2 \leq 1} = 25\int_{0}^{1}se^{-5s} ds$
    \item If $M\brak{t}$ denotes the number of male customers who have arrived by time $t$, then $P\brak{M\brak{\frac{1}{3}} = 0|M\brak{1} = 1} = \frac{1}{3}$
    \item $E\sbrak{\brak{N\brak{t}}^2} = 100t^2 + 10t$
    \item $E\sbrak{N\brak{t}N\brak{2t}} = 200t^2 + 10t$ \\
\end{enumerate}
\item Let $X_{\brak{1}} < X_{\brak{2}} < X_{\brak{3}} < X_{\brak{4}} < X_{\brak{5}}$ be the order statistics corresponding to a random sample of size 5 from a uniform distribution on $\sbrak{0, \theta}$, where $\theta \in \brak{0, \infty}$. Then which of the following statements is/are true?\\
P : $3X_{\brak{2}}$ is an unbiased estimator of $\theta$.\\
Q : The variance of $E\sbrak{2X_{\brak{3}}|X_{\brak{5}}}$ is less than or equal to the variance of $2X_{\brak{3}}$.
\begin{enumerate}
    \item P only
    \item Q only
    \item Both P and Q
    \item Neither P nor Q \\ 
\end{enumerate}
\item Let $X_1$, $X_2$, $\cdots$ , $X_n$ be a random sample of size $n\brak{\geq 2}$ from a distribution having the probability density function
\begin{align*}
    f\brak{x;\theta} = \begin{cases}
        \frac{1}{\theta}e^{-\frac{x}{\theta}}, & x > 0, \\
        0, & \text{otherwise},
    \end{cases}
\end{align*}
where $\theta\in\brak{0, \infty}$. Let $X_{\brak{1}} = \min\{X_1$, $X_2$, $\cdots$ , $X_n\}$ and $T = \sum_{i=1}^nX_i$. Then $E\brak{X_{\brak{1}}|T}$ equals
\begin{enumerate}
    \item $\frac{T}{n^2}$
    \item $\frac{T}{n}$
    \item $\frac{\brak{n+1}T}{2n}$
    \item $\frac{\brak{n+1}^2T}{4n^2}$ \\
\end{enumerate}
\item Let $X_1, X_2, \cdots, X_n$ be a random sample of size $n\brak{\geq 2}$ from a uniform distribution on $\sbrak{-\theta, \theta}$, where $\theta \in \brak{0, \infty}$. Let $X_{\brak{1}} = \min\{X_1, X_2, \cdots, X_n\}$ and $X_{\brak{n}} = \max\{X_1, X_2, \cdots, X_n\} $. Then which of the following statements is/are true ?
P : $\brak{X_{\brak{1}}, X_{\brak{n}}}$ is a complete statistic. \\
Q : $X_{\brak{n}} - X_{\brak{1}}$ is an ancillary statistic.
\begin{enumerate}
    \item P only
    \item Q only
    \item Both P and Q
    \item Neither P nor Q \\ 
\end{enumerate}
\item Let $\{X_n\}_{n \geq 1}$ be a sequence of independent and identically distributed random variables having common distribution function $ F\brak{\cdot}$. Let $a < b$ be two real numbers such that $F\brak{x} = 0$ for all $x \leq a$, $0 < F\brak{x} < 1$ for all $a < x < b$, and $F\brak{x} = 1$ for all $x \geq b$. Let $S_n\brak{x}$ be the empirical distribution function at $x$ based on $X_1, X_2, \cdots, X_n$, $n \geq 1$. Then which one of the following statements is NOT true ?
\begin{enumerate}
    \item $P\sbrak{\lim_{n \rightarrow \infty} \sup_{-\infty < x < \infty} \abs{S_n\brak{x} - F\brak{x}} = 0} = 1$
    \item For fixed $x \in \brak{a, b}$ and $t \in (-\infty, \infty)$,
    $\lim_{n \to \infty} P\sbrak{\frac{\sqrt{n}\abs{S_n\brak{x} - F\brak{x}}}{\sqrt{S_n\brak{x}(1 - S_n\brak{x})}} \leq t} = P\brak{Z \leq t)},$
    where $Z$ is the standard normal random variable
    \item The covariance between $S_n\brak{x}$ and $S_n\brak{y}$ equals $\frac{1}{n} F\brak{x}(1 - F\brak{y})$ for all $n \geq 2$ and for fixed $-\infty < x, y < \infty$
    \item If $Y_n = \sup_{-\infty < x < \infty} \brak{S_n\brak{x} - F\brak{x}}^2$, then $\{4n\ Y_n\}_{n \geq 1}$ converges in distribution to a central chi-square random variable with 2 degrees of freedom \\
\end{enumerate}
\item Let the joint distribution of random variables $X_1, X_2, X_3$ and $X_4$ be $N_4\brak{\underline{\mu}, \Sigma}$, where
\begin{align*}
\underline{\mu} = \myvec{1 \\ 0 \\ 0 \\ 1}\ \text{and}\ \Sigma = \sbrak{\begin{matrix}
    1 & 0.2 & 0 & 0 \\ 0.2 & 2 & 0 & 0 \\ 0 & 0 & 2 & 0.2 \\ 0 & 0 & 0.2 & 1
\end{matrix}}.
\end{align*}
Then which one of the following statements is true ?
\begin{enumerate}
    \item $\frac{5}{17} \sbrak{\brak{X_1 + X_2}^2 + \brak{X_3 + X_4 - 1}^2}$ follows a central chi-square distribution with 2 degrees of freedom
    \item $\frac{1}{3} \sbrak{\brak{X_1 + X_2}^2 + \brak{X_3 + X_4 - 1}^2}$ follows a central chi-square distribution with 2 degrees of freedom
    \item $E \sbrak{\sqrt{\abs{\frac{X_1 + X_2 - 1}{X_3 + X_4 - 1}}}}$ is NOT finite
    \item $E \sbrak{\abs{\frac{X_1 + X_2 + X_3 + X_4 - 2}{X_1 + X_2 - X_3 - X_4}}}$ is NOT finite \\
\end{enumerate}

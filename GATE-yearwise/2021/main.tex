\iffalse
\title{Assignment}
\author{EE24BTECH11038}
\section{ce}
\chapter{2021}
\fi
\item Contractor X is developing his bidding strategy against Contractor Y.The  ratio of Y`s bid price to X`s cost for the 30 previous bids in which contractor X has competed against Contractor Y in the given table 

\begin{tabular}{|c|c|}
\hline
\textbf{Ratio of Y`s bid price to x`s cost} & \textbf{Number of bids} \\
\hline
1.02 & 6\\
\hline
1.04 & 12\\
\hline
1.06 & 3\\
\hline 
1.10 & 6\\
\hline 
1.12 & 3\\
\hline
\end{tabular}

 Based on bidding behaviour of the contractor Y,the probability of winning against Contractor Y at a mark up of 8\% for the next project is 
 \begin{enumerate}
     \item 0\%
     \item more than 0\% and less than 50\%
     \item more than 0\% and less than 100\%
     \item 100\%
 \end{enumerate}
\bigskip
 \item Based on drained triaxial shear test on sands and clays the reprsentative variations of volumetric strain $\frac{\triangle v}{v}$ with the shear strain $\brak{\gamma}$ is shown in the figure. 
 \begin{figure}[!ht]
\centering
\resizebox{0.25\textwidth}{!}{%
\begin{circuitikz}
\tikzstyle{every node}=[font=\small]
\draw [<->, >=Stealth] (6,14.25) -- (6,10.75);
\draw [->, >=Stealth] (6,12.5) -- (9.5,12.5);
\draw [short] (6,12.5) .. controls (6.5,10.75) and (8,10.75) .. (9.25,11);
\draw [short] (6,12.5) .. controls (6.75,11.75) and (6.75,12) .. (7.25,12.5);
\draw [short] (7.25,12.5) .. controls (8,14) and (8.5,13.75) .. (9.5,14);
\node [font=\small] at (9,14.25) {Curve P};
\node [font=\small] at (9,10.75) {Curve Q};
\node [font=\small] at (8.5,12.25) {sheer strain};
\node [font=\small] at (6,14.5) {expansion};
\node [font=\small] at (6,10.5) {compression};
\node [font=\small, rotate around={-269:(0,0)}] at (5.5,12.75) {volume strain};
\end{circuitikz}
}%
\end{figure}\\
choose the CORRECT option regarding the representative behaviour exhibited by curve P and curve Q.
\begin{enumerate}
    \item Curve P represents dense sand and overconsolidated clay,while curve Q represents  loose sand and normally consolidated clay
    \item Curve P represents dense sand and normally consolidated  clay,while curve Q represents  loose sand and  overconsolidated clay
    \item Curve P represents loose sand and overconsolidated clay,while curve Q represents  dense sand and normally consolidated cay
    \item Curve P represents loose sandand overconsolidated clay,while curve Q represents  dense sand and  consolidated clay
\end{enumerate}
\bigskip
\item A fluid flowing steadily in a cirucar pipe of radius R has a velocity that is everywhere parllel to the axis of the pipe The velocity distribution along the radial direction is $V_r=U\brak{1-\frac{r^3}{R^2}}$ where r is the radial distance as measured fromthe pipe axis and U is the maximum velocity at r=0.the average velocity in the pipe is 
\begin{enumerate}
    \item $\frac{U}{2}$
    \item $\frac{U}{3}$
    \item $\frac{U}{4}$
    \item $\frac{5U}{6}$
\end{enumerate}
\bigskip
\item A water sample is analyzed for coliform organisms by the multiple-tube fermentation method. The results of confirmed test are as follow: 
\begin{tabular}{|c|c|c|}
\hline
\textbf{sample size} & \textbf{Number of positive results out of 5 test tubes}&\textbf{number of negative results of five test tubes}\\
\hline
0.01 & 5 & 0\\
\hline
0.001 & 3 & 2\\
\hline 
0.0001 & 1 & 4\\
\hline 
\end{tabular}
the most proable number(MPN)of coliform organisms for the above results is to obtained using the following MPN index.
\begin{tabular}{|c|c|}
\hline
\multicolumn{2}{|c|}{\textbf{MPN index for various combinations of positive results when five tubes are used per dilution of 10.0ml, 1ml, 0.1 ml}} \\
\hline
\textbf{Combination of positive tubes} & \textbf{MPN index per 100ml} \\
\hline
0-2-4 & 11 \\
\hline
1-3-5 & 19 \\
\hline
4-2-0 & 22 \\
\hline 
5-3-1 & 110 \\
\hline
\end{tabular}

\begin{enumerate}
    \item 1100000
    \item 110000
    \item 1100
    \item 110
\end{enumerate}
\bigskip
\item Ammonia nitrogen is present in a given wastewater sample as the ammonium ion $\brak{NH^+_4}$ and ammonia $\brak{NH_3}$. If pH is the only deciding factor for the proportion of these two constituents, which of the following is a correct statement?
\begin{enumerate}
    \item At pH above 9.25, only $\brak{NH^+_4}$  will be present
    \item At pH below 9.25, $\brak{NH_3}$ will be predominant
    \item At pH 7.0, $\brak{NH^+_4}$ and $\brak{NH_3}$  will be found in equal measures.
    \item At pH 7.0, $\brak{NH^+_4}$  will be predominant.

\end{enumerate}
\bigskip
\item On a road, the speed-density relationship of a traffic stream is given by u = 70 - 0.7k At the capacity condition, the average time headway will be
\begin{enumerate}
    \item 0.5 s
    \item 1.0 s
    \item 1.6 s
    \item 2.1 s
\end{enumerate}
\bigskip
\item the values of abscissa(x) and ordinate (y) of a curve are as follow 

\begin{tabular}{|c|c|}
\hline
\textbf{X} & \textbf{Y} \\
\hline
2 & 5\\
\hline
2.5 & 7.25\\
\hline
3 & 10\\
\hline 
3.5 & 13.25\\
\hline 
4 & 17\\
\hline
\end{tabular}

By simpson`s $\frac{1}{3}$rd rule the area under the curve is 
\bigskip
\item Refer the truss as shown in the figure 
\begin{figure}[!ht]
\centering
\resizebox{0.5\textwidth}{!}{%
\begin{circuitikz}
\tikzstyle{every node}=[font=\large]
\draw [short] (11,14.5) -- (17.25,14.5);
\draw [short] (17.25,14.5) -- (22,9.5);
\draw [short] (22,9.5) -- (23.25,8.75);
\draw [short] (22,9.5) -- (21,8.75);
\draw [short] (21,8.75) -- (23.25,8.75);
\draw  (21.5,8.5) circle (0.25cm);
\draw  (22.25,8.5) circle (0.25cm);
\draw  (23,8.5) circle (0.25cm);
\draw [line width=1.3pt, short] (21.25,8.25) -- (23.5,8.25);
\draw [line width=1.3pt, short] (21.25,8.25) -- (21,8.25);
\draw [line width=0.3pt, short] (21,8.25) -- (21.25,8);
\draw [line width=0.3pt, short] (21.25,8.25) -- (21.5,8);
\draw [line width=0.3pt, short] (21.5,8.25) -- (21.75,8);
\draw [line width=0.3pt, short] (21.75,8.25) -- (22,8);
\draw [line width=0.3pt, short] (22,8.25) -- (22.25,8);
\draw [line width=0.3pt, short] (22.25,8.25) -- (22.5,8);
\draw [line width=0.3pt, short] (22.5,8.25) -- (22.75,8);
\draw [line width=0.3pt, short] (22.75,8.25) -- (23,8);
\draw [line width=0.3pt, short] (23,8.25) -- (23.25,8);
\draw [line width=0.3pt, short] (23.25,8.25) -- (23.5,8);
\draw [line width=0.3pt, short] (22,9.5) -- (5.75,9.5);
\draw [line width=0.3pt, short] (5.75,9.5) -- (11,14.5);
\draw [line width=0.3pt, short] (11,14.5) -- (14,9.5);
\draw [line width=0.3pt, short] (17.25,14.5) -- (14,9.5);
\draw [line width=0.3pt, short] (5.75,9.5) -- (7,8.5);
\draw [line width=0.3pt, short] (5.75,9.5) -- (4.75,8.5);
\draw [line width=1.3pt, short] (4.75,8.5) -- (7,8.5);
\draw [line width=0.2pt, short] (4.75,8.5) -- (5,8.25);
\draw [line width=0.2pt, short] (5,8.5) -- (5.25,8.25);
\draw [line width=0.2pt, short] (5.5,8.5) -- (5.75,8.25);
\draw [line width=0.2pt, short] (5.5,8.5) -- (7.5,8.5);
\draw [line width=0.2pt, short] (5.25,8.5) -- (5.5,8.25);
\draw [line width=0.2pt, short] (6,8.5) -- (6.25,8.25);
\draw [line width=0.2pt, short] (5.75,8.5) -- (6,8.25);
\draw [line width=0.2pt, short] (6.5,8.5) -- (6.75,8.25);
\draw [line width=0.2pt, short] (6.5,8.5) -- (6.25,8.5);
\draw [line width=0.2pt, short] (6.25,8.5) -- (6.5,8.25);
\draw [line width=1.5pt, ->, >=Stealth] (14,9.5) -- (14,7);
\node [font=\large] at (14.5,7.25) {\textbf{F}};
\node [font=\large] at (11,15) {$P$};
\node [font=\large] at (5.5,9.75) {$Q$};
\node [font=\large] at (17,14.75) {$R$};
\node [font=\large] at (22.25,9.75) {$S$};
\node [font=\large] at (14.25,9) {$T$};
\draw [line width=0.5pt, <->, >=Stealth] (5.5,7.5) -- (13.5,7.5);
\draw [line width=0.5pt, <->, >=Stealth] (14.25,7.75) -- (22.25,7.5);
\node [font=\large] at (9,7) {$L$};
\node [font=\large] at (18.5,7) {$L$};
\end{circuitikz}
}%
\end{figure}
If load ,$\mathbf{F}=10\sqrt{3}KN$ moment of inertia $\mathbf{I}=8.33 \times 10^6 mm^4$area of crossection ,A=$10^4mm^2$ and length L=2m for all the members of the truss the compressive stress carried by member Q-R is 
\bigskip
\item  A prismatic cantilever prestressed concrete beam of span length, L = 1.5m has one straight tendon placed in the cross-section as shown in the following figure (not to scale). The total prestressing force of 50 kN in the tendon is applied at $d_p$ = 50mm from the top in the cross-section of width,  b = 200mm and depth, d = 300mm
\begin{figure}[!ht]
\centering
\resizebox{0.5\textwidth}{!}{%
\begin{circuitikz}
\tikzstyle{every node}=[font=\large]

% Vertical line
\draw [line width=1.5pt, short] (0.75,17.5) -- (0.75,15.5);

% Zig-zag pattern
\draw [short] (0.75,17.5) -- (0.5,17.25);
\draw [short] (0.75,17.25) -- (0.5,17);
\draw [short] (0.75,17) -- (0.5,16.75);
\draw [short] (0.75,16.75) -- (0.5,16.5);
\draw [short] (0.75,16.5) -- (0.5,16.25);
\draw [short] (0.75,16.25) -- (0.5,16);
\draw [short] (0.75,16) -- (0.5,15.75);

% Left block
\draw [fill={rgb,255:red,36; green,31; blue,49}] (0.75,16.75) rectangle (1,16.5);
\draw [fill={rgb,255:red,119; green,118; blue,123}] (0.75,16.75) rectangle (8.25,16.25);

% Load arrow
\draw [->, >=Stealth] (8,18.25) -- (8,16.75);
\node [font=\large] at (8.25,18.25) {\textbf{P}};

% Length marker
\draw [<->, >=Stealth] (0.75,15.25) -- (8.25,15.25);
\node [font=\large] at (3.25,15.5) {\textbf{L}};

% Right block
\draw [fill={rgb,255:red,119; green,118; blue,123}] (10.25,18) rectangle (12.5,14.25);

% Circle with white fill
\fill [white] (11.25,17.5) circle (0.25cm);  % Fills the circle with white
\draw (11.25,17.5) circle (0.25cm);          % Draws the circle outline

% Dimension markers
\draw [<->, >=Stealth] (9.75,18) -- (9.75,14.5);
\draw [<->, >=Stealth] (10.5,14) -- (12.5,14);
\node [font=\large] at (9.5,16.75) {\textbf{d}};
\node [font=\large] at (11,13.75) {\textbf{b}};

% Prestressing tendon arrow
\draw [->, >=Stealth] (11.25,19) -- (11.25,17.5);
\node [font=\large] at (11.25,19.25) {$\text{Prestressing Tendon}$};

\end{circuitikz}
}%
\end{figure}
If the concentrated load,P = 5KN the resultant stress (in MPa, in integer) experienced at point `Q' will be 
\bigskip
\item A column is subjected to a total load P of 60 kN supported through a bracket connection, as shown in the figure (not to scale).
\begin{figure}[!ht]
\centering
\resizebox{0.5\textwidth}{!}{%
\begin{circuitikz}
\tikzstyle{every node}=[font=\normalsize]
\draw [line width=1.1pt, short] (1.25,14) -- (1.25,6.5);
\draw [line width=1.1pt, short] (1.25,13) -- (8.5,13);
\draw [line width=1.1pt, short] (8.5,13) -- (8.5,11);
\draw [line width=1.1pt, short] (1.25,8.75) -- (6.75,8.75);
\draw [line width=1.1pt, short] (6.75,8.75) -- (8.5,11);
\draw [line width=0.7pt, short] (0.75,13.75) -- (3.25,13.75);
\draw [line width=0.7pt, short] (3.25,13.75) -- (3.5,13.25);
\draw [line width=0.7pt, short] (3.5,13.25) -- (3.75,14.25);
\draw [line width=0.7pt, short] (3.75,14.25) -- (4,13.75);
\draw [line width=0.7pt, short] (4,13.75) -- (6.5,13.75);
\draw [line width=0.7pt, short] (5.75,13.75) -- (5.75,7.5);
\draw [line width=0.7pt, short] (1.25,7.5) -- (3.25,7.5);
\draw [line width=0.7pt, short] (4,7.5) -- (6.25,7.5);
\draw [ line width=0.7pt ] (2.25,12.25) circle (0.25cm);
\draw [ line width=0.7pt ] (2.25,11) circle (0.25cm);
\draw [ line width=0.7pt ] (2.25,9.75) circle (0.25cm);
\draw [ line width=0.7pt ] (5.25,12.25) circle (0.25cm);
\draw [ line width=0.7pt ] (5.25,11) circle (0.25cm);
\draw [ line width=0.7pt ] (5.25,9.75) circle (0.25cm);
\draw [line width=0.7pt, dashed] (4,14.5) -- (4,6.75);
\draw [line width=0.7pt, ->, >=Stealth] (7.5,14) -- (7.5,13);
\draw [line width=0.7pt, <->, >=Stealth] (4,15) -- (7.5,15);
\draw [line width=0.7pt, <->, >=Stealth] (4,14.5) -- (5.5,14.5);
\draw [line width=0.7pt, <->, >=Stealth] (2.25,14.5) -- (3.75,14.5);
\draw [short] (3.25,7.5) -- (3.5,7);
\draw [short] (3.5,7) -- (3.75,8);
\draw [short] (3.75,8) -- (4,7.5);
\draw [<->, >=Stealth] (0.75,12.25) -- (0.75,11.25);
\draw [<->, >=Stealth] (0.75,11) -- (0.75,9.75);
\node [font=\LARGE] at (9,13.75) {P=60KN};
\node [font=\small] at (5.5,15.5) {$100mm$};
\node [font=\small] at (4.75,14.75) {$40mm$};
\node [font=\small] at (3,14.75) {$40mm$};
\node [font=\small] at (0.25,11.75) {$30mm$};
\node [font=\small] at (0,10.25) {$30mm$};
\node [font=\normalsize] at (4.75,11) {$R$};
\end{circuitikz}
}%
\end{figure}
The resultant force in the bolt R
\bigskip
\item Employ stiffness matrix approach for the simply supported as shown in the figure to calculate unknown displacements/rotations. Take length ,L=8m modulus of elasticity ,E=$3\times 10^4N/mm^2$ moment of inertia I=225$\times 10^6 mm^4$
\begin{figure}[!ht]
\centering
\resizebox{0.5\textwidth}{!}{%
\begin{circuitikz}
\tikzstyle{every node}=[font=\normalsize]
\draw  (-0.5,13.75) rectangle (8.25,12.5);
\draw  (8.25,13.5) rectangle (13,13);
\draw [short] (-0.5,12.5) -- (0.25,11.5);
\draw [short] (-0.5,12.5) -- (-1.25,11.5);
\draw [short] (-1.25,11.5) -- (0.25,11.5);
\draw [short] (-1.25,11.5) -- (-1,11.25);
\draw [short] (-1,11.5) -- (-0.75,11.25);
\draw [short] (-0.75,11.5) -- (-0.5,11.25);
\draw [short] (-0.5,11.5) -- (-0.25,11.25);
\draw [short] (-0.25,11.5) -- (0,11.25);
\draw [short] (0,11.5) -- (0.25,11.25);
\draw [short] (13,13) -- (13.75,12.5);
\draw [short] (13,13) -- (12.25,12.5);
\draw [short] (12.25,12.5) -- (13.75,12.5);
\draw  (12.25,12.25) circle (0.25cm);
\draw  (13,12.25) circle (0.25cm);
\draw  (13.75,12.25) circle (0.25cm);
\draw [short] (11.75,12) -- (14.25,12);
\draw [short] (11.75,12) -- (12,11.75);
\draw [short] (12.25,12) -- (12.5,11.75);
\draw [short] (12.75,12) -- (13,11.75);
\draw [short] (13.25,12) -- (13.5,11.75);
\draw [short] (13.75,12) -- (14,11.75);
\draw [->, >=Stealth] (8,15.75) -- (8,14);
\node [font=\normalsize] at (8.25,15.25) {\textbf{P}};
\node [font=\normalsize] at (3,12.25) {$E,2l$};
\node [font=\normalsize] at (10.5,12.75) {$E,l$};
\draw [<->, >=Stealth] (-0.5,10.5) -- (8.5,10.5);
\draw [<->, >=Stealth] (8.25,11) -- (13,11);
\node [font=\normalsize] at (4.5,9.75) {$L/2$};
\node [font=\normalsize] at (10,10.5) {$L/2$};
\node [font=\normalsize] at (-0.5,14.5) {\textbf{A}};
\node [font=\normalsize] at (12.75,14.25) {\textbf{C}};
\end{circuitikz}
}%

\end{figure}
The mid-span deflection of the beam in mm under P=100KN in downward direction will be
\bigskip


\item A square plate O-P-Q-R of a linear elastic material with sides 1.0m is loaded in a state of plane stress under a given stress condition the plate deforms ti a new configuration O-P'-Q'-R' as shown in the figure Under the given deformation the edges of  aplate remain straight.
\begin{figure}[!ht]
\centering
\resizebox{0.25\textwidth}{!}{%
\begin{circuitikz}
\tikzstyle{every node}=[font=\normalsize]
\draw [line width=1.1pt, ->, >=Stealth] (-4.75,9.25) -- (-4.75,15.75);
\draw [line width=1.1pt, ->, >=Stealth] (-4.75,9.25) -- (3.5,9.25);
\draw [ line width=0.6pt ] (-4.75,13) rectangle (-0.25,9.25);
\draw [line width=0.6pt, dashed] (-4.75,13.75) -- (-2.25,13.75);
\draw [line width=0.6pt, dashed] (-4.75,9.25) -- (1,10);
\draw [line width=0.6pt, dashed] (-2.25,13.75) -- (1,10);
\node [font=\normalsize] at (1.75,9.75) {10mm};
\draw [line width=0.6pt, <->, >=Stealth] (1,10) -- (1,9.25);
\draw [line width=0.6pt, <->, >=Stealth] (-0.25,9) -- (1,9);
\draw [line width=0.6pt, <->, >=Stealth] (-5,13.75) -- (-5,13);
\draw [line width=0.6pt, <->, >=Stealth] (-2.25,14) -- (-0.25,14);
\node [font=\normalsize] at (0.25,8.75) {20mm};
\node [font=\normalsize] at (-5.5,13.5) {10mm};
\node [font=\normalsize] at (-1.25,14.5) {10mm};
\node [font=\normalsize] at (-4.5,14.25) {R`};
\node [font=\normalsize] at (-2.5,14.25) {Q`};
\node [font=\normalsize] at (-5,12.75) {R};
\node [font=\normalsize] at (0,13) {Q};
\node [font=\normalsize] at (1.25,10.25) {P`};
\node [font=\normalsize] at (0,9.5) {P};
\node [font=\normalsize] at (-5,9) {O};
\node [font=\normalsize] at (3.25,8.75) {X};
\node [font=\normalsize] at (-5.25,15.25) {Y};
\end{circuitikz}
}%


\end{figure}\\
\bigskip
\item A small project has 12 activities -N,P,Q,R,S,T,U,V,W,X,Y and Z the relationship among these activities and the duration of these activities are given in the table. The total float of activity "V" in integer

\begin{tabular}{|c|c|c|}
\hline
\textbf{Activity} & \textbf{Duration} & \textbf{Depends upon}\\
\hline 
N & 2 & - \\
\hline
P & 5 & N \\
\hline
Q&3&5\\
\hline
R&4&p \\
\hline
S&5&Q\\
\hline
T&8&R\\
\hline
U&7&R,S\\
\hline
V&2&U\\
\hline
W&3&U\\
\hline
X&5&T,V\\
\hline
Y&1&W\\
\hline
Z&3&X,Y\\
\hline
\end{tabular}

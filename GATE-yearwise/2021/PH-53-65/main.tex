\iffalse
\title{GATE Questions 18}
\author{EE24BTECH11012 - Bhavanisankar G S}
\section{ph}
\chapter{2021}
\fi
%\begin{enumerate}
	\item P and Q are two Hermititan matrices and there exists a matrix R, which diagonalizes both of them, such that $RPR^{-1} = S_1$ and $RQR^{-1} = S_2$, where $S_1$ and $S_2$ are diagonal matrices. The correct statement(s) is(are) :
		\begin{enumerate}
			\item All the elements of both matrices $S_1$ and $S_2$ are real
			\item The matrix PQ can have complex eigenvalues.
			\item The matrix QP can have complex eigenvalues.
			\item The matrices P and Q commute.
		\end{enumerate}
	\item A uniform block of mass $M$ slides on a smooth horizontal bar. Another mass $m$ is connected to it by an inextensible string of length $l$ of negligible mass, and is constrained to oscillate in the X-Y plane only. Neglect the sizes of the masses. The number of degrees of freedom of the system is two and the generalized coordinates are chosen as $x$ and $\theta$ as shown in the figure. \\
		If $p_x$ and $p_{\theta}$ are the generalised momenta corresponding to $x$ and $\theta$, respectively, then the correct option(s) is(are)
		\begin{enumerate}
	\item $p_x = \brak{m+M} \overline{x} + mI\cos{\theta} \overline{\theta}$
	\item $p_{\theta} = mI^2 \overline{\theta} - mI \cos{\theta} \overline{x}$
	\item $p_x$ is conserved
	\item $p_{\theta}$ is conserved
		\end{enumerate}
	\item The Gell-Mann-Okuba mass formula defines the mass of baryons as 
		$$ M = M_0 + aY + b \sbrak{I(I+1) - \frac{1}{4} Y^2}, \text{where} M_0, a \text{and} b \text{are constants}$$
		If the mass of $\sigma$ hyperons is same as that of $\Lambda$ hyperons, then the correct option(s) is(are)
		\begin{enumerate}
				\begin{multicols}{2}
				\item $M \propto I(I+1)$
				\item $M \propto Y$
				\item $M$ does not depend on $I$
				\item $M$ does not depend on $Y$
				\end{multicols}
		\end{enumerate}
	\item The time derivative of a differentiable function $g(q_{i},t)$ is added to a Lagrangian $L(q_i, \overline{q_i},t)$ such that
		$$ L^{\prime} = L(q_i, \overline{q_i}, t) + \frac{d\brak{g(q_i,t)}}{dt} $$
		where $q_i, \overline{q_i}, t$ are the generalised coordinates, generalizes velocities and time respectively. Let $p_i$ be the generalized momentum and H the Hamiltonian associated with $L(q_i, \overline{q_i}, t)$. If $p_{i}^{\prime} \text{and} H^{\prime}$ are those associated with $L^{\prime}$, then the correct option(s) is(are)
		\begin{enumerate}
			\item Both $L$ and $L^{\prime}$ sarisfy Euler-Lagrange's equations of motion.
			\item $p_{i}^{\prime} = p_i + \frac{\partial}{\partial q_i} g(q_i,t) $
			\item If $p_i$ is conserved, then $p_{i}^{\prime}$ is necassarily conserved.
			\item $H^{\prime} = H + \frac{d}{dt} g(q_i, t) $
		\end{enumerate}
	\item A linear charged particle accelerator is driven by an alternating voltage source operating at 10 MHz. Assume that it is used to accelerate electrons. After a few drift-tubes, the electrons attain a velocity $2.9 \times 10^8 ms^{-1}$. The minimum length of each drift-tube, in m, to accelerate the electrons further is \underline{   }
	\item The Coulomb energy component in the binding energy of a nucleus is 18.432 MeV. If the radius of the uniform and spherical charge distribution in the nucleus is 3 fm, the corresponding atomic number is \underline{  }
	\item For a two-nucleon system, in spin singlet state, the spin is represented through the Pauli matrices $\sigma_1, \sigma_2$ for particles 1 and 2, respectively. The value of $\brak{\sigma_1 \cdot \sigma_2}$ is
	\item A contour is defined as 
		$$ I_n = \int \frac{dz}{(z-n)^2 + \pi^2} $$
		where $n$ is a positive integer and C is the closed contour, as shown in the figure, consisting of the line from -100 to 100 and the semicircle traversed in the counter-clockwise sense. The value of $\sum_{n=1}^{5} I_n $ is
	\item The normalised radial wave function of the second excited state of hydrogen atom is 
		$$ R(r) = \frac{1}{\sqrt{24}} a^{\frac{3}{2}} \frac{r}{a} e^{\frac{-r}{2a}} $$
		where $a$ is the Bohr radius and $r$ is the distance from the centre of the atom. The distance at which the electron is most likely to be found is $y \times a$, the value of $y$ is
	\item Consider an atomic gas with number density $n = 10^{20} m^{-3} $, in the ground state at 300 K. The valence electronic configuration of atoms is $f^7$. The paramagetic susceptibility of the gas $\chi = m \times 10^{-11}$. The value of $m$ is
	\item Consider a cross-section of an electromagnet having an air-gap of 5 cm as shown. It consists of a magnetic material with $\mu = 20000 \mu_0$ and is driven by a coil having $NI = 10^4$ where $N$ is the number of turns and $I$ is the current in Ampere.
		Ignoring the fringe fields, the magnitude of the magnetic field $\overline{B}$ in the air-gap between the magnetic poles is
	\item The spin $\vec{S}$ and orbital angular momentum $\vec{L}$ of an atom precess about $\vec{J}$, the total angular momentum. $\vec{J}$ precesses about an axis fixed by a magnetic field $\vec{B}_1 = 2B_0 \hat{z}$, where $B_0$ is a constant. Now the magnetic field is changed to
 $\vec{B}_2 = B_0 (\hat{x} + \sqrt{2} \hat{y} + \hat{z}).$ Given the orbital angular momentum quantum number $l = 2$ and spin quantum number $s = 1/2$, $\theta$ is the angle between $\vec{B}_1$ and $\vec{J}$ for the largest possible values of total angular quantum number $j$ and its $z$-component $j_z$. The value of $\theta$ (in degree, rounded off to the nearest integer) is \underline{  }.
	\item The spin-orbit effect splits the $^2P \rightarrow \; ^2S$ transition (wavelength, $\lambda = 6521 \; \text{\AA}$) in Lithium into two lines with separation of $\Delta \lambda = 0.14 \; \text{\AA}$. The corresponding positive value of energy difference between the above two lines, in eV, is $m \times 10^{-5}$. The value of $m$ (rounded off to the nearest integer) is \underline{ }.



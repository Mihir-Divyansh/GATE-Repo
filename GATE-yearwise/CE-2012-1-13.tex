 \iffalse
 \chapter{2012}
 \author{Prajwal naik}
 \section{ce}
 \fi







%16
\item The estimate of $\int_{0.5}^{1.5} \frac{dx}{x}$ obtained using Simpson's rule with three-point function evaluation exceeds the exact value by\hfill{(2012)}
    \begin{multicols}{4}
			\begin{enumerate}
\item 0.235
\item 0.068
\item 0.024
\item 0.012
        \end{enumerate}
			\end{multicols}

	\item The annual precipitation data of a city is normally distributed with mean and standard deviation as 1000 mm and 200 mm , respectively. The probability that the annual precipitation will be more than 1200 mm is\hfill{(2012)}

   \begin{multicols}{4}
			\begin{enumerate}

\item  $<50 \%$
\item $50 \%$
\item $75 \%$
\item $100 \%$
\end{enumerate}
		\end{multicols}


	\item The infinite series $1+x+\frac{x^{2}}{2!}+\frac{x^{3}}{3!}+\frac{x^{4}}{4!}+\ldots$ corresponds to\hfill{(2012)}

   \begin{multicols}{4}
			\begin{enumerate}
\item  $\sec x$
\item ${e}^{x}$
\item $\cos x$
\item $1+\sin ^{2} x$
  \end{enumerate}
		\end{multicols}


	\item The Poisson's ratio is defined as\hfill{(2012)}

   \begin{multicols}{4}
			\begin{enumerate}
\item $\abs{\frac{\text { axial stress }}{\text { lateral stress }}}$
\item  $\abs{\frac{\text { lateral strain }}{\text { axial strain }}}$
\item $\abs{\frac{\text { lateral stress }}{\text { axial stress }}}|$
\item $\abs{\frac{\text { axial strain }}{\text { lateral strain }}}$
   \end{enumerate}
		\end{multicols}


  \item The following statements are related to bending of beams:

I The slope of the bending moment diagram is equal to the shear force.\\
II The slope of the shear force diagram is equal to the load intensity.\\
III The slope of the curvature is equal to the flexural rotation.\\
IV The second derivative of the deflection is equal to the curvature.\\
The only FALSE statement is\hfill{2012}

   \begin{multicols}{4}
			\begin{enumerate}
   \item I
\item II
\item III
\item IV
\end{enumerate}
		\end{multicols}


	\item If a small concrete cube is submerged deep in still water in such a way that the pressure exerted on all faces of the cube is $p$, then the maximum shear stress developed inside the cube is\hfill{(2012)}

   \begin{multicols}{1}
			\begin{enumerate}
   \item  0
\item$\frac{p}{2}$
\item $p$
\item$2 p$
\end{enumerate}
		\end{multicols}


	\item As per IS 456:2000, in the Limit State Design of a flexural member, the strain in reinforcing bars under tension at ultimate state should not be less than\hfill{(2012)}

  \begin{multicols}{4}
			\begin{enumerate}
\item $\frac{f_{y}}{E_{s}}$
\item $\frac{f_{y}}{E_{s}}+0.002$
\item $\frac{f_{y}}{1.15 E_{s}}$
\item $\frac{f_{y}}{1.15 E_{s}}+0.002$
\end{enumerate}
		\end{multicols}


	\item Which one of the following is categorised as a long-term loss of prestress in a prestressed concrete member?\hfill{(2012)}

   \begin{multicols}{1}
			\begin{enumerate}
  \item Loss due to elastic shortening
\item Loss due to friction
\item  Loss due to relaxation of strands
\item  Loss due to anchorage slip
\end{enumerate}
		\end{multicols}


	\item In a steel plate with bolted connections, the rupture of the net section is a mode of failure under\hfill{(2012)}

\begin{multicols}{1}
			\begin{enumerate}

\item  tension
\item compression
\item flexure
\item shear
   \end{enumerate}
		\end{multicols}


	\item The ratio of the theoretical critical buckling load for a column with fixed ends to that of another column with the same dimensions and material, but with pinned ends, is equal to\hfill{(2012)}

  \begin{multicols}{1}
			\begin{enumerate}
   \item 0.5
\item 1.0
\item  2.0
\item 4.0
 \end{enumerate}
		\end{multicols}


	\item The effective stress friction angle of a saturated, cohesionless soil is $38^{\circ}$. The ratio of shear stress to normal effective stress on the failure plane is\hfill{(2012)}

  \begin{multicols}{4}
			\begin{enumerate}
   \item 0.781
\item 0.616
\item 0.488
\item 0.438
 \end{enumerate}
		\end{multicols}


  \item  Two series of compaction tests were performed in the laboratory on an inorganic clayey soil employing two different levels of compaction energy per unit volume of soil. With regard to the above tests, the following two statements are made.\\\\
  I The optimum moisture content is expected to be more for the tests with higher energy. II The maximum dry density is expected to be more for the tests with higher energy.
The CORRECT option evaluating the above statements is\hfill{(2012)}

\begin{multicols}{1}
			\begin{enumerate}
   \item Only I is TRUE
\item Only II is TRUE
\item Both I and II are TRUE
\item Neither I nor II is TRUE
\end{enumerate}
		\end{multicols}


	\item  As per the Indian Standard soil classification system, a sample of silty clay with liquid limit of 40\% and plasticity index of $28 \%$ is classified as\hfill{(2012)}

  \begin{multicols}{4}
			\begin{enumerate}
   \item CH
\item CI
\item CL
\item CL-ML
  \end{enumerate}
		\end{multicols}


 

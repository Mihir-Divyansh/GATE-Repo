\iffalse
\chapter{2021}
\author{AI24BTECH11018}
\section{ma}
\fi

\item The geometric mean radius of a conductor having four equal strands with each strand $r$, as shown in the figure below, is
\begin{figure}[!ht]
\centering
\resizebox{0.2\textwidth}{!}{%
\begin{circuitikz}
\tikzstyle{every node}=[font=\scriptsize]
\draw  (4,10) circle (0.5cm);
\draw  (5,10) circle (0.5cm);
\draw  (4,9) circle (0.5cm);
\draw  (5,9) circle (0.5cm);
\node at (4,10) [circ] {};
\draw [->, >=Stealth] (4,10) -- (4.25,10.5);
\node [font=\scriptsize] at (4,10.25) {r};
\end{circuitikz}
}%

\label{fig:my_label}
\end{figure}
\begin{enumerate}
    \item $4 r$ 
    \item $1.414 r$
    \item $2 r$
    \item $1.723 r$
\end{enumerate}
\item THe valid positive, negitive and zero sequence in $\brak{in p\cdot u}$,respectively for a $220 kV$ fully transposed three-phase transmission line, from the given choices are
\begin{enumerate}
    \item $1.1,0.15$ and $0.08$
    \item $0.15,0.15$ and $0.35$
    \item $0.2,0.2$ and $0.2$
    \item $0.1,0.3$ and $0.1$
\end{enumerate}
\item The steady state output $\brak{V_out}$ of the circuit shown below, will
\begin{figure}[!ht]
\centering
\resizebox{0.5\textwidth}{!}{%
\begin{circuitikz}
\tikzstyle{every node}=[font=\small]

\draw (5.5,9) node[op amp,scale=1] (opamp2) {};
\draw (opamp2.+) to[short] (4,8.5);
\draw  (opamp2.-) to[short] (4,9.5);
\draw (6.7,9) to[short](7,9);
\draw (7,9) to[short, -o] (8,9) ;
\draw (4,8.5) to[R] (4,6.25);
\draw (4,6.5) to (4,6.25) node[ground]{};
\draw (4,9.5) to[R] (2,9.5);
\draw (2,9.5) to[american voltage source] (2,6.75);
\draw (2,6.75) to (2,6.25) node[ground]{};
\draw [short] (4,9.5) -- (4,10.75);
\draw [short] (6.75,9) -- (6.75,10.75);
\draw (6.75,10.75) to[C,l={ \small $V_{DD}$}] (4,10.75);
\draw [short] (5.5,9.5) -- (5.5,10);
\draw [short] (5.5,8.5) -- (5.5,7.75);
\node [font=\small] at (1.25,8.25) {0.1 V};
\node [font=\small] at (3,10) {$R_1$};
\node [font=\small] at (4.5,7.25) {$R_2$};
\node [font=\small] at (5.5,7.5) {$-V_{EE}$};
\node [font=\small] at (8,8.75) {$V_{out}$};
\end{circuitikz}
}%

\label{fig:my_label}
\end{figure}
\begin{enumerate}
    \item saturate to $+V_{DD}$
    \item saturate to $-V_{EE}$
    \item becomes equal to $0.1 V$
    \item becomes equal to $-0.1 V$
\end{enumerate}
\item The Body magnitude plot of a first order stable system is constant with frequency. The asymptotic value of high frequency phase,for the system is $-180\circ$. This system has
\begin{figure}[!ht]
\centering
\resizebox{0.3\textwidth}{!}{%
\begin{circuitikz}
\tikzstyle{every node}=[font=\small]

\draw [<->, >=Stealth] (2.5,10) -- (2.5,6.5);
\draw (2.5,9.25) to[short] (5.5,9.25);
\draw [short] (2.5,8.25) -- (3,8);
\draw [short] (3,8) -- (3.5,7.25);
\draw [short] (3.5,7.25) -- (5.5,7.25);
\node [font=\small] at (3.5,9.5) {magnitude};
\draw [->, >=Stealth] (2.5,8.25) -- (5.5,8.25)node[pos=0.5, fill=white]{log(f)};
\node [font=\small] at (4.5,7) {phase};
\node [font=\small] at (2,7.25) {-180$\circ$};
\node [font=\small] at (2.25,8.25) {0$\circ$};
\end{circuitikz}
}%

\label{fig:my_label}
\end{figure}
\begin{enumerate}
    \item one LHP pole one RHP zero at the same frequency  
    \item one LHP pole one LHP zero at the same frequency
    \item two LHP poles and one RHP zero.
    \item two RHP poles and one LHP zero
\end{enumerate}
\item A balanced wheatstone bridge has the following arm resistance:\\
$R_{AB}=1K\Omega \pm 2.1$ percent; $R_{BC}=100\Omega \pm 0.5$ percent; $R_{CD}$ is a unknown resistance; $R_{DA}=300\Omega \pm 0.4$ percent. The value of $R_{CD}$ and its accuracy is 
\begin{enumerate}
    \item $30\Omega \pm 3\Omega$
    \item $30\Omega \pm 0.9\Omega$
    \item $3000\Omega \pm 90\Omega$
    \item $3000\Omega \pm 3\Omega$
\end{enumerate}
\item The open loop transfer function of a unity gain negive feedback system is given by $G\brak{s}=\frac{k}{s^2+4s-5}$.The of $K$ for which the system is stable, is 
\begin{enumerate}
    \item $k\textgreater 3$
    \item $k\textless 3$
    \item $k\textgreater 5$
    \item $k\textless 5$
\end{enumerate}
\item Consider a $3 \times 3$ matrix whose \brak{i,j}-th element $a_{i,j}=\brak{i-j}^2$.Then the matrix $A$ will be 
\begin{enumerate}
    \item symmetric
    \item skew-symmetric
    \item unitary
    \item null.
\end{enumerate}
\item In the circuit shown below, a three phase star-connected unbalanced load is connected to a balanced three-phase supply of $100\sqrt{3}V$ with phase sequnce $ABC$. the star the voltage differnce across the nodes $n$ and $n\prime$ is zero is  \\\\\\
\begin{figure}[!ht]
\centering
\resizebox{0.5\textwidth}{!}{%
\begin{circuitikz}
\tikzstyle{every node}=[font=\normalsize]

\draw [ line width=0.5pt](3,7.25) to[short] (10.25,7.25);
\draw [line width=0.5pt, short] (3,8) -- (3,7.25);
\draw [line width=0.5pt, short] (10.25,8) -- (10.25,7.25);
\draw [ line width=0.5pt](3,8) to[sinusoidal voltage source, sources/symbol/rotate=auto] (4.25,9.25);
\node at (4.25,9.25) [circ] {};
\draw [ line width=0.5pt](4.25,9.25) to[sinusoidal voltage source, sources/symbol/rotate=auto] (5.25,8.25);
\draw [line width=0.5pt, short] (5.25,8.25) -- (7.5,8.25);
\draw [ line width=0.5pt](10.25,8) to[european resistor] (8.75,9.5);
\draw [ line width=0.5pt](8.75,9.5) to[european resistor] (7.5,8.25);
\node at (8.75,9.5) [circ] {};
\draw [ line width=0.5pt](4.25,11) to[short] (8.75,11);
\draw [ line width=0.5pt](8.75,11) to[european resistor] (8.75,9.5);
\draw [ line width=0.5pt](4.25,11) to[sinusoidal voltage source, sources/symbol/rotate=auto] (4.25,9.25);
\node [font=\small] at (4.5,9.25) {n};
\node [font=\small] at (5.25,9) {$E_B$};
\node [font=\normalsize] at (5.25,8.75) {+};
\node [font=\normalsize] at (2.75,8.25) {+};
\node [font=\normalsize] at (3.5,9.25) {$E_C$};
\node [font=\normalsize] at (4,10.75) {+};
\node [font=\normalsize] at (4.75,10.5) {$E_A$};
\node [font=\normalsize] at (7.75,9.25) {$Z_B$};
\node [font=\normalsize] at (9,9.5) {n};
\node [font=\normalsize] at (10,9) {$Z_C$};
\node [font=\normalsize] at (9.25,10.25) {$Z_A$};
\end{circuitikz}
}%

\label{fig:my_label}
\end{figure}
\begin{enumerate}
    \item $20 \textless -30\circ$
    \item $20 \textless 30\circ$
    \item $20 \textless -60\circ$
    \item $20 \textless 600\circ$
\end{enumerate}
\item A charger supplies $100W$ at $20V$ for charging of battery of a laptop. The power devices, used in the converter inside the charger operate at a switching at a frequency of $200kHz$ which power devices is best suited for this purpose?
\begin{enumerate}
    \item IGBT
    \item Thyristor
    \item MOSFET
    \item BJT
\end{enumerate}
\item A long conducting cylinder having a radius $b$ is placed along $Z$ axis. The current density is $J=J_ar^3\hat{z}$ for the region $r\textless b$ where $r$ is the distance in the radial direction. THe magnitude feild intensity $\brak{H}$ for the region inside of the conductor $\brak{i.e for r \textless b}$ is
\begin{enumerate}
    \item $\frac{j_a}{4}r^4$
    \item $\frac{j_a}{3}r^3$
    \item $\frac{j_a}{5}r^5$
    \item $j_ar^3$
\end{enumerate}
\item The type of single phase induction motor, expected to have the maximum power factor during steady state running condition is 
\begin{enumerate}
    \item split phase \brak{resistance  start}
    \item shaded pole 
    \item capacitor start
    \item capacitor start, capacitor run.
\end{enumerate}
\item For the circuit shown below with ideak diodes, the output will be \\\\\\\\\\\\
\begin{figure}[!ht]
\centering
\resizebox{0.5\textwidth}{!}{%
\begin{circuitikz}
\tikzstyle{every node}=[font=\normalsize]

\draw [ line width=0.5pt](2.25,10) to[D] (4.75,10);
\draw [line width=0.5pt, short] (4.5,10) -- (6.5,10);
\draw [ line width=0.5pt](2.25,10) to[short, -o] (2,10) ;
\draw [ line width=0.5pt](2,8) to[D] (5,8);
\draw [ line width=0.5pt](2,8) to[short, -o] (2,8) ;
\draw [ line width=0.5pt](2.75,8) to[short, -o] (2,8) ;
\draw [ line width=0.5pt](6,10) to[short, -o] (6.5,10) ;
\draw [ line width=0.5pt](5,8) to[short, -o] (6.5,8) ;
\draw [ line width=0.5pt](5.25,10) to[R,l={ \normalsize $R_1$}] (5.25,8);
\draw [ line width=0.5pt](1.75,9) to[short] (3.75,9);
\draw [line width=0.5pt, short] (3,9) .. controls (3,8.5) and (3.25,8.75) .. (3.5,8.5);
\draw [line width=0.5pt, short] (3.5,8.5) .. controls (3.75,8.75) and (3.75,8.75) .. (3.75,9);
\node [font=\normalsize] at (1.75,9.25) {$V_{in}$};
\node [font=\normalsize] at (3.5,7.5) {$D_2$};
\node [font=\normalsize] at (3.5,10.5) {$D_1$};
\node [font=\normalsize] at (6.5,9) {$V_{out}$};
\node [font=\normalsize] at (6.5,9.75) {$+$};
\node [font=\normalsize] at (6.5,8.25) {$-$};
\node [font=\normalsize] at (2,10.25) {$+$};
\node [font=\normalsize] at (2,7.75) {$-$};
\draw [line width=0.5pt, short] (2,9) .. controls (1.75,9.5) and (2.25,9.5) .. (2.5,9.5);
\draw [line width=0.5pt, short] (2.5,9.5) .. controls (2.75,9.5) and (3,9.5) .. (3,9);
\end{circuitikz}
}%

\label{fig:my_label}
\end{figure}
\begin{enumerate}
    \item $V{out}=V_{in}$ for $V_{in}\textgreater 0$
    \item $V{out}=V_{in}$ for $V_{in}\textless 0$
    \item $V{out}=-V_{in}$ for $V_{in}\textgreater 0$
    \item $V{out}=-V_{in}$ for $V_{in}\textless 0$
\end{enumerate}
\item A MOD $2$ and a MOD $5$ up-counter when cascaded together results in a MOD counter \underline{\hspace{2cm}}


\iffalse
\chapter{2008}
\author{EE24BTECH11058}
\section{xe}
\fi
 
%\begin{enumerate}
    \item An LC circuit is shown in the figure. The inductor current,i, when the switch S is opened at t=0 is best represented by\\
    
     
           

    \item In the figure shown, power supplied by the current source is 
      
    \begin{enumerate}
        \item $0.0 W$
        \item $5.0 W$, delivered
        \item $10.0 W$, delivered
        \item $10.0 W$, absorbed \\
    \end{enumerate}


    \item An inductor of $0.4 H$ was constructed with $20$ turns on an iron core. If $10$ additional turns in the same sense are added to the coil on the same core, the new inductance will be
    \begin{enumerate}
        \item $0.9 H$
        \item $0.8 H$
        \item $0.7 H$
        \item $0.6 H$ \\
    \end{enumerate}


    \item A three-phase star-connected, slip-ring induction motor has per-phase standstill rotor resistance, $r_{2} = 0.01 \ohm$ and reactance, $x_{2} = 0.05 \ohm$. To achieve maximum torque at starting, the external per-phase resistance to be connected at the slip-rings is 
    \begin{enumerate}
        \item $ 0.01 \ohm$
        \item $ 0.02 \ohm$
        \item $ 0.03 \ohm$
        \item $ 0.04 \ohm$ \\
    \end{enumerate}

    \item The device structure shown in the figure is that of a 
    \begin{enumerate}
        \item pnp BJT
        \item p-channel MOSFET
        \item npn BJT
        \item n-channel MOSFET \\
    \end{enumerate}

    \item The input voltage applied to the rectifier circuit shown in the figure is $v_{in}=V_{m}\sin\brak{2\pi50t}.$The steady state output voltage $V_{o}$ of the rectifier,under no-load condition,is
    \begin{enumerate}
        \item $V_{m}$
        \item $\sqrt{2} V_{m}$
        \item $2 V_{m}$
        \item $2 \sqrt{2}V_{m}$ \\   
    \end{enumerate}

    \item IN the figure shown, the diode is ideal and the zener voltage is $10 V.$ The input voltage,$v_{in} = 10 \sqrt{2}\sin\brak{100\pi t} V.$ The wave-shape of the current through the resistor, R is represented by 
    


    \item For the counter shown in figure, the present count, $Q_{1}Q_{2}Q_{3}Q_{4}$ is $0100.$ The count after two clock pulses will be 
    \begin{enumerate}
        \item $0100$
        \item $0001$
        \item $0010$
        \item $1000$\\
    \end{enumerate}

    \item An incandescent lamp is rated for $200 V ,100 W$. Neglect temperature effects. When the lamp consumes $121 W$ , the supply voltage is 
    \begin{enumerate}
        \item $242 V$
        \item $220 V$
        \item $180 V$
        \item $165 V$\\
    \end{enumerate}

    \item In the circuit shown in the figure, the load resistance, $R_{L}$ draws $15 A$ when it is $10\ohm$ and $20 A$ when it is $5\ohm.$ The open circuit voltage across $XY$ is 
    \begin{enumerate}
        \item $100 V$
        \item $150 V$
        \item $200 V$
        \item $300 V$ \\
    \end{enumerate}
    

    \item Three $15 V$ batteries are connected to a resistive network as shown in the figure. The current in each resistor is 
    \begin{enumerate}
        \item $0.0 A$
        \item $0.667 A$
        \item $1.0 A$
        \item $1.5 A$ \\
    \end{enumerate}


    \item At a particular frequency,the impedance across terminals $AB$ in the figure shown is $\brak{6.0 + j 0.0}$ ohms. If $R_{1} = 12 \ohm , C_{1} = 10 \mu F , L_{1} = 0.2 H , L_{2} = 0.1 H$ , then $C_{2}$ is 
    \begin{enumerate}
        \item $1.414 \mu F$
        \item $5 \mu F$
        \item $12 \mu F$
        \item $20 \mu F$\\
    \end{enumerate}

    \item A transformer is feeding a $2.5 kVA$ load at $0.8 pf \brak{lag}.$ If its efficiently is $95 \%$ and the copper losses equal $55 W$ , the core loss is  
    \begin{enumerate}
        \item $25.13 W$
        \item $0,26 W$
        \item $100.54$
        \item $125.26 W$\\
    \end{enumerate}

    \item The total winding resistance of a single-phase ,two-winding transformer is half the magnitude of total impedance of the winding, both referred to the primary side. Considering the input and output voltages to be practically in- phase, the transformer will have zero regulation when the load power factor is 
    \begin{enumerate}
        \item $60 \degree$ lagging
        \item $60 \degree$ leading
        \item $30 \degree$ leading
        \item $30 \degree$ lagging\\  
    \end{enumerate}


    \item A $220V$ dc shunt motor having an armature resistance $r_{a} = 0.5 \ohm$  draws an armature current of $40 V$ when running at $1400 rpm.$ If the load torque is halved at the same field current and maintaining the same terminal voltage, then $\brak{neglecting armature reaction}$ the speed of the motor will be 
    \begin{enumerate}
        \item $1510 rpm$
        \item $1485 rpm$
        \item $1470 rpm$
        \item $1370 rpm$ \\
    \end{enumerate}


    
    \item A $230 V$ separately excited dc  motor having an armature resistance $r_{a} = 2 \ohm.$ It draws $15 A$ when running at a speed $N_{1}.$ If the supply to the armature is disconnected, the field excitation and speed remaining unchanged,the voltage at the armature terminals will be 
    \begin{enumerate}
        \item $0 V$
        \item $200 V$
        \item $210 V$
        \item $240 V$ \\
    \end{enumerate}


    \item In an induction motor the phase-difference, $\phi,$ between the voltage applied at the stator terminals and the magnetizing current is
    \begin{enumerate}
       \item $\phi = 0\degree$
       \item $0\degree < \phi = 90\degree$
       \item $\phi = 90\degree$
       \item $90 \degree < \phi < 180 \degree$ \\
    \end{enumerate}

    
%\end{enumerate}

 

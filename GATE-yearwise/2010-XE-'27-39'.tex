\iffalse
\chapter{2010}
\author{AI24BTECH11027}
\section{xe}
\fi

\item Let $f(x+iy)=u(x,y)+iv(x,y)$ be an analytic defined on the complex plane satisfying $2u^2+3v^2=1.$ then 
\begin{enumerate}
    \item f is a constant
    \item $f(z)=kz$ some nonzero real number $k$
    \item $u(x,y)=\frac{\cos(x+y)}{\sqrt{2}}$
    \item $v(x,y)=\frac{\sin(x-y)}{\sqrt{3}}$ \\
\end{enumerate}

\item The value of $\oint_c (xy^2+2x)dx+(x^2y+4x)dy$ along the circle $C:x^2+y^2=4$ in the anticlockwise direction is
\begin{enumerate}
    \item $-16\pi$
    \item $-4\pi$
    \item $4\pi$
    \item $16\pi$ \\
\end{enumerate}

\item The volume of the prism whose base is the triangle in the $xy$-plane bounded by the x-axis and the lines $y=x$ and $x=2$ and whose top lies in the plane $z=5-x-y$ is 
\begin{enumerate}
    \item $2$
    \item $4$
    \item $6$
    \item $10$ \\
\end{enumerate}

\item The general solution of $x(z^2-y^2)\frac{\partial z}{\partial x}+y(x^2-z^2)\frac{\partial z}{\partial y}=z(y^2-x^2)$ is 
\begin{enumerate}
    \item $F(x^2+y^2+z^2,xyz)=0$
    \item $F(x^2+y^2-Z^2,xyz)=0$
    \item $F(x^2-y^2+Z^2,xyz)=0$
    \item $F(-x^2+y^2+Z^2,xyz)=0$ \\
\end{enumerate}

\item Choose a point uniformly distributed at random on the disc $x^2+y^2\leq1$. Let the random variable $X$ denote the distance of this point from the center of the disc. Then the variance of $X$ is 
\begin{enumerate}
    \item $\frac{1}{16}$
    \item $\frac{1}{17}$
    \item $\frac{1}{18}$
    \item $\frac{1}{19}$ \\
\end{enumerate}
    
\item If Runge-kutta method of order 4 is used to solve the differential equation $\frac{dy}{dx}=f(x),y(0)=0$ in the interval $[0.h]$ with step size $h$, then 
\begin{enumerate}
    \item $y(h)=\frac{h}{6}[f(0)+4f(h/2)+f(h)]$
    \item $y(h)=\frac{h}{6}[f(0)+4f(h)]$
    \item $y(h)=\frac{h}{2}[f(0)+4f(h)]$
    \item $y(h)=\frac{h}{6}[f(0)+2f(h/2)+f(h)]$ \\
\end{enumerate}

\item If a polynomial of degree three interpolates a function $f(x)$ at the points $(0, 3), (1,13),(3,99)$ and $(4,187),$ then $f(2)$ is 
\begin{enumerate}
    \item $20$
    \item $36$
    \item $43$
    \item $58$ \\
\end{enumerate}

\item Let $f:R\rightarrow R$ be defined by $f(x)=x^2$ for $-\pi \leq x \leq \pi$ and $f(x+2\pi)=f(x).$ 
The Fourier series of $f$ in $[-\pi,\pi]$ is 
\begin{enumerate}
    \item $\frac{\pi^2}{3}+4\sum_{n=1}^\infty \frac{\cos nx}{n^2}$
    \item $\frac{\pi^2}{3}+\sum_{n=1}^\infty \frac{(-1)^n \cos nx}{n^2}$
    \item $\frac{\pi^2}{3}+4\sum_{n=1}^\infty \frac{(-1)^n \cos nx}{n^2}$
    \item $\frac{\pi^2}{3}+\sum_{n=1}^\infty \frac{\cos nx}{n^2}$ \\
\end{enumerate}


\item Let $f:R\rightarrow R$ be defined by $f(x)=x^2$ for $-\pi \leq x \leq \pi$ and $f(x+2\pi)=f(x).$  
The sum of the absolute value of the Fourier coefficients of $f$ is
\begin{enumerate}
    \item $\frac{\pi^2}{6}$
    \item $\frac{\pi^2}{3}$
    \item $\frac{2\pi^2}{3}$
    \item $\pi^2$ \\
\end{enumerate}

\item Let $y(x)=\sum_{n=0}^\infty a_n x^n$ be a solution of the differential equation $\frac{d^2 y}{dx^2}+xy=0.$
The value of $a_{11}$ is 
\begin{enumerate}
    \item 0
    \item 1
    \item 2 
    \item 3 \\
\end{enumerate}

\item Let $y(x)=\sum_{n=0}^\infty a_n x^n$ be a solution of the differential equation $\frac{d^2 y}{dx^2}+xy=0.$
The value of $a_11$ is The solution of the differential equation given above satisfying $y(0)=1$ and $y'(0)=0$ is 
\begin{enumerate}
    \item $y(x)=1+\frac{1}{2.3}x^2-\frac{1}{2.3.5.6}x^4+\frac{1}{2.3.5.6.8.9}x^6-....$
    \item $y(x)=1-\frac{1}{2.3}x^2+\frac{1}{2.3.5.6}x^4-\frac{1}{2.3.5.6.8.9}x^6+....$
    \item $y(x)=1+\frac{1}{2.3}x^3-\frac{1}{2.3.5.6}x^6+\frac{1}{2.3.5.6.8.9}x^9-....$
    \item $y(x)=1-\frac{1}{2.3}x^3-\frac{1}{2.3.5.6}x^6-\frac{1}{2.3.5.6.8.9}x^9+....$ \\
\end{enumerate}

\item The potential $u(x,y)$ satisfies the equation $\frac{\partial^2 u}{\partial x^2}+\frac{\partial^2 u}{\partial y^2}=0$ in the square $0 \leq x \leq \pi, 0\leq y \leq \pi.$ Three of the edges $x=0$ and $y=0$ of the square are kept at zero potential and the edge $y=\pi$ is kept at nonzero potential. the potential $u(x,y)$ is given by 
\begin{enumerate}
 \item $u(x,y)=\sum_{n=1}^\infty A_n \cos h nx \sin ny$
 \item $u(x,y)=\sum_{n=1}^\infty A_n \sin nx \cosh ny$
 \item $u(x,y)=\sum_{n=1}^\infty A_n \sin h nx \sin ny$
 \item $u(x,y)=\sum_{n=1}^\infty A_n \sin nx \sin h ny$ \\
\end{enumerate}

\item The potential $u(x,y)$ satisfies the equation $\frac{\partial^2 u}{\partial x^2}+\frac{\partial^2 u}{\partial y^2}=0$ in the square $0 \leq x \leq \pi, 0\leq y \leq \pi.$ Three of the edges $x=0$ and $y=0$ of the square are kept at zero potential and the edge $y=\pi$ is kept at nonzero potential. If the edge $y=\pi$ is kept at the potential $\sin x$, then the potential $u(x,y)$ is given by 
\begin{enumerate}
    \item$u(x,y)=\sum_{n=1}^\infty \frac{\sin nx \sin h ny}{\sin h n \pi}$
    \item$u(x,y)=\frac{\sin x \sin h y}{\sin h \pi}$
    \item$u(x,y)=\frac{\sin x \cos h y}{\cos h \pi}$
    \item$u(x,y)=\sum_{n=1}^\infty \frac{\cos hnx \sin ny}{\cos h n \pi}$
\end{enumerate}
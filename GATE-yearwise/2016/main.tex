\iffalse
    \title{Assignment}
    \author{EE24BTECH11034}
    \section{ma}
    \chapter{2016}
  \fi
\item Let $u\brak{x,y} = x^3 + a x^2 y + b x y^2 + 2 y^3$ be a harmonic function and $v\brak{x,y}$ its harmonic conjugate. If $v\brak{0,0} = 1$, then $\abs{a + b + v\brak{1,1}}$ is equal to \underline{\hspace{2cm}}

\item Let $\gamma$ be the triangular path connecting the points $\brak{0,0}$, $\brak{2,2}$ and $\brak{0,2}$ in the counter-clockwise direction in $\mathbb{R}^2$. Then

    $$I = \oint_{\gamma}  
    \sin\brak{x^3}dx + 6 x y dy$$

    is equal to \underline{\hspace{2cm}}

\item Let $y$ be the solution of 

    $$y\prime + y = \abs{x}, \quad x \in \mathbb{R}, \quad y\brak{-1} = 0.$$

    Then $y\brak{1}$ is equal to

    \begin{enumerate}
        \item $\dfrac{2}{e} - \dfrac{2}{e^2}$
        \item $\dfrac{2}{e} - 2 e^2$
        \item $2 - \dfrac{2}{e}$
        \item $2 - 2 e$
    \end{enumerate}

\item Let $X$ be a random variable with the following cumulative distribution function:

    $$F\brak{x} = \begin{cases}
        0 & x < 0 \\
        x^2 & 0 \leq x < \dfrac{1}{2} \\
        \dfrac{3}{4} & \dfrac{1}{2} \leq x < 1 \\
        1 & 2 \leq x < 1.
    \end{cases}$$

    Then $P\brak{\dfrac{1}{4} < X < 1}$ is equal to \underline{\hspace{2cm}}

\item Let $\gamma$ be the curve which passes through $\brak{0,1}$ and intersects each curve of the family $y = c x^2$ orthogonally. Then $\gamma$ also passes through the point

    \begin{enumerate}
        \item $\brak{\sqrt{2},0}$
        \item $\brak{0,\sqrt{2}}$
        \item $\brak{1,1}$
        \item $\brak{-1,1}$
    \end{enumerate}

\item Let $S\brak{x} = a_0 + \sum_{n=1}^{\infty}\brak{a_n \cos\brak{n x} + b_n \sin\brak{n x}}$ be the Fourier series of the $2 \pi$ periodic function defined by $f\brak{x} = x^2 + 4 \sin\brak{x} \cos\brak{x}, -\pi \le x \le \pi$. Then 

    $$\abs{\sum_{n=0}^{\infty} a_n - \sum_{n=1}^{\infty} b_n}$$

    is equal to \underline{\hspace{2cm}}

\item Let $y\brak{t}$ be a continuous function on $\sbrak{0,\infty}$. If

    $$y\brak{t} = t\brak{1 - 4\int_{0}^{t}y\brak{x}dx} + 4\int_{0}^{t}x y\brak{x}dx,$$

    then $\int_{0}^{\pi/2} y\brak{t} dt$ is equal to \underline{\hspace{2cm}}

\item Let $S_n = \sum_{k=1}^{n} \dfrac{1}{k}$ and $I_n = \int_{1}^{n} \dfrac{x - \sbrak{x}}{x^2} dx$. Then $S_{10} + I_{10}$ is equal to

    \begin{enumerate}
        \item $\ln 10 + 1$
        \item $\ln 10 - 1$
        \item $\ln 10 - \dfrac{1}{10}$
        \item $\ln 10 + \dfrac{1}{10}$
    \end{enumerate}

\item For any $\brak{x,y} \in \mathbb{R}^2 \setminus B\brak{0,1}$, let

    $$f\brak{x,y} = \text{distance}\brak{\brak{x,y}, B\brak{0,1}}$$
    $$= \text{inf}\sbrak{\sqrt{\brak{x - x_1}^2 + \brak{y - y_1}^2} : \brak{x_1,y_1} \in B\brak{0,1}}.$$

    Then, $\abs{\nabla f\brak{3,4}}$ is equal to \underline{\hspace{2cm}}

\item Let $f\brak{x} = \int_{0}^{x} e^{t^2 - t^2} dt$ and $g\brak{x} = \int_{0}^{x} \dfrac{e^{t^2 + t^2}}{1 + t^2} dt$. Then $f\brak{\sqrt{\pi}} + g\brak{\sqrt{\pi}}$ is equal to \underline{\hspace{2cm}}

\item Let $M = \begin{bmatrix}
        a & b & c \\
        d & e & f \\
        g & h & i
    \end{bmatrix}$ be a real matrix with eigenvalues $1$, $0$, and $3$. If the eigenvectors corresponding to $1$ and $0$ are $\sbrak{1,1,1}^T$ and $\sbrak{1,-1,0}^T$, respectively, then  
 the value of $3 f$ is equal to \underline{\hspace{2cm}}

\item Let $M = \begin{bmatrix}
        1 & 1 & 0 \\
        0 & 1 & 1 \\
        0 & 0 & 1
    \end{bmatrix}$ and $e^M = I + M + \dfrac{1}{2!} M^2 + \dfrac{1}{3!} M^3 + \cdots$. If $e^M = \sbrak{b_{ij}}$, then 

    $$\dfrac{1}{e} \sum_{i=1}^3 \sum_{j=1}^3 b_{ij}$$

    is equal to \underline{\hspace{2cm}}

\item Let the integral $I = \int_{0}^{4} f\brak{x} dx$, where $f\brak{x} = \begin{cases}
        x & 0 \leq x \leq 2 \\
        4 - x & 2 \leq x \leq 4.
    \end{cases}$

    Consider the following statements $P$ and $Q$:

    $P$: If $I_2$ is the value of the integral obtained by the composite trapezoidal rule with two equal sub-intervals, then  
 $I_2$ is exact.

    $Q$: If $I_3$ is the value of the integral obtained by the composite trapezoidal rule with three equal sub-intervals, then $I_3$ is exact.

    Which of the above statements hold TRUE?  

    \begin{enumerate}
        \item both $P$ and $Q$
        \item only $P$
        \item only $Q$
        \item Neither $P$ nor $Q$
    \end{enumerate}



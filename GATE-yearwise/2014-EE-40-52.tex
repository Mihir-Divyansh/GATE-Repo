\iffalse
\chapter{2014}
\author{AI24BTECH11004-Bheri Sai Likith Reddy}
\section{ee}
\fi

       \item In the figure, the value of resistor R is (25 + I/2) ohms, where I is the current in amperes. The
       current I is  \rule{1cm}{0.15mm}
       \hfill{(2014)}
		
\centering
\resizebox{0.3\textwidth}{!}{%
\begin{circuitikz}
    \tikzstyle{every node}=[font=\small]
    \draw (1.75,17.75) to[short] (3.75,17.75);
    \draw (3.75,17.75) to[short] (3.75,17.25);
    \draw (3.75,17.25) to[european resistor] (3.75,16.25);
    \draw (3.75,16.25) to[short] (3.75,15.75);
    \draw (3.75,15.75) to[short] (1.75,15.75);
    \draw (1.75,17.75) to[battery] (1.75,15.75);
    \draw [->, >=Stealth] (1.75,18) -- (3.5,18);
    \node [font=\LARGE] at (4.25,17) {R};
    \node [font=\LARGE] at (2.5,18.5) {I};
    \node [font=\small] at (1,16.75) {300V};
\end{circuitikz}}%




	\item In an unbalanced three phase system, phase current $I_\alpha=1\angle \brak{-90^{\circ}}$ pu, negative sequence current $I_(b_2)=4\angle \brak{150_{\circ}}$ pu, zero sequence current $I_c0=3 \angle 90_{\circ}$ pu. The magnitude of phase current $I_b$ in pu is
	\hfill{(2014)}

               \begin{enumerate}
			        \begin{multicols}{4}  
		       \item $1.00$
		       \item $7.81$
		       \item $11.53$
		       \item $13.00$
                    \end{multicols}   
	       \end{enumerate}	
       \item The following four vector fields are given in Cartesian co-ordinate system. The vector field which
does not satisfy the property of magnetic flux density is
\hfill{(2014)}
		\begin{enumerate}
			\item $y^2a_x+z^2a_x+x^2a_z$
			\item $z^2a_x+x^2a_y+y^2a_z$
			\item $x^2a_x+y^2a_y+z^2a_z$
			\item $y^2z^2a_x+x^2z^2a_y+x^2y^2a_z$
		\end{enumerate}
	\item  The function shown in the figure can be represented as
	\hfill{(2014)}
		
\centering
\resizebox{0.3\textwidth}{!}{%
\begin{circuitikz}
    \tikzstyle{every node}=[font=\small]
    \draw [->, >=Stealth] (1.25,13.5) -- (1.25,17.75);  % Vertical arrow
    \draw [->, >=Stealth] (0.5,14.5) -- (4.75,14.5);    % Horizontal arrow
    \draw [short] (1.25,16.5) -- (2.25,16.5);           % Horizontal line
    \draw [short] (2.25,16.5) -- (2.25,14.5);           % Vertical line
    \draw [short] (2.25,14.5) -- (3.25,16.5);           % Diagonal line
    \draw [short] (3.25,16.5) -- (4,16.5);              % Horizontal line
    \node [font=\small] at (1,16.5) {1};                % Node at (1,16.5)
    \node [font=\small] at (1,14.25) {0};               % Node at (1,14.25)
    \node [font=\small] at (2.25,14.25) {T};            % Node at (2.25,14.25)
    \draw [short] (3.25,14.75) -- (3.25,14.25);         % Vertical short line
    \node [font=\small] at (3.25,14) {2T};              % Node at (3.25,14)
    \node [font=\small] at (4.5,14.75) {t};             % Node at (4.5,14.75)
\end{circuitikz}

}%



		\begin{enumerate}
			\item $u\brak{t}-u\brak{t-T}+\frac{t-T}{T}u\brak{t-T}-\frac{tu-2T}{T}u\brak{t-2T}$
			\item $u\brak{t}+ \frac{t}{T}u\brak{t-T}-\frac{t}{T}u{t-2T}$
			\item $u\brak{t}-u\brak{t-T}+\frac{t-T}{T}u\brak{t}-\frac{t-2T}{T}u\brak{t}$
	        \item $u\brak{t}+\frac{t-T}{T}u\brak{t-T}-\frac{tu-2T}{T}u\brak{t-2T}$
        	\end{enumerate}
	\item Let $X\brak{z}=\frac{1}{1-z^{-3}}$ be the $Z$-transform of a causal signal $x\sbrak{n}$. Then, the values of $x\sbrak{2}$ and $x\sbrak{3}$ are
	\hfill{(2014)}
		\begin{enumerate}
		       \item $0$ and $0$
		       \item $0$ and $1$
		       \item $1$ and $0$
		       \item $1$ and $1$
        	\end{enumerate}	
	\item Let $f\brak{t}$ be a continuous time signal and let $F\brak{w}$ be its Fourier Transform defined by
 \begin{align*}
     F\brak{w}=\int_{-\infty}^{\infty}f\brak{t}e^{-jwt}dt
 \end{align*}
 Define $g\brak{t}$
 \begin{align*}
     g\brak{w}=\int_{-\infty}^{\infty}F\brak{u}e^{-jut}du
 \end{align*}
 What is the relationship between $f\brak{t}$ and $g\brak{t}$?
 \hfill{(2014)}
		\begin{enumerate}
			\item $g\brak{t}$ would always be proportional to $f\brak{t}$
			\item $g\brak{t}$ would be proportional to $f\brak{t}$ if $f\brak{t}$ is an even function.
			\item $g\brak{t}$ would be proportional to $f\brak{t}$ only if $f\brak{t}$ is a sinusoidal function.
			\item $g\brak{t}$ would never be proportional to $f\brak{t}$
        	\end{enumerate}
	\item The core loss of a single phase, $230/115 V$, $50 Hz$ power transformer is measured from $230 V$ side
by feeding the primary ($230 V$ side) from a variable voltage variable frequency source while
keeping the secondary open circuited. The core loss is measured to be $1050 W$ for $230 V$, $50 Hz$
input. The core loss is again measured to be $500 W$ for $138 V$, $30 Hz$ input. The hysteresis and eddy
current losses of the transformer for $230 V$, $50 Hz$ input are respectively,
\hfill{(2014)}
		\begin{enumerate}
		       \item $508 W$ and $542 W$. 
		       \item $468 W$ and $582 W$.
		       \item $498 W$ and $552 W$. 
		       \item $488 W$ and $562 W$.
        	\end{enumerate}	
	\item A $15 kW$, $230 V$ dc shunt motor has armature circuit resistance of $0.4 \omega$ and field circuit resistance
of $230 \omega$. At no load and rated voltage, the motor runs at $1400 rpm$ and the line current drawn by
the motor is $5 A$. At full load, the motor draws a line current of $70 A$. Neglect armature reaction.
The full load speed of the motor in rpm is \rule{1cm}{0.15mm}
\hfill{(2014)}
    
	\item  A $3$ phase, $50 Hz$, six pole induction motor has a rotor resistance of $0.1 \omega$ and reactance of $0.92\omega$.
Neglect the voltage drop in stator and assume that the rotor resistance is constant. Given that the
full load slip is $3\%$, the ratio of maximum torque to full load torque is 
\hfill{(2014)}
		\begin{enumerate}
            \begin{multicols}{4}
			\item $1.567$
			\item $ 1.712$
			\item $ 1.948$
			\item $ 2.134$
   \end{multicols}
        	\end{enumerate}	
	\item A three phase synchronous generator is to be connected to the infinite bus. The lamps are connected
as shown in the figure for the synchronization. The phase sequence of bus voltage is R-Y-B and
that of incoming generator voltage is R'-Y'-B'.


\centering
\resizebox{0.3\textwidth}{!}{%
\begin{circuitikz}
    \tikzstyle{every node}=[font=\large]

    % Vertical lines
    \draw [short] (1.25,18.25) -- (1.25,13.25);
    \draw [short] (2,18.25) -- (2,13.25);
    \draw [short] (2.75,18.25) -- (2.75,13.25);

    % Nodes
    \node at (1.25,17.25) [circ] {};
    \node at (2,16.25) [circ] {};
    \node at (3.75,17.25) [circ] {};
    \node at (3.25,16.25) [circ] {};
    \node at (2.75,15) [circ] {};
    \node at (3,15) [circ] {};

    % Horizontal lines with open-end
    \draw (1.25,17.25) to[short, -o] (4.5,17.25);
    \draw (2,16.25) to[short, -o] (4.5,16.25);
    \draw (2.75,15) to[short, -o] (4.5,15);

    % Connections and circles
    \draw (3.75,17.25) to[short] (3.75,14.25);
    \draw (3.25,16.25) to[short] (3.25,13.5);
    \draw (3,15) to[short] (3,12.25);
    \draw (3.75,14.25) to[short] (4.5,14.25);
    \draw (3.25,13.5) to[short] (4.5,13.5);
    \draw (3,12.25) to[short] (4.5,12.25);
    \draw (4.75,14.25) circle (0.25cm);
    \draw (4.75,13.5) circle (0.25cm);
    \draw (4.75,12.25) circle (0.25cm);

    % Labels
    \node [font=\small] at (4.75,14.25) {L\_a};
    \node [font=\small] at (4.75,13.5) {L\_b};
    \node [font=\small] at (4.75,12.25) {L\_c};

    % Horizontal connections
    \draw (5,14.25) to[short] (5.75,14.25);
    \draw (5,13.5) to[short] (6.25,13.5);
    \draw (5,12.25) to[short] (7,12.25);
    \draw (5.75,14.25) to[short] (5.75,17.25);
    \draw (6.25,13.5) to[short] (6.25,16.25);
    \draw (7,12.25) to[short] (7,15);
    \draw (5.75,17.25) to[short] (5.25,17.25);
    \draw (5.25,16.25) to[short] (7.75,16.25);
    \draw (5.25,15) to[short] (7.5,15);

    % Nodes on R', Y', B'
    \node at (5.75,17.25) [circ] {};
    \node at (6.25,16.25) [circ] {};
    \node at (7,15) [circ] {};

    % Additional lines and sinusoidal sources
    \draw (5.25,17.25) to[short] (4.5,18);
    \draw (5.25,16.25) to[short] (4.5,17);
    \draw (5.25,15) to[short] (4.5,15.75);
    \draw (4.75,17.75) to[short] (4.75,15.5);
    \draw (5,17.5) to[short] (5,15.25);

    % Generator part
    \draw (7.5,15) to[short] (7.5,13.5);
    \draw (7.5,13.5) to[short] (8,13.5);
    \draw (8,13.5) to[short] (8.5,14);
    \draw (8.5,14) to[short] (9,13.5);
    \draw (9,13.5) to[short] (9.5,13.5);
    \draw (9.5,13.5) to[short] (9.5,14.5);
    \draw (9.5,14.5) to[sinusoidal voltage source, sources/symbol/rotate=auto] (11.25,16.25);
    \draw (11.25,16.25) to[sinusoidal voltage source, sources/symbol/rotate=auto] (11.25,17.25);
    \draw (11.25,16.25) to[sinusoidal voltage source, sources/symbol/rotate=auto] (13,14.5);

    % Infinite bus and dashed box
    \draw (8.25,16.25) to[short] (8.25,12.75);
    \draw (8.25,12.75) to[short] (13,12.75);
    \draw (13,12.75) to[short] (13,14.5);
    \draw (9,17.25) to[short] (11.25,17.25);
    \draw [dashed] (9.25,17.75) -- (9.25,12);
    \draw [dashed] (9.25,17.75) -- (13.75,17.75);
    \draw [dashed] (13.75,17.75) -- (13.75,12);
    \draw [dashed] (9.25,12) -- (13.75,12);

    % Labels for R, Y, B
    \node [font=\large] at (1,18.75) {R};
    \node [font=\large] at (2,18.75) {Y};
    \node [font=\large] at (2.75,18.75) {B};
    \node [font=\large] at (7.25,17.5) {R'};
    \node [font=\large] at (7.25,16.5) {Y'};
    \node [font=\large] at (7.25,15.25) {B'};
    \node [font=\large] at (1.5,12.75) {Infinite Bus};
    \node [font=\large] at (11,11.5) {Incoming Generator};
\end{circuitikz}

}%


It was found that the lamps are becoming dark in the sequence $L_a -Lb -L_c$ . It means that the phase
sequence of incoming generator is
\hfill{(2014)}
                \begin{enumerate}
                
			\item opposite to infinite bus and its frequency is more than infinite bus
			\item opposite to infinite bus but its frequency is less than infinite bus
			\item same as infinite bus and its frequency is more than infinite bus
			\item same as infinite bus and its frequency is less than infinite bus
        	
         \end{enumerate}		
	\item  A distribution feeder of $1$ km length having resistance, but negligible reactance, is fed from both the
ends by $400V$, $50 Hz$ balanced sources. Both voltage sources $S_1$ and $S_2$ are in phase. The feeder
supplies concentrated loads of unity power factor as shown in the figure.

\centering
\resizebox{0.3\textwidth}{!}{%
\begin{circuitikz}
    \tikzstyle{every node}=[font=\small]
    
    % Horizontal line and voltage sources
    \draw (0.75,19.25) to[short] (9.25,19.25);
    \draw (0.75,19.25) to[sinusoidal voltage source, sources/symbol/rotate=auto] (0,19.25);
    \draw (9.25,19.25) to[sinusoidal voltage source, sources/symbol/rotate=auto] (10,19.25);
    
    % Vertical lines and arrow markers
    \draw (1.75,19.75) to[short] (1.75,18.75);
    \draw [->, >=Stealth] (4.75,19.25) -- (4.75,18.5);
    \draw [->, >=Stealth] (6.25,19.25) -- (6.25,18.5);
    \draw [->, >=Stealth] (7.75,19.25) -- (7.75,18.5);
    
    % Distance indicators
    \draw [<->, >=Stealth] (1.75,20.25) -- (4.75,20.25);
    \draw [<->, >=Stealth] (5,20.25) -- (6.25,20.25);
    \draw [<->, >=Stealth] (6.5,20.25) -- (7.75,20.25);
    \draw [<->, >=Stealth] (8,20.25) -- (9.25,20.25);
    
    % Additional vertical line
    \draw [short] (9,19.75) -- (9,18.75);
    
    % Voltage, frequency, and distances labels
    \node [font=\large] at (0.25,20.25) {S\_1};
    \node [font=\large] at (9.5,20.25) {S\_2};
    \node [font=\large] at (0.25,18.5) {400V};
    \node [font=\large] at (9.5,18.5) {400V};
    \node [font=\large] at (0,18) {50Hz};
    \node [font=\large] at (9.5,18) {50Hz};
    \node [font=\small] at (3,19.75) {400m};
    \node [font=\small] at (5.5,19.75) {200m};
    \node [font=\small] at (7,19.75) {200m};
    \node [font=\small] at (8.25,19.75) {200m};
    
    % Nodes and current labels
    \node at (4.75,19.25) [circ] {};
    \node at (6.25,19.25) [circ] {};
    \node at (7.75,19.25) [circ] {};
    \node [font=\small] at (4.75,18.25) {200A};
    \node [font=\small] at (6.25,18.25) {100A};
    \node [font=\small] at (7.75,18.25) {200A};
    
    % Small vertical lines at distances
    \draw (4.75,20.5) to[short] (4.75,20);
    \draw (6.25,20.5) to[short] (6.25,20);
    \draw (7.75,20.5) to[short] (7.75,20);
    
    % Label for point P
    \node [font=\small] at (6.25,19.5) {P};
\end{circuitikz}

}%


The contributions of $S_1$ and $S_2$ in $100 A$ current supplied at location P respectively, are
\hfill{(2014)}
		\begin{enumerate}
			\item $75 A$ and $25 A$
			\item $0 A$ and $50 A$
			\item $5 A$ and $75 A$
			\item$ 0A$ and $100A$
        	\end{enumerate}	
	\item A two bus power system shown in the figure supplies load of $1.0+j0.5$ p.u.
		
\centering
\resizebox{0.3\textwidth}{!}{%
\begin{circuitikz}
    \tikzstyle{every node}=[font=\normalsize]
    
    % Voltage source and inductance
    \draw (1,20.5) to[sinusoidal voltage source, sources/symbol/rotate=auto] (0.25,20.5);
    \draw (1,20.5) to[L] (7.25,20.5);
    
    % Bus 1 grounding and label
    \draw (1.5,21.25) to[short] (1.5,19.75);
    \draw (1.5,20) to[short] (2,20);
    \draw (2,20) to[L] (2,18);
    \draw (2,18) to (2,17.75) node[ground]{};
    
    % Bus 2 and arrow to show current direction
    \draw (7.25,21.25) to[short] (7.25,20);
    \draw (7.25,20.25) to[short] (7.75,20.25);
    \draw [->, >=Stealth] (7.75,20.25) -- (7.75,19);
    
    % Node labels
    \node [font=\large] at (0.5,21.25) {$G_1$};
    \node [font=\normalsize] at (1.5,22.5) {Bus 1};
    \node [font=\normalsize] at (1.5,22) {$V_1 < 0^\circ$};
    \node [font=\normalsize] at (2.5,19) {$j2$};
    \node [font=\normalsize] at (4.25,20) {$j0.1$};
    \node [font=\normalsize] at (7.25,22.75) {Bus 2};
    \node [font=\normalsize] at (7,22) {$1<\delta_2$};
    \node [font=\normalsize] at (8.5,20) {$1.0 + j0.5$};
\end{circuitikz}

}%


 The values of $V_1$ is p.u. and $\delta ^2$ respectively are
 \hfill{(2014)}
		\begin{enumerate}
			\item $0.95$ and $6.00^{\circ}$
			\item $1.05$ and $-5.44^{\circ}$
			\item $1.1$ and $-6.00^{\circ}$
			\item $1.1$ and $-27.12^{\circ}$
        	\end{enumerate}	
    \item The fuel cost functions of two power plants are 
    \begin{center}
          Plant $P_1$    $C_1=0.05Pg_1^2+APg_1+B$\\
         Plant $P_2$    $C_2=0.10Pg_2^2+3APg_2+B$\\
    \end{center}
    where. $P_{g1}$ and $P_{g2}$ are the generated powers of two plants, and A and B are the constants. If the two plants optimally share $1000 MW$ load at incremental fuel cost of $100Rs/MWh$, the ratio of load shared by plants$P_1$ and $P_2$ is
    \hfill{(2014)}
    
		\begin{enumerate}
			\item $1:4$
			\item $2:3$
			\item $3:2$
			\item $4:1$
        	\end{enumerate}	


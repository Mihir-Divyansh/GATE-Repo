\iffalse
\chapter{2014}
\author{Prajwal naik}
\section{ph}
\fi


 
%16
    \item  The donor concentration in a sample of n-type silicon is increased by a factor of 100. The shift in the position of the Fermi level at 300 K , assuming the sample to be non degenerate is $\quad$ meV .
$$
(k_{B} T=25 {meV} \text { at } 300 {~K})
$$\hfill(2014)
   
        
  \item   A particle of mass $m$ is subjected to a potential,
$$
V(x, y)=\frac{1}{2} m \omega^{2}(x^{2}+y^{2}),-\infty \leq x \leq \infty,-\infty \leq y \leq \infty
$$

The state with energy $4 \hbar \omega$ is $g$-fold degenerate. The value of $g$ is $\quad$ .\hfill{(2014)}
 
  
  \item A hydrogen atom is in the state
$\Psi=\sqrt{\frac{8}{21}} \psi_{200}-\sqrt{\frac{3}{7}} \psi_{310}+\sqrt{\frac{4}{21}} \psi_{321}$
where $n, l, m$ in $\psi_{n l m}$ denote the principal, orbital and magnetic quantum numbers, respectively. If $\vec{L}$ is the angular momentum operator, the average value of $L^{2}$ is $(\quad$) $\hbar^{2}$\hfill{(2014)}
  
  \item A planet of mass $m$ moves in a circular orbit of radius $r_{0}$ in the gravitational potential $V(r)=-\frac{k}{r}$, where $k$ is a positive constant. The orbital angular momentum of the planet is\hfill{(2014)}
  \begin{multicols}{4}
			\begin{enumerate}
   \item $2 r_{0} {~km}$
\item $\sqrt{2 r_{0} k m}$
\item $r_{0} {~km}$
\item  $\sqrt{r_{0} k m}$
\end{enumerate}
		\end{multicols}
   
%33
\item The moment of inertia of a rigid diatomic molecule $A$ is 6 times that of another rigid diatomic molecule $B$. If the rotational energies of the two molecules are equal, then the corresponding values of the rotational quantum numbers $J_{{A}}$ and $J_{{B}}$ are\hfill{(2014)}
   \begin{multicols}{4}
			\begin{enumerate}
   \item $J_{{A}}=2, J_{{B}}=1$
\item  $J_{{A}}=3, J_{{B}}=1$
\item$J_{{A}}=5, J_{{B}}=0$
\item  $J_{A}=6, J_{B}=1$
\end{enumerate}
		\end{multicols}
  \item The value of the integral
$$
\oint_{C} \frac{z^{2}}{e^{z}+1} d z
$$
where $C$ is the circle $|z|=4$, is\hfill{(2014)}
\begin{enumerate}
    \item $2 \pi i$
    \item $2 \pi^{2} i$
    \item $4 \pi^{2} i$
    \item $4 \pi^{2} i$
\end{enumerate}
  \item   A ray of light inside Region 1 in the $x y$-plane is incident at the semicircular boundary that carries no free charges. The electric field at the point $P\left(r_{0}, \pi / 4\right)$ in plane polar coordinates is $\vec{E}_{1}=7 \hat{e}_{r}-3 \hat{e}_{\varphi}$, where $\hat{e}_{r}$ and $\hat{e}_{\varphi}$ are the unit vectors. The emerging ray in Region 2 has the electric field $\vec{E}_{2}$ parallel to $x$ - axis. If $\varepsilon_{1}$ and $\varepsilon_{2}$ are the dielectric constants of Region 1 and Region 2 respectively, then $\frac{\varepsilon_{2}}{\varepsilon_{1}}$ is\hfill{(2014)}
  \begin{tikzpicture}

    % Draw the semicircle with radius r
    \def\radius{3}
    \draw[thick] (0, -\radius) arc[start angle=-90, end angle=90, radius=\radius];

    % Label the regions
    \node at (1.5, 0.8) {$\varepsilon_1$ \\ Region 1}; % Inner region label
    \node at (4, 1) {$\varepsilon_2$ \\ Region 2}; % Outer region label

    % Label point P on the circumference in the first quadrant
    \coordinate (P) at ({\radius*cos(45)}, {\radius*sin(45)});
    \fill (P) circle (2pt); % Mark the point with a dot
    \node[above right] at (P) {$P(r_0, \pi/4)$};

    % Draw the x and y axes
    \draw[->] (-0.5, 0) -- (4, 0) node[right] {$x$};
    \draw[->] (0, -3.5) -- (0, 3.5) node[above] {$y$};

\end{tikzpicture}


  
  
\item     The solution of the differential equation
$$
\frac{d^{2} y}{d t^{2}}-y=0
$$
subject to the boundary conditions $y(0)=1$ and $y(\infty)=0$, is\hfill{(2014)}
\begin{multicols}{4}
			\begin{enumerate}

\item $\cos t+\sin t$
\item $\cosh t+\sinh t$
\item $\cos t-\sin t$
\item  $\cosh t-\sinh t$
   \end{enumerate}
		\end{multicols}
  \item Given that the linear transformation of a generalized coordinate $q$ and the corresponding momentum $p$,

\begin{align*}
Q=q+4 a p \\
P=q+2 p
\end{align*}

is canonical, the value of the constant $a$ is $\quad$ .\hfill{(2014)}

  \item The value of the magnetic field required to maintain non-relativistic protons of energy 1 MeV in a circular orbit of radius 100 mm is $\quad$ Tesla.
(Given: $m_{p}=1.67 \times 10^{-27} {~kg}, e=1.6 \times 10^{-19} {C}$ )\hfill{(2014)}

\item  For a system of two bosons, each of which can occupy any of the two energy levels 0 and $\varepsilon$, the mean energy of the system at a temperature $T$ with $\beta=\frac{1}{k_{B} T}$ is given by\hfill{(2014)}
 \begin{multicols}{4}
			\begin{enumerate}
   
 
   \item$\frac{\varepsilon e^{-\beta \varepsilon}+2 \varepsilon e^{-2 \beta \varepsilon}}{1+2 e^{-\beta \varepsilon}+e^{-2 \beta \varepsilon}}$
\item  $\frac{1+\varepsilon e^{-\beta \varepsilon}}{2 e^{-\beta \varepsilon}+e^{-2 \beta \varepsilon}}$
\item  $\frac{2 \varepsilon e^{-\beta \varepsilon}+\varepsilon e^{-2 \beta \varepsilon}}{2+e^{-\beta \varepsilon}+e^{-2 \beta \varepsilon}}$
\item $\frac{\varepsilon e^{-\beta \varepsilon}+2 \varepsilon e^{-2 \beta \varepsilon}}{2+e^{-\beta \varepsilon}+e^{-2 \beta \varepsilon}}$
 \end{enumerate}
	\end{multicols}
  \item In an interference pattern formed by two coherent sources, the maximum and the minimum of the intensities are $9 I_{0}$ and $I_{0}$, respectively. The intensities of the individual waves are\hfill{(2014)}
  

\begin{multicols}{4}
			\begin{enumerate}
   \item $3 I_{0}$ and $I_{0}$
\item  $4 I_{0}$ and $I_{0}$
\item $5 I_{0}$ and $4 I_{0}$
\item $9 I_{0}$ and $I_{0}$
\end{enumerate}
		\end{multicols}
  \item  $\quad \psi_{1}$ and $\psi_{2}$ are two orthogonal states of a spin $\frac{1}{2}$ system. It is given that
$$
\psi_{1}=\frac{1}{\sqrt{3}}\myvec{{1}\\{0}}+\sqrt{\frac{2}{3}}\myvec{{0}\\{1}}
$$
where $\myvec{{1}\\{0}}$ and $\myvec{{0} \\{1}}$ represent the spin-up and spin-down states, respectively. When the system is in the state $\psi_{2}$, its probability to be in the spin-up state is $\quad$ .\hfill{(2014)}
 
			
		
 

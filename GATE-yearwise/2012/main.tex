
\iffalse
    \title{Assignment}
    \author{EE24BTECH11035}
    \section{ae}
    \chapter{2012}
  \fi
    \item The integration $ \int_0^1 x^3 \, dx $ computed using the trapezoidal rule with $ n = 4 $ intervals is.
    
    \item An aircraft has a steady rate of climb of $300 \, \frac{m}{s}$ at sea level and $150 \, \frac{m}{s}$ at $2500 \, \text{m}$ altitude. The time taken $\brak{in sec}$ for this aircraft to climb from $500 \, \text{m}$ altitude to $3000 \, \text{m}$ altitude is.
    
    \item An airfoil generates a lift of $80 \, \text{N}$ when operating in a freestream flow of $60 \, \frac{m}{s}$. If the ambient pressure and temperature are $100 \, \text{kPa}$ and $290 \, \text{K}$ respectively specific gas constant is $287 units$, the circulation on the airfoil in $ \text{m}^2/\text{s} $ is.
    
    \item A rocket motor has combustion chamber temperature of $2600 \, \text{K}$ and the products have molecular weight of $25 \, \frac{g}{mol}$ and ratio of specific heats $1.2$. The universal gas constant is $8314 \, \frac{J}{kg-mole-K}$. The value of theoretical $ c^* $ $\brak{in \frac{m}{s}}$ is.
    
    \item The mode shapes of an un-damped two degrees of freedom system are $ \begin{Bmatrix} 1 \\ 0.5 \end{Bmatrix}^T $ and $ \begin{Bmatrix} 1 \\ -0.675 \end{Bmatrix}^T $. The corresponding natural frequencies are $0.45 \, \text{Hz}$ and $1.2471 \, \text{Hz}$. The maximum amplitude $\brak{in mm}$ of vibration of the first degree of freedom due to an initial displacement of $ \begin{Bmatrix} 2 \\ 1 \end{Bmatrix}^T $ $\brak{in mm}$ and zero initial velocities is.
    
    \item The $n^{\text{th}}$ derivative of the function $ y = \frac{1}{x+3} $ is:
    \begin{enumerate}
        \item $ \frac{\brak{-1}^n n!}{\brak{x+3}^{n+1}} $
        \item $ \frac{\brak{-1}^{n+1} n!}{\brak{x+3}^{n+1}} $
        \item $ \frac{\brak{-1}^n \brak{n+1}!}{\brak{x+3}^n} $
        \item $ \frac{\brak{-1}^n n!}{\brak{x+3}^n} $
    \end{enumerate}

    \item The volume of a solid generated by rotating the region between semi-circle $ y = 1 - \sqrt{1 - x^2} $ and straight line $ y = 1 $, about $ x $-axis, is:
    \begin{enumerate}
        \item $ \pi^2 - \frac{4}{3} \pi $
        \item $ 4 \pi^2 - \frac{1}{3} \pi $
        \item $ \pi^2 - \frac{3}{4} \pi $
        \item $ \frac{3}{4} \pi^2 - \pi $
    \end{enumerate}

    \item One eigenvalue of the matrix $ A = \begin{bmatrix} 2 & 7 & 10 \\ 5 & 2 & 25 \\ 1 & 6 & 5 \end{bmatrix} $ is $ -9.33 $. One of the other eigenvalues is:
    \begin{enumerate}
        \item $ 18.33 $
        \item $ -18.33 $
        \item $ 18.33 - 9.33i $
        \item $ 18.33 + 9.33i $
    \end{enumerate}

    \item If an aircraft takes off with $10\%$ less fuel in comparison to its standard configuration, its range is:
    \begin{enumerate}
        \item Lower by exactly $10\%$.
        \item Lower by more than $10\%$.
        \item Lower by less than $10\%$.
        \item An unpredictable quantity.
    \end{enumerate}

    \item An aircraft has an approach speed of $144 \, \text{kmph}$ with a descent angle of $ 6.6^\circ $. If the aircraft load factor is $1.2$ and constant deceleration at touch down is $0.25g \, \brak{g = 9.81 \, \frac{m}{s}^2} $, its total landing distance approximately over a $15 \, \text{m}$ high obstacle is:
    \begin{enumerate}
        \item $1830 \, \text{m}$
        \item $1380 \, \text{m}$
        \item $830 \, \text{m}$
        \item $380 \, \text{m}$
    \end{enumerate}

    \item An oblique shock wave with a wave angle $ \beta $ is generated from a wedge angle of $ \theta $. The ratio of the Mach number downstream of the shock to its normal component is:
    \begin{enumerate}
        \item $ \sin\brak{\beta - \theta} $
        \item $ \cos\brak{\beta - \theta} $
        \item $ \sin\brak{\theta - \beta} $
        \item $ \cos\brak{\theta - \beta} $
    \end{enumerate}

    \item In a closed-circuit supersonic wind tunnel, the convergent-divergent $\brak{C-D}$ nozzle and test section are followed by a $C-D$diffuser to swallow the starting shock. Here, we should have the:
    \begin{enumerate}
        \item Diffuser throat larger than the nozzle throat and the shock located just at the diffuser throat.
        \item Diffuser throat larger than the nozzle throat and the shock located downstream of the diffuser throat.
        \item Diffuser throat of the same size as the nozzle throat and the shock located just at the diffuser throat.
        \item Diffuser throat of the same size as the nozzle throat and the shock located downstream of the diffuser throat.
    \end{enumerate}

    \item An aircraft is trimmed straight and level at true air speed $\brak{TAS}$ of $100 \frac{m}{s}$ at standard sea level $\brak{SSL}$. Further, pull of $5 N$ holds the speed at $90 \frac{m}{s}$ without re-trimming at $SSL$ air density = $1.22 kg \text{m}^3$. To fly at 3000 m altitude air density = $0.91 kg \text{m}^3$ and $120 \brak{\frac{m}{s}}$ $TAS$ without re-trimming, the aircraft needs:
    \begin{enumerate}
        \item $1.95 N$ upward force
        \item $1.95 N$ downward force
        \item $1.85 N$ upward force
        \item $1.75 N$ downward force
    \end{enumerate}
    
   






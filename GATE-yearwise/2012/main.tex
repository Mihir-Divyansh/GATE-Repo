
\iffalse
\chapter{2012}
\author{AI24BTECH11008}
\section{ae}
\fi

    \item The stagnation temperatures at the inlet and exit of a combustion chamber are 600 K and 1200 K,
    respectively. If the heating value of the fuel is $44 \frac{MJ}{kg}$ and specific heat at constant pressure for
    air and hot gases are $1.005\frac{kJ}{kg.K}$ and $1.147\frac{kJ}{kg.K}$ respectively, the fuel-to-air ratio is  \hfill (2012)
    \begin{enumerate}[label=(\Alph*)]
        \item 0.0018
        \item 0.018
        \item 0.18
        \item 1.18
    \end{enumerate}
    \item A solid propellant of density 1800 $\frac{kg}{m^3}$ has a burning rate law $r = 6.65 \times 10^{-3}p^{0.45} \frac{mm}{s}$, where p is pressure in Pascals. It is used in a rocket motor with a tubular grain with an initial burning area of $0.314 m^2$. The characteristic velocity is 1450 $\frac{m}{s}$. What should be the nozzle throat diameter to achieve an equilibrium chamber pressure of 50 bar at the end of the ignition transient?\hfill (2012)
    \begin{enumerate}[label=(\Alph*)]
        \item 35 mm
        \item 38 mm
        \item 41 mm
        \item 45 mm
    \end{enumerate}
    \item A bipropellant liquid rocket motor operates at a chamber pressure of 40 bar with a nozzle throat
    diameter of 50 mm. The characteristic velocity is 1540 $\frac{m}{s}$. If the fuel-oxidizer ratio of the
    propellant is 1.8, and the fuel density is 900 $\frac{kg}{m^3}$, what should be the minimum fuel tank volume
    for a burn time of 8 minutes \hfill (2012)
    \begin{enumerate}[label=(\Alph*)]
        \item $1.65m^3$
        \item $1.75m^3$
        \item $1.85m^3$
        \item $1.95m^3$
    \end{enumerate}
    \item The propellant in a single stage sounding rocket occupies 60\% of its initial mass. If all of it is
    expended instantaneously at an equivalent exhaust velocity of $3000\frac{m}{s}$, what would be the altitude
    attained by the payload when launched vertically?\\
    Neglect drag and assume acceleration due to gravity to be constant at 9.81 $\frac{m}{s^2}$. \hfill (2012)
    \begin{enumerate}[label=(\Alph*)]
        \item 315 km
        \item 335 km
        \item 365 km
        \item 385 km
    \end{enumerate}
    \item The Airy stress function, $\phi = \alpha x^2 + \beta xy + \gamma y^2$ for a thin square panel of size $l\times l$ automatically
    satisfies compatibility. If the panel is subjected to uniform tensile stress,$\sigma _0$ on all four edges, the
    traction boundary conditions are satisfied by \hfill (2012)
    \begin{enumerate}[label=(\Alph*)]
        \item $\alpha = \frac{\sigma_0}{2};\beta = 0;\gamma = \frac{\sigma_0}{2}$
        \item $\alpha = \sigma_0;\beta = 0;\gamma = \sigma_0$
        \item $\alpha = 0;\beta = \frac{\sigma_0}{4};\gamma = 0$
        \item $\alpha = 0;\beta = \frac{\sigma_0}{2};\gamma = 0$
    \end{enumerate}
    \item The boundary condition of a rod under longitudinal vibration is changed from fixed-fixed to fixedfree. The fundamental natural frequency of the rod is now k times the original frequency, where k is \hfill (2012)
    \begin{enumerate}[label=(\Alph*)]
        \item $\frac{1}{2}$
        \item 2
        \item $\frac{1}{\sqrt{2}}$
        \item $\sqrt{2}$
    \end{enumerate}
    \item A spring-mass system is viscously damped with a viscous damping constant c. The energy
    dissipated per cycle when the system is undergoing a harmonic vibration $x\cos w_0t$ is given by \hfill (2012)
    \begin{enumerate}[label=(\Alph*)]
        \item $\pi\omega_dcX^2$
        \item $\pi\omega_dX^2$
        \item $\pi\omega_dcX$
        \item $\pi\omega_d^2cX$
    \end{enumerate}
    \item[47.] Buckling of the fuselage skin can be delayed by \hfill (2012)
    \begin{enumerate}[label=(\Alph*)]
        \item  increasing internal pressure. 
        \item placing stiffeners farther apart. 
        \item  reducing skin thickness. 
        \item  placing stiffeners farther and decreasing internal pressure. 
    \end{enumerate} 
    \textbf{Common Data Questions}
    Common Data for Questions 48 and 49:
    A wing and tail are geometrically similar, while tail area is one-third of the wing area and distance between
two aerodynamic centres is equal to wing semi-span $\brak{\frac{b}{2}}$. In addition, following data is applicable: 
$\epsilon_{\alpha}=0.3,C_L = 1.0,C_{L_{\alpha}} = 0.08/deg, \bar{c}=2.5m, b=30m,C_{M_{ac}}=0, \eta_t = 1$. The symbols have their usual meanings.
    \item The maximum distance that the centre of gravity can be behind aerodynamic centre without
    destabilizing the wing-tail combination is     \hfill (2012)
    \begin{enumerate}[label=(\Alph*)]
        \item 0.4m
        \item 1.4m
        \item 2.4m
        \item 3.4m
    \end{enumerate}
    \item The angle of incidence of tail to trim the wing-tail combination for a 5\% static margin is\hfill (2012) 
     \begin{enumerate}[label=(\Alph*)]
        \item $-1.4^{\circ}$
        \item $-0.4^{\circ}$
        \item $0.4^{\circ}$
        \item $1.4^{\circ}$
     \end{enumerate}
     \textbf{Common Data for Questions 50 and 51:}\\
     A thin long circular pipe of 10 mm diameter has porous walls and spins at 60 rpm about its own axis. Fluid
     is pumped out of the pipe such that it emerges radially relative to the pipe surface at a velocity of 1 $\frac{m}{s}$.
     [Neglect the effect of gravity.] 
    \item What is the radial component of the fluid's velocity at a radial location 0.5 m from the pipe axis? \hfill (2012)
    \begin{enumerate}[label=(\Alph*)]
        \item $0.01\frac{m}{s}$
        \item $0.1\frac{m}{s}$
        \item $1\frac{m}{s}$
        \item $10\frac{m}{s}$
    \end{enumerate}
    \item What is the tangential component of the fluid's velocity at the same radial location as above? \hfill (2012)
    \begin{enumerate}[label=(\Alph*)]
        \item $0.01\frac{m}{s}$
        \item $0.03\frac{m}{s}$
        \item $0.10\frac{m}{s}$
        \item $0.31\frac{m}{s}$
    \end{enumerate}
    \textbf{Linked Answer Questions 52 and 53}\\
    \textbf{Statement for Linked Answer Questions 52 and 53:}\\
    Air at a stagnation temperature of $15^{\circ}$C and stagnation pressure 100 kPa enters an axial compressor with an
absolute velocity of $120\frac{m}{s}$. Inlet guide vanes direct this absolute velocity to the rotor inlet at an angle of
$18^{\circ}$ to the axial direction. The rotor turning angle is $27^{\circ}$ and the mean blade speed is 200 $\frac{m}{s}$. The axial
velocity is assumed constant through the stage.
    \item The blade angle at the inlet of the rotor is \hfill (2012)
    \begin{enumerate}[label=(\Alph*)]
        \item $25.5^{\circ}$ 
        \item $38.5^{\circ}$ 
        \item $48.5^{\circ}$ 
        \item $59.5^{\circ}$ 
    \end{enumerate} 


\iffalse
\chapter{2018}
\author{AI24BTECH11027}
\section{ee}
\fi

\item Let $f(x)=3X^3-7x^2+5x+6.$ The maximum value of $f(x)$ over the interval $\sbrak{0,2}$ is $\underline{\hspace{2cm}}$ (up to $1$ decimal place).\\

\item Let $A=\myvec{1 & 0 & -1\\ -1 & 2 & 0\\ 0 & 0 & -2}$ and $B=A^3-A^2-4A+5I$, where I is the $3\times3$ identity determinant of $B$ is $\underline{\hspace{2cm}}$ (up to $1$ decimal place).\\

\item The capacitance of an air-filled parallel-plate capacitor is $60 pF.$ When a dielectric slab whose thickness is half the distance between the plates, is placed on one of the plates covering it entirely, the capacitance becomes $86 pF.$ Neglecting the fringing effects, the relative permittivity of the dielectric is $\underline{\hspace{2cm}}$  (up to $2$ decimal place).\\ 

\item The unit step response $y(t)$ of a unity feedback system with open loop transfer function $G(s)H(s)=\frac{K}{(s+1)^2(s+2)}$ is shown in the figure. The value of $k$ is $\underline{\hspace{2cm}}$  (up to $2$ decimal place).

\begin{figure}[H]
\centering
\resizebox{0.8\textwidth}{!}{%
\begin{circuitikz}
\tikzstyle{every node}=[font=\LARGE]
\draw [short] (-14.75,29) -- (-14.75,-7.25);
\draw [short] (-14.75,-7.25) -- (37.75,-7.25);
\draw [short] (-9.5,-7.25) -- (-9.5,-6.75);
\draw [short] (-4.25,-7.25) -- (-4.25,-6.75);
\draw [short] (1,-7.25) -- (1,-6.75);
\draw [short] (6.25,-7.25) -- (6.25,-6.75);
\draw [short] (11.5,-7.25) -- (11.5,-6.75);
\draw [short] (16.75,-7.25) -- (16.75,-6.75);
\draw [short] (22,-7.25) -- (22,-6.75);
\draw [short] (27.25,-7.25) -- (27.25,-6.75);
\draw [short] (32.5,-7.25) -- (32.5,-6.75);
\draw [short] (37.75,-7.25) -- (37.75,-6.75);
\draw [short] (-14.75,-2) -- (-14,-2);
\draw [short] (-14.75,3.25) -- (-14,3.25);
\draw [short] (-14.75,8.5) -- (-14,8.5);
\draw [short] (-14.75,13.5) -- (-14,13.5);
\draw [short] (-14.75,18.75) -- (-14,18.75);
\draw [short] (-14.75,24) -- (-14,24);
\draw [short] (-14.75,29) -- (-14,29);
\node [font=\LARGE] at (-15.25,-6.75) {$0$};
\node [font=\LARGE] at (-14.25,-8.25) {$0$};
\node [font=\LARGE] at (-15.75,-2) {$0.2$};
\node [font=\LARGE] at (-15.5,3.25) {$0.4$};
\node [font=\LARGE] at (-15.5,8.75) {$0.6$};
\node [font=\LARGE] at (-15.25,13.5) {$0.8$};
\node [font=\LARGE] at (-15.25,18.75) {$1$};
\node [font=\LARGE] at (-15.75,24) {$1.2$};
\node [font=\LARGE] at (-15.5,28.75) {$1.4$};
\node [font=\LARGE] at (-9.5,-8) {$2$};
\node [font=\LARGE] at (-4.25,-8.25) {$4$};
\node [font=\LARGE] at (1,-8.25) {$6$};
\node [font=\LARGE] at (6.25,-8) {$8$};
\node [font=\LARGE] at (11.5,-8) {$10$};
\node [font=\LARGE] at (17,-8) {$12$};
\node [font=\LARGE] at (22,-8) {$14$};
\node [font=\LARGE] at (27,-8) {$16$};
\node [font=\LARGE] at (93.25,-16.25) {$18$};
\node [font=\LARGE] at (37.5,-8.25) {$20$};
\node [font=\LARGE] at (35.5,-9.75) {$t(sec)$};
\node [font=\LARGE] at (-19,27.5) {$y(t)$};
\draw [dashed] (-14.5,13.5) -- (37,13.5);
\draw [line width=0.6pt, short] (-14.75,-7.25) .. controls (-7,2) and (-14.5,38.75) .. (-5,9.75);
\draw [line width=0.6pt, short] (-5,9.75) .. controls (-4,7.75) and (-2.25,4.25) .. (-0.25,14.25);
\draw [line width=0.6pt, short] (-0.25,14.25) .. controls (4.5,26.5) and (4.25,3.5) .. (10.75,14.75);
\draw [line width=0.6pt, short] (10.75,14.75) .. controls (11.75,17.25) and (12.75,9.75) .. (16.75,14.25);
\draw [line width=0.6pt, short] (16.75,14.25) .. controls (18.25,15.75) and (18.75,10.5) .. (21.75,14.25);
\end{circuitikz}
}%

\label{fig:my_label}
\end{figure}

\item A three-phased load is connected to a three-phase balanced supply as shown in the figure. If $v_{an}=100 \angle{0}\degree V,b_{bn}=100\angle{-120}\degree V$ and $V_{cn}=100\angle{-240}\degree V$ (angle are considered positive in the anti-clockwise direction), the value of $R$ for zero current in the neutral wire is $\underline{\hspace{2cm}}\Omega$ (up to $2$ decimal places).\\
\begin{figure}[H]
\centering
\resizebox{0.3\textwidth}{!}{%
\begin{circuitikz}
\tikzstyle{every node}=[font=\LARGE]
\draw (-23.5,-37.5) to[short, -o] (-25,-37.5) ;
\draw (-23.75,-35) to[short, -o] (-25,-35) ;
\draw (-23.75,-32.5) to[short, -o] (-25,-32.5) ;
\draw (-23.75,-30) to[short, -o] (-25,-30) ;
\draw (-23.75,-30) to[short] (-17.5,-30);
\draw (-23.5,-37.5) to[short] (-13.75,-37.5);
\draw (-13.75,-37.5) to[short] (-13.75,-36.25);
\draw (-23.75,-35) to[short] (-20,-35);
\draw (-20,-35) to[short] (-20,-36.25);
\draw (-20,-36.25) to[short] (-19,-35.25);
\draw (-18.25,-34.5) to[short] (-17,-33.25);
\draw (-17.25,-33.5) to[short] (-16.75,-33);
\draw (-17.5,-32.5) to[short] (-24,-32.5);
\draw (-16.75,-33) to[short] (-16.75,-31.75);
\draw (-17.5,-30) to[short] (-16.75,-30);
\draw (-16.75,-30) to[short] (-16.75,-30.75);
\draw (-18,-34.25) to[C] (-19.5,-35.75);
\draw (-16.75,-30.5) to[R] (-16.75,-32);
\node at (-16.75,-33) [circ] {};
\draw (-17.25,-32.5) to[short] (-17.75,-32.5);
\draw (-15.75,-34) to[L ] (-14,-35.75);
\draw (-17.25,-32.5) to[short] (-15.5,-34.25);
\draw (-14.25,-35.5) to[short] (-13.75,-36);
\draw (-13.75,-36) to[short] (-13.75,-36.75);
\draw [->, >=Stealth] (-23.75,-30) -- (-22.5,-30);
\draw [->, >=Stealth] (-21.5,-32.5) -- (-22.5,-32.5);
\draw [->, >=Stealth] (-23.5,-35) -- (-22.5,-35);
\draw [->, >=Stealth] (-23.75,-37.5) -- (-22.5,-37.5);
\node [font=\LARGE] at (-25.75,-30) {$a$};
\node [font=\LARGE] at (-25.5,-32.5) {$n$};
\node [font=\LARGE] at (-25.5,-35) {$c$};
\node [font=\LARGE] at (-25.75,-37.5) {$b$};
\node [font=\LARGE] at (-16,-31.5) {$R$};
\node [font=\LARGE] at (-14,-34) {$j10$};
\node [font=\LARGE] at (-17.75,-35.5) {$-j10$};
\end{circuitikz}
}%

\label{fig:my_label}
\end{figure}

\item The voltage across the circuit in the figure, and the current through it, are given by the following expressions: \\
$v(t)=5-10 \cos{\omega t+60\degree}V$\\
$i(t)=5+X \cos{\omega t}A$\\
where $\omega=100\pi$ radian/s. If the average power delivered to the circuit is zero, then the value of X (in Ampere) if $\underline{\hspace{2cm}}$  (up to $2$ decimal place).\\
\begin{figure}[H]
\centering
\resizebox{0.2\textwidth}{!}{%
\begin{circuitikz}
\tikzstyle{every node}=[font=\LARGE]
\draw [->, >=Stealth] (0,8.75) -- (2.25,8.75);
\draw  (2.5,7.5) rectangle (5,5);
\draw [->, >=Stealth] (0,3.75) -- (2.25,3.75);
\draw (2.25,8.75) to[short] (3.75,8.75);
\draw (3.75,8.75) to[short] (3.75,7.5);
\draw (2.25,3.75) to[short] (3.75,3.75);
\draw (3.75,3.75) to[short] (3.75,4.75);
\draw (3.75,4.5) to[short] (3.75,5);
\node at (0,3.75) [circ] {};
\node at (0,8.75) [circ] {};
\node [font=\LARGE] at (3.75,6.5) {$Electrical$};
\node [font=\LARGE] at (3.75,5.75) {$Circuit$};
\node [font=\LARGE] at (1.75,8.25) {$+$};
\node [font=\LARGE] at (1.75,3.25) {$-$};
\node [font=\LARGE] at (1.5,9.5) {$i(t)$};
\node [font=\LARGE] at (1.5,6.25) {$v(t)$};
\end{circuitikz}
}%

\label{fig:m}
\end{figure}

\item A phase controlled single rectifier, supplied by an AC source, feeds power to an R-L-E load as shown in the figure. The rectifier output  voltage has an average value gven by$v_0=\frac{v_m}{2\pi}(3+\cos{\alpha})$, where $v_m=80\pi$ volts and $\alpha$ is the firing angle. If the power delivered to the lossless battery is $1600 W, \alpha$ in degree is $\underline{\hspace{2cm}}$  (up to $2$ decimal place).\\
\begin{figure}[H]
\centering
\resizebox{0.4\textwidth}{!}{%
\begin{circuitikz}
\tikzstyle{every node}=[font=\LARGE]
\draw (-2.5,16.25) to[sinusoidal voltage source, sources/symbol/rotate=auto] (-2.5,10);
\draw (-2.5,16.25) to[short] (-1.25,16.25);
\draw (-2.5,10) to[short] (-1.25,10);
\draw (-1.25,17.5) to[short] (1.25,17.5);
\draw (-1.25,17.5) to[short] (-1.25,8.75);
\draw (-1.25,8.75) to[short] (1.25,8.75);
\draw (1.25,8.75) to[short] (1.25,17.5);
\draw (1.25,16.25) to[short, -o] (2.25,16.25) ;
\draw (1.25,10) to[short, -o] (2.25,10) ;
\draw (2.25,16.25) to[short] (3.75,16.25);
\draw (2.25,10) to[short] (3.75,10);
\draw (3.75,16.25) to[R] (3.75,13.75);
\draw (3.75,10) to[battery1] (3.75,11.25);
\draw (3.75,14) to[L ] (3.75,11.25);
\draw (0,12.25) node[ieeestd buffer port, anchor=in, rotate=-270](port){} (port.out) to[short] (0,14.75);
\draw (port.in) to[short] (0,11.25);
\draw [short] (-1,13.5) -- (1,13.5);
\draw (0,13.5) to[short, -o] (0.5,14) ;
\draw [->, >=Stealth] (2.5,11.25) -- (2.5,15.25);
\node [font=\LARGE] at (-4.75,13.25) {$V_m sin(\omega t$)};
\node [font=\LARGE] at (2,13.75) {$v_0$};
\node [font=\LARGE] at (4,10.75) {$+$};
\node [font=\LARGE] at (2.5,9.75) {$-$};
\node [font=\LARGE] at (4.5,15.25) {$2\Omega$};
\node [font=\LARGE] at (5,12.5) {$10mH$};
\node [font=\LARGE] at (4.75,10.75) {$80 V$};
\node [font=\LARGE] at (4.75,10) {$battery$};
\node [font=\LARGE] at (4,10.5) {$-$};
\node [font=\LARGE] at (2.5,16.5) {$+$};
\end{circuitikz}
}%

\label{fig:my_label}
\end{figure}

\item The load resistance is $1 \Omega.$ The capacitor voltage has negligible ripple. Both converters operate in the continuous conduction mode. The switching frequency is $1 kHz$, and the switch control signals arc as shown. The circuit operates in the steady state. Assuming that the converters share the load equally, the average value $i_S1$ , the current of switch $S1$ (in Ampere), is $\underline{\hspace{2cm}}$  (up to $2$ decimal place).\\
\begin{figure}[H]
\centering
\resizebox{0.7\textwidth}{!}{%
\begin{circuitikz}
\tikzstyle{every node}=[font=\LARGE]
\draw (-7.5,14.75) to[short] (-7.5,10);
\draw (-7.5,10) to[short] (-2,10);
\draw (-7.5,16.25) to[short] (-7.5,18.75);
\draw (-5,18.75) to[short] (-5,13.25);
\draw (-7.5,18.75) to[short] (-5,18.75);
\draw (-5.25,18.75) to[short] (-3.75,18.75);
\draw (-2.5,18.75) to[short] (-1.25,18.75);
\draw (-7.5,13.25) to[short] (3.5,13.25);
\draw (-2,15.25) node[ieeestd buffer port, anchor=in, rotate=-270](port){} (port.out) to[short] (-2,18.75);
\draw (port.in) to[short] (-2,13.25);
\draw (-1.25,18.75) to[L ] (3.5,18.75);
\draw (3.5,18.75) to[R] (3.5,13.25);
\draw (2.5,18.75) to[C] (2.5,13.25);
\draw (1.75,18.75) to[short] (1.75,12);
\draw (1.75,12) to[L ] (-2.75,12);
\draw (-2,10.25) node[ieeestd buffer port, anchor=in, rotate=-270](port){} (port.out) to[short] (-2,12);
\draw (port.in) to[short] (-2,10);
\draw (-5,13.5) to[short] (-5,12);
\draw (-5,12) to[short] (-4.5,12);
\draw (-4.5,12) to[short] (-3.75,12.75);
\draw (-2.75,12) to[short] (-3.5,12);
\draw (-3.75,18.75) to[short] (-3,19.5);
\draw (-2.25,16.5) to[short] (-1.75,16.5);
\draw (-2.25,11.5) to[short] (-1.75,11.5);
\draw  (-7.5,15.5) circle (0.75cm);
\node at (-7.5,13.25) [circ] {};
\node at (-4.5,12) [circ] {};
\node at (-3.5,12) [circ] {};
\node at (2.5,13.25) [circ] {};
\node at (2.5,18.75) [circ] {};
\node at (1.75,18.75) [circ] {};
\node at (-2,18.75) [circ] {};
\node at (-2.5,18.75) [circ] {};
\node at (-3,19.5) [circ] {};
\node at (-3.75,18.75) [circ] {};
\node at (-5,18.75) [circ] {};
\draw [dashed] (6,13.5) -- (6,13.5);
\draw [dashed] (5,13.75) -- (15,13.75);
\draw [dashed] (5.75,16.25) -- (5.75,10.25);
\draw [dashed] (5,11.25) -- (14.75,11.25);
\draw (5,13.75) to[short] (5.75,13.75);
\draw (5.75,13.75) to[short] (5.75,15.75);
\draw (5.75,15.75) to[short] (8.75,15.75);
\draw (8.75,15.75) to[short] (8.75,13.75);
\draw (8.75,13.75) to[short] (11.25,13.75);
\draw (11.25,13.75) to[short] (11.25,15.75);
\draw (11.25,15.75) to[short] (13.75,15.75);
\draw (13.75,15.75) to[short] (13.75,13.75);
\draw (5,13) to[short] (5.75,13);
\draw (5.75,13) to[short] (5.75,11.25);
\draw (5.75,11.25) to[short] (8.75,11.25);
\draw (8.75,11.25) to[short] (8.75,13);
\draw (8.75,13) to[short] (11.25,13);
\draw (11.25,13) to[short] (11.25,11.25);
\draw (11.25,11.25) to[short] (13.5,11.25);
\draw (13.75,11.25) to[short] (13.75,13);
\draw (13.75,11.25) to[short] (12.75,11.25);
\draw (13.75,13) to[short] (14.75,13);
\draw [->, >=Stealth] (13.75,13.75) -- (14.75,13.75);
\draw [->, >=Stealth] (14.25,11.25) -- (14.75,11.25);
\node [font=\LARGE] at (-9,15.5) {$100v$};
\node [font=\LARGE] at (-4.5,19.75) {$i_{S1}$};
\node [font=\LARGE] at (-0.5,11.25) {$L$};
\node [font=\LARGE] at (2.25,15.5) {$C$};
\node [font=\LARGE] at (4.25,16.25) {1$\Omega$};
\node [font=\LARGE] at (-3,18) {$S1$};
\node [font=\LARGE] at (-3.75,11.25) {$S2$};
\node [font=\LARGE] at (1,18) {L};
\node [font=\LARGE] at (5.25,12.25) {$S1$};
\node [font=\LARGE] at (5.25,14) {$S2$};
\node [font=\LARGE] at (5.25,10.25) {$0$};
\node [font=\LARGE] at (8.5,10.5) {$0.5 ms$};
\node [font=\LARGE] at (11.25,10.5) {$1 ms$};
\node [font=\LARGE] at (9.5,16.5) {Switch control signals};
\node [font=\LARGE] at (15.25,13.5) {$t$};
\node [font=\LARGE] at (15.25,11) {$t$};
\draw [->, >=Stealth] (-5,19.25) -- (-3.75,19.25);
\end{circuitikz}
}%

\label{fig:my_label}
\end{figure}

\item A $3$-phase $900$ kVA, $3$ $kV / \sqrt{3} $k V $(\Delta/Y)$ $50$ Hz transformer has primary (high voltage side) resistance per phase of $0.3 \Omega$ and secondary (low voltage side ) resistance per phase of $0.02 \Omega$. Iron loss of the transformer is $10$ kW. The full load \% efficiency of the transformer operated at unity power factor is $\underline{\hspace{2cm}}$  (up to $2$ decimal place).\\

\item A $200$ V DC series motor, when operating from voltage while driving a certain load, draws $10 A$ current and runs $1000$ r.p.m. The total series resistance is $1 \Omega$. The magnetic circuit is assumed to be linear. At the same voltage, the load torque is increased by $44\%$. The speed of the motor in r.p.m. (rounded to the nearest integer) is $\underline{\hspace{2cm}}$ \\

\item A dc to dc converter shown in the figure is charging a batery bank, B$2$ whose voltage is constant at $150$ V. B$1$ is another battery bank whose voltage is constant at $50$ V. The value of the inductor, L is $5$mH and the ideal switch, S is operated with a switching frequency of $5$ kHz with a duty ratio of $0.4.$ Once the circuit has attained steady state and assuming the diode D to be ideal, The power transferred from B$1$ to B$2$
(in watt) is $\underline{\hspace{2cm}}$  (up to $2$ decimal place).\\
\begin{figure}[H]
\centering
\resizebox{0.4\textwidth}{!}{%
\begin{circuitikz}
\tikzstyle{every node}=[font=\LARGE]
\draw (2.5,13.75) to[battery1] (2.5,8.75);
\draw (2.5,13.75) to[L ] (8.75,13.75);
\draw (9.875,13.75) node[ieeestd buffer port, anchor=in](port){} (port.out) to[short] (12.5,13.75);
\draw (port.in) to[short] (8.75,13.75);
\draw (12.5,13.75) to[battery1] (12.5,8.75);
\draw [->, >=Stealth] (2.5,13.75) -- (3.75,13.75);
\node [font=\LARGE] at (2.75,11.5) {$+$};
\node [font=\LARGE] at (2.75,11) {$-$};
\node [font=\LARGE] at (1.5,10.5) {$50 V$};
\node [font=\LARGE] at (3.5,11.25) {$B1$};
\node [font=\LARGE] at (3.25,14.25) {$i_L$};
\node [font=\LARGE] at (6,14.75) {$L=5 mH$};
\node [font=\LARGE] at (7,11.5) {$S$};
\node [font=\LARGE] at (10.5,14.75) {$D$};
\node [font=\LARGE] at (11.75,11.25) {$B2$};
\node [font=\LARGE] at (13,11.5) {$+$};
\node [font=\LARGE] at (12.75,10.75) {$-$};
\node [font=\LARGE] at (14,11.25) {$150 V$};
\draw (11,14.25) to[short] (11,13.25);
\draw (7.5,13.75) to[short] (7.5,11.75);
\draw (7.5,11.75) to[short] (8.25,11);
\draw (7.5,10.5) to[short] (7.5,8.75);
\draw (2.5,8.75) to[short] (12.5,8.75);
\end{circuitikz}
}%

\label{fig:my_label}
\end{figure}

\item The equivalent circuit of a single phase induction motor is sown in the figure, where the parameters are $R_1=R_2^1=X_{l1}^1=X_{l2}^1=12\Omega, X_M=240\Omega$ and $S$ is the slip. At no-load, the motor speed can be approximated to be the synchronous speed. The no-load lagging power factor of the motor is $\underline{\hspace{2cm}}$  (up to $3$ decimal place).\\
\begin{figure}[H]
\centering
\resizebox{0.4\textwidth}{!}{%
\begin{circuitikz}
\tikzstyle{every node}=[font=\LARGE]
\node at (0,17.5) [circ] {};
\draw (0,17.5) to[R] (2.5,17.5);
\draw (2.5,17.5) to[L ] (5,17.5);
\draw (5,17.5) to[short] (5,16.75);
\draw (4.25,16.75) to[short] (5.75,16.75);
\draw (5.75,16.75) to[R] (5.75,14.25);
\draw (5.75,14.25) to[L ] (5.75,12);
\draw (4.25,16.75) to[L ] (4.25,12);
\draw (4.25,12) to[short] (5.75,12);
\draw (5,12) to[short] (5,11.25);
\draw (4.25,11.25) to[short] (5.75,11.25);
\draw (5.75,11.25) to[R] (5.75,8.75);
\draw (5.75,8.75) to[L ] (5.75,6.5);
\draw (4.25,11.25) to[L ] (4.25,6.5);
\draw (4.25,6.5) to[short] (5.75,6.5);
\draw (5,6.5) to[short] (5,5.5);
\draw (5,5.5) to[short] (0,5.5);
\node at (0,5.5) [circ] {};
\draw [->, >=Stealth] (0,12.25) -- (0,17);
\draw [->, >=Stealth] (0,11.25) -- (0,6.25);
\node [font=\LARGE] at (1.25,11.75) {$V\angle 0\degree$};
\node [font=\LARGE] at (1.25,18.25) {$R_1$};
\node [font=\LARGE] at (3.75,18.5) {$jX_{l1}$};
\node [font=\LARGE] at (3.25,14.5) {$j \frac{X_M}{2}$};
\node [font=\LARGE] at (7.5,13.25) {$j\frac {X_l2^1}{2}$};
\node [font=\LARGE] at (3.25,8.75) {$j \frac{X_M}{2}$};
\node [font=\LARGE] at (7.75,10) {$\frac{R_2^1}{2(2-s)}$};
\node [font=\LARGE] at (7.75,8) {$j \frac{X_{l2}^1}{2}$};
\node [font=\LARGE] at (7.25,15.5) {$\frac{R_2^1}{2s}$};
\end{circuitikz}
}%

\label{fig:my_label}
\end{figure}

\item The voltage $v(t)$ across the terminals $a$ and $b$ as shown in the figure, is a sinusoidal voltage having a frequency $\omega=100$ radian/s. when the inductor current $i(t)$ is in phase with the voltage $v(t)$, the magnitude of the impedance $Z$(in $\Omega$) seen between the terminals $a$ and $b$ is $\underline{\hspace{2cm}}$  (up to $2$ decimal place).\\
\begin{figure}[H]
\centering
\resizebox{0.4\textwidth}{!}{%
\begin{circuitikz}
\tikzstyle{every node}=[font=\LARGE]
\draw (-1.25,17.5) to[L ] (5,17.5);
\draw (5,17.5) to[R] (5,15);
\draw (3.25,17.5) to[C] (3.25,15);
\draw (-1.25,15) to[short] (5,15);
\draw [->, >=Stealth] (1.25,18.25) -- (2.25,18.25);
\draw [->, >=Stealth] (-1.25,16.25) -- (-0.25,16.25);
\node at (-1.25,17.5) [circ] {};
\node at (-1.25,15) [circ] {};
\node at (3.25,15) [circ] {};
\node at (3.25,17.5) [circ] {};
\node [font=\LARGE] at (-2,17.5) {$v(t)$};
\node [font=\LARGE] at (-1.75,15) {$b$};
\node [font=\LARGE] at (0.75,18) {$i(t)$};
\node [font=\LARGE] at (2,15.75) {$100 \mu F$};
\node [font=\LARGE] at (6.25,16) {$100\Omega$};
\node [font=\LARGE] at (-1.25,14.75) {$-$};
\node [font=\LARGE] at (-1.25,17.75) {$+$};
\node [font=\LARGE] at (1.75,16.75) {$L$};
\end{circuitikz}
}%

\label{fig:my_label}
\end{figure}
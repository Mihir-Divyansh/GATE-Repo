\iffalse
\chapter{2013}
\author{EE24BTECH11003}
\section{ph}
\fi
\item $f\brak{x}$ is a symmetric periodic function of $x$ i.e. $f\brak{x} = f\brak{-x}$. Then, in general, the Fourier series of the function f(x) will be of the form
\hfill{\brak{2013}}
\begin{enumerate}
\item $f\brak{x} = \sum_{n=1}^{\infty} \brak{a_n \cos\brak{nkx} + b_n\sin\brak{nkx}}$
\item $f\brak{x} = a_0 + \sum_{n=1}^{\infty} \brak{a_n \cos\brak{nkx}}$
\item $f\brak{x} = \sum_{n=1}^{\infty} \brak{b_n\sin\brak{nkx}}$
\item $f\brak{x} = a_0 + \sum_{n=1}^{\infty} \brak{b_n\sin\brak{nkx}}$
\end{enumerate}

\item In the most general case, which one of the following quantities is NOT a second order tensor?
\hfill{\brak{2013}}
\begin{enumerate}
\item Stress
\item Strain
\item Moment of inertia
\item Pressure
\end{enumerate}

\item An electron is moving with a velocity of $0.85c$ in the same direction as that of a moving photo. The relative velocity of the electron with respect to photon is
\hfill{\brak{2013}}
\begin{enumerate}
\item $c$
\item $-c$
\item $0.15c$
\item $-0.15c$
\end{enumerate}

\item If Planck's constant were zero, then the total energy contained in a box filled with radiation of all frequencies at temperature T would be $\brak{\text{k is the Boltzmann constant and T is nonzero}}$
\hfill{\brak{2013}}
\begin{enumerate}
\item Zero
\item Infinite
\item $\frac{3}{2}kT$
\item $kT$
\end{enumerate}

\item Across a first order phase transition, the free energy is
\hfill{\brak{2013}}
\begin{enumerate}
\item proportional to the temperature
\item a discontinuous function of the temperature
\item a continuous function of the temperature but its first derivative is discontinuous
\item such that the first derivative with respect to temperature is continuous
\end{enumerate}

\item Two gases separated by an impermeable but movable partition are allowed to freely exchange energy. At equilibrium, the two sides will have the same
\hfill{\brak{2013}}
\begin{enumerate}
\item pressure and temperature
\item volume and temperature
\item pressure and volume
\item volume and energy
\end{enumerate}

\item The entropy function of a system is given by $S\brak{E} = aE\brak{E_0 - E}$ where $a$ and $E_0$ are positive constants. The temperature of the system is
\hfill{\brak{2013}}
\begin{enumerate}
\item negative for some energies 
\item increases monotonically with energy
\item decreases monotonically with energy
\item Zero
\end{enumerate}

\item Consider a linear collection of $N$ independent spin $\frac{1}{2}$ particles, each at a fixed location. The entropy of this system is $\brak{\text{k is the Boltzmann constant}}$
\hfill{\brak{2013}}
\begin{enumerate}
\item Zero
\item $Nk$
\item $\frac{1}{2}Nk$
\item $Nk\ln\brak{2}$
\end{enumerate}

\item The decay process $n \to p^+ + e^- + \bar{\nu_e}$ violates
\hfill{\brak{2013}}
\begin{enumerate}
\item baryon number
\item lepton number
\item isospin
\item strangeness
\end{enumerate}

\item The isospin $\brak{I}$ and baryon number $\brak{B}$ of the up quark is
\hfill{\brak{2013}}
\begin{enumerate}
\item $I=1$, $B=1$
\item $I=1$, $B=\frac{1}{3}$
\item $I=\frac{1}{2}$, $B=1$
\item $I=\frac{1}{2}$, $B=\frac{1}{3}$
\end{enumerate}

\item Consider the scattering of neutrons by protons at very low energy due to a nuclear potential of range $r_0$. Given that,
\begin{align*}
\cot\brak{kr_0 + \delta} \approx -\frac{\gamma}{k}
\end{align*}
where $\delta$ is the phase shift, $k$ the wave number and $\brak{-\gamma}$ the logarithmic derivative of the deuteron ground state wave function, the phase shift is
\hfill{\brak{2013}}
\begin{enumerate}
\item $\delta \approx -\frac{k}{\gamma} - kr_0$
\item $\delta \approx -\frac{\gamma}{k} - kr_0$
\item $\delta \approx \frac{\pi}{2} - kr_0$
\item $\delta \approx -\frac{\pi}{2} - kr_0$
\end{enumerate}

\item In the $\beta$ decay process, the transition $2^+ \to 3^+$, is
\hfill{\brak{2013}}
\begin{enumerate}
\item allowed both by Fermi and Gamow-Teller selection rule
\item allowed by Fermi and but not by Gamow-Teller selection rule
\item not allowed by Fermi but allowed by Gamow-Teller selection rule
\item not allowed both by Fermi and Gamow-Teller selection rule
\end{enumerate}

\item At a surface current, which one of the magnetostatic boundary condition is NOT CORRECT?
\hfill{\brak{2013}}
\begin{enumerate}
\item Normal component of the magnetic field is continuous.
\item Normal component of the magnetic vector potential is continuous.
\item Tangential component of the magnetic vector potential is continuous.
\item Tangential component of the magnetic vector potential is not continuous.
\end{enumerate}



\iffalse
\title{2017-XE-53-65}
\author{EE24BTECH11010 - BALAJI B}
\section{xe}
\chapter{2017}
\fi

    \item Copper is an $FCC$ metal with lattice parameter of 3.62 \AA . Hall effect measurement shows electron mobility to be $3.2 \times 10^{-3} {m}^2 {V}^{-1}{s}^{-1}$. Electrical resistivity of copper is $1.7 \times 10^{-8}  \Omega {m}$. The average number of free electrons per atom in copper is \rule{2.5cm}{0.5pt}(Charge of an electron: $1.6 \times 10^{-19}$ C) \hfill(2017-XE)
    \item In an ionic solid the cation and the anion have ionic radii as 0.8 \AA \ and 1.6 \AA \ respectively. The maximum coordination number of the cation in the structure will be \hfill(2017-XE)
 \begin{enumerate}
     \begin{multicols}{4}
         \item 3
         \item 4
         \item 6
         \item 8
     \end{multicols}
 \end{enumerate}
\item Which of the following statement(s) is / are true regarding susceptibility of a material
\begin{enumerate}[label=\roman*.]
    \item  Magnetic susceptibility is positive for a diamagnetic material 
    \item  Magnetic susceptibility is negative for a diamagnetic material 
    \item  Magnetic susceptibility is negative for an ferromagnetic material 
    \item Magnetic susceptibility is positive for a paramagnetic material
\end{enumerate}
\hfill(2017-XE)
\begin{enumerate}
    \begin{multicols}{4}
        \item (ii) and (iv)\\
        \item (i) and (iii)\\
        \item (ii) and (iii)\\
        \item (i) and (iv)
    \end{multicols}
\end{enumerate}
\item In the truss shown, a mass $m = 10kg$ is hung from the node J. The magnetic of net force(in Newtons) transferred by the truss 
 EFGHIJ onto the truss JKLMNO at the node J is \rule{2.5cm}{0.6pt} \\\\ Assume acceleration due to gravity $g = 10m/s^2$\hfill(2017-XE)
\begin{figure}[H]
\centering
\resizebox{8.5cm}{!}{%
\begin{circuitikz}
\tikzstyle{every node}=[font=\Huge]
\draw [line width=2pt, short] (2.5,38.25) -- (22.5,38);
\draw [line width=2pt, short] (2.5,38.25) -- (2.5,28.5);
\draw [line width=2pt, short] (2.5,28.5) -- (12.75,38);
\draw [line width=2pt, short] (12.75,38) -- (22.5,27.75);
\draw [line width=2pt, short] (22.5,38) -- (22.5,27.75);
\draw [line width=2pt, short] (7.75,38.25) -- (7.75,33.5);
\draw [line width=2pt, short] (7.75,33.5) -- (2.5,33.5);
\draw [line width=2pt, short] (17.5,38) -- (17.5,33);
\draw [line width=2pt, short] (17.5,33) -- (22.5,33);
\draw [line width=2pt, short] (7.75,33.5) -- (2.5,38.25);
\draw [line width=2pt, short] (17.5,33) -- (22.5,38);
\draw [line width=2pt, short] (2.5,28.5) -- (1.5,27.5);
\draw [line width=2pt, short] (2.5,28.5) -- (3.5,27.5);
\draw [line width=2pt, short] (1.5,27.5) -- (3.5,27.5);
\draw [line width=2pt, short] (22.5,27.75) -- (21.5,27);
\draw [line width=2pt, short] (22.5,27.75) -- (23.5,27);
\draw [line width=2pt, short] (21.5,27) -- (23.5,27);
\draw [line width=1pt, short] (12.75,38) -- (12.75,33.25);
\draw [line width=2pt, short] (11.75,33.25) -- (13.75,33.25);
\draw [line width=2pt, short] (11.75,33.25) -- (11.25,31.5);
\draw [line width=2pt, short] (13.75,33.25) -- (14.25,31.5);
\draw [line width=2pt, short] (11.25,31.5) -- (14.25,31.5);
\draw [line width=2pt, short] (0.5,27.5) -- (4.75,27.5);
\draw [line width=2pt, short] (20.75,27) -- (24.5,27);
\draw [line width=2pt, short] (17.75,33.25) -- (18,33.5);
\draw [line width=2pt, short] (17.75,27.25) -- (17.75,27.25);
\draw [line width=2pt, short] (22.5,28) -- (22.5,28.25);
\node at (2.5,38.25) [circ] {};
\node at (7.75,38.25) [circ] {};
\node at (7.75,33.5) [circ] {};
\node at (17.5,33) [circ] {};
\node at (22.5,38) [circ] {};
\node at (17.5,38) [circ] {};
\node at (22.5,33) [circ] {};
\node at (22.5,27.75) [circ] {};
\node at (2.5,28.5) [circ] {};
\node [font=\Huge] at (2,39) {G};
\node [font=\Huge] at (7.75,39) {H};
\node [font=\Huge] at (12.75,39) {J};
\node [font=\Huge] at (17.5,39) {K};
\node [font=\Huge] at (23,39.25) {M};
\node [font=\Huge] at (23.25,32.75) {N};
\node [font=\Huge] at (23.5,28) {O};
\node [font=\Huge] at (17,32.25) {L};
\node [font=\Huge] at (8.25,33.25) {I};
\node [font=\Huge] at (1.25,33.25) {F};
\node [font=\Huge] at (1.5,28.5) {E};
\node [font=\Huge] at (12.75,32.5) {$m$};
\draw [line width=1.2pt, short] (2.5,41.75) -- (7.5,41.75);
\draw [line width=1.2pt, short] (7.5,42.5) -- (7.5,41);
\draw [line width=1.2pt, short] (7.5,41.75) -- (22.5,41.5);
\draw [line width=1.2pt, short] (2.5,42.5) -- (2.5,41);
\draw [line width=1.2pt, short] (12.75,42.5) -- (12.75,41);
\draw [line width=1.2pt, short] (17.5,42.5) -- (17.5,41);
\draw [line width=1.2pt, short] (22.5,42.5) -- (22.5,40.75);
\node [font=\Huge] at (5.5,43) {$1m$};
\node [font=\Huge] at (10.25,43) {$1m$};
\node [font=\Huge] at (15.25,42.75) {$1m$};
\node [font=\Huge] at (20,43) {$1m$};
\draw [line width=1.2pt, short] (25.75,38) -- (25.75,33);
\draw [line width=1.2pt, short] (24.75,38) -- (26.75,38);
\draw [line width=1.2pt, short] (24.75,33) -- (27,33);
\draw [line width=1.2pt, short] (25.75,33) -- (25.75,28);
\draw [line width=1.2pt, short] (25,28) -- (27,28);
\node [font=\Huge] at (26.75,35.5) {1m};
\node [font=\Huge] at (27,30.5) {1m};
\end{circuitikz}
}%

\label{fig:my_label}
\end{figure}

\item A ball moves along a plannar frictionless slot as shown. Which one of the paths shown closely matches the path by the ball after it exits the slot at E \hfill(2017-XE)
\begin{enumerate}
    \begin{multicols}{4}
        \item path $m$
        \item path $n$
        \item path $p$
        \item path $q$
    \end{multicols}
\end{enumerate}
\begin{figure}[!ht]
\centering
\resizebox{6cm}{!}{%
\begin{circuitikz}
\tikzstyle{every node}=[font=\normalsize]
\draw [line width=0.5pt, short] (3,12.5) .. controls (2.5,14.5) and (3,14.75) .. (4.25,14.25);
\draw [line width=0.5pt, short] (2.75,12.5) .. controls (2.25,13.75) and (2.25,15.25) .. (4,14.75);
\draw [line width=0.5pt, short] (4,14.75) -- (6.5,14);
\draw [line width=0.5pt, short] (4.25,14.25) -- (6.5,13.5);
\draw [line width=0.5pt, short] (6.5,13.5) .. controls (8.25,13.5) and (8.75,13.75) .. (8.75,14.5);
\draw [line width=0.5pt, short] (6.5,14) .. controls (8,13.75) and (8.5,14) .. (8.5,14.75);
\draw [line width=0.5pt, short] (8.75,14.5) .. controls (9,15.25) and (8.75,15.5) .. (8,16.25);
\draw [line width=0.5pt, short] (8.5,14.75) .. controls (8.25,15.5) and (8.25,15.75) .. (7.5,16);
\draw [line width=0.5pt, short] (7.5,16) .. controls (7.75,16) and (8,16) .. (8,16.25);
\draw [line width=0.5pt, short] (7.5,16) .. controls (7.75,16.25) and (7.75,16.25) .. (8,16.25);
\draw [line width=0.5pt, ->, >=Stealth, dashed] (8,16.25) -- (8.5,18);
\draw [line width=0.5pt, ->, >=Stealth, dashed] (8,16.25) .. controls (7,17) and (6.5,17) .. (6.25,18.25);
\draw [line width=0.5pt, ->, >=Stealth, dashed] (7.75,16.25) -- (5.25,17);
\draw [line width=0.5pt, ->, >=Stealth, dashed] (7.75,16) .. controls (6.25,16.25) and (6.25,16.25) .. (5.25,15.75);
\node [font=\normalsize] at (8.25,16.25) {$E$};
\node [font=\normalsize] at (6.75,18) {$p$};
\node [font=\normalsize] at (8.75,18) {$q$};
\node [font=\normalsize] at (6,17.25) {$n$};
\node [font=\normalsize] at (6,15.75) {$m$};
\end{circuitikz}
}%

\label{fig:my_label}
\end{figure}

\item A rod $EF$ moving in a plane has velocity $V_E$ at $E$ and $V_F$ that are parallel to each other. Which of the following \textbf{CANNOT} be true? \hfill(2017-XE)
\begin{figure}[!ht]
\centering
\resizebox{6.5cm}{!}{%
\begin{circuitikz}
\tikzstyle{every node}=[font=\large]
\draw [line width=0.7pt, short] (3.25,13.25) -- (10.25,11.5);
\draw [line width=0.7pt, short] (3.25,13.25) -- (3.5,13.5);
\draw [line width=0.7pt, short] (3.5,13.5) -- (10.5,11.75);
\draw [line width=0.7pt, short] (10.5,11.75) -- (10.25,11.5);
\draw [line width=0.7pt, ->, >=Stealth] (3.5,13.5) -- (4.5,14.25);
\draw [line width=0.7pt, ->, >=Stealth] (10.5,11.75) -- (11.75,13);
\node [font=\large] at (3.25,12.75) {$E$};
\node [font=\large] at (10,11.25) {$F$};
\node [font=\large] at (4,14.25) {$V_E$};
\node [font=\large] at (12,12.5) {$V_F$};
\end{circuitikz}
}%

\label{fig:my_label}
\end{figure}
\begin{enumerate}
    \item Both $V_E$ and $V_F$ are perpendicular to $EF$.
    \item Magnitude of $V_E$ is equal to the magnitude of $V_F$ and the angular velocity of $EF$ is zero.
    \item The velocity $V_E$ is not perpendicular to $EF$ and the angular velocity of $EF$ is nonzero.
    \item Magnitude of $V_E$ is not equal to the magnitude of $V_F$ and the angular velocity of $EF$ is nonzero.
\end{enumerate}
\item The beam shown below carries two external moments. A counterclockwise moment of magnitude $2M$ acts at point $B$ and a clockwise moment of magnitude $M$ acts at the free end, $C$. The beam is fixed at $A$. The shear force at a section close to the fixed end is equal to \hfill(2017-XE)
\begin{figure}[H]
\centering
\resizebox{8cm}{!}{%
\begin{circuitikz}
\tikzstyle{every node}=[font=\large]
\draw [line width=1pt, short] (2.75,13.25) -- (2.75,11.75);
\draw [line width=1pt, short] (2.75,12.25) -- (10.25,12.25);
\draw [line width=1pt, short] (2.75,12.5) -- (10.25,12.5);
\draw [line width=1pt, short] (10.25,12.5) -- (10.25,12.25);
\draw [line width=1pt, short] (2.75,13.25) -- (2.5,13);
\draw [line width=1pt, short] (2.75,13) -- (2.5,12.75);
\draw [line width=1pt, short] (2.75,12.75) -- (2.5,12.5);
\draw [line width=1pt, short] (2.75,12.5) -- (2.5,12.25);
\draw [line width=1pt, short] (2.75,12.25) -- (2.5,12);
\draw [line width=1pt, short] (2.75,12) -- (2.5,11.75);
\draw [line width=1pt, ->, >=Stealth] (6.75,11.75) .. controls (7,12.5) and (7,12.25) .. (6.75,13) ;
\draw [line width=1pt, ->, >=Stealth] (10.25,13) .. controls (10.75,12.5) and (10.5,12.25) .. (10.25,11.75) ;
\draw [line width=1pt, <->, >=Stealth] (2.75,11.25) -- (6.75,11.25);
\draw [line width=1pt, <->, >=Stealth] (6.75,11.25) -- (10.25,11.25);
\node [font=\large] at (7,13.5) {$2M$};
\node [font=\large] at (10.25,13.25) {$M$};
\node [font=\large] at (10,11.75) {$C$};
\node [font=\large] at (8.5,11.5) {$L/2$};
\node [font=\large] at (4.5,11.75) {$L/2$};
\node [font=\large] at (3,11.5) {$A$};
\draw [ line width=1pt](2.75,11.5) to[short] (2.75,11);
\draw [ line width=1pt](6.75,11.5) to[short] (6.75,11);
\draw [ line width=1pt](10.25,11.5) to[short] (10.25,11);
\node [font=\large] at (6.5,11.75) {$B$};
\end{circuitikz}
}%

\label{fig:my_label}
\end{figure}

\begin{enumerate}
    \begin{multicols}{4}
        \item $\frac{2M}{L}$
        \item $\frac{M}{L}$
        \item 0
        \item $-\frac{M}{L}$
    \end{multicols}
\end{enumerate}
\item Two pendulums are shown below. \textbf{\textit{Pendulum-A}} carries a bob of mass $m,$ hung using a hinged massless rigid rod of length $L$ whereas \textbf{\textit{Pendulum-B}} carries a bob of mass $4m$ and length $L/4$. The ratio of the natural frequencies of \textbf{\textit{Pendulum-A}} and \textbf{\textit{Pendulum-B}} is given by \hfill(2017-XE)
\begin{figure}[H]
\centering
\resizebox{8cm}{!}{%
\begin{circuitikz}
\tikzstyle{every node}=[font=\large]
\draw [line width=1pt, short] (1.75,14.25) -- (3.5,14.25);
\draw [line width=1pt, short] (2,14.25) -- (1.75,14.5);
\draw [line width=1pt, short] (2.25,14.25) -- (2,14.5);
\draw [line width=1pt, short] (2.5,14.25) -- (2.25,14.5);
\draw [line width=1pt, short] (2.75,14.25) -- (2.5,14.5);
\draw [line width=1pt, short] (3,14.25) -- (2.75,14.5);
\draw [line width=1pt, short] (3.25,14.25) -- (3,14.5);
\draw [line width=1pt, short] (2.5,14.25) -- (2.75,14);
\draw [line width=1pt, short] (2.75,14.25) -- (3.5,14.25);
\draw [line width=1pt, short] (2.75,14.25) -- (3.5,14.25);
\draw [line width=1pt, short] (2.75,14.25) -- (3.75,14.25);
\draw [line width=1pt, short] (3.25,14.25) -- (3,14);
\draw [line width=1pt, short] (2.75,14) -- (2.75,11.5);
\draw [line width=1pt, short] (3,14) -- (3,11.5);
\draw [ line width=1pt ] (2.85,11) circle (0.5cm);
\node [font = \large] at (3.8, 11) {$m$};
\draw [line width=1pt, short] (3.5,14.25) -- (3.25,14.5);
\draw [line width=1pt, short] (7,14.25) -- (8.75,14.25);
\draw [line width=1pt, short] (7,14.25) -- (6.75,14.5);
\draw [line width=1pt, short] (7,14.5) -- (7.25,14.25);
\draw [line width=1pt, short] (7.25,14.5) -- (7.5,14.25);
\draw [line width=1pt, short] (7.5,14.5) -- (7.75,14.25);
\draw [line width=1pt, short] (7.75,14.5) -- (8,14.25);
\draw [line width=1pt, short] (8,14.5) -- (8.25,14.25);
\draw [line width=1pt, short] (8.25,14.5) -- (8.5,14.25);
\draw [line width=1pt, short] (7.5,14.25) -- (7.75,14);
\draw [line width=1pt, short] (8.25,14.25) -- (8,14);
\draw [line width=1pt, short] (7.75,14) -- (7.75,12.75);
\draw [line width=1pt, short] (8,14) -- (8,12.75);
\draw [ line width=1pt ] (7.85,12.25) circle (0.5cm);
\node [font=\large] at (2.25,12.5) {$L$};
\node [font=\large] at (7.25,13.25) {$\frac{L}{4}$};
\node [font=\large] at (9,12.25) {$4m$};
\node [font=\normalsize] at (2.75,10) {\textbf{\textit{pendulum-A}}};
\node [font=\normalsize] at (8,11.25) {\textbf{\textit{Pendulum-B}}};
\end{circuitikz}
}%

\label{fig:my_label}
\end{figure}
\begin{enumerate}
    \begin{multicols}{4}
        \item 1 : 2
        \item 1 : 1
        \item $\sqrt{2}$ : 1
        \item 2 : 1 
    \end{multicols}
\end{enumerate}
\item A closed thin-walled cylindrical steel pressure vessel of wall thickness $t = 1 \ \text{mm}$ is subjected to internal pressure. The maximum value of pressure $p$ (in kPa) that the wall can withstand based on the maximum shear stress failure theory is given by (Yield strength of steel is $200  MPa$ and mean radius of the cylinder $r = 1 m$).

\hfill(2017-XE)
\begin{enumerate}
    \begin{multicols}{4}
        \item 100
        \item 200
        \item 300
        \item 400
    \end{multicols}
\end{enumerate}
\item The state of stress at a point in a body is represented using components of stresses along $X$ and $Y$ directions as shown. Which one of the following represents the state of the stress along $X^\prime$ and $Y^\prime$ axes?($X^\prime$- axis at $45^{\degree}$ clockwise with respect to $X$- axis)

\hfill(2017-XE)
\begin{figure}[H]
\centering
\resizebox{6cm}{!}{%
\begin{circuitikz}
\tikzstyle{every node}=[font=\normalsize]

\draw [line width=0.5pt, short] (3.75,13.5) -- (3.75,11.25);
\draw [line width=0.5pt, short] (3.75,13.5) -- (6.5,13.5);
\draw [line width=0.5pt, short] (6.5,13.5) -- (6.5,11.25);
\draw [line width=0.5pt, short] (3.75,11.25) -- (6.5,11.25);
\draw [line width=0.5pt, ->, >=Stealth] (2.25,12.5) -- (3.75,12.5);
\draw [line width=0.5pt, ->, >=Stealth] (8,12.5) -- (6.5,12.5);
\draw [line width=0.5pt, ->, >=Stealth, dashed] (5,12.5) -- (5,13.25);
\draw [line width=0.5pt, ->, >=Stealth, dashed] (5,12.5) -- (5.75,12.5);
\node [font=\normalsize] at (7,12.75) {$\sigma$};
\node [font=\normalsize] at (5.5,12.75) {$X$};
\node [font=\normalsize] at (4.75,13) {$Y$};
\end{circuitikz}
}%

\label{fig:my_label}
\end{figure}

\begin{enumerate}
    \item \begin{figure}[H]
\centering
\resizebox{2cm}{!}{%
\begin{circuitikz}
\tikzstyle{every node}=[font=\normalsize]
\draw [line width=0.5pt, short] (5,14) -- (3.5,12.5);
\draw [line width=0.5pt, short] (5,14) -- (6.5,12.5);
\draw [line width=0.5pt, short] (3.5,12.5) -- (5,11);
\draw [line width=0.5pt, short] (5,11) -- (6.5,12.5);
\draw [line width=0.5pt, short] (6.5,12.75) -- (5.25,14);
\draw [line width=0.5pt, short] (5.25,14) -- (5.5,14);
\draw [line width=0.5pt, short] (6.5,12.25) -- (5.25,11);
\draw [line width=0.5pt, short] (5.25,11) -- (5.75,11);
\draw [line width=0.5pt, short] (3.5,12.75) -- (4.75,14);
\draw [line width=0.5pt, short] (4.75,14) -- (4.5,14);
\draw [line width=0.5pt, short] (3.5,12.25) -- (4.75,11);
\draw [line width=0.5pt, short] (4.75,11) -- (4.5,11);
\draw [line width=0.5pt, ->, >=Stealth] (3,14.25) -- (4,13.25);
\draw [line width=0.5pt, ->, >=Stealth] (6.75,14) -- (6,13.25);
\draw [line width=0.5pt, ->, >=Stealth] (6.75,11) -- (6,11.75);
\draw [line width=0.5pt, ->, >=Stealth] (3.25,11) -- (4,11.75);
\draw [line width=0.5pt, ->, >=Stealth, dashed] (4.5,12.5) -- (5.25,13.25);
\draw [line width=0.5pt, ->, >=Stealth, dashed] (4.5,12.5) -- (5.25,11.75);
\node [font=\normalsize] at (5,13.25) {$Y$};
\node [font=\normalsize] at (5.5,12) {$X$};
\node [font=\normalsize] at (5,14.25) {$\tau = \sigma/2 $};
\node [font=\normalsize] at (7,13.5) {$\sigma/2$};
\node [font=\normalsize] at (6.75,11.75) {$\sigma/2$};
\node [font=\normalsize] at (5,10.75) {$\tau = \sigma/2$};
\end{circuitikz}
}%

\label{fig:my_label}
\end{figure}

    \item \begin{figure}[H]
\centering
\resizebox{2cm}{!}{%
\begin{circuitikz}
\tikzstyle{every node}=[font=\normalsize]
\draw [line width=0.5pt, short] (5,14) -- (3.5,12.5);
\draw [line width=0.5pt, short] (5,14) -- (6.5,12.5);
\draw [line width=0.5pt, short] (3.5,12.5) -- (5,11);
\draw [line width=0.5pt, short] (5,11) -- (6.5,12.5);
\draw [line width=0.5pt, short] (6.5,12.75) -- (5.25,14);
\draw [line width=0.5pt, short] (5.25,14) -- (5.5,14);
\draw [line width=0.5pt, short] (6.5,12.25) -- (5.25,11);
\draw [line width=0.5pt, short] (5.25,11) -- (5.75,11);
\draw [line width=0.5pt, short] (3.5,12.75) -- (4.75,14);
\draw [line width=0.5pt, short] (4.75,14) -- (4.5,14);
\draw [line width=0.5pt, short] (3.5,12.25) -- (4.75,11);
\draw [line width=0.5pt, short] (4.75,11) -- (4.5,11);
\draw [line width=0.5pt, ->, >=Stealth] (4,13.25) -- (3.25,14);
\draw [line width=0.5pt, ->, >=Stealth] (6.75,14) -- (6,13.25);
\draw [line width=0.5pt, ->, >=Stealth] (6,11.5) -- (6.75,10.75);
\draw [line width=0.5pt, ->, >=Stealth] (3.25,11) -- (4,11.75);
\draw [line width=0.5pt, ->, >=Stealth, dashed] (4.5,12.5) -- (5.25,13.25);
\draw [line width=0.5pt, ->, >=Stealth, dashed] (4.5,12.5) -- (5.25,11.75);
\node [font=\normalsize] at (5,13.25) {$Y$};
\node [font=\normalsize] at (5.5,12) {$X$};
\node [font=\normalsize] at (5,14.25) {$\tau = \sigma/2 $};
\node [font=\normalsize] at (7,13.5) {$\sigma/2$};
\node [font=\normalsize] at (6.75,11.75) {$\sigma/2$};
\node [font=\normalsize] at (5,10.75) {$\tau = \sigma/2$};
\end{circuitikz}
}%

\label{fig:my_label}
\end{figure}

    \item \begin{figure}[!ht]
\centering
\resizebox{2cm}{!}{%
\begin{circuitikz}
\tikzstyle{every node}=[font=\normalsize]
\draw [line width=0.5pt, short] (5,14) -- (3.5,12.5);
\draw [line width=0.5pt, short] (5,14) -- (6.5,12.5);
\draw [line width=0.5pt, short] (3.5,12.5) -- (5,11);
\draw [line width=0.5pt, short] (5,11) -- (6.5,12.5);
\draw [line width=0.5pt, short] (6.5,12.75) -- (5.25,14);
\draw [line width=0.5pt, short] (5.25,14) -- (5.5,14);
\draw [line width=0.5pt, short] (6.5,12.25) -- (5.25,11);
\draw [line width=0.5pt, short] (5.25,11) -- (5.75,11);
\draw [line width=0.5pt, short] (3.5,12.75) -- (4.75,14);
\draw [line width=0.5pt, short] (4.75,14) -- (4.5,14);
\draw [line width=0.5pt, short] (3.5,12.25) -- (4.75,11);
\draw [line width=0.5pt, short] (4.75,11) -- (4.5,11);
\draw [line width=0.5pt, ->, >=Stealth] (4,13.25) -- (3,14.25);
\draw [line width=0.5pt, ->, >=Stealth] (6,13.25) -- (6.75,14);
\draw [line width=0.5pt, ->, >=Stealth] (6,11.75) -- (6.75,11);
\draw [line width=0.5pt, ->, >=Stealth] (4,11.75) -- (3.25,11);
\draw [line width=0.5pt, ->, >=Stealth, dashed] (4.5,12.5) -- (5.25,13.25);
\draw [line width=0.5pt, ->, >=Stealth, dashed] (4.5,12.5) -- (5.25,11.75);
\node [font=\normalsize] at (5,13.25) {$Y$};
\node [font=\normalsize] at (5.5,12) {$X$};
\node [font=\normalsize] at (5,14.25) {$\tau = \sigma/2 $};
\node [font=\normalsize] at (7,13.5) {$\sigma/2$};
\node [font=\normalsize] at (6.75,11.75) {$\sigma/2$};
\node [font=\normalsize] at (5,10.75) {$\tau = \sigma/2$};
\end{circuitikz}
}%

\label{fig:my_label}
\end{figure}

    \item \begin{figure}[!ht]
\centering
\resizebox{2cm}{!}{%
\begin{circuitikz}
\tikzstyle{every node}=[font=\normalsize]
\draw [line width=0.5pt, short] (5,14) -- (3.5,12.5);
\draw [line width=0.5pt, short] (5,14) -- (6.5,12.5);
\draw [line width=0.5pt, short] (3.5,12.5) -- (5,11);
\draw [line width=0.5pt, short] (5,11) -- (6.5,12.5);
\draw [line width=0.5pt, short] (6.5,12.75) -- (5.25,14);
\draw [line width=0.5pt, short] (5.25,14) -- (5.5,14);
\draw [line width=0.5pt, short] (6.5,12.25) -- (5.25,11);
\draw [line width=0.5pt, short] (5.25,11) -- (5.75,11);
\draw [line width=0.5pt, short] (3.5,12.75) -- (4.75,14);
\draw [line width=0.5pt, short] (4.75,14) -- (4.5,14);
\draw [line width=0.5pt, short] (3.5,12.25) -- (4.75,11);
\draw [line width=0.5pt, short] (4.75,11) -- (4.5,11);
\draw [line width=0.5pt, ->, >=Stealth] (4,13.25) -- (3,14.25);
\draw [line width=0.5pt, ->, >=Stealth] (6,13.25) -- (6.75,14);
\draw [line width=0.5pt, ->, >=Stealth] (6,11.75) -- (6.75,11);
\draw [line width=0.5pt, ->, >=Stealth] (4,11.75) -- (3.25,11);
\draw [line width=0.5pt, ->, >=Stealth, dashed] (4.5,12.5) -- (5.25,13.25);
\draw [line width=0.5pt, ->, >=Stealth, dashed] (4.5,12.5) -- (5.25,11.75);
\node [font=\normalsize] at (5,13.25) {$Y$};
\node [font=\normalsize] at (5.5,12) {$X$};
\node [font=\normalsize] at (7.25,12.5) {$\tau = \sigma/2 $};
\node [font=\normalsize] at (7,13.5) {$\sigma/2$};
\node [font=\normalsize] at (6.75,11.75) {$\sigma/2$};
\node [font=\normalsize] at (5,10.75) {$\tau = \sigma/2$};
\node [font=\normalsize] at (3,12.5) {$\tau = \sigma/2 $};
\end{circuitikz}
}%

\label{fig:my_label}
\end{figure}

\end{enumerate}
\item An aluminum specimen with an initial gauge diameter $d_0 = 10mm$ and a gauge length $l_0 = 10mm$ is subjected to tension test. A tensile force $P = 50kN$ is applied at the ends of the specimen as shown resulting in an elongation of $1mm$ in the gauge length. The Poisson's ratio ($\gamma$) of the specimen is \rule{2cm}{0.4pt}\\ \\
Shear modulus of the material $G = 25GPa$. Consider engineering stress-strain conditions. \hfill(2017-XE)
\begin{figure}[H]
\centering
\resizebox{1.9cm}{!}{%
\begin{circuitikz}
\tikzstyle{every node}=[font=\large]
\draw [line width=1.2pt, short] (3.75,14) -- (4.25,13.25);
\draw [line width=1.2pt, short] (4.75,13.25) -- (5.25,14);
\draw [line width=0.7pt, short] (4.25,13.25) -- (4.25,10.5);
\draw [line width=0.7pt, short] (4.75,13.25) -- (4.75,10.5);
\draw [line width=0.7pt, short] (3.75,14) -- (3.75,16);
\draw [line width=0.7pt, short] (5.25,14) -- (5.25,16);
\draw [line width=1.2pt, short] (3.75,16) .. controls (4.25,15.5) and (5,15.5) .. (5.25,16);
\draw [line width=1.2pt, short] (3.75,16) .. controls (4.25,16.5) and (4.75,16.5) .. (5.25,16);
\draw [line width=1.2pt, short] (4.25,10.5) -- (3.75,9.75);
\draw [line width=1.2pt, short] (4.75,10.5) -- (5.25,9.75);
\draw [line width=1.2pt, short] (3.75,9.75) -- (3.75,9.5);
\draw [line width=0.7pt, short] (3.75,9.75) -- (3.75,8);
\draw [line width=0.7pt, short] (5.25,9.75) -- (5.25,8);
\draw [line width=0.7pt, short] (5.25,8) -- (5.25,7.75);
\draw [line width=0.7pt, short] (3.75,8) -- (3.75,7.75);
\draw [line width=0.7pt, short] (3.75,7.75) .. controls (4.25,7) and (5,7.25) .. (5.25,7.75);
\draw [line width=0.7pt, dashed] (3.75,7.75) .. controls (4.5,8.5) and (4.75,8.25) .. (5.25,7.75);
\draw [ line width=0.7pt](4.5,12.5) to[short] (6.25,12.5);
\draw [ line width=0.7pt](4.5,11) to[short] (6.25,11);
\node at (4.5,12.5) [circ] {};
\node at (4.5,11) [circ] {};
\draw [line width=0.7pt, <->, >=Stealth] (5.75,12.5) -- (5.75,11);
\draw [line width=0.7pt, ->, >=Stealth] (4.5,16) -- (4.5,17.25);
\draw [line width=0.7pt, ->, >=Stealth] (4.5,7.75) -- (4.5,6.75);
\node [font=\large] at (4.75,16.75) {$P$};
\node [font=\large] at (5,7) {$P$};
\node [font=\large] at (6,11.75) {$I_0$};
\node [font=\large] at (3.5,13) {$d_0$};
\draw [short] (3.25,12.75) -- (5,12.75);
\draw [short] (4.5,13) -- (4,12.5);
\draw [short] (5,13) -- (4.5,12.5);
\end{circuitikz}
}%

\label{fig:my_label}
\end{figure}

\item A rectangular sheet $ABCD$ of dimensions $a$ and $b$ along $X$ and $Y$ directions, respectively, is stretched to a rectangle $AB'C'D'$, as shown. The maximum principal strain ($\varepsilon_1$) and minimum principal strain ($\varepsilon_2$) due to the stretch are given by

\hfill(2017-XE)
\begin{figure}[H]
\centering
\resizebox{6cm}{!}{%
\begin{circuitikz}
\tikzstyle{every node}=[font=\normalsize]
\draw [line width=0.9pt, short] (3.5,14.5) -- (3.5,11.75);
\draw [line width=0.9pt, short] (3.5,14.5) -- (7.25,14.5);
\draw [line width=0.9pt, short] (7.25,14.5) -- (7.25,11.75);
\draw [line width=0.9pt, short] (3.5,11.75) -- (7.25,11.75);
\draw [line width=0.9pt, dashed] (3.5,14.5) -- (3.5,15.75);
\draw [line width=0.9pt, dashed] (3.5,15.75) -- (8.25,15.75);
\draw [line width=0.9pt, dashed] (7.25,11.75) -- (8.5,11.75);
\draw [line width=0.9pt, dashed] (8.5,11.75) -- (8.5,15.75);
\draw [line width=0.9pt, dashed] (8,15.75) -- (8.5,15.75);
\node [font=\normalsize] at (3.25,14.25) {$D$};
\node [font=\normalsize] at (7,14.25) {$C$};
\node [font=\normalsize] at (7,11.5) {$B$};
\node [font=\normalsize] at (3.25,11.5) {$A$};
\node [font=\normalsize] at (3.25,16) {$D^{\prime}$};
\node [font=\normalsize] at (8.75,16.25) {$C^{\prime}$};
\node [font=\normalsize] at (9,11.75) {$B^{\prime}$};
\draw [line width=0.9pt, <->, >=Stealth] (3.5,10.5) -- (7.25,10.5);
\draw [line width=0.9pt, <->, >=Stealth] (7.25,10.5) -- (8.5,10.5);
\draw [line width=0.9pt, <->, >=Stealth] (2.5,15.75) -- (2.5,14.5);
\draw [line width=0.9pt, <->, >=Stealth] (2.5,14.5) -- (2.5,11.75);
\draw [line width=0.5pt, short] (2.5,11.75) -- (3.5,11.75);
\draw [line width=0.5pt, short] (3.5,11.75) -- (3.5,10);
\draw [line width=0.5pt, short] (8.5,11.5) -- (8.5,10);
\draw [line width=0.5pt, short] (7.25,11.75) -- (7.25,10);
\draw [line width=0.5pt, dashed] (1.75,13) -- (10.25,13);
\draw [line width=0.5pt, ->, >=Stealth] (1.25,13) -- (1.25,14);
\draw [line width=0.5pt, ->, >=Stealth] (1.25,13) -- (2.25,13);
\node [font=\normalsize] at (1.25,14.25) {$Y$};
\node [font=\normalsize] at (2,13.25) {$X$};
\node [font=\normalsize] at (2.25,13.75) {$b$};
\node [font=\normalsize] at (1.75,15) {$0.001b$};
\node [font=\normalsize] at (5.25,10.75) {$a$};
\node [font=\normalsize] at (8,10.75) {$0.001a$};
\draw [line width=0.5pt, short] (3.5,15.75) -- (2,15.75);
\draw [line width=0.5pt, short] (2,14.5) -- (3.5,14.5);
\end{circuitikz}
}%

\label{fig:my_label}
\end{figure}

\begin{enumerate}
    \begin{multicols}{2}
        \item $\varepsilon_1 = 0.001$ and $\varepsilon_2 = 0.001$
        \item $\varepsilon_1 = -0.001$ and $\varepsilon_2 = 0.001$
        \item $\varepsilon_1 = 0.001$ and $\varepsilon_2 = -0.001$
        \item  $\varepsilon_1 = -0.001$ and $\varepsilon_2 = -0.001$
    \end{multicols}
\end{enumerate}
\item A solid bar of uniform square cross-section of side $b$ and length $L$ is rigidly fixed to the supports at the two ends. When the temperature in the rod is increased uniformly by $T$, the bar undergoes elastic buckling. Assume Young's modulus $E$ and coefficient of thermal expansion $\alpha$ to be independent of temperature. The coefficient of thermal expansion $\alpha$ is given by \hfill(2017-XE)
\begin{enumerate}
    \begin{multicols}{4}
        \item $\frac{3\pi^2b^2}{T_cL^2}$
         \item $\frac{\pi^2b^2}{T_cL^2}$
          \item $\frac{\pi^2b^2}{2T_cL^2}$
           \item $\frac{\pi^2b^2}{3T_cL^2}$
    \end{multicols}
\end{enumerate}

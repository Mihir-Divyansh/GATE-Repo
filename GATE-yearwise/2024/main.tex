\iffalse
    \title{Assignment}
    \author{EE24BTECH11066}
    \section{xe}
    \chapter{2024}
  \fi
%27
\item The hydraulic diameter for a circular pipe of radius $R$ is \hfill{[2024-XE]}\\
\begin{enumerate}
    \item $0.5R$\\
    \item $R$\\
    \item $2R$\\
    \item $4R$\\
\end{enumerate}

%28
\item For incompressible, laminar, fully-developed flow through a circular pipe, Darcy friction factor and Fanning friction factor are represented as $f$ and $C_{f}$, respectively. Which one of the following options is correct? \hfill{[2024-XE]}\\
\begin{enumerate}
    \item $f=0.25 C_{f}$\\
    \item $f=0.5 C_{f}$\\
    \item $f=2 C_{f}$\\
    \item $f=4 C_{f}$\\
\end{enumerate}

%29
\item For an immersed neutrally buoyant body to be in stable equilibrium, the center of gravity of the body is directly \hfill{[2024-XE]}\\
\begin{enumerate}
    \item above the metacenter.\\
    \item below the metacenter.\\
    \item above the center of buoyancy.\\
    \item below the center of buoyancy.\\
\end{enumerate}

%30
\item The absolute pressure in a chamber is measured as $400 \, \text{mm Hg}$ at a location where the atmospheric pressure is $700 \, \text{mm Hg}$. A vacuum gauge connected to the chamber reads \underline{\hspace{1cm}} $\text{mm Hg}$ $\brak{\text{answer in integer}}$ \hfill{[2024-XE]}\\

%31
\item A thin film of incompressible, Newtonian liquid $\brak{\text{density } \rho, \text{ viscosity } \mu}$ with an uniform thickness $\brak{\text{h}}$ is flowing down on a vertical plate. The flow is driven by gravity $\brak{\text{g}}$ alone. Assume zero shear stress condition at the free surface. \hfill{[2024-XE]}\\

\scalebox{0.75}{
\begin{circuitikz}
\tikzstyle{every node}=[font=\large]
\draw [line width=1.9pt, short] (5.75,15.75) -- (5.75,6.75);
\draw [line width=0.5pt, short] (5.75,13) -- (12.75,13);
\draw [line width=0.5pt, short] (12.75,16) -- (12.75,6.5);
\draw [line width=0.5pt, short] (5.75,13) .. controls (7.5,10) and (9.5,9) .. (12.75,8.75);
\draw [line width=0.5pt, ->, >=Stealth] (6.75,13) -- (6.75,11.5);
\draw [line width=0.5pt, ->, >=Stealth] (7.5,13) -- (7.5,10.75);
\draw [line width=0.5pt, ->, >=Stealth] (8.25,13) -- (8.25,10);
\draw [line width=0.5pt, ->, >=Stealth] (9,13) -- (9,9.5);
\draw [line width=0.5pt, ->, >=Stealth] (9.75,13) -- (9.75,9.25);
\draw [line width=0.5pt, ->, >=Stealth] (10.5,13) -- (10.5,9);
\draw [line width=0.5pt, ->, >=Stealth] (11.25,13) -- (11.25,9);
\draw [line width=0.5pt, ->, >=Stealth] (12,13) -- (12,8.75);
\draw [line width=0.5pt, ->, >=Stealth] (5.75,7) -- (5.75,5.75);
\draw [line width=0.5pt, ->, >=Stealth] (9,7.25) -- (12.75,7.25);
\draw [line width=0.5pt, ->, >=Stealth] (9.75,7.25) -- (5.75,7.25);
\draw [line width=0.5pt, ->, >=Stealth] (9.75,16.5) -- (9.75,14);
\draw [line width=0.5pt, ->, >=Stealth] (5.75,15.75) -- (6.75,15.75);
\node [font=\Large, rotate around={90:(0,0)}] at (5.25,11.25) {Vertical plane};
\node [font=\large] at (7.25,15.75) {y,v};
\node [font=\large] at (10.25,14) {g};
\node [font=\large] at (9.25,7.75) {h};
\node [font=\Large, rotate around={90:(0,0)}] at (13.25,10.75) {Free surface};
\node [font=\large] at (5.75,5.25) {x,u};
\end{circuitikz}
}
The maximum velocity is given by\\
\begin{enumerate}
    \item $\frac{1}{2\mu}\rho g h^2$\\
    \item $\frac{1}{4\mu}\rho g h^2$\\
    \item $\frac{1}{\mu}\rho g h^2$\\
    \item $\frac{1}{8\mu}\rho g h^2$\\
\end{enumerate}

%32
\item A one-eighth scale model of a car is to be tested in a wind tunnel. If the air velocity over the car is $16 \frac{m}{s}$, what should be the air velocity $\brak{\text{in} \frac{m}{s}}$ in the wind tunnel in order to achieve similarity between the model and the prototype? \hfill{[2024-XE]}\\
\begin{enumerate}
    \item 2\\
    \item 16\\
    \item 64\\
    \item 128\\
\end{enumerate}

%33
\item A set of basic dimensions, mass, length, and time are represented by $M, L$, and $T$ respectively. What will be the dimensions of pressure in $M-L-T$ system? \hfill{[2024-XE]}\\
\begin{enumerate}
    \item $ML^{-1}T^{-2}$\\
    \item $MLT^{-2}$\\
    \item $MLT^{-1}$\\
    \item $ML^{-1}T^{-1}$\\
\end{enumerate}

%34
\item Consider a fluid flow around an airfoil as shown in figure.
\scalebox{0.75}{
\begin{circuitikz}
\tikzstyle{every node}=[font=\normalsize]
\draw [short] (12.25,13.5) .. controls (3.25,16.5) and (5.75,10) .. (12.25,13.5);
\draw [->, >=Stealth] (13,13.5) -- (15.25,13.5);
\draw [->, >=Stealth] (13,13.5) -- (13,15.5);
\draw [->, >=Stealth] (13,13.5) -- (12,15);
\draw [->, >=Stealth] (13,13.5) -- (15,14.75);
\draw [short] (14,13.5) .. controls (13.5,13.5) and (14.5,13.5) .. (13.75,14);
\draw [->, >=Stealth] (1.5,13.75) -- (3,14.5);
\draw [->, >=Stealth] (1.75,13.25) -- (3.25,14);
\draw [->, >=Stealth] (2,12.75) -- (3.5,13.5);
\draw [->, >=Stealth] (2.25,12.25) -- (3.75,13);
\draw [->, >=Stealth] (2.5,11.75) -- (4,12.5);
\draw [->, >=Stealth] (2.75,11.25) -- (4.25,12);
\draw [short] (2.75,11.25) -- (4.5,11.25);
\draw [->, >=Stealth] (7.5,13.25) -- (7.5,14);
\draw [->, >=Stealth] (7.5,13.25) -- (8.25,13.25);
\draw [short] (5.75,18.75) -- (5.75,18.75);
\draw [short] (3.25,11.5) .. controls (3.5,11.5) and (3.75,11.5) .. (3.5,11.25);
\node [font=\large, rotate around={25:(0,0)}] at (2,14.75) {Flow Direction};
\node [font=\normalsize, rotate around={13:(0,0)}] at (4,11.5) {$30^{\circ}$};
\node [font=\normalsize] at (7.75,14) {y};
\node [font=\normalsize, rotate around={45:(0,0)}] at (8,14.25) {};
\node [font=\normalsize] at (8.5,13.25) {x};
\node [font=\large] at (11.75,15.25) {D};
\node [font=\large] at (13,15.75) {C};
\node [font=\large] at (15.25,15) {B};
\node [font=\large] at (15.5,13.25) {A};
\node [font=\large] at (12.75,13) {O};
\node [font=\normalsize, rotate around={31:(0,0)}] at (14.5,14) {$30^{\circ}$};
\node [font=\large, rotate around={45:(0,0)}] at (1.5,14.75) {};
\end{circuitikz}
}
The directions of drag force and lift force, respectively are along \hfill{[2024-XE]}\\
\begin{enumerate}
    \item $OA$ and $OC$\\
    \item $OA$ and $OD$\\
    \item $OB$ and $OC$\\
    \item $OB$ and $OD$\\
\end{enumerate}

%35
\item A vessel which contains a volatile liquid and its vapor is connected with a mercury manometer as shown in figure. Both the liquid and vapor phases are at equilibrium. The vapor pressure and density of the volatile liquid are $107.6 \text{kPa}$ and $700 \frac{kg}{m^3}$, respectively. The density of the mercury is $13600 \frac{kg}{m^3}$. Acceleration due to gravity $\brak{g}$ is $10 \frac{m}{s^2}$ and atmospheric pressure is $101 \text{kPa}$. Hydrostatic pressure created by the weight of the vapor is neglected.

\scalebox{0.5}{
\begin{circuitikz}
\tikzstyle{every node}=[font=\large]
\draw [ fill={rgb,255:red,61; green,56; blue,70} , rounded corners = 42.0] (4,15.75) rectangle (8,6.5);
\draw [short] (5.75,6.5) -- (5.75,4.25);
\draw [short] (6.5,6.5) -- (6.5,5);
\draw [short] (6.5,5) -- (10,5);
\draw [short] (5.75,4.25) -- (10,4.25);
\draw [short] (10,5) -- (10,14.75);
\draw [short] (10,4.25) -- (10.75,4.25);
\draw [short] (10.75,4.25) -- (10.75,14);
\draw [short] (10,14.75) -- (13.75,14.75);
\draw [short] (10.75,14) -- (13,14);
\draw [short] (13,14) -- (13,4.5);
\draw [short] (13.75,14.75) -- (13.75,5.25);
\draw [short] (13,4.5) -- (16.75,4.5);
\draw [short] (13.75,5.25) -- (16,5.25);
\draw [short] (16,5.25) -- (16,17.75);
\draw [short] (16.75,4.5) -- (16.75,17.75);
\draw [ fill={rgb,255:red,61; green,56; blue,70} ] (5.75,6.5) rectangle (6.5,4.25);
\draw [ fill={rgb,255:red,61; green,56; blue,70} ] (5.75,5) rectangle (10.75,4.25);
\draw [ fill={rgb,255:red,61; green,56; blue,70} ] (10,4.25) rectangle (10.75,14.75);
\draw [ fill={rgb,255:red,61; green,56; blue,70} ] (10,14.75) rectangle (13.75,14);
\draw [ fill={rgb,255:red,61; green,56; blue,70} ] (13,14.75) rectangle (13.75,13.25);
\draw [ color={rgb,255:red,192; green,191; blue,188} , fill={rgb,255:red,192; green,191; blue,188}] (13,13.25) rectangle (13.75,4.5);
\draw [ color={rgb,255:red,192; green,191; blue,188} , fill={rgb,255:red,192; green,191; blue,188}] (13,5.25) rectangle (16.75,4.5);
\draw [ color={rgb,255:red,192; green,191; blue,188} , fill={rgb,255:red,192; green,191; blue,188}] (16,4.5) rectangle (16.75,15.75);
\draw [ fill={rgb,255:red,61; green,56; blue,70} ] (4,15.75) rectangle (8,7.75);
\draw [dashed] (8,13.25) -- (13,13.25);
\draw [dashed] (16.75,13.25) -- (17.75,13.25);
\draw [dashed] (16.75,15.75) -- (18,15.75);
\draw [dashed] (18,15.75) -- (17.75,15.75);
\draw [dashed] (1.25,15.75) -- (2.25,15.75);
\draw [dashed] (1.25,13.25) -- (2.25,13.25);
\draw  (3,13.25) circle (0.5cm);
\draw [short] (3.5,13.5) -- (4,13.5);
\draw [short] (3.5,13) .. controls (3.75,13) and (3.75,13) .. (4,13);
\draw [->, >=Stealth] (3,13.25) -- (2.75,13.5);
\draw [ color={rgb,255:red,246; green,245; blue,244}, ->, >=Stealth] (2.75,10.25) -- (5,10.25);
\draw [short] (2.75,10.25) -- (4,10.25);
\draw [->, >=Stealth] (17.75,9.25) -- (16.25,9.25);
\draw [->, >=Stealth] (17,14.5) -- (17,15.5);
\draw [->, >=Stealth] (17,14.5) -- (17,13.5);
\draw [->, >=Stealth] (16.5,17.5) -- (16.5,18.5);
\draw [->, >=Stealth] (11.5,18) -- (11.5,16.25);
\draw [->, >=Stealth] (1.75,14.5) -- (1.75,15.5);
\draw [->, >=Stealth] (1.75,14.75) -- (1.75,13.5);
\draw [->, >=Stealth] (3.25,16.75) -- (5.5,16.75);
\node [font=\Large] at (2,16.75) {Vapour};
\node [font=\large] at (1.75,10.25) {Volatile};
\node [font=\large] at (12,16.25) {g};
\node [font=\large] at (16.5,19.25) {Open to };
\node [font=\large] at (17.25,14.5) {h};
\node [font=\large] at (19,9.25) {Mercury};
\node [font=\large] at (1.75,9.75) {liquid};
\node [font=\large, rotate around={90:(0,0)}] at (0.5,14.75) {};
\node [font=\large, rotate around={90:(0,0)}] at (1.25,14.5) {1 m};
\draw (5,17.25) to[short] (7,17.25);
\draw [short] (4,15.75) .. controls (4.25,16.75) and (4.25,17) .. (5,17.25);
\draw [short] (7,17.25) .. controls (8,17) and (7.75,16.75) .. (8,15.75);
\node [font=\large] at (17.75,18.75) {atmosphere};
\end{circuitikz}
}

The height, $h$, $\brak{\text{in m, rounded off to two decimal places}}$ of the mercury column in figure is \underline{\hspace{1cm}}. \hfill{[2024-XE]}\\

%36
\item The velocity in a one-dimensional flow is given by $u \brak{x} = \frac{a}{\brak{b-x}^2} \frac{m}{s}$, where $a=8 \frac{m}{s^2}$ and $b=4 m$. The acceleration $\brak{\text{in} \frac{m}{s^2}, \text{ answer in integer}}$ at $x=2 m$ is \underline{\hspace{1cm}}.\hfill{[2024-XE]}\\

%37
\item Consider two parallel plates separated by a distance of $1 cm$ filled with a Newtonian fluid of viscosity $10^{-3} \text{Pa.s}$. The top plate is moving with a velocity of $1 \frac{m}{s}$ whereas the bottom plate is stationary. The shear stress $\brak{\text{in Pa, rounded off to one decimal}}$ on the top plate is \underline{\hspace{1cm}}.\hfill{[2024-XE]}\\

%38
\item A circular water jet of diameter $50 mm$ impinges with a velocity of $18 \frac{m}{s}$ normal to a plate. The density of water is $1000 \frac{m^3}{s}$ and gravity force is neglected.
\scalebox{0.75}{
\begin{circuitikz}
\tikzstyle{every node}=[font=\large]
\draw [ fill={rgb,255:red,192; green,191; blue,188} , line width=0.8pt ] (7.25,15.75) rectangle (14.25,14.75);
\draw [short] (7.25,13.25) -- (10,13.25);
\draw [short] (10,13.25) -- (10,9.5);
\draw [short] (11.25,13.25) -- (11.25,9.5);
\draw [short] (11.25,13.25) .. controls (11.75,13.25) and (12.75,13.25) .. (14.25,13.25);
\draw [line width=1.6pt, short] (10,9.5) -- (11.25,9.5);
\draw [line width=1.6pt, short] (10,9.5) -- (9.75,9.25);
\draw [line width=1.6pt, short] (9.75,9.25) -- (9.75,8.5);
\draw [line width=1.6pt, short] (9.75,8.5) -- (11.5,8.5);
\draw [line width=1.6pt, short] (11.25,9.5) -- (11.5,9.25);
\draw [line width=1.6pt, short] (11.5,9.25) -- (11.5,8.5);
\draw [line width=0.5pt, ->, >=Stealth] (8.5,14) -- (7,14);
\draw [line width=0.5pt, ->, >=Stealth] (13,14) -- (14.25,14);
\draw [line width=0.5pt, ->, >=Stealth] (12.5,11.5) -- (11.25,11.5);
\draw [line width=0.5pt, ->, >=Stealth] (8.75,11.5) -- (10,11.5);
\draw [line width=0.5pt, ->, >=Stealth] (10.5,9.5) -- (10.5,10.25);
\draw [line width=0.5pt, ->, >=Stealth] (10.75,14.25) -- (10.75,14.75);
\node [font=\large] at (11,16.25) {Stationary plate};
\node [font=\large] at (11,13.75) {V=18 m/s};
\node [font=\large, rotate around={90:(0,0)}] at (10.5,11.5) {Water jet};
\node [font=\large] at (13,12) {d=50 mm};
\node [font=\normalsize] at (10.5,9) {Nozzle};
\end{circuitikz}
}
The magnitude of net force $\brak{\text{in N, rounded off to two decimal places}}$ imparted by the jet on the stationary plate is \underline{\hspace{1cm}}. \hfill{[2024-XE]}\\

%39
\item Consider the steady, incompressible flow of water in a horizontal pipe of constant diameter $1 m$ with an inlet velocity of $12 \frac{m}{s}$.
\scalebox{0.5}{
\begin{circuitikz}
\tikzstyle{every node}=[font=\LARGE]
\draw [line width=1.2pt, short] (6.25,9.75) -- (17.75,9.75);
\draw [line width=1.1pt, short] (6.25,11.75) -- (10.5,11.75);
\draw [line width=1.1pt, short] (12.5,11.75) -- (17.5,11.75);
\draw [line width=0.6pt, ->, >=Stealth] (4,10.75) -- (6.5,10.75);
\draw [line width=0.6pt, ->, >=Stealth] (17,10.75) -- (18.75,10.75);
\draw [line width=0.6pt, ->, >=Stealth] (11.5,11.75) -- (11.5,13);
\draw [line width=0.6pt, ->, >=Stealth] (12,11.75) .. controls (11.75,12.75) and (12.5,13.25) .. (12.75,12.75) ;
\node [font=\Large] at (5,11.25) {Inlet};
\node [font=\Large] at (11.5,13.5) {Circular hole};
\node [font=\Large] at (18,11.25) {Outlet};
\draw [->, >=Stealth] (11,11.75) .. controls (11.5,12.5) and (11.25,13.25) .. (10.25,12.75) ;
\end{circuitikz}
}
As shown in the figure, water is lost through a circular hole of diameter $0.6  \text{m}$ at the rate of $4.53 \frac{m^3}{s}$. The outlet velocity $\brak{\text{in } \frac{m}{s} \text{, rounded off to two decimal places}}$ of water in the pipe is \underline{\hspace{1cm}}. \hfill{[2024-XE]}\\


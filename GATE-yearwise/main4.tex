\iffalse
\title{Assignment3}
\author{ee24btech11064}
\chapter{2010}
\section{ma}
\fi

%\begin{enumerate}
\item For the linear programming problem
\begin{align*}
    \text{Minimize } z = x - y \text{ , subject to } 2x + 3y \leq 6, 0 \leq x \leq 3, 0 \leq y \leq 3,
\end{align*}
the number of extreme points of its feasible region and the number of basic feasible solutions respectively, are
\begin{multicols}{2}
    \begin{enumerate}
        \item 3 and 3
        \item 4 and 4 
        \item 3 and 5
        \item 4 and 5
    \end{enumerate}
\end{multicols}
\item Which one of the following statements is correct?
\begin{enumerate}
        \item If a Linear Programming Problem (LPP) is infeasible, then its dual is also infeasible
        \item If an LPP is infeasible, then its dual always has unbounded solution
        \item If an LPP has unbounded solution, then its dual also has unbounded solution
        \item If an LPP has unbounded solution, then its dual is infeasible
\end{enumerate}
\item Which one of the following groups is simple?
\begin{multicols}{2}
\begin{enumerate}
        \item $S_3$
        \item $GL\brak{2,R}$
        \item $Z_2 \times Z_2$
        \item $A_5$
\end{enumerate}
\end{multicols}
\item Consider the algebraic extension $E = Q \brak{\sqrt{2}, \sqrt{3}, \sqrt{5}}$ of the field $\mathbb{Q}$ of rational numbers. Then $\sbrak{E \colon Q}$, the degree of $E$ over $Q$, is
\begin{multicols}{2}
    \begin{enumerate}
        \item 3
        \item 4
        \item 7
        \item 8
    \end{enumerate}
\end{multicols}

\item The general solution of the partial differential equation $\frac{\partial^2 z}{\partial x \partial y} = x + y$ is of the form
\begin{multicols}{2}
    \begin{enumerate}
        \item $\frac{1}{2} xy \brak{x+y} + F\brak{x} + G\brak{y}$
        \item $\frac{1}{2} xy \brak{x-y} + F\brak{x} + G\brak{y}$
        \item $\frac{1}{2} xy \brak{x-y} + F\brak{x} G\brak{y}$
        \item $\frac{1}{2} xy \brak{x+y} + F\brak{x} G\brak{y}$
    \end{enumerate}
\end{multicols}

\item The numerical value obtained by applying the two-point trapezoidal rule to the integral $\int_0^1 \frac{\ln{\brak{1+x}}}{x}, dx$ is
\begin{multicols}{2}
    \begin{enumerate}
        \item $\frac{1}{2} \brak{\ln{2}+1}$
        \item $\frac{1}{2}$
        \item $\frac{1}{2} \brak{\ln{2}-1}$
        \item $\frac{1}{2} \ln{2}$
    \end{enumerate}
\end{multicols}

\item Let $l_k \brak{x} =, k =0,1, \dots , n$ denote the Lagrange's fundamental polynomials of degree $n$ for the nodes $x_0, x_1, \dots , x_n$. Then the value of $\sum_{k=0}^{n} l_k \brak{x}$ is
\begin{multicols}{2}
    \begin{enumerate}
        \item 0
        \item 1
        \item $x^n + 1$
        \item $x^n - 1$
    \end{enumerate}
\end{multicols}

\item Let $X$ and $Y$ be normed linear spaces and $\cbrak{T_n}$ be a sequence of bounded linear operators from $X$ to $Y$. Consider the statements:
\begin{align*}
    P \colon \cbrak{\abs{\abs{T_n x}} \colon n \in \mathbb{N}} \text{ is bounded for each } x \in X\\
    Q \colon \cbrak{\abs{\abs{T_n}} \colon n \in \mathbb{N}} \text{ is bounded}
\end{align*}
    \begin{enumerate}
        \item If $P$ implies $Q$, then both $X$ and $Y$ are Banach spaces
        \item If $P$ implies $Q$, then only one of $X$ and $Y$ are Banach space
        \item If $X$ is a Banach space, then $P$ implies $Q$
        \item If $Y$ is a Banach space, then $P$ implies $Q$
    \end{enumerate}

\item Let $X = C\sbrak{0,1}$ with the norm $\abs{\abs{x}}_t  = \int_0^1 \abs{x\brak{t}}, dt$, $x \in C \sbrak{0,1}$ and $\Omega = \cbrak{f \in X^{\prime} \colon \abs{\abs{f}} = 1}$, where $X^{\prime}$ denotes the dual space of $X$. Let $C\brak{\Omega}$ be the linear space of continuous functions on $\Omega$ with the norm $\abs{\abs{u}} = sup _{s \in \Omega} \abs{u\brak{s}}$, $u \in C\brak{\Omega}$. Then
    \begin{enumerate}
        \item $X$ is linearly isometric with $C\brak{\Omega}$
        \item $X$ is linearly isometric with a proper subspace of $C\brak{\Omega}$ 
        \item there does not exist a linear isometry from $X$ into $C\brak{\Omega}$
        \item every linear isometry from $X$ to $C\brak{\Omega}$ is onto
    \end{enumerate}

\item Let $X = R$ equipped with the topology generated by open intervals of the form $\brak{
		a,b}$ and sets of the form $\brak{a,b} \cup Q$. The which one of the following statements is correct?
\begin{multicols}{2}
    \begin{enumerate}
        \item $X$ is regular
        \item $X$ is normal
        \item $\frac{X}{Q}$ is dense in $X$
        \item $Q$ is dense in $X$
    \end{enumerate}
\end{multicols}

\item Let $H$, $T$ and $V$ denote the Hamiltonian, the kinetic energy and the potential energy respectively of a mechanical system at time $t$. If $H$ contains $t$ explicitly, then $\frac{\partial H}{\partial t}$ is equal to
\begin{multicols}{2}
    \begin{enumerate}
        \item $\frac{\partial T}{\partial t} + \frac{\partial V}{\partial t}$
        \item $\frac{\partial T}{\partial t} - \frac{\partial V}{\partial t}$
        \item $\frac{\partial V}{\partial t} - \frac{\partial T}{\partial t}$
        \item $-\frac{\partial V}{\partial t} - \frac{\partial V}{\partial t}$
    \end{enumerate}
\end{multicols}

\item The Euler's equation for the variational problem: Minimize \\$I\sbrak{y\brak{x}} = \int_0^1 \brak{2x - xy - y^{\prime}} y^{\prime}, dx$, is 
\begin{multicols}{2}
    \begin{enumerate}
        \item $2y^{{\prime}{\prime}} - y = 2$
        \item $2y^{{\prime}{\prime}} + y = 2$
        \item $y^{{\prime}{\prime}} + 2y = 0$
        \item $2y^{{\prime}{\prime}} - y = 0$
    \end{enumerate}
\end{multicols}
\item Let X have a binomial distribution with parameter n and p, $n=3$. For testing the hypothesis $H_0:p=\frac{2}{3}$ against $H_1:p=\frac{1}{3}$, let a test be :"Reject $H_0$ if $X\geq2$ and accept $H_0$ if $X\leq 1$". Then the probabilities of Type 1 and Type 2 errors respectively are 
\begin{multicols}{2}
    \begin{enumerate}
        \item $\frac{20}{27}$ and $\frac{20}{27}$
        \item $\frac{7}{27}$ and $\frac{20}{27}$
        \item $\frac{20}{27}$ and $\frac{7}{27}$
        \item $\frac{7}{27}$ and $\frac{7}{27}$
    \end{enumerate}
\end{multicols}
%\end{enumerate}


 \iffalse
        \chapter{2016}
        \author{AI24BTECH11034}
        \section{ma}
 \fi
\item Let $X$ be a random variable with the following cumulative distribution function:
\begin{align*}
F(x) = \begin{cases}
0 & x < 0 \\
x^2 & 0 \leq x < \frac{1}{2} \\
\frac{3}{4} & \frac{1}{2} \leq x < 1 \\
1 & x \geq 1.
\end{cases}
\end{align*}
Then $P\brak{\frac{1}{4} < X < 1}$ is equal to $\underline{\hspace{2cm}}$\\

\item Let $y$ be the curve which passes through $\brak{0,1}$ and intersects each curve of the family $y = cx^2$ orthogonally. Then $y$ also passes through the point
\begin{multicols}{2}
\begin{enumerate}
    \item $(\sqrt{2}, 0)$
    \item $(0, \sqrt{2})$
    \item $(1, 1)$
    \item $(-1, 1)$
\end{enumerate}
\end{multicols}

\item Let $S(x) = a_0 + \sum_{n=1}^{\infty}\brak{a_n \cos\brak{nx} + b_n \sin\brak{nx}}$ be the Fourier series of the $2\pi$ periodic function defined by $f\brak{x} = x^2 + 4\sin\brak{x}\cos\brak{x}, -\pi \leq x \leq \pi$. Then 
\begin{align*}
\left|\sum_{n=0}^{\infty} a_n - \sum_{n=1}^{\infty} b_n \right|
\end{align*}
is equal to $\underline{\hspace{2cm}}$\\

\item Let $y\brak{t}$ be a continuous function on $[0,\infty)$. If
\begin{align*}
y\brak{t}=t\brak{1-4\int_{0}^{t}y\brak{x}dx}+4\int_{0}^{t}xy\brak{x}dx
\end{align*}
then $\int_{0}^{\pi/2}y\brak{t}dt$ is equal to  $\underline{\hspace{2cm}}$\\

\item Let $S_{n}=\sum_{k=1}^{n}\frac{1}{k}$ and $I_{n}=\int_{1}^{n}\frac{x-[x]}{x^{2}}dx$. Then $S_{10}+I_{10}$ is equal to
\begin{multicols}{2}
\begin{enumerate}
    \item $\ln 10 + 1$
    \item $\ln 10 - 1$
    \item $\ln 10 - \frac{1}{10}$
    \item $\ln 10 + \frac{1}{10}$
\end{enumerate}
\end{multicols}

\item For any $(x, y) \in \mathbb{R}^2 \setminus \overline{B(0,1)}$, let

\begin{align*}
    f(x, y) &= \text{distance}\brak{(x, y), \overline{B(0,1)}} \\
    &= \text{infimum} \cbrak{\sqrt{(x-x_1)^2 + (y-y_1)^2} : (x_1, y_1) \in \overline{B(0,1)}}.
\end{align*}

Then, $\|\nabla f(3,4)\|$ is equal to $\underline{\hspace{2cm}}$\\

\item If$f\brak{x} = \brak{\int_0^x e^{-t^2} \, dt }^2 \text{and} g(x) = \int_0^1 \frac{e^{-x^2\brak{1 + t^2}}}{1 + t^2} \, dt.$
Then $f'\brak{\sqrt{\pi}} + g'\brak{\sqrt{\pi}}$ is equal to \underline{\hspace{2cm}}.\\

\item Let $M = \begin{bmatrix} a & b & c \\ b & d & e \\ c & e & f \end{bmatrix}$ be a real matrix with eigenvalues $1$, $0$, and $3$. If the eigenvectors corresponding to $1$ and $0$ are $\brak{1, 1, 1}^T$ and $\brak{1, -1, 0}^T$ respectively, then the value of $3f$ is equal to \underline{\hspace{2cm}}.\\

\item Let $M = \begin{bmatrix} 1 & 1 & 0 \\ 1 & 1 & 1 \\ 0 & 0 & 1 \end{bmatrix}$ and $e^M = Id + M + \frac{1}{2!} M^2 + \frac{1}{3!} M^3 + \dots$. If $e^M = [b_{ij}]$, then
\begin{align*}
\frac{1}{e} \sum_{i=1}^3 \sum_{j=1}^3 b_{ij}
\end{align*}
is equal to \underline{\hspace{2cm}}\\

\item Let the integral $I = \int_0^4 f(x) \, dx$, where 

$f(x) = 
\begin{cases} 
x & 0 \leq x \leq 2 \\ 
4 - x & 2 \leq x \leq 4 
\end{cases}$

Consider the following statements P and Q:\\
(P): The integral values \( I_2 \) and \( I_3 \) are exact for any function.\\
(Q): The integral values \( I_2 \) and \( I_3 \) are exact only for certain types of functions.\\

Which of the above statements hold TRUE?
\begin{multicols}{2}
\begin{enumerate}
    \item Both P and Q
    \item Only P
    \item Only Q
    \item Neither P nor Q
\end{enumerate}
\end{multicols}

\item The difference between the least two eigenvalues of the boundary value problem
\begin{align*}
y'' + \lambda y = 0, \quad 0 < x < \pi
\end{align*}
\begin{align*}
y\brak{0} = 0,  y'\brak{\pi} = 0
\end{align*}
is equal to \underline{\hspace{2cm}}.

\item The number of roots of the equation $x^2 - \cos\brak{x} = 0$ in the interval $\sbrak{-\frac{\pi}{2}, \frac{\pi}{2} }$ is equal to \underline{\hspace{2cm}}\\

\item For the fixed point iteration $x_{k+1} = g\brak{x_k}$, $k = 0, 1, 2, \dots$, consider the following statements P and Q:\\

(P): If $g\brak{x} = 1 + \frac{2}{x}$, then the fixed point iteration converges to $2$ for all $x_0 \in \sbrak{1, 100}$.\\

(Q): If $g\brak{x} = \sqrt{2 + x}$, then the fixed point iteration converges to $2$ for all $x_0 \in \sbrak{0, 100}$.\\

Which of the above statements hold TRUE?
\begin{multicols}{2}
\begin{enumerate}
    \item Both P and Q
    \item Only P
    \item Only Q
    \item Neither P nor Q
\end{enumerate}
\end{multicols}



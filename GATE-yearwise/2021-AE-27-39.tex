\iffalse
\chapter{2021}
\author{AI24BTECH11004}
\section{ae}
\fi
       \item The figure shows three glasses $P,Q$ and $R$ with water and floating ice cube. Glass $P$ has a solid ice cube, glass $Q$ has an air bubble. After the ice cube melts, the level of water in glasses $P,Q$ and $R$, respectively:
       \hfill{(2021)}
     \begin{center}
\resizebox{0.5\textwidth}{!}{%
\begin{circuitikz}
\tikzstyle{every node}=[font=\footnotesize]
\draw (0.75,17.25) to[short] (0.75,13.75);
\draw (0.75,13.75) to[short] (3,13.75);
\draw (3,13.75) to[short] (3,17.25);
\draw (4,17.25) to[short] (4,13.75);
\draw (4,13.75) to[short] (6.25,13.75);
\draw (6.25,13.75) to[short] (6.25,17.25);
\draw (7.25,17.25) to[short] (7.25,13.75);
\draw (7.25,13.75) to[short] (9.25,13.75);
\draw (9.25,13.75) to[short] (9.25,17.25);
\draw  (1.5,16.25) rectangle (2.25,15.5);
\draw [short] (0.75,15.75) -- (1.5,15.75);
\draw [short] (2.25,15.75) -- (3,15.75);
\draw  (4.75,16.25) rectangle (5.5,15.5);
\draw  (8,16.25) rectangle (8.75,15.5);
\draw [short] (4,15.75) -- (4.75,15.75);
\draw [short] (5.5,15.75) -- (6.25,15.75);
\draw [short] (7.25,15.75) -- (8,15.75);
\draw [short] (8.75,15.75) -- (9.25,15.75);
\draw [ fill={rgb,255:red,94; green,92; blue,100} ] (1.5,16.25) rectangle (2.25,15.5);
\draw [ fill={rgb,255:red,94; green,92; blue,100} ] (4.75,16.25) rectangle (5.5,15.5);
\draw [ fill={rgb,255:red,94; green,92; blue,100} ] (8,16.25) rectangle (8.75,15.5);
\node [font=\small] at (1.75,16.75) {ICE CUBE};
\draw [ fill={rgb,255:red,0; green,0; blue,0} ] (5,16) circle (0.25cm);
\draw [ fill={rgb,255:red,154; green,153; blue,150} ] (8.25,16) circle (0.25cm);
\node [font=\tiny] at (5,17.75) {ICE CUBE WITH SOLID STEEL BALL};
\node [font=\tiny] at (9.75,17.75) {ICE CUBE WITH AIR BUBBLE};
\node [font=\footnotesize] at (1.75,14.75) {WATER};
\node [font=\footnotesize] at (5,14.75) {WATER};
\node [font=\footnotesize] at (8.25,14.75) {WATER};
\node [font=\footnotesize] at (1.75,13.25) {P};
\node [font=\footnotesize] at (5,13.25) {Q};
\node [font=\footnotesize] at (8,13.5) {R};
\end{circuitikz}
}%



\end{center}


       \begin{enumerate}
           \item remains same, increases, and decreases
           \item increases, decreases, and increases
           \item remains same, decreases, and decreases
           \item remains same, decreases, and increases
       \end{enumerate}
       \item To estimate aerodynamic loads on an aircraft flying at $100km/h$ at standard sea-level conditions, a one-fifth scale model is tsted in a varibel-density wind tunnel ensuring similarity of inertial and viscous forces. The pressure used in the wind runnel is $10$ times the atmospheric pressure. Assuming ideal gas law to hold and the same temperature conditions in model and prototype, the velocity needed in the wind tunnel test-section is \rule{1cm}{0.15mm}
       \hfill{(2021)}
       \begin{enumerate}
           \item $25km/h$
           \item $50km/h$
           \item $100km/h$
           \item $20km/h$
       \end{enumerate}
       \item The figure shows schematic of a set-up for visualization of non-uniform density field in the test section o fa supersonic wind tunnel. This technique of visualization of high speed flows is known as: 
       \hfill{(2021)}
\begin{center}
\resizebox{0.5\textwidth}{!}{%

\begin{circuitikz}
\tikzstyle{every node}=[font=\footnotesize]

% Rays and lens
\draw [short] (3,17) -- (3,12.75);
\draw [short] (3,17) -- (2.75,16.75);
\draw [short] (3,17) -- (3.25,16.75);
\draw [short] (3,12.75) -- (2.75,13);
\draw [short] (3,12.75) -- (3.25,13);

% Light source rays
\draw [->, >=Stealth] (0.75,14.75) -- (3,13.25);
\draw [->, >=Stealth] (0.75,14.75) -- (3,13.75);
\draw [->, >=Stealth] (0.75,14.75) -- (3,14.25);
\draw [->, >=Stealth] (0.75,14.75) -- (3,15.5);
\draw [->, >=Stealth] (0.75,14.75) -- (3,16);
\draw [->, >=Stealth] (0.75,14.75) -- (3,16.5);

% Test section and walls
\draw (4.75,17) rectangle (5,12.75);
\draw [ fill={rgb,255:red,119; green,118; blue,123} ] (4.75,18.75) rectangle (5,17);
\draw [ fill={rgb,255:red,119; green,118; blue,123} ] (4.75,12.75) rectangle (5,11);
\draw [ fill={rgb,255:red,119; green,118; blue,123} ] (7.25,18.75) rectangle (7.5,17);
\draw [ fill={rgb,255:red,119; green,118; blue,123} ] (7.25,12.75) rectangle (7.5,10.75);
\draw (7.25,17) rectangle (7.5,12.75);

% Screen and light rays after test section
\draw [line width=1.2pt, short] (10.25,19.25) -- (10.25,10.75);
\draw [short] (3,16.5) -- (7.25,16.5);
\draw [short] (7.25,16.5) -- (10.25,16.25);
\draw [short] (3,16) -- (7.5,16);
\draw [short] (7.5,16) -- (10.25,15.25);
\draw [short] (3,15.5) -- (5.25,15.5);
\draw [short] (5.25,15.5) -- (7.25,15.25);
\draw [short] (7.25,15.25) -- (10.25,14.5);
\draw [short] (3,14.25) -- (4.75,14.25);
\draw [short] (4.75,14.25) -- (7.5,14);
\draw [short] (7.5,14) -- (10.25,13.5);
\draw [short] (3,13.75) -- (4.75,13.75);
\draw [short] (4.75,13.75) -- (7.25,13.75);
\draw [short] (7.25,13.75) -- (10.25,13.25);
\draw [short] (3,13.25) -- (3,13.75);
\draw [short] (3,13.25) -- (5,13.25);
\draw [short] (5,13.25) -- (7.25,13.5);
\draw [short] (7.25,13.5) -- (10.25,13);

% Labels and annotations
\node at (0.75,14.75) [circ] {};
\draw [->, >=Stealth] (6,11) -- (6,13)node[pos=0.5, fill=white]{FLOW};
\node [font=\footnotesize] at (0.75,14) {LIGHT SOURCE};
\node [font=\footnotesize] at (3,12.25) {CONVEX LENS};
\node [font=\footnotesize] at (6,17) {TEST SECTION};
\draw [->, >=Stealth] (7.75,18.25) -- (8.75,18.25)node[pos=0.5, fill=white]{TUNNEL WALL};
\draw [->, >=Stealth] (9,18.25) -- (9.5,18.25);
\draw [->, >=Stealth] (7.5,16.75) -- (8,16.75);
\node [font=\footnotesize] at (8.5,16.75) {WINDOW};
\node [font=\footnotesize] at (11,15.75) {Lighter area};
\node [font=\footnotesize] at (11,13.25) {Brighter area};
\end{circuitikz}
}%
\end{center}
 
		\begin{enumerate}
           \item schlieren 
           \item interferometry
           \item shadowgraph
           \item holography
       \end{enumerate}
	\item For a conventional fixed-wing aicraft in a $360^\circ$ inverted vertical loop maneuver, what is the load factor \brak{n} at the topmost point of the loop? Assume the flight to be steady at the topmost point.
	\hfill{(2021)}
               \begin{enumerate}
			        \begin{multicols}{4}  
		       \item $n=1$
		       \item $n<1$
		       \item $n=-1$
		       \item $n>-1$
                    \end{multicols}   
	       \end{enumerate}	
\textbf{The next 5 question sare multiple select queestions and carry TWO mark each}
       \item Which of the following statement\brak{s} is/are true about the function defined as $f\brak{x} = e^{-x}\abs{\cos x} \text{ for } x>0 $?
       \hfill{(2021)}
             \begin{enumerate}
                 \item Differentiable at $x=\frac{\pi}{2}$
                 \item Differentiable at $x=\pi$
                 \item Differentiable at $x=\frac{3\pi}{2}$
                 \item Continuous at $x=2\pi$
             \end{enumerate}
	\item  A two degree of freedom spring-mass system undergoing free vibration with generalized coordinates $x_1$ and $x_2$ has natural frequencies $\omega_1=233.9rad/s$ and $\omega_2=324.5rad/s$, respectively. The corresponding mode shapes are $\phi_1=\myvec{1\\-3.16}$ and $\phi_2=myvec{1\\3.16}$. If the system is disturbed with certain deflection sand zero initial velocities, then which of the following statement\brak{s} is/are true?
	\hfill{(2021)}
		\begin{enumerate}
			\item An initial deflection of $ x_1\brak{0}=6.32cm$ and $x_2\brak{0}=-3.16cm$ would make the system oscillate with only the second natural frequency. 
			\item An initial deflection of $ x_1\brak{0}=2cm$ and $x_2\brak{0}=-6.32cm$ would make the system oscillate with only the first natural frequency. 
			\item  An initial deflection of $ x_1\brak{0}=62cm$ and $x_2\brak{0}=-2cm$ would make the system oscillate with a linear combination of first and second natural frequencies. 
	        \item  An initial deflection of $ x_1\brak{0}=1cm$ and $x_2\brak{0}=-6.32cm$ would make the system oscillate with only the first natural frequency. 
        	\end{enumerate}
	\item A shock moving into a stationary gas can be transformedd to a stationary shock by a change in reference frame, as shown in the figure. Which of the following is/are true relation the flow properties in the two  reference frames?
	\hfill{(2021)}
\begin{center}
\resizebox{0.5\textwidth}{!}{%

\begin{circuitikz}
\tikzstyle{every node}=[font=\footnotesize]

% Left and right vertical lines for normal shock boundaries
\draw [line width=1pt, short] (2.75,17.25) -- (2.75,14);
\draw [line width=1pt, short] (9,17.25) -- (9,14);

% Arrow indicating transformation to shock speed
\draw [line width=2pt, ->, >=Stealth] (5.25,15.5) -- (6.25,15.5);
\node [font=\footnotesize] at (5.5,16) {Transformed to };

% Shock speed arrow
\draw [line width=0.5pt, ->, >=Stealth] (3.25,17.5) -- (0.75,17.5) node[pos=0.5, fill=white]{shock speed, $V_s$};

% Labels for moving and stationary shock problems
\node [font=\footnotesize] at (2.75,13.75) {Moving};
\node [font=\footnotesize] at (2.75,13.5) {Normal shock};
\node [font=\footnotesize] at (9,13.5) {Normal shock};
\node [font=\footnotesize] at (9,13.75) {stationary};

% Pressure, temperature, and density labels on the left
\node [font=\footnotesize] at (1.75,16.75) {$p_1$};
\node [font=\footnotesize] at (3.25,16.75) {$p_2$};
\node [font=\footnotesize] at (1.75,16) {$T_1$};
\node [font=\footnotesize] at (3.25,16) {$T_2$};
\node [font=\footnotesize] at (1.75,15.5) {$\rho_1$};
\node [font=\footnotesize] at (3.25,15.25) {$\rho_2$};
\node [font=\footnotesize] at (1.75,14.75) {$p_{01}$};
\node [font=\footnotesize] at (3.25,14.75) {$p_{02}$};
\node [font=\footnotesize] at (1.75,14.25) {$T_{01}$};
\node [font=\footnotesize] at (3.25,14.25) {$T_{02}$};

% Labels on the right for stationary shock problem
\node [font=\footnotesize] at (9.5,14.25) {$T_{02}'$};
\node [font=\footnotesize] at (8.25,14.25) {$T'_{01}$};
\node [font=\footnotesize] at (8.25,14.75) {$p'_{01}$};
\node [font=\footnotesize] at (9.5,14.75) {$p'_{02}$};
\node [font=\footnotesize] at (9.5,15.5) {$\rho'_{2}$};
\node [font=\footnotesize] at (8.25,16) {$T'_{1}$};
\node [font=\footnotesize] at (9.5,16) {$T'_2$};
\node [font=\footnotesize] at (8.25,16.75) {$p'_1$};
\node [font=\footnotesize] at (9.5,16.75) {$p'_2$};
\node [font=\footnotesize] at (8.25,15.25) {$\rho'_1$};

% Titles for the two sections
\node [font=\footnotesize] at (2.5,13) {\underline{MOVING SHOCK PROBLEM}};
\node [font=\footnotesize] at (9,13) {\underline{STATIONARY SHOCK PROBLEM}};

\end{circuitikz}
	}%
\end{center}


		\begin{enumerate}
             \item $T_1'>T_1,T_{01}'>T_{01},p_{01}'>p_{01},\rho_2'>\rho_1'$
             \item $T_1'=T_1,T_{2}'<T_{01},p_{01}'>p_{01},\rho_2'=\rho_2$
             \item $T_1'<T_1,p_{1}'>p_{1},p_{01}'>p_{01},\rho_2'>\rho_1$
             \item $T_1'=T_1,p_{2}>p_{01},T_{01}'>T_{01},p_{01}'>p_{10}$
         \end{enumerate}
	\item For a conventional fixed-wing aircraft, which of rhe following statements are true?
	\hfill{(2021)}
          \begin{enumerate}
              \item Making $C_{m_\alpha}$ more negative leads to an increase in the frequency of its short-period mode.
              \item Making $C_{m_q}$ more negative leads to a decreased damping off the short-period mode.
              \item The peimary contribution towards $C_{l_p}$ is from the aircraft wing.
              \item  Increase the size of he vertical fin leads to a higher yaw damping.
          \end{enumerate}
	\item Which of the folloing statement\brak{s} is/are true?
	\hfill{(2021)}
         \begin{enumerate}
             \item Service ceiling is higher than absolute ceiling for a piston-propeller aircraft.
             \item For a given aircraft, teh stall speed increase with increase in altitude.
             \item Everything else remaining the same, a tailwind increase the range of an aircraft.
             \item For a jet aircraft comstrained ro fly at constant altitude, there exists an altitude where its range is maximum.
         \end{enumerate}
	\item A conventional fixed-wing aircraft, with a horizontal tail and vertical fin, in steady and level flight is subjected to small perturbations. Which of the following statement\brak{s} is/are true?
	\hfill{(2021)}
          \begin{enumerate}
              \item Vertical fin has a stabilizing effecgt on the lateral stabilithy of the aircraft.
              \item Vertical fin has a destabilizing effect on the directional stability of the aircraft.
              \item Presence of wing anhedral increase the lateral stability of the aircraft. 
              \item Horizontal tail has a stabilizing effect on the longitudinal static stability of the aircraft.
          \end{enumerate}
\textbf{The next 19 questions are Numerical answer type \brak{NAT}, carry TWO mark each (no negative marks)}
    \item The ratio of the product of eigenvalues to the sum of the eigenvalues of the given matrix
            \begin{align*}
                \myvec{3&1&2\\2&-3&-1\\1&2&1}
            \end{align*}
            is \rule{1cm}{0.15mm} \brak{\text{round off to nearest integer}}
            \hfill{(2021)}
	\item The definite integral $\int_1^5x^2dx$ is evaluated using four equal interval by two methods-first by the trapezoidal rule and then by the Simpson's onw-third rule. The absolute value of the difference between the two calculations is \rule{1cm}{0.15mm} (round off to two decimal places).
	\hfill{(2021)}
	\item  The deflection $y$ of a certain beam of length $l$ and uniform weight per unit length $w$, is given as $y=\frac{w}{48EI}\brak{2x^4-3lx^3+l^3x}$, where $x$ is the distance from the point of support andd $EI$ us a constant. The non-dimensional location $\frac{x}{l}$, where the deflection of the beam is maximum,is \rule{1cm}{0.15mm} (round off to two decimal places).  
	\hfill{(2021)}
	



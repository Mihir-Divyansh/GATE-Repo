\iffalse
\title{EE-2021-14-26}
\author{EE24BTECH11041-Mohit}
\section{ee}
\chapter{2021}
\fi
\item For the network shown,the equivalent Thevenin voltage and Thevenin impedance as seen across terminals 'ab' is
\hfill{(EE 2021)}
\begin{center}
\begin{circuitikz}
\tikzstyle{every node}=[font=\large]
\draw (9.5,12) to[R] (9.5,9.75);
\draw (7.5,9.75) to[american current source] (7.5,12);
\draw (13.5,12) to[R] (15.25,12);
\draw [short] (7.5,12) -- (11.25,12);
\draw [short] (7.5,9.75) -- (15.25,9.75);
\draw (15,9.75) to[short, -o] (15.5,9.75) ;
\draw (15,12) to[short, -o] (15.5,12) ;
\draw (13.5,12) to[american controlled voltage source] (11.25,12);
\draw [->, >=Stealth] (10,11.5) -- (10,10.25);
\node [font=\large] at (10.25,11) {$i_1$};
\node [font=\large] at (12.5,12.75) {$3i_1$};
\node [font=\large] at (15.4,12.3) {a};
\node [font=\large] at (15.4,10.2) {b};
\node [font=\large] at (14.25,12.5) {2$\Omega$};
\node [font=\large] at (6.75,11) {5A};
\node [font=\large] at (8.75,11) {10$\Omega$};
\end{circuitikz}
\end{center}

\begin{enumerate}
\item 10 V in series with 12 \ohm
\item 65 V in series with 15 \ohm
\item 50 V in series with 2 \ohm
\item 35 V in series with 2 \ohm
\end{enumerate}

\item Which one of the following vector functions represents a magnetic field $\vec{B}$ ? ($\hat{x}$, $\hat{y}$, and $\hat{z}$ unit vectors along x-axis, y-axis and z-axis respectively)
\hfill{(EE 2021)}
\begin{enumerate}
\item $10$x$\hat{x}+20$y$\hat{y}-30$z$\hat{z}$
\item $10$y$\hat{x}+20$x$\hat{y}-30$z$\hat{z}$
\item $10$z$\hat{x}+20$y$\hat{y}-30$x$\hat{z}$
\item $10$x$\hat{x}+20$z$\hat{y}-30$y$\hat{z}$
\end{enumerate}


\item If the input $x(t)$ and output $y(t)$ of s system are related as $y(t)=$max$\brak{0,x(t)}$,then the system is \\
\text{ } \hfill{(EE 2021)}

\begin{enumerate}
\item linear and time-variant
\item linear and time-invariant
\item non-linear and time-variant
\item non-linear and time-invariant
\end{enumerate}

\item Two discreate-time linear time-invariant systems with impulse responses $h_1[n]=\delta[n-1]+\delta[n+1]$ and $h_2[n]=\delta[n]+\delta[n-1]$ are connected in cascade, where $\delta[n]$ is the kronecker delta. The impulse resoponse of the casacded system is 
\hfill{(EE 2021)}
\begin{enumerate}
\item $\delta[n-2]+\delta[n+1]$
\item $\delta[n-2]\delta[n]+\delta[n+1]\delta[n-1]$
\item $\delta[n-2]+\delta[n-1]+\delta[n]+\delta[n+1]$
\item $\delta[n]\delta[n-1]+\delta[n-2]\delta[n+1]$
\end{enumerate}
\item Consider the table given :
\begin{table}[h!]    
  \centering
\begin{tabular}[12pt]{ |c| c| c|}
    \hline
    \textbf{Constructional feature} & \textbf{Machine type} & \textbf{Mitigation}  \\ 
    \hline
    (P) Damper bars & (S) Induction motor & (X)Hunting\\
    \hline
    (Q)Skewed rotor slots & (T)Transformer & (Y) Magnetic locking \\   
    \hline
    (R) Compensating winding & (U) Synchronous machine & (Z) Armature reaction \\
    \hline
      & (V) DC machine & \\
    \hline
    \end{tabular}
\end{table}


The correct combination that relates the constructional feature,machine type and migretion is \\
\text{ } \hfill{(EE 2021)}
\begin{enumerate}
\item P-V-X, Q-U-Z, R-T-X
\item P-U-X, Q-S-Y, R-V-Z
\item P-T-Y, Q-V-Z, R-S-X
\item P-U-X, Q-V-Y, R-T-Z
\end{enumerate}
\item Consider a power system consisting of $N$ number of buses. Buses in this power system are categorized into slack bus, PV buses, and PQ buses for load flow study. The number of PQ buses is $N_L$. The balanced Newton-Raphson method is used to carry out load flow study in polar form. $H,S,M,$ and $R$ are sub-matrices of the Jacobian matrix $J$ as shown below :\\
$\myvec{\Delta{P} \\ \Delta{Q}} =J\myvec{\Delta{\delta} \\ \Delta{V}}$, where $J=\myvec{H & S \\ M&R}$ \\
The dimension of the sub-matrix $M$ is
\hfill{(EE 2021)}
\begin{enumerate}
\item $N_L\times(N-1)$
\item $(N-1)\times(N-1-N_L)$
\item $N_L\times(N-1+N_L)$
\item $(N-1)\times(N-1+N_L)$
\end{enumerate}
\item Two generators have cost functions $F_1$ and $F_2$. Their incremental-cost charcterstics are \\
$\frac{dF_1}{dP_1} = 40 + 0.2P_1$ \\
$\frac{dF_2}{dP_2} = 32 + 0.4P_2$ \\
They need to deliver a combined load of 260 $MW$.Ignoring the network losses, for economic operations, the generations $P_1$ and $P_2$ (in $MW$) are \\
\text{ } \hfill{(EE 2021)}
\begin{enumerate}
\item $P_1 = P_2 = 130$
\item $P_1=160$, $P_2=100$
\item $P_1=140$, $P_2=120$
\item $P_1=120$, $P_2=140$
\end{enumerate}
\item For the closed-loop system shown,the transfer function $\frac{E(s)}{R(s)}$ is 
\hfill{(EE 2021)}
\begin{center}
\begin{circuitikz}
\tikzstyle{every node}=[font=\Large]
\draw  (10,14.75) rectangle (12.5,12.25);
\draw  (10,10.25) rectangle (12.5,9.25);
\draw  (6.25,13.5) circle (0.5cm);
\draw [->, >=Stealth] (6.75,13.5) -- (10,13.5);
\draw [->, >=Stealth] (12.5,13.5) -- (15.75,13.5);
\draw [->, >=Stealth] (14.5,9.75) -- (12.5,9.75);
\draw [->, >=Stealth] (3.5,13.5) -- (5.25,13.5);
\draw [->, >=Stealth] (6.25,9.75) -- (6.25,12);
\draw [short] (5,13.5) -- (5.75,13.5);
\draw [short] (6.25,12) -- (6.25,13);
\draw [short] (6.25,9.75) -- (10,9.75);
\draw [short] (14.5,9.75) -- (14.5,13.5);
\node [font=\Large] at (11.25,13.5) {G};
\node [font=\Large] at (11.25,9.75) {H};
\node [font=\Large] at (5.5,14) {+};

\draw [short] (6.5,12.75) -- (6.75,12.75);
\node [font=\Large] at (4,14) {R(s)};
\node [font=\Large] at (8,14) {E(s)};
\node [font=\Large] at (14.25,14) {C(s)};
\end{circuitikz}
\end{center}
\begin{enumerate}
\item $\frac{G}{1+GH}$
\item $\frac{GH}{1+GH}$
\item $\frac{1}{1+GH}$
\item $\frac{1}{1+G}$
\end{enumerate}
\item Inductance is measured by
\hfill{(EE 2021)}
\begin{enumerate}
\item Schering bridge
\item Maxwell bridge
\item Kelvin bridge
\item Wein bridge
\end{enumerate}
\item Suppose the circles $x^2 + y^2 =1$ and $(x-1)^2 + (y-1)^2 =r^2$ intersect each other orthogonally at the point $\brak{u,v}$. Then $u+v=$ \rule{2cm}{0.4pt} .
\hfill{(EE 2021)}
\item In the given circuit, the value of capacitor $C$ that makes current $I =0$ is \rule{2cm}{0.4pt} $\mu F$ .
\hfill{(EE 2021)}
\begin{center}
\begin{circuitikz}
\tikzstyle{every node}=[font=\LARGE]
\draw (7.5,9.75) to[sinusoidal voltage source, sources/symbol/rotate=auto] (7.5,12.25);
\draw (7.5,12.25) to[R] (11,12.25);
\draw (10.75,12.25) to[L ] (12.5,12.25);
\draw (12.75,12.25) to[L ] (12.75,9.75);
\draw [short] (7.5,9.75) -- (16,9.75);
\draw (16,12.25) to[C] (16,9.75);
\node at (12.75,12.25) [circ] {};
\node at (12.75,9.75) [circ] {};
\draw [->, >=Stealth] (7.5,12.25) -- (8,12.25);
\node [font=\large] at (6.5,11) {10 V};
\node [font=\large] at (7.75,12.75) {I};
\node [font=\large] at (9.25,12.75) {10$\Omega$};
\node [font=\large] at (11.5,13) {j5$\Omega$};
\node [font=\large] at (14.25,13) {j5$\Omega$};
\node [font=\large] at (16.75,11) {C};
\node [font=\large] at (13.5,11) { j5$\Omega$};
\node [font=\large] at (9.5,11) {$\omega$ = 5k rad/s};
\draw [short] (12.5,12.25) -- (16,12.25);
\end{circuitikz}
\end{center}

\item Two single-core power cables have total conductors resistance of 0.7 $\ohm$ \text{ }and 0.5 $\ohm$, respectively, and their insulation resistance (between core and sheath) are 600 $M\ohm$ and 900 $M\ohm$, respectively. When the two cables are joined in series, the ratio of insulation resistance to conductor resistance is \rule{2cm}{0.4pt} $\times 10^6$.
\hfill{(EE 2021)}
\item In the given circuit, for voltage $V_y$ to be zero, the value of $\beta$ should by \rule{2cm}{0.4pt}.(Round off to 2 decimal places).
\hfill{(EE 2021)}
\begin{center}
\begin{circuitikz}
\tikzstyle{every node}=[font=\LARGE]
\draw [short] (6.25,7.25) -- (17.5,7.25);
\draw (6.25,11) to[american voltage source] (6.25,7.25);
\draw (10,11) to[R] (10,7.25);
\draw (13.75,7.25) to[american current source] (13.75,11);
\draw (17.5,11) to[american controlled voltage source] (17.5,7.25);
\draw (6.25,11) to[R] (10,11);
\draw (10,11) to[R] (13.75,11);
\draw (13.75,11) to[R] (17.5,11);
\node at (10,7.25) [circ] {};
\node at (10,11) [circ] {};
\node at (13.75,11) [circ] {};
\node at (13.75,7.25) [circ] {};
\draw [short] (13.75,7.25) -- (13.75,6.75);
\draw (13.75,7.25) to (13.75,7) node[ground]{};
\node [font=\LARGE] at (10,11.5) {$V_x$};
\node [font=\LARGE] at (13.75,11.5) {$V_y$};
\node [font=\LARGE] at (14.6,9) {2A};
\node [font=\LARGE] at (9.25,9) {4$\Omega$};
\node [font=\LARGE] at (8,11.5) {1$\Omega$};
\node [font=\LARGE] at (12,11.5) {2$\Omega$};
\node [font=\LARGE] at (15.5,11.5) {3$\Omega$};
\node [font=\LARGE] at (5.25,9.25) {6V};
\node [font=\LARGE] at (18.5,9) {$\beta$$V_x$};
\end{circuitikz}
\end{center}


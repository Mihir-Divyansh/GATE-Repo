\iffalse
	\title{2007-CE-18-34}
	\author{EE24Btech11024 - G. Abhimanyu Koushik}
	\section{ce}
	\chapter{2007}
\fi

\item The consistency and flow resistance of bitumen can be determined from the following

\hfill{\brak{\text{CE 2007}}}
\begin{enumerate}
\begin{multicols}{4}
\item Ductility test
\item Penetration test
\item Softening point test
\item Viscosity test
\end{multicols}
\end{enumerate}

\item If a two-lane national highway and a two-lane state highway intersect at right angles, the number of potential conflict points at the intersection, assuming both the roads are two-way is

\hfill{\brak{\text{CE 2007}}}
\begin{enumerate}
\begin{multicols}{4}
\item $11$
\item $17$
\item $24$
\item $32$
\end{multicols}
\end{enumerate}

\item In signal design as per Indian Roads Congress specifications, if the sum of the ratios of normal flows to saturation flow of two directional traffic flow is $0.50$ and the total lost time per cycle is $10$ seconds, the optimum cycle length in seconds is

\hfill{\brak{\text{CE 2007}}}
\begin{enumerate}
\begin{multicols}{4}
\item $100$
\item $80$
\item $60$
\item $40$
\end{multicols}
\end{enumerate}

\item For what values of $\alpha$ and $\beta$ the following simultaneous equations have an infinite number of solutions?\newline $x+y+z=5$; $x+3y+3z=9$; $x+2y+\alpha z=\beta$;

\hfill{\brak{\text{CE 2007}}}
\begin{enumerate}
\begin{multicols}{4}
\item $2$, $7$
\item $3$, $8$
\item $8$, $3$
\item $7$, $2$
\end{multicols}
\end{enumerate}

\item A velocity vector is given as $\vec{V}=5xy\hat{i}+2y^2\hat{j}+3yz^2\hat{k}$. The divergence of this velocity vector at $\brak{1,1,1}$ is

\hfill{\brak{\text{CE 2007}}}
\begin{enumerate}
\begin{multicols}{4}
\item $9$
\item $10$
\item $14$
\item $15$
\end{multicols}
\end{enumerate}

\item A body originally at $60^{\degree} C$ cools down to $40^{\degree} C$ in $15$ minutes when kept in air at a temperature of $25^{\degree} C$. What will be the temperature of the body at the end of $30$ minutes?

\hfill{\brak{\text{CE 2007}}}
\begin{enumerate}
\begin{multicols}{4}
\item $35.2^{\degree} C$
\item $31.5^{\degree} C$
\item $28.7^{\degree} C$
\item $15^{\degree} C$
\end{multicols}
\end{enumerate}

\item The following equation needs to be numerically solved using the Newton-Raphson method.\newline $x^3+4x-9=0$\newline The iterative equation for this purpose is $\brak{k\text{ indicates the iteration level}}$ 

\hfill{\brak{\text{CE 2007}}}
\begin{enumerate}
\begin{multicols}{4}
\item $x_{k+1}=\frac{2{x_{k}}^3+9}{3{x_{k}}^3+4}$
\item $x_{k+1}=\frac{3{x_{k}}^3+4}{2{x_{k}}^3+9}$
\item $x_{k+1}=x_k-3{x_k}^2+4$
\item $x_{k+1}=\frac{4{x_{k}}^3+3}{9{x_{k}}^3+2}$
\end{multicols}
\end{enumerate}

\item Evaluate $\int_{0}^{\infty}\frac{\sin{t}}{t}$

\hfill{\brak{\text{CE 2007}}}
\begin{enumerate}
\begin{multicols}{4}
\item $\pi$
\item $\frac{\pi}{2}$
\item $\frac{\pi}{4}$
\item $\frac{\pi}{8}$
\end{multicols}
\end{enumerate}

\item Potential function $\phi$ is given as $\phi=x^2-y^2$. What will be the stream function $\brak{\psi}$ with the condition $\psi=0$ at $x=y=0$?

\hfill{\brak{\text{CE 2007}}}
\begin{enumerate}
\begin{multicols}{4}
\item $2xy$
\item $x^2+y^2$
\item $x^2-y^2$
\item $2x^2y^2$
\end{multicols}
\end{enumerate}

\item The inverse of the $2\times 2$ matrix $\myvec{1&2\\5&7}$ is,

\hfill{\brak{\text{CE 2007}}}
\begin{enumerate}
\begin{multicols}{4}
\item $\frac{1}{3}\myvec{-7&2\\5&-1}$
\item $\frac{1}{3}\myvec{7&2\\5&1}$
\item $\frac{1}{3}\myvec{7&-2\\-5&1}$
\item $\frac{1}{3}\myvec{-7&-2\\-5&-7}$
\end{multicols}
\end{enumerate}

\item Given that one root of the equation $x^3-10x^2+31x-30=0$ is $5$, the other two roots are

\hfill{\brak{\text{CE 2007}}}
\begin{enumerate}
\begin{multicols}{4}
\item $2$ and $3$
\item $2$ and $4$
\item $3$ and $4$
\item $-2$ and $-3$
\end{multicols}
\end{enumerate}

\item If the standard deviation of the spot speed of vehicles in a highway is $8.8kmph$ and the mean speed of the vehicles is $33kmph$, the coefficient of variation in speed is 

\hfill{\brak{\text{CE 2007}}}
\begin{enumerate}
\begin{multicols}{4}
\item $0.1517$
\item $0.1867$
\item $0.2666$
\item $0.3646$
\end{multicols}
\end{enumerate}

\item A metal bar of length $100mm$ is inserted between two rigid supports and its temperature is increased by $10^{\degree} C$. If the coefficient of thermal expansion is $12\times 10^{-6}$ per $^{\degree} C$ and the Young's modulus is $2\times 10^5 MPa$, the stress in the bar is

\hfill{\brak{\text{CE 2007}}}
\begin{enumerate}
\begin{multicols}{4}
\item $0$
\item $12MPa$
\item $24MPa$
\item $2400MPa$
\end{multicols}
\end{enumerate}

\item A rigid bar is suspended by three rods made of the same material as shown in the figure. The area and length of the central rod are $3A$ and $L$, respectively while that of the two outer rods are $2A$ and $2L$, respectively. If a downward force of $50kN$ is applied to the rigid bar, the forces in the central and each of the outer rods will be
\\\begin{center}
   \scalebox{0.5}{
   \usetikzlibrary{patterns}
\begin{tikzpicture}

\draw[pattern = north west lines] (-0.5,0.5) rectangle (0.5,0);
\draw[pattern = north west lines] (9.5,0.5) rectangle (10.5,0);
\draw[pattern = north west lines] (4.5,-4.5) rectangle (5.5,-5);

\draw[thick] (-0.5,0) -- (0.5,0);
\draw[thick] (9.5,0) -- (10.5,0);
\draw[thick] (4.5,-5) -- (5.5,-5);

\draw[thick] (0,0) -- (0,-10) -- (10,-10) -- (10,0);
\draw[thick] (5,-5) -- (5,-10);
\draw[thin] (0,-10) -- (10,-10) -- (10,-10.2) -- (0,-10.2) -- (0,-10);

\draw[thick, ->] (5,-10.2) -- (5,-10.5) node[below] {50 kN};

\fill[gray!30] (0,-10) rectangle (10,-10.2);

\end{tikzpicture}}
\end{center}

\hfill{\brak{\text{CE 2007}}}
\begin{enumerate}
\begin{multicols}{4}
\item $16.67kN$ each
\item $30kN$ and $15kN$
\item $30kN$ and $10kN$
\item $21.4kN$ and $14.3kN$
\end{multicols}
\end{enumerate}

\item The maximum and minimum shear stresses in a hollow circular shaft of outer diameter $20mm$ and thickness $2mm$, subjected to torque of $92.7Nm$ will be

\hfill{\brak{\text{CE 2007}}}
\begin{enumerate}
\item $59MPa$ and $47.2MPa$
\item $100MPa$ and $80MPa$
\item $118MPa$ and $160MPa$
\item $200MPa$ and $160MPa$
\end{enumerate}

\item The shear stress at the neutral axis in a beam of triangular section with a base of $40mm$ and height $20mm$, subjected to shear force of $3kN$ is

\hfill{\brak{\text{CE 2007}}}
\begin{enumerate}
\item $3MPa$
\item $6MPa$
\item $10MPa$
\item $20MPa$
\end{enumerate}

\item $U_1$ and $U_2$ are the strain energies stored in a prismatic bar due to axial tensile forces $P_1$ and $P_2$, respectively. The strain energy $U$ stored in the same bar due to combined action of $P_1$ and $P_2$ will be

\hfill{\brak{\text{CE 2007}}}
\begin{enumerate}
\item $U=U_1+U_2$
\item $U=U_1U_2$
\item $U<U_1+U_2$
\item $U>U_1+U_2$
\end{enumerate}



\iffalse
\title{EE-2024-27-39}
\author{EE24BTECH11041-Mohit}
\section{ee}
\chapter{2024}
\fi

\item If the following switching devices have similar power ratings, which one of them is the fastest?
\hfill{(EE 2024)}
\begin{enumerate}
\item SCR
\item GTO
\item IGBT
\item Power MOSFET
\end{enumerate}
\item  A single-phase triac based AC voltage controller feeds a series RL load. The input AC supply is $230 V$, $50 Hz$. The values of R and L are $10 \ohm$ and $18.37 mH$, respectively. The minimum triggering angle of the triac to obtain controllable output voltage is
\hfill{(EE 2024)}
\begin{enumerate}
\item $15^{\circ}$
\item $30^{\circ}$
\item $45^{\circ}$
\item $60^{\circ}$
\end{enumerate}
\item Let $X$ be a discrete random variable that is uniformly distributed over the set
\{-10, -9, \dots , 0, \dots , 9, 10\}. Which of the following random variables is/are
uniformly distributed?
\hfill{(EE 2024)}
\begin{enumerate}
\item $X^2$
\item $X^3$
\item $(X-5)^2$
\item $(X+10)^2$
\end{enumerate}
\item Which of the following complex functions is/are analytic on the complex plane?
\hfill{(EE 2024)}
\begin{enumerate}
\item $f(z)=Re(z)$
\item $f(z)=Im(z)$
\item $f(z)=e^{\abs{z}}$
\item $f(z)=z^2-z$
\end{enumerate}
\item Consider the complex function $f(z)= \cos{z}+e^{z^{2}}$.The cofficient of $z^5$ in the Taylor series expansion of $f(z)$ about the origin is \rule{2cm}{0.4pt} (rounded off to 1
decimal place).
\hfill{(EE 2024)}
\item The sum of the eigenvalues of the matrix $A=\myvec{1&2\\3&4}^2$ is \rule{2cm}{0.4pt} (rounded off to the nearest integer).
\hfill{(EE 2024)}
\item Let $X(w)$ be the Fourier transform of the signal
\begin{align}
x(t)=e^{-t^{4}}\cos{t},\qquad -\infty < t < \infty.
\end{align}
The value of the derivative of $X(w)$ at $w=0$ is \rule{2cm}{0.4pt} (rounded off to 1 decimal place).
\hfill{(EE 2024)}
\item The incremental cost curves of two generators (Gen A and Gen B) in a plant
supplying a common load are shown in the figure. If the incremental cost of supplying the common load is Rs. 7400 per $MWh$, then the common load in $MW$ is\rule{2cm}{0.4pt} (rounded off to the nearest integer).
\hfill{(EE 2024)}
\begin{center}
{\scalebox{0.4}{
\begin{circuitikz}
\tikzstyle{every node}=[font=\Large]
\draw [short] (13.75,13.5) -- (13.75,-6.5);
\draw [short] (15,13.5) -- (15,-6.5);
\draw [short] (16.25,13.5) -- (16.25,-6.5);
\draw [short] (17.5,13.5) -- (17.5,-6.5);
\draw [short] (18.75,13.5) -- (18.75,-6.5);
\draw [short] (20,13.5) -- (20,-6.5);
\draw [short] (21.25,13.5) -- (21.25,-6.5);
\draw [short] (22.5,13.5) -- (22.5,-6.5);
\draw [short] (23.75,13.5) -- (23.75,-6.5);
\draw [short] (25,13.5) -- (25,-6.5);
\draw [short] (26.25,13.5) -- (26.25,-6.5);
\draw [short] (27.5,13.5) -- (27.5,-6.5);
\draw [short] (28.75,13.5) -- (28.75,-6.5);
\draw [short] (30,13.5) -- (30,-6.75);
\draw [short] (31.25,13.5) -- (31.25,-6.5);
\draw [short] (32.5,13.5) -- (32.5,-6.5);
\draw [short] (33.75,13.5) -- (33.75,-6.5);
\draw [short] (35,13.5) -- (35,-6.5);
\draw [short] (36.25,13.5) -- (36.25,-6.5);
\draw [short] (36.25,-6.5) -- (13.75,-6.5);
\draw [short] (13.75,13.5) -- (36.25,13.5);
\draw [short] (13.75,-4) -- (36.5,-4);
\draw [short] (36.25,-1.5) -- (13.75,-1.5);
\draw [short] (36.25,1) -- (13.75,1);
\draw [short] (13.75,6) -- (36.25,6);
\draw [short] (36.25,3.5) -- (13.75,3.5);
\draw [short] (13.75,8.5) -- (36.25,8.5);
\draw [short] (36.25,11) -- (13.75,11);
\node [font=\Large] at (15,-7) {0};
\node [font=\Large] at (17.5,-7) {20};
\node [font=\Large] at (20,-7) {40};
\node [font=\Large] at (22.5,-7) {60};
\node [font=\Large] at (25,-7) {80};
\node [font=\Large] at (27.5,-7) {100};
\node [font=\Large] at (30,-7) {120};
\node [font=\Large] at (32.5,-7) {140};
\node [font=\Large] at (35,-7) {160};
\node [font=\Large] at (12.5,-6.5) {5000};
\node [font=\Large] at (12.5,-4) {6000};
\node [font=\Large] at (12.5,-1.5) {7000};
\node [font=\Large] at (12.5,1) {8000};
\node [font=\Large] at (12.5,3.5) {9000};
\node [font=\Large] at (12.5,6) {10000};
\node [font=\Large] at (12.5,8.5) {11000};
\node [font=\Large] at (12.5,11) {12000};
\node [font=\Large] at (12.5,13.5) {13000};
\node [font=\LARGE] at (24,-8.25) {Genration (MW)};
\node [font=\LARGE, rotate around={90:(0,0)}] at (9.25,3.5) {Incremental cost (rs/MWH))};
\draw [line width=1.5pt, dashed] (15,-4) -- (35,11.75);
\draw [line width=1.5pt, short] (15,1) -- (35,9);
\node [font=\Large, rotate around={45:(0,0)}] at (31.75,10.25) {Gen B};
\node [font=\Large] at (34.5,8) {Gen A};
\end{circuitikz}
}
}
\end{center}

\item A forced commutated thyristorized step-down chopper is shown in the figure. Neglect the ON-state drop across the power devices. Assume that the capacitor is initially charged to $50 V$ with the polarity shown in the figure. The load current ($I_l$ ) can be assumed to be constant at $10 A$. Initially, $Th_M$ is ON and $Th_A$ is OFF. The turn-off time available to $Th_M$ in microseconds, when $Th_A$ is triggered, is \rule{2cm}{0.4pt} (rounded off to the nearest integer).
\hfill{(EE 2024)}

\begin{center}
{\scalebox{0.7}{
\begin{circuitikz}
\tikzstyle{every node}=[font=\large]
\draw (6.25,14.75) to[battery1] (6.25,7.25);
\draw (6.25,14.75) to[D] (20,14.75);
\draw (17.5,7.25) to[D] (17.5,14.75);
\draw (20,14.75) to[european resistor] (20,7.25);
\draw (15,12.75) to[D] (12.75,12.75);
\draw (13.25,12.75) to[L ] (11.25,12.75);
\draw (11.25,11) to[D] (15,11);
\draw (15,14.75) to[curved capacitor] (15,12.75);
\draw (11.25,14.75) to[short] (11.25,11);
\draw (15,13) to[short] (15,11);
\draw (6.25,7.25) to[short] (20,7.25);
\draw [->, >=Stealth] (17.5,14.75) -- (19,14.75);
\node [font=\large] at (5,11) {50 V};
\node [font=\large] at (6.5,11.5) {+};

\node [font=\large] at (6.5,10.75) {-};
\node [font=\large] at (14.75,14.25) {+};
\node [font=\large] at (14.5,13.5) {-};
\node [font=\large] at (16.25,13.75) {10 $\mu$ F};
\node [font=\large] at (18.75,15.25) {$I_L=10 A$};
\node [font=\large] at (12.75,15.25) {$Th_M$};
\draw [short] (13.25,11) -- (13.5,11);
\draw [short] (13.25,14.75) -- (13.5,15);
\draw [short] (13.25,11) -- (13.5,11.25);
\node [font=\large] at (13,10.25) {$Th_A$};
\node [font=\large] at (21.5,11.5) {Constant};
\node [font=\large] at (21.25,11) {Current};
\node [font=\large] at (21.25,10.5) {Load};
\end{circuitikz}
}}
\end{center}
\item Consider a vector $\vec{u}=2\hat{x}+\hat{y}+2\hat{z}$, where $\hat{x},\hat{y},\hat{z}$ represent unit vectors along the coordinate axes $x,y,z$ respectively. The directional derivative of the function $f\brak{x,y,z} = 2\ln\brak{xy} +3\ln\brak{yz}+3\ln\brak{xz}$ at the point $\brak{x,y,z}=\brak{1,1,1}$ in the direction of $\vec{u}$ is
\hfill{(EE 2024)}
\begin{enumerate}
\item 0
\item $\frac{7}{5\sqrt{2}}$
\item 7
\item 21
\end{enumerate} 
\item The input  $x\brak{t}$ and the output $y\brak{t}$ of a system are related as
\hfill{(EE 2024)}
\begin{align}
y(t) = e^{-t} \int_{-infty}^{t} e^{\tau}x\brak{\tau}d\tau,\qquad -\infty < t < \infty.
\end{align}
The system is 
\hfill{(EE 2024)}
\begin{enumerate}
\item nonlinear.
\item linear and time-invariant.
\item linear but not time-invariant.
\item noncausal.
\end{enumerate}
\item Consider the discrete-time systems $T_1$ and $T_2$ defined as follows:
\begin{align}
\{T_1x\}[n] = x[0]+x[1]+\dots+x[n]\\
\{T_2x\}[n] = x[0]+\frac{1}{2}x[1]+\dots+\frac{1}{2^n}x[n]
\end{align}
Which one of the following statements is true?
\hfill{(EE 2024)}
\begin{enumerate}
\item $T_1$ and $T_2$ are BIBO stable.
\item $T_1$ and $T_2$ are not BIBO stable.
\item $T_1$ is BIBO stable and $T_2$ is not BIBO stable.
\item $T_1$ is not BIBO stable and $T_2$ is BIBO stable.
\end{enumerate}
\item If the Z-transform of a finite-duration discrete-time signal $x[n]$ is $X\brak{z}$, then the  Z-transformation of the signal $y[n]=x[2n]$ is
 \hfill{(EE 2024)}
\begin{enumerate}
\item $Y\brak{z}=X(z^{2})$
\item $Y\brak{z}=\frac{1}{2}[X(z^{-\frac{1}{2}})+X(-z^{-\frac{1}{2}})]$
\item $Y\brak{z}=\frac{1}{2}[X(z^{\frac{1}{2}})+X(-z^{\frac{1}{2}})]$
\item $Y\brak{z}=\frac{1}{2}[X(z^{2})+X(-z^{2})]$
\end{enumerate}


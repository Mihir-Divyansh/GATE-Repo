\iffalse
\title{2020-MA}
\author{EE24BTECH11020 -  Ellanti Rohith}
\section{ma}
\chapter{2020}
\fi
    \item  Let $\mathcal{P}\brak{\mathbb{R}}$ denote the power set of $\mathbb{R}$, equipped with the metric
    \begin{align*}
    d\brak{U, V} = \sup_{x \in \mathbb{R}} | \chi_U\brak{x} - \chi_V\brak{x} |,
    \end{align*}
    where $\chi_U$ and $\chi_V$ denote the characteristic functions of the subsets $U$ and $V$, respectively, of $\mathbb{R}$. The set $\cbrak{\cbrak{m} : m \in \mathbb{Z}}$ in the metric space $\brak{\mathcal{P}\brak{\mathbb{R}}, d}$ is
    \hfill{[GATE 2020]}\begin{multicols}{2}\begin{enumerate}
        \item bounded but not totally bounded
        \item totally bounded but not compact
        \item compact
        \item not bounded
    \end{enumerate}\end{multicols}

    \item  Let $f : \mathbb{R} \to \mathbb{R}$ be defined by
    \begin{align*}
    f\brak{x} = \sum_{n=0}^{\infty} \frac{1}{2^n} \chi_{\brak{n, n+1}}\brak{x},
    \end{align*}
    where $\chi_{\brak{n, n+1}}$ is the characteristic function of the interval $(n, n+1]$. For $\alpha \in \mathbb{R}$, let $S_\alpha = \cbrak{ x \in \mathbb{R} : f\brak{x} > \alpha }$. Then
    \hfill{[GATE 2020]}\begin{multicols}{2}\begin{enumerate}
        \item $S_{\frac{1}{2}}$ is open
        \item $S_{\frac{\sqrt{3}}{2}}$ is not measurable
        \item $S_0$ is closed
        \item $S_{\frac{1}{\sqrt{2}}}$ is measurable
    \end{enumerate}\end{multicols}

    \item  For $n \in \mathbb{N}$, let $f_n, g_n : \brak{0, 1} \to \mathbb{R}$ be functions defined by
    \begin{align*}
    f_n\brak{x} = x^n \quad \text{and} \quad g_n\brak{x} = x^n \brak{1 - x}.
    \end{align*}
    Then
    \hfill{[GATE 2020]}\begin{enumerate}
        \item $\cbrak{f_n}$ converges uniformly but $\cbrak{g_n}$ does not converge uniformly
        \item $\cbrak{g_n}$ converges uniformly but $\cbrak{f_n}$ does not converge uniformly
        \item both $\cbrak{f_n}$ and $\cbrak{g_n}$ converge uniformly
        \item neither $\cbrak{f_n}$ nor $\cbrak{g_n}$ converge uniformly
    \end{enumerate}

    \item  Let $u$ be a solution of the differential equation $y' + xy = 0$ and let $\phi = u\psi$ be a solution of the differential equation $y'' + 2xy' + \brak{x^2 + 2}y = 0$ satisfying $\phi\brak{0} = 1$ and $\phi'\brak{0} = 0$. Then $\phi\brak{x}$ is
    \hfill{[GATE 2020]}\begin{multicols}{2}\begin{enumerate}
        \item $\brak{\cos^2 x}e^{-\frac{x^2}{2}}$
        \item $\brak{\cos x}e^{-\frac{x^2}{2}}$
        \item $\brak{1 + x^2}e^{-\frac{x^2}{2}}$
        \item $\brak{\cos x}e^{-x^2}$
    \end{enumerate}\end{multicols}
     \item For $n \in \mathbb{N} \cup \cbrak{0}$, let $y_n$ be a solution of the differential equation
    \begin{align*}
    xy'' + \brak{1 - x}y' + n \, y = 0
    \end{align*}
    satisfying $y_n\brak{0} = 1$. For which of the following functions $w\brak{x}$, the integral
    \begin{align*}
    \int_0^{\infty} y_p\brak{x} \, y_q\brak{x} \, w\brak{x} \, dx, \quad \brak{p \neq q}
    \end{align*}
    is equal to zero?
    \hfill{[GATE 2020]}\begin{multicols}{2}\begin{enumerate}
        \item $e^{-x^2}$
        \item $e^{-x}$
        \item $x e^{-x^2}$
        \item $x e^{-x}$
    \end{enumerate}\end{multicols}

    \item  Suppose that
    \begin{align*}
    X = \cbrak{\brak{0,0}} \cup \cbrak{ \brak{ x, \sin \dfrac{1}{x} } : x \in \mathbb{R} \setminus \cbrak{0} }
    \end{align*}
    and
    \begin{align*}
    Y = \cbrak{\brak{0,0}} \cup \cbrak{ \brak{ x, x \sin \dfrac{1}{x} } : x \in \mathbb{R} \setminus \cbrak{0} }
    \end{align*}
    are metric spaces with metrics induced by the Euclidean metric of $\mathbb{R}^2$. Let $B_X$ and $B_Y$ be the open unit balls around $\brak{0,0}$ in $X$ and $Y$, respectively. Consider the following statements:
    
    \hfill{[GATE 2020]}\begin{multicols}{2}\begin{enumerate}
        \item[I.] The closure of $B_X$ in $X$ is compact.
        \item[II.] The closure of $B_Y$ in $Y$ is compact.
    \end{enumerate}\end{multicols}
    \begin{multicols}{2}\begin{enumerate}
        \item both I and II are true
        \item I is true II is false 
         \item I is false II is true 
         \item both I and II are false
    \end{enumerate}\end{multicols}
    
\item If $f: \mathbb{C} \setminus \cbrak{0} \to \mathbb{C}$ is a function such that $f\brak{z} = f\brak{ \dfrac{z}{|z|}}$ and its restriction to the unit circle is continuous, then
    \hfill{[GATE 2020]}
       \begin{enumerate}
            \item $f$ is continuous but not necessarily analytic
            \item $f$ is analytic but not necessarily a constant function
            \item $f$ is a constant function
            \item $\lim\limits_{z \to 0} f\brak{z}$ exists
        \end{enumerate}

    \item  For a subset $S$ of a topological space, let $\text{Int}\brak{S}$ and $\overline{S}$ denote the interior and closure of $S$, respectively. Then which of the following statements is TRUE?
    \hfill{[GATE 2020]}
      \begin{enumerate}
            \item If $S$ is open, then $S = \text{Int}\brak{S}$
            \item If the boundary of $S$ is empty, then $S$ is open
            \item If the boundary of $S$ is empty, then $S$ is not closed
            \item If $\overline{S} \setminus S$ is a proper subset of the boundary of $S$, then $S$ is open
        \end{enumerate}
    
    \item  Suppose $\mathcal{T}_1, \mathcal{T}_2$ and $\mathcal{T}_3$ are the smallest topologies on $\mathbb{R}$ containing $S_1, S_2$ and $S_3$, respectively, where
    \begin{align*}
    S_1 = \cbrak{ \brak{ a, a + \dfrac{\pi}{n} } : a \in \mathbb{Q}, n \in \mathbb{N} },
    \end{align*}
    \begin{align*}
    S_2 = \cbrak{ \brak{a, b} : a < b, \; a, b \in \mathbb{Q} },
    \end{align*}
    \begin{align*}
    S_3 = \cbrak{ \brak{a, b} : a < b, \; a, b \in \mathbb{R} }.
    \end{align*}
    Then\hfill{[GATE 2020]}

        \begin{multicols}{2}\begin{enumerate}
            \item $\mathcal{T}_3 \supseteq \mathcal{T}_1$
            \item $\mathcal{T}_3 \supseteq \mathcal{T}_2$
            \item $\mathcal{T}_1 = \mathcal{T}_2$
            \item $\mathcal{T}_1 \subseteq \mathcal{T}_2$
        \end{enumerate}
        \end{multicols}
  

  
  
   \item     Let $ M = \myvec{\alpha & 3 & 0 \\ \beta & 3 & 1 \\ 0 & 1 & 2 } $. Consider the following statements:
    
    \begin{enumerate}
        \item[I:] There exists a lower triangular matrix $ L $ such that $ M = L L^t $, where $ L^t $ denotes the transpose of $ L $.
        \item[II:] Gauss-Seidel method for $ Mx = b $ $\brak{ b \in \mathbb{R}^3 }$ converges for any initial choice $ x_0 \in \mathbb{R}^3 $.
    \end{enumerate}
    Then:
    
    \hfill{[GATE 2020]}
    \begin{multicols}{2}
    \begin{enumerate}
        \item I is not true when $ \alpha > \dfrac{9}{2}, \beta = 3 $
        
        \item I is true when $ \alpha = 5, \beta = 3 $
        \item II is not true when $ \alpha > \dfrac{9}{2}, \beta = -1 $
        \item II is not true when $ \alpha = 4, \beta = \dfrac{3}{2} $
        
    \end{enumerate}
    \end{multicols}

    \item    Let $ I $ and $ J $ be the ideals generated by $ \cbrak{5, \sqrt{10}} $ and $ \cbrak{4, \sqrt{10}} $ in the ring $ \mathbb{Z}[\sqrt{10}] = \cbrak{a + b\sqrt{10} \mid a, b \in \mathbb{Z}} $, respectively. Then:
        \hfill{[GATE 2020]}\begin{multicols}{2}\begin{enumerate}
        \item both $ I $ and $ J $ are maximal ideals
        \item $ I $ is a maximal ideal but $ J $ is not a prime ideal
        \item $ I $ is not a maximal ideal but $ J $ is a prime ideal
        \item neither $ I $ nor $ J $ is a maximal ideal
    \end{enumerate}\end{multicols}
   

    \item   Suppose $ V $ is a finite dimensional vector space over $ \mathbb{R} $. If $ W_1, W_2 $ and $ W_3 $ are subspaces of $ V $, then which of the following statements is \textbf{TRUE}?
   
    \hfill{[GATE 2020]}\begin{enumerate}
        \item If $ W_1 + W_2 + W_3 = V $ then $ \text{span}\brak{W_1 \cup W_2} \cup \text{span}\brak{W_2 \cup W_3} \cup \text{span}\brak{W_3 \cup W_1} = V $
        \item If $ W_1 \cap W_2 = \cbrak{0} $ and $ W_1 \cap W_3 = \cbrak{0} $, then $ W_1 \cap \brak{W_2 + W_3} = \cbrak{0} $
        \item If $ W_1 + W_2 = W_1 + W_3 $, then $ W_2 = W_3 $
        \item If $ W_1 \neq V $, then $ \text{span}\brak{V \setminus W_1} = V $
    \end{enumerate}
    

    \item    Let $ \alpha, \beta \in \mathbb{R}, \alpha \neq 0 $. The system 
    \begin{align*}
    x_1 - 2x_2 + \alpha x_3 &= 8 \\
    x_1 - x_2 + x_4 &= \beta \\
    x_1, x_2, x_3, x_4 &\geq 0
\end{align*}

    has NO basic feasible solution if:
    
    \hfill{[GATE 2020]}\begin{multicols}{2}\begin{enumerate}
        \item $ \alpha < 0, \beta > 8 $
        \item $ \alpha > 0, 0 < \beta < 8 $
        \item $ \alpha > 0, \beta < 0 $
        \item $ \alpha < 0, \beta < 8 $
    \end{enumerate}
    \end{multicols}


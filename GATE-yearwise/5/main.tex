\iffalse
\chapter{2013}
\author{EE24BTECH11003}
\section{xe}
\fi

\item Flow past a circular cylinder can be produced by superposition of the following elementary potential flows:
\hfill{\brak{2013}}
\begin{enumerate}
\item Uniform flow, doublet
\item Uniform flow, vortex
\item Source, vortex
\item Sink, vortex
\end{enumerate}

\item Let $\delta$, $\delta_1$ and $\delta_2$ denote respectively the boundary-layer thickness, displacement thickness and the momentum thickness for laminar boundary layer flow of an incompressible fluid over a flat plate. The correct relation among these quantities is
\hfill{\brak{2013}}
\begin{enumerate}
\item $\delta < \delta_1 < \delta_2$
\item $\delta > \delta_1 > \delta_2$
\item $\delta > \delta_1 < \delta_2$
\item $\delta < \delta_1 > \delta_2$
\end{enumerate}

\item In the hydrodynamic entry region of a circular duct, the pressure forces balance the sum of
\hfill{\brak{2013}}
\begin{enumerate}
\item viscous and buoyancy forces
\item inertia and buoyancy forces
\item inertia and surface tension forces
\item inertia and viscous forces
\end{enumerate}

\item Bodies with various cross-sectional shapes subjected to cross-flow of air are shown in the following figures. The characteristic dimension of all the shapes is the same. The cross-sectional shape with the largest coefficient of drag $\brak{\text{i.e. sum of the pressure and skin-friction drags}}$, at any moderately large Reynolds number, is
\hfill{\brak{2013}}
\begin{enumerate}
\item 
\begin{circuitikz}
\tikzstyle{every node}=[font=\LARGE]
\draw [->, >=Stealth] (5,15.5) -- (9,15.5);
\draw [->, >=Stealth] (5,14.75) -- (9,14.75);
\draw [->, >=Stealth] (5,14) -- (9,14);
\draw [->, >=Stealth] (5,12.5) -- (9,12.5);
\draw [->, >=Stealth] (5,13.25) -- (9,13.25);
\draw [short] (11,14.75) -- (15.25,14);
\draw [short] (11,13) -- (15.25,14);
\draw [short] (11,14.75) .. controls (10,14.25) and (10,13.25) .. (11,13);
\end{circuitikz}
\begin{circuitikz}
\tikzstyle{every node}=[font=\LARGE]
\draw [->, >=Stealth] (5,15.5) -- (9,15.5);
\draw [->, >=Stealth] (5,14.75) -- (9,14.75);
\draw [->, >=Stealth] (5,14) -- (9,14);
\draw [->, >=Stealth] (5,12.5) -- (9,12.5);
\draw [->, >=Stealth] (5,13.25) -- (9,13.25);
\draw  (12.5,14) ellipse (2.5cm and 1.25cm);
\end{circuitikz}i
\item \begin{circuitikz}
\tikzstyle{every node}=[font=\LARGE]
\draw [->, >=Stealth] (5,15.5) -- (9,15.5);
\draw [->, >=Stealth] (5,14.75) -- (9,14.75);
\draw [->, >=Stealth] (5,14) -- (9,14);
\draw [->, >=Stealth] (5,12.5) -- (9,12.5);
\draw [->, >=Stealth] (5,13.25) -- (9,13.25);
\draw  (10.5,15.25) rectangle (13.5,12.75);
\end{circuitikz}
\begin{circuitikz}
\tikzstyle{every node}=[font=\LARGE]
\draw [->, >=Stealth] (5,15.5) -- (9,15.5);
\draw [->, >=Stealth] (5,14.75) -- (9,14.75);
\draw [->, >=Stealth] (5,14) -- (9,14);
\draw [->, >=Stealth] (5,12.5) -- (9,12.5);
\draw [->, >=Stealth] (5,13.25) -- (9,13.25);
\draw  (11.5,14) circle (1.5cm);
\end{circuitikz}
\end{enumerate}

\item A U-tube of a very small bore, with its limbs in a vertical plane and filled with a liquid of density $\rho$, up to a height of $h$, is rotated about a vertical axis, with an angular velocity of $\omega$, as shown in the Figure. The radius of each limb from the axis of rotation is $R$. Let $p_a$ be the atmospheric pressure and $g$, the gravitational acceleration. The angular velocity at which the pressure at the point $O$ becomes half of the atmospheric pressure is given by
\begin{center}
\begin{circuitikz}
\tikzstyle{every node}=[font=\Huge]
\draw [short] (8.75,15.5) -- (8.75,10);
\draw [short] (8.75,10) -- (14,10);
\draw [short] (14,15.5) -- (14,10);
\draw [short] (9,15.5) -- (9,10.25);
\draw [short] (9,10.25) -- (13.75,10.25);
\draw [short] (13.75,10.25) -- (13.75,15.5);
\draw [short] (8.75,14.5) -- (9,14.5);
\draw [short] (13.75,14.5) -- (14,14.5);
\draw [->, >=Stealth] (10.75,12.5) .. controls (10.75,11.75) and (12,11.5) .. (12.25,12.5) ;
\draw [dashed] (11.5,15.5) -- (11.5,8.75);
\draw [<->, >=Stealth] (14.75,14.5) -- (14.75,10);
\draw [<->, >=Stealth] (11.5,9.5) -- (14,9.5);
\draw [short] (14.25,14.5) -- (15.25,14.5);
\draw [short] (14.25,10) -- (15.25,10);
\draw [short] (14,9.75) -- (14,9);
\node [font=\large] at (11,10.5) {O};
\node [font=\Large] at (12.25,12.75) {$\omega$};
\node [font=\Large] at (15.25,12) {h};
\node [font=\Large] at (12.75,9) {R};
\node [font=\Huge] at (11.5,10.25) {\textbf{.}};
\end{circuitikz}
\end{center}
\hfill{\brak{2013}}
\begin{enumerate}
\item $\sqrt{\frac{p_a + 2\rho gh}{\rho R^2}}$
\item $\sqrt{\frac{2\brak{p_a + \rho gh}}{\rho R^2}}$
\item $\sqrt{\frac{p_a + 2\rho gh}{2\rho R^2}}$
\item $\sqrt{\frac{p_a + \rho gh}{2\rho R^2}}$
\end{enumerate}

\item An incompressible fluid at a pressure of $150$ kPa $\brak{\text{absolute}}$ flows steadily through a two-dimensional channel with a velocity of $5$ m/s as shown in the Figure. The channel has a $90^{\circ}$ bend. The fluid leaves the channel with a pressure of $100$ kPa $\brak{\text{absolute}}$ and linearly-varying velocity profile. $v_{max}$ is four times $v_{min}$. The density of the fluid is $914.3$ kg/$m^3$. The velocity $v_{min}$, in m/s, is
\begin{circuitikz}
\tikzstyle{every node}=[font=\Large]
\draw [short] (9.5,12.25) -- (13.25,12.25);
\draw [short] (9.5,10) -- (15.25,10);
\draw [short] (13.25,12.25) -- (13.25,16.25);
\draw [short] (15.25,16.25) -- (15.25,10);
\draw [short] (13.25,16.25) -- (15.25,16.25);
\draw [short] (13.25,19.5) -- (15.25,17.5);
\draw [->, >=Stealth] (7.75,12.25) -- (9.5,12.25);
\draw [->, >=Stealth] (7.75,12) -- (9.5,12);
\draw [->, >=Stealth] (7.75,11.75) -- (9.5,11.75);
\draw [->, >=Stealth] (7.75,11.5) -- (9.5,11.5);
\draw [->, >=Stealth] (7.75,11.25) -- (9.5,11.25);
\draw [->, >=Stealth] (7.75,11) -- (9.5,11);
\draw [->, >=Stealth] (7.75,10.75) -- (9.5,10.75);
\draw [->, >=Stealth] (7.75,10.5) -- (9.5,10.5);
\draw [->, >=Stealth] (7.75,10.25) -- (9.5,10.25);
\draw [->, >=Stealth] (7.75,10) -- (9.5,10);
\draw [->, >=Stealth] (13.25,16.25) -- (13.25,19.5);
\draw [->, >=Stealth] (13.75,16.25) -- (13.75,19);
\draw [->, >=Stealth] (15.25,16.25) -- (15.25,17.5);
\draw [->, >=Stealth] (14.75,16.25) -- (14.75,18);
\draw [->, >=Stealth] (14.25,16.25) -- (14.25,18.5);
\draw [short] (7.75,12.25) -- (7.75,10);
\draw [short] (9.5,12.25) -- (9.5,10);
\draw [->, >=Stealth] (15,16.25) -- (15,17.75);
\draw [->, >=Stealth] (14.5,16.25) -- (14.5,18.25);
\draw [->, >=Stealth] (14,16.25) -- (14,18.75);
\draw [->, >=Stealth] (13.5,16.25) -- (13.5,19.25);
\draw [<->, >=Stealth] (13.25,14.25) -- (15.25,14.25);
\draw [<->, >=Stealth] (10.75,12.25) -- (10.75,10);
\node [font=\Large] at (6.5,11.25) {5 m/s};
\node [font=\Large] at (12,11.25) {50 mm};
\node [font=\Large] at (14.25,14.75) {60 mm};
\node [font=\Large] at (12.5,17.75) {$V_{max}$};
\node [font=\Large] at (16,16.75) {$V_{min}$};
\end{circuitikz}
\hfill{\brak{2013}}
\begin{enumerate}
\item $25$
\item $2.5$
\item $2.0$
\item $0.2$
\end{enumerate}

\item The velocity vector corresponding to a flow field is given, with usual notation, by $\vec{V} = 3x\vec{i} + 4xy\vec{j}$. The magnitude of rotation at the point $\brak{2,2}$ in rad/s is
\hfill{\brak{2013}}
\begin{enumerate}
\item $0.75$
\item $1.33$
\item $2$
\item $4$
\end{enumerate}

\item The stream function for a potential flow field is given by $\psi = x^2 - y^2$. The corresponding potential function, assuming zero potential at the origin, is
\hfill{\brak{2013}}
\begin{enumerate}
\item $x^2 + y^2$
\item $2xy$
\item $x^2 - y^2$
\item $x - y$
\end{enumerate}

\item Fully developed flow of an oil takes place in a pipe of inner diameter $50$ mm. The pressure drop per metre length of the pipe is $2$ kPa. Determine the shear stress, in Pa, at the pipe wall.$\rule{2cm}{0.1pt}$
\hfill{\brak{2013}}

\item The Darcy friction factor $f$ for a smooth pipe is given by $f = \frac{64}{Re}$ for laminar flow and by $f = \frac{0.3}{Re^{0.25}}$ for turbulent flow, where $Re$ is the Reynolds number based on the diameter. For fully developed flow of a fluid of density $1000$ kg/$m^3$ and dynamic viscosity $0.001$ Pa.s through a smooth pipe of diameter $10$ mm with a velocity of $1$ m/s, determine the Darcy friction factor.$\rule{2cm}{0.1pt}$
\hfill{\brak{2013}}

\item Air flows steadily through a channel. The stagnation and static pressures at a point in the flow are measured by a Pitot tube and a wall pressure tap, respectively. The pressure difference is found to be $20$ mm Hg. The densities of air, water and mercury, in kg/$m^3$, are $1.18$, $1000$ and $13600$, respectively. The gravitational acceleration is $9.81$ m/$s^2$. Determine the air speed in m/s.$\rule{2cm}{0.1pt}$
\hfill{\brak{2013}}\\

\textbf{Common Data for Questions 12 and 13:}\\

The velocity field within a laminar boundary layer is given by the expression:
\begin{align*}
\vec{V} = \frac{Bu_{\infty}y}{x^{\frac{3}{2}}}\vec{i} + \frac{Bu_{\infty}y^{2}}{4x^{\frac{5}{2}}}\vec{j} 
\end{align*}
where $B = 100m^{\frac{1}{2}}$ and the free stream velocity $u_{\infty} = 0.1$ m/s.

\item Calculate the x-direction component of the acceleration in m/$s^2$ at the point $x = 0.5$ m and $y = 50$ mm.$\rule{2cm}{0.1pt}$
\hfill{\brak{2013}}

\item Find the slope of the streamline passing through the point $x = 0.5$ m and $y = 50$ mm.$\rule{2cm}{0.1pt}$
\hfill{\brak{2013}}


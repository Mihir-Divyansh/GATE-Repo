\iffalse
\chapter{2011}
\author{AI24BTECH11022}
\section{xe}
\fi

\item The integral $\oint_{c}\frac{z^{3}e^{z}}{\brak{z - 1}^{3}}dz$ the curve $C:\abs{z}=\frac{\pi}{2}$, oriented counter-clockwise, equals\hfill(2011)
\begin{multicols}{2}
\begin{enumerate}
\item $0$
\item $2\pi i$
\item $13e\pi i$
\item $20e\pi i$
\end{enumerate}
\end{multicols}


\item Consider the function $f\brak{x,y,z}=x^{3}e^{y}\sin{z}$ and the point $P=\brak{1,0,\frac{\pi}{2}}$. The value of $f$ DOES NOT change due to a small displacement of P along the direction of\hfill(2011)
\begin{multicols}{2}
\begin{enumerate}
\item $\sbrak{1,0,\frac{\pi}{2}}$
\item $\sbrak{1,-1,1}$
\item $\sbrak{1,-3,0}$
\item $\sbrak{2,0,-1}$
\end{enumerate}
\end{multicols}


\item A solution of the differential equation $\frac{d^{2}y}{dx^{2}}-5\frac{dy}{dx}+6y=36x$ is\hfill(2011)
\begin{multicols}{2}
\begin{enumerate}
\item $e^{2x}+e^{3x}+6x+5$
\item $e^{-3x}+6x+5$
\item $e^{2x}+e^{-3x}+6x+\frac{5}{6}$
\item $e^{2x}+e^{3x}+x+\frac{5}{6}$
\end{enumerate}
\end{multicols}

\item For any positive numbers $a$ and $b$, the matrix $$P=\begin{bmatrix}
1\\a\\b\\
\end{bmatrix}
\begin{bmatrix}
4&5&6\\
\end{bmatrix}$$ is\hfill(2011)
\begin{multicols}{2}
\begin{enumerate}
\item orthogonal
\item diagonalizable
\item nonsingular
\item of rank $2$
\end{enumerate}
\end{multicols}


\item Suppose $x_{n}$ is the $n-th$ iterated value while finding the positive square root of $7$ by the Newton-Raphson method with a positive initial guess $x_{0}\brak{\neq \sqrt{7}}$. If $e_{n}=\sqrt{7}-x_{n}$ for $n\geq 1$, then\hfill(2011)
\begin{multicols}{2}
\begin{enumerate}
\item $e_{n+1}=\frac{e_{n}}{2x_{n}^{2}}$
\item $e_{n+1}=\frac{-\sqrt{7}e_{n}^{2}}{2x_{n}}$
\item $e_{n+1}=\frac{e_{n}^{2}}{\sqrt{7}}$
\item $e_{n+1}=\frac{-e_{n}^{2}}{2x_{n}}$
\end{enumerate}
\end{multicols}

\item The solution of the initial boundary value problem partial $$\frac{\partial u}{\partial t}=\frac{\partial^{2}u}{\partial x^{2}},0<x<\pi,t>0,\text{with boundary and initial conditions}$$ $$\frac{\partial u}{\partial x}\brak{0,t}=0=u\brak{\pi,t},t>0\text{ and }u\brak{x,0}=f\brak{x},0<x<\pi\text{ is}$$\hfill(2011)
\begin{enumerate}
\item $u\brak{x,t}=\sum_{n=0}^{\infty}A_{n}exp\brak{-\brak{\frac{2n+1}{2}}^{2}t}\cos{\brak{\frac{2n+1}{2}x}}$, with $A_{n}=\frac{2}{\pi}\int_{0}^{\pi}f\brak{x}\cos{\brak{\frac{2n+1}{2}x}}dx$
\item $u\brak{x,t}=\sum_{n=0}^{\infty}A_{n}exp\brak{-n^{2}t}\cos{\brak{nx}}$, with $A_{n}=\frac{2}{\pi}\int_{0}^{\pi}f\brak{x}\cos{\brak{nx}}dx$
\item $u\brak{x,t}=\sum_{n=1}^{\infty}A_{n}exp\brak{-\brak{\frac{2n+1}{2}}^{2}t}\sin{\brak{\frac{2n+1}{2}x}}$, with $A_{n}=\frac{2}{\pi}\int_{0}^{\pi}f\brak{x}\sin{\brak{\frac{2n+1}{2}x}}dx$
\item $u\brak{x,t}=\sum_{n=1}^{\infty}A_{n}exp\brak{-n^{2}t}\sin{\brak{nx}}$, with $A_{n}=\frac{2}{\pi}\int_{0}^{\pi}f\brak{x}\sin{\brak{nx}}dx$
\end{enumerate}

\item The function $f\brak{x}$ defined by $f(x)=
\begin{cases} 
3-x^{2},&x\leq1,\\ 
3-x,&1<x\leq 2,\\ 
x-1,&x>2. 
\end{cases}
$ has\hfill(2011)
\begin{enumerate}
\item a local maxima at $x=3$ and a local minima at $x=0$
\item a local maxima at $x=0$ and no local minima
\item a local maxima at $x=0$ and a local minima at $x=2$
\item no local maxima and a local minima at $x=1$
\end{enumerate}


\item In a biased die experiment, the random variable $x$ of the outcome has the (cumulative) distribution function $F\brak{x}$ shown below.\\
\begin{tikzpicture}
\draw[->] (0,0) -- (7,0) node[right] {$x$};
\draw[->] (0,0) -- (0,6) node[above] {$F(x)$};
\node[left] at (0,5) {1};
\node[left] at (0,4.5) {0.9};
\node[left] at (0,3.5) {0.7};
\node[left] at (0,2.5) {0.5};
\node[left] at (0,1.5) {0.3};
\node[left] at (0,0.5) {0.1};
\foreach \x in {1,2,3,4,5,6}
\node[below] at (\x,0) {\x};
\draw (1,0) -- (1,0.5) -- (2,0.5) -- (2,1.5) -- (3,1.5) -- (3,2.5)-- (4,2.5) -- (4,3.5) -- (5,3.5) -- (5,4.5) -- (6,4.5) -- (6,5) -- (7,5);
\end{tikzpicture}

The variance of $x$ is\hfill(2011)
\begin{multicols}{2}
\begin{enumerate}
\item $1.5$
\item $2.25$
\item $3.5$
\item $4.25$
\end{enumerate}
\end{multicols}


\item For a boundary layer on a flat plate \rule{1cm}{0.15mm} forces and \rule{1cm}{0.15mm} forces are of the same order of magnitude\hfill(2011)
\begin{multicols}{2}
\begin{enumerate}
\item body, inertia
\item viscous, body
\item inertia, viscous
\item viscous, pressure
\end{enumerate}
\end{multicols}


\item The temperature field in a fluid flow is given by $(60-0.2xy)\degree C$. The velocity field is $\overrightarrow{V}=2xy\hat{i}+ty\hat{j}m/s$. The rate of change of the temperature measured by a thermometer moving along with the flow at $\brak{2,-4}m$ at $t=4s$ is\hfill(2011)
\begin{multicols}{2}
\begin{enumerate}
\item $-12.8\degree C/s$
\item $-10.6\degree C/s$
\item $-6.4\degree C/s$
\item $-4.8\degree C/s$
\end{enumerate}
\end{multicols}


\item Two tanks $A$ and $B$, with the same height are filled with water till the top. The volume of tank $A$ is $10$ times the volume of tank $B$. What can you say about the pressures $p_{A}$ and $p_{B}$ at the bottom of the tanks $A$ and $B$ respectively?\hfill(2011)
\begin{enumerate}
\item $p_{A}=10p_{B}$
\item $p_{B}=10p_{A}$
\item $p_{A}=p_{B}$
\item Additional data is required to compare the two pressures
\end{enumerate}


\item A velocity field in a plane flow is given by $\overrightarrow{V}=2xy\hat{i}+3y\hat{j}$. The vorticity at the point $\brak{2,4}m$ is\hfill(2011)
\begin{multicols}{2}
\begin{enumerate}
\item $-4\hat{k}rad/s$
\item $-3\hat{j}rad/s$
\item $-2\hat{k}rad/s$
\item $-3\hat{i}rad/s$
\end{enumerate}
\end{multicols}


\item Separation is said to occur at a wall when \rule{1cm}{0.15mm} at the wall becomes zero.\hfill(2011)
\begin{multicols}{2}
\begin{enumerate}
\item internal energy
\item pressure
\item shear stress
\item density
\end{enumerate}
\end{multicols}
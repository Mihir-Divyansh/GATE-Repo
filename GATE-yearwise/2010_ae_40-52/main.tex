\iffalse
\chapter{2010}
\author{EE24BTECH11005}
\section{ae}
\fi
\item In finding a root of the equation: $x^2-6x+5=0$ the Newton-Ralphson method achieves an order of convergance equal to,
	\begin{enumerate}
			\begin{multicols}{2}
			\item $1.0$
				\columnbreak
			\item $1.67$
			\end{multicols}
			\begin{multicols}{2}
			\item $2.0$
				\columnbreak
			\item $2.5$
			\end{multicols}
	\end{enumerate}
\item Consider a $1-D$ adiabatic, inviscid, compressible Now of air $\brak{R = 287 J/Kg-K, c_r = 718 J/Kg-K}$ through a duct of constant cross-sectional area $A=1 m$. If the volumetric flow rate is $Q=680 m^3/s$ and stagnation temperature is $T, = 580.05 K$, then the air temperature inside the duct is
	\begin{enumerate}
			\begin{multicols}{2}
			\item $300K$
				\columnbreak
			\item $350K$
			\end{multicols}
			\begin{multicols}{2}
			\item $400K$
				\columnbreak
			\item $450K$ 
			\end{multicols}
	\end{enumerate}
\item A two stage chemical rocket, having the same specific impulse $\brak{l_{sp}}$ of $300 s$ for both the stages is designed in such a way that the payload ratio and the structural ratio are same for both the stages. The second stage of the rocket has following mass distribution:\\
	Propellant Mass = $10208 kg$\\
	Structural Mass = $1134 kg$\\
	Payload Mass = $1700 kg$
	$g_e = 9.8 m/s2$\\
	If the rocket is fired from rest and it flies in a zero gravity field and a drag free environment, the final velocity attained by the payload is
	\begin{enumerate}
			\begin{multicols}{2}
			\item  $9729.3m/s$
				\columnbreak
			\item  $897.3m/s$
			\end{multicols}
			\begin{multicols}{2}
			\item $9360.2m/s$ 
				\columnbreak
			\item $8973.2m/s$
			\end{multicols}
	\end{enumerate}
\item A missile with a Ramjet engine is flying in air. The temperature at the inlet and the outlet of the combustor are $1200 K$ and $2500 K$ respectively. The heating value of the fuel is $43 MJ/kg$ and the burner efficiency is $90\%$. Considering the working fluid to be air $\brak{C_p = 1005 J/kgK, \gamma = 1.4}$, The for this engine is equal to: $\frac{\text{fuel}}{\text{air}}$ ratio $\brak{f=\frac{m_f}{m_a}}$
	\begin{enumerate}
			\begin{multicols}{2}
			\item $0.032$
				\columnbreak
			\item $0.036$
			\end{multicols}
			\begin{multicols}{2}
			\item $0.042$
				\columnbreak
			\item $0.026$
			\end{multicols}
	\end{enumerate}
\item The trim curves of an aircraft are of the form $C_m \brak{0.05-0.2\delta_r}-0.1C_l$, where the elevator deflection angle, $\delta_e$, is in radians. The change in elevator deflection needed to increase the lift coefficient from $0.4$ to $0.9$ is:
	\begin{enumerate}
			\begin{multicols}{2}
			\item $-0.5$ radians
				\columnbreak
			\item $-0.25$ radians
			\end{multicols}
			\begin{multicols}{2}
			\item $0.25$ radians
				\columnbreak
			\item $0.5$ radians
			\end{multicols}
	\end{enumerate}
\item If $e$ is the base of the natural logarithms then the equation of the tangent from the origin to the curve $y=e^x$ is
	\begin{enumerate}
			\begin{multicols}{2}
			\item $y=x$
				\columnbreak
			\item $y=\pi x$
			\end{multicols}
			\begin{multicols}{2}
			\item $y=\frac{x}{e}$
				\columnbreak
			\item $y=ex$
			\end{multicols}
	\end{enumerate}
\item Consider a potential flow over a finite wing with the following circulation distribution
	\begin{figure}[H]
		\centering
		\begin{tikzpicture}

			\draw [-Triangle ] (5,8.5) -- (12.25,15.75);
			\draw  (5,8.5) -- (15,8.5);
			\draw (10,13.5) -- (16.25,13.5);
			\draw [dashed] (7.5,11) -- (7.5,14.75);
			\draw [-Triangle] (7.5,11) -- (7.5,12.5);
			\draw [-Triangle ] (7.5,11) -- (11.25,11);
			\draw [-Triangle ] (1.25,14.75) -- (2.5,14.75);
			\draw [-Triangle ] (1.25,13.5) -- (2.5,13.5);
			\draw [-Triangle ] (1.25,12.25) -- (2.5,12.25);
			\draw [-Triangle ] (1.25,11) -- (2.5,11);
			\draw [-Triangle ] (1.25,9.75) -- (2.5,9.75);
			\draw [-Triangle ] (1.25,8.5) -- (2.5,8.5);

			\draw [-Triangle, thick] (14,13.5) arc[start angle=0, end angle=-270, radius=0.6cm];
			\draw [-Triangle, thick] (13,8.5) arc[start angle=0, end angle=270, radius=0.6cm];
			\draw [dashed, thick] (7.5,14.75) .. controls (6,12.75) .. (5,8.5);
			\draw [dashed, thick] (7.5,14.75) .. controls (9,14.25) .. (10,13.5);
			\node [font=\LARGE] at (4,12.75) {$\Gamma=\Gamma(y)$};
			\node [font=\LARGE] at (7,15.75) {$\Gamma_0=100m^2/s$};
			\node [font=\LARGE] at (11,13) {$y=2m$};
			\node [font=\LARGE] at (13,15.75) {$y$};
			\node [font=\LARGE] at (12,10.75) {$x$};
			\node [font=\LARGE] at (7,12.75) {$z$};
			\node [font=\LARGE] at (5,7.75) {$y=-2m$};
			%\draw [dashed, thick] (7.5,14.75) arc[start angle=90, end angle=110, radius=7cm];
			%\draw [dashed, thick] (7.5,14.75) .. controls (5,13) .. (5,8.5);
			%\draw [dashed, thick] (7.5,14.75) arc[start angle=90, end angle=230, radius=7.5cm];

		\end{tikzpicture}
	\end{figure}
	\begin{align*}
		\Gamma\brak{y}=100\sqrt{1-\brak{\frac{2y}{4}}^2} m^2/s
	\end{align*}
	\begin{enumerate}
			\begin{multicols}{2}
			\item $0.125$ radians
				\columnbreak
			\item $-0.125$ radians
			\end{multicols}
			\begin{multicols}{2}
			\item $0.125\sqrt{1=\brak{\frac{y}{2}}^2}$ radians
				\columnbreak
			\item $-0.125\sqrt{1=\brak{\frac{y}{2}}^2}$ radians

			\end{multicols}
	\end{enumerate}
\item The inlet stagnation temperature for a single stage axial compressor is $300 K$ and the stage efficiency is $0.80$. Following conditions exist at the mean radius of the rotor blade: \\
	Blade speed $= 200 m/s$\\
	Axial flow velocity $=160 m/s$\\
	Intet blade angle $\beta_1 = 44\degree$\\
	Outlet blade angle $\beta_2 = 14\degree$\\
	$C_p 1005 J/kgK \text{and} y= 1.4$\\
	What is the stagnation pressure ratio $\brak{\text{PRS}}$ for this compressor?

	\begin{enumerate}
			\begin{multicols}{2}
			\item $1.41$
				\columnbreak
			\item $1.37$
			\end{multicols}
			\begin{multicols}{2}
			\item $1.51$
				\columnbreak
			\item $1.23$
			\end{multicols}
	\end{enumerate}
	\textbf{Common Data for Questions 48 and 49:}\\
	Consider a simply supported beam of length $L$, carrying a bracket welded at its center. The bracket carries a vertical load, $P$, as shown in the figure. Dimensions of bracket are $a=0.1L$. The beam has a square cross-section of dimension $h \times h$.
\item Bending moment diagram is given by,
	\begin{figure}[H]
		\centering
		\begin{tikzpicture}

			\draw  (5,12.25) rectangle (13.75,12.75);
			\draw  (5.5,12.25) -- (5,11.75);
			\draw  (5.5,12.25) -- (6,11.75);
			\draw  (6,11.75) -- (5,11.75);
			\draw  (13.25,12.25) -- (13.75,11.75);
			\draw  (13.75,11.75) -- (12.75,11.75);
			\draw  (10,12.75) -- (10,14.75);
			\draw  (10,14.75) -- (8.75,14.75);
			\draw  (8.5,14.5) -- (9.75,14.5);
			\draw  (9.75,14.5) -- (9.75,12.75);
			\draw  (8.75,14.75) -- (9,15);
			\draw  (9,15) -- (8,15);
			\draw  (8,15) -- (8.5,14.5);
			\draw [Triangle-Triangle,  dashed] (10.25,12.75) -- (10.25,14.75);
			\draw [Triangle-Triangle,  dashed] (8.5,14.25) -- (9.75,14.25);
			\draw [Triangle-Triangle] (5.5,11) -- (13.25,11);
			\draw  (12.75,11.75) -- (13.25,12.25);
			\draw [-Triangle] (8.25,16) -- (8.25,15);
			\node [font=\LARGE] at (7.5,16.25) {$P$};
			\node [font=\LARGE] at (10.5,13.75) {$a$};
			\node [font=\LARGE] at (9,13.75) {$a$};
			\node [font=\LARGE] at (9,11.25) {$L$};

		\end{tikzpicture}
	\end{figure}

	\begin{enumerate}

			\begin{multicols}{2}
			\item .\begin{figure}[H]
					\centering
					\begin{tikzpicture}

						\draw  (6.25,9.75) -- (10,13.5);
						\draw  (10,13.5) -- (10,14.25);
						\draw  (10,14.25) -- (14.5,9.75);
						\draw [->, >=Stealth] (6.25,9.75) -- (16.5,9.75);
						\draw [->, >=Stealth] (6.25,8.5) -- (6.25,16.75);
						\draw  (6,13.5) -- (6.5,13.5);
						\draw  (6,14.25) -- (6.5,14.25);
						\draw  (10,9.5) -- (10,10);
						\node [font=\LARGE] at (10,9.0) {$\frac{L}{2}$};
						\node [font=\LARGE] at (5.5,17.25) {$M$};
						\node [font=\LARGE] at (5,14.25) {$0.3PL$};
						\node [font=\LARGE] at (5,13.5) {$0.2PL$};

					\end{tikzpicture}
			\end{figure}
			\columnbreak
		\item .\begin{figure}[H]
				\centering
				\begin{tikzpicture}

					\draw [-Triangle] (6.25,9.75) -- (16.25,9.75);
					\draw [-Triangle] (6.25,6) -- (6.25,15);
					\draw  (6.25,9.75) -- (10,6);
					\draw  (10,6) -- (10,13.5);
					\draw  (10,13.5) -- (13.75,9.75);
					\draw  (6,6) -- (6.5,6);
					\draw  (6,13.5) -- (6.5,13.5);
					\node [font=\LARGE] at (6,15.25) {$M$};
					\node [font=\LARGE] at (5,13.25) {$0.2PL$};
					\node [font=\LARGE] at (5,5.75) {$-0.2PL$};
					\node [font=\LARGE] at (10.5,9.25) {$\frac{L}{2}$};
				\end{tikzpicture}
		\end{figure}
			\end{multicols}
			\begin{multicols}{2}
			\item .\begin{figure}[H]
					\centering
					\begin{tikzpicture}
						\draw [-Triangle, ] (6.25,9.75) -- (16.25,9.75);
						\draw [-Triangle, ] (6.25,8.5) -- (6.25,15);
						\draw  (6.25,9.75) -- (10,13.5);
						\draw  (10,13.5) -- (10,12.25);
						\draw  (10,12.25) -- (12.5,9.75);
						\draw  (6,13.5) -- (6.5,13.5);
						\draw  (6,12.25) -- (6.5,12.25);
						\draw  (10,9.5) -- (10,10);
						\node [font=\LARGE] at (10,9) {$\frac{L}{2}$};
						\node [font=\LARGE] at (5,12.25) {$0.2PL$};
						\node [font=\LARGE] at (5,13.75) {$0.3PL$};
						\node [font=\LARGE] at (6,15.25) {$M$};
					\end{tikzpicture}
			\end{figure}
			\columnbreak
		\item .\begin{figure}[H]
				\centering
				\begin{tikzpicture}
					\draw [-Triangle] (6.25,11) -- (15,11);
					\draw [-Triangle] (6.25,6.25) -- (6.25,16.5);
					\draw  (6.25,11) -- (10,14.75);
					\draw  (10,14.75) -- (10,7.25);
					\draw  (10,7.25) -- (13.75,11);
					\node [font=\LARGE] at (6,17.25) {$M$};
					\draw  (6,14.75) -- (6.5,14.75);
					\draw  (6,7.25) -- (6.75,7.25);
					\node [font=\LARGE] at (5,14.75) {$0.3PL$};
					\node [font=\LARGE] at (5,7.25) {$-0.2PL$};
					\node [font=\LARGE] at (10.5,10.25) {$\frac{L}{2}$};
				\end{tikzpicture}
		\end{figure}
			\end{multicols}
	\end{enumerate}
\item Maximum value of shear stress is,
	\begin{enumerate}
			\begin{multicols}{2}
			\item $0.67P/h^2$
				\columnbreak
			\item $1.33P/h^2$
			\end{multicols}
			\begin{multicols}{2}
			\item $1.5P/h^2$
				\columnbreak
			\item $0.9 P/h^2$
			\end{multicols}
	\end{enumerate}
	\textbf{Statement for Linked Answer Questions 50 and 51:}\\
	Consider a potential flow over a spinning cylinder. The stream function is given as,
	\begin{align*}
		\psi =\brak{V_{\infty}r\sin\theta}\brak{1=\frac{R^2}{r^2}}+\frac{\Gamma}{2\pi}\ln\frac{r}{R}
	\end{align*}
	where \\
	Free stream velocity, $V_{\infty}=25m/s$\\
	Cylinder radius, $R=1m$\\
	Circulation, $\Gamma =50\pi m^2/s$
	\begin{figure}[H]
		\centering
		\begin{tikzpicture}

			\draw [dashed] (5.75,14.25) -- (15.25,14.25);
			\draw  (9.5,14.25) circle (2cm);
			\draw [-Triangle] (5.75,14.25) -- (7.25,14.25);
			\draw [-Triangle] (5.75,15) -- (7.25,15);
			\draw [-Triangle] (5.75,13.5) -- (7.25,13.5);
			\draw [-Triangle,dashed] (9.5,14.25) -- (10.9,12.8);
			\node [font=\LARGE] at (10,13.25) {R};
			\node [font=\LARGE] at (9.5,15.25) {$\Gamma$};
			\draw [Triangle-] (11.3, 14.25) arc (0:120:1.8);
			\draw [-Triangle] (11.7, 14.25) arc (0:60:2.2);
			\node [font=\LARGE] at (11.5,15.75) {$\theta$};
		\end{tikzpicture}
	\end{figure}
\item The radial and azimuthal velocities on the cylinder surface at $\theta=\frac{\pi}{2}$ are,
	\begin{enumerate}
			\begin{multicols}{2}
			\item $V_r=0m/s, V_0=-75m/s$
				\columnbreak
			\item $V_r=0m/s, V_0=75m/s$
			\end{multicols}
			\begin{multicols}{2}
			\item $V_r=0m/s, V_0=-25m/s$
				\columnbreak
			\item $V_r=0m/s, V_0=25m/s$
			\end{multicols}
	\end{enumerate}
\item The stagnantation points are located at
	\begin{enumerate}
			\begin{multicols}{2}
			\item $210\degree$ and $330\degree$
				\columnbreak
			\item $240\degree$ and $300\degree$
			\end{multicols}
			\begin{multicols}{2}
			\item $30\degree$ and $150\degree$
				\columnbreak
			\item $60\degree$ and $120\degree$
			\end{multicols}
	\end{enumerate}
	\textbf{Statement for Linked Answer Questions 52 and 53:}\\
	An aircraft with an IDEAL Turbojet engine is flying at $200 m/s$ at an altitude where the ambient pressure is equal to $0.8 bar$. The stagnation pressure and temperature at the inlet of the turbine are $6 bar$ and $1400 K$ respectively. The change in specific enthalpy across the compressor is $335 kJ/kg$. Assume the fuel flow rate to be very small in comparison to the air flow rate and consider $C_p = 1117 J/kgK$ and $\gamma = 1.3$.
\item What is the stagnantation pressure at the inlet of the nozzle,
	\begin{enumerate}
			\begin{multicols}{2}
			\item $2.8 bar$
				\columnbreak
			\item $5.7 bar$
			\end{multicols}
			\begin{multicols}{2}
			\item $2.1 bar$
				\columnbreak
			\item $6.3 bar$
			\end{multicols}
	\end{enumerate}



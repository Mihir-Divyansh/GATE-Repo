\iffalse
\chapter{2007}
\author{EE24BTECH11037}
\section{xe}
\fi

%\begin{enumerate}
    \item The volume of the prism whose base is the triangle in the $xy$ - plane bounded by the $x$-axis and the lines $y=x$ and $x=2$ and whose top lies in the plane $z=5-x-y$ is
    \begin{enumerate}
    \item $2$
    \item $4$
    \item $6$
    \item $10$
    \end{enumerate}
    \item The general solution of $x\brak{z^2-y^2}\frac{\partial z}{\partial x}+y\brak{x^2-z^2}\frac{\partial z}{\partial y}=z\brak{y^2-x^2}$ is
    \begin{enumerate}
        \item $F\brak{x^2+y^2+z^2,xyz}=0$
        \item $F\brak{x^2+y^2-z^2,xyz}=0$
        \item $F\brak{x^2-y^2+z^2,xyz}=0$
        \item $F\brak{-x^2+y^2+z^2,xyz}=0$
    \end{enumerate}
    \item Choose a point uniformly distributed at random on the disc $x^2+y^2 \leq 1$. Let the random variable $X$ denote the distance of this point from the center of the disc. Then the variance of $X$ is 
    \begin{enumerate}
        \item $\frac{1}{16}$
        \item $\frac{1}{17}$
        \item $\frac{1}{18}$
        \item $\frac{1}{19}$
    \end{enumerate}
    \item If Range-Kutta method of order 4 is used to solve the differential equation $\frac{dy}{dx}=f(x)$, $y(0)=0$ in the interval $\sbrak{0,h}$ with step size $h$, then 
    \begin{enumerate}
    \item $y(h)=\frac{h}{6}\sbrak{f\brak{0}+4f\brak{h/2}+f\brak{h}}$
    \item $y(h)=\frac{h}{6}\sbrak{f\brak{0}+f\brak{h}}$
    \item $y(h)=\frac{h}{2}\sbrak{f\brak{0}+f\brak{h}}$
    \item $y(h)=\frac{h}{6}\sbrak{f\brak{0}+2f\brak{h/2}+f\brak{h}}$
    \end{enumerate}
    \item If a polynomial of degree three interpolates a function $f(x)$ at the points $\brak{0,3}$, $\brak{1,13}$, $\brak{3,99}$ and $\brak{4,187}$, then $f\brak{2}$
    \begin{enumerate}
        \item $20$
        \item $36$
        \item $43$
        \item $58$
    \end{enumerate}
    \textbf{Common Data Questions}\\
    Common Data for Questions $23, 24:$\\
    Let $f:R\rightarrow R$ be defined by $f\brak{x}=x^2$ for $-\pi\leq x\leq\pi$ and $f\brak{x+2\pi}=f\brak{x}$.\\
    
    \item The Fourier series of $f$ in $\sbrak{-\pi,\pi}$ is
    \begin{enumerate}
        \item $\frac{\pi^2}{3}+4\displaystyle \sum_{n=1}^{\infty}\frac{\cos{nx}}{n^2}$
        \item $\frac{\pi^2}{3}+\displaystyle \sum_{n=1}^{\infty}\frac{\brak{-1}^n\cos{nx}}{n^2}$
        \item $\frac{\pi^2}{3}+4\displaystyle \sum_{n=1}^{\infty}\frac{\brak{-1}^n\cos{nx}}{n^2}$
        \item $\frac{\pi^2}{3}+\displaystyle \sum_{n=1}^{\infty}\frac{\cos{nx}}{n^2}$
    \end{enumerate}
    \item The sum of the absolute values of the Fourier coefficients of $f$ is
    \begin{enumerate}
        \item $\displaystyle\frac{\pi^2}{6}$
        \item $\displaystyle\frac{\pi^2}{3}$
        \item $\displaystyle\frac{2\pi^2}{3}$
        \item $\displaystyle\pi^2$
    \end{enumerate}
    \textbf{Linked Answer Questions: Q.25 to Q.28 carry two marks each.}\\
    Statement for Linked Answer Questions 25\&26\\
    Let $y\brak{x}=\displaystyle \sum_{n=0}^{\infty}a_n x^n$ be a solution of the differential equation $\frac{d^2y}{dx^2}+xy=0$.\\
    \item The value of $a_11$ is
    \begin{enumerate}
        \item $0$
        \item $1$
        \item $2$
        \item $3$
    \end{enumerate}
    \item The solution of the differential equation given above satisfying $y\brak{0}=1$ and $y'\brak{0}=0$ is 
    \begin{enumerate}
        \item $\displaystyle y\brak{x}=1+\frac{1}{2.3}x^2-\frac{1}{2.3.5.6}x^4+\frac{1}{2.3.5.6.8.9}x^6-...$
        \item $\displaystyle y\brak{x}=1-\frac{1}{2.3}x^2+\frac{1}{2.3.5.6}x^4-\frac{1}{2.3.5.6.8.9}x^6+...$
        \item $\displaystyle y\brak{x}=1+\frac{1}{2.3}x^3-\frac{1}{2.3.5.6}x^6+\frac{1}{2.3.5.6.8.9}x^9-...$
        \item $\displaystyle y\brak{x}=1-\frac{1}{2.3}x^3+\frac{1}{2.3.5.6}x^6-\frac{1}{2.3.5.6.8.9}x^9+...$
    \end{enumerate}
    Statement for Linked Answer Questions 27\&28:\\
    The potential $u(x,y)$ satisfies the equation $\displaystyle\frac{\partial^2u}{\partial x^2}+\frac{\partial^2u}{\partial y^2}=0$ in the square $0\leq x\leq\pi, 0\leq y\leq\pi$. Three of the edges $x=0,x=\pi$ and $y=0$ of the square are kept at zero potential and the edge $y=\pi$ is kept at nonzero potential.\\

    \item The potential $u\brak{x,y}$ is given by
    \begin{enumerate}
        \item $u\brak{x,y}=\displaystyle\sum_{n=1}^{\infty}A_n \cosh{nx}\sin{ny}$
        \item $u\brak{x,y}=\displaystyle\sum_{n=1}^{\infty}A_n \sin{nx}\cosh{ny}$
        \item $u\brak{x,y}=\displaystyle\sum_{n=1}^{\infty}A_n \sinh{nx}\sin{ny}$
        \item $u\brak{x,y}=\displaystyle\sum_{n=1}^{\infty}A_n \sin{nx}\sinh{ny}$
    \end{enumerate}
    \item If the edge $y=\pi$ is kept at the potential $\sin{x}$, then the potential $u\brak{x,y}$ is given by
    \begin{enumerate}
        \item $u\brak{x,y}=\displaystyle\sum_{n=1}^{\infty}\frac{\sin{nx}\sinh{ny}}{\sinh{n\pi}}$
        \item $u\brak{x,y}=\displaystyle\frac{\sin{x}\sinh{y}}{\sinh{\pi}}$
        \item $u\brak{x,y}=\displaystyle\frac{\sin{x}\cosh{y}}{\cosh{\pi}}$
        \item $u\brak{x,y}=\displaystyle\sum_{n=1}^{\infty}\frac{\cosh{nx}\sin{ny}}{\cosh{n\pi}}$
    \end{enumerate}
\vspace{0.5cm}
\textbf{B: Computational Science}\\
    \item If the $7$-base representation of a number is $123$, then its octal representation is
    \begin{enumerate}
        \item $102$
        \item $103$
        \item $111$
        \item $112$
    \end{enumerate}
    \item Consider the following four FORTRAN statements 
    \begin{enumerate}
        \item[(1)] {$S1: X=5^{**}3$}
        \item[(2)] {$S2: X=\brak{-5}^{**}3.0$}
        \item[(3)] {$S3: X=5^**-3$}
        \item[(4)] {$S3: X=5^**3.0$}
    \end{enumerate} Which of the following sets contain the set of valid statements from above?
    \begin{enumerate}
        \item $\{ S1, S3\}$
        \item $\{ S1, S4\}$
        \item $\{ S2, S3\}$
        \item $\{ S2, S4\}$
    \end{enumerate}
    \item Which of the following sets contains the set of the basic data types in $C$?
    \begin{enumerate}
        \item \{char,int,float,logical\}
        \item \{char,boolean,int,float\}
        \item \{char,int,long,short,float,double\}
        \item \{char,int,float,void\}
    \end{enumerate}
    \item If a root of $f\brak{x}=x^2+-2x+1=0$ is obtained by using the iterative scheme $x_{n+1}=x_n-\frac{f\brak{x_n}}{f'\brak{x_n}}$ with initial value $x_o=0.5$, then the convergence rate is
    \begin{enumerate}
        \item $1$
        \item $1.62$
        \item $1.84$
        \item $2$
    \end{enumerate}
    \item Let $S1$ be the sum of the eigen values of 2 x 2  matrix $P$ and $S2$ be the sum of the eigen values of another 2 x 2 matrix $Q$. If $S1=S2$,then $P$ and $Q$ are
    \begin{enumerate}
        \item \[
\begin{bmatrix}
4 & 1 \\
3 & 5
\end{bmatrix}
\] and \[
\begin{bmatrix}
1 & 4 \\
2 & 3
\end{bmatrix}
\]
     \item \[
\begin{bmatrix}
3 & 4 \\
5 & 1
\end{bmatrix}
\] and \[
\begin{bmatrix}
2 & 4 \\
3 & 1
\end{bmatrix}
\]
\item \[
\begin{bmatrix}
4 & 1 \\
3 & 5
\end{bmatrix}
\] and \[
\begin{bmatrix}
3 & 4 \\
1 & 5
\end{bmatrix}
\]
\item \[
\begin{bmatrix}
1 & 3 \\
4 & 5
\end{bmatrix}
\] and \[
\begin{bmatrix}
4 & 3 \\
1 & 2
\end{bmatrix}
\]
    \end{enumerate}
\item If $y_i$ denotes the value of $y\brak{x}$ at $x=x_i$ in $x_o<x_1<...<x_i<...x_n$ and $x_i-x_{i-1}=h$ for $1\leq i\leq n$, then $\displaystyle\frac{d^2y}{dx^2}$ at $x = x_i$, $1\leq i\leq n-1$ is approximated using finite difference scheme by
\begin{enumerate}
    \item $\frac{1}{2h}\brak{y_{i+1}-2y_i+y_{i-1}}$
    \item $\frac{1}{2h}\brak{y_{i+1}-y_i+y_{i-1}}$
    \item $\frac{1}{h^2}\brak{y_{i+1}-2y_i+y_{i-1}}$
    \item $\frac{1}{h^2}\brak{y_{i+1}-y_i+y_{i-1}}$
\end{enumerate}
%\end{enumerate}




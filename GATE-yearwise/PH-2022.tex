\iffalse
\chapter{2022}
\author{AI24BTECH11012}
\section{ph}
\fi
    \item In cylindrical coordinates $(s, \varphi, z)$, which of the following is a Hermitian operator?
    \begin{enumerate}
        \item $\frac{1}{i} \frac{\partial}{\partial s}$
        \item $\frac{1}{i} \left( \frac{\partial}{\partial s} + \frac{1}{s} \right)$
        \item $\frac{1}{i} \left( \frac{\partial}{\partial s} + \frac{1}{2s} \right)$
        \item $\left( \frac{\partial}{\partial s} + \frac{1}{s} \right)$
    \end{enumerate}

    \item A particle of mass 1 kg is released from a height of 1 m above the ground. When it reaches the ground, what is the value of Hamilton's action for this motion in J s? ( $g$ is the acceleration due to gravity; take gravitational potential to be zero on the ground)
    \begin{enumerate}
        \item $-\frac{2}{3} \sqrt{2g}$
        \item $\frac{5}{3} \sqrt{2g}$
        \item $3 \sqrt{2g}$
        \item $-\frac{1}{3} \sqrt{2g}$
    \end{enumerate}
    
    \item If $(\dot{x} y + \alpha xy)$ is a constant of motion of a two-dimensional isotropic harmonic oscillator with Lagrangian
    \[
    L = \frac{m(\dot{x}^2 + \dot{y}^2)}{2} - \frac{k(x^2 + y^2)}{2}
    \]
    then $\alpha$ is
    \begin{enumerate}
        \item $+\frac{k}{m}$
        \item $-\frac{k}{m}$
        \item $-\frac{2k}{m}$
        \item 0
    \end{enumerate}

    \item In a two-dimensional square lattice, frequency $\omega$ of phonons in the long wavelength limit changes linearly with the wave vector $k$. Then the density of states of phonons is proportional to
    \begin{enumerate}
        \item $\omega$
        \item $\omega^2$
        \item $\sqrt{\omega}$
        \item $\frac{1}{\sqrt{\omega}}$
    \end{enumerate}
    
   \item At $T=0$K, which of the following diagram represents the occupation probability P(E) of energy states of electrons in a BCS type superconductor?
   \begin{enumerate}
   \item 
 
   
\begin{circuitikz}
\tikzstyle{every node}=[font=\small]
\draw [->, >=Stealth] (2,11.75) -- (2,16.5);
\draw [->, >=Stealth] (1,12.75) -- (8.5,12.75);
\draw [short] (2,14.5) -- (5.25,14.5);
\draw [short] (5.25,14.5) .. controls (6.75,14) and (5,13) .. (7.5,12.75);
\draw [dashed] (5.75,15) -- (5.75,12.75);
\node [font=\small] at (1.5,16.75) {P(E)};
\node [font=\small] at (5.75,12.5) {$E_F$};
\node [font=\small] at (8.5,12.5) {E};
\node [font=\small] at (1.75,14.5) {1};
\end{circuitikz}



  
   \item 

   
\begin{circuitikz}
\tikzstyle{every node}=[font=\small]
\draw [->, >=Stealth] (2,11.75) -- (2,16.5);
\draw [->, >=Stealth] (1,12.75) -- (8.5,12.75);
\draw [short] (2,14.5) -- (5.25,14.5);
\node [font=\small] at (1.5,16.75) {P(E)};
\node [font=\small] at (5.75,12.5) {$E_F$};
\node [font=\small] at (8.5,12.5) {E};
\node [font=\small] at (1.75,14.5) {1};
\draw [short] (5.25,14.5) -- (5.75,14.5);
\draw [short] (5.75,14.5) -- (5.75,12.75);
\end{circuitikz}


   
   \item 
   
   
\begin{circuitikz}
\tikzstyle{every node}=[font=\small]
\draw [->, >=Stealth] (1,12.25) -- (1,16.75);
\draw [->, >=Stealth] (0.5,12.75) -- (8.25,12.75);
\draw [short] (3,12.75) .. controls (4.5,13) and (3.75,13.75) .. (4.75,14);
\draw [short] (4.75,14) -- (8,14);
\draw [dashed] (4,14.25) -- (4,12.75);
\node [font=\small] at (1.5,16.5) {P(E)};
\node [font=\small] at (4,12.5) {$E_F$};
\node [font=\small] at (0.5,14) {1};
\node [font=\small] at (8.25,12.25) {E};
\end{circuitikz}

   
   \item 
   
   
\begin{circuitikz}
\tikzstyle{every node}=[font=\small]
\draw [->, >=Stealth] (1,12.25) -- (1,16.75);
\draw [->, >=Stealth] (0.5,12.75) -- (8.25,12.75);
\draw [dashed] (4,14.25) -- (4,12.75);
\node [font=\small] at (1.5,16.5) {P(E)};
\node [font=\small] at (4,12.5) {$E_F$};
\node [font=\small] at (0.5,14) {1};
\node [font=\small] at (8.25,12.25) {E};
\draw [short] (2.75,12.75) .. controls (3.75,13) and (3.75,13.25) .. (4,14);
\draw [short] (4,14) .. controls (4.5,13) and (4.5,13) .. (5.25,12.75);
\end{circuitikz}



   
   \end{enumerate}
   
    \item For a one-dimensional harmonic oscillator, the creation operator $(a^{\dagger})$ acting on the $n$th state $|\psi_n\rangle$, where $n = 0, 1, 2, \ldots$, gives $a^{\dagger} |\psi_n\rangle = \sqrt{n + 1} |\psi_{n+1}\rangle$. The matrix representation of the position operator 
    \[
    x = \sqrt{\frac{\hbar}{2m\omega}} (a + a^{\dagger})
    \] 
    for the first three rows and columns is:
    \begin{enumerate}
        \item $\sqrt{\frac{\hbar}{2m\omega}} \begin{pmatrix}
        1 & 0 & 0 \\
        0 & \sqrt{2} & 0 \\
        0 & 0 & \sqrt{3}
        \end{pmatrix}$

        \item $\sqrt{\frac{\hbar}{2m\omega}} \begin{pmatrix}
        0 & 1 & 0 \\
        1 & 0 & 1 \\
        0 & 1 & 0
        \end{pmatrix}$

        \item $\sqrt{\frac{\hbar}{2m\omega}} \begin{pmatrix}
        0 & 1 & 0 \\
        1 & 0 & \sqrt{2} \\
        0 & \sqrt{2} & 0
        \end{pmatrix}$

        \item $\sqrt{\frac{\hbar}{2m\omega}} \begin{pmatrix}
        1 & 0 & \sqrt{3} \\
        0 & 0 & 0 \\
        \sqrt{3} & 0 & 1
        \end{pmatrix}$
    \end{enumerate}

    \item A piston of mass $ m $ is fitted to an airtight horizontal cylindrical jar. The cylinder and piston have identical unit area of cross-section. The gas inside the jar has volume $ V $ and is held at pressure $ P = P_{\text{atmosphere}} $. The piston is pushed inside the jar very slowly over a small distance. On releasing, the piston performs an undamped simple harmonic motion of low frequency. Assuming that the gas is ideal and no heat is exchanged with the atmosphere, the frequency of the small oscillations is proportional to
    \begin{enumerate}
        \item $ \sqrt{\frac{P}{\gamma m V}} $
        \item $ \sqrt{\frac{P \gamma}{V m}} $
        \item $ \sqrt{\frac{P}{m V^{\gamma - 1}}} $
        \item $ \sqrt{\frac{\gamma P}{m V^{\gamma - 1}}} $
    \end{enumerate}

    \item A paramagnetic salt of mass $ m $ is held at temperature $ T $ in a magnetic field $ H $. If $ S $ is the entropy of the salt and $ M $ is its magnetization, then $ dG = -S dT - M dH $, where $ G $ is the Gibbs free energy. If the magnetic field is changed adiabatically by $ \Delta H \to 0 $ and the corresponding infinitesimal changes in entropy and temperature are $ \Delta S $ and $ \Delta T $, then which of the following statements are correct
    \begin{enumerate}
        \item $ \Delta S = - \frac{1}{T} \left( \frac{\partial G}{\partial T} \right)_{H} \Delta T $
        \item $ \Delta S = 0 $
        \item $ \Delta T = \left( \frac{\partial M}{\partial T} \right)_{H} \Delta H $
        \item $ \Delta T = 0 $
    \end{enumerate}

    \item A particle of mass $ m $ is moving inside a hollow spherical shell of radius $ a $ so that the potential is
    \[
    V(r) = 
    \begin{cases} 
      0 & \text{for } r < a \\
      \infty & \text{for } r \geq a 
    \end{cases}
    \]
    The ground state energy and wavefunction of the particle are $ E_0 $ and $ R(r) $, respectively. Then which of the following options are correct?
    \begin{enumerate}
        \item $ E_0 = \frac{\hbar^2 \pi^2}{2 m a^2} $
        \item $ -\frac{\hbar^2}{2 m r^2} \frac{d}{dr} \left( r^2 \frac{d R}{dr} \right) = E_0 R \quad (r < a) $
        \item $ -\frac{\hbar^2}{2 m r^2} \frac{d^2 R}{dr^2} = E_0 R \quad (r < a) $
        \item $ R(r) = \frac{1}{r} \sin \left( \frac{\pi r}{a} \right) \quad (r < a) $
    \end{enumerate}

    \item A particle of unit mass moves in a potential $ V(r) = -V_0 e^{-r^2} $. If the angular momentum of the particle is $ L = 0.5 \sqrt{V_0} $, then which of the following statements are true?
    \begin{enumerate}
        \item There are two equilibrium points along the radial coordinate
        \item There is one stable equilibrium point at $ r_1 $ and one unstable equilibrium point at $ r_2 > r_1 $
        \item There are two stable equilibrium points along the radial coordinate
        \item There is only one equilibrium point along the radial coordinate
    \end{enumerate}
    \item In a diatomic molecule of mass $ M $, electronic, rotational and vibrational energy scales are of magnitude $ E_e $, $ E_R $, and $ E_V $, respectively. The spring constant for the vibrational energy is determined by $ E_e $. If the electron mass is $ m $ then
    \begin{enumerate}
        \item $ E_R \sim \frac{m}{M} E_e $
        \item $ E_R \sim \sqrt{\frac{m}{M}} E_e $
        \item $ E_V \sim \sqrt{\frac{m}{M}} E_e $
        \item $ E_V \sim \left( \frac{m}{M} \right)^{1/4} E_e $
    \end{enumerate}

    \item Electronic specific heat of a solid at temperature $ T $ is $ C = \gamma T $, where $ \gamma $ is a constant related to the thermal effective mass ($ m_{\text{eff}} $) of the electrons. Then which of the following statements are correct?
    \begin{enumerate}
        \item $ \gamma \propto m_{\text{eff}} $
        \item $ m_{\text{eff}} $ is greater than free electron mass for all solids
        \item Temperature dependence of $ C $ depends on the dimensionality of the solid
        \item The linear temperature dependence of $ C $ is observed at $ T \ll $ Debye temperature
    \end{enumerate}
    \item In a Hall effect experiment on an intrinsic semiconductor, which of the following statements are correct?
    \begin{enumerate}
        \item Hall voltage is always zero
        \item Hall voltage is negative if the effective mass of holes is larger than those of electrons
        \item Hall coefficient can be used to estimate the carrier concentration in the semiconductor
        \item Hall voltage depends on the mobility of the carriers
    \end{enumerate}


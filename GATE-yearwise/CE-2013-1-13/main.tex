\iffalse
\title{CE-2013-1-13}
\author{EE24BTECH11036 - Krishna Patil}
\section{ce}
\chapter{2013}
\fi
\item There is no value of $x$ that can simultaneously satisfy both the given equations . Therefore , find the 'Least Squares error' solution to the two equations , i.e. , find the value of $x$ that minimizes the sum of squares of the errors in the two equations . \rule{2cm}{0.4pt}
 \begin{center}
     $2x=3$ \\ $4x=1$ \\
 \end{center}
\item What is the minimum number of multiplications involved in computing the matrix product $PQR$ ? Matrix $P$ has $4$ rows and $2$ columns ,  matrix $Q$ has $2$ rows and $4$ columns , and matrix $R$ has $4$ rows and $1$ column. \rule{2cm}{0.4pt} \\
\item A $1-h$ rainfall of $10cm$ magnitude at a station has a return period of $50$ years . The probablity that a $1-h$ rainfall of magnitude of $10cm$  or more will occur in each of two successive years is :  
\begin{enumerate}
\begin{multicols}{2}
\item $ 0.04 $
\item $ 0.2 $ 
\item $ 0.02 $
\item $ 0.0004 $
\end{multicols}
\end{enumerate}
\item Maximum possible value of Compacting Factor for fresh \brak{\text{green}} concrete is:
\begin{enumerate}
\begin{multicols}{2}
\item $0.5$
\item $1.0$
\item $1.5$
\item $2.0$
\end{multicols}
\end{enumerate}
\item As per is $800:2007$ , the cross-section in which the extreme fiber can reach the yield stress , but cannot develop the plastic moment of resistance due to failure by local buckling is classified as
\begin{enumerate}
\begin{multicols}{2}
\item plastic section
\item  compact section  
\item semi-compact section
\item slender section
\end{multicols}
\end{enumerate}
\item the creep strains are 
\begin{enumerate}
\begin{multicols}{2}
\item  caused due to dead load only  
\item  caused due to live loads only 
\item caused due to cyclic loads only 
\item independent of loads 
\end{multicols}
\end{enumerate}
\item As per IS $456:2000$ for $M20$ grade concrete and plain bars in tension, the design bond stress $\tau_{bd} = 1.2 $ , $\text{MPa}$ . Further, IS $456:2000$ permits this design bond stress value to be increased by $60\%$ for HSD bars. The stress in the HSD reinforcing steel bars in tension, $\sigma_s = 360$ ,
$MPa$ . Find the required development length, $L_d$, for HSD bars in terms of the bar diameter, $\phi$. \rule{2cm}{0.4pt} \\
\item The 'plane section remains plane' assumption in bending theory implies: 
\begin{enumerate}
\item strain profile is linear 
\item stress profile is linear
\item both stress and strain profiles are linear 
\item shear deformations are neglected
\end{enumerate}
\item Two steel columns $P$ (length $L$ and yield strength $f_y = 250MPa$) and Q (length $2L$ and yield strength $f_y = 500MPa$) have the same crossections and end-conditions . The ratio of buckling load of column $P$ to that of column $Q$ is: 
\begin{enumerate}
\begin{multicols}{2}
\item $0.5$
\item $1.0$
\item $2.0$
\item $4.0$
\end{multicols}
\end{enumerate}
\item The pin-jointed 2-D truss is loaded with a horizontal force $15kN$ at joint $S$ and another $15$ kN vertical force at joint $U$ , as shown .Find the force in member $RS$ \brak{in \, \, kN} and report your answer taking trnsion as positive and compression as negative   . \rule{2cm}{0.4pt} \\ 
\begin{figure}[!ht]
\centering
\resizebox{0.5\textwidth}{!}{%
\begin{circuitikz}
\tikzstyle{every node}=[font=\normalsize]
\draw  (10,14.75) rectangle (13,12);
\draw [short] (10,14.75) -- (13,12);
\draw [short] (10,14.75) -- (7,12);
\draw [short] (10,12) -- (7,12);
\draw [short] (13,14.75) -- (16,12);
\draw [short] (13,12) -- (16,12);
\draw [short] (10,12) -- (10,10.25);
\draw [short] (10,10.25) -- (9.5,9.75);
\draw [short] (10,10.25) -- (10.5,9.75);
\draw [short] (9.5,9.75) -- (10.5,9.75);
\draw [short] (9.5,9.75) -- (9.25,9.5);
\draw [short] (10,9.75) -- (9.75,9.5);
\draw [short] (10.5,9.75) -- (10.25,9.5);
\draw [short] (16,12) -- (15.5,11.5);
\draw [short] (16,12) -- (16.5,11.5);
\draw [short] (15.5,11.5) -- (16.5,11.5);
\draw [short] (15.5,11.5) -- (15.25,11.25);
\draw [short] (16,11.5) -- (15.75,11.25);
\draw [short] (16.5,11.5) -- (16.25,11.25);
\node [font=\normalsize] at (10,15) {R};
\node [font=\normalsize] at (13,15) {S};
\node [font=\normalsize] at (13.25,12.25) {U};
\node [font=\normalsize] at (10.25,12.25) {V};
\node [font=\normalsize] at (10.25,10.5) {W};
\node [font=\normalsize] at (16.25,12.25) {T};
\draw [->, >=Stealth] (13,12) -- (13,10.75)node[pos=0.5, fill=white]{15 kN};
\draw [->, >=Stealth] (13,14.75) -- (14.5,14.75);
\node [font=\normalsize] at (15,14.75) {15 kN};
\node [font=\normalsize] at (6.75,12.25) {Q};
\draw [<->, >=Stealth] (17,15) -- (17,12)node[pos=0.5, fill=white]{4m};
\draw [<->, >=Stealth] (13,16) -- (15.75,16);
\draw [<->, >=Stealth] (10,16) -- (13,16);
\draw [<->, >=Stealth] (7,16) -- (10,16);
\node [font=\normalsize] at (8.5,16.25) {4m};
\node [font=\normalsize] at (11.5,16.25) {4m};
\node [font=\normalsize] at (14.25,16.25) {4m};
\draw [<->, >=Stealth] (7,12) -- (7,10);
\node [font=\normalsize] at (6.5,11) {4m};
\end{circuitikz}
}%
\label{fig:my_label}
\end{figure}
\item A symmetric I-section with (width of each flange = $10mm$, depth of web = $100mm$ , and thickness of web = $10mm$) of steel is subjected to a shear force of $100kN$. Find the magnitude of the shear stress in $N/mm^2$ in the web at its junction with the top flange. \rule{2cm}{0.4pt} 
\item In its natural condition , a soil sample  has a mass of $1.980kg$ and a volume of $0.001m^3$ . After being completely dried in an oven, the mass of the sample is $1.800kg$ . Specific gravity $G$ is $2.7$ . Unit weight of water is $10kN/m^3$. The degree of saturation of the soil is:  
\begin{enumerate}
\begin{multicols}{2}
\item  $ 0.65 $
\item $ 0.70 $
\item $ 0.54 $
\item $ 0.61 $
\end{multicols}
\end{enumerate}
\item The ratio of $N_f$/$N_d$ is known as shape factor , where $N_f$ is the number of flow lines and $N_d$ is the number of equipotential drops .  flow net is always drawn with a constant $b/a$ ratio , where $b$ and $a$ are distances between two consecutive flow lines and equipotential lines , respectively . Assuming that $b/a$ ratio remains the same, the shape factor of aflow net will change if the
\begin{enumerate}
\item upstream and downstream heads are interchanged
\item soil in the flow space is changed 
\item dimensions of the flow space are changed
\item head difference causing the flow is changed  
\end{enumerate}
     

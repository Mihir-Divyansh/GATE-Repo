\iffalse
    \title{2008-MA-69-85}
    \author{EE24BTECH11001 -  ADITYA TRIPATHY}
    \section{ma}
    \chapter{2008}
\fi

	\item[69.] 
	    The initial value problem $u_x + u_y = 1$, $u\brak{s, s} = \sin s, 0 \le s \le 1,$ has
		\hfill{\brak{2008-MA}}
	\begin{multicols}{4}
		\begin{enumerate}
			\item two solutions 
			\columnbreak
			\item a unique solution
			\columnbreak
			\item no solution
			\columnbreak
			\item infinitely many solutions
		\end{enumerate}
	\end{multicols}

\item[70.] Let $u\brak{x, t} = 0,$ be the solution of $u_tt - u_xx = 1, x \in \mathbb{R}, t > 0$
    with $u\brak{x, 0} = 0, u_t\brak{x, 0} = 0, x \in \mathbb{R}.$ Then $u\brak{\frac{1}{2}, \frac{1}{2}}$ 
        is equal to 
		
		\hfill{\brak{2008-MA}}
	\begin{multicols}{4}
		\begin{enumerate}
            \item $\frac{1}{8}$ \columnbreak
            \item $-\frac{1}{8}$ \columnbreak
            \item $\frac{1}{4}$ \columnbreak
            \item $-\frac{1}{4}$ 
		\end{enumerate}
	\end{multicols}

        \textbf{Common Data Questions}\\
        \textbf{Common Data for 71, 72, 73}\\
        Let $X = C\brak{\sbrak{0, 1}}$ with sup norm $\norm{\quad}_{\infty}$
    \item[71.] Let $S = \{ x \in X : \norm{x}_{\infty} \le 1 \}$. Then  
		\hfill{\brak{2008-MA}}
		\begin{enumerate}
			\item $S$ is convex and compact 
			\item $S$ is not convex but compact
			\item $S$ is convex but not compact 
			\item $S$ is neither convex nor compact
		\end{enumerate}
		
	\item[72.] Which one of the following is true?

		\hfill{\brak{2008-MA}}
		\begin{enumerate}
            \item $C^{\infty}\brak{\sbrak{0, 1}}$ is dense in $X$
            \item $X$ is dense $L^{\infty}\brak{\sbrak{0, 1}}$
			\item $X$ has a countable basis
            \item There is a sequence in $X$ which is uniformly Cauchy on on $\sbrak{0, 1}$ but does not converge uniformly on $\sbrak{0, 1}$
		\end{enumerate}

    \item[73.] Let $I = \cbrak{x \in X : x\brak{0} = 0}$. Then 
		
		\hfill{\brak{2008-MA}}
		\begin{enumerate}
			\item $I$ is not an ideal of $X$
			\item $I$ is an ideal, nut not a prime ideal of $X$
			\item $I$ is a prime ideal, but not a maximal ideal of $X$
			\item $I$ is a maximal ideal of $X$
		\end{enumerate}


        \textbf{Common Data for 74, 75}\\
        Let $X = C^1\brak{\sbrak{0, 1}}$ and $Y = C\brak{\sbrak{0, 1}}$, both with sup norm. Define
        $F : X \rightarrow Y$ by $F\brak{x} = x + x^{\prime}$ and $f\brak{x} = x\brak{1} + x^{\prime}\brak{1}$
        for $x \in X$.
	\item[74.] Then
		\hfill{\brak{2008-MA}}
\begin{enumerate}
			\item $F$ and $f$ are continuos
			\item $F$ is continuous and $f$ is discontinuous
			\item $F$ is discontinuous $f$ is contiuous
			\item $F$ and $f$ are discontinuous
		\end{enumerate}
\item[75.] Then
		\hfill{\brak{2008-MA}}
\begin{enumerate}
			\item $F$ and $f$ are closed maps
			\item $F$ is a closed map and $f$ is not a closed map 
			\item $F$ is not a closed map $f$ is a closed map
			\item neither $F$ nor $f$ is a closed map
		\end{enumerate}
        
        \textbf{Linked Answer Questions: Q76. to Q85 carry two marks each}\\
\textbf{Statement for Linked Answer Questions 76 , 77}\\
 Let $N = \mydet{\frac{3}{5} & -\frac{4}{5} & 0 \\ \frac{4}{5} & \frac{3}{5} & 0\\ 0 & 0 & 1}$
	\item[76.] Then N is
		\hfill{\brak{2008-MA}}
\begin{enumerate}
			\item non-invertible 
			\item skew-symmetric
			\item symmetric
			\item orthogonal
		\end{enumerate}
		
    \item[77.] If $M$ is any $3 \times 3$ real matrix, then trace$\brak{NMN^{\top}}$ is equal to
		\hfill{\brak{2008-MA}}
\begin{enumerate}
    \item $\sbrak{\textnormal{trace}\brak{N}}^2 \textnormal{trace}\brak{M}$ 
    \item $2\textnormal{trace}\brak{N} + \textnormal{trace}\brak{M}$
    \item $\textnormal{trace}\brak{M}$
    \item $\sbrak{\textnormal{trace}\brak{N}}^2 + \textnormal{trace}\brak{M}$ 
		\end{enumerate}
\textbf{Statement for Linked Answer Questions 78 , 79}\\
Let $f\brak{z} = \cos z - \frac{\sin z}{z}$ for non-zero $z \in \mathbb{C}$ and $f\brak{0} = 0$.
Also, let $g\brak{z} = \sinh z$ for $z \in \mathbb{C}$.
    \item[78.] Then $f\brak{z}$ has a zero $z = 0$ or order 
		\hfill{\brak{2008-MA}}
        \begin{enumerate}
    \item 0 
    \item 1
    \item 2
    \item greater than 2
        \end{enumerate}
\item[79.] Then $\frac{g\brak{z}}{z\brak{z}}$ has a pole at $z = 0$ or order 
		\hfill{\brak{2008-MA}}
        \begin{enumerate}
    \item 1
    \item 2
    \item 3 
    \item greater than 3
        \end{enumerate}
\textbf{Statement for Linked Answer Questions 80 , 81}\\
Let $n \ge 3$ be an integer. Let $y$ be the polynomial solution of $\brak{1-x^2}y^{\prime \prime}
-2xy^{\prime} + n\brak{n-1}y = 0$ satisfying $y\brak{1} = 1$
	\item[80.] Then the degree of $y$ is
		\hfill{\brak{2008-MA}}
        \begin{multicols}{4}
		\begin{enumerate}
			\item $n$ 
			\columnbreak
			\item $n -1$
			\columnbreak
			\item less than $n-1$
			\columnbreak
			\item greater than $n + 1$
		\end{enumerate}
	\end{multicols}
\item[81.] If $I = \int_{-1}^{1} y\brak{x}x^{n-3} \, dx$ and $J = \int_{-1}^{1} y\brak{x}x^{n} \, dx$, then 
		\hfill{\brak{2008-MA}}
        \begin{multicols}{4}
		\begin{enumerate}
			\item $I \ne 0, J \ne 0$ 
			\columnbreak
			\item $I \ne 0, J = 0$
			\columnbreak
			\item $I = 0, J \ne 0$
			\columnbreak
        \item $I = 0, J = 0$
		\end{enumerate}
	\end{multicols}

\textbf{Statement for Linked Answer Questions 82 , 83}\\
Consider the boundary value problem
\begin{align}
    u_xx + u_yy = 0, x \in \brak{0, \pi}, y \in \brak{0, \pi}, \\
    u\brak{x, 0} = u\brak{x, \pi} = u\brak{0, y} = 0.
\end{align}
\item[82.] Any solution of this boundary value problem is of the form
		\hfill{\brak{2008-MA}}
		\begin{enumerate}
            \item $\sum_{n = 1}^{\infty} a_n \sinh nx \sin ny$ 
            \item $\sum_{n = 1}^{\infty} a_n \cosh nx \sin ny$ 
			\item $\sum_{n = 1}^{\infty} a_n \sinh nx \cos ny$ 
		\item $\sum_{n = 1}^{\infty} a_n \cosh nx \cos ny$ 
		\end{enumerate}
\item[83.] If an additional boundary condition $u_x\brak{\pi, y} = \sin y$ is satisfied, then 
    $u\brak{x, \frac{\pi}{2}}$ is equal to
		\hfill{\brak{2008-MA}}
		\begin{enumerate}
            \item $\frac{\pi}{2}\brak{e^x - e^{-x}}\brak{e^x + e^{-x}}$ 
            \item $\frac{\pi\brak{e^x - e^{-x}}}{\brak{e^x + e^{-x}}}$ 
            \item $\frac{\pi\brak{e^x + e^{-x}}}{\brak{e^x - e^{-x}}}$ 
            \item $\frac{\pi}{2}\brak{e^{\pi} + e^{-\pi}}\brak{e^x - e^{-x}}$ 
		\end{enumerate}
\textbf{Statement for Linked Answer Questions 84 , 85}\\
Let the random variable $X$ follow the exponential distribution with mean 2. Define
$Y = \sbrak{X - 2 | X > 2}$.
\item[84.] The value of $\Pr{\brak{Y \ge t}}$
		\hfill{\brak{2008-MA}}
		\begin{enumerate}
            \item $e ^ {\frac{-t}{2}}$ 
            \item $e ^ {2t}$ 
            \item $\frac{1}{2}e ^ {\frac{-t}{2}}$ 
            \item $\frac{1}{2}e ^ {-t}$ 
		\end{enumerate}
\item[85.] The value of $E\brak{Y}$ is equal to
		\hfill{\brak{2008-MA}}
		\begin{enumerate}
            \item $\frac{1}{4}$ 
            \item $\frac{1}{2}$ 
            \item 1
            \item 2
		\end{enumerate}


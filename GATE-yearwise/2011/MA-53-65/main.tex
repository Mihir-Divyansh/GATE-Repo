\iffalse
\title{GATE Questions 6}
\author{EE24BTECH11012 - Bhavanisankar G S}
\section{ma}
\chapter{2011}
\fi
%\begin{enumerate}
	\item $\lim_{n \to \infty} \int_{0}^{1} s_n(x) dx = 1 $
		\begin{enumerate}
			\item by dominated convergence theorem
			\item by Fatou's theorem
			\item by the fact that $\cbrak{s_n}$ converges uniformly on \sbrak{0,1}
			\item by the fact that $\cbrak{s_n}$ converges pointwise on \sbrak{0,1}
		\end{enumerate}
	The matrix $A = \myvec{1 && 1 && 1 \\ 2 && 1 && 2 \\ 1 && 3 && 2 }$ can be decomposed into the product of a lower triangularmatrix L and an upper triangular matrix U as A=LU where
		$$ L = \myvec{1 && 0 && 0 \\ l_{21} && 1 && 0 \\ l_{31} && l_{32} && 1} ; U = \myvec{u_{11} && u_{12} && u_{13} \\ 0 && u_{22} && u_{23} \\ 0 && 0 && u_{33} } $$
		Let $x, z \in \mathbb{R}^3$ and $b = \myvec{1 && 1 && 1}$
	\item The solution $z = \myvec{z_1 && z_2 && z_3}$ of the system $Lz=b$ is
		\begin{enumerate}
				\begin{multicols}{2}
				\item $\myvec{-1 && -1 && -2}$
				\item $\myvec{1 && -1 && 2}$
				\item $\myvec{1 && -1 && -2}$
				\item $\myvec{-1 && 1 && 2}$
				\end{multicols}
		\end{enumerate}
	\item The solution $x = \myvec{x_1 && x_2 && x_3}$ of the system $Ux=z$ is
		\begin{enumerate}
				\begin{multicols}{2}
				\item $\myvec{2 && 1 && -2}$
				\item $\myvec{2 && 1 && 2}$
				\item $\myvec{-2 && -1 && -2}$
				\item $\myvec{-2 && 1 && -2}$
				\end{multicols}
		\end{enumerate}
	\item Choose the most appropriate word from the options given below to complete the following sentence. \\
		\textbf{It was her view that the country's problems had been (blank) by foreign technocrats, so that to invite them to come back would be counter productive}
		\begin{enumerate}
				\begin{multicols}{4}
				\item identified
				\item ascertained
				\item exacerbated
				\item analysed
				\end{multicols}
		\end{enumerate}
	\item There are two candidates P and Q in an election. During the campaign, 40 \% of the voters promised to vote for P and rest for Q. However, on the day of election, 15 \% of the voters went back on their promise to vote for P and instead voted for Q. 25 \% of the voters went back on their promise tovote for Q instead of P. Suppose, P lost by 2 votes, then what was the total numbers of voters.
		\begin{enumerate}
				\begin{multicols}{4}
				\item 100
				\item 110
				\item 90
				\item 95
				\end{multicols}
		\end{enumerate}
	\item The question beow consists of a pair of related words followed by four pairs of words. Select the pair that best expresses the relation in the original pair. \\
		\textbf{Gladiator : Arena}
		\begin{enumerate}
				\begin{multicols}{2}
				\item dancer:stage
				\item commuter:train
				\item teacher:classroom
				\item lawyer:courtroom
				\end{multicols}
		\end{enumerate}
	\item Choose the most appropriate word from the options to complete the sentence. \\
		\textbf{Under ethical guidelines recently adopted by the Indian Medical Association, human genes are to be manipulated only to correct diseases for which (blank) treatments are unsatisfactory}
		\begin{enumerate}
				\begin{multicols}{4}
				\item similar
				\item most
				\item uncommon
				\item available
				\end{multicols}
		\end{enumerate}
	\item Choose the word that is nearly the opposite to the given word.\\
		\textbf{Frequency}
		\begin{enumerate}
				\begin{multicols}{4}
				\item periodicity
				\item rarity
				\item gradualness
				\item persistency
				\end{multicols}
		\end{enumerate}
	\item Three friends R, S and T shared from a bowl took $\frac{1}{3}^{rd}$ of the toffees, but returned four to the bowl. S took $\frac{1}{4}^{th}$ of what was left but returned three toffees to the bowl. T took half of the remainder but returned 2 back. If the had 17 toffees left, how many toffees were originally there in the bowl ?
	\begin{enumerate}
			\begin{multicols}{4}
			\item 38
			\item 31
			\item 48
			\item 41
			\end{multicols}
	\end{enumerate}
\item The fuel consumed by a motorcycle during a journey while travelling at various speeds is indicated in the graph below.
	\begin{figure}[H]
		\centering
		\begin{circuitikz}
\tikzstyle{every node}=[font=\normalsize]
\draw  (1.75,15.5) rectangle (10,9.75);
\draw [short] (2.75,11.5) -- (3.5,12.5);
\draw [short] (3.5,12.5) -- (6,13.75);
\draw [short] (6,13.75) -- (7.75,13.25);
\node [font=\normalsize] at (2,9.5) {0};
\node [font=\normalsize] at (3,9.5) {15};
\node [font=\normalsize] at (4,9.5) {30};
\node [font=\normalsize] at (5,9.5) {45};
\node [font=\normalsize] at (6,9.5) {60};
\node [font=\normalsize] at (6.75,9.5) {75};
\node [font=\normalsize] at (7.5,9.5) {90};
\node [font=\normalsize] at (1.25,11) {30};
\node [font=\normalsize] at (1.25,12) {60};
\node [font=\normalsize] at (1.25,13.25) {90};
\node [font=\normalsize] at (1.25,14.25) {120};
\node [font=\normalsize, rotate around={90:(0,0)}] at (0.5,13) {Fuel consumption};
\node [font=\small, rotate around={90:(0,0)}] at (1,13) {(kilometres per litre)};
\node [font=\small] at (5,8.75) {(kilometres per hour)};
\node [font=\normalsize] at (4.75,9.25) {Speed};
\end{circuitikz}
		\caption{}
		\label{25}
	\end{figure}
	The distances covered during four laps o the journey are listed in the table below.
	\begin{table}[h!]
    \centering
    \begin{tabular}{|c|c|c|}
        \hline
        \textbf{LAP} & \textbf{Distance} (kilometres) & \textbf{Average Speed} (kilometres per hour) \\
        \hline
        P & 15 & 15 \\
        \hline
        Q & 75 & 45 \\
        \hline
        R & 40 & 75 \\
        \hline
        S & 10 & 10 \\
        \hline
    \end{tabular}
\end{table}	
	From the given data, we can conclude that the fuel consumed per kilometre was least during the lap
	\begin{enumerate}
			\begin{multicols}{4}
			\item P
			\item Q
			\item R
			\item S
			\end{multicols}
	\end{enumerate}
\item The horse has played a little known but very important role in the field of medicine. Horses were injected with toxins of diseases until their blood built p immunities. Then a serum was made from their blood. Serums to fight with diphtheria and tetanus were developed this way.
	It can be inferred from the passage that horses were
	\begin{enumerate}
		\item given immunity to diseases
		\item generally quite immune to diseases
		\item given medicines to fight toxins
		\item given diphtheria and tetanus serums
	\end{enumerate}
\item The sum of n terms of the series 4 + 44 + 444 + $\dots$ is
	\begin{enumerate}
			\begin{multicols}{2}
			\item $\frac{4}{81} \brak{10^{n+1} - 9n - 1}$
			\item $\frac{4}{81} \brak{10^{n-1} - 9n - 1}$
			\item $\frac{4}{81} \brak{10^{n+1} - 9n - 10}$
			\item $\frac{4}{81} \brak{10^{n} - 9n - 10}$
			\end{multicols}
	\end{enumerate}
\item Given that $f(y) = \frac{\abs{y}}{y}$ and q is any non-zero real number, the value of $\abs{f(q) - f(-q}$ is
	\begin{enumerate}
			\begin{multicols}{4}
			\item 0
			\item -1
			\item 1
			\item 2
			\end{multicols}
	\end{enumerate}



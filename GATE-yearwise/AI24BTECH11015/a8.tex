\iffalse
\chapter{2019}
\author{AI24BTECH11015 - Harshvardhan Patidar}
\section{ma}
\fi
    \item Let $1 \leq p < q < \infty$. Consider the followings statements: 
        \begin{itemize}
            \item [I.] $l^p \subset l^q$
            \item [II.] $L^p \sbrak{0,1} \subset L^q\sbrak{0,1}$,
        \end{itemize}

        where $l^p = \cbrak{\brak{x_1,x_2, \ldots} : x_i \in \mathbb{R}, \sum_{i=1}^{\infty} \abs{x_i}^p < \infty}$ and 
        \begin{align*}
            L^p \sbrak{0,1} = \lcbrak{f:\sbrak{0,1} \rightarrow \mathbb{R}:f \text{ is } \mu-\text{measurable,} \vphantom{\int_{1}^{0}}} \\  \vphantom{\int_{1}^{0}} \rcbrak{\int_{\sbrak{0,1}} \abs{f}^p d \mu < \infty, \text{where } \mu \text{ is the Lebesgue measure}}
        \end{align*}
        $\brak{\mathbb{R} \text{ is the set of all real numbers}}$\\
        Which of the following statements is/are TRUE?

        \begin{enumerate}
            \item Both I and II
            \item I only
            \item II only
            \item Neither I nor II
        \end{enumerate}

        \item Consider the differential equation 
            \begin{align*}
                t \frac{d^2 y}{dt^2} + 2 \frac{dy}{dt} + ty = 0, t>0, y\brak{0+} = 1, \brak{\frac{dy}{dt}}_{t=0+} = 0
            \end{align*}
            If $Y\brak{s}$ is the Laplace transform of $y\brak{t}$, then the value of $Y\brak{1}$ is \_\_\_\_ (round off to $2$ places of decimal). \\
            (Here, the inverse of trigonometric functions assume principal values only)

    \item Let $R$ be the region in the $xy$-plane bounded by the curves $y = x^2$, $y=4x^2$, $xy=1$ and $xy=5$.\\ 
            Then the value of the integral $\int \int_{R} \frac{y^2}{x} dy dx$ is equal to \_\_\_\_.

    \item Let $V$ be the vector space of all $3 \times 3$ matrices with complex entries over the real field. If
            \begin{align*}
                W_1 = \cbrak{A \in V : A = \bar{A}^T} \text{ and } W_2 = \cbrak{A \in V : \text{trace of } A = 0},
            \end{align*}
            then the dimension of $W_1 + W_2$ is equal to \_\_\_\_.\\
            ($\bar{A}^T$ denotes the conjugate transpose of $A$)

    \item The number of elements of order $15$ in the additive group $\mathbb{Z}_{60} \times \mathbb{Z}_{50}$ is \_\_\_\_. \\
            ($\mathbb{Z}_n$ denotes the group of integers modulo $n$, under the operation of addition modulo $n$, for any positive integer $n$)

    \item Consider the following cost matrix of assigning four jobs to four persons:
            \begin{table}[h!]
                \centering
                \begin{tabular}{|c|c|c|c|c|c|}
    \hline
    \multicolumn{2}{|c|}{} & \multicolumn{4}{|c|}{Jobs} \\ \cline{3-6}
    \multicolumn{2}{|c|}{} & J$_1$ & J$_2$ & J$_3$ & J$_4$ \\ \hline
    \multirow{4}{*}{Persons} & P$_1$ & 5 & 8 & 6 & 10 \\ \cline{2-6}
     &P$_2$ & 2 & 5 & 4 & 8 \\ \cline{2-6}
     &P$_3$ & 6 & 7 & 6 & 9 \\ \cline{2-6}
     &P$_4$ & 6 & 9 & 8 & 10 \\ \hline
    \end{tabular}

% \begin{tabular}{|c|c|c|c|c|}
%     \hline
%     & J$_1$ & J$_2$ & J$_3$ & J$_4$ \hline
%     P$_1$ & 5 & 8 & 6 & 10 \hline
%     P$_2$ & 2 & 5 & 4 & 8 \hline
%     P$_3$ & 6 & 7 & 6 & 9 \hline
%     P$_4$ & 6 & 9 & 8 & 10 \hline
% \end{tabular}
                \caption{}
                \label{{48t}}
            \end{table}

        Then the minimum cost of the assignment problem subject to the constraint that job J$_4$ is assigned to the person P$_2$, is \_\_\_\_.
    
    \item Let $y : \sbrak{-1,1} \rightarrow \mathbb{R}$ with $y\brak{1} = 1 $ satisfy the Legendre differential equation
            \begin{align*}
                \brak{1 - x^2} \frac{d^2 y}{d x ^2} - 2 x \frac{dy}{dx} + 6y = 0 \text{ for } \abs{x} < 1.
            \end{align*}
        Then the value of $\int_{-1}^1 y \brak{x} \brak{x+x^2}dx$ is equal to \_\_\_\_ (round off to 2 places of decimal).
        
    \item Let $\mathbb{Z}_{125}$ be the ring of integers modulo $125$ under the operations of addition modulo $125$ and multiplication modulo $125$. If $m$ is the number of maximal ideals of $\mathbb{Z}_{125}$ and $n$ is the number of non-units of $\mathbb{Z}_{125}$, then $m+n$ is equal to \_\_\_\_.

    \item The maximum value of the error term of the composite Trapezoidal rule when it is used to evaluate the definite integral
            \begin{align*}
                \int_{0.2}^{1.4} \brak{\sin x - \log_e x} dx
            \end{align*}
        with $12$ sub-intervals of equal length, is equal to \_\_\_\_ (round off to $3$ places of decimal).

    \item By the Simplex method, the optimal table of the linear programming problem:
            \begin{align*}
                \text{Maximize } &Z = \alpha x_1 + 3 x_2 \\
                \text{subject to } &\beta x_1 + x_2 + x_3 = 8,\\
                &2x_1 + x_2 + x_4 = \gamma \\
                &x_1, x_2, x_3, x_4 \geq 0,
            \end{align*}
        where $\alpha, \beta, \gamma$ are real constants, is
        \begin{table}[h!]
            \centering
            \begin{figure}[!ht]
\centering
\resizebox{0.7\textwidth}{!}{%
\begin{circuitikz}
\tikzstyle{every node}=[font=\normalsize]
\draw (7.5,12.25) to[R,l={ \normalsize 3k$\Omega$}, *-] (7.5,9.75);
\draw (7.5,7) to[Tnpn, transistors/scale=1.19] (7.5,9);
\draw (5,9.75) to[short] (9.25,9.75);
\draw (6.5,8) to[short] (5,8);
\draw (5,8) to[R,l={ \normalsize 150k$\Omega$}] (5,9.75);
\draw (9.25,9.75) to[curved capacitor,l={ \normalsize 20$\mu$F}] (11.25,9.75);
\draw (5,8) to[curved capacitor,l={ \normalsize 20$\mu$F}] (3,8);
\draw (7.5,9.75) to[short] (7.5,9);
\draw (7.5,7) to[R,l={ \normalsize 3k$\Omega$}] (7.5,5.75);
\draw (7.5,5.75) to (7.5,5.5) node[ground]{};
\node [font=\normalsize] at (2.75,8) {$V_i$};
\node [font=\normalsize] at (11.5,9.75) {$V_o$};
\node [font=\normalsize] at (7.5,12.5) {$12V$};
\draw  (7.35,8) circle (0.6cm);
\end{circuitikz}
}%

\label{fig:my_label}
\end{figure}
            \caption{}
            \label{{52t}}
        \end{table}

    Then the value of $\alpha + \beta + \gamma$ is \_\_\_\_.

    \item Consider the inner product space $P_2$ of all polynomials of degree at most $2$ over the field of real numbers with the inner product $\langle f,g \rangle = \int_0^1 f \brak{t} g \brak{t} dt$ for $f,g \in P_2$. \\ Let $\cbrak{f_0, f_1, f_2}$ be an orthogonal set in $P_2$, where $f_0 = 1, f_1 = t+c_1, f_2 = t^2 + c_2 f_1 + c_3$ and $c_1,c_2,c_3$ are real constants. Then the value of $2c_1 + c_2 + 3 c_3$ is equal to \_\_\_\_.

    \item Consider the system of linear differential equations
        \begin{align*}
            \frac{dx_1}{dt} = 5x_1 - 2x_2,\\
            \frac{dx_2}{dt} = 4x_1 - x_2,
        \end{align*}
        with the initial conditions $x_1 \brak{0} = 0, x_2 \brak{0} =1$. \\ Then $\log_e \brak{x_2\brak{2} - x_1 \brak{2}}$ is equal to \_\_\_\_.

    \item Consider the differential equation
        \begin{align*}
            x \brak{1 + x^2} \frac{d^2y}{dx^2} - 9\frac{dy}{dx} + 7y =0.
        \end{align*}
        The sum of the roots of the indicial equation of the Frobenius series solution for the above differential equation in a neighborhood of $x=0$ is equal to \_\_\_\_.

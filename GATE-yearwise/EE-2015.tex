\iffalse
\chapter{2015}
\author{AI24BTECH11028}
\section{ee}
\fi

    \item Consider a function $\overrightarrow f = \frac{1}{r^2}\hat{r}$, where r is the distance from the origin and $\hat{r}$ is the unit vector in the radial direction. The divergence of this function over a sphere of radius R, which includes the origin is  \hfill{[2015-EE]}
    \begin{enumerate}
        \item $0$
        \item $2\pi$
        \item $4\pi$
        \item $R\pi$
    \end{enumerate}

    \item When the wheatstone bridge shown in the figure is used to find the value of resistor $R_X$, the galvanometer G indicates zero current when $R_1 = 50\ohm, R_2 = 65\ohm$ and $R_3 = 100\ohm$. If $R_3$ is known with $\pm 5\%$ tolerance on its nominal value of $100\ohm$, what is the range of $R_X$ in Ohms? \hfill{[2015-EE]}
    \begin{figure}[!ht]
\centering
\resizebox{0.2\textwidth}{!}{%
\begin{circuitikz}
\tikzstyle{every node}=[font=\footnotesize]
\draw (0.75,11.25) to[short] (1.75,11.25);
\draw (0.75,11.25) to[short] (0.75,8);
\draw (2,8) to[battery1] (4,8);
\draw (0.75,8) to[short] (2.5,8);
\draw (4,8) to[short] (6,8);
\draw (6,8) to[short] (6,11.25);
\draw (6,11.25) to[short] (5.25,11.25);
\draw (1.75,11.25) to[R] (3.25,12.75);
\draw (1.75,11.25) to[R] (3.25,9.75);
\draw (3.25,9.75) to[R] (4.75,11.25);
\draw (3.25,12.75) to[R] (4.75,11.25);
\draw (4.75,11.25) to[short] (5.25,11.25);
\draw  (3.25,11.25) circle (0.25cm);
\draw [short] (3.25,12.75) -- (3.25,11.5);
\draw [short] (3.25,11) -- (3.25,9.75);
\node [font=\large] at (3.25,11.25) {G};
\node [font=\large] at (2,12.25) {$R_1$};
\node [font=\large] at (4.5,12.25) {$R_2$};
\node [font=\large] at (2,10.5) {$R_3$};
\node [font=\large] at (4.5,10.25) {$R_X$};
\node [font=\large] at (3,7.5) {$V$};
\node [font=\footnotesize] at (2.75,8.25) {$+$};
\node [font=\footnotesize] at (3.25,8.25) {$-$};
\end{circuitikz}
}%

\label{fig:my_label}
\end{figure}

    \begin{enumerate}
        \item $[123.50, 136.50]$
        \item $[125.89, 134.12]$
        \item $[117.00, 143.00]$
        \item $[120.25, 139.75]$
    \end{enumerate}


    \item A $\brak{0 - 50A}$ moving coil ammeter has a voltage drop of 0.1V across its terminals at full scale deflection. The external shunt resistance (in  milliohms) needed to extend its range to (0-500 A) is \underline{\hspace{2cm}} \hfill{[2015-EE]}

    \item Of the four characteristics given below, which are the major requirements for an instrumentation amplifier? \hfill{[2015-EE]}
    P. High common mode rejection ratio
    Q. High input impedance
    R. High linearity
    S. High output impedance
    \begin{enumerate}
        \item P, Q and R only
        \item P and R only
        \item P, Q and S only
        \item Q, R and S only
    \end{enumerate}

    \item In the following chopper, the duty ratio of swithc S is 0.4. If the inductor and capacitor are sufficiently large to ensure continous inductor current and ripple free capacitor voltage, the changing current(in Ampere) of the 5V battery, under steady-state, is \underline{\hspace{2cm}} \hfill{[2015-EE]}

    \begin{figure}[!ht]
\centering
\resizebox{0.22\textwidth}{!}{%
\begin{circuitikz}
\tikzstyle{every node}=[font=\large]
\draw (1,10.5) to[american voltage source] (1,9.25);
\draw (1,10.5) to[short] (1,11);
\draw (1,9.25) to[short] (1,8.25);
\draw (1,8.25) to[short] (7,8.25);
\draw (7,9) to[battery1] (7,8.25);
\draw (7,10.25) to[R] (7,9);
\draw (7,10.25) to[short] (7,11);
\draw (7,11) to[short] (6,11);
\draw (6,11) to[short] (6,10);
\draw (6,10) to[C] (6,9.25);
\draw (6,9.25) to[short] (6,8.25);
\draw (6,11) to[L ] (4.25,11);
\draw (4.25,11) to[short] (4.25,10);
\draw (4.25,9.5) to[D] (4.25,10);
\draw (4.25,9.5) to[short] (4.25,8.25);
\draw (1,11) to[short] (1.75,11);
\draw (1.75,11) to[short] (1.75,10.5);
\draw  (1.75,10.5) ellipse (0.25cm and 0cm);
\draw (2.5,11) to[short] (4.25,11);
\draw (3.25,11) to[short] (3.25,10.5);
\draw  (3.25,10.5) ellipse (0.25cm and 0cm);
\draw [->, >=Stealth] (2.5,11) -- (2.5,10.5);
\draw (1.5,10.5) to[short] (2,10.5);
\draw (1.5,10.25) to[short] (3.25,10.25);
\draw (3.25,10.25) to[short] (3.25,9.5);
\draw (2.25,10.5) to[short] (2.75,10.5);
\node [font=\large] at (0.5,9.25) {20V};
\node [font=\large] at (2.5,11.5) {S};
\node [font=\large] at (5,11.5) {L};
\node [font=\large] at (7.75,8.75) {5V};
\node [font=\large] at (7.5,9.75) {$3\Omega$};
\node [font=\large] at (5.25,9.5) {$C$};
\end{circuitikz}
}%

\label{fig:my_label}
\end{figure}


    \item  A moving average function is given by $y\brak{t} = \frac{1}{T}\int_{t-T}^{t}u\brak{\tau}d\tau$. If the input $u$ is a sinusoidal signal of frequency $\frac{1}{2T}$Hz, then in steady state, the output y will lag $u$ (in degree) by \underline{\hspace{2cm}} \hfill{[2015-EE]}

    \item The impulse response $g\brak{t}$ of a system, G, is as shown in Figure(a). What is the maximum value attained by the impulse response of two cascaded blocks of G as shown in Figure(b) \hfill{[2015-EE]}
    \begin{figure}[!ht]
\centering
\resizebox{0.6\textwidth}{!}{%
\begin{circuitikz}
\tikzstyle{every node}=[font=\normalsize]
\draw [->, >=Stealth] (1.25,8.5) -- (1.25,12.25);
\draw [->, >=Stealth] (1.25,8.5) -- (5,8.5);
\draw [short] (1.25,10.5) -- (3,10.5);
\draw [short] (3,10.5) -- (3,8.5);
\draw [->, >=Stealth] (6,10) -- (8,10);
\draw  (8,10.75) rectangle (10.25,9);
\draw [->, >=Stealth] (10.25,10) -- (11.75,10);
\draw  (11.75,10.75) rectangle (14.25,9);
\draw [->, >=Stealth] (14.25,10) -- (15.75,10);
\node [font=\large] at (9.25,10) {G};
\node [font=\large] at (13,10) {G};
\node [font=\large] at (1,10.5) {1};
\node [font=\large] at (3,8.25) {1};
\node [font=\large] at (1,8.25) {0};
\node [font=\normalsize] at (1.25,12.5) {g(t)};
\node [font=\normalsize] at (5.25,8.5) {t};
\node [font=\normalsize] at (3,7.5) {(a)};
\node [font=\normalsize] at (11.5,7.5) {(b)};
\end{circuitikz}
}%

\label{fig:my_label}
\end{figure}
\begin{enumerate}
    \item $\frac{2}{3}$
    \item $\frac{3}{4}$
    \item $\frac{4}{5}$
    \item $1$
\end{enumerate}

    \item Consider a one-turn rectangular loop of wire placed in a uniform magnetic field as shown in the figure. The plane of the loop is perpendicular to the field lines. The resistance of the loop is $0.4\ohm$, and its inductance negligible. The magnetic flux density(in Tesla) is a function of time and is given by $B\brak{t} = 0.25 sin\omega t$, where $\omega = 2\pi\times 50$ radian/second. The power absorbed (in Wat) by the loop from the magnetic field is  \hfill{[2015-EE]} \begin{figure}[!ht]
\centering
\resizebox{0.3\textwidth}{!}{%
\begin{circuitikz}
\tikzstyle{every node}=[font=\large]
\draw [ line width=0.5pt ] (-1.5,15.75) circle (0.75cm);
\draw [line width=0.5pt, short] (-2,15.25) -- (-1,16.25);
\draw [line width=0.5pt, short] (-1,15.25) -- (-2,16.25);
\draw [ line width=0.5pt ] (1.25,15.75) circle (0.75cm);
\draw [line width=0.5pt, short] (0.75,15.25) -- (1.75,16.25);
\draw [line width=0.5pt, short] (1.75,15.25) -- (0.75,16.25);
\draw [ line width=0.5pt ] (3.75,15.75) circle (0.75cm);
\draw [line width=0.5pt, short] (3.25,15.25) -- (4.25,16.25);
\draw [line width=0.5pt, short] (4.25,15.25) -- (3.25,16.25);
\draw [ line width=0.5pt ] (6.5,15.75) circle (0.75cm);
\draw [line width=0.5pt, short] (6,15.25) -- (7,16.25);
\draw [line width=0.5pt, short] (7,15.25) -- (6,16.25);
\draw [ line width=0.5pt ] (9.25,15.75) circle (0.75cm);
\draw [line width=0.5pt, short] (8.75,15.25) -- (9.75,16.25);
\draw [line width=0.5pt, short] (9.75,15.25) -- (8.75,16.25);
\draw [ line width=0.5pt ] (-1.5,13.25) circle (0.75cm);
\draw [line width=0.5pt, short] (-2,12.75) -- (-1,13.75);
\draw [line width=0.5pt, short] (-1,12.75) -- (-2,13.75);
\draw [ line width=0.5pt ] (1.25,13.25) circle (0.75cm);
\draw [line width=0.5pt, short] (0.75,12.75) -- (1.75,13.75);
\draw [line width=0.5pt, short] (1.75,12.75) -- (0.75,13.75);
\draw [ line width=0.5pt ] (3.75,13.25) circle (0.75cm);
\draw [line width=0.5pt, short] (3.25,12.75) -- (4.25,13.75);
\draw [line width=0.5pt, short] (4.25,12.75) -- (3.25,13.75);
\draw [ line width=0.5pt ] (6.5,13.25) circle (0.75cm);
\draw [line width=0.5pt, short] (6,12.75) -- (7,13.75);
\draw [line width=0.5pt, short] (7,12.75) -- (6,13.75);
\draw [ line width=0.5pt ] (9.25,13.25) circle (0.75cm);
\draw [line width=0.5pt, short] (8.75,12.75) -- (9.75,13.75);
\draw [line width=0.5pt, short] (9.75,12.75) -- (8.75,13.75);
\draw [ line width=0.5pt ] (-1.5,10.25) circle (0.75cm);
\draw [line width=0.5pt, short] (-2,9.75) -- (-1,10.75);
\draw [line width=0.5pt, short] (-1,9.75) -- (-2,10.75);
\draw [ line width=0.5pt ] (1.25,10.25) circle (0.75cm);
\draw [line width=0.5pt, short] (0.75,9.75) -- (1.75,10.75);
\draw [line width=0.5pt, short] (1.75,9.75) -- (0.75,10.75);
\draw [ line width=0.5pt ] (3.75,10.25) circle (0.75cm);
\draw [line width=0.5pt, short] (3.25,9.75) -- (4.25,10.75);
\draw [line width=0.5pt, short] (4.25,9.75) -- (3.25,10.75);
\draw [ line width=0.5pt ] (6.5,10.25) circle (0.75cm);
\draw [line width=0.5pt, short] (6,9.75) -- (7,10.75);
\draw [line width=0.5pt, short] (7,9.75) -- (6,10.75);
\draw [ line width=0.5pt ] (9.25,10.25) circle (0.75cm);
\draw [line width=0.5pt, short] (8.75,9.75) -- (9.75,10.75);
\draw [line width=0.5pt, short] (9.75,9.75) -- (8.75,10.75);
\draw [ line width=0.5pt ] (-1.5,7.5) circle (0.75cm);
\draw [line width=0.5pt, short] (-2,7) -- (-1,8);
\draw [line width=0.5pt, short] (-1,7) -- (-2,8);
\draw [ line width=0.5pt ] (1.25,7.5) circle (0.75cm);
\draw [line width=0.5pt, short] (0.75,7) -- (1.75,8);
\draw [line width=0.5pt, short] (1.75,7) -- (0.75,8);
\draw [ line width=0.5pt ] (3.75,7.5) circle (0.75cm);
\draw [line width=0.5pt, short] (3.25,7) -- (4.25,8);
\draw [line width=0.5pt, short] (4.25,7) -- (3.25,8);
\draw [ line width=0.5pt ] (6.5,7.5) circle (0.75cm);
\draw [line width=0.5pt, short] (6,7) -- (7,8);
\draw [line width=0.5pt, short] (7,7) -- (6,8);
\draw [ line width=0.5pt ] (9.25,7.5) circle (0.75cm);
\draw [line width=0.5pt, short] (8.75,7) -- (9.75,8);
\draw [line width=0.5pt, short] (9.75,7) -- (8.75,8);
\draw [ line width=1.4pt ] (-0.5,14.5) rectangle (8.5,9.25);
\draw [short] (-0.5,14.5) -- (-0.25,14.5);
\draw [short] (-0.5,14.75) -- (-0.5,18);
\draw [short] (8.5,14.75) -- (8.5,18);
\draw [short] (8.75,14.5) -- (11.75,14.5);
\draw [short] (8.75,9.25) -- (11.75,9.25);
\draw [line width=0.6pt, <->, >=Stealth] (-0.5,17.75) -- (8.5,17.75);
\draw [line width=0.6pt, <->, >=Stealth] (11.5,14.5) -- (11.5,9.25);
\node [font=\large] at (4.25,18.25) {10 cm};
\node [font=\large] at (12.25,11.75) {5 cm};
\end{circuitikz}
}%

\label{fig:my_label}
\end{figure}

\item A steady current I is flowing in -x direction through each of the two infinitely long wires at $y = \pm \frac{L}{2}$ as shown in the figure. The permeability of the medium is $\mu_{0}$. The $\overrightarrow B$- field at $\brak{0, L, 0}$ is  \hfill{[2015-EE]}
\begin{figure}[!ht]
\centering
\resizebox{0.3\textwidth}{!}{%
\begin{circuitikz}
\tikzstyle{every node}=[font=\LARGE]
\draw [ line width=0.5pt ] (1.75,11) circle (0.75cm);
\draw [line width=0.5pt, short] (1.25,10.5) -- (2.25,11.5);
\draw [line width=0.5pt, short] (2.25,10.5) -- (1.25,11.5);
\draw [ line width=0.5pt ] (9.75,11) circle (0.75cm);
\draw [line width=0.5pt, short] (9.25,10.5) -- (10.25,11.5);
\draw [line width=0.5pt, short] (10.25,10.5) -- (9.25,11.5);
\draw [line width=0.6pt, short] (2.5,11) -- (9,11);
\draw [line width=0.6pt, ->, >=Stealth] (10.5,11) -- (14,11);
\draw [line width=0.6pt, ->, >=Stealth] (1,11) -- (-2.75,11);
\draw [line width=0.6pt, ->, >=Stealth] (5.75,11) -- (5.75,19);
\draw [line width=0.6pt, <->, >=Stealth] (10.25,16.5) -- (1,5);
\node [font=\Large] at (0.5,9.75) {Current=I};
\node [font=\Large] at (10.5,9.75) {Current=I};
\node [font=\Large] at (14,11.5) {y};
\node [font=\Large] at (5.5,19.5) {z};
\node [font=\LARGE] at (9.5,12.5) {y = L/2};
\node [font=\LARGE] at (1.5,12.25) {y = -L/2};
\node [font=\LARGE] at (6,10.5) {0};
\end{circuitikz}
}%

\label{fig:my_label}
\end{figure}

\begin{enumerate}
    \item $-\frac{4\mu_{0}I}{3\pi L} \hat{Z}$
    \item $+\frac{4\mu_{0}I}{3\pi L} \hat{Z}$
    \item $0$
    \item $-\frac{3\mu_{0}I}{4\pi L} \hat{Z}$
\end{enumerate}


\item Consider the circuit shown in the figure. In this circuit $R = 1 k\ohm$ and $C = 1 \mu F$. The input voltage is sinusoidal with a frequency of 5-Hz, represented as a phasor with magnitude $V_i$ and phase angle 0 radian as shown in the figure. The output voltage is represented as phasor with magnitude $V_0$ and phase angle $\delta$ radian. What is the value of the output phase angle $\delta$(in radian) relative to the phase angle of the input voltage? \hfill{[2015-EE]}
\begin{figure}[!ht]
\centering
\resizebox{0.4\textwidth}{!}{%
\begin{circuitikz}
\tikzstyle{every node}=[font=\LARGE]
\draw [ line width=0.6pt](-0.25,13.5) to[short, -o] (-2.25,13.5) ;
\draw [line width=0.6pt](-0.25,13.5) to[C] (0.5,13.5);
\draw [ line width=0.6pt](0.5,13.5) to[short] (1.5,13.5);
\draw [ line width=0.6pt](1.5,13.5) to[short] (3.5,13.5);
\draw [ line width=0.6pt](-0.25,11.5) to[short, -o] (-2.25,11.5) ;
\draw [line width=0.6pt](-0.25,11.5) to[C] (0.5,11.5);
\draw [ line width=0.6pt](0.5,11.5) to[short] (1.5,11.5);
\draw [ line width=0.6pt](1.5,11.5) to[short] (3.5,11.5);
\draw [ line width=0.6pt](3.5,14) to[short] (3.5,11);
\draw [line width=0.6pt, short] (3.5,11) -- (6.5,12.5);
\draw [line width=0.6pt, short] (3.5,14) -- (6.5,12.5);
\draw [ line width=0.6pt](6.5,12.5) to[short] (7.75,12.5);
\node at (7.75,12.5) [circ] {};
\draw [ line width=0.6pt](7.75,12.5) to[short, -o] (9,12.5) ;
\draw [ line width=0.6pt](7.75,12.5) to[short] (7.75,15.75);
\draw [ line width=0.6pt](7.75,15.75) to[R] (1.5,15.75);
\draw [ line width=0.6pt](1.5,15.75) to[short] (1.5,13.5);
\node at (1.5,13.5) [circ] {};
\node at (1.5,11.5) [circ] {};
\draw [ line width=0.6pt](1.5,11.5) to[R] (1.5,8.75);
\draw [line width=0.6pt](1.5,8.75) to (1.5,8.5) node[ground]{};
\draw [line width=0.6pt, ->, >=Stealth] (-1.75,12) -- (-1.75,13);
\node [font=\LARGE] at (-4,12.5) {$v_i = V_i$};
\draw [ line width=0.6pt](-3,12.25) to[short] (-2.5,12.75);
\draw [ line width=0.6pt](-3,12.25) to[short] (-2.25,12.25);
\node [font=\large] at (-2.5,12.5) {$0$};
\node [font=\large] at (0.25,14.25) {$C$};
\node [font=\large] at (0,10.75) {$C$};
\node [font=\large] at (2,10) {$R$};
\node [font=\large] at (4.75,16.25) {$R$};
\node [font=\large] at (3.75,13.5) {$-$};
\node [font=\large] at (3.75,11.75) {$+$};
\node [font=\large] at (10,12.5) {$v_0 = V_0$};
\draw [line width=0.6pt, short] (10.75,12.25) -- (11.25,12.25);
\draw [line width=0.6pt, short] (10.75,12.25) -- (11,12.75);
\node [font=\normalsize] at (11.25,12.5) {$\delta$};
\end{circuitikz}
}%

\label{fig:my_label}
\end{figure}

\begin{enumerate}
    \item $0$
    \item $\pi$
    \item $\frac{\pi}{2}$
    \item $-\frac{\pi}{2}$
\end{enumerate}

\item In the given circuit, the silicon tranistor has $\beta = 75$ and a collector voltage $V_C = 9V$. Then the ratio of $R_B$ and $R_C$ is \underline{\hspace{2cm}} \hfill{[2015-EE]}
\begin{figure}[!ht]
\centering
\resizebox{0.25\textwidth}{!}{%
\begin{circuitikz}
\tikzstyle{every node}=[font=\large]
\draw (1.5,11) to[short] (1.5,9.5);
\draw (1.5,11) to[R] (4,11);
\draw (4,11) to[R] (4,13);
\draw (4,11) to[short] (4,10.25);
\draw (1.5,9.5) to[short] (3.25,9.5);
\draw (3.25,10) to[short] (3.25,9);
\draw [short] (3.25,9.75) -- (4,10.25);
\draw [->, >=Stealth] (3.25,9.25) -- (4,8.75);
\draw [short] (4,8.75) -- (4,8.25);
\draw (4,8.25) to (4,8) node[ground]{};
\draw (4,11) to[short] (5,11);
\node at (4,11) [circ] {};
\node at (5,11) [circ] {};
\node at (4,13) [circ] {};
\node [font=\large] at (2.75,11.5) {$R_B$};
\node [font=\large] at (4.5,12.25) {$R_C$};
\node [font=\large] at (5.5,11) {$V_c$};
\end{circuitikz}
}%

\label{fig:my_label}
\end{figure}

\item In the $4 \times 1$, the output F is given by $F = A\oplus B$. Find the required input $'I_3I_2I_1I_0'$ \hfill{[2015-EE]}
\begin{figure}[!ht]
\centering
\resizebox{0.25\textwidth}{!}{%
\begin{circuitikz}
\tikzstyle{every node}=[font=\large]
\draw  (3.25,13) rectangle (6.5,8.75);
\draw (2,12.25) to[short] (3.25,12.25);
\draw (2,11.5) to[short] (3.25,11.5);
\draw (2,10.5) to[short] (3.25,10.5);
\draw (2,9.5) to[short] (3.25,9.5);
\draw (6.5,11.25) to[short] (7.5,11.25);
\draw (4.5,8.75) to[short] (4.5,7.75);
\draw (5.5,8.75) to[short] (5.5,7.75);
\node at (2,12.25) [circ] {};
\node at (2,11.5) [circ] {};
\node at (2,10.5) [circ] {};
\node at (2,9.5) [circ] {};
\node at (4.5,7.75) [circ] {};
\node at (5.5,7.75) [circ] {};
\node at (7.5,11.25) [circ] {};
\node [font=\large] at (3.5,12.25) {$I_0$};
\node [font=\large] at (3.5,11.5) {$I_1$};
\node [font=\large] at (3.5,10.5) {$I_2$};
\node [font=\large] at (3.5,9.5) {$I_3$};
\node [font=\large] at (4.75,11.5) {$4 \times 1$};
\node [font=\normalsize] at (4.5,9) {$s_1$};
\node [font=\normalsize] at (5.5,9) {$s_0$};
\node [font=\large] at (4.75,10.75) {MUX};
\node [font=\large] at (7.75,11.25) {F};
\node [font=\large] at (4.5,7.25) {A};
\node [font=\large] at (5.5,7.25) {B};
\end{circuitikz}
}%

\label{fig:my_label}
\end{figure}

\begin{enumerate}
    \item 1010
    \item 0110
    \item 1000
    \item 1110
\end{enumerate}

\item Consider a HVDC link which used thryistor based line-commutated converters as shown in the figure. For a power flow of 750 MW from a System 1 to System 2, the voltage at the two ends, and the current, are given by : $V_1 = 500 kV, V_2 = 485 kV$ and $I = 1.5 kA$. If the direction of power flow is to be reversed(that is, from System 2 to System 1) without changing the electrical connections, then which one of the following combinations is feasible? \hfill{[2015-EE]}
\begin{figure}[!ht]
\centering
\resizebox{0.5\textwidth}{!}{%
\begin{circuitikz}
\tikzstyle{every node}=[font=\large]
\draw  (1.25,11.5) rectangle (3.25,9.75);
\draw  (7.5,11.5) rectangle (9.5,9.75);
\draw (3.25,11.25) to[R] (7.5,11.25);
\draw (3.25,10) to[short] (4.75,10);
\draw (5,10) to[short] (5.5,10);
\draw (5.75,10) to[short] (7.5,10);
\draw (4.5,10) to[short] (6,10);
\draw (1.25,11.25) to[short] (0.25,11.25);
\draw (1.25,10.75) to[short] (0.25,10.75);
\draw (1.25,10.25) to[short] (0.25,10.25);
\draw (9.5,11.25) to[short] (10.75,11.25);
\draw (9.5,10.75) to[short] (10.75,10.75);
\draw (9.5,10.25) to[short] (10.75,10.25);
\draw [->, >=Stealth] (4.5,12) -- (6.25,12);
\draw [->, >=Stealth] (7.25,10.25) -- (7.25,11.25);
\draw [->, >=Stealth] (3.5,10.25) -- (3.5,11.25);
\draw [short] (1.75,10) -- (2.75,10);
\draw [short] (1.75,10) -- (2.25,10.75);
\draw [short] (2.75,10) -- (2.25,10.75);
\draw [short] (1.75,10.75) -- (2.75,10.75);
\draw [short] (2.25,10.75) -- (2.75,11.25);
\draw [short] (2.25,10.75) -- (2.25,11.25);
\draw [short] (7.75,11.25) -- (9,11.25);
\draw [short] (7.75,11.25) -- (8.5,10.5);
\draw [short] (9,11.25) -- (8.5,10.5);
\draw [short] (7.75,10.5) -- (9,10.5);
\draw [short] (8.5,10.5) -- (8,10);
\draw [short] (8.5,10.5) -- (8.5,10);
\node [font=\normalsize] at (1.5,11.75) {System 1};
\node [font=\normalsize] at (9,12) {System 2};
\node [font=\normalsize] at (3.75,10.6) {\( V_1 \)};
\node [font=\normalsize] at (7,10.6) {\( V_2 \)};
\node [font=\small] at (3.75,11) {+};
\node [font=\small] at (7,11) {+};
\node [font=\small] at (3.75,10.25) {-};
\node [font=\small] at (7,10.25) {-};
\node [font=\large] at (6.5,12) {\( I \)};
\end{circuitikz}
}%

\label{fig:my_label}
\end{figure}

\begin{enumerate}
    \item $V_1 = -500kV, V_2 = -485kV$ and $I = 1.5kA$
    \item $V_1 = -485kV, V_2 = -500kV$ and $I = 1.5kA$
    \item $V_1 = 500kV, V_2 = 485kV$ and $I = -1.5kA$
    \item $V_1 = -500kV, V_2 = -485kV$ and $I = -1.5kA$
\end{enumerate}




    




\iffalse
\author{EE24BTECH11049}
\section{xe}
\chapter{2018}
\fi

\subsection{General Aptitude}
\subsubsection{Carry One Mark each}
	%1st Question
	\item
	"Going by the \rule{1cm}{0.1pt} that many hands make light work, the school \rule{1cm}{0.1pt} involved all the students in the task."
	The words that best fill the blanks in the above sentence are

	\hfill{\brak{\text{2018-XE}}}

	\begin{multicols}{2}
		\begin{enumerate}
			\item principle, principal
			\item principal, principle
			\item principle, principle
			\item principal, principal
		\end{enumerate}
	\end{multicols}

	%2nd Question
	\item 
	"Her \rule{1cm}{0.1pt} should not be confused with miserliness; she is ever willing to assist those in need."

	The word that best fills the blank in the above sentence is
	
	\hfill{\brak{\text{2018-XE}}}

	\begin{multicols}{2}
		\begin{enumerate}
			\item cleanliness 
			\item punctuality
			\item frugality
			\item greatness
		\end{enumerate}
	\end{multicols}

	%3rd Question
	\item
	Seven machines take 7 minutes to make 7 identical toys. At the same rate, how many minutes would it take for 100 machines to make 100 toys?

	\hfill{\brak{\text{2018-XE}}}

	\begin{multicols}{4}
		\begin{enumerate}
			\item 1
			\item 7
			\item 100
			\item 700
		\end{enumerate}
	\end{multicols}

	%4th Question
	\item
	A rectangle becomes a square when its length and breadth are reduced by $10m$ and $5m$, respectively. During this process, the rectangle loses $650 m^2$ of area. What is the area of the original rectangle in square meters?

	\hfill{\brak{\text{2018-XE}}}

	\begin{multicols}{4}
		\begin{enumerate}
			\item 1125
			\item 2250
			\item 2924
			\item 4500
		\end{enumerate}
	\end{multicols}

	%5th Question
	\item 
	A number consists of two digits. The sum of the digits is 9. If 45 is subtracted from the number, its digits are interchanged. What is the number?

	\hfill{\brak{\text{2018-XE}}}

	\begin{multicols}{4}
		\begin{enumerate}
			\item 63
			\item 72
			\item 81
			\item 90
		\end{enumerate}
	\end{multicols}

\subsubsection{Carry two marks each}
	
	%6th Question 
	\item 
	For integers $a$, $b$ and $c$, what would be the minimum and maximum values respectively of $a + b + c$ if $\log{\abs{x}} + \log{\abs{b} + \log{\abs{c}} = 0}$?

	\hfill{\brak{\text{2018-XE}}}
	\begin{multicols}{4}
		\begin{enumerate}
			\item -3 and 3
			\item -1 and 1
			\item -1 and 3
			\item 1 and 3
		\end{enumerate}
	\end{multicols}

	%7th Question
	\item 
	Given that $a$ and $b$ are integers and $a + a^2b^3$ is odd, which one of the following statements is correct?

	\begin{enumerate}
		\item $a$ and $b$ are both odd
		\item $a$ and $b$ are both even 
		\item $a$ is even and $b$ is odd 
		\item $a$ is odd and $b$ is even 
	\end{enumerate}

	%8th Question 
	\item 
	From the time the front of a train enters a platform, it takes 25 seconds for the back of the train to leave the platform, while travelling at a constant speed of $54 \frac{km}{h}$. At the same speed, it takes 14 seconds to pass a man running at $9\frac{km}{h}$ in the same direction as the train. What is the length of the train and that of the platform in meters, respectively?

	\hfill{\brak{\text{2018-XE}}}
	\begin{multicols}{2}
		\begin{enumerate}
			\item 210 and 140
			\item 162.5 and 187.5
			\item 245 and 130
			\item 175 and 200
		\end{enumerate}
	\end{multicols}

	%9th Question
	\item 
	Which of the following functions describe the graph shown in the below figure?
	
	\begin{figure}[H]
		\centering
		\resizebox{0.4\textwidth}{!}{\begin{circuitikz}
\tikzstyle{every node}=[font=\large]
\draw [dashed] (2.25,11.25) -- (2.25,5.25);
\draw [dashed] (3,11.25) -- (3,5.25);
\draw [<->, >=Stealth] (1.5,8.25) -- (7.5,8.25);
\draw [<->, >=Stealth] (4.5,11.25) -- (4.5,5.25);
\draw [dashed] (3.75,11.25) -- (3.75,5.25);
\draw [dashed] (5.25,11.25) -- (5.25,5.25);
\draw [dashed] (6,11.25) -- (6,5.25);
\draw [dashed] (6.75,11.25) -- (6.75,5.25);
\draw [dashed] (1.5,10.5) -- (7.5,10.5);
\draw [dashed] (1.5,9.75) -- (7.5,9.75);
\draw [dashed] (1.5,9) -- (7.5,9);
\draw [dashed] (1.5,7.5) -- (7.25,7.5);
\draw [dashed] (1.5,6.75) -- (7.25,6.75);
\draw [dashed] (1.5,6) -- (7.5,6);
\draw  (1.5,11.25) rectangle (7.5,5.25);
\draw (2.25,9) to[short] (3.75,7.5);
\draw (3.75,7.5) to[short] (4.5,8.25);
\draw (4.5,8.25) to[short] (5.25,7.5);
\draw (5.25,7.5) to[short] (6.75,9);
\node [font=\large] at (2,8) {-3};
\node [font=\large] at (2.75,8) {-2};
\node [font=\large] at (3.5,8) {-1};
\node [font=\large] at (5.5,8) {1};
\node [font=\large] at (6.25,8) {2};
\node [font=\large] at (7,8) {3};
\node [font=\large] at (4.25,8.5) {0};
\node [font=\large] at (4.75,10.75) {3};
\node [font=\large] at (4.75,10) {2};
\node [font=\large] at (4.75,9.25) {1};
\node [font=\large] at (4.75,7.25) {-1};
\node [font=\large] at (4.75,6.5) {-2};
\node [font=\large] at (4.75,5.75) {-3};
\draw [ line width=1pt ] (1.5,11.25) rectangle (7.5,5.25);
\end{circuitikz}}
	\end{figure}

	\hfill{\brak{\text{2018-XE}}}
	\begin{multicols}{2}
		\begin{enumerate}
			\item $y = \abs{\abs{x} + 1} - 2$
			\item $y = \abs{\abs{x} - 1} - 1$
			\item $y = \abs{\abs{x} + 1} - 1$
			\item $y = \abs{\abs{x - 1} - 1}$
		\end{enumerate}
	\end{multicols}

	%10th Question 
	\item 
	Consider the following three statements:
	\begin{enumerate}[label=\roman*]
		\item Some roses are red 
		\item All the flowers fade quickly
		\item some roses fade quickly
	\end{enumerate}
	Which of the following statements can be logically inferred from the above statements?

	\hfill{\brak{\text{2018-XE}}}
	\begin{enumerate}
		\item If \brak{i} is true \brak{ii} is false, then \brak{iii} is false
		\item If \brak{i} is true \brak{ii} is false, then \brak{iii} is true	
		\item If \brak{i} and \brak{ii} is true, then \brak{iii} is true
		\item If \brak{i} and \brak{ii} are false, then \brak{iii} is false
	\end{enumerate}

\subsection{XE - A}
\subsubsection{Carry One mark each}
	%11th Question 
	\item 
	The largest interval in which the initial value problem 
	\begin{align*}
		e^x \frac{d^2y}{dx^2} + \frac{1}{x - 5}\frac{dy}{dx} + \brak{\sqrt{x}}y = \ln{x}
	\end{align*}
	$y\brak{1} = 0$ and $\frac{dy}{dx}\brak{1} = 1$ has a unique solution is 

	\hfill{\brak{\text{2018-XE}}}
	\begin{multicols}{4}
		\begin{enumerate}
			 \item $\brak{-\infty,\infty}$
			 \item $\brak{-5,5}$
			 \item $\brak{0,\infty}$
			 \item $\brak{0,5}$
		\end{enumerate}
	\end{multicols}

	%12th Question 
	\item 
	The sum of the roots of the indicial equation at $x = 0$ of the differential equation
	\begin{align*}
		x^3 \frac{d^2y}{dx^2} + \brak{x\sin{x}}\frac{dy}{dx} - \brak{\tan{x}}y = 0, x > 0
	\end{align*}
	is 

	\hfill{\brak{\text{2018-XE}}}
	\begin{multicols}{4}
		\begin{enumerate}
			\item 0
			\item 1
			\item 2
			\item -2
		\end{enumerate}
	\end{multicols}

	%13th Question
	\item 
	let $f$ be a three times continuously diferentiable real valued function on $\brak{0,5}$ such that its third derivative $f^{\prime\prime\prime}\brak{x} = \frac{1}{100} \forall x \in \brak{0,5}$. If $P\brak{x}$ is a polynomial of the degree $\leq 2$ such that $P\brak{1} = f\brak{1}, P\brak{2} = f\brak{2} \text{ and } P\brak{3} = f\brak{3}$ then $\abs{f\brak{4} - P\brak{4}}$ equals \rule{1cm}{0.1pt}

	\hfill{\brak{\text{2018-XE}}}


\iffalse
\author{EE24BTECH11030}
\chapter{2013}
\section{xe}
%\author{EE24BTECH11030)
\fi
    \item Consider the function \( f(z) = z^2 \overline{z} \), \( z \in \mathbb{C} \). At \( z = 0 \), the function \( f \)
    \begin{enumerate}
        \item does not satisfy the Cauchy-Riemann equations
        \item satisfies the Cauchy-Riemann equations but is not differentiable
        \item is differentiable but not analytic
        \item is analytic
    \end{enumerate}
    \bigskip
    \item The integral \( \oint_C \frac{(z+4)}{(z+1)(z-2)^3} \, dz \) along the contour \( C : |z - (1 + i)| = 2 \), oriented anti-clockwise, is equal to
    \begin{multicols}{4}
    \begin{enumerate}
        \item 0
        \item \( \frac{2 \pi i}{9} \)
        \item \( -\frac{2 \pi i}{9} \)
        \item \( \frac{4 \pi i}{9} \)
    \end{enumerate}
    \end{multicols}
\bigskip
    \item The integral \( \int_0^1 \int_{x}^{x^2} \left( \frac{x}{y} \right) e^{-x^2/y} \, dy \, dx \) equals
    \begin{multicols}{4}
    \begin{enumerate}
        \item \( \frac{e - 2}{e} \)
        \item \( \frac{e - 1}{2e} \)
        \item \( \frac{e - 1}{2} \)
        \item \( \frac{e - 2}{2e} \)
    \end{enumerate}
    \end{multicols}
\bigskip
    \item If the mean and variance of a binomial distribution are 6 and 2 respectively, then the probability of two failures is
    \begin{multicols}{4}
    \begin{enumerate}
        \item \( 4 \left( \frac{2}{3} \right)^7 \)
        \item \( 4 \left( \frac{22}{37} \right) \)
        \item \( 17 \left( \frac{2}{3} \right)^7 \)
        \item \( 17 \left( \frac{22}{37} \right) \)
    \end{enumerate}
    \end{multicols}
\bigskip
    \item For the matrix \( M = \begin{pmatrix} 1 & 0 & -1 \\ 0 & 1 & -1 \\ 1 & 1 & -2 \end{pmatrix} \), consider the following statements:
    
    \begin{itemize}
        \item[(P)] The characteristic equation of \( M \) is \( \lambda^3 - \lambda = 0 \).
        \item[(Q)] \( M^{-1} \) does not exist.
        \item[(R)] The matrix \( M \) is diagonalizable.
    \end{itemize}

    Which of the above statements are true?
    \begin{multicols}{2}
    \begin{enumerate}
        \item P, Q and R
        \item P and R but not Q
        \item P and Q but not R
        \item Q and R but not P
    \end{enumerate}
    \end{multicols}
\bigskip
    \item The work done by the force \( \vec{F} = (x + x^2) \, \hat{i} + (x^2 + y^3) \, \hat{j} \) in moving a particle once along the triangle with vertices \( (0,0), (1,0) \) and \( (0,1) \) in the anti-clockwise direction is
    \begin{multicols}{4}
    \begin{enumerate}
        \item 0
        \item \( \frac{1}{6} \)
        \item \( \frac{1}{3} \)
        \item \( \frac{5}{3} \)
    \end{enumerate}
    \end{multicols} 
\bigskip   
    \item The general solution of the differential equation
    \[
    x^3 \frac{d^3 y}{dx^3} + x^2 \frac{d^2 y}{dx^2} + x \frac{dy}{dx} - y = 0, \quad x > 0
    \]
    is
    \begin{enumerate}
        \item \( C_1 e^x + e^{x/2} \left\{ C_2 \cos\left( \frac{\sqrt{3}}{2} x \right) + C_3 \sin\left( \frac{\sqrt{3}}{2} x \right) \right\} \)
        \item \( C_1 x + x^{-1/2} \left\{ C_2 \cos\left( \frac{\sqrt{3}}{2} \log_e x \right) + C_3 \sin\left( \frac{\sqrt{3}}{2} \log_e x \right) \right\} \)
        \item \( C_1 e^x + e^{-x/2} \left\{ C_2 \cos\left( \frac{\sqrt{3}}{2} x \right) + C_3 \sin\left( \frac{\sqrt{3}}{2} x \right) \right\} \)
        \item \( C_1 x + x^{1/2} \left\{ C_2 \cos\left( \frac{\sqrt{3}}{2} \log_e x \right) + C_3 \sin\left( \frac{\sqrt{3}}{2} \log_e x \right) \right\} \)
    \end{enumerate}
\bigskip
    \item Using Euler's method to solve the differential equation
    \[
    \frac{dy}{dx} = 2 \cos \left( \frac{4 \pi x}{3} \right) - y, \quad y(0) = 1
    \]
    with step-size \( h = 0.25 \), the value of \( y(0.5) \) is
    \begin{multicols}{4}
    \begin{enumerate}
        \item 1.3125
        \item 1.1875
        \item 1.125
        \item 1.0625
    \end{enumerate}
    \end{multicols}
\bigskip    
    \item The gauge pressure inside a soap bubble of radius $R$, with $\sigma$ denoting the surface tension between the soap solution and air, is:
    \begin{multicols}{4}
    \begin{enumerate}
        \item $\frac{\sigma}{2 \pi R}$
        \item $\frac{4 \sigma}{R}$
        \item $\frac{2 \sigma}{R}$
        \item $\frac{\sigma}{4 \pi R}$
    \end{enumerate}
    \end{multicols}
\bigskip
    \item Let $M$, $B$, and $G$ represent respectively the metacentre, centre of buoyancy, and the centre of mass of a floating buoy. Which of the following statements is correct?
    \begin{multicols}{2}
    \begin{enumerate}
        \item $M$ is above $G$; Buoy unstable
        \item $B$ is above $G$; Buoy stable
        \item $M$ is above $G$; Buoy stable
        \item $B$ is above $G$; Buoy unstable
    \end{enumerate}
    \end{multicols}
\bigskip
    \item A reservoir connected to a pipeline is being filled with water, as shown in the Figure. At any time $t$, the free surface level in the reservoir is $h$. Find the time in seconds for the reservoir to get filled up to a height of 1 m, if the initial level is 0.2 m.
\begin{figure}[!ht]
    \centering
    \resizebox{1\textwidth}{!}{%
        \begin{circuitikz}
            % Set font size for all nodes
            \tikzstyle{every node}=[font=\large]

            % Drawing circuit elements
            \draw (9.25,18.75) to[short] (9.25,18.75); % Starting point (possibly not necessary)
            \draw (10.5,18.25) to[short] (10.5,15.5);  % Vertical line on left
            \draw (14.5,18.25) to[short] (14.5,15.5);  % Vertical line on right
            \draw (13,15.5) to[short] (14.5,15.5);     % Horizontal line connecting right section
            \draw (10.5,15.5) to[short] (12,15.5);     % Horizontal line from left to center
            \draw (12,15.5) to[short] (12,14.5);       % Vertical line down
            \draw (13,15.5) to[short] (13,14.5);       % Parallel vertical line down on right
            \draw (10,14.5) to[short] (12,14.5);       % Horizontal line connecting to left vertical
            \draw (13,14.5) to[short] (15,14.5);       % Horizontal line connecting to right vertical
            \draw (10,13.75) to[short] (15,13.75);     % Bottom horizontal line

            % Internal short connections
            \draw (12.5,12.75) to[short] (13.5,12.75); % Short connection in bottom section
            \draw (15.5,17) to[short] (17.5,17);       % Top connection on the right
            \draw (24.25,22) to[short] (26.25,22);     % Possibly separate element on top right
            \draw (10.5,17) to[short] (14.5,17);       % Top horizontal line connecting left and right
            \draw (15.5,15.5) to[short] (17.5,15.5);   % Middle right connection

            % Adding arrows
            \draw [->, >=Stealth] (9.5,14) .. controls (10.5,14) and (10,14) .. (10.5,14); % Curved arrow left
            \draw [->, >=Stealth] (13.5,12.75) -- (14,13.75);                              % Straight arrow upward
            \draw [->, >=Stealth] (14.75,14) .. controls (14,14) and (15.25,14) .. (15.5,14); % Curved arrow right
            \draw [->, >=Stealth] (16.5,16.5) .. controls (16.5,17) and (16.5,16.75) .. (16.5,17); % Curved arrow upward
            \draw [->, >=Stealth] (16.5,16) -- (16.5,15.5);                                 % Straight arrow downward
            \draw [->, >=Stealth] (12.5,14.5) -- (12.5,15.25);                              % Straight arrow upward in center

            % Adding labels
            \node [font=\LARGE] at (8.5,14) {2 m/s};                % Left label
            \node [font=\LARGE] at (16.5,14) {1 m/s};               % Right label
            \node [font=\large] at (16.5,16.25) {h};                % Label in middle right
            \node [font=\large] at (11.25,12.75) {Dia = 0.1 m};     % Diameter label bottom
            \node [font=\large] at (12.5,17.5) {Dia = 0.5 m};       % Diameter label top center
        \end{circuitikz}
    }%
\end{figure}
\newpage
    \item Bernoulli's equation is valid for the following type of flow:
    \begin{multicols}{2}
    \begin{enumerate}
        \item Compressible, steady, inviscid
        \item Incompressible, steady, viscous
        \item Compressible, unsteady, viscous
        \item Incompressible, steady, inviscid
    \end{enumerate}
    \end{multicols}
\bigskip
    \item If $A$ is the area of a circle of radius $r$ enclosing a plane forced vortex flow, with origin at the centre of the vortex and if $\omega$ is the angular velocity, $\zeta$ is the vorticity, $\vec{V}$ is the velocity vector, then the circulation around the contour of the circle is given by
    \begin{multicols}{4}
    \begin{enumerate}
        \item $2 \omega A$
        \item $2 \zeta A$
        \item $2 \vec{V} A$
        \item 0
    \end{enumerate}
    \end{multicols}

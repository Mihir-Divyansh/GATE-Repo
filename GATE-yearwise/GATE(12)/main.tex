\iffalse
\chapter{2024}
\section{ma}
\author{EE24BTECH11030}
\fi
    \item Consider the following statements.
\begin{enumerate}
    \item[I.] There exists a proper subgroup $G$ of $(\mathbb{Q}, +)$ such that $\mathbb{Q}/G$ is a finite group.
    \item[II.] There exists a subgroup $G$ of $(\mathbb{Q}, +)$ such that $\mathbb{Q}/G$ is isomorphic to $(\mathbb{Z}, +)$.
\end{enumerate}
Which one of the following is correct?
\begin{enumerate}
    \item Both I and II are TRUE
    \item I is TRUE and II is FALSE
    \item I is FALSE and II is TRUE
    \item Both I and II are FALSE
\end{enumerate}
\bigskip
\item Let $X$ be the space $\mathbb{R}/\mathbb{Z}$ with the quotient topology induced from the usual topology on $\mathbb{R}$. Consider the following statements.
\begin{enumerate}
    \item[I.] $X$ is compact.
    \item[II.] $X \setminus \{x\}$ is connected for any $x \in X$.
\end{enumerate}
Which one of the following is correct?
\begin{enumerate}
    \item Both I and II are TRUE
    \item I is TRUE and II is FALSE
    \item I is FALSE and II is TRUE
    \item Both I and II are FALSE
\end{enumerate}
\bigskip
\item Let $\langle \cdot, \cdot \rangle$ denote the standard inner product on $\mathbb{R}^7$. Let $\Sigma = \{v_1, \ldots, v_5\} \subset \mathbb{R}^7$ be a set of unit vectors such that $\langle v_i, v_j \rangle$ is a non-positive integer for all $1 \leq i \neq j \leq 5$. Define $N(\Sigma)$ to be the number of pairs $(r, s)$, $1 \leq r, s \leq 5$, such that $\langle v_r, v_s \rangle \neq 0$. The maximum possible value of $N(\Sigma)$ is equal to
\begin{enumerate}
    \item 9
    \item 10
    \item 14
    \item 5
\end{enumerate}
\bigskip
\item Let $f(x) = |x| + |x - 1| + |x - 2|, x \in [-1, 2]$. Which one of the following numerical integration rules gives the exact value of the integral
\[
\int_{-1}^2 f(x) \, dx
\]
\begin{enumerate}
    \item The Simpson's rule
    \item The trapezoidal rule
    \item The composite Simpson's rule by dividing $[-1, 2]$ into 4 equal subintervals
    \item The composite trapezoidal rule by dividing $[-1, 2]$ into 3 equal subintervals
\end{enumerate}

\item Consider the initial value problem (IVP)
\[
y' = e^{-y^2} + 1, \quad y(0) = 0.
\]
\begin{enumerate}
    \item[I.] IVP has a unique solution on $\mathbb{R}$.
    \item[II.] Every solution of IVP is bounded on its maximal interval of existence.
\end{enumerate}
Which one of the following is correct?
\begin{enumerate}
    \item Both I and II are TRUE
    \item I is TRUE and II is FALSE
    \item I is FALSE and II is TRUE
    \item Both I and II are FALSE
\end{enumerate}

\item Let $A$ be a $2 \times 2$ non-diagonalizable real matrix with a real eigenvalue $\lambda$ and $v$ be an eigenvector of $A$ corresponding to $\lambda$. Which one of the following is the general solution of the system $y' = Ay$ of first-order linear differential equations?
\begin{enumerate}
    \item $c_1 e^{\lambda t} + c_2 te^{\lambda t}$, where $c_1, c_2 \in \mathbb{R}$
    \item $c_1 e^{\lambda t} v + c_2 t e^{\lambda t} v$, where $c_1, c_2 \in \mathbb{R}$
    \item $c_1 e^{\lambda t} v + c_2 e^{\lambda t} (tv + u)$, where $c_1, c_2 \in \mathbb{R}$ and $u$ is a $2 \times 1$ real column vector such that $(A - \lambda I)u = v$
    \item $c_1 e^{\lambda t} v + c_2 te^{\lambda t} (v + u)$, where $c_1, c_2 \in \mathbb{R}$ and $u$ is a $2 \times 1$ real column vector such that $(A - \lambda I)u = v$
\end{enumerate}
\item Let $D = \{(x, y) \in \mathbb{R}^2 : x > 0 \text{ and } y > 0\}$. If the following second-order linear partial differential equation
\[
y^2 \frac{\partial^2 u}{\partial x^2} - x^2 \frac{\partial^2 u}{\partial y^2} + y \frac{\partial u}{\partial y} = 0 \text{ on } D
\]
is transformed to
\[
\left( \frac{\partial^2 u}{\partial \eta^2} - \frac{\partial^2 u}{\partial \xi^2} \right) + \left( a \frac{\partial u}{\partial \eta} + \frac{\partial u}{\partial \xi} \right) \frac{1}{2 \eta} + \left( a \frac{\partial u}{\partial \xi} + \frac{\partial u}{\partial \eta} \right) \frac{1}{2 \xi} = 0 \text{ on } D
\]
for some $a, b \in \mathbb{R}$, via the coordinate transform $\eta = \frac{x^2}{2}$ and $\xi = \frac{y^2}{2}$, then which one of the following is correct?
\begin{enumerate}
    \item $a = 2, b = 0$
    \item $a = 0, b = -1$
    \item $a = 1, b = -1$
    \item $a = 1, b = 0$
\end{enumerate}

\bigskip

\item Let $\ell^p = \left\{ x = (x_n)_{n \geq 1} : x_n \in \mathbb{R}, \|x\|_p = \left( \sum_{n=1}^\infty |x_n|^p \right)^{1/p} < \infty \right\}$ for $p = 1, 2$. Let
\[
c_{00} = \left\{ (x_n)_{n \geq 1} : x_n = 0 \text{ for all but finitely many } n \geq 1 \right\}.
\]
For $x = (x_n)_{n \geq 1} \in c_{00}$, define $f(x) = \sum_{n=1}^\infty \frac{x_n}{\sqrt{n}}$. Consider the following statements:
\begin{enumerate}
    \item[I.] There exists a continuous linear functional $F$ on $\ell^1, \|\cdot\|_1$ such that $F = f$ on $c_{00}$.
    \item[II.] There exists a continuous linear functional $G$ on $\ell^2, \|\cdot\|_2$ such that $G = f$ on $c_{00}$.
\end{enumerate}
Which one of the following is correct?
\begin{enumerate}
    \item Both I and II are TRUE
    \item I is TRUE and II is FALSE
    \item I is FALSE and II is TRUE
    \item Both I and II are FALSE
\end{enumerate}
    \item Let $\ell_2^Z = \left\{ (x_j)_{j \in \mathbb{Z}} : x_j \in \mathbb{R} \text{ and } \sum_{j=-\infty}^{\infty} x_j^2 < \infty \right\}$ endowed with the inner product
    \[
    \langle x, y \rangle = \sum_{j=-\infty}^{\infty} x_j y_j, \quad x = (x_j)_{j \in \mathbb{Z}}, \; y = (y_j)_{j \in \mathbb{Z}} \in \ell_2^Z.
    \]
    Let $T : \ell_2^Z \rightarrow \ell_2^Z$ be given by $T((x_j)_{j \in \mathbb{Z}}) = (y_j)_{j \in \mathbb{Z}}$, where
    \[
    y_j = \frac{x_j + x_{-j}}{2}, \quad j \in \mathbb{Z}.
    \]
    Which of the following is/are correct?
    \begin{enumerate}
        \item[(A)] $T$ is a compact operator
        \item[(B)] The operator norm of $T$ is 1
        \item[(C)] $T$ is a self-adjoint operator
        \item[(D)] $\text{Range}(T)$ is closed
    \end{enumerate}

    \bigskip

    \item Let $X$ be the normed space $(\mathbb{R}^2, \|\cdot\|)$, where
    \[
    \|(x, y)\| = |x| + |y|, \quad (x, y) \in \mathbb{R}^2.
    \]
    Let $S = \{(x, 0) : x \in \mathbb{R}\}$ and $f : S \rightarrow \mathbb{R}$ be given by $f((x, 0)) = 2x$ for all $x \in \mathbb{R}$. Recall that a Hahn-Banach extension of $f$ to $X$ is a continuous linear functional $F$ on $X$ such that $F|_S = f$ and $\|F\| = \|f\|$, where $\|F\|$ and $\|f\|$ are the norms of $F$ and $f$ on $X$ and $S$, respectively. Which of the following is/are true?
    \begin{enumerate}
        \item $F(x, y) = 2x + 3y$ is a Hahn-Banach extension of $f$ to $X$
        \item $F(x, y) = 2x + y$ is a Hahn-Banach extension of $f$ to $X$
        \item $f$ admits infinitely many Hahn-Banach extensions to $X$
        \item $f$ admits exactly two distinct Hahn-Banach extensions to $X$
    \end{enumerate}

    \bigskip

    \item Let $\{[a, b) : a, b \in \mathbb{R}, a < b\}$ be a basis for a topology $\tau$ on $\mathbb{R}$. Which of the following is/are correct?
    \begin{enumerate}
        \item[(A)] Every $(a, b)$ with $a < b$ is an open set in $(\mathbb{R}, \tau)$
        \item[(B)] Every $[a, b]$ with $a < b$ is a compact set in $(\mathbb{R}, \tau)$
        \item[(C)] $(\mathbb{R}, \tau)$ is a first-countable space
        \item[(D)] $(\mathbb{R}, \tau)$ is a second-countable space
    \end{enumerate}
    \bigskip
    \item Let $T, S : \mathbb{R}^4 \rightarrow \mathbb{R}^4$ be two non-zero, non-identity $\mathbb{R}$-linear transformations. Assume $T^2 = T$. Which of the following is/are TRUE?
    \begin{enumerate}
        \item[(A)] $T$ is necessarily invertible
        \item[(B)] $T$ and $S$ are similar if $S^2 = S$ and $\text{Rank}(T) = \text{Rank}(S)$
        \item[(C)] $T$ and $S$ are similar if $S$ has only 0 and 1 as eigenvalues
        \item[(D)] $T$ is necessarily diagonalizable
    \end{enumerate}

    \bigskip

    \item Let $p_1 < p_2$ be the two fixed points of the function $g(x) = e^x - 2$, where $x \in \mathbb{R}$. For $x_0 \in \mathbb{R}$, let the sequence $(x_n)_{n \geq 1}$ be generated by the fixed point iteration
    \[
    x_n = g(x_{n-1}), \quad n \geq 1.
    \]
    Which one of the following is/are correct?
    \begin{enumerate}
        \item[(A)] $(x_n)_{n \geq 0}$ converges to $p_1$ for any $x_0 \in (p_1, p_2)$
        \item[(B)] $(x_n)_{n \geq 0}$ converges to $p_2$ for any $x_0 \in (p_1, p_2)$
        \item[(C)] $(x_n)_{n \geq 0}$ converges to $p_2$ for any $x_0 > p_2$
        \item[(D)] $(x_n)_{n \geq 0}$ converges to $p_1$ for any $x_0 < p_1$
    \end{enumerate}


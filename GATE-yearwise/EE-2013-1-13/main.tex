\iffalse
\title{CE-2013-1-13}
\author{EE24BTECH11036 - Krishna Patil}
\section{ee}
\chapter{2013}
\fi
\item In the circuit shown below What is the output voltage \brak{V_{out}} in Volts if a silicon transistor $Q$ and an ideal op-amp are used ? \\
\begin{figure}[!ht]
\centering
\resizebox{0.6\textwidth}{!}{
\begin{circuitikz}
\tikzstyle{every node}=[font=\LARGE]
\draw (12.75,14.25) node[op amp,scale=1] (opamp2) {};
\draw (opamp2.+) to[short] (11.25,13.75);
\draw  (opamp2.-) to[short] (11.25,14.75);
\draw (13.95,14.25) to[short](14.25,14.25);
\node at (11.25,14.75) [circ] {};
\draw (9.25,14.75) to[R] (11.25,14.75);
\draw (11.25,13.75) to[short] (11.25,12.25);
\draw (11.25,12.25) to (11.25,11.75) node[ground]{};
\draw (9.25,14.75) to[american voltage source] (9.25,12.5);
\draw (9.25,12.5) to (9.25,11.75) node[ground]{};
\draw (11.25,14.75) to[short] (11.25,17.25);
\draw (11.25,17.25) to[short] (16.5,17.25);
\draw (16.5,14.25) to[Tnpn, transistors/scale=1.19] (16.5,17.25);
\draw (15.5,15.75) to (15.25,15.75) node[ground]{};
\draw (14.25,14.25) to[short] (16.25,14.25);
\draw (16.25,14.25) to[short, -o] (18.25,14.25) ;
\draw (12.5,14.75) to[short, -o] (12.5,15.75) ;
\draw (12.5,13.75) to[short, -o] (12.5,12.75) ;
\node [font=\LARGE] at (19,14.25) {$V_{out}$};
\node [font=\LARGE] at (16.75,15.75) {Q};
\node [font=\LARGE] at (12.5,16.25) {+15V};
\node [font=\LARGE] at (12.5,12.25) {-15V};
\node [font=\LARGE] at (10,15.25) {1k$\Omega$};
\node [font=\LARGE] at (8.25,13.5) {5V};
\end{circuitikz}
}
\label{fig:my_label}
\end{figure}
\begin{enumerate}
\begin{multicols}{2}
\item $ -15 $
\item $ -0.7 $ 
\item $ +0.7 $
\item $ +15 $
\end{multicols}
\end{enumerate}
\item The transfer function $\frac{V_2 \brak{s}}{V_1 \brak{s}}$ of the circuit shown below is 
\begin{figure}[!ht]
\centering
\resizebox{0.6\textwidth}{!}{
\begin{circuitikz}
\tikzstyle{every node}=[font=\large]
\draw (7.5,17.5) to[curved capacitor] (9,17.5);
\draw (7.5,17.5) to[short, -o] (6.25,17.5) ;
\draw (9,17.5) to[short] (10.75,17.5);
\draw (10.75,17.5) to[R] (10.75,15.75);
\draw (10.75,15.75) to[curved capacitor] (10.75,14.5);
\draw (7,14.5) to[short, -o] (12.25,14.5) ;
\draw (7,14.5) to[short, -o] (6.25,14.5) ;
\draw (10.75,17.5) to[short, -o] (12.25,17.5) ;
\node [font=\normalsize] at (8.40,18.1) {100${\mu}F$};
\node [font=\normalsize] at (9.75,15) {100${\mu}F$};
\node [font=\normalsize] at (10,16.5) {10k$\Omega$};
\node [font=\normalsize] at (12.25,16) {$V_2\brak{s}$};
\node [font=\normalsize] at (6.25,16) {$V_1\brak{s}$};
\node [font=\large] at (6.25,17.25) {+};
\node [font=\large] at (12.25,17.25) {+};
\draw (12.25,14.75) to[short] (12.5,14.75);
\draw (12,14.75) to[short] (12.5,14.75);
\draw (6,14.75) to[short] (6.5,14.75);
\end{circuitikz}
}
\label{fig:my_label}
\end{figure}
\begin{enumerate}
\begin{multicols}{4}
\item $ \frac{0.5s+1}{s+1} $
\item $ \frac{0.6s+1}{s+2} $ 
\item $ \frac{s+2}{s+1} $
\item $ \frac{s+1}{s+2} $
\end{multicols}
\end{enumerate}
\newpage
\item Assuming zero initial condition , the response $y\brak{t}$ of the system given below to a unit step input $u\brak{t}$ is 
\begin{figure}[!ht]
\centering
\resizebox{0.6\textwidth}{!}{
\begin{circuitikz}
\tikzstyle{every node}=[font=\normalsize]
\draw  (8.5,15.75) rectangle (12,14.25);
\node [font=\normalsize] at (10.25,15) {$\frac{1}{s}$};
\draw [->, >=Stealth] (7,15) -- (8.5,15);
\draw [->, >=Stealth] (12,15) -- (13.5,15);
\node [font=\normalsize] at (7.25,15.25) {$U\brak{s}$};
\node [font=\normalsize] at (13.25,15.25) {$Y\brak{s}$};
\end{circuitikz}
}
\label{fig:my_label}
\end{figure}
\begin{enumerate}
\begin{multicols}{4}
\item $ u\brak{t} $
\item $ tu\brak{t} $ 
\item $ \frac{t^2}{2}u\brak{t} $
\item $ e^{-t}u\brak{t}$
\end{multicols}
\end{enumerate}
\item The impulse response of a system is $h\brak{t}=tu\brak{t}$ .For an input $u\brak{t-1}$ , the output is 
\begin{enumerate}
\begin{multicols}{4}
\item $ \frac{t^2}{2}u\brak{t} $
\item $ \frac{t\brak{t-1}}{2}u\brak{t-1} $
\item $ \frac{(t-1)^2}{2}u\brak{t-1} $
\item $ \frac{t^2 - 1}{2}u\brak{t-1} $
\end{multicols}
\end{enumerate}
\item Which one of the following statements is NOT TRUE for a continous time casual and stable LTI system ? 
\begin{enumerate}
\item All the poles of the system must lie on the left side of the $j\omega$ axis.
\item  Zeroes of the system can lie anywhere in the $s$ plane.
\item All the poles must lie within $\abs{s}=1$.
\item All the roots of the characteristic equation must be located on the left side of the $j\omega$ axis.
\end{enumerate}
\item Two systems with impulse reponses of $h_1\brak{t}$ and $h_2\brak{t}$ are connected in cascade . Then the overall impulse response of the cascaded system is given by
\begin{enumerate}
\item product of $h_1\brak{t}$ and $h_2\brak{t}$  
\item sum of $h_1\brak{t}$ and $h_2\brak{t}$
\item convolution of $h_1\brak{t}$ and $h_2\brak{t}$ 
\item subtraction of $h_2\brak{t}$ and $h_1\brak{t}$ 
\end{enumerate}
\item A source of $V_s\brak{t}=V\cos{100\pi t}$ has a internal impedance of $\brak{4+3j}\Omega$.If a purely resistive load connected to this source has to extract the maximum power out of the source , its value in $\Omega$ should be 
\begin{enumerate}
\begin{multicols}{4}
\item $3$ 
\item $4$
\item $5$ 
\item $7$
\end{multicols}
\end{enumerate}
\item A single-phase load is supplied by a single-phase voltage source . If the current flowing from the load to the source $10\angle -150^\circ $A and if the volatage at the load terminal is  $100\angle 60^\circ$V , then the
\begin{enumerate}
\item load absorbs real power and delivers reactive power. 
\item load absorbs real power and absorbs reactive power.
\item load delivers real power and delivers reactive power.
\item load delivers real power and absorbs reactive power.
\end{enumerate}
\newpage
\item A single-phase transformer has no-load loss of $64W$ as obtained from an open-circuit test.When a short-circuit test is performed on it with $90\%$ of the rated currents flowing in its both $LV$ and $HV$ windings , the measured loss is $81W$ . The transformer  has maximum efficiency when operated at 
\begin{enumerate}
\item $ 50.0\% $ of the rated current.
\item $ 64.0\% $ of the rated current.
\item $ 80.0\% $ of the rated current.
\item $ 88.8\% $ of the rated current.
\end{enumerate}
\item The flux density at a point in space is given by $B = 4xa_x + 2k ya_y + 8a_z\text{Wb/m}^2$. The value of constant $k$ must be equal to
\begin{enumerate}
\begin{multicols}{4}
\item $-2$
\item $-0.5$
\item $+0.5$
\item $+2$
\end{multicols}
\end{enumerate}
\item A continuous random variable $X$ has a probability density function $f\brak{x} = e^{-x}, \; 0 < x < \infty$. Then $P\cbrak{ X > 1 }$ is
\begin{enumerate}
\begin{multicols}{4}
\item $0.368$
\item $0.5$
\item $0.632$
\item $1.0$ 
\end{multicols}
\end{enumerate} 
\item The curl of the gradient of the scalar field defined by $V = 2x^2 y + 3y^2 z + 4z^2 x$ is
\begin{enumerate}
\item $4xy \, a_x + 6yz \, a_y + 8zx \, a_z$
\item $4 \, a_x + 6 \, a_y + 8 \, a_z$
\item $\brak{4xy + 4z^2} a_x + \brak{2x^2 + 6yz} a_y + \brak{3y^2 + 8zx} a_z$
\item $0$
\end{enumerate}
\item In the feedback network shown below, if the feedback factor $k$ is increased, then the
\begin{figure}[!ht]
\centering
\resizebox{0.6\textwidth}{!}{%
\begin{circuitikz}
\tikzstyle{every node}=[font=\normalsize]
\draw  (8.75,17) rectangle (11.75,15.25);
\draw  (8.75,13.25) rectangle (11.75,11.5);
\draw (8.75,16.75) to[short, -o] (5.5,16.75) ;
\draw (6,15.75) to[short, -o] (5.5,15.75) ;
\draw (6,15.75) to[short] (6,11.75);
\draw (6,11.75) to[short] (8.75,11.75);
\draw (8.75,15.75) to[short] (8.25,15.75);
\draw (8.25,15.75) to[short] (8.25,13);
\draw (8.25,13) to[short] (8.75,13);
\draw (11.75,16.75) to[short, -o] (13.75,16.75) ;
\draw (11.75,15.75) to[short, -o] (13.75,15.75) ;
\draw (11.75,13) to[short] (12.5,13);
\draw (12.5,13) to[short] (12.5,16.75);
\draw (13,15.75) to[short] (13,11.75);
\draw (11.75,11.75) to[short] (13,11.75);
\node [font=\normalsize] at (13.75,16.25) {$V_{out}$};
\node [font=\normalsize] at (6.25,16.25) {$V_{in}$};
\node [font=\normalsize] at (5.5,16.5) {+};
\node [font=\normalsize] at (8.5,16.5) {+};
\node [font=\normalsize] at (12,12.75) {+};
\node [font=\normalsize] at (8.5,12.75) {+};
\node [font=\normalsize] at (12,16.5) {+};
\node [font=\LARGE] at (5.5,16) {-};
\node [font=\LARGE] at (8.5,12) {-};
\node [font=\LARGE] at (12,12) {-};
\node [font=\LARGE] at (8.5,16) {-};
\node [font=\LARGE] at (12,16) {-};
\node [font=\normalsize] at (8,16.25) {$V_1$};
\node [font=\normalsize] at (7.5,12.25) {$V_f = kV_{out}$};
\node [font=\normalsize] at (10.25,16.25) {$A_0$};
\node [font=\normalsize] at (10,12.5) {$k$};
\end{circuitikz}
}%
\label{fig:my_label}
\end{figure}
\begin{enumerate}
\item[(A)] input impedance increases and output impedance decreases.
\item[(B)] input impedance increases and output impedance also increases.
\item[(C)] input impedance decreases and output impedance also decreases.
\item[(D)] input impedance decreases and output impedance increases.
\end{enumerate}    

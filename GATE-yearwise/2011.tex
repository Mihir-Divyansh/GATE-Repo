\iffalse
  \title{GateAssignment-2}
  \author{EE24BTECH11048-NITHIN.K} 
  \section{ae}
  \chapter{2011}
\fi
%\begin{enumerate}

%1
\item Consider x, y, z to be right-handed Cartesian coordinates. A vector function is defined in this coordinate system as $\vec{v} = 3x\vec{i} + 3xy\vec{j} - yz^2\vec{k}$, where $\vec{i}, \vec{j} \text{and} \vec{k}$ are the unit vectors along x, y and z axes, respectively. The curl of $\vec{v}$ is given by
	\begin{enumerate}
		\item $z^2\vec{i} - 3y\vec{k}$
		\item $z^2\vec{j} + 3y\vec{k}$
		\item $z^2\vec{i} + 3y\vec{j}$
		\item $-z^2\vec{i} + 3y\vec{k}$
	\end{enumerate}
%2
\item Which of the following functions is periodic?
	\begin{enumerate}
		\item $f\brak{x} = x^2$
		\item $f\brak{x} = \log{x}$
		\item $f\brak{x} = e^x$ 
		\item $f\brak{x}$ = const.
	\end{enumerate}
%3
\item The function $f\brak{x_1, x_2, x_3} = x_1^2 + x_2^2 + x_3^2 - 2x_1 - 4x_2 - 6x_3 + 14$ has its minimum value at
	\begin{enumerate} 
		\item $\brak{1,2,3}$
		\item $\brak{0,0,0}$
		\item $\brak{3,2,1}$       
		\item $\brak{1,1,3}$     
	\end{enumerate}
%4
\item Consider the function $f\brak{x_1, x_2} = x_1^2 + 2x_2^2 + e^{-x_1 - x_2}$. The vector pointing in the direction of maximum increase of the function at the point $\brak{1, -1}$ is
	\begin{enumerate}
		\item $\myvec = {2 \\
			-5}$
		\item $\myvec = {1 \\
			-5}$
		\item $\myvec = {-0.73 \\
			-6.73}$
		\item $\myvec = {2 \\
			-4}$
	\end{enumerate}
%5
\item Two simultaneous equations given by $y = \pi + x$ and $y = x - \pi$ have
	\begin{enumerate}
		\item a unique solution
		\item infinitely many solutions
		\item no solution
		\item a finite number of multiple solutions
	\end{enumerate}
%6
\item In three-dimensional linear elastic solids, the number of non-trivial stress-strain relations, strain-displacement equations and equations of equilibrium are, respectively
	\begin{enumerate}
		\item 3, 3 and 3
		\item 6, 3 and 3
		\item 6, 6 and 3
		\item 6, 3 and 6
	\end{enumerate}
%7
\item An Euler-Bernoulli beam in bending is assumed to satisfy
	\begin{enumerate}
		\item both plane stress as well as plane strain conditions
		\item plane strain condition but not plane stress condition
		\item plane stress condition but not plane strain condition
		\item neither plane strain condition nor plane stress condition
	\end{enumerate}
%8
\item A statically indeterminate frame structure has
	\begin{enumerate}
		\item same number of joint degrees of freedom as the number of equilibrium equations
		\item number of joint degrees of freedom greater than the number of equilibrium equations
		\item number of joint degrees of freedom less than the number of equilibrium equations
		\item unknown number of joint degrees of freedom, which cannot be solved using laws of mechanics
	\end{enumerate}
%9
\item Consider a single degree of freedom spring-mass-damper system with mass, damping and stiffness of m, c and k, respectively. The logarithmic decrement of this system can be calculated using
	\begin{enumerate}
		\item $\frac{2\pi c}{\sqrt{4mk - c^2}}$
		\item $\frac{\pi c}{\sqrt{4mk - c^2}}$
		\item $\frac{2\pi c}{\sqrt{mk - c^2}}$
		\item $\frac{2\pi c}{\sqrt{mk - 4c^2}}$
	\end{enumerate}
%10
\item Consider a single degree of freedom spring-mass system of spring stiffness $k_1$ and mass m which has a natural frequency of 10 rad/s. Consider another single degree of freedom spring-mass system of spring stiffness $k_2$ and mass m which has a natural frequency of 20 rad/s. The spring stiffness $k_2$ is equal to
	\begin{enumerate}
		\item $k_1$
		\item $2k_1$
		\item $\frac{k_1}{4}$
		\item $4k_1$
	\end{enumerate}
%11
\item Consider a simply supported two-dimensional beam
	\begin{figure}[H]
		\centering
		\resizebox{0.5\textwidth}{!}{%
			\begin{circuitikz}
				\tikzstyle{every node}=[font=\LARGE]
				\draw [short] (6.25,10.5) -- (7,12);
				\draw [short] (7,12) -- (7.75,10.5);
				\draw [short] (6.25,10.5) -- (7.75,10.5);
				\draw [short] (7,12) -- (11.5,12);
				\draw [short] (11.5,12) -- (10.75,10.75);
				\draw [short] (11.5,12) -- (12.25,10.75);
				\draw [short] (10.75,10.75) -- (12.25,10.75);
				\draw  (11,10.5) circle (0.25cm);
				\draw  (11.75,10.75) circle (0cm);
				\draw  (12,10.5) circle (0.25cm);
				\draw [short] (10.5,10.25) -- (12.75,10.25);
			\end{circuitikz}
			}%
	\end{figure}
	If the beam is converted into a fixed-fixed beam as
	\begin{figure}[H]
		\centering
		\resizebox{0.5\textwidth}{!}{%
			\begin{circuitikz}
				\tikzstyle{every node}=[font=\LARGE]
				\draw [line width=0.6pt, short] (7,13.5) -- (7,12.5);
				\draw [line width=0.6pt, short] (11,13.5) -- (11,12.5);
				\draw [line width=0.6pt, short] (6.5,13) -- (7,13.5);
				\draw [line width=0.6pt, short] (6.5,12.5) -- (7,13);
				\draw [line width=0.6pt, short] (7,12.5) -- (6.5,12);
				\draw [line width=0.6pt, short] (11,13.5) -- (11.5,14);
				\draw [line width=0.6pt, short] (11,13) -- (11.5,13.5);
				\draw [line width=0.6pt, short] (11,12.5) -- (11.5,13);
				\draw [ line width=0.6pt](7,13) to[short] (11,13);
			\end{circuitikz}
			}%
	\end{figure}
	then the degree of static indeterminacy will
	\begin{enumerate}
		\item increase by 3
		\item increase by 2
		\item decrease by 1
		\item decrease by 3
	\end{enumerate}
%12
\item An impulsive launch of a rocket minimizes the loss of burn-out velocity due to
	\begin{enumerate}
		\item aerodynamic drag force only
		\item gravitational force only
		\item both aerodynamic drag and gravitational forces
                \item reaction jet control force
        \end{enumerate}
%13
\item Multi-staging in rockets improves the burn-out performance by increasing mainly stage-wise
	\begin{enumerate}
                \item payload mass ratios
                \item structural mass efficiencies
                \item propellant masses
                \item control system masses
        \end{enumerate}
%\end{enumerate}

\iffalse
	\chapter{2008}
	\author{AI24BTECH11003}
	\section{ae}
\fi
%1
    \item The function defined by
    $
    f(x)=
    \begin{cases}
        \sin x, & x<0 \\
        0, & x=0\\
        3x^3, & x>0
    \end{cases}$
    
    \hfill{(2008)}

        \begin{enumerate}
            \item Is neither continuous nor differentiable at $x=0$
            \item Is continuous and differentiable at $x=0$
            \item Is differentiable but not continuous at $x=0$
            \item Is continuous but not differentiable at $x=0$
        \end{enumerate}

%2
    \item The product of the eigenvalues of the matrix $\myvec{1 & 0 & 1 \\ 0 & 2 & 1 \\ 1 & 1 & -3}$ is
    
    \hfill{(2008)}
    
        \begin{multicols}{4}
            \begin{enumerate}
                \item $4$
                \item $0$
                \item $-6$
                \item $-9$
            \end{enumerate}
        \end{multicols}

%3
    \item Which of the following equations is a LINEAR ordinary differential equation?
    
    \hfill{(2008)}

        \begin{multicols}{2}
            \begin{enumerate}
                \item $\frac{d^2y}{dx^2}+\frac{dy}{dx}+2y^2=0$
                \item $\frac{d^2y}{dx^2}+y\frac{dy}{dx}+2y=0$
                \item $\frac{d^2y}{dx^2}+x\frac{dy}{dx}+2y=0$
                \item $\brak{\frac{dy}{dx}}^2+\frac{dy}{dx}+2y=0$
            \end{enumerate}
        \end{multicols}

%4
    \item To transfer a satellite from an elliptical orbit to a circular orbit having radius equal to the apogee distance of the elliptical orbit, the speed of the satellite should be
    
    \hfill{(2008)}

		\begin{multicols}{2}
			\begin{enumerate}
				\item increased at the apogee
				\item decreased at the apogee
				\item increased at the perigee
				\item decreased at the perigee
			\end{enumerate}
		\end{multicols}

%5
    \item The service ceiling of a transport aircraft is defined as the altitude
    
    \hfill{(2008)}

       \begin{enumerate}
            \item That is halfway between sea level and absolute ceiling
            \item at which it can cruise at one engine operational
            \item at which it's maximum rate of climb is zero
            \item at which it's maximum rate of climb is $0.508\frac{m}{s}$
        \end{enumerate}
  
%6
    \item The drag of an aircraft in steady climbing flight at a given forward speed is 
    
    \hfill{(2008)}

        \begin{enumerate}
            \item inversely proportional to climb angle
            \item higher than drag in steady level flight at the same forward speed
            \item lower than drag in steady level flight at the same forward speed
            \item independent of climb angle
        \end{enumerate}

%7
    \item In steady, level turning flight of an aircraft at a load factor $'n'$, the ratio of horizontal component of lift and aircraft weight is
    
    \hfill{(2008)}

        \begin{multicols}{4}
            \begin{enumerate}
                \item $\sqrt{n-1}$
                \item $\sqrt{n+1}$
                \item $\sqrt{n^2-1}$
                \item $\sqrt{n^2+1}$
            \end{enumerate}
        \end{multicols}
		
%8
    \item The parameters that remain constant in a cruise climb of an aircraft are 
    
    \hfill{(2008)}

        \begin{multicols}{2}
            \begin{enumerate}
                \item equivalent airspeed and lift coefficient
                \item altitude and lift coefficient
                \item equivalent airspeed and altitude
                \item lift coefficient and aircraft mass
            \end{enumerate}
        \end{multicols}

%9
    \item The maximum thickness to chord ratio for the NACA 24012 airfoil is
    
    \hfill{(2008)}

        \begin{multicols}{4}
            \begin{enumerate}
                \item $0.01$
                \item $0.12$
                \item $0.24$
                \item $0.40$
            \end{enumerate}
        \end{multicols}

%10
    \item The maximum possible value of pressure coefficient $C_p$ in incompressible flow is
    
    \hfill{(2008)}

        \begin{multicols}{4}
            \begin{enumerate}
                \item $0.5$
                \item $1$
                \item $\pi$
                \item $\infty$
            \end{enumerate}
        \end{multicols}
        
%11
    \item An irrotational and inviscid flow can become rotational on passing through a
    
    \hfill{(2008)}

        \begin{multicols}{2}
            \begin{enumerate}
                \item normal shock wave
                \item oblique shock wave
                \item curved shock wave
                \item Mach wave
            \end{enumerate}
        \end{multicols}

%12
    \item Laminar flow airfoils are used to reduce 
    
    \hfill{(2008)}

        \begin{multicols}{4}
            \begin{enumerate}
                \item trim drag
                \item skin friction drag
                \item induced drag
                \item wave drag
            \end{enumerate}
        \end{multicols}
        
%13
    \item The degree of reaction of an impulse turbine is 
    
    \hfill{(2008)}

        \begin{multicols}{4}
            \begin{enumerate}
                \item $2$
                \item $0.75$
                \item $0.5$
                \item $0$
            \end{enumerate}
        \end{multicols}

%14
    \item In a convergent-divergent \brak{CD} nozzle of a rocket motor, the wall heat flux is maximum at
    
    \hfill{(2008)}

        \begin{enumerate}
            \item the exit of the divergent portion of the CD nozzle
            \item the entry to the convergent portion of the CD nozzle
            \item the throat of the CD nozzle
            \item the mid-length of the divergent portion of the CD nozzle
        \end{enumerate}

%15
    \item In a scramjet engine, the Mach number at the entry to the combustion chamber is around
    
    \hfill{(2008)}

        \begin{multicols}{4}
            \begin{enumerate}
                \item $0$
                \item $0.3$
                \item $2$
                \item $6$
            \end{enumerate}
        \end{multicols}

%16
    \item DB denotes double base solid propellant.\\
    LOX-RPI denotes liquid oxygen - kerosene combination.\\
    LOX-LH$_2$ denotes liquid oxygen - hydrogen combination.\\

    The correct order of increasing specific impulse is 
    
    \hfill{(2008)}

        \begin{multicols}{2}
            \begin{enumerate}
                \item DB$<$LOX-RPI$<$LOX-LH$_2$
                \item LOX-RPI$<$DB$<$LOX-LH$_2$
                \item LOX-LH$_2<$DB$<$LOX-RPI
                \item DB$<$LOX-LH$_2<$LOX-RPI
            \end{enumerate}
        \end{multicols}

%17
    \item In the absence of body movements, the symmetry of the stress tensor is derived from
    
    \hfill{(2008)}

        \begin{enumerate}
            \item force equilibrium conditions
            \item movement equilibrium conditions
            \item linear relations between stresses and strains
            \item compatibility conditions
        \end{enumerate}


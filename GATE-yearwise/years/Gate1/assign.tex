\iffalse

\chapter{2009}
\section{ph}
\author{AI24BTECH11023 - Tarun Reddy Pakala}
\fi
    \item The value of the contour integral, $\abs{\int_{C} \overrightarrow{r} \times d\overrightarrow{\theta} \, }$, for a circle $C$ of radius $r$ with center at origin is
    \begin{enumerate}
    \item $2\pi r$
    \item $r^2/2$
    \item $\pi r^2$
    \item $r$
    \end{enumerate}
    \item An electrostatic field $\overrightarrow{E}$ exists in a given region R. Choose the WRONG statement.
    \begin{enumerate}
    \item Circulation of $\overrightarrow{E}$ is zero
    \item $\overrightarrow{E}$ can always be expressed as the gradient of a scalar field
    \item The potential difference between any two arbitrary points in the region R is zero
    \item The work  done in a closed path lying entirely in R is zero
    \end{enumerate}
    \item The Lagrangian of a free particle in spherical polar co-ordinates is given by $L = \frac{1}{2} m \left( \dot{r}^2 + r^2 \dot{\theta}^2 + r^2 \dot{\phi}^2 \sin^2 \theta \right)$. The quantity that is conserved is
    \begin{enumerate}
    \item $\frac{\partial L}{\partial \dot{r}}$
    \item $\frac{\partial L}{\partial \dot{\theta}}$
    \item $\frac{\partial L}{\partial \dot{\phi}}$
    \item $\frac{\partial L}{\partial \dot{\phi}} + \dot{r} \dot{\theta}$
    \end{enumerate}
    \item A conducting loop $L$ of surface area $S$ is moving with a velocity $\overrightarrow{v}$ in a magnetic field $\overrightarrow{B}(\overrightarrow{r},t)$=$\overrightarrow{B_0}t^2$, $B_o$ is a positive constant of suitable dimensions. The emf induced, $V_{emf}$, in the loop is given by
    \begin{enumerate}
        \item -$\int_S \frac{\partial \overrightarrow{B}}{\partial t} \cdot d\overrightarrow{S}$
        \item $\oint_L (\overrightarrow{v} \times \overrightarrow{B}) \cdot d\overrightarrow{L}$
        \item -$\int_S \frac{\partial \overrightarrow{B}}{\partial t} \cdot d\overrightarrow{S}$-$\oint_L (\overrightarrow{v} \times \overrightarrow{B}) \cdot d\overrightarrow{L}$
        \item -$\int_S \frac{\partial \overrightarrow{B}}{\partial t} \cdot d\overrightarrow{S}$+$\oint_L (\overrightarrow{v} \times \overrightarrow{B}) \cdot d\overrightarrow{L}$
    \end{enumerate}
    \item The eigenvalues of the matrix $A= 
\begin{bmatrix}
0 & i \\
i & 0
\end{bmatrix}
$ are
    \begin{enumerate}
    \item real and distinct
    \item complex and distinct
    \item complex and coinciding
    \item real and coinciding
    \end{enumerate}
    \item $\sigma_i(i=1, 2, 3)$ represent the Pauli spin matrices. Which one of the following is NOT true ?
    \begin{enumerate}
        \item $\sigma_i \sigma_j$+$\sigma_j \sigma_i$=2$\delta_{ij}$
        \item $Tr(\sigma_i)=0$
        \item The eigenvalues of $\sigma_i$ are $\pm1$
        \item det($\sigma_i$)=1
    \end{enumerate}
    \item Which one of the functions given below represents the bound state eigenfunction of the operator $ - \frac{d^2}{dx^2} $ in the region, $0 \leq x < \infty$ with the eigenvalue -4 ? 
    \begin{enumerate}
    \item $A_o e^{2x}$
    \item $A_o \cosh{2x}$
    \item $A_o e^{-2x}$
    \item $A_o \sinh{2x}$
    \end{enumerate}
    \item Pick the WRONG statement.
    \begin{enumerate}
        \item The nuclear force is independent of electric charge
        \item The Yukawa potential is proportional to $r^{-1} \exp\left(\frac{mc}{\hbar} r\right)$, where $r$ is the separation between two nucleons
        \item The range of nuclear force is order of $10^{-15}m-10^{-14}m$
        \item The nucleons interact among each other by the exchange of mesons
    \end{enumerate}
    \item If $p$ and $q$ are the position and momentum variables, which one of the following is NOT a canonical transformation ?
    \begin{enumerate}
        \item $Q=\alpha q$ and $P= \frac{1}{\alpha}p$, for $\alpha \neq 0$
        \item $Q=\alpha q+\beta p$ and $P=\beta q+\alpha p$ for $\alpha,\beta$ real and $\alpha^2-\beta^2=1$
        \item $Q=p$ and $P=q$
        \item $Q=p$ and $P=-q$
    \end{enumerate}
    \item The Common Mode Rejection Ratio (CMRR) of a differential amplifier using an operational amplifier is 100 dB. The output voltage for a differential input of 200 $\mu$V is 2 V. The common mode gain is
    \begin{enumerate}
        \item 10
        \item 0.1
        \item 30 dB
        \item 10 dB
    \end{enumerate}
    \item In an insulating solid which one of the following physical phenomena is a consequence of Pauli's exclusion principle ?
    \begin{enumerate}
        \item Ionic conductivity
        \item Ferromagnetism
        \item Paramagnetism
        \item Ferroelectricity
    \end{enumerate}
    \item Which one of the following curves gives the solution of the differential equation $k_1 \frac{dx}{dt} + k_2 x = k_3$, where $k_1,k_2$ and $k_3$ are positive constants with initial conditions $x=0$ at $t=0$ ?
    \begin{enumerate}
	    \item 
\begin{circuitikz}
\tikzstyle{every node}=[font=\small]
\draw [->, >=Stealth] (2,11.75) -- (2,16.5);
\draw [->, >=Stealth] (1,12.75) -- (8.5,12.75);
\draw [short] (2,14.5) -- (5.25,14.5);
\draw [short] (5.25,14.5) .. controls (6.75,14) and (5,13) .. (7.5,12.75);
\draw [dashed] (5.75,15) -- (5.75,12.75);
\node [font=\small] at (1.5,16.75) {P(E)};
\node [font=\small] at (5.75,12.5) {$E_F$};
\node [font=\small] at (8.5,12.5) {E};
\node [font=\small] at (1.75,14.5) {1};
\end{circuitikz}



        \item \resizebox{0.3\textwidth}{!}{%
\begin{circuitikz}
\tikzstyle{every node}=[font=\Large]
\draw [line width=0.5pt, ->, >=Stealth] (7.5,14.25) -- (14.25,14.25);
\draw [short] (11,16.5) -- (11,13.25);
\foreach \x in {0,...,4}{
  \draw [ line width=0.5pt] (8.75+\x*1,14.25) -- ++(0,1) -- ++ (0.5, 0) -- ++(0, -1) -- ++(0.5,0);
}
\node [font=\Large] at (10.5,13.75) {0};
\node [font=\Large] at (10.5,16.25) {f(t)};
\node [font=\Large] at (14.5,14.25) {t};
\end{circuitikz}
}%


        \item \begin{circuitikz}
    % Grid
    \draw[very thin, gray] (-2, -1.5) grid (4, 2.5);
    
    % Plot the curve (example curve for approximation)
    \draw[thick] plot[smooth, tension=1] coordinates {(-1.8,-1) (-1, 0.5) (0, 1.5) (1, 0.5) (2, 1.2) (3, 0.5) (4, 1.4)};
    
\end{circuitikz}

        \item \resizebox{0.3\textwidth}{!}{%
\begin{circuitikz}
\tikzstyle{every node}=[font=\Large]
\draw  (6,13.75) circle (0cm);
\draw  (6.75,14.25) circle (0.25cm);
\draw  (6.75,14.25) circle (0.5cm);
\draw [short] (6.25,14.25) -- (6.25,13.25);
\draw [short] (7.25,14.25) -- (7.25,13.25);
\draw [short] (6.25,13.25) -- (7.25,13.25);
\draw [short] (6.25,13.25) -- (6,13);
\draw [short] (6.75,13.25) -- (6.5,13);
\draw [short] (7.25,13.25) -- (7,13);
\draw [ line width=0.6pt ] (13.25,14) circle (0cm);
\draw [ line width=0.6pt ] (14,14.5) circle (0.25cm);
\draw [ line width=0.6pt ] (14,14.5) circle (0.5cm);
\draw [line width=0.6pt, short] (13.5,14.5) -- (13.5,13.5);
\draw [line width=0.6pt, short] (14.5,14.5) -- (14.5,13.5);
\draw [line width=0.6pt, short] (13.5,13.5) -- (14.5,13.5);
\draw [line width=0.6pt, short] (13.5,13.5) -- (13.25,13.25);
\draw [line width=0.6pt, short] (14,13.5) -- (13.75,13.25);
\draw [line width=0.6pt, short] (14.5,13.5) -- (14.25,13.25);
\draw [short] (6.75,14.25) -- (9.75,17.75);
\draw [short] (10.25,17.75) -- (14,14.5);
\draw [ rotate around={138:(10, 18)}] (9.5,18.25) rectangle (10.5,17.75);
\draw  (10,18) circle (0.25cm);
\draw [short] (8.5,19.5) -- (9.75,18.25);
\draw [short] (7.25,14.25) -- (8.75,14.25);
\draw [short] (7.5,15.25) .. controls (8,15) and (8.25,14.5) .. (8,14.25);
\node [font=\Large] at (8.25,15) {$60\degree$};
\node [font=\Large] at (5.5,14.5) {A};
\node [font=\Large] at (9.25,17.75) {B};
\node [font=\Large] at (14.75,15) {C};
\node [font=\Large] at (9.25,19.5) {D};
\node [font=\Large] at (13.5,19.25) {AB=250};
\node [font=\Large] at (13.5,18.5) {BC=250$\sqrt{3}$};
\node [font=\Large] at (13.5,17.5) {AC=500};
\end{circuitikz}
}%


    \end{enumerate}


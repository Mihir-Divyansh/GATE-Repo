
\iffalse
\author{Golla Shriram - AI24BTech11010}
\section{ce}
\chapter{2009 }
\fi
%\maketitle
%\newpage
%\bigskip

%\renewcommand{\thefigure}{\theenumi}
%\renewcommand{\thetable}{\theenumi}



		
	 \item A square matrix $\vec{B}$ is skew-symmetric if\hfill{(2009-CE)}

 \begin{enumerate}
    \begin{multicols}{4} 
  \item $\vec{B}^{\top} = - \vec{B}$
  \item $\vec{B}^{\top} = \vec{B}$
  \item $\vec{B}^{-1} =  \vec{B}$
  \item $\vec{B}^{-1} =  \vec{B}^{T}$
      \end{multicols}
 \end{enumerate}

\item For a scalar function $f(x,y,z) = x^2+3y^2+3z^2$, the gradient at the point P(1, 2, -1) is\hfill{(2009-CE)}
\begin{enumerate}
   \begin{multicols}{4} 
	\item $2\overrightarrow{i}+6\overrightarrow{j}+4\overrightarrow{k} $
	\item $2\overrightarrow{i}+6\overrightarrow{j}-4\overrightarrow{k} $
	\item $2\overrightarrow{i}+12\overrightarrow{j}+4\overrightarrow{k} $
	\item $\sqrt{56}$
	    \end{multicols}
\end{enumerate}

\item The analytic function $f(z) = \frac{z-1}{z^{2}+1}$ has singularities at \hfill{(2009-CE)}
	\begin{enumerate}
	   \begin{multicols}{4} 
	\item 1 and $-1$
	\item 1 and $i$
	\item 1 and $-i$
	\item $i$ and $-i$
	    \end{multicols}
	\end{enumerate}


\item A thin walled cylindrical pressure vessel having a radius of 0.5 m and wall thickness of 25 mm is subjected to an internal pressure of 700 kPa. The hoop stress developed is \hfill{(2009-CE)}
	\begin{enumerate}
	   \begin{multicols}{4} 
		\item 14 MPa
		\item 1.4 MPa
		\item 0.14 MPa
		\item 0.014 MPa
		    \end{multicols}
	\end{enumerate}		

\item The modulus of rupture of concrete in terms of its characteristic cube compressive strength ($f_{ck}$) in MPa according to IS 456:2000 is\hfill{(2009-CE)}
	\begin{enumerate}
	   \begin{multicols}{4} 
		\item $5000f_{ck}$
		\item $0.7f_{ck}$
		\item $5000\sqrt{f_{ck}}$
		\item $0.7\sqrt{f_{ck}}$
		    \end{multicols}
	\end{enumerate}


\item In the theory of plastic bending of beams, the ratio of plastic moment to yield moment is called                     \hfill{(2009-CE)}
	\begin{enumerate}
	   \begin{multicols}{2} 
		\item shape factor
		\item plastic section modulus
		\item modulus of resilience
		\item rigidity modulus
		    \end{multicols}
	\end{enumerate}

\item For limit state of collapse, the partial saftey factors recommended by IS 456:2000 for estimating design strength of concrete and reinforcing steel are respectively\hfill{(2009-CE)}
	\begin{enumerate}
	   \begin{multicols}{4} 
		\item 1.15 and 1.5
		\item 1.0 and 1.0
		\item 1.5 and 1.15
		\item 1.5 and 1.0
		    \end{multicols}
	\end{enumerate}

\item The point within the cross sectional plane of a beam through which the resultant of the external loading on the beam has to pass through to ensure pure bending without twisting of the cross-section of beam is called   \hfill{(2009-CE)}
	\begin{enumerate}
	   \begin{multicols}{4} 
		\item moment centre
		\item centroid
		\item shear centre
		\item elastic centre
		    \end{multicols}
	\end{enumerate}

\item The square root of the ratio of moment of inertia of the cross section to its cross sectional area is called     \hfill{(2009-CE)}
	\begin{enumerate}
	   \begin{multicols}{2} 
		\item second moment of area
		\item slenderness ratio
		\item section modulus
		\item radius of gyration
		    \end{multicols}
	\end{enumerate}

\item Deposit with flocculated structure is formed when\hfill{(2009-CE)}

	\begin{enumerate}
	   \begin{multicols}{2} 
		\item clay particles settle on sea bed
		\item clay particles settle on fresh water lake bed
		\item sand particles settle on river bed
		\item sand particles settle on sea bed
		    \end{multicols}
        \end{enumerate}

\item Dilatancy correction is required when a strata is \hfill{(2009-CE)}

	\begin{enumerate}
	   \begin{multicols}{2} 
		\item cohesive and saturated and also has N value \\ of STP $>$ 15
		\item saturated slit/fine sand and N value of SPT\\ $<$ 10 after the  overburden correction	\\
		\item saturated slit/fine sand and N value of SPT $>$ 15 after the   overburden correction	
				 \item coarse sand under dry condition and N value of SPT $<$ 10 after   the overburden correction	
		    \end{multicols}
	\end{enumerate}

\item A precast concrete pile is driven with a 50 kN hammer falling through a height of 1.0 m with an efficiency of 0.6. The set value observed is 4 mm per blow and the combined temporary compression of the pile, cushion and the ground is 6mm. As per Modified Hiley Formula, the ultimate resistance of the pile is\hfill{(2009-CE)}

	\begin{enumerate}
	   \begin{multicols}{4} 
		\item 3000 kN
		\item 4285.7 kN
	    \item 8333 kN
		\item 11905 kN
		    \end{multicols}
	\end{enumerate}









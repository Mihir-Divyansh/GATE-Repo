
\iffalse
\title{GATE}
\author{EE}
\chapter{2016}
\section{AE}
\fi



	\item A channel section shown in the figure has uniform thickness. It is subjected to an anticlockwise torque of $62.5$ Nmm. The maximum possible thichness of the channel section, such that the shear stress induced in it does not exceed $100$ $\text{N/mm}^2$, is $\rule{2cm}{0.4pt}$ mm.\\
\begin{figure}[H]
    \centering
    \begin{circuitikz}
\tikzstyle{every node}=[font=\large]
\draw (10.25,16.75) to[short] (15,16.75);
\draw (10.25,16.75) to[short] (10.25,12.5);
\draw (10.25,12.5) to[short] (15,12.5);
\draw [<->, >=Stealth] (10.25,17) -- (15,17);
\draw [<->, >=Stealth] (10,16.75) -- (10,12.5);
\draw [<->, >=Stealth] (10.25,12.25) -- (15,12.25);
\node [font=\large] at (8.75,14.5) {100 mm};
\node [font=\large] at (12.75,11.75) {100 mm};
\node [font=\large] at (12.75,17.25) {100 mm};
\end{circuitikz}
\end{figure}

	\item The governing differential equation of motion of a damped system is given by $m \frac{d^2 x}{dt^2} + c \frac{dx}{dt} + kx = 0$. If $m = 1$ kg, $c = 2$ Ns/m and $k = 2$ N/m then the frequency of the damped oscillation of this system is $\rule{1.5cm}{0.4pt}$ rad/s.\\

	\item The two dimensional state of stress in a body is described by the Airy's stress function: $\phi = 5\frac{x^4}{12} + \frac{x^3y}{6} + 3\frac{x^2y^2}{2} + 7\frac{xy^3}{6} + E\frac{y^4}{12}$. The Airy's stress function will satisfy the equilibrium and the compatibility requirements if and only if the value of the coefficient E is \rule{1.5cm}{0.4pt}.\\
		
	\item The value of definite integral $\int_{0}^{\pi}\brak{x\sin{x}}dx$ is \rule{1.5cm}{0.4pt}.\\
		
	\item Use Newton-Raphson method to solve the equation: $xe^{x}=1$ . Begin with the initial guess $x_0=0.5$ . The solution after one step is $x =$ $\rule{1.5cm}{0.4pt}$.\\

	\item A wall of thickness $5$ mm is heated by a hot gas flowing along the wall. The gas is at a temperature of $3000$ K, and the convective heat transfer coefficient is $160$ $\text{W/m}^2$K. The wall thermal conductivity is $40$ W/mK. If the colder side of the wall is held at $500$ K, the temperature of the side exposed to the hot gas is $\rule{1.5cm}{0.4pt}$ K.\\
		
	\item A launch vehicle has a main rocket engine with two identical strap-on motors, all of which fire simultaneously during the operation. The main engine delivers a thrust of $6300$ kN with a specific impulse of $428$ s. Each strap-on motor delivers a thrust of $12000$ kN with specific impulse of $292$ s. The acceleration due to gravity is $9.81$ $\text{m/s}^2$. The effective $\brak{\text{combined}}$ specific impulse of the vehicle is $\rule{1.5cm}{0.4pt}$ s.\\
		
	\item A substance experiences an entropy change of $\Delta \text{s} > 0$ in a quasi-steady process. The rise in temperature $\brak{\text{corresponding to the entropy change $\Delta$s}}$ is highest for the following process:
		\begin{enumerate}
			\item isenthalpic
			\item isobaric
			\item isochoric
			\item isothermal\\
		\end{enumerate}
		
	\item In a particular rocket engine, helium propellant is heated to $6000$ K and $95\%$ of its total enthalpy is recovered as kinetic energy of the nozzle exhaust. Consider helium to be a calorically perfect gas with specific heat at constant pressure of $5200$ J/kgK. The exhaust velocity for such a rocket for an optimum expansion is $\rule{1.5cm}{0.4pt}$ m/s.\\
		
	\item An aircraft is flying level in the North direction at a velocity of $55$ m/s under cross wind from East to West of $5$ m/s. For the given aircraft $\text{C}_{n\beta} = 0.012/$deg and $\text{C}_{n\delta r} = -0.0072/$deg, where $\delta r$ is the rudder deflection and $\beta$ is the side slip angle. The rudder deflection exerted by pilot is $\rule{1.5cm}{0.4pt}$ degrees.\\
		
	\item An aircraft weighing $10000$ N is flying level at $100$ m/s and it is powered by a jet engine. The thrust required for level flight is $1000$ N. The maximum possible thrust produced by the jet engine is $5000$ N. The minimum time required to climb $1000$ m, when flight speed is $100$ m/s, is $\rule{1.5cm}{0.4pt}$ s.\\

	\item The aircraft velocity $\brak{\text{m/s}}$ components in body axes are given as $\sbrak{u, v, w} = \sbrak{100, 10, 10}$. The air velocity $\brak{\text{m/s}}$, angle of attack $\brak{\text{deg}}$ and sideslip angle $\brak{\text{deg}}$ in that order are
		\begin{enumerate}
			\item $\sbrak{120,0.1,0.1}$
			\item $\sbrak{100,0.1,0.1}$
			\item $\sbrak{100.995,0.1,5.73}$
			\item $\sbrak{100.995,5.71,5.68}$\\
		\end{enumerate}
		
	\item The Dutch roll motion of the aircraft is described by following relationship
$$\begin{bmatrix}
\Delta \dot{\beta} \\ 
\Delta \dot{r} 
\end{bmatrix} 
= 
\begin{bmatrix} 
-0.26 & -1 \\ 
4.49 & -0.76 
\end{bmatrix}
\begin{bmatrix} 
\Delta \beta \\ 
\Delta r 
\end{bmatrix}$$
The undamped natural frequency $\brak{\text{rad/s}}$ and damping ratio for the Dutch roll motion in that order are:
               \begin{enumerate}
		       \item $4.68,1.02$
		       \item $4.49,1.02$
		       \item $2.165,0.235$
		       \item $2.165,1.02$
	       \end{enumerate}

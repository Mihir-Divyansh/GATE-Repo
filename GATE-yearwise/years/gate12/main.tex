\iffalse
                       
                        
                        
                        
                    
                        \author{AI24BTECH11006 - Bugada Roopansha}
                        \section{ph}
                        \chapter{2018}
                        \fi
 
   \item The eigenvalues of a Hermitian matrix are all
   \begin{enumerate}
       \item real
\item imaginary
\item of modulus one
\item  real and positive
   \end{enumerate}


\item Which one of the following represents the $3p$ radial wave function of the hydrogen atom? $\brak{a_0 \text{ is the Bohr radius}}$

\begin{enumerate}

\item
\begin{tikzpicture}[scale=0.6] % Adjust the scale factor as needed
    % Axes
    \draw[->] (-1,0) -- (6,0) node[right] {$r/a_0$};
    \draw[->] (0,-1) -- (0,4) node[above] {R(r)};

    % Bézier curve from (0,0) to (5,3) with control points
    \draw[thick, color=black]
        (0,3) .. controls (0.5,-3) and (2,1) .. (5,0.1);

    % Mark point
    \fill (0,0) circle (2pt) node[below left] { O};
\end{tikzpicture}


\item
\begin{tikzpicture}[scale=0.6] % Adjust the scale factor as needed
    % Axes
    \draw[->] (-1,0) -- (6,0) node[right] {$r/a_0$};
    \draw[->] (0,-1) -- (0,4) node[above] {R(r)};

    % Bézier curve from (0,0) to (5,3) with control points
    \draw[thick, color=black]
        (0,0) .. controls (0.5,6) and (1,-4) .. (3,-0.1);

    % Mark point
    \fill (0,0) circle (2pt) node[below left] { O};
\end{tikzpicture}


\item
\begin{tikzpicture}[scale=0.6] % Adjust the scale factor as needed
    % Axes
    \draw[->] (-1,0) -- (6,0) node[right] {$r/a_0$};
    \draw[->] (0,-1) -- (0,4) node[above] {R(r)};

    % Bézier curve from (0,0) to (5,3) with control points
    \draw[thick, color=black]
        (0,3) .. controls (2,-1) .. (5,-0.1);

    % Mark point
    \fill (0,0) circle (2pt) node[below left] { O};
\end{tikzpicture}


\item
\begin{tikzpicture}[scale=0.6] % Adjust the scale factor as needed
    % Axes
    \draw[->] (-1,0) -- (6,0) node[right] {$r/a_0$};
    \draw[->] (0,-1) -- (0,4) node[above] {R(r)};

    % Bézier curve from (0,0) to (5,3) with control points
    \draw[thick, color=black]
        (0,0) .. controls (0.5,5) and (1,-0.5) .. (3,0.1);

    % Mark point
    \fill (0,0) circle (2pt) node[below left] { O};
\end{tikzpicture}


\end{enumerate}



\item Given the following table,

\begin{table}[h!]
    \centering
    \begin{tabular}{|p{3cm}|p{5cm}|}
        \hline
        \textbf{Group I} &  \textbf{Group II}  \\ \hline
        P: Stern-Gerlach experiment & $1$: Wave nature of particles \\ \hline
        Q:  Zeeman effect & $2$: Quantization of energy of electrons in the atoms \\ \hline
        R:  Frank-Hertz experiment & $3$: Existence of electron spin \\ \hline
        S: Davisson-Germer experiment & $4$: Space quantization of angular momentum \\ \hline
    \end{tabular}
\end{table}

Which one of the following correctly matches the experiments from Group I to their inferences in Group II?
\begin{enumerate}
    \item P-$2$, Q-$3$, R-$4$, S-$1$
    \item P-$1$, Q-$3$, R-$2$, S-$4$
    \item P-$3$, Q-$4$, R-$2$, S-$1$
    \item P-$2$, Q-$1$, R-$4$, S-$3$
\end{enumerate}

\item In spherical polar coordinates $ \brak{r, \theta, \phi} $, the unit vector $\hat{\theta}$ at $\brak{10, \frac{\pi}{4}, \frac{\pi}{2}}$ is
\begin{enumerate}
    \item $\hat{k}$
    \item $\frac{1}{\sqrt{2}} \brak{\hat{j} + \hat{k}}$
    \item $\frac{1}{\sqrt{2}} \brak{-\hat{j} + \hat{k}}$
    \item $\frac{1}{\sqrt{2}} \brak{\hat{j} - \hat{k}}$
\end{enumerate}

\item The scale factors corresponding to the covariant metric tensor $g$ in spherical polar coordinates are
\begin{enumerate}
    \item $1, r^2, r^2 \sin^2 \theta$
    \item $1, r^2, \sin^2 \theta$
    \item $1, 1, 1$
    \item $1, r, r \sin \theta$
\end{enumerate}

\item In the context of small oscillations, which one of the following does NOT apply to the normal coordinates?
\begin{enumerate}
    \item Each normal coordinate has an eigen-frequency associated with it
    \item The normal coordinates are orthogonal to one another
    \item The normal coordinates are all independent
    \item The potential energy of the system is a sum of squares of the normal coordinates with constant coefficients
\end{enumerate}

\item For the given unit cells of a two-dimensional square lattice, which option lists all the primitive cells?\\
\begin{figure}[!ht]
\centering
\resizebox{0.6\textwidth}{!}{%
\begin{circuitikz}
\tikzstyle{every node}=[font=\LARGE]
\draw (7.75,8) to[short] (7.75,8);
\draw (7.75,7.5) to[short] (7.75,7.5);
\node [font=\Large] at (14,8.5) {};
\node [font=\Large] at (14,8.5) {};


\node at (4.5,10.5) [circ] {};
\node at (4.5,9.75) [circ] {};
\node at (4.5,9) [circ] {};
\node at (4.5,8.25) [circ] {};
\node at (5.25,10.5) [circ] {};
\node at (6,10.5) [circ] {};
\node at (6.75,10.5) [circ] {};
\node at (7.5,10.5) [circ] {};
\node at (8.25,10.5) [circ] {};
\node at (9,10.5) [circ] {};
\node at (5.25,9.75) [circ] {};
\node at (5.25,9) [circ] {};
\node at (6,9.75) [circ] {};
\node at (6.75,9.75) [circ] {};
\node at (7.5,9.75) [circ] {};
\node at (8.25,9.75) [circ] {};
\node at (9,9.75) [circ] {};
\node at (9,9) [circ] {};
\node at (9,8.25) [circ] {};
\node at (5.25,8.25) [circ] {};
\node at (6,8.25) [circ] {};
\node at (6,9) [circ] {};
\node at (6.75,9) [circ] {};
\node at (7.5,9) [circ] {};
\node at (8.25,9) [circ] {};
\node at (8.25,8.25) [circ] {};
\node at (7.5,8.25) [circ] {};
\node at (6.75,8.25) [circ] {};
\draw (4.5,10.5) to[short] (4.5,9.75);
\draw (4.5,9.75) to[short] (5.25,9.75);
\draw (4.5,10.5) to[short] (5.25,10.5);
\draw (5.25,10.5) to[short] (5.25,9.75);
\draw (6,10.5) to[short] (7.5,10.5);
\draw (7.5,10.5) to[short] (7.5,9.75);
\draw (7.5,9.75) to[short] (6,9.75);
\draw (6,9.75) to[short] (6,10.5);
\draw (4.5,8.25) to[short] (5.25,9);
\draw (5.25,9) to[short] (6,9);
\draw (6,9) to[short] (5.25,8.25);
\draw (5.25,8.25) to[short] (4.5,8.25);
\draw (6.25,9.25) to[short] (6.25,8.5);
\draw (6.25,9.25) to[short] (7.25,9.25);
\draw (7.25,8.5) to[short] (7.25,9.25);
\draw (6.25,8.5) to[short] (7.25,8.5);
\draw (8,10) to[short] (8,9.25);
\draw (8,10) to[short] (8.75,10);
\draw (8,9.25) to[short] (8.75,9.25);
\draw (8.75,10) to[short] (8.75,9.25);
\node [font=\small] at (5,10) {1};
\node [font=\small] at (5,8.5) {2};
\node [font=\small] at (6.5,10) {5};
\node [font=\small] at (8.5,9.5) {4};
\node [font=\small] at (7,9) {3};
\end{circuitikz}
}%
\end{figure}

\begin{enumerate}
    \item $1$ and $2$
    \item $1$,$2$ and $3$
  \item $1$,$2$,$3$ and $4$
   \item  $1$,$2$,$3$,$4$ and $5$
\end{enumerate}

\item Among electric field $\brak{\vec{E}}$, magnetic field $\brak{\vec{B}}$, angular momentum $\brak{\vec{L}}$, and vector potential $\brak{\vec{A}}$, which is/are odd under parity \brak{\text{space inversion}} operation?
\begin{enumerate}
    \item $\vec{E}$ only
    \item $\vec{E}$ \& $\vec{A}$ only
    \item $\vec{E}$ \& $\vec{B}$ only
    \item $\vec{B}$ \& $\vec{L}$ only
\end{enumerate}

\item The expression for the second overtone frequency in the vibrational absorption spectra of a diatomic molecule in terms of the harmonic frequency $\omega$ and anharmonicity constant $x_e$, is
\begin{enumerate}
    \item $2 \omega_e \brak{1 - x_e}$
    \item $2 \omega_e \brak{1 - 3x_e}$
    \item $3 \omega_e \brak{1 - 2x_e}$
    \item $3 \omega_e \brak{1 - 4x_e}$
\end{enumerate}

\item Match the physical effects and order of magnitude of their energy scales given below, where $\alpha=\frac{e^2}{4\pi\epsilon_o\hbar c}$ is the fine structure constant, $m_e$ and $m_p$ are the electron and proton masses, respectively.

\begin{tabular}{|c|c|}
\hline
Group I & Group II \\ \hline
P: Lamb shift & $1: \sim 0 \brak{\alpha^2 m_e c^2}$ \\ \hline
Q: Fine structure & $2: \sim 0 \brak{\alpha^4 m_e c^2}$ \\ \hline
R: Bohr energy & $3: \sim 0 \brak{\frac{\alpha^4 m_e^2 c^2}{ m_p}}$ \\ \hline
S: Hyperfine structure & $4: \sim 0 \brak{\alpha^5 m_p c^2}$ \\ \hline
\end{tabular}


\begin{enumerate}
    \item $P-3, Q-1, R-2, S-4$
    \item $P-2, Q-3, R-1, S-4$
    \item $P-4, Q-2, R-1, S-3$
    \item $P-2, Q-4, R-1, S-3$
\end{enumerate}

\item The logic expression $\bar{A}BC + \bar{A}\bar{B}C + AB\bar{C} + A\bar{B}\bar{C}$ can be simplified to
\begin{enumerate}
    \item $A XOR C$
    \item $A AND C$
    \item $0$
    \item $1$
\end{enumerate}

\item At low temperatures $\brak{T}$, the specific heat of common metals is described by $\brak{\text{with} \alpha \text{and} \beta \text{as constants}}$
\begin{enumerate}
    \item $\alpha T + \beta T^3$
    \item $\beta T^3$
    \item $\exp\brak{-\frac{\alpha}{T}}$
    \item $\alpha T + \beta T^5$
\end{enumerate}

\item In a $2$-to-$1$ multiplexer as shown below, the output $X=A_0$ if $C=0$ and $X=A_1$ if $C=1$.\\
\begin{figure}[!ht]
\centering
\resizebox{0.3\textwidth}{!}{%
\begin{circuitikz}
\tikzstyle{every node}=[font=\LARGE]
\draw (7.75,8) to[short] (7.75,8);
\draw (7.75,7.5) to[short] (7.75,7.5);

\node [font=\Large] at (14,8.5) {};
\node [font=\Large] at (14,8.5) {};
\draw  (10.25,9) rectangle (11.75,7);

\draw (11.75,8) to[short] (13,8);
\draw (11,10) to[short] (11,9);
\draw (9,8.25) to[short] (10.25,8.25);
\draw (9,7.5) to[short] (10.25,7.5);
\node [font=\LARGE] at (11,10.25) {C};
\node [font=\LARGE] at (13.5,8) {X};
\node [font=\LARGE] at (8.5,8) {$A_0$};
\node [font=\LARGE] at (8.5,7.5) {$A_1$};
\end{circuitikz}
}%
\end{figure}

Which one of the following is the correct implementation of this multiplexer?
\begin{enumerate}
    \item 
\centering
\resizebox{0.3\textwidth}{!}{%
\begin{circuitikz}
\tikzstyle{every node}=[font=\LARGE]
% Connections for the first AND gate
\draw (5.5,9.75) to[short] (7.5,9.75);
\draw (7.5,9.75) to[short] (7.75,9.75);
\draw (7.5,9.25) to[short] (7.75,9.25);
\draw (7.75,9.75) node[ieeestd and port, anchor=in 1, scale=0.89](port1){} (port1.out) to[short] (9.5,9.5);
\draw (5.75,9.25) to[short, -o] (7.75,9.25);
\draw (5.75,8) to[short] (5.75,9.25);

% Connections for input C
\draw (4.5,8.5) to[short] (5.75,8.5);

% Connections for the second AND gate
\draw (5.75,8) to[short] (7.75,8);
\draw (7.5,8) to[short] (7.75,8);
\draw (7.5,7.5) to[short] (7.75,7.5);
\draw (7.75,8) node[ieeestd and port, anchor=in 1, scale=0.89](port2){} (port2.out) to[short] (9.5,7.75);
\draw (6.5,7.5) to[short] (7.75,7.5);

% Connections for OR gate
\draw (9.5,9.5) to[short] (9.5,8.75);
\draw (9.5,8.25) to[short] (11.5,8.25);
\draw (9.5,8.75) to[short] (11.5,8.75);
\draw (9.5,7.75) to[short] (9.5,8.25);
\draw (11.25,8.75) to[short] (11.5,8.75);
\draw (11.25,8.25) to[short] (11.5,8.25);
\draw (11.5,8.75) node[ieeestd or port, anchor=in 1, scale=0.89](port3){} (port3.out) to[short] (13.25,8.5);

% Output node and labels
\node at (13.25,8.5) [circ] {};
\node [font=\LARGE] at (14,8.5) {X};

% Input labels
\node [font=\LARGE] at (5,9.5) {$A_0$};
\node [font=\LARGE] at (4,8.5) {C};
\node [font=\LARGE] at (6,7.5) {$A_1$};

\end{circuitikz}
}%


\item 
\centering
\resizebox{0.3\textwidth}{!}{%  % Set to 50% of the text width
\begin{circuitikz}
\tikzstyle{every node}=[font=\LARGE]
% Connections for the first AND gate
\draw (5.5,9.75) to[short] (7.5,9.75);
\draw (7.5,9.75) to[short] (7.75,9.75);
\draw (7.5,9.25) to[short] (7.75,9.25);
\draw (7.75,9.75) node[ieeestd and port, anchor=in 1, scale=0.89](port1){} (port1.out) to[short] (9.5,9.5);

% Connections for other components
\draw (5.75,8) to[short] (5.75,9.25);
\draw (4.5,8.5) to[short] (5.75,8.5);
\draw (7.75,8) to[short] (7.75,8);
\draw (7.75,7.5) to[short] (7.75,7.5);
\draw (7.75,8) node[ieeestd and port, anchor=in 1, scale=0.89](port2){} (port2.out) to[short] (9.5,7.75);
\draw (6.5,7.5) to[short] (7.75,7.5);

% OR gate connections
\draw (9.5,9.5) to[short] (9.5,8.75);
\draw (9.5,8.25) to[short] (11.5,8.25);
\draw (9.5,8.75) to[short] (11.5,8.75);
\draw (9.5,7.75) to[short] (9.5,8.25);
\draw (11.25,8.75) to[short] (11.5,8.75);
\draw (11.25,8.25) to[short] (11.5,8.25);
\draw (11.5,8.75) node[ieeestd or port, anchor=in 1, scale=0.89](port3){} (port3.out) to[short] (13.25,8.5);

% Output node and labels
\node at (13.25,8.5) [circ] {};
\node [font=\Large] at (14,8.5) {X};

% Input labels
\node [font=\Large] at (5,9.5) {$A_0$};
\node [font=\Large] at (4,8.5) {C};
\node [font=\Large] at (6,7.5) {$A_1$};

% Additional short connections for clarity
\draw (7.5,9.25) to[short] (5.75,9.25);
\draw (7.5,8) to[short] (7.75,8);
\draw (7.5,7.5) to[short] (7.75,7.5);
\draw (5.75,8) to[short, -o] (7.75,8);

\end{circuitikz}
}%


\item 


\resizebox{0.3\textwidth}{!}{
\begin{circuitikz}
\tikzstyle{every node}=[font=\LARGE]
\draw (5.5,9.75) to[short] (7.5,9.75);
\draw (5.75,8) to[short] (5.75,9.25);
\draw (4.5,8.5) to[short] (5.75,8.5);
\draw (7.75,8) to[short] (7.75,8);
\draw (7.75,7.5) to[short] (7.75,7.5);
\draw (7.75,8) node[ieeestd and port, anchor=in 1, scale=0.89](port1){} (port1.out) to[short] (9.5,7.75);
\draw (6.5,7.5) to[short] (7.75,7.5);
\draw (9.5,9.5) to[short] (9.5,8.75);
\draw (9.5,8.25) to[short] (11.5,8.25);
\draw (9.5,8.75) to[short] (11.5,8.75);
\draw (9.5,7.75) to[short] (9.5,8.25);
\draw (11.25,8.75) to[short] (11.5,8.75);
\draw (11.25,8.25) to[short] (11.5,8.25);
\draw (11.5,8.75) node[ieeestd or port, anchor=in 1, scale=0.89](port2){} (port2.out) to[short] (13.25,8.5);

% Output node and labels
\node at (13.25,8.5) [circ] {};
\node [font=\Large] at (14,8.5) {X};

% Input labels
\node [font=\Large] at (5,9.5) {$A_0$};
\node [font=\Large] at (4,8.5) {C};
\node [font=\Large] at (6,7.5) {$A_1$};

% Additional connections
\draw (7.5,8) to[short] (5.75,8);
\draw (7.5,8) to[short] (7.75,8);
\draw (7.5,7.5) to[short] (7.75,7.5);
\draw (7.5,9.75) to[short] (7.75,9.75);
\draw (7.5,9.25) to[short] (7.75,9.25);
\draw (7.75,9.75) node[ieeestd or port, anchor=in 1, scale=0.89](port4){} (port4.out) to[short] (9.5,9.5);
\draw (5.75,9.25) to[short, -o] (7.75,9.25);
\end{circuitikz}
}%


\item 



\resizebox{0.3\textwidth}{!}{%  % Set to 60% of the text width
\begin{circuitikz}
\tikzstyle{every node}=[font=\LARGE]
\draw (5.5,9.75) to[short] (7.5,9.75);
\draw (7.5,9.75) to[short] (7.75,9.75);
\draw (7.5,9.25) to[short] (7.75,9.25);
\draw (7.75,9.75) node[ieeestd and port, anchor=in 1, scale=0.89](port1){} (port1.out) to[short] (9.5,9.5);
\draw (5.75,8) to[short] (5.75,9.25);
\draw (4.5,8.5) to[short] (5.75,8.5);
\draw (7.75,8) to[short] (7.75,8);
\draw (7.75,7.5) to[short] (7.75,7.5);
\draw (7.75,8) node[ieeestd and port, anchor=in 1, scale=0.89](port2){} (port2.out) to[short] (9.5,7.75);
\draw (6.5,7.5) to[short] (7.75,7.5);
\draw (9.5,9.5) to[short] (9.5,8.75);
\draw (9.5,8.25) to[short] (11.5,8.25);
\draw (9.5,8.75) to[short] (11.5,8.75);
\draw (9.5,7.75) to[short] (9.5,8.25);
\draw (11.25,8.75) to[short] (11.5,8.75);
\draw (11.25,8.25) to[short] (11.5,8.25);
\draw (11.5,8.75) node[ieeestd or port, anchor=in 1, scale=0.89](port3){} (port3.out) to[short] (13.25,8.5);

% Output node and labels
\node at (13.25,8.5) [circ] {};
\node [font=\Large] at (14,8.5) {X};

% Input labels
\node [font=\Large] at (5,9.5) {$A_0$};
\node [font=\Large] at (4,8.5) {C};
\node [font=\Large] at (6,7.5) {$A_1$};

% Additional connections
\draw (7.5,9.25) to[short] (5.75,9.25);
\draw (7.5,8) to[short] (5.75,8);

\end{circuitikz}
}%



\end{enumerate}

                                        






\iffalse
\title{GATE}
\author{EE}
\chapter{2018}
\section{EE}
\fi


\item The Fourier transform of a continuous-time signal $x\brak{t}$ is given by $X\brak{\omega}=\frac{1}{\brak{10+j\omega}^2}$, $-\infty<\omega<\infty$, where $j=\sqrt{-1}$ and $\omega$ denotes frequency. Then the value of $\abs{\ln{x\brak{t}}}$ at $t=1$ is $\rule{2cm}{0.4pt}$ \brak{\text{up to 1 decimal place}}. \brak{\text{$\ln$ denotes the logarithm to base $e$}}\\

	\item In the circuit shown in the figure, the bipolar junction transistor $\brak{\text{BJT}}$ has a current gain $\beta = 100$. The base-emitter voltage drop is a constant, $V_{BE} = 0.7 V$. The value of the Thevenin equivalent resistance $R_{Th}$ $\brak{\text{in \ohm}}$ as shown in the figure is $\rule{2cm}{0.4pt}$ $\brak{\text{up to 2 decimal places}}$.\\
\begin{figure}[H]
    \centering
  \begin{circuitikz}
\tikzstyle{every node}=[font=\large]

\draw (8.5,14) to[Tnpn, transistors/scale=1.19] (6.5,14);
\draw (4.5,14) to[american voltage source] (4.5,9.75);
\draw (7.5,11.5) to[american voltage source] (7.5,9.75);
\draw (4.5,14) to[R] (6.5,14);
\draw (7.5,13.25) to[R] (7.5,11.5);
\draw (4.5,9.75) to[short] (6.5,9.75);
\draw (4.5,9.75) to[short] (11.75,9.75);
\draw (10,14) to[R] (10,9.75);
\draw (8.25,14) to[short] (11.75,14);
\node [font=\large] at (12,14) {a};
\node [font=\large] at (12,9.75) {b};
\node [font=\large] at (5.5,14.5) {10 $\Omega$};
\node [font=\large] at (6.5,12.5) {10 k$\Omega$};
\node [font=\large] at (3.5,11.75) {15 V};
\node [font=\large] at (6.25,10.5) {10.7 V};
\node [font=\large] at (9.25,12) {1 k$\Omega$};
\node at (7.5,9.75) [circ] {};
\node at (10,14) [circ] {};
\node at (10,9.75) [circ] {};
\node at (11.75,9.75) [circ] {};
\node at (11.75,14) [circ] {};
\draw [->, >=Stealth] (12,12) -- (11,12);
\node [font=\large] at (12.5,12) {$R_{TH}$};
\end{circuitikz}

\end{figure}
		

	\item As shown in the figure, $C$ is the arc from the point $\brak{3,0}$ to the point $\brak{0,3}$ on the circle $x^2 + y^2 = 9$. The value of the integral $\int_{C} \brak{y^2+2yx}dx + \brak{2xy+x^2}dy$ is $\rule{2cm}{0.4pt}$ $\brak{\text{up to 2 decimal places}}$.\\
\begin{figure}[H]
    \centering
    \begin{tikzpicture}
    % Draw axes
    \draw[->] (-0.5, 0) -- (4, 0) node[right] {$x$};
    \draw[->] (0, -0.5) -- (0, 4) node[above] {$y$};
    
    % Draw semicircle with an arrow pointing along the curve toward C
    \draw[thick] (0,3) arc[start angle=90, end angle=0, radius=3];
    \draw[thick,->] (3,0) arc[start angle=0, end angle=45, radius=3]; % Arrow along the curve ending at C
    
    % Add labels for points and curve
    \node at (0,3) [left] {$(0,3)$};
    \node at (3,0) [below] {$(3,0)$};
    \node at (1.5,1.5) [above right] {$C$};
\end{tikzpicture}

\end{figure}
		

	\item Let $f\brak{x}=3x^3-7x^2+5x+6$. The maximum value of $f\brak{x}$ over the interval $\sbrak{0, 2}$ is $\rule{2cm}{0.4pt}$ $\brak{\text{up to 1 decimal place}}$.\\

	\item Let $A = \begin{bmatrix} 
	1 & 0 & -1 \\
	-1 & 2 & 0 \\
	0 & 0 & -2
	\end{bmatrix}$ and $B = A^3 - A^2 - 4A + 5I$, where $I$ is the $3\times3$ identity matrix. The determinant of B is $\rule{2cm}{0.4pt}$ (up to 1 decimal place).\\

        \item The capacitance of an air-filled parallel-plate capacitor is $60$ pF. When a dielectric slab whose thickness is half the distance between the plates, is placed on one of the plates covering it entirely, the capacitance becomes $86$ pF. Neglecting the fringing effects, the relative permittivity of the dielectric is $\rule{2cm}{0.4pt}$ (up to 2 decimal places).\\

	\item he unit step response $y\brak{t}$ of a unity feedback system with open loop transfer function $G\brak{s}H\brak{s}=\frac{K}{\brak{s+1}^2\brak{s+2}}$ is shown in the figure. The value of K is $\rule{2cm}{0.4pt}$ $\brak{\text{up to 2 decimal places}}$.\\
\begin{figure}[H]
    \centering
    \begin{circuitikz}
\tikzstyle{every node}=[font=\normalsize]
\draw (6.5,18.5) to[short] (6.5,10.25);
\draw (5.5,11) to[short] (19.25,11);
\node [font=\LARGE] at (6.5,17.75) {-};
\node [font=\LARGE] at (6.5,16.75) {-};
\node [font=\LARGE] at (6.5,15.75) {-};
\node [font=\LARGE] at (6.5,14.75) {-};
\node [font=\LARGE] at (6.5,13.75) {-};
\node [font=\LARGE] at (6.5,12.75) {-};
\node [font=\LARGE] at (6.5,11.75) {-};
\node [font=\LARGE, rotate around={-90:(0,0)}] at (7.75,11) {-};
\node [font=\LARGE, rotate around={-90:(0,0)}] at (10.25,11) {-};
\node [font=\LARGE, rotate around={-90:(0,0)}] at (9,11) {-};
\node [font=\LARGE, rotate around={-90:(0,0)}] at (15.25,11) {-};
\node [font=\LARGE, rotate around={-90:(0,0)}] at (14,11) {-};
\node [font=\LARGE, rotate around={-90:(0,0)}] at (12.75,11) {-};
\node [font=\LARGE, rotate around={-90:(0,0)}] at (11.5,11) {-};
\node [font=\LARGE, rotate around={-90:(0,0)}] at (16.5,11) {-};
\node [font=\LARGE, rotate around={-90:(0,0)}] at (17.75,11) {-};
\node [font=\LARGE, rotate around={-90:(0,0)}] at (18.75,11) {-};
\draw [dashed] (6.5,14.75) -- (19.25,14.75);
\draw [short] (6.5,11) .. controls (7,10.5) and (7.25,21) .. (8.5,14.75);
\draw [short] (8.5,14.75) .. controls (9.25,11.75) and (9.5,16.75) .. (11,14.75);
\draw [short] (11,14.75) .. controls (12,13.5) and (12.25,15.5) .. (13.25,14.75);
\draw [short] (13.25,14.75) .. controls (14.75,14.5) and (14.75,15) .. (15.75,14.75);
\draw [short] (15.75,14.75) .. controls (17.25,14.5) and (17.5,15) .. (18.75,14.75);
\node [font=\LARGE] at (4.25,16.75) {y(t)};
\node [font=\LARGE] at (18,9.75) {t (sec)};
\node [font=\normalsize] at (7.75,10.5) {2};
\node [font=\normalsize] at (9,10.5) {4};
\node [font=\normalsize] at (10.25,10.5) {6};
\node [font=\normalsize] at (11.5,10.5) {8};
\node [font=\normalsize] at (12.75,10.5) {10};
\node [font=\normalsize] at (14,10.5) {12};
\node [font=\normalsize] at (15.25,10.5) {14};
\node [font=\normalsize] at (16.5,10.5) {16};
\node [font=\normalsize] at (17.75,10.5) {18};
\node [font=\normalsize] at (19,10.5) {20};
\node [font=\normalsize] at (6,11.75) {0.2};
\node [font=\normalsize] at (6,12.75) {0.4};
\node [font=\normalsize] at (6,13.75) {0.6};
\node [font=\normalsize] at (6,14.75) {0.8};
\node [font=\normalsize] at (6.25,15.75) {1};
\node [font=\normalsize] at (6,16.75) {1.2};
\node [font=\normalsize] at (6,17.75) {1.4};
\draw [short] (18.75,14.75) -- (19,14.75);
\end{circuitikz}

\end{figure}
		


	\item A three-phase load is connected to a three-phase balanced supply as shown in the figure. If $V_{an}=100\angle0^\circ,V_{bn}=100\angle-120^\circ$ and $V_{cn}=100\angle-240^\circ$ $\brak{\text{angles are considered positive in the anti-clockwise direction}}$, the value of R for zero current in the neutral wire is $\rule{2cm}{0.4pt}$ $\brak{\text{up to 2 decimal places}}$.\\
\begin{figure}[H]
    \centering
   \begin{circuitikz}
\tikzstyle{every node}=[font=\normalsize]
\draw (8,13.75) to[short, -o] (7.5,13.75) ;
\draw (8,12.5) to[short, -o] (7.5,12.5) ;
\draw (8,11.5) to[short, -o] (7.5,11.5) ;
\draw (8,10) to[short, -o] (7.5,10) ;
\draw [->, >=Stealth] (8,13.75) -- (9,13.75);
\draw [->, >=Stealth] (10.25,12.5) -- (8.75,12.5);
\draw [->, >=Stealth] (8,11.5) -- (9.5,11.5);
\draw [->, >=Stealth] (8,10) -- (9.5,10);
\draw (7.75,12.5) to[short] (9,12.5);
\draw (10.25,12.5) to[short] (10.75,12);
\draw (10.75,13.75) to[R] (10.75,12);
\draw (9,13.75) to[short] (10.75,13.75);
\draw (10.75,12) to[C] (9.75,11);
\draw (9.75,11.5) to[short] (9.75,11);
\draw (8.5,11.5) to[short] (9.75,11.5);
\node at (10.75,12) [circ] {};
\draw (10.75,12) to[L ] (12,10.75);
\draw (9.5,10) to[short] (12,10);
\draw (12,10.75) to[short] (12,10);
\node [font=\normalsize] at (7,13.75) {a};
\node [font=\normalsize] at (7,12.5) {n};
\node [font=\normalsize] at (7,11.5) {c};
\node [font=\normalsize] at (7,10) {b};
\node [font=\normalsize] at (10.75,11) {-j10};
\node [font=\normalsize] at (12,11.75) {j10};
\node [font=\normalsize] at (11.25,13) {R};
\end{circuitikz}

\end{figure}
		

	\item The voltage across the circuit in the figure, and the current through it, are given by the following expressions:\\$v\brak{t}=5-10\cos{\brak{\omega t+60^\circ}}V$\\$i\brak{t}=5+X\cos{\brak{\omega t}}A$\\
	where $\omega =100\pi$ radian/s. If the average power delivered to the circuit is zero, then the value of $X\brak{\text{in Ampere}}$ is $\rule{2cm}{0.4pt}$ $\brak{\text{up to 2 decimal places}}$.\\
\begin{figure}[H]
    \centering
    \begin{circuitikz}
\tikzstyle{every node}=[font=\Large]
\draw [->, >=Stealth] (8.25,14) -- (9.5,14);
\draw (9.5,14) to[short] (10.75,14);
\draw (10.75,14) to[short] (10.75,13.25);
\draw  (9.5,13.25) rectangle (12,10.75);
\node [font=\large] at (10.75,12.25) {Electrical};
\node [font=\large] at (10.75,11.75) {Circuit};
\draw (10.75,10.75) to[short] (10.75,10);
\draw (8.25,10) to[short] (10.75,10);
\node [font=\Large] at (8.75,13.5) {+};
\node [font=\Large] at (8.75,10.5) {-};
\node [font=\Large] at (9.25,14.75) {i(t)};
\node [font=\Large] at (8.75,12) {v(t)};
\node at (8.25,14) [circ] {};
\node at (8.25,10) [circ] {};
\end{circuitikz}

\end{figure}
		

        \item A phase controlled single phase rectifier, supplied by an AC source, feeds power to an R-L-E load as shown in the figure. The rectifier output voltage has an average value given by $V_0=\frac{v_m}{2\pi}\brak{3+\cos{\alpha}}$, where $V_m=80\pi$ volts and $\alpha$ is the firing angle. If the power delivered to the lossless battery is $1600$ W, $\alpha$ in degree is $\rule{2cm}{0.4pt}$ (up to $2$ decimal places).\\
\begin{figure}[H]
    \centering
   \begin{circuitikz}
\tikzstyle{every node}=[font=\Large]
\draw (8,15.25) to[sinusoidal voltage source, sources/symbol/rotate=auto] (8,10.75);
\draw (8,15.25) to[short] (9.25,15.25);
\draw (8,10.75) to[short] (9.25,10.75);
\draw  (9.25,16) rectangle (11,10);
\draw (11,15.25) to[short] (11.5,15.25);
\draw (11,10.75) to[short] (11.5,10.75);
\draw (13,10.75) to[short, -o] (11.5,10.75) ;
\draw (13,15.25) to[short, -o] (11.5,15.25) ;
\draw (13,15.25) to[R] (13,13.5);
\draw (13,13.5) to[L ] (13,12);
\draw [->, >=Stealth] (12,11.5) -- (12,14.5);
\draw (10.25,12.25) to[D] (10.25,13.75);
\draw (10.25,13.25) to[short, -o] (10.75,13.75) ;
\draw [short] (12.75,11.25) -- (13.25,11.25);
\draw [short] (12.5,11.5) -- (13.5,11.5);
\draw [short] (13,12.25) -- (13,11.5);
\draw [short] (13,11.25) -- (13,10.75);
\node [font=\large] at (13.5,11.75) {+};
\node [font=\large] at (13.5,11) {-};
\node [font=\normalsize] at (14.25,11.75) {80 V};
\node [font=\normalsize] at (14.25,11.25) {Battery};
\node [font=\normalsize] at (14,12.75) {10 mH};
\node [font=\normalsize] at (13.5,14.5) {2 $\Omega$};
\node [font=\normalsize] at (11.75,15.5) {+};
\node [font=\Large] at (11.75,10.5) {-};
\node [font=\normalsize] at (11.5,13) {$V_0$};
	\node [font=\Large] at (6,13) {$V_m \sin{\brak{\omega t}}$};
\end{circuitikz}

\end{figure}
		

	\item The figure shows two buck converters connected in parallel. The common input dc voltage for the converters has a value of $100$ V. The converters have inductors of identical value. The load resistance is $1\ohm$ The capacitor voltage has negligible ripple. Both converters operate in the continuous conduction mode. The switching frequency is $1$ kHz, and the switch control signals are as shown. The circuit operates in the steady state. Assuming that the converters share the load equally, the average value of $i_{S1}$, the current of switch $S1$ (in Ampere), is $\rule{2cm}{0.4pt}$ (up to $2$ decimal places).\\
\begin{figure}[H]
    \centering
  \begin{circuitikz}
\tikzstyle{every node}=[font=\normalsize]
\draw (3,19.25) to[american voltage source] (3,17.25);
\draw (3,17.25) to[short] (10.25,17.25);
\draw (10.25,19.25) to[R] (10.25,17.25);
\draw (9.25,19.25) to[C] (9.25,17.25);
\draw (8.5,19.25) to[short] (8.5,15.75);
\draw (6.75,15.75) to[L ] (8.5,15.75);
\draw (4,19.25) to[normal open switch] (6.75,19.25);
\draw (6.75,19.25) to[L ] (8.5,19.25);
\draw (8.5,19.25) to[short] (10.25,19.25);
\draw (6.25,17.25) to[D] (6.25,19.25);
\draw (4,19.25) to[short] (4,15.75);
\draw (3,19.25) to[short] (4,19.25);
\draw (4,15.75) to[normal open switch] (6.75,15.75);
\draw (6.25,14) to[D] (6.25,15.75);
\draw (3,17.25) to[short] (3,14);
\draw (3,14) to[short] (6.25,14);
\node at (3,17.25) [circ] {};
\node at (4,19.25) [circ] {};
\node at (6.25,17.25) [circ] {};
\node at (6.25,19.25) [circ] {};
\node at (5.5,19.25) [circ] {};
\node at (5.25,19.25) [circ] {};
\node at (8.5,19.25) [circ] {};
\node at (9.25,19.25) [circ] {};
\node at (9.25,17.25) [circ] {};
\node at (5.5,15.75) [circ] {};
\node at (5.25,15.75) [circ] {};
\node at (6.25,15.75) [circ] {};
\node [font=\large] at (2,18.25) {100 V};
\node [font=\large] at (4.75,20) {$i_{S1}$};
\node [font=\large] at (5.25,18.75) {S1};
\node [font=\large] at (7.5,20) {L};
\node [font=\large] at (8.75,18.25) {C};
\node [font=\large] at (11,18.25) {1 $\Omega$};
\node [font=\large] at (7.75,15.25) {L};
\node [font=\large] at (5.25,15.25) {S2};
\draw [->, >=Stealth] (4.5,19.5) -- (5.25,19.5);
\draw [dashed] (12.25,17.5) -- (12.25,13.5);
\draw (11.75,16) to[short] (12.25,16);
\draw (11.75,15.5) to[short] (12.25,15.5);
\draw (12.25,15.5) to[short] (12.25,14);
\draw (12.25,14) to[short] (14,14);
\draw (14,14) to[short] (14,15.5);
\draw (14,15.5) to[short] (15.75,15.5);
\draw (15.75,15.5) to[short] (15.75,14);
\draw (15.75,14) to[short] (17.5,14);
\draw (17.5,14) to[short] (17.5,15.5);
\draw (17.5,15.5) to[short] (18.25,15.5);
\draw (12.25,17.25) to[short] (12.25,16);
\draw (12.25,17.25) to[short] (14,17.25);
\draw (14,17.25) to[short] (14,16);
\draw (14,16) to[short] (15.75,16);
\draw (15.75,16) to[short] (15.75,17.25);
\draw (15.75,17.25) to[short] (17.5,17.25);
\draw (17.5,17.25) to[short] (17.5,16);
\draw (17.5,16) to[short] (18.25,16);
\draw [dashed] (11.5,14) -- (17.75,14);
\draw [->, >=Stealth] (17.5,16) -- (18.75,16);
\draw [->, >=Stealth, dashed] (17.5,14) -- (18.75,14);
\node [font=\large] at (11.75,16.75) {S1};
\node [font=\large] at (11.5,14.75) {S2};
\node [font=\large] at (11.75,13.75) {0};
\node [font=\normalsize] at (14.25,13.75) {0.5 ms};
\node [font=\normalsize] at (16,13.75) {1 ms};
\node [font=\large] at (19.25,16) {t};
\node [font=\large] at (19.25,14) {t};
\draw [dashed] (11.75,16) -- (17.75,16);
\node [font=\normalsize] at (14.75,18) {Switch control signals};
\end{circuitikz}

\end{figure}
		

	\item A $3$-phase $900$ kVA, $3$ kV /$\sqrt{3}$ kV ($\Delta$/Y), $50$ Hz transformer has primary (high voltage side) resistance per phase of $0.3 \ohm$ and secondary (low voltage side) resistance per phase of $0.02 \ohm$. Iron loss of the transformer is $10$ kW. The full load $\%$ efficiency of the transformer operated at unity power factor is $\rule{2cm}{0.4pt}$ (up to $2$ decimal places).\\

	\item A $200$ V DC series motor, when operating from rated voltage while driving a certain load, draws $10$ A current and runs at $1000$ r.p.m. The total series resistance is $1\ohm$. The magnetic circuit is assumed to be linear. At the same supply voltage, the load torque is increased by $44\%$. The speed of the motor in r.p.m. (rounded to the nearest integer) is $\rule{2cm}{0.4pt}$ .


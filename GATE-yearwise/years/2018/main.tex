\iffalse
\chapter{2018}
\author{EE24BTECH11004}
\section{xe}
\fi



%\begin{enumerate}
    \item Which of the following is a quasi-linear partial differential equation? \\
    (A) $x \frac{\partial u}{\partial x} + y \frac{\partial u}{\partial y} = 0$ \\
    (B) $\frac{\partial^2 u}{\partial x^2} + 2 \frac{\partial^2 u}{\partial y^2} = 0$ \\
    (C) $\left( \frac{\partial u}{\partial x} \right)^2 + \left( \frac{\partial u}{\partial y} \right)^2 = 0$ \\
    (D) $\left( \frac{\partial u}{\partial x} \right) - \left( \frac{\partial u}{\partial y} \right)^2 = 0$

    \item Let $P(x)$ and $Q(x)$ be the polynomials of degree 5, generated by Lagrange and Newton interpolation techniques respectively, both passing through six distinct points on the $xy$-plane. Which of the following is correct? \\
    (A) $P(x) \equiv Q(x)$ \\
    (B) $P(x) - Q(x)$ is a polynomial of degree 2 \\
    (C) $P(x) - Q(x)$ is a polynomial of degree 3 \\
    (D) $P(x) - Q(x)$ is a polynomial of degree 5

    \item The Laurent series of $f(z) = \frac{1}{(z^3 - z^4)}$ with center at $z = 0$ in the region $|z| > 1$ is: \\
    (A) $\sum_{n=0}^{\infty} z^{n-3}$ \\
    (B) $-\sum_{n=0}^{\infty} \frac{1}{z^{n+4}}$ \\
    (C) $\sum_{n=0}^{\infty} z^n$ \\
    (D) $\sum_{n=0}^{\infty} \frac{1}{z^n}$

    \item The value of the surface integral $\iint_S \vec{F} \cdot d\vec{S}$ over the sphere $S$ given by $x^2 + y^2 + z^2 = 1$, where $\vec{F} = 4x \hat{i} - z \hat{k}$, and $n$ denotes the outward unit normal, is: \\
    (A) $\pi$ \\
    (B) $2\pi$ \\
    (C) $3\pi$ \\
    (D) $4\pi$

    \item A diagnostic test for a certain disease is 90\% accurate. That is, the probability of a person having (respectively, not having) the disease tested positive (respectively, negative) is 0.9. Fifty percent of the population has the disease. What is the probability that a randomly chosen person has the disease given that the person tested negative?

    \item Let $M = \begin{pmatrix} 1 & 1 \\ 0 & 1 \end{pmatrix}$. Which of the following is correct? \\
    (A) Rank of $M$ is 1 and $M$ is not diagonalizable \\
    (B) Rank of $M$ is 2 and $M$ is diagonalizable \\
    (C) 1 is the only eigenvalue and $M$ is not diagonalizble \\
    (D) 1 is the only eigenvalue and $M$ is diagonalizable

    \item Let $f(x) = 2x^3 - 3x^2 + 69$, $-5 \leq x \leq 5$. Find the point at which $f$ attains the global maximum.

    \item Calculate $\int_{C_1} \vec{F} \cdot d\vec{r} - \int_{C_3} \vec{F} \cdot d\vec{r}$, where $C_1 : \vec{r}(t) = (t, t^2)$ and $C_2 : \vec{r}(t) = (t, \sqrt{t})$, $t$ varying from 0 to 1 and $F = xy \hat{j}$.
\item In the parallel-plate configuration shown, steady flow of an incompressible Newtonian fluid is established by moving the top plate with a constant speed, $U_0 = 1 \, \text{m/s}$. If the force required on the top plate to support this motion is $0.5 \, \text{N}$ per unit area (in $\text{m}^2$) of the plate, then the viscosity of the fluid between the plates is $\underline{\hspace{2cm}}$ $ Ns/m^2$.

\item For a newly designed vehicle by some students, the volume of fuel consumed per unit distance traveled $\left( q_f \text{ in m}^3/\text{m} \right)$ depends on the viscosity $(\mu)$ and density $(\rho)$ of the fuel, as well as the speed $(U)$ and size $(L)$ of the vehicle, given by
$q_f = c \frac{\mu^2 U}{\rho L}$
  where $c$ is a constant. The dimensions of the constant $c$ are:
\begin{figure}[t]
\centering
\resizebox{0.5\textwidth}{!}{%
\begin{circuitikz}
\tikzstyle{every node}=[font=\normalsize]
\draw [line width=1pt, short] (3.25,8.5) -- (7.75,8.5);
\draw [line width=1pt, short] (3.25,10.5) -- (7.75,10.5);
\draw [line width=0.7pt, <->, >=Stealth] (3.75,10.5) -- (3.75,8.5);
\draw [line width=0.7pt, dashed] (5.25,8.5) -- (7.25,8.5);
\draw [line width=0.7pt, dashed] (5.25,8.5) -- (7.25,8.5);
\draw [line width=0.7pt, dashed] (5.25,10.5) -- (5.25,8.5);
\draw [line width=0.7pt, dashed] (5.25,8.5) -- (7,10.5);
\draw [line width=0.7pt, ->, >=Stealth] (5.25,10.25) -- (6.75,10.25);
\draw [line width=0.7pt, ->, >=Stealth] (5.25,9.75) -- (6.25,9.75);
\draw [line width=0.7pt, ->, >=Stealth] (5.25,9.25) -- (5.75,9.25);
\draw [line width=0.7pt, ->, >=Stealth] (5.25,10.5) -- (7,10.5);
\node [font=\normalsize] at (6,10.75) {$U_0$};
\node [font=\normalsize] at (4.25,9.5) {$10 mm$};
\end{circuitikz}
}%
\end{figure}
\item A semi-circular gate of radius $1 \text{ m }$ is placed at the bottom of a water reservoir as shown in the figure below. The hydrostatic force per unit width of the cylindrical gate in the $y$-direction is $\underline{\hspace{2cm}}$. The gravitational acceleration is $g = 9.8 \, \text{m/s}^2$ and the density of water is $1000 \, \text{kg/m}^3$.

\begin{figure}[!ht]
\centering
\resizebox{0.4\textwidth}{!}{%
\begin{circuitikz}
\tikzstyle{every node}=[font=\scriptsize]

\draw [short] (4.25,10.25) -- (4.25,8);
\draw [short] (4.25,8) -- (6.5,8);
\draw [short] (6.5,10.25) -- (6.5,9);
\draw [short] (6.5,9) .. controls (5.75,8.75) and (6,8.25) .. (6.5,8);
\draw [short] (6.5,10) -- (7,10);
\draw [short] (6.5,9) -- (7,9);
\draw [dashed] (6.5,10) -- (4.25,10);
\draw [<->, >=Stealth] (6.75,10) -- (6.75,9);
\draw [->, >=Stealth] (5,8.5) -- (5,9);
\draw [->, >=Stealth] (5,8.5) -- (5.75,8.5);
\draw [short] (5.5,10) -- (5.25,9.75);
\draw [short] (5.5,10) -- (5.75,9.75);
\draw [short] (5.25,9.75) -- (5.75,9.75);
\draw [line width=0.2pt, ->, >=Stealth] (6.75,8.25) -- (6.25,8.75);
\node [font=\normalsize] at (6.75,8.25) {x};
\node [font=\scriptsize] at (7.25,9.5) {2 m};
\node [font=\small] at (7.25,8.25) {gate};
\node [font=\small] at (5.75,8.25) {x};

\end{circuitikz}
}%
\end{figure}
    
    \item The velocity vector in $\text{m/s}$ for a 2-D flow is given in Cartesian coordinates $(x, y)$ as $\vec{V} = \left( \frac{x^2}{2} - \frac{y^2}{2} \right) \hat{i} + xy \hat{j}$. At a point in the flow field, the $x$- and $y$-components of the acceleration vector are given as $1 \, \text{m/s}^2$ and $-0.5 \, \text{m/s}^2$, respectively. The velocity magnitude at that point is $\underline{\hspace{2cm}}$.
\item  If $\phi(x, y)$ is the velocity potential and $\psi(x, y)$ is the stream function for a 2-D, steady, incompressible, and irrotational flow, which one of the following is incorrect?
\begin{enumerate}
    \item[(A)] $\left( \frac{\partial \psi}{\partial x} \right)_{\phi= \text{const}}  = -\frac{1}{\left( \frac{\partial \psi}{\partial y} \right)_{\psi= \text{const}}}$
    \item[(B)] $\frac{\partial^2 \psi}{\partial x^2} + \frac{\partial^2 \psi}{\partial y^2} = 0$
    \item[(C)] $\left( \frac{\partial \psi}{\partial y} \right)_{\phi= \text{const}} =  \frac{1}{\left( \frac{\partial \psi}{\partial x} \right)_{\psi= \text{const}}}$
    \item[(D)] $\frac{\partial^2 \phi}{\partial x^2} + \frac{\partial^2 \phi}{\partial y^2} = 0$
\end{enumerate}


%\end{enumerate}



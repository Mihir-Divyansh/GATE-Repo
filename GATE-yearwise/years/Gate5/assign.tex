\iffalse
\chapter{2014}
\section{ce}
\author{AI24BTECH11023 - Tarun Reddy Pakala}
\fi
\item A box of weight $100\;kN$ shown in figure is to be lifted without swinging. If all forces are coplanar, the magnitude and direction $\brak{\theta}$ of the force $\brak{F}$ with respect to $x$-axis should be
	%input of figure 1
	\begin{figure}[!ht]
\centering
\resizebox{3cm}{3cm}{%
\begin{circuitikz}
\tikzstyle{every node}=[font=\normalsize]

\draw  (5.25,9.75) rectangle (9,8.25);
\draw [dashed] (7,9.75) -- (7,12);
\draw [dashed] (9,9.75) -- (10.25,9.75);
\draw [->, >=Stealth] (7,9.75) -- (5.25,11);
\draw [->, >=Stealth] (7,9.75) -- (8.75,11.5);
\draw [->, >=Stealth] (7,9.75) -- (9,10.5);
\node [font=\small] at (7,12.25) {y};
\node [font=\small] at (10.5,9.75) {x};
\node [font=\small] at (8.25,10) {$\theta$};
\node [font=\small] at (7.75,10.25) {$45\degree$};
\node [font=\small] at (6.5,10) {$30\degree$};
\node [font=\normalsize] at (7,9) {$100 kN$};
\node [font=\normalsize] at (4.5,11.25) {$90 kN$};
\node [font=\normalsize] at (9.25,11.75) {$40 kN$};
\node [font=\normalsize] at (9.25,10.5) {F};
\end{circuitikz}
}%

\label{fig:my_label}
\end{figure}

	\begin{enumerate}
    \item $F=56.389 kN$ and $\theta=28.28\degree$
    \item $F=-56.389 kN$ and $\theta=-28.28\degree$
    \item $F=9.055 kN$ and $\theta=1.414\degree$
    \item $F=-9.055 kN$ and $\theta=-1.414\degree$
\end{enumerate}
\item A particle moves along a curve whose parametric equations are: $x=t^3+2t$, $y=-3e^{-2t}$ and $z=2\sin{5t}$, where $x,\;y$ and $z$ show variations of the distance covered by the particle \brak{\text{in} \;cm} with time $t$ \brak{\text{in}\;s}. The magnitude of the acceleration of the particle  ({in}\;$\frac{cm}{s^2}$) at $t=0$ is $\underline{\hspace{2cm}}$
\item A traffic office imposes on an average $5$ number of penalties  daily on traffic violators. Assume that the number of penalties on different days is independent and follows a Poisson distribution. The probability that there will be less than $4$ penalties in a day is $\underline{\hspace{2cm}}$
\item Mathematical idealization of a crane has three had three bars with their vertices arranged as shown in the figure with a load of $80\;kN$ hanging vertically. The coordinates of the vertices are given in parentheses. The force in the member $QR$, $F_{QR}$ will be 
	%input of figure 2
	\begin{figure}[!ht]
\centering
\resizebox{3cm}{3cm}{%
\begin{circuitikz}
\tikzstyle{every node}=[font=\small]
\draw [short] (6,11.75) -- (7.75,7.25);
\draw [short] (6,11.75) -- (10.75,7.25);
\draw [short] (7.75,7.25) -- (10.75,7.25);
\draw [dashed] (6,10) -- (6,7.25);
\draw [dashed] (6,7.25) -- (7.75,7.25);
\draw [->, >=Stealth] (6,11.75) -- (6,10.5);
\node [font=\normalsize] at (6,12) {P(0,4)};
\node [font=\normalsize] at (7.75,7) {Q(1,0)};
\node [font=\normalsize] at (10.75,7) {R(3,0)};
\node [font=\small] at (9.75,7.5) {$53.13\degree$};
\node [font=\small] at (8,7.5) {$104.03\degree$};
\node [font=\small] at (6.75,11.5) {$22.84\degree$};
\node [font=\small] at (6,10.25) {80 kN};
\node [font=\small] at (6.5,7.5) {x};
\node [font=\small] at (5.75,9.75) {y};
\end{circuitikz}
}%

\label{fig:my_label}
\end{figure}

\begin{enumerate}
    \item $30\;kN$ Compressive
    \item $30\;kN$ Tensile
    \item $50\;kN$ Compressive
    \item $50\;kN$ Tensile
\end{enumerate}
\item For the cantilever beam of span $3\;m$ \brak{\text{shown below}}, a concentrated load of $20\;kN$ applied at the free end causes a vertical displacement of $2\;mm$ at a section located at a distance of $1\;m$ from the fixed end. If a concentrated vertically downward load of $10\; kN$ is applied at the section located at a distance of $1\;m$ from the fixed end \brak{\text{with no other load on the beam }}, the maximum vertical displacement in the same beam \brak{\text{in}\;mm} is \underline{\hspace{2cm}}
	%input for figure 3
	\begin{center}
\begin{tikzpicture}[>=latex]
    % Ground support (fixed end)
    \draw[pattern=north east lines] (-0.2,-0.5) rectangle (0,1);
    \draw (-0.5,0) -- (-0.2,0);
    
    % Main beam (solid line)
    \draw[thick] (0,0) -- (3,0);
    
    % Deflected beam (dashed line)
    \draw[dashed] (0,0) -- (1,-0.05) -- (3,-0.4);
    
    % Vertical dimensions and arrows
    \draw[<->] (1,0.3) -- (1,0.5) node[midway,right] {2 mm};
    
    % Force arrow and label
    \draw[thick,->] (3,0.5) -- (3,0) node[above] {20 kN};
    
    % Horizontal dimensions
    \draw[<->] (0,-0.8) -- (1,-0.8) node[midway,below] {1 m};
    \draw[<->] (1,-0.8) -- (3,-0.8) node[midway,below] {2 m};
    
    % Vertical marker at deflection point
    \draw[->] (1,-0.05) -- (1,0);
\end{tikzpicture}
\end{center}

\item For the truss shown below, the member $PQ$ is short by $3\;mm$. The magnitude of vertical displacement of joint $R\brak{\text{in} \;mm}$ is \underline{\hspace{2cm}}\\
	%input for figure 4
	\begin{tikzpicture}[scale=0.7]

% Define coordinates for the main points
\coordinate (P) at (0,0);
\coordinate (Q) at (8,0);
\coordinate (R) at (4,3);

% Draw the main triangle structure
\draw[line width=0.8pt] (P) -- (Q);
\draw[line width=0.8pt] (P) -- (R);
\draw[line width=0.8pt] (R) -- (Q);

% Draw the supports
% Left support (pin) - simplified to match original
\draw[line width=0.8pt] ($(P)+(-0.4,-0.1)$) -- ($(P)+(0.4,-0.1)$);
\draw[thick] ($(P)+(-0.4,-0.2)$) -- ($(P)+(0.4,-0.2)$);
\draw[thick] ($(P)+(-0.4,-0.3)$) -- ($(P)+(0.4,-0.3)$);

% Right support (roller) - simplified to match original
\draw[line width=0.8pt] ($(Q)+(-0.4,-0.3)$) -- ($(Q)+(0.4,-0.3)$);
\draw[thick] ($(Q)+(-0.4,-0.4)$) -- ($(Q)+(0.4,-0.4)$);
\draw[thick] ($(Q)+(-0.4,-0.5)$) -- ($(Q)+(0.4,-0.5)$);
\draw[thick] ($(Q)+(.2,-.2)$) circle (.1);
\draw[thick] ($(Q)+(-.2,-.2)$) circle (.1);

% Add dimensions
% Height dimension
\draw[<->] ($(P)+(-0.5,0)$) -- ($(P)+(-0.5,3)$);
\node[left] at ($(P)+(-0.5,1.5)$) {3 m};

% Width dimensions
\draw[<->] ($(P)+(0,-0.8)$) -- ($(P)+(4,-0.8)$);
\draw[<->] ($(P)+(4,-0.8)$) -- ($(Q)+(0,-0.8)$);
\node[below] at ($(P)+(2,-0.8)$) {4 m};
\node[below] at ($(P)+(6,-0.8)$) {4 m};

% Add labels for points
\node[anchor=east] at ($(P)+(-0.1,0.1)$) {P};
\node[anchor=west] at ($(Q)+(0.1,0.1)$) {Q};
\node[anchor=south] at ($(R)+(0,0.1)$) {R};

% Add small circles at joints
\fill (P) circle (1pt);
\fill (Q) circle (1pt);
\fill (R) circle (1pt);

\end{tikzpicture}



\item A rectangular beam of width \brak{b} $230\;mm$ and effective depth \brak{d} $450\;mm$ is reinforced with four bars of $12\;mm$ diameter. The grade of concrete is $M20$ and grade of steel is $Fe500$. Given that for $M20$ grade of concrete the ultimate shear strength, $\tau_{uc}=0.36\;\frac{N}{mm^2}$ for steel percentage, $p=0.25$, and $\tau_{uc}=0.48\;\frac{N}{mm^2}$ for $p=0.50$. For a factored shear force of $45\;kN$, the diameter \brak{\text{in}\;mm} of $Fe500$ steel two legged stirrups to be used at spacing of $375\;mm$, should be
\begin{enumerate}
    \item $8$
    \item $10$
    \item $12$
    \item $16$
\end{enumerate}
\item The tension and shear force \brak{\text{both in}\; kN} in the bolt of the joint, as shown below, respectively are
	% input for figure 5
	\begin{figure}[!ht]
\centering
\resizebox{3cm}{3cm}{%
\begin{circuitikz}
\tikzstyle{every node}=[font=\small]
\draw [short] (2,12.75) -- (2,8);
\draw [short] (2,12.75) -- (2,12);
\draw [short] (2.25,12.75) -- (2.25,8);
\draw [short] (2,12.75) -- (3.25,12.75);
\draw [short] (3.25,12.75) -- (3.25,13);
\draw [short] (3.25,13) -- (3.75,12.5);
\draw [short] (3.75,12.5) -- (3.75,12.75);
\draw [short] (3.75,12.75) -- (5,12.75);
\draw [short] (5,12.75) -- (5,8);
\draw [short] (5,12.75) -- (5.25,12.75);
\draw [short] (5.25,12.75) -- (5.25,8);
\draw [short] (2,8) -- (3.25,8);
\draw [short] (3.25,8) -- (3.25,7.75);
\draw [short] (3.25,7.75) -- (3.75,8.25);
\draw [short] (3.75,8.25) -- (3.75,8);
\draw [short] (3.75,8) -- (5.25,8);
\draw [dashed] (3.5,13.25) -- (3.5,7.25);
\draw [short] (5.5,11.25) -- (5.5,9.5);
\draw [short] (5.25,11.25) -- (6.5,11.25);
\draw [short] (6.5,11.25) -- (7.5,10.25);
\draw [short] (7.5,10.25) -- (7,9.25);
\draw [short] (5.25,9.5) -- (5.5,9.5);
\draw [short] (5.5,9.5) -- (7,9.25);
\draw [dashed] (5.75,10.5) -- (7.25,9.75);
\draw [->, >=Stealth] (7.25,9.75) -- (7.75,9.5);
\draw [dashed] (7.75,9.5) -- (8.25,9.25);
\draw [short] (9.25,12.75) -- (9.25,8);
\draw [short] (9.25,12.75) -- (10.5,12.75);
\draw [short] (10.5,12.75) -- (10.5,13);
\draw [short] (10.5,13) -- (11,12.25);
\draw [short] (11,12.25) -- (11,12.75);
\draw [short] (11,12.75) -- (12.25,12.75);
\draw [short] (12.25,12.75) -- (12.25,8);
\draw [short] (9.25,8) -- (10.5,8);
\draw [short] (10.5,8) -- (10.5,8.25);
\draw [short] (10.5,8.25) -- (11.25,7.75);
\draw [short] (11.25,7.75) -- (11.25,8);
\draw [short] (11.25,8) -- (12.25,8);
\draw [short] (9.75,11.25) -- (11.75,11.25);
\draw [short] (9.75,11.25) -- (9.75,9);
\draw [short] (10.75,11.25) -- (10.75,12.75);
\draw [short] (11,11.25) -- (11,12.25);
\draw [line width=0.2pt, short] (11.75,11.25) -- (11.75,9);
\draw [line width=0.2pt, short] (9.75,9) -- (11.75,9);
\draw [line width=0.2pt, dashed] (10.75,11.25) -- (10.75,9.25);
\draw [line width=0.2pt, dashed] (11,11.25) -- (11,9);
\draw [line width=0.2pt, dashed] (10.75,9.5) -- (10.75,9);
\draw [line width=0.2pt, short] (10.75,9) -- (10.75,8.25);
\draw [line width=0.2pt, short] (11,9) -- (11,8);
\draw [ line width=0.8pt ] (10,11) circle (0.25cm);
\draw [ line width=0.8pt ] (10,10.25) circle (0.25cm);
\draw [ line width=0.8pt ] (10,9.5) circle (0.25cm);
\draw [ line width=0.8pt ] (11.25,11) circle (0.25cm);
\draw [ line width=0.8pt ] (11.25,10.25) circle (0.25cm);
\draw [ line width=0.8pt ] (11.25,9.5) circle (0.25cm);
\draw [short] (6,10.5) -- (6.75,10.5);
\draw [short] (6.75,10.5) -- (6.75,10);
\node [font=\small] at (7,10.25) {3};
\node [font=\small] at (6.25,10.75) {4};
\node [font=\small] at (6.25,10) {5};
\node [font=\small] at (8,9.75) {$P_u=250\; kN$};
\end{circuitikz}
}%

\label{fig:my_label}
\end{figure}

\begin{enumerate}
    \item $30.33$ and $20.00$
    \item $30.33$ and $25.00$
    \item $33.33$ and $20.00$
    \item $33.33$ and $25.00$
\end{enumerate}
\item For a beam of cross-section, width=$230\;mm$ and effective depth=$500\;mm$, the number of rebars of $12\;mm$ diameter required to satisfy 
minimum tension reinforcement requirement specified by $IS:456-2000$ \brak{\text{assuming grade of steel reinforcement as Fe500}} is \underline{\hspace{2cm}}
\item In a reinforced concrete section, the stress at the extreme fiber in compression is $5.8\;MPa$. The depth of neutral axis in the section is $58\;mm$ and the grade of concrete is $M25$. Assuming linear elastic behavior of the concrete, the effective curvature of the section $\brak{\text{in per}\;mm} $
\begin{enumerate}
    \item $2.0 \times 10^{-6}$
    \item $3.0 \times 10^{-6}$
    \item $4.0 \times 10^{-6}$
    \item $5.0 \times 10^{-6}$
\end{enumerate}
\item Group $I$ contains representative load-settlement curves for different modes of bearing capacity failures of sandy soil. Group $II$ enlists the various failure characteristics. Match the load-settlement curves with the corresponding failure characteristics.   \\
	%input for figure 6
	\begin{figure}[!ht]
\centering
\resizebox{3cm}{3cm}{%
\begin{circuitikz}
\tikzstyle{every node}=[font=\small]
\draw [->, >=Stealth] (4.75,12.25) -- (9.25,12.25);
\draw [->, >=Stealth] (4.75,12.25) -- (4.75,8.25);
\draw [short] (4.75,12.25) -- (5.5,11.25);
\draw [short] (5.5,11.25) -- (5.5,9);
\draw [short] (4.75,12.25) -- (6.5,10.75);
\draw [short] (6.5,10.75) -- (7,10);
\draw [short] (7,10) -- (6.75,9);
\draw [short] (4.75,12.25) -- (7.25,11.25);
\draw [short] (7.25,11.25) -- (8.5,9.5);
\node [font=\small] at (9,11.75) {Load};
\node [font=\small] at (4.75,8) {Settlement};
\node [font=\small] at (5.25,9.25) {J};
\node [font=\small] at (7,9) {K};
\node [font=\small] at (8.75,9.75) {L};
\end{circuitikz}
}%

\label{fig:my_label}
\end{figure}

\begin{tabular}{ll}
    \textbf{Group I} & \textbf{Group II} \\
    \brak{P} Curve $J$          & \brak{i} No apparent heaving of soil around the footing         \\
    \brak{Q} Curve $K$         & \brak{ii} Rankine's passive zone develops imperfectly       \\
    \brak{R} Curve $L$        & \brak{iii} Well defined slip surface extends to ground surface    \\
\end{tabular}

\begin{enumerate}
    \item $P-1,\;Q-3,\;R-2$
    \item $P-3,\;Q-2,\;R-1$
    \item $P-3,\;Q-1,\;R-2$
    \item $P-1,\;Q-2,\;R-3$
\end{enumerate}
\item A given cohesionless soil has $e_{max}=0.85$ and $e_{min}=0.50$. In the field, the soil is compacted to a mass density of $1800\;\frac{kg}{m^3}$ at a water content of $8\%$. Take the mass density of water as $1000\;\frac{kg}{m^3}$ and $G_s$ as $2.7$. The relative density \brak{\text{in}\; \%} of the soil is
\begin{enumerate}
    \item $56.43$
    \item $60.25$
    \item $62.87$
    \item $65.71$
\end{enumerate}
\item The following data are given for the laboratory sample.\\
$\sigma_0'=175\;kPa;\:e_o=1.1;\:\sigma_o'+\Delta \sigma_o'=300\;kPa;\:e=0.9$\\
If the thickness of the clay specimen is $25\;mm$, the value of the coefficient of volume compressibility is \underline{\hspace{2cm}}$\times 10^{-4}\;\frac{m^2}{kN}$



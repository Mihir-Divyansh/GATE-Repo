\iffalse
\author{Manvik Muthyapu - AI24BTECH11021}
\section{ae}
\chapter{2015}
\fi
%27
\item An aircraft in level and unaccelerated flight with a velocity of $v_\infty = 300$ m/s requires a power of $9 \times 10^6 W$. If the aircraft weighs $1.5 \times 10^5 N$, the lift-to-drag ratio $\frac{L}{D}$ is \underline{\hspace{2cm}}. 

%28
\item The percentage change in the lift-off distance for a 20 $\%$ increase in aircraft weight is \underline{\hspace{2cm}}.

%29
\item Consider a monoplane wing and a biplane wing with identical airfoil sections, wingspans and incidence angles in identical conditions in a wind tunnel. As compared to the monoplane, the biplane experiences 

\begin{enumerate}
	\item a higher lift and a higher drag
	\item a higher lift and a lower drag
	\item a lower lift and a lower drag
	\item a lower lift and a higher drag
\end{enumerate}

%30
\item A statically stable trimmed aircraft experiences a gust and the angle of attack reduces momentarily. As a result, the center of pressure of the aircraft

\begin{enumerate}
	\item shifts forward
	\item shifts rearward
	\item does not shift
	\item coincides with the neutral point
\end{enumerate}

%31
\item Consider a wing of elliptic planform, with its aspect ratio $AR \rightarrow \infty$. Its lift-curve slope, $\frac{dC_{L}}{d\alpha}=$ \underline{\hspace{2cm}}.

%32
\item An ideal gas in a reservoir has a specific stagnation enthalpy of $h_0$. The gas is isentropically expanded to a new specific stagnation enthalpy of $\frac{h_{0}}{2}$ and velocity $u$. The flow is one-dimensional and steady. Then $\frac{u^2}{h_{0}}=$ \underline{\hspace{2cm}}.

%33
\item The Reynolds number, $Re$ is defined as $\frac{U_{\infty}L}{v}$ where $L$ is the length scale for a flow, $U_{\infty}$ is its reference velocity and $v$ is the coefficient of kinematic viscosity. In the laminar boundary layer approximation, comparison of the dimensions of the convection term $u\frac{\partial u}{\partial x}$ and the viscous term $v\frac{\partial^{2}u}{\partial x^{2}}$ leads to the following relation between the boundary layer thickness $\delta$ and $Re$

\begin{enumerate}
	\item $\delta \propto \sqrt{Re}$
	\item $\delta \propto 1/\sqrt{Re}$
	\item $\delta \propto Re$
	\item $\delta \propto 1/Re$
\end{enumerate}

%34
\item Isentropic efficiencies of an aircraft engine operating at typical subsonic cruise conditions with the following components - intake, compressor, turbine and nozzle - are denoted by $\eta _i, \eta _c, \eta _t$ and $\eta _n$, respectively. Which one of the following is correct?

\begin{enumerate}
	\item $\eta _i < \eta _c < \eta _t < \eta _n$
	\item $\eta _t < \eta _i < \eta _c < \eta _n$
	\item $\eta _c < \eta _t < \eta _i < \eta _n$
	\item $\eta _c < \eta _i < \eta _t < \eta _n$
\end{enumerate}

%35
\item A rocket nozzle is designed to produce maximum thrust at an altitude, $H = 8km$ from the sea level. The nozzle operates in

\begin{enumerate}
	\item under-expanded condition for $H>8km$
	\item under-expanded condition for $H<8km$
	\item sonic exit condition for $H<8km$
	\item unchoked condition for $H<8km$
\end{enumerate}

%36
\item In the solution of $\frac{d^2y}{dx^2}-2\frac{dy}{dx}+y=0$, if the values of the integration constants are identical and one of the initial conditions is specified as $y\brak{0}=1$, the other initial condition $y'(0)=$ \underline{\hspace{2cm}}.

%37
\item For $x>0$, the general solution of the differential equation $\frac{dy}{dx}=1-2y$ asymptotically approaches \underline{\hspace{2cm}}.

%38
\item For a parabola defined by $y=ax^2 + bx + c$, $a \neq 0$, the coordinates $\brak{x,y}$ of the extremum are 

\begin{enumerate}
	\item $\brak{ \frac{-b}{2a}+\frac{\sqrt{b^2-4ac}}{2a}, 0}$
	\item $\brak{ \frac{-b}{2a}, \frac{-b^2+4ac}{2a}}$
	\item $\brak{ \frac{-b}{2a}, \frac{-b^2+4ac}{4a}}$
	\item $\brak{0,c}$
\end{enumerate}

%39
\item The $2$-D stress state at a point $P$ in the $x$-$y$ coordinate system is $\begin{bmatrix} 60 & 50 \\ 50 & -40 \end{bmatrix} MPa$. The magnitude of the tangential stress $\brak{\text{in} MPa}$ on a surface normal to the $x$- axis at $P$ is \underline{\hspace{2cm}}.



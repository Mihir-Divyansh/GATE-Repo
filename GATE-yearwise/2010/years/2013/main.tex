\iffalse
\chapter{2013}
\author{AI24BTECH11032}
\section{ph}
\fi
\item A proton is confined to a cubic box, whose sides have length $10^{-12}m$. What is the minimum kinetic energy of the proton? The mass of proton is $1.67\times10^{-27}$kg and Planck's constant is
$6.63\times10^{-34}$Js.
\begin{multicols}{4}
    \begin{enumerate}
        \item $1.1\times10^{-17}$J
        \item $3.3\times10^{-17}$J
        \item $9.9\times10^{-17}$J
        \item $6.6\times10^{-17}$J
    \end{enumerate}
\end{multicols}
\bigskip
\item For the function $f\brak{\text{z}}=\frac{16\textbf{z}}{\brak{z+3}\brak{z-1}^{2}}$ the residue at the pole z = 1 is $\brak{\text{your answer should be an integer}}$
\bigskip
\item 2 The degenerate eigenvalue of the matrix\\ $\begin{bmatrix}
4 & -1 & -1 \\
-1 & 4 & -1 \\
-1 & -1 & 4 \\
\end{bmatrix}  \textbf{is} \brak{\text{your answer should be an integer}}$
\bigskip
\item Consider the decay of a pion into a muon and an anti-neutrino $\pi^{-}\to\mu^{-}+v_{\mu}^{-}$ in the pion rest frame $m_{\pi}=139.6\frac{MeV}{c^{2}},m_{\mu}=105.7\frac{MeV}{c^{2}},m_{v}\approx 0$ The energy \brak{\text{in MeV}} of the emitted neutrino, to the nearest integer is
\bigskip
\item In a constant magnetic field of 0.6 Tesla along the z direction, find the value of the path integral $\oint\overrightarrow{A}\cdot\overrightarrow{dl}$ in the units of on $\brak{\text{Tesla $m^{2}$}}$ a square loop of side length $\brak{\frac{1}{\sqrt{2}}}$ meters. The normal to the loop makes an angle of $60^\circ$ to the z-axis, as shown in the figure. The answer should be up to two decimal places.
\begin{figure}[H]
\centering
\resizebox{0.3\textwidth}{!}{%
\begin{circuitikz}
\tikzstyle{every node}=[font=\large]
\draw [->, >=Stealth] (1.25,7.5) -- (2.25,6);
\begin{scope}[rotate around={-90:(3.5,7.5)}]
\draw[domain=3.5:4.25,samples=100,smooth] plot (\x,{0.9*sin(1*\x r -3.5 r ) +7.5});
\end{scope}
\draw [->, >=Stealth] (2.5,5.5) -- (3,6);
\draw [short] (2.25,6) -- (2.5,5.5);
\draw [short] (1.25,7.5) -- (1.5,7.75);
\draw [->, >=Stealth] (2.25,8.5) -- (1.5,7.75);
\draw [->, >=Stealth] (3.5,6.5) -- (2.75,7.75);
\draw [short] (3,6) -- (3.5,6.5);
\draw [short] (2.25,8.5) -- (2.75,7.75);
\draw [dashed] (-1.25,6.75) -- (7.25,6.75);
\draw [dashed] (-1.25,5.5) -- (7.25,5.5);
\draw [dashed] (-1.25,4.5) -- (7.25,4.5);
\draw [dashed] (-1.25,8) -- (7.25,8);
\draw [dashed] (-1.25,9) -- (7.25,9);
\draw [->, >=Stealth] (7,9) -- (7.25,9);
\draw [->, >=Stealth] (7,8) -- (7.25,8);
\draw [->, >=Stealth] (7,6.75) -- (7.25,6.75);
\draw [->, >=Stealth] (6.75,5.5) -- (7.25,5.5);
\draw [->, >=Stealth] (6.75,4.5) -- (7.25,4.5);
\draw [->, >=Stealth] (2.75,7) -- (4,7.75);
\node [font=\large] at (4.25,7.25) {$60^\circ$};  % Use math mode for the degree symbol
\node [font=\large] at (6.5,7) {z};
\end{circuitikz}
}%
\label{fig:my_label}
\end{figure}


\bigskip
\item A spin-half particle is in a linear superposition $0.8 \lvert \uparrow \rangle + 0.6 \lvert \downarrow \rangle$  of its spin-up and spin-down states. If $\lvert \uparrow \rangle \quad \text{and} \quad \lvert \downarrow \rangle$ are the eigenstates of $\sigma_{z}$ then what is the expectation value, up to one decimal place, of the operator$10\sigma_{z}+5\sigma_{x}?$  Here,symbols have their usual meanings.
\bigskip
\item Consider the wave function $\text{A} e^{ikr} \brak{\frac{r_{0}}{r}}$
where A is the normalization constant. For r=2$r_{0}$ the
magnitude of probability current density up to two decimal places, in units of $\frac{A^{2}\hbar k}{m}$ is 
\bigskip
\item An n-channel junction field effect transistor has $5mA$ source to drain current at shorted gate $\brak{I_{DSS}}$ and $5V$ pinch off voltage $\brak{V_{P}}$. Calculate the drain current in mA for a gate-source voltage $\brak{V_{GS}}$ of $-2.5V.$ The answer should be up to two decimal places.
\bigskip
$$\textbf{Common Data Questions }$$
$\textbf{Common Data for Questions 48 and 49:}$
\bigskip
 There are four energy levels E, 2E, 3E and 4E $\brak{\text{where}E >0}$. The canonical partition function of two particles is , if these particles are
\item two identical fermions
\begin{enumerate}
    \item $e^{-2 \beta E}+e^{-4\beta E}+e^{-6\beta E}+e^{-8\beta E}$
    \item $e^{-3 \beta E}+e^{-4\beta E}+2e^{-5\beta E}+e^{-6\beta E}+e^{-7\beta E}$
    \item $\brak{e^{- \beta E}+e^{-2\beta E}+e^{-3\beta E}+e^{-4\beta E}}^{2}$
    \item $e^{-2 \beta E}-e^{-4\beta E}+e^{-6\beta E}-e^{-8\beta E}$
\end{enumerate}
\bigskip
\item two distinguishable particles 
\begin{enumerate}
    \item $e^{-2 \beta E}+e^{-4\beta E}+e^{-6\beta E}+e^{-8\beta E}$
    \item $e^{-3 \beta E}+e^{-4\beta E}+2e^{-5\beta E}+e^{-6\beta E}+e^{-7\beta E}$
    \item $\brak{e^{- \beta E}+e^{-2\beta E}+e^{-3\beta E}+e^{-4\beta E}}^{2}$
    \item $e^{-2 \beta E}-e^{-4\beta E}+e^{-6\beta E}-e^{-8\beta E}$
\end{enumerate}
$\textbf{Common Data for Questions 50 and 51: }$ To the given unperturbed Hamiltonian 
\begin{align*}
\begin{bmatrix}
5 & 2 & 0 \\
2 & 5 & 0 \\
0 & 0 & 2
\end{bmatrix}
\end{align*} we add a small perturbation given by 
\begin{align*}
\epsilon
    \begin{bmatrix}
1 & 1 & 1 \\
1 & 1 & -1 \\
1 & -1 & 1
\end{bmatrix}
\end{align*}, where $\epsilon$ is a small quantity
\bigskip
\item The ground state eigenvector of the unperturbed Hamiltonian is
\begin{multicols}{4}
    \begin{enumerate}
        \item $\brak{\frac{1}{\sqrt{2}},\frac{1}{\sqrt{2}},0}$
        \item $\brak{\frac{1}{\sqrt{2}},\frac{-1}{\sqrt{2}},0}$
        \item $\brak{0,0,1}$
        \item $\brak{1,0,0}$
    \end{enumerate}
\end{multicols}
\item A pair of eigenvalues of the perturbed Hamiltonian, using first order perturbation theory, is 
\begin{multicols}{4}
    \begin{enumerate}
        \item $3+2\epsilon+7+2\epsilon$
        \item $3+2\epsilon,2+\epsilon$
        \item 3,7+2$\epsilon$
        \item 3,2+2$\epsilon$
    \end{enumerate}
\end{multicols}
\bigskip
$$\textbf{Linked Answer Questions}$$
$\textbf{Statement for Linked Answer Questions 52 and 53: }$
\bigskip In the Schmidt model of nuclear magnetic moments,
we have, $\overrightarrow{\mu}=\frac{e\hbar}{2Mc}\brak{g_{l}\overrightarrow{l}+g_{s}\overrightarrow{S}}$ where the symbols have their usual meaning 
\bigskip
\item  For the case $j=l+\frac{1}{2}$ where J is the total angular momentum, the expectation value of $\overrightarrow{S}\cdot\overrightarrow{J}$ in
the nuclear ground state is equal to, 
\begin{multicols}{4}
    \begin{enumerate}
        \item $\frac{\brak{J-1}}{2}$
        \item $\frac{\brak{J+1}}{2}$
        \item $\frac{J}{2}$
        \item $\frac{-J}{2}$
    \end{enumerate}
\end{multicols}
\bigskip
\item For the $O^{17}$ nucleus $\brak{A=17,Z=8}$ the effective magnetic moment is given by,$\overrightarrow{\mu}=\frac{e\hbar}{2Mc} g\overrightarrow{J}$ where is equal to,$\brak{g_{s} = 5.59 \text{ for proton and } -3.83 \text{ for neutron}}$
\begin{multicols}{4}
    \begin{enumerate}
        \item $1.12$
        \item $0.77$
        \item $-1.28$
        \item $1.28$
    \end{enumerate}
\end{multicols}







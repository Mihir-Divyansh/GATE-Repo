\iffalse
\chapter{2015}
\author{AI24BTECH11032}
\section{ee}
\fi
    \item  A random variable X has probability density function f$\brak{x}$ as given below:
\begin{align*}
    f(x) = 
    \begin{cases} 
      a + b x & \text{if } 0 < x < 1, \\
      0 & \text{otherwise}
   \end{cases}
\end{align*}
If the expected value E$\sbrak{X}=\frac{2}{3},$ then Pr$\sbrak{X<0.5}$ is
\bigskip
\item If a continuous function f$\brak{x}$ does not have a root in the interval $\sbrak{a,b}$ then which one of the following statements is TRUE?
\begin{enumerate}
    \item $f\brak{a}\cdot f\brak{b}=0$
    \item $f\brak{a}\cdot f\brak{b}<0$
    \item $f\brak{a}\cdot f\brak{b}>0$
    \item $\frac{f\brak{a}} {f\brak{b}}\leq0$
\end{enumerate}
\bigskip
\item If the sum of the diagonal elements of a $2\times2$ matrix is $-6$, then the maximum possible value of determinant of the matrix is $\underline{\hspace{2cm}}.$
\bigskip
\item Consider a function $\overrightarrow{f}=\frac{1}{r^{2}}\hat{r}$ where r is the distance from the origin and $\hat{r}$ is the unit vector in the radial direction. The divergence of this function over a sphere of radius R, which includes the origin, is
\begin{multicols}{4}
    \begin{enumerate}
        \item $0$
        \item $2\pi$
        \item $4\pi$
        \item R$\pi$
    \end{enumerate}
\end{multicols}
\bigskip
\item When the Wheatstone bridge shown in the figure is used to find the value of resistor R$_{X}$, the galvanometer G indicates zero current when $R_{1}=50\ohm, R_{2}=65\ohm \text{ and } R_{3}=100\ohm. $ If R$_{3}$ is known with $\pm 5\%$ tolerance on its nominal value of $100\ohm,$ what is the range of R$_{X}$ in Ohms ?
\begin{figure}[H]
\centering
\resizebox{0.3\textwidth}{!}{%
\begin{circuitikz}
\tikzstyle{every node}=[font=\small]
\draw (2.25,11.75) to[short] (3.5,11.75);
\draw (2.25,11.75) to[short] (2.25,8);
\draw (3.5,11.75) to[R] (5.5,13.75);
\draw (5.5,13.75) to[R] (7.5,11.75);
\draw (3.5,11.75) to[R] (5.5,9.75);
\draw (5.5,9.75) to[R] (7.5,11.75);
\draw (7.5,11.75) to[short] (8.75,11.75);
\draw (8.75,11.75) to[short] (8.75,8);
\draw (2.25,8) to[battery1] (8.75,8);
\node [font=\small] at (5.25,8.25) {+};
\node [font=\small] at (5.75,8.25) {-};
\node [font=\small] at (5.5,8.75) {V};
\node [font=\small] at (4.25,13) {$R_1$};
\node [font=\small] at (4.25,10.5) {$R_3$};
\node [font=\small] at (6.75,13) {$R_2$};
\node [font=\small] at (6.75,10.5) {$R_x$};
\draw (5.5,13.75) to[short] (5.5,12);
\draw  (5.5,11.5) circle (0.5cm);
\draw (5.5,11) to[short] (5.5,9.75);
\node [font=\small] at (5.5,11.5) {$G$};
\end{circuitikz}
}%

\end{figure}
\begin{enumerate}
        \item $\sbrak{123.50,136.50}$
        \item $\sbrak{125.89,134.12}$
        \item $\sbrak{117.00,143.00}$
        \item $\sbrak{120.25,139.75}$
\end{enumerate}
\bigskip
\item A $\brak{0-50\text{A}}$ moving coil ammeter has a voltage drop of $0.1$V V across its terminals at full scale deflection. The external shunt resistance $\brak{\text{in milliohms}}$ needed to extend its range to $\brak{0-50\text{A}}$ is $\underline{\hspace{2cm}}.$
\bigskip
\item Of the four characteristics given below, which are the major requirements for an instrumentation amplifier? 
\begin{enumerate}[label=\Alph*]
    \item[P.] High common mode rejection ratio
    \item[Q.] High input impedance
    \item[R.] High linearity
    \item[S.] High output impedance
\end{enumerate}
\begin{multicols}{2}
    \begin{enumerate}
        \item P, Q and R only
        \item P and R only
        \item P, Q and S only
        \item Q, R and S only
    \end{enumerate}
\end{multicols}
\bigskip
\item In the following chopper, the duty ratio of switch S is $0.4$. If the inductor and capacitor are sufficiently large to ensure continuous inductor current and ripple free capacitor voltage, the charging current $\brak{\text{in Ampere}}$ of the $5$V battery, under steady-state,is $\underline{\hspace{2cm}}.$
\begin{figure}[H]
\centering
\resizebox{0.5\textwidth}{!}{%
\begin{circuitikz}
\tikzstyle{every node}=[font=\small]
\draw (1,12.75) to[american voltage source] (1,7);
\draw (1,12.75) to[short] (2.75,12.75);
\draw (2.75,12.75) to[short] (2.75,11.5);
\draw (3.25,12.75) to[short] (5.75,12.75);
\draw (3.75,12.75) to[short] (5.75,12.75);
\draw (4.25,12.75) to[short] (4.25,11.5);
\draw (2.5,11.5) to[short] (3,11.5);
\draw (4,11.5) to[short] (4.5,11.5);
\draw (2.5,11.25) to[short] (3.75,11.25);
\draw (4.25,11.25) to[short] (4.25,10.25);
\draw [->, >=Stealth] (3.5,12.75) -- (3.5,11.5);
\draw [short] (5,18.5) -- (5.25,18.5);
\draw (3.25,11.5) to[short] (3.75,11.5);
\draw (3.5,11.25) to[short] (4.25,11.25);
\draw (5.75,12.75) to[L ] (8,12.75);
\draw (8,12.75) to[short] (8.75,12.75);
\draw (1,7) to[short] (8.75,7);
\draw (8.75,7) to[short] (8.75,7.5);
\draw (8.75,12.75) to[R] (8.75,9.5);
\draw (8.75,9.5) to[battery1] (8.75,7.25);
\draw (7.5,12.75) to[C] (7.5,7);
\draw (5.5,7) to[D] (5.5,12.75);
\node [font=\small] at (9.5,11.25) {3\ohm};
\node [font=\small] at (9,8.75) {+};
\node [font=\small] at (9.25,8.25) {5V};
\node [font=\small] at (7,9.75) {C};
\node [font=\small] at (0.25,9.75) {20V};
\node [font=\small] at (3,13) {S};
\node [font=\small] at (6.75,13.25) {L};
\end{circuitikz}
}%

\end{figure}
\bigskip
\item A moving average function is given by $y\brak{t} = \frac{1}{T} \int_{t-T}^{t} u\brak{\tau} d\tau.$ If the input $\mu$ is a sinusoidal signal of frequency $\frac{1}{2T}$Hz,then in steady state, the output y will lag $\mu$ $\brak{\text{n degree}}$ by $\underline{\hspace{2cm}}.$
\bigskip
\item The impulse response g$\brak{\text{t}}$ of a system, G, is as shown in Figure $\brak{\text{a}}$. What is the maximum value attained by the impulse response of two cascaded blocks of G as shown in Figure $\brak{\text{b}}$ ?
\begin{figure}[!ht]
\centering
\resizebox{0.5\textwidth}{!}{%
\begin{circuitikz}
\tikzstyle{every node}=[font=\small]
\draw [->, >=Stealth] (4,7.5) -- (4,12.75);
\draw [->, >=Stealth] (3.75,8) -- (8.25,8);
\draw [short] (4,10) -- (6.25,10);
\draw [short] (6.25,10) -- (6.25,8);
\draw [->, >=Stealth] (3.75,10.75) -- (3.75,11);
\node [font=\small] at (3.25,10.5) {g$\brak{t}$};
\node [font=\small] at (3.75,10) {1};
\node [font=\small] at (3.75,7.75) {0};
\node [font=\small] at (6.25,7.75) {1};
\node [font=\small] at (8.5,8) {t};
\node [font=\small] at (5.75,7.25) {$\brak{a}$};
\draw  (3.75,5) rectangle (5.75,3.75);
\draw [->, >=Stealth] (2.5,4.25) -- (3.75,4.25);
\draw [->, >=Stealth] (5.75,4.25) -- (7,4.25);
\draw  (7,5) rectangle (9.5,3.75);
\draw [->, >=Stealth] (9.5,4.25) -- (10.75,4.25);
\node [font=\small] at (4.75,4.25) {G};
\node [font=\small] at (8.25,4.25) {G};
\node [font=\small] at (6.25,3.25) {$\brak{b}$};
\end{circuitikz}
}%


\end{figure}
\begin{multicols}{4}
    \begin{enumerate}
        \item $\frac{2}{3}$
        \item $\frac{3}{4}$
        \item $\frac{4}{5}$
        \item $1$
    \end{enumerate}
\end{multicols}
\bigskip
\item Consider a one-turn rectangular loop of wire placed in a uniform magnetic field as shown in the figure. The plane of the loop is perpendicular to the field lines. The resistance of the loop is $0.4\ohm,$ and its inductance is negligible. The magnetic flux density $\brak{\text{in Tesla}}$ is a function of time, and is given by B$\brak{t}=0.25\sin\omega t,$ where $\omega=2\pi\times50\frac{radian}{second}.$ The power absorbed $\brak{\text{n Watt}}$ by the loop from the magnetic field is $\underline{\hspace{2cm}}.$
\begin{figure}[!ht]
\centering
\resizebox{0.3\textwidth}{!}{%
\begin{circuitikz}
\tikzstyle{every node}=[font=\small]
\draw  (2.75,10.75) rectangle (8.75,7.5);
\draw (9,10.75) to[short] (11,10.75);
\draw (8.75,11) to[short] (8.75,12.5);
\draw (2.75,11) to[short] (2.75,12.5);
\draw [->, >=Stealth] (6,12.5) -- (8.75,12.5);
\draw [->, >=Stealth] (5.5,12.5) -- (2.75,12.5);
\draw [->, >=Stealth] (11,9) -- (11,7.5);
\draw [->, >=Stealth] (11,9.5) -- (11,10.75);
\draw [short] (9,7.5) -- (11,7.5);
\draw  (1.75,11.75) circle (0.25cm) node {\small X} ;
\draw  (1.75,10.25) circle (0.25cm);
\draw  (1.75,8.25) circle (0.25cm);
\draw  (1.75,6.75) circle (0.25cm);
\draw  (3.5,11.75) circle (0.25cm);
\draw  (5.5,11.75) circle (0.25cm);
\draw  (7.5,11.75) circle (0.25cm);
\draw  (3.5,10.25) circle (0.25cm);
\draw  (5.5,10.25) circle (0.25cm);
\draw  (7.5,10.25) circle (0.25cm);
\draw  (3.5,8.25) circle (0.25cm);
\draw  (5.5,8.25) circle (0.25cm);
\draw  (7.5,8.25) circle (0.25cm);
\draw  (3.75,6.75) circle (0.25cm);
\draw  (5.5,6.75) circle (0.25cm) node {\small X} ;
\draw  (7.5,6.75) circle (0.25cm);
\draw  (9.25,11.75) circle (0.25cm);
\draw  (9.25,10.25) circle (0.25cm);
\draw  (9.25,8.25) circle (0.25cm);
\draw  (9.25,6.75) circle (0.25cm);
\node [font=\normalsize] at (3.5,11.75) {X};
\node [font=\normalsize] at (5.5,11.75) {X};
\node [font=\normalsize] at (1.75,10.25) {X};
\node [font=\normalsize] at (3.5,10.25) {X};
\node [font=\normalsize] at (5.5,10.25) {X};
\node [font=\normalsize] at (1.75,8.25) {X};
\node [font=\normalsize] at (3.5,8.25) {X};
\node [font=\normalsize] at (5.5,8.25) {X};
\node [font=\normalsize] at (1.75,6.75) {X};
\node [font=\normalsize] at (3.75,6.75) {X};
\node [font=\small] at (7.5,11.75) {X};
\node [font=\small] at (7.5,10.25) {X};
\node [font=\small] at (7.5,8.25) {X};
\node [font=\small] at (7.5,6.75) {X};
\node [font=\small] at (9.25,11.75) {X};
\node [font=\small] at (9.25,10.25) {X};
\node [font=\small] at (9.25,8.25) {X};
\node [font=\small] at (9.25,6.75) {X};
\node [font=\small] at (5.75,12.5) {10 cm};
\node [font=\small] at (11,9.25) {5 cm};
\end{circuitikz}
}%

\end{figure}
\bigskip
\item A steady current I is flowing in the $-\text{x}$ direction through each of two infinitely long wires at $y = \pm \frac{L}{2}$ as shown in the figure. The permeability of the medium is $\mu_0$ . The $\overrightarrow{B}$-field at $\brak{0,L,0}$ is
\begin{figure}[H]
\centering
\resizebox{0.5\textwidth}{!}{%
\begin{circuitikz}
\tikzstyle{every node}=[font=\small]
\draw [->, >=Stealth] (6.25,9.25) -- (6.25,13.5);
\draw (6.25,9.25) to[lamp] (9,9.25);
\draw (6.25,9.25) to[lamp] (3.75,9.25);
\draw [->, >=Stealth] (9,9.25) -- (10.5,9.25);
\draw [->, >=Stealth] (3.75,9.25) -- (1.75,9.25);
\draw [<->, >=Stealth] (9.25,12.75) -- (3.25,5.75);
\node [font=\small] at (4.25,8.75) {Current=I};
\node [font=\small] at (8.25,8.75) {Current=I};
\node [font=\small] at (10.25,9.5) {y};
\node [font=\small] at (6,13) {z};
\node [font=\small] at (3.25,6.25) {x};
\node [font=\small] at (5,9.75) {{$y = -\frac{L}{2}$}};
\node [font=\small] at (8.25,9.75) {{$y = \frac{L}{2}$}};
\node [font=\small] at (6.25,9) {0};
\end{circuitikz}
}%

\end{figure}
\begin{multicols}{4}
    \begin{enumerate}
        \item $- \frac{4 \mu_0 I}{3 \pi L}\hat{Z}$
        \item $+ \frac{4 \mu_0 I}{3 \pi L}\hat{Z}$
        \item $0$
        \item $- \frac{3 \mu_0 I}{4 \pi L}\hat{Z}$
    \end{enumerate}
\end{multicols}
\bigskip
\item Consider the circuit shown in the figure. In this circuit R$=1 \text{k}\ohm,$ and C$=1 \mu \text{F}.$ The input voltage is sinusoidal with a frequency of $50$Hz,represented as a phasor with magnitude V$_{i}$ and phase angle $0$  radian as shown in the figure. The output voltage is represented as a phasor with magnitude V$_{o}$ and  phase angle $\delta$ radian . What is the value of the output phase angle $\delta\brak{\text{in radian}}$ relative to the phase
angle of the input voltage?
\begin{figure}[H]
\centering
\resizebox{0.3\textwidth}{!}{%
\begin{circuitikz}
\tikzstyle{every node}=[font=\small]
% Draw resistors and op-amp
\draw (3.75,11.5) to[R] (8.5,11.5);
\draw (3.75,11.5) to[short] (3.75,9.75);
\node at (3.75,9.75) [circ] {};
\draw (5.25,9.25) node[op amp,scale=1] (opamp2) {};
\draw (opamp2.+) to[short] (3.75,8.75);
\draw  (opamp2.-) to[short] (3.75,9.75);
\draw (6.45,9.25) to[short](6.75,9.25);
\draw (6.75,9.25) to[short, -o] (10.25,9.25);
\draw (8.5,11.5) to[short] (8.5,9.25);
\node at (8.5,9.25) [circ] {};
\draw (3.75,9.75) to[C] (2.75,9.75);
\draw (2.75,9.75) to[short, -o] (2.25,9.75);
\node at (3.75,8.75) [circ] {};
\draw (3.75,8.75) to[C] (2.75,8.75);
\draw (2.75,8.75) to[short, -o] (2.25,8.75);
\draw [->, >=Stealth] (2.25,9) -- (2.25,9.5);
\draw (3.75,8.75) to[R] (3.75,7);
\draw (3.75,7) to (3.75,6.75) node[ground]{};

% Labels
\node [font=\small] at (9.75,9.5) {$v_o = v_o \angle \delta$};
\node [font=\small] at (4.25,7.75) {$R$};
\node [font=\small] at (3.25,8.25) {$C$};
\node [font=\small] at (3.25,10.25) {$C$};
\node [font=\small] at (6,12) {$R$};
\node [font=\small] at (1.5,9.25) {$v_i = V_1 \angle \delta$};
\end{circuitikz}
} % End of resizebox
\caption{Circuit diagram with capacitors, resistors, and an op-amp}
\end{figure}

\begin{multicols}{4}
    \begin{enumerate}
        \item $0$
        \item $\pi$
        \item $\frac{\pi}{2}$
        \item $-\frac{\pi}{2}$
    \end{enumerate}
\end{multicols}



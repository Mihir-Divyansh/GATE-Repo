\iffalse
\chapter{2019}
\author{AI24BTECH11032}
\section{xe}
\fi
\item If the transformation u$\brak{x,t}=e^{x}v\brak{x,t}$ reduces the partial $$ \text{ differential equation }$$ $\frac{\partial^{2}u}{\partial x^{2}}-2\frac{\partial u}{\partial x}-\frac{\partial u}{\partial x}+u=9$  to the equation $\frac{\partial v}{\partial t}-\frac{\partial^{2}v}{\partial x^{2}}=9 \text{ f }\brak{x}$ then $\text{f}\brak{x}$ equals 

\begin{multicols}{4}
    \begin{enumerate}
        \item $-e^{-x}$
        \item $e^{-x}$
        \item $-2e^{-x}$
        \item $2e^{-x}$
    \end{enumerate}
\end{multicols}

\bigskip

%15
\item The value of a for which the system of equations 
\begin{align*}
    x-y-3z=3\\
    2x+z=0\\
    -2y-7z=\alpha
\end{align*}
has a solution is $\underline{\hspace{2cm}}.$
\bigskip

%16
\item The value of the line integral $\frac{2}{\pi}\oint_{\gamma}\brak{-y^{3}dx+x^{3}dy},$ where $\gamma$ is the circle $x^{2}+y^{2}=1$ oriented counter clockwise,is $\underline{\hspace{2cm}}.$
\bigskip

%17
\item Let $y_{1}\brak{x}\text{ and } y_{2}\brak{x}$  be two linearly independent solutions of the differential equation $x^{2}\frac{d^{2}y}{dx^{2}}+x\frac{dy}{dx}-4y=0,x>0\text{ If } y_{1}\brak{x}=x^{2},\text{ then }\lim_{x\to\infty}y_{2}\brak{x}$ is $\underline{\hspace{2cm}}.$
\bigskip

%18
\item If Q $= \myvec{3&2&4\\2&0&2\\4&2&3} $ and P $=\brak{v_{1},v_{2},v_{3}}$ is the matrix where $v_{1},v_{2}\text{ and }v_{3}$ are linearly independent eigenvector of the matrix Q, then the sum of the absolute values of all the elements of the matrix P$^{-1}$QP

\begin{multicols} {4}
    \begin{enumerate}
        \item $6$
        \item $10$
        \item $14$
        \item $22$
    \end{enumerate}
\end{multicols}
\bigskip

%19
\item If P$\brak{x}=ax^{3}+bx^{2}+cx+d$ is the polynomial obtained by Lagrange interpolation satisfying $P\brak{0}=-8,P\brak{1}=-7,P\brak{2}=-6\text{ and }P\brak{4}=20$ hen the value of $a+b+c$ is 

\begin{multicols} {4}
    \begin{enumerate}
        \item $1$
        \item $3$
        \item $5$
        \item $7$
    \end{enumerate}
\end{multicols}
\bigskip

%20
\item The number of critical points of the function f$\brak{x,y}=x^{3}+3xy^{2}-15x-12y$ at which there is neither maximum nor minimum is $\underline{\hspace{2cm}}.$
\bigskip

%21
\item Let I $=\frac{10^{5}i}{2\pi}\oint_{\gamma}\frac{dz}{\brak{z-4}\brak{z^{7}-1}},\text{ where } i = \sqrt{-1} \text{ and } \gamma$ is the circle $z=2$ oriented counter clockwise. Then, the value of I rounded off to one decimal place
\bigskip

%22
\item For stable equilibrium of a floating body, which one of the following statements is correct?

\begin{enumerate}
    \item Centre of gravity must be located below the centre of buoyancy.
     \item Centre of buoyancy must be located below the centre of gravity.
    \item Metacentre must be located below the centre of gravity.
     \item Centre of gravity must be located below the metacentre.
\end{enumerate}
\bigskip

%23
\item f u and v are the velocity components in the x- and y-directions respectively, the z-component of vorticity $\omega_{z}$ at a point in a flow field is

\begin{multicols} {4}
    \begin{enumerate}
        \item $\frac{\partial v}{\partial x}+\frac{\partial u}{\partial y}$
        \item $\frac{\partial v}{\partial x}-\frac{\partial u}{\partial y}$
        \item $\frac{\partial v}{\partial y}+\frac{\partial u}{\partial x}$
        \item$\frac{\partial v}{\partial y}-\frac{\partial u}{\partial x}$
    \end{enumerate}
\end{multicols}
\bigskip 

%24
\item In which one of the following devices the difference between static and total pressure is used to determine the flow velocity?

\begin{multicols} {4}
    \begin{enumerate}
     \item Piezometer
     \item Pitot static tube
    \item Orificemeter
    \item Venturimeter
    \end{enumerate}
\end{multicols}
\bigskip 

%25
\item A golf ball is dimpled to make the flow turbulent and consequently to reduce the drag. Turbulent flow reduces the drag on the golf ball because

\begin{enumerate}
    \item skin friction coefficient is lower in a turbulent flow.
    \item skin friction coefficient is higher in a turbulent flow.
    \item turbulent flow has a lower tendency to separate.
    \item turbulent flow has a higher tendency to separate.
\end{enumerate}
\bigskip

%26
\item For a steady laminar incompressible boundary layer flow over a sharp-edged flat plate at zero incidence,

\begin{enumerate}
     \item the edge of the boundary layer is a streamline.
    \item the edge of the boundary layer is a pathline.
    \item he skin friction coefficient decreases as the distance from the leading edge increases.
     \item the skin friction coefficient remains constant all along the plate.
\end{enumerate}


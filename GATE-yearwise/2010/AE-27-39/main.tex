\iffalse
\title{GATE Questions 5}
\author{EE24BTECH11012 - Bhavanisankar G S}
\section{ae}
\chapter{2010}
\fi
	\item Following stress state is proposed for a 2-D problem with no body forces:
		$$ \sigma_{xx} = 3x^2y + 4y^2 $$
		$$ \sigma_{yy} = -3xy^2 - 7x^2 $$
		It satisfies
		\begin{enumerate}
			\item Equilibrium equations but not compatibility equation
			\item Compatibility equation but not equilibrium equations
			\item Neither equilibrium equations nor compatibility equations
			\item Both equilibrium and compatibilit equations holds good.
		\end{enumerate}
	\item A uniform cross-section rigid rod of mass $m$ and length $l$ is hinged at its upper end and suspended like a pendulum. Its natural frequency for small oscillations is
		\begin{figure}[H]
			\centering
			\begin{circuitikz}
\tikzstyle{every node}=[font=\small]
\draw [dashed] (2.75,14.75) -- (2.75,12.25);
\draw [short] (2.25,14.75) -- (3.5,14.75);
\draw [short] (2.25,14.75) -- (2.5,15.25);
\draw [short] (2.5,14.75) -- (2.75,15.25);
\draw [short] (2.75,14.75) -- (3,15.25);
\draw [short] (3,14.75) -- (3.25,15.25);
\draw [short] (3.25,14.75) -- (3.5,15.25);
\draw [short] (2.75,14.75) -- (4.25,12.5);
\draw [short] (3,14.75) -- (4.5,12.5);
\draw [short] (4.25,12.5) -- (4.5,12.5);
\draw [<->, >=Stealth, dashed] (3.5,14.75) -- (4.75,12.5);
\node [font=\LARGE] at (4.5,13.75) {l};
\node [font=\small] at (3,13.75) {$\theta$};
\end{circuitikz}
			\caption{}
			\label{25}
		\end{figure}
		\begin{enumerate}
				\begin{multicols}{4}
			\item $\sqrt{\frac{g}{2l}} $
			\item $\sqrt{\frac{g}{l}} $
			\item $\sqrt{\frac{2g}{l}} $
			\item $\sqrt{\frac{3g}{2l}}$
				\end{multicols}
		\end{enumerate}
	\item The thin rectangular plate shown in the figure is loaded with uniform shear, $\tau$ along all edges and uniform uniaxial tension in the y-direction. The appropriate Airy's stress function to solve for stresses is given by
		\begin{figure}[H]
			\centering
			\begin{circuitikz}
\tikzstyle{every node}=[font=\small]
\draw  (3,13) rectangle (7.75,10.75);
\draw [->, >=Stealth] (3.5,13) -- (3.5,13.5);
\draw [->, >=Stealth] (3.75,13) -- (3.75,13.5);
\draw [->, >=Stealth] (4,13) -- (4,13.5);
\draw [->, >=Stealth] (4.5,13) -- (4.5,13.5);
\draw [->, >=Stealth] (5,13) -- (5,13.5);
\draw [->, >=Stealth] (5.5,13) -- (5.5,13.5);
\draw [->, >=Stealth] (6,13) -- (6,13.5);
\draw [->, >=Stealth] (6.5,13) -- (6.5,13.5);
\draw [->, >=Stealth] (7,13) -- (7,13.5);
\draw [->, >=Stealth] (7.5,13) -- (7.5,13.5);
\draw [->, >=Stealth] (3.5,10.75) -- (3.5,10.25);
\draw [->, >=Stealth] (4,10.75) -- (4,10.25);
\draw [->, >=Stealth] (4.5,10.75) -- (4.5,10.25);
\draw [->, >=Stealth] (5,10.75) -- (5,10.25);
\draw [->, >=Stealth] (5.5,10.75) -- (5.5,10.25);
\draw [->, >=Stealth] (6,10.75) -- (6,10.25);
\draw [->, >=Stealth] (6.5,10.75) -- (6.5,10.25);
\draw [->, >=Stealth] (7,10.75) -- (7,10.25);
\draw [->, >=Stealth] (7.5,10.75) -- (7.5,10.25);
\draw [->, >=Stealth] (2.75,13) -- (2.75,12.25);
\draw [->, >=Stealth] (2.75,11.75) -- (2.75,11);
\draw [->, >=Stealth] (8,11) -- (8,11.5);
\draw [->, >=Stealth] (8,12) -- (8,12.75);
\draw [->, >=Stealth] (3.25,13.25) -- (4.25,13.25);
\draw [->, >=Stealth] (4.5,13.25) -- (5.75,13.25);
\draw [->, >=Stealth] (6.25,13.25) -- (7.5,13.25);
\draw [->, >=Stealth] (7.5,10.5) -- (6.75,10.5);
\draw [->, >=Stealth] (6.5,10.5) -- (5.5,10.5);
\draw [->, >=Stealth] (5,10.5) -- (4.25,10.5);
\draw [->, >=Stealth] (4,10.5) -- (3.5,10.5);
\draw [->, >=Stealth, dashed] (5.25,12) -- (5.25,12.75);
\draw [->, >=Stealth, dashed] (5.25,12) -- (6.5,12);
\node [font=\small] at (6.75,11.75) {x};
\node [font=\small] at (4.75,12.75) {y};
\node [font=\small] at (5.25,14.25) {$\sigma_0$};
\node [font=\small] at (8.25,12) {$\tau$};
\end{circuitikz}
			\caption{}
			\label{25}
		\end{figure}
		\begin{enumerate}
				\begin{multicols}{2}
				\item $-\tau xy - \sigma \frac{x^2}{2} + \sigma \brak{x^4-y^4}$
				\item $-\tau xy + \sigma \frac{x^2}{2} + \sigma \brak{x^4-y^4}$
				\item $ \tau xy - \sigma \frac{x^2}{2} $
				\item $ -\tau xy + \sigma \frac{x^2}{2} $
				\end{multicols}
		\end{enumerate}
	\item A propeller powered aircraft, trimmed to attain maximum range and flying in a straight line, travels a distance R from its take-off point, when it has consumed a weight of fuel equal to 20 \% of its take-off weight. If the aircraft continues to fly and consumes a total weight of fuel equal to 50 \% of its take-off weight, the distance between it and its take-off point becomes
		\begin{enumerate}
				\begin{multicols}{4}
				\item 2.5R
				\item 3.1R
				\item 2.1R
				\item 3.9R
				\end{multicols}
		\end{enumerate}
	\item The given wall section of uniform thickess $t$ is symmetric about x-axis. Moment of inertia is given to be $I_{xx} = \frac{35}{12} th^3 $. Shear centre for this section is located at
		\begin{figure}[H]
			\centering
			\begin{circuitikz}
\tikzstyle{every node}=[font=\small]
\draw  (2,14.75) rectangle (2.25,9);
\draw  (2,14.5) rectangle (4.75,14.75);
\draw  (2,9) rectangle (4.75,9.25);
\draw  (2.25,13.25) rectangle (0.75,13.5);
\draw [->, >=Stealth] (2,14.5) -- (2,15.5);
\draw [->, >=Stealth] (0.75,12) -- (6.5,12);
\draw [<->, >=Stealth, dashed] (5.5,12) -- (5.5,14.75)node[pos=0.5, fill=white]{h};
\draw [<->, >=Stealth, dashed] (2,8.75) -- (4.75,8.75)node[pos=0.5, fill=white]{h};
\draw  (2.25,10.5) rectangle (0.75,10.75);
\draw [<->, >=Stealth, dashed] (0.75,12) -- (0.75,10.75)node[pos=0.5, fill=white]{h/2};
\draw [<->, >=Stealth, dashed] (0.75,10.25) -- (2.25,10.25)node[pos=0.5, fill=white]{h/2};
\node [font=\small] at (6.75,11.75) {\textbf{x}};
\node [font=\small] at (2.5,15.75) {\textbf{y}};
\end{circuitikz}
			\caption{}
			\label{25}
		\end{figure}
		\begin{enumerate}
				\begin{multicols}{4}
				\item x = $\frac{-3}{8}$h
				\item x = $\frac{-9}{28}$h
				\item x = $\frac{-35}{36}$h
				\item x = $\frac{-17}{35}$h
				\end{multicols}
		\end{enumerate}
	\item During an under-damped oscillation of a single-degree of freedom system, in the time-displacement plot the third peak is of magnitude 100 and the tenth peak is of magnitude 10. The damping ratio $\zeta$ is approximately
		\begin{enumerate}
				\begin{multicols}{4}
				\item 0.052
				\item 0.023
				\item 0.366
				\item 0.159
				\end{multicols}
		\end{enumerate}
	\item Given the Laplace transform of $y(t) = e^{-t}\brak{2cos(2t)-sin(2t)}$ is $Y(s) = \frac{2s}{(s+1)^2 + 4}$, the Laplace transform of $y(t) = e^t\brak{2cos(2t)-sin(2t)}$ is
		\begin{enumerate}
				\begin{multicols}{4}
				\item $\frac{2(s-2)}{(s-1)^2+4}$
				\item $\frac{2(s+2)}{(s+3)^2+4}$
				\item $\frac{2(s+2)}{(s+1)^2+4}$
				\item $\frac{2(s-1)}{(s-1)^2+4}$
				\end{multicols}
		\end{enumerate}
	\item In a certain region a hill is described by the shape $z(x,y) = \frac{1}{50}x^4 + y^2 - xy - 3y $, where the axes x and y ar in the horizontal plane and axis z points vertically upward. If $\vec{i}, \vec{j}, \vec{k}$ are unit vectors along x, y, z respectively, then at the point x=5, y=10 the unit vector in the direction of the steepest slope of the hill will be
		\begin{enumerate}
				\begin{multicols}{4}
				\item $\vec{i}$
				\item $\vec{j}$
				\item $\vec{k}$
				\item $\vec{i+j+k}$
				\end{multicols}
		\end{enumerate}
	\item An aircraft is cruising at an altitude of 9km. The free-stream static pressure and density at this altitude are $3.08 \times 10^4$ and $0.467 kg/m^2$ respectively. A Pitot tube mounted on the wing senses a pressure of $3.31 \times 10^4 N/m^2$. Ignoring compressibility effects, the cruising speed of the aircraft is approximately
		\begin{enumerate}
				\begin{multicols}{4}
				\item 50 m/s
				\item 100 m/s
				\item 150 m/s
				\item 200 m/s
				\end{multicols}
		\end{enumerate}
	\item The Irim curves of an aircraft are of the form $C_m = (0.05 - 0.2 \delta) - 0.1 C_l$, where the elevator deflection angle, $\delta_m$ is in radians. The static margin of the aircraft is
		\begin{enumerate}
				\begin{multicols}{4}
				\item 0.5
				\item 0.2
				\item 0.1
				\item 0.05
				\end{multicols}
		\end{enumerate}
	\item The function $f(x,y) = x^2 + y^2 - xy - 3y$ has an extremum at the point
		\begin{enumerate}
				\begin{multicols}{4}
				\item \brak{1,2}
				\item \brak{3,0}
				\item \brak{2,2}
				\item \brak{1,1}
				\end{multicols}
		\end{enumerate}
	\item Consider the flow of air $\brak{\rho = 1.23 kg/m^2}$ over a wing of chord of length 0.5m and span 3m. Let the free stream velocity be $U=100m/s$ and the average circulation around the wing be $\tau = 10 m^2/s$ per unit span. The lift force acting on the wing is
		\begin{enumerate}
				\begin{multicols}{4}
				\item 615 N
				\item 1845 N
				\item 3690 N
				\item 4920 N
				\end{multicols}
		\end{enumerate}
	\item The stagnation pressure and stagnation temperature inside the combustion chaber of a liquid rocket engine are 1.5 MPa and 2500 K respectively. The burned gases have $\gamma = 1.2$ and $R=692.83 J/kgK$. The rocket has a converging-diverging nozzle with a throat area of 0.025 $m^2$ and the flow at the exit of the nozzle is supersonic. If the flow through the nozzle is isentropic, what is the mass flow rate of the gases out of the nozzle ?
		\begin{enumerate}
				\begin{multicols}{4}
				\item 18.5 kg/s
				\item 31.2 kg/s
				\item 29.7 kg/s
				\item 19.4 kg/s
				\end{multicols}
		\end{enumerate}



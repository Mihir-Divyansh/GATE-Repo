\iffalse
\chapter{2023}
\author{AI24BTECH11009}
\section{xe}
\fi

\item Water (density = 1000 kg/$\text{m}^3$) flows steadily with a flow rate of 0.05 $\text{m}^3$/s through a venturimeter having throat diameter of 100 mm. If the pipe diameter is 200 mm and losses are negligible, the pressure drop (in kPa, $rounded\ off\ to\ one\ decimal\ place$) between an upstream location in the pipe and the throat (both at the same elevation) is $\_\_\_\_$. \\
\item Water flows around a thin flat plate (0.25 m long, 2 m wide) with a free stream velocity $\brak{U_{\infty}}$ of 1 m/s, as shown in the figure. Consider linear velocity profile $\brak{\frac{u}{U_{\infty}} = \frac{y}{\delta}}$ for which the laminar boundary layer thickness is expressed as $\delta = \frac{3.5x}{\sqrt{Re_x}}$. For water, density = 1000 kg/$\text{m}^3$ and dynamic viscosity = 0.001 kg/m.s. Net drag force (in N, $rounded\ off\ to\ two\ decimal\ places$) acting on the plate, neglecting the end effects, is $\_\_\_\_$. 
\begin{figure}[!ht]
\centering
\resizebox{0.7\textwidth}{!}{%
\input{figs/2023-XE/Q2.tex}
}%
\end{figure} \\
\item Axial velocity profile $u\brak{r}$ for an axisymmetric flow through a circular tube of radius $R$ is given as,
\begin{align*}
    \frac{u\brak{r}}{U} = \brak{1 - \frac{r}{R}}^{\frac{1}{n}}
\end{align*}
where $U$ is the centerline velocity. If $V$ refers to the area-averaged velocity (volume flow rate per unit area), then the ratio $\frac{V}{U}$ for $n = 1$ \brak{rounded\ off\ to\ two\ decimal\ places} is $\_\_\_\_$. \\
\item A stationary circular pipe of radius $R = 0.5$ m is half filled withwater (density = 1000 kg/$\text{m}^3$), whereas the upper half is filled with air at atmospheric pressure, as shown in the figure. Acceleration due to gravity is $g$ = 9.81 m/$\text{s}^2$. The magnitude of the force per unit length (in kN/m, $rounded\ off\ to\ one\ decimal\ place$) applied by water on the pipe section AB is $\_\_\_\_$.
\begin{figure}[!ht]
\centering
\resizebox{0.5\textwidth}{!}{%
\input{figs/2023-XE/Q4.tex}
}%
\end{figure} \\
\item In age-hardening of an aluminium alloy, the purpose of solution treatment
followed by quenching is to
\begin{enumerate}
    \item form martensitic structure
    \item increase the size of the precipitates
    \item form supersaturated solid solution
    \item form precipitates at the grain boundaries \\
\end{enumerate}
\item The magnetization (M) - magnetic field (H) curves for four different materials are given below. Which one of these materials is most suitable for use as a permanent magnet?
\begin{enumerate}
    \item \resizebox{0.4\textwidth}{!}{%
\input{figs/2023-XE/A6.1.tex}
}%
\item \resizebox{0.4\textwidth}{!}{%
\input{figs/2023-XE/A6.2.tex}
}%
\item \resizebox{0.4\textwidth}{!}{%
\input{figs/2023-XE/A6.3.tex}
}%
\item \resizebox{0.4\textwidth}{!}{%
\input{figs/2023-XE/A6.4.tex}
}%
\\
\end{enumerate}
\item The band gap of a semiconducting material is $\sim$ 2 eV. Which one of the following absorption (A) vs. energy (in eV) curves is correct ?
\begin{enumerate}
    \item \resizebox{0.4\textwidth}{!}{%
\input{figs/2023-XE/A7.1.tex}
}%
\item \resizebox{0.4\textwidth}{!}{%
\input{figs/2023-XE/A7.2.tex}
}%
\item \resizebox{0.4\textwidth}{!}{%
\input{figs/2023-XE/A7.3.tex}
}%
\item \resizebox{0.4\textwidth}{!}{%
\input{figs/2023-XE/A7.4.tex}
}%
\\
\end{enumerate}
\item Figures (i) and (ii) show a binary phase diagram and the corresponding Gibbs free energy \brak{G} vs. composition \brak{X_B} diagram, respectively. Figure (ii) corresponds to which one of the temperatures shown in Figure (i)?
\begin{figure}[!ht]
\centering
\resizebox{0.9\textwidth}{!}{%
\input{figs/2023-XE/Q8.tex}
}%
\end{figure}
\begin{enumerate}
    \item T1
    \item T2
    \item T3
    \item T4 \\
\end{enumerate}
\item Aliovalent doping of $MgCl_2$ in $NaCl$ leads to the formation of defects. Which one of the following is the correct defect reaction ?
\begin{enumerate}
    \item $Mg^{\bullet}_{Cl} + Na_{Na} + V'_{Cl} = \varnothing$
    \item $Mg^{\bullet}_{Na} + Cl_{Cl} + V'_{Na} = \varnothing$
    \item $Mg_{Na} + Cl_{Cl} = \varnothing$
    \item $Mg'_{Na} + Cl_{Cl} + V^{\bullet}_{Na} = \varnothing$ \\
\end{enumerate}
\item A screw dislocation in a FCC crystal has Burgers vector of $\frac{a}{2}\sbrak{1 1 0}$, where $a$ is the lattice constant. The possible slip plane(s) is/are:
\begin{enumerate}
    \item $\brak{1 1 \bar{1}}$
    \item $\brak{1 1 1}$
    \item $\brak{\bar{1} 1 1}$
    \item $\brak{1 \bar{1} 1}$ \\
\end{enumerate}
\item The tensile true stress $\brak{\sigma}$ - true strain $\brak{\epsilon}$ curve follows the Hollomon equation:
\begin{align*}
    \sigma = 500\epsilon^{0.15}\ \text{MPa}
\end{align*}
At the maximum load, the work-hardening rate $\brak{\frac{d\sigma}{d\epsilon}}$ is (in MPa): $\_\_\_\_$ (rounded off to nearest integer) \\
\item A metal has a certain vacancy fraction at a temperature of 600 K. On increasing the temperature to 900 K, the vacancy fraction increases by a factor of $\_\_\_\_$ (rounded off to one decimal place) \\\\
Given: Gas constant, R = 8.31 J $\text{mol}^{-1}\text{K}^{-1}$ and activation energy for vacancy formation, Q = 68 kJ $\text{mol}^{-1}$ \\
\item In a semiconductor, the ratio of electronic mobility to hole mobility is 10. The density of electrons and holes are $10^{15} m^{-3}$ and $10^{16} m^{-3}$, respectively. If the conductivity of the material is 1.6 $\Omega^{-1} m^{-1}$, then the mobility of holes is (in $m^2V^{-1}s^{-1}$) : $\_\_\_\_$ (rounded off to nearest integer) \\\\
Given: Charge of an electron: $1.6 \times 10^{-19}\ C$ \\

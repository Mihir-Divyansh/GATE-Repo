\iffalse
\chapter{2023}
\author{EE24BTECH11037}
\section{st}
\fi

%\begin{enumerate}
    \item Suppose that $X_{1}, X_{2}, ..., X_{n}$ are independent and identically distributed random variables each having probability density function $f(\cdot)$ and median  $\theta.$ We want to test\\

$H_{0}: \theta = \theta_{0}$ against $H_{1}: \theta > \theta_{0}.$\\

Consider a test that rejects $H_{0}$ if $S > c$ for some $c$ depending on the size of the test, where $S$ is the cardinality of the set $i: X_{i} > \theta_{0}, 1 \leq i \leq n$. Then which one of the following statements is true?
    \begin{enumerate}
        \item Under $H_o$, the distribution of $S$ depends on $f\brak{\cdot}$
        \item Under $H_1$, the distribution of $S$ does not depend on $f\brak{\cdot}$
        \item The power of function depends on $\theta$
        \item The power of function does not depend on $\theta$
    \end{enumerate}
    \item Suppose that $x$ is an observed sample of size $1$ from a population with probability density function $f\brak{\cdot}$. Based on $x$, consider testing\\$H_o:f\brak{y}=\frac{1}{\sqrt{2\pi}}e^{\frac{-y^2}{2}}$; $y\in R$ against $H_1:f\brak{y}=\frac{1}{2}e^{-|y|}$;$y\in R$
    \\Then which of the following statements is true?
    \begin{enumerate}
        \item The most powerful test rejects $H_o$ if $|x|>c$ for some $c>0$
        \item The most powerful test rejects $H_o$ if $|x|<c$ for some $c>0$
        \item The most powerful test rejects $H_o$ if $||x|-1|>c$ for some $c>0$
        \item The most powerful test rejects $H_o$ if $||x|-1|<c$ for some $c>0$
    \end{enumerate}
    \item Let $f:f : \mathbb{R}^2 \to \mathbb{R}$ be defined by $f\brak{x,y}=xy$. Then the maximum value(rounded off to two decimal places) of $f$ on the ellipse $x^2+2y^2=1$ equals \underline{\hspace{1.5cm}}.
    \item Let $A$ be a $2x2$ real matrix such that $AB=BA$ for all $2x2$ real matrices $B$. If trace of $A$ equals $5$, then the determinant $A$(rounded off to two decimal places) equals \underline{\hspace{1.5cm}}.
    
    \item Two defective bulbs are present in a set of five bulbs. To remove the two defective bulbs, the bulbs are chosen randomly one by one and tested. If $X$ denotes the minimum number of bulbs that must be tested to find out the two defective bulbs, then $P\brak{X=3}$ (rounded off to two decimal places) equals \underline{\hspace{1.5cm}}.
    \item Let $\{X_n\}_{n \geq 1}$ be a sequence of independent and identically distributed random
variables each having mean $4$ and variance $9$. If  $Y_n = \frac{1}{n} \sum_{i=1}^{n} X_i$ for $n \geq 1$, then $\lim_{n \to \infty} E\left[ \frac{(Y_n - 4)^2}{(\sqrt{n})^2} \right]$ (in integer) equals  \underline{\hspace{2cm}}.

    \item Let $\{W_t\}_{t\geq 0}$ be a standard Brownian motion. Then $E\brak{W_4^2|W_2=2}$ (in integer) equals \underline{\hspace{2cm}}.
    \item Let $\{x_n\}_{n\geq 1}$ be a Markov chain with state space $\{1,2,3\}$ and transition probability matrix\\
    $\displaystyle \begin{pmatrix}
    \frac{1}{2} \frac{1}{4} \frac{1}{4}\\
    \frac{1}{3} \frac{1}{3} \frac{1}{3}\\
    0 \frac{1}{2} \frac{1}{2}
      \end{pmatrix}$\\ Then $P\brak{X_2=1|X_1=1,X_3=2}$ (rounded off to two decimal places) equals \underline{\hspace{2cm}}. 
    \item Suppose that $\brak{X_1,X_2,X_3}$ has $N_3\brak{\mu,\sum}$ distribution with $\mu = \begin{pmatrix}
        0\\0\\0
    \end{pmatrix}$ and $\sum=\begin{pmatrix}
        2 2 1\\2 5 1\\1 1 1
    \end{pmatrix}$. Given that $\phi\brak{-0.5}=0.3085$, where $\phi\brak{\cdot}$ denotes the cumulative distribution function of a standard normal random variable, $P\brak{\brak{X_1-2X_2+2X+3}^2\leq \frac{7}{2}}$ (rounded off to two decimal places) equals \underline{\hspace{2cm}}.  
    \item Let $A$ be an $nxn$ real matrix. Consider the following statements.\\
    $\brak{I}$ If $A$ is symmetric, then there exists $c\geq 0$ such that $A+cI_n$ is symmetric and positive definite, where $I_n$ is the $nxn$ identity matrix.\\
    $\brak{II}$ If $A$ is symmetric and positive definite, then there exists a symmetric and positive definite matrix $B$ such that $A=B^2$.\\Which of the above statements is/are true?
    \begin{enumerate}
        \item Only $\brak{I}$
        \item Only $\brak{II}$
        \item Both $\brak{I}$ and $\brak{II}$
        \item Neither $\brak{I}$ nor $\brak{II}$
    \end{enumerate} 
    \item Let $X$ be a random variable with probability density function\\$f\brak{x}=\begin{cases}
        \frac{1}{x^2}, & x\geq 1\\ 0 & \text{otherwise},
    \end{cases}$\\
      If $Y=log_e X$, then $P\brak{Y<1|Y<2}$ equals  
    \begin{enumerate}
        \item $\frac{e}{e+1}$
        \item $\frac{e-1}{e+1}$
        \item $\frac{1}{e+1}$ 
        \item $\frac{1}{e-1}$
    \end{enumerate}
    \item Let $\{N\brak{t}\}_{t\geq 0}$ be a Poisson process with rate $1$. Consider the following statements.\\$\brak{I}$ $P\brak{N\brak{3}=3|N\brak{5}=5}=\frac{5}{3}\frac{3}{5}^2\frac{2}{5}^2$.\\ $\brak{II}$ If $S_5$ denotes the time of occurrence of the $5^{th}$ event for the above Poisson process, then $E\brak{S_5|N\brak{5}=3}=7$.\\Which of the above statements is/are true?
    \begin{enumerate}
        \item Only $\brak{I}$
        \item Only $\brak{II}$
        \item Both $\brak{I}$ and $\brak{II}$
        \item Neither $\brak{I}$ nor $\brak{II}$
    \end{enumerate}
    \item Let $X_1, X_2, ..., X_n$ be random sample of size $n$ from a population having probability density function\\$f\brak{x;\mu}=\begin{cases} e^{-(x-\mu)} & \text{if } \mu \leq x < \infty \\ 0 & \text{otherwise,} \end{cases}$\\ where $\mu \in \mathbb{R}$ is an unknown parameter. If $\hat{M}$ is the maximum likelihood estimator of the median of $X_1$, then which one of the following statements is true?
\begin{enumerate}
    \item $P\brak{\hat{M}\leq 2}=1-e^{-n\brak{1-log_e2}}$ if $\mu=1$
    \item $P\brak{\hat{M}\leq 1}=1-e^{-nlog_e2}$ if $\mu=1$
    \item $P\brak{\hat{M}\leq 3}=1-e^{-n\brak{1-log_e2}}$ if $\mu=1$
    \item $P\brak{\hat{M}\leq 4}=1-e^{-n\brak{2log_e2-1}}$ if $\mu=1$
\end{enumerate}
%\end{enumerate}

  

 \iffalse
    \title{Assignment}
    \author{EE24BTECH11035}
    \section{ph}
    \chapter{2023}
  \fi
\item ``You are delaying the completion of the task. Send \_\_\_\_\_ contributions at the earliest.''
  \begin{enumerate}
    \item you are
    \item your
    \item you're
    \item yore
  \end{enumerate}

\item References : \_\_\_\_\_ :: Guidelines : Implement (By word meaning)
  \begin{enumerate}
    \item Sight
    \item Site
    \item Cite
    \item Plagiarise
  \end{enumerate}
 \item In the given fgure, PQRS is a parallelogram with PS = 7 cm, PT = 4cm and PV = 5cm . What is the length of RS in cm ? (The diagram is representative.)
 \begin{enumerate}
     \item $\frac{20}{7}$
     \item $\frac{28}{5}$
     \item $\frac{9}{2}$
     \item $\frac{35}{4}$
 \end{enumerate}
 
 \item In 2022, June Huh was awarded the Fields medal, which is the highest prize in
Mathematics.
When he was younger, he was also a poet. He did not win any medals in the
International Mathematics Olympiads. He dropped out of college.
Based only on the above information, which one of the following statements can be logically inferred with certainity?
\begin{enumerate}
 \item Every Fields medalist has won a medal in an International Mathematics Olympiad.
\item Everyone who has dropped out of college has won the Fields medal.
\item All Fields medalists are part-time poets.
\item Some Fields medalists have dropped out of college.
\end{enumerate}
\item A line of symmetry is defined as a line that divides a figure into two parts in a way such that each part is a mirror image of the other part about that line. \\
The given figure consists of 16 unit squares arranged as shown. In addition to the three black squares, what is the minimum number of squares that must be colored black, such that both $PQ$ and $MN$ form lines of symmetry? (The figure is representative)

\begin{enumerate}
    \item 3
    \item 4
    \item 5
    \item 6
\end{enumerate}
\item Human beings are one among many creatures that inhabit an imagined world. In this imagined world, some creatures are cruel. If in this imagined world, it is given that the statement ``Some human beings are not cruel creatures'' is \textbf{FALSE}, then which of the following set of statement(s) can be logically inferred with \textit{certainty}?
  \begin{enumerate}
    \item[(i)] All human beings are cruel creatures.
    \item[(ii)] Some human beings are cruel creatures.
    \item[(iii)] Some creatures that are cruel are human beings.
    \item[(iv)] No human beings are cruel creatures.
  \end{enumerate}

  \begin{enumerate}
    \item only (i)
    \item only (iii) and (iv)
    \item only (i) and (ii)
    \item (i), (ii) and (iii)
  \end{enumerate}

\item To construct a wall, sand and cement are mixed in the ratio of 3:1. The cost of sand and that of cement are in the ratio of 1:2. If the total cost of sand and cement to construct the wall is 1000 rupees, then what is the cost (in rupees) of cement used?
  \begin{enumerate}
    \item 400
    \item 600
    \item 800
    \item 200
  \end{enumerate}
\item The World Bank has declared that it does not plan to offer new financing to Sri
Lanka, which is battling its worst economic crisis in decades, until the country has
an adequate macroeconomic policy framework in place. In a statement, the World
Bank said Sri Lanka needed to adopt structural reforms that focus on economic
stabilisation and tackle the root causes of its crisis. The latter has starved it of
foreign exchange and led to shortages of food, fuel, and medicines. The bank is
repurposing resources under existing loans to help alleviate shortages of essential
items such as medicine, cooking gas, fertiliser, meals for children, and cash for
vulnerable households.\\

Based only on the above passage, which one of the following statements can be
inferred with certainty?
\begin{enumerate}
    \item According to the World Bank, the root cause of Sri Lanka's economic crisis is that
it does not have enough foreign exchange.
  \item The World Bank has stated that it will advise the Sri Lankan government about how
to tackle the root causes of its economic crisis.
  \item According to the World Bank, Sri Lanka does not yet have an adequate
macroeconomic policy framework.
  \item The World Bank has stated that it will provide Sri Lanka with additional funds for
essentials such as food, fuel, and medicines.
\end{enumerate}
\item The coefficient of $x^4$ in the polynomial $(x-1)^3$$(x-2)^3$ is equal to $\dots$.
\begin{enumerate}
    \item $33$
    \item $-3$
    \item $30$
    \item $21$
\end{enumerate}
\item Which one of the following shapes can be used to tile (completely cover by
repeating) a flat plane, extending to infinity in all directions, without leaving any
empty spaces in between them? The copies of the shape used to tile are identical
and are not allowed to overlap.
\begin{enumerate}
    \item circle
    \item regular octagon
    \item regular pentagon
    \item rhombus
\end{enumerate}
\item Which one of the following entropy (S) - temperature (T) diagrams CORRECTLY
represents the Carnot cycle $abcda$ shown \textbf{P-V} in the diagram?


% Main figure for the question
\begin{figure}[!ht]
\centering
\resizebox{0.2\textwidth}{!}{%
\begin{circuitikz}
    % Define the style for nodes
    \tikzstyle{every node}=[font=\normalsize]

    % Drawing the main structure
    \draw [line width=0.7pt] (8.25,15.75) -- (8.25,12.5); % Vertical line
    \draw [line width=0.7pt] (8.25,12.5) -- (12,12.5);    % Horizontal line

    % Drawing arrows
    \draw [->, >=Stealth] (9.25,14.25) -- (9.25,15.5);   % Up arrow
    \draw [->, >=Stealth] (10.75,13.75) -- (10.75,15);   % Another up arrow
    \draw [->, >=Stealth] (10.75,13.75) .. controls (10,13.5) and (9.75,13.75) .. (9.25,14.25); % Curved arrow
    \draw [->, >=Stealth] (9.25,15.5) .. controls (9.75,15) and (10,14.75) .. (10.75,15); % Another curved arrow

    % Adding labels
    \node [font=\large] at (7.75,15) {P};          % Label for P
    \node [font=\large] at (11.5,12.25) {V};       % Label for V
    \node [font=\large] at (10,15.5) {$T_2$};      % Label for T2
    \node [font=\large] at (9.75,13.5) {$T_1$};    % Label for T1
    \node [font=\normalsize] at (9,15.5) {a};      % Label for a
    \node [font=\normalsize] at (11,15.25) {b};    % Label for b
    \node [font=\normalsize] at (11,13.75) {c};    % Label for c
    \node [font=\normalsize] at (9,14.25) {d};     % Label for d
\end{circuitikz}
}
\label{fig:main_diagram}
\end{figure}


% Multicol setup for options
\begin{multicols}{2}
\begin{enumerate}

    % Option (a)
    \item
    \begin{center}
    \resizebox{0.4\linewidth}{!}{%
    \begin{circuitikz}
        % Diagram for Option (a)
        \draw[thick,->] (0,0) -- (5,0) node[below] {$T$};
        \draw[thick,->] (0,0) -- (0,4) node[left] {$S$};
        \draw[dashed] (2,0) -- (2,3);
        \draw[dashed] (4,0) -- (4,3);
        \draw[->,>=stealth] (2,3) -- (4,3) node[midway, above] {b};
        \draw[->,>=stealth] (4,1) -- (2,1) node[midway, above] {c};
        \draw[->,>=stealth] (2,1) -- (2,3) node[midway, left] {d};
        \draw[->,>=stealth] (4,3) -- (4,1) node[midway, right] {a};
    \end{circuitikz}
    }
    \end{center}

    % Option (b)
    \item
    \begin{center}
    \resizebox{0.4\linewidth}{!}{%
    \begin{circuitikz}
        % Diagram for Option (b)
        \draw[thick,->] (0,0) -- (5,0) node[below] {$T$};
        \draw[thick,->] (0,0) -- (0,4) node[left] {$S$};
        \draw[dashed] (2,0) -- (2,2);
        \draw[dashed] (4,0) -- (4,2);
        \draw[->,thick] (2,2) -- (4,2) node[midway, above] {d};
        \draw[->,thick] (4,1) -- (2,1) node[midway, below] {b};
        \draw[->,thick] (2,1) -- (2,2) node[midway, left] {c};
        \draw[->,thick] (4,2) -- (4,1) node[midway, right] {a};
    \end{circuitikz}
    }
    \end{center}

    % Option (c)
    \item
    \begin{center}
    \resizebox{0.4\linewidth}{!}{%
    \begin{circuitikz}
        % Diagram for Option (c)
        \draw [ line width=0.7pt](8.25,15.75) to[short] (8.25,12.5);
        \draw [ line width=0.7pt](8.25,12.5) to[short] (12,12.5);
        \node [font=\large] at (7.75,15) {S};
        \node [font=\large] at (11.5,12.25) {T};
        \draw [->, >=Stealth] (9,15) -- (9,13.75);
        \draw [->, >=Stealth] (10.75,13.25) -- (10.75,14.5);
        \draw [->, >=Stealth] (10.75,14.5) -- (9,15);
        \draw [->, >=Stealth] (9,13.75) -- (10.75,13.25);
        \node [font=\normalsize] at (11,13.25) {a};
        \node [font=\normalsize] at (11,14.75) {b};
        \node [font=\normalsize] at (8.75,15.25) {c};
        \node [font=\normalsize] at (8.75,13.75) {d};
        \draw [dashed] (9,13.75) -- (9,12.5);
        \draw [dashed] (10.75,13.25) -- (10.75,12.5);
        \node [font=\large] at (9,12.25) {$T_1$};
        \node [font=\large] at (10.75,12.25) {$T_2$};
    \end{circuitikz}
    }
    \end{center}

    % Option (d)
    \item
    \begin{center}
    \resizebox{0.4\linewidth}{!}{%
    \begin{circuitikz}
        % Diagram for Option (d)
        \draw [ line width=0.7pt](8.25,15.75) to[short] (8.25,12.5);
        \draw [ line width=0.7pt](8.25,12.5) to[short] (12,12.5);
        \node [font=\large] at (7.75,15) {S};
        \node [font=\large] at (11.5,12.25) {T};
        \draw [->, >=Stealth] (10.75,14.5) -- (10.75,13.25);
        \draw [->, >=Stealth] (9,15) -- (10.75,14.5);
        \draw [->, >=Stealth] (10.75,13.25) -- (9,13.75);
        \node [font=\normalsize] at (11,13.25) {a};
        \node [font=\normalsize] at (11,14.75) {b};
        \node [font=\normalsize] at (8.75,15.25) {c};
        \node [font=\normalsize] at (8.75,13.75) {d};
    \end{circuitikz}
    }
    \end{center}

\end{enumerate}
\end{multicols}

\item Which one of the following is a dimensionless constant ?
\begin{enumerate}
    \item Permittivity of free space
    \item Permeability of free space
    \item Bohr magneton
    \item Fine structure constant
\end{enumerate}
\item Choose the most appropriate matching of the items in \textbf{Column 1} with those in
\textbf{Column 2}\\
\begin{table}[h]
    \centering
    \begin{tabular}{|l|l|}
        \hline
        \textbf{Column 1} & \textbf{Column 2} \\
        \hline
        (i) PIN diode &     P. Voltage regulation \\
        (ii) Tunnel diode & Q. Radio frequency and microwave devices \\
        (iii) Zener diode & R. Optoelectronic detection \\
        (iv) Photo diode &  S. Oscillator \\
        \hline
    \end{tabular}
\end{table}
\begin{enumerate}
    \item (i) - Q; (ii) - S; (iii) - P; (iv) - R
    \item (i) - R; (ii) - Q; (iii) - P; (iv) - S
    \item (i) - R; (ii) - S; (iii) - P; (iv) - Q
    \item (i) - P; (ii) - Q; (iii) - R; (iv) - S
\end{enumerate}



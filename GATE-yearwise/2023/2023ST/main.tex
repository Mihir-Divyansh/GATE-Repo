\iffalse
\chapter{2023}
\author{EE24BTECH11011}
\section{st}
\fi
\item Let $X_1,X_2,\dots,X_{10}$ be a random sample size $10$ from a population having $N\brak{0,\theta^2}$ distribution where $\theta > 0$ is and unknowm parameter.Let $T = \frac{1}{10}\sum_{i=1}^{10} {X_i}^2$. If the mean square error of $cT\brak{c>0}$, as an estimator of $\theta^2$, is minimized at $c = c_0$, then the value of $c_0$ equals
\begin{multicols}{4}
\begin{enumerate}
    \item $\frac{5}{6}$
    \item $\frac{2}{3}$
    \item $\frac{3}{5}$
    \item $\frac{1}{2}$
\end{enumerate}
\end{multicols}
\item Suppose that $X_1,X_2,\dots,X_{10}$ are independent and identically istributed random
vectors each having $N_2\brak{\mu , \sum}$ distribution, where $\sum$ is non-singular. If
\begin{align}
    U = \frac{1}{1+\brak{\overline{X}-\mu}^\top \sum^{-1}\brak{\overline{X}-\mu}}
\end{align}
where $\overline{X} = \frac{1}{10}\sum_{i=1}^{10}X_i$ then the value of $\log_e P\brak{U \leq \frac{1}{2}}$ equals
\begin{multicols}{4}
    \begin{enumerate}
        \item $-5$
        \item $-10$
        \item $-2$
        \item $-1$
    \end{enumerate}
\end{multicols}
\item Suppose that $\brak{X, Y}$ has joint probability mass function
    \begin{align}
    P\brak{X=0, Y=0} = P\brak{X=1, Y=1} = \theta,
    \end{align}
    \begin{align}
    P\brak{X=1, Y=0} = P\brak{X=0, Y=1} = \frac{1}{2} - \theta,
\end{align}
    where $0 \leq \theta \leq \frac{1}{2}$ is an unknown parameter. Consider testing $H_0: \theta = \frac{1}{4}$
    against $H_1: \theta = \frac{1}{3}$ based on a random sample $\{\brak{X_1, Y_1}, \brak{X_2, Y_2}, \dots , \brak{X_n, Y_n}\}$
    from the above probability mass function. Let $M$ be the cardinality of the set
    $\cbrak{i: X_i = Y_i, 1 \leq i \leq n}$. If $m$ is the observed value of $M$, then which one of the
    following statements is true?

    \begin{enumerate}
        \item The likelihood ratio test rejects $H_0$ if $m > c$ for some $c$
        \item The likelihood ratio test rejects $H_0$ if $m < c$ for some $c$
        \item The likelihood ratio test rejects $H_0$ if $c_1 < m < c_2$ for some $c_1$ and $c_2$
        \item The likelihood ratio test rejects $H_0$ if $m < c_1$ or $m > c_2$ for some $c_1$ and $c_2$\\
    \end{enumerate}
    \item  Let $g\brak{x} = f\brak{x} + f\brak{2-x}$ for all $x \in \brak{0,2}$, where $f \colon\brak{0,2} \to \mathbb{R}$ is continuous on $\brak{0,2}$ and twice differentiable on $\brak{0,2}$. If $g^\prime$ denotes the derivative of $g$ and $f^{\prime\prime}$ denotes the second derivative of $f$, then which one of the following statements is NOT true?  


    \begin{enumerate}
        \item There exists $c \in \brak{0,2}$ such that $g^\prime\brak{c} = 0$
        \item If $f^{\prime\prime} > 0$ on $\brak{0,2}$, then $g$ is strictly decreasing on $\brak{0,1}$
        \item  If $f^{\prime\prime} < 0$ on $\brak{0,2}$, then $g$ is strictly increasing on $\brak{1,2}$
        \item If $f^{\prime\prime} = 0$ on $\brak{0,2}$, then $g$ is a constant function\\
    \end{enumerate}
    \item For any subset $U$ of $\mathbb{R}^n$, let $L\brak{U}$ denote the span of $U$. For any two subsets $T$ and $S$ of $\mathbb{R}^n$, which one of the following statements is NOT true?

    \begin{enumerate}
        \item If $T$ is a proper subset of $S$, then $L\brak{T}$ is a proper subset of $L\brak{S}$
        \item $L\brak{L\brak{S}} = L\brak{S}$
        \item $L\brak{T \cup S} = \cbrak{u + v\colon u \in L\brak{T}, v \in L\brak{S}}$
        \item If $\alpha, \beta$, and $\gamma$ are three vectors in $\mathbb{R}^n$ such that $\alpha + 2\beta + 3\gamma = 0$, then $L\brak{\cbrak{\alpha, \beta}} = L\brak{\cbrak{\beta, \gamma}}$\\
    \end{enumerate}
    \item Let $f$ be a continuous function from $\sbrak{0,1}$ to the set of all real numbers. Then which one of the following statements is NOT true?  
\begin{enumerate}
\item For any sequence $\cbrak{x_n}_{n \geq 1}$ in $[0,1]$, $\sum_{n=1}^{\infty} \frac{f(x_n)}{n^2}$ is absolutely convergent.
\item If $\abs{f\brak{x}} = 1$ for all $x \in \sbrak{0,1}$, then $\abs{\int_{0}^{1} f(x) dx} = 1$.
\item If $\cbrak{x_n}_{n \geq 1}$ is a sequence in $\sbrak{0,1}$ such that $\cbrak{f\brak{x_n}}_{n \geq 1}$ is convergent, then $\cbrak{x_n}_{n \geq 1}$ is convergent.
\item If $f$ is also monotonically increasing, then the image of $f$ is given by $\sbrak{f\brak{0}, f\brak{1}}$.
\end{enumerate}
\item Let $X$ be a random variable with cumulative distribution function
\begin{align}
F\brak{x} = \begin{cases}
    0 & \text{if } x < -1 \\
    \frac{1}{4}\brak{x+1} & \text{if } -1 \leq x < 0 \\
    \frac{1}{4}\brak{x+3} & \text{if } 0 \leq x < 1 \\
    1 & \text{if } x \geq 1.
\end{cases}
\end{align}

Which one of the following statements is true?

\begin{enumerate}
\item $\lim_{n \to \infty} P\brak{-\frac{1}{2} + \frac{1}{n} < X < -\frac{1}{n}} = \frac{5}{8}$
\item $\lim_{n \to \infty} P\brak{-\frac{1}{2} - \frac{1}{n} < X < \frac{1}{n}} = \frac{5}{8}$
\item $\lim_{n \to \infty} P\brak{X = \frac{1}{n}} = \frac{1}{2}$
\item $P\brak{X = 0} = \frac{1}{3}$\\
\end{enumerate}
\item Let $\brak{X,Y}$ have joint probability mass function
\begin{align}p\brak{x,y} = \begin{cases}
    \frac{c}{2^{x+y+2}} & \text{if } x = 0, 1, 2, ...; \ y = 0, 1, 2, ...; \ x \neq y \\
    0 & \text{otherwise.}
\end{cases}\end{align}

Then which one of the following statements is true?

\begin{enumerate}
\item $c = \frac{1}{2}$
\item $c = \frac{1}{4}$
\item $c > 1$
\item $X$ and $Y$ are independent
\end{enumerate}
\item Let $X_1, X_2, ..., X_{10}$ be a random sample of size 10 from a $N_3(\mu, \Sigma)$ distribution, where $\mu$ and non-singular $\Sigma$ are unknown parameters. If 
\begin{align}\overline{X}_1 = \frac{1}{5}\sum_{i=1}^{5}X_i, \quad \overline{X}_2 = \frac{1}{5}\sum_{i=6}^{10}X_i,\end{align}
\begin{align}S_1 = \frac{1}{4}\sum_{i=1}^{5}\brak{X_i - \overline{X}_1}\brak{X_i - \overline{X}_1}^\prime, \quad S_2 = \frac{1}{4}\sum_{i=6}^{10}\brak{X_i - \overline{X}_2}\brak{X_i - \overline{X}_2}^\prime,\end{align}
then which one of the following statements is NOT true?

\begin{enumerate}
\item $\frac{5}{6}\brak{\overline{X}_1 - \mu}^\prime S_1^{-1}\brak{\overline{X}_1 - \mu}$ follows an F-distribution with $3$ and $2$ degrees of freedom.\\
\item  $\frac{6}{\brak{\overline{X}_1 - \mu}^\prime S_1^{-1}\brak{\overline{X}_1 - \mu}}$ follows an F-distribution with $2$ and $3$ degrees of freedom.\\
\item $4\brak{S_1 + S_2}$ follows a Wishart distribution of order 3 with 8 degrees of freedom.
\item $5\brak{S_1 + S_2}$ follows a Wishart distribution of order 3 with 10 degrees of freedom.
\end{enumerate}  
\item  Which of the following sets is/are countable?

\begin{enumerate}
\item The set of all functions from $\cbrak{1, 2, 3, ..., 10}$ to the set of all rational numbers
\item The set of all functions from the set of all natural numbers to $\cbrak{0,1}$
\item The set of all integer valued sequences with only finitely many non-zero terms
\item The set of all integer valued sequences converging to $1$
\end{enumerate}

\item For a given real number $a$, let $a^+ = \max\cbrak{a,0}$ and $a^- = \max\cbrak{-a,0}$.
      If $\cbrak{x_n}_{n\ge1}$ is a sequence of real numbers, then which of the following statements is/are true?

\begin{enumerate}
\item If $\cbrak{x_n}_{n\ge1}$ converges, then both $\cbrak{x_n^+}_{n\ge1}$ and $\cbrak{x_n^-}_{n\ge1}$ converge
\item If $\cbrak{x_n}_{n\ge1}$ converges to $0$, then both $\cbrak{x_n^+}_{n\ge1}$ and $\cbrak{x_n^-}_{n\ge1}$ converge to $0$
\item If both $\cbrak{x_n^+}_{n\ge1}$ and $\cbrak{x_n^-}_{n\ge1}$ converge, then $\cbrak{x_n}_{n\ge1}$ converges
\item If $\cbrak{x_n^2}_{n\ge1}$ converges, then both $\cbrak{x_n^+}_{n\ge1}$ and $\cbrak{x_n^-}_{n\ge1}$ converge
\end{enumerate}
\item Let $A$ be a $3\times3$ real matrix such that 
$A\myvec{1\\1\\0} = \myvec{0\\0\\4}, \quad A\myvec{0\\1\\1} = \myvec{4\\0\\0}, \quad A\myvec{1\\0\\1} = \myvec{0\\4\\0}$
Then which of the following statements is/are true?

\begin{enumerate}
\item $A\myvec{1\\0\\0} = \myvec{2\\2\\-2}$
\item $A\myvec{0\\1\\0} = \myvec{2\\-2\\2}$
\item $A\myvec{1\\1\\1} = \myvec{2\\0\\2}$
\item $A\myvec{1\\2\\3} = \myvec{8\\4\\0}$
\end{enumerate}
\item Let $X$ be a positive valued continuous random variable with finite mean.
      If $Y = \sbrak{X}$, the largest integer less than or equal to $X$, then which of the
      following statements is/are true?

\begin{enumerate}
\item $P\brak{Y \leq u} \leq P\brak{X \leq u}$ for all $u \geq 0$
\item $P\brak{Y \geq u} \leq P\brak{X \geq u}$ for all $u \geq 0$
\item $E\brak{X} < E\brak{Y}$
\item $E\brak{X} > E\brak{Y}$
\end{enumerate}


\iffalse
\title{MA-2020-40-52}
\author{EE24BTECH11041-Mohit}
\section{xe}
\chapter{2020}
\fi


\item Let \\
\begin{align}
f(x)=\frac{x^2}{x^2+(1-nx)^2}, x \in [0,1], n = 1,2,3,\dots  .
\end{align}
Then, which of the following statements is TRUE?
\hfill{(MA 2022)}
\begin{enumerate}
\item $\{f_n\}$ is not equicontinous in $[0,1]$
\item $\{f_n\}$ is uniformly convergent on $[0,1]$
\item $\{f_n\}$ is equicontinous on $[0,1]$
\item $\{f_n\}$ is uniformly bounded and has a subsquence coverging uniformly in $[0,1]$
\end{enumerate}
\item Let $(\mathbb{Q},d)$ be the metric space with $d\brak{x,y} = \abs{x-y}$. Let $E = \{p \in \mathbb{Q} : 2 < p^2 < 3 \}$ . Then, the set $E$ is 
\hfill{(MA 2022)}
\begin{enumerate}
\item closed but not compact
\item not closed but not compact
\item compact
\item neither closed nor compact
\end{enumerate}

\item Let $T:L^{2}[-1,1]\rightarrow L^{2}[-1,1]$ be defined by  $Tf = \tilde{f}$, where  $\tilde{f}(x) = f(-x)$ almost everywhere.
If $M$ is the kernel of $I-T$, then the distance between the function  $\phi(t) = e^t$  and $M$ is 
\hfill{(MA 2022)}
\begin{multicols}{2}
\begin{enumerate}
\item \quad $\frac{1}{2}\sqrt{(e^{2}-e^{-2}+4)}$
\item \quad $\frac{1}{2}\sqrt{(e^{2}-e^{-2}-2)}$ 
\item \quad $\frac{1}{2}\sqrt{(e^{2}-4)} $
\item \quad $\frac{1}{2}\sqrt{(e^{2}-e^{-2}-4)}$
\end{enumerate}
\end{multicols}
\item $X, Y $ and $Z$ be Banach spaces. Suppose that $T:X\rightarrow Y$ is linear and $S:Y \rightarrow Z$ is linear, bounded and injective. In addition, if $ S \circ T : X \rightarrow Z$ is bounded, then, which of the following statements is TRUE?
\hfill{(MA 2022)}
\begin{enumerate}
\item T is surjective
\item T is bounded but not continuous
\item T is bounded 
\item T is not bounded 
\end{enumerate}
\item The first derivative of a function $f \in C^{\infty}(-3, 3)$ is approximated by an interpolating polynomial of degree 2, using the data
\begin{align}
(-1, f(-1)),(0, f(0)) \text{ and } (2, f(2)).
\end{align}
It is found that
\begin{align}
f'(0) \approx \frac{2}{3}f(-1) + \alpha f(0) + \beta f(2).
\end{align}
Then, the value of $\frac{1}{\alpha \beta}$ is
\hfill{(MA 2022)}
\begin{multicols}{4}
\begin{enumerate}
\item 3 
\item 6
\item 9
\item 12
\end{enumerate}
\end{multicols}
\item The work done by the force $F = \brak{x+y}\hat{i} - \brak{x^2+y^2}\hat{j}$, where $\hat{i}$ and $\hat{j}$ are unit
vectors in $\vec{OX}$ and $\vec{OY}$ directions, respectively, along the upper half of the circle $x^2+y^2=1$ from $\brak{1,0}$ to $\brak{-1,0}$ in the $xy-$plane is
\hfill{(MA 2022)}
\begin{multicols}{4}
\begin{enumerate}
\item $-\pi $
\item $-\frac{\pi}{2}$
\item $\frac{\pi}{2}$
\item $\pi$
\end{enumerate} 
\end{multicols}
\item Let $u(x, t)$ be the solution of the wave equation
\hfill{(MA 2022)}
\begin{align}
\frac{\partial^2u}{\partial^2} - \frac{\partial^2u}{\partial x^2} = 0,0<x<\pi,t>0, 
\end{align}
with the initial conditions
\begin{align}
u\brak{x,0} = \sin{x}+\sin{2x}+\sin{3x},\frac{\partial u}{\partial t}\brak{x,0} = 0, 0<x<\pi
\end{align}
and the boundary comditions $u\brak{0, t} = u\brak{n, t} = 0, t \geq 0$.Then, the value of $u\brak{\frac{\pi}{2},\pi}$ is
\begin{multicols}{4}
\begin{enumerate}
\item $-\frac{1}{2}$
\item 0
\item $\frac{1}{2}$
\item 1
\end{enumerate}
\end{multicols}
\item Let $T : \mathbb{R}^2 \rightarrow \mathbb{R}^2$ be a linear transformation defined by
\begin{align}
T\brak{\brak{1,2}}=\brak{1,0} \text{ and } T\brak{\brak{2,1}}=\brak{1,1}
\end{align}
For $p,q \in \mathbb{R}^2,\text{let } T^{-1}\brak{\brak{p,q}}=\brak{x,y}$.\\
Which of the following statements is TRUE?
\hfill{(MA 2022)}

\begin{enumerate}
\item $x=p-q; y=2p-q$
\item $x=p+q; y=2p-q$
\item $x=p+q; y=2p+q$
\item $x=p-q; y=2p+q$
\end{enumerate}
\item Let $y = (\alpha,-1)^T , \alpha \in \mathbb{R}$ be a feasible solution for the dual problem of the linear
programming problem
\begin{align}
\text{Maximize : } 5x_1 + 12x_2 \\
\text{subject to : } x_1+2x_2+x_3 \leq 10 \\
2x_1-x_2+3x_3 = 8 \\
x_1,x_2,x_3 \geq 0.
\end{align}
Which of the following statements is TRUE?
\hfill{(MA 2022)}
\begin{multicols}{4}
\begin{enumerate}

\item $\alpha < 3$
\item $3 \leq \alpha < 5.5$
\item $5.5 \leq \alpha < 7$
\item $\alpha \geq 7$
\end{enumerate}
\end{multicols}
\item Let $K$ denote the subset of $\mathbb{C}$ consisting of elements algebraic over $\mathbb{Q}$. Then, which
of the following statements are TRUE?
\hfill{(MA 2022)}
\begin{enumerate}
\item No element of $\mathbb{C}\backslash K$ is algebraic over $\mathbb{Q}$
\item $K$ is an algebraically closed field
\item For any bijective ring homomorphism $f : \mathbb{C} \rightarrow \mathbb{C} $, we have $f(K)=K$
\item There is no bijection between $K$ and $\mathbb{Q}$
\end{enumerate}
\item Let $T$ be a Mobius transformation such that $T\brak{0}=\alpha,T\brak{\alpha}=0$ and $T(\infty)=-\alpha $, where $\alpha = \frac{-1+i}{\sqrt{2}}$ Let $L$ denote the straight line passing through the origin with slope  $-1$, and let $C$ denote the circle of unit radius centred at the origin. Then, which of the following statements are TRUE?
\hfill{(MA 2022)}
\begin{enumerate}
\item $T$ maps $L$ to a straight line
\item $T$ maps $L$ to a circle
\item $T^{-1}$ maps $C$ to a straight line
\item $T^{-1}$ maps $C$ to a circle
\end{enumerate}
\item Let $a>0$. Define $D_a : L^2\brak{\mathbb{R}} \rightarrow L^2\brak{\mathbb{R}}$ by $\brak{D_af}\brak{x}=\frac{1}{\sqrt{a}}f\brak{\frac{x}{a}}$,almost everywhere, for $f \in L^2\brak{\mathbb{R}}$. Then, which of the following statements are TRUE?
\hfill{(MA 2022)}
\begin{enumerate}
\item $D_a$ is a linear isometry
\item $D_a$ is a bijection
\item $D_a \circ D_b=D_{a+b},b>0$
\item $D_a$ is bounded from below
\end{enumerate}
\item Let $\{ \phi_0,\phi_1,\phi_2,\dots\}$ be an orthonormal set in $L^2[-1, 1]$ such that $\phi_n = C_nP_n$, where
$C_n$ is a constant and $P_n$ is the Legendre polynomial of degree $n$, for each $n \in \mathbb{N} \cup \{0\}$. Then, which of the following statements are TRUE?
\hfill{(MA 2022)}
\begin{multicols}{4}
\begin{enumerate}
\item $\phi_6\brak{1}=1$
\item $\phi_7\brak{-1}=1$
\item $\phi_7\brak{1}=\sqrt{\frac{15}{2}}$
\item $\phi_7\brak{-1}=\sqrt{\frac{13}{2}}$
\end{enumerate}
\end{multicols}


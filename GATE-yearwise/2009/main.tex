\iffalse
    \title{Assignment}
    \author{EE24BTECH11066}
    \section{ma}
    \chapter{2009}
  \fi
\item The dimension of the vector space $V=\cbrak{A=\brak{a_{ij}}_{n \times n}: a_{ij} \in \mathbb C,a_{ij}=-a_{ji}}$ over the field $\mathbb R$ is \hfill{[2009-MA]}
\begin{enumerate}
\begin{multicols}{4}
\item $n^2$
\item $n^2-1$
\item $n^2-n$
\item $\frac{n^2}{2}$
\end{multicols}
\end{enumerate}

\item The minimal polynomial associated with the matrix $\myvec{0&0&3\\1&0&2\\0&1&1}$ is \hfill{[2009-MA]}
\begin{enumerate}
\begin{multicols}{2}
\item $x^3-x^2-2x-3$
\item $x^3-x^2+2x-3$
\item $x^3-x^2-3x-3$
\item $x^3-x^2+3x-3$
\end{multicols}
\end{enumerate}

\item For the function $f(z)=\sin \brak{\frac{1}{\cos \brak{\frac{1}{z}}}}$, the point $z=0$ is \hfill{[2009-MA]}
\begin{enumerate}
\begin{multicols}{2}
\item a removable singularity
\item a pole
\item an essential singularity
\item a non-isolated singularity
\end{multicols}
\end{enumerate}

\item Let $f(z)=\sum _{n=0}^{15} z^{n}$ for $z \in \mathbb C$. If $C:\abs{z-i}=2$ then $\oint_C
\frac{f(z)}{\brak{z-i}^15}=$ \hfill{[2009-MA]}
\begin{enumerate}
\begin{multicols}{4}
\item $2\pi i \brak{1+15i}$
\item $2\pi i \brak{1-15i}$
\item $4\pi i \brak{1+15i}$
\item $2\pi i$
\end{multicols}
\end{enumerate}

\item For what values of $\alpha$ and $\beta$, the quadrature formula $\int \limits_{-1}^1 f(x)\, \text{dx} \approx \alpha f\brak{-1}+f\brak{\beta}$ is exact for all polynomials of degree $\leq 1 ?$ \hfill{[2009-MA]}
\begin{enumerate}
\begin{multicols}{4}
\item $\alpha=1, \beta=1$
\item $\alpha=-1, \beta=1$
\item $\alpha=1, \beta=-1$
\item $\alpha=-1, \beta=-1$
\end{multicols}
\end{enumerate}

\item Let $f:\sbrak{0,4} \rightarrow \mathbb R$ be a three times continuously differential function. Then the value of $f\sbrak{1,2,3,4}$ is \hfill{[2009-MA]}
\begin{enumerate}
\begin{multicols}{2}
\item $\frac{f''\brak{\xi}}{3}$ for some $\xi \in \brak{0,4}$
\item $\frac{f''\brak{\xi}}{6}$ for some $\xi \in \brak{0,4}$
\item $\frac{f'''\brak{\xi}}{3}$ for some $\xi \in \brak{0,4}$
\item $\frac{f'''\brak{\xi}}{6}$ for some $\xi \in \brak{0,4}$
\end{multicols}
\end{enumerate}

\item Which one of the following is TRUE ? \hfill{[2009-MA]}

\begin{enumerate}
    \item Every linear programming problem has a feasible solution.
    \item If a linear programming problem has an optimal solution then it is unique.
    \item The union of two convex sets is necessarily convex.
    \item Extreme points of the disk $x^2+y^2 \leq 1$ are the points on the circle $x^2+y^2=1.$
\end{enumerate}

\item The dual of the linear programming problem: \hfill{[2009-MA]}
\begin{align*}
    \text{Minimize } \textbf{c}^{\mathrm{T}}\textbf{x} \text{ subject to } A\textbf{x}\geq b \text{ and } \textbf{x} \geq 0 \text{ is}
\end{align*} 
\begin{enumerate}
    \item Maximize $\textbf{b}^{\mathrm{T}}\textbf{w}$ subject to $A^{\mathrm{T}}\textbf{w}\geq c \text{ and } \textbf{w}\geq 0$ 
    \item Maximize $\textbf{b}^{\mathrm{T}}\textbf{w}$ subject to $A^{\mathrm{T}}\textbf{w}\leq c \text{ and } \textbf{w}\geq 0$ 
    \item Maximize $\textbf{b}^{\mathrm{T}}\textbf{w}$ subject to $A^{\mathrm{T}}\textbf{w}\leq c \text{ and } \textbf{w}$ is unrestricted
    \item Maximize $\textbf{b}^{\mathrm{T}}\textbf{w}$ subject to $A^{\mathrm{T}}\textbf{w}\geq c \text{ and } \textbf{w}$ is unrestricted
\end{enumerate}

\item The resolvent kernel for the integral equation $u(x)=F(x)+\int \limits_{log 2}^x e^{\brak{t-x}}u(t)\, \text{dt}$ is 
    
     \hfill{[2009-MA]}
\begin{enumerate}
\begin{multicols}{4}
\item $\cos \brak{x-t}$
\item 1
\item $e^{\brak{t-x}}$
\item $e^{2\brak{t-x}}$
\end{multicols}
\end{enumerate}

\item Consider the metrics $d_2 \brak{f,g}=\brak{\int \limits_{a}^b \abs{f(t)-g(t)}^2\, \text{dt}}^\frac{1}{2}$ and $d_{\infty}\brak{f,g}=\sup_{t \in \sbrak{a,b}} \abs{f(t)-g(t)}$ on the space $X=C\sbrak{a,b}$ of all real valued continuous functions on $\sbrak{a,b}$. Then which of the following is TRUE ? \hfill{[2009-MA]}
\begin{enumerate}
    \item Both $\brak{X,d_2}$ and $\brak{X,d_{\infty}}$ are complete.
    \item $\brak{X,d_2}$ is complete but $\brak{X,d_{\infty}}$ is NOT complete.
    \item $\brak{X,d_{\infty}}$ is complete but $\brak{X,d_2}$ is NOT complete.
    \item Both $\brak{X,d_2}$ and $\brak{X,d_{\infty}}$ are NOT complete.
\end{enumerate}

\item A function $f:\mathbb R \rightarrow \mathbb R$ need NOT be Lebesgue measurable if \hfill{[2009-MA]}
\begin{enumerate}
    \item $f$ is monotone
    \item $\cbrak{x \in \mathbb R:f(x) \geq \alpha}$ is measurable for each $\alpha \in \mathbb Q$
    \item $\cbrak{x \in \mathbb R:f(x) = \alpha}$ is measurable for each $\alpha \in \mathbb R$
    \item For each open set $G$ in $\mathbb R, f^{-1}\brak{G}$ is measurable
\end{enumerate}

\item Let $\cbrak{e_n}_{n=1}^\infty$ be an orthonormal sequence in a Hilbert space $H$ and let $x\brak{\neq 0} \in H$. Then \hfill{[2009-MA]}
\begin{enumerate}
    \item $\lim\limits_{x\to \infty} \langle x, e_n \rangle $ does not exist
    \item $\lim\limits_{x\to \infty} \langle x, e_n \rangle =\abs{\abs{x}}$
    \item $\lim\limits_{x\to \infty} \langle x, e_n \rangle =1$
    \item $\lim\limits_{x\to \infty} \langle x, e_n \rangle =0$
\end{enumerate}

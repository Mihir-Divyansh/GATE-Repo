
\iffalse
\chapter{2009}
\author{AI24BTECH11008}
\section{ph}
\fi

    \item[49.] The Lagrangian of a particle of mass m moving in one dimension is $ L = exp\brak{\alpha t}\big[\frac{m\dot{x}^2}{2}-\frac{kx^2}{2}],$ where $\alpha$ and k are positive constants. The equation of motion of the particle is  \hfill (2009)
     \begin{enumerate}[label=(\Alph*)]
        \item $\ddot{x}+\alpha \dot{x} = 0$
        \item $\ddot{x}+\frac{k}{m}x = 0$
        \item $\ddot{x}-\alpha \dot{x}+\frac{k}{m}x = 0$
        \item $\ddot{x}+\alpha \dot{x}+\frac{k}{m}x = 0$
     \end{enumerate}
     \item[50.] Two monochromatic waves having frequencies $\omega$ and $\omega + \triangle \omega \brak{\Delta\omega<<<\omega}$ and corresonding wavelength $\lambda$ and $\lambda-\Delta\lambda\brak{\Delta\lambda<<<\lambda}$ of same polarization, travelling along x-axis are superimposed on each other. The phase velocity and group velocity of the resultant wave are respectively given by \hfill (2009)
     \begin{enumerate}[label=(\Alph*)]
         \item $\frac{\omega\lambda}{2\pi},\frac{\Delta\lambda ^2}{2\pi\Delta\lambda}$
         \item $\omega\lambda,\frac{\Delta\lambda ^2}{\Delta\lambda}$
         \item $\frac{\omega\Delta\lambda}{2\pi},\frac{\Delta\lambda}{2\pi\Delta\lambda}$
         \item $\omega\Delta\lambda,\omega\Delta\lambda$
     \end{enumerate}
     \textbf{Common data questions}
     Common data questions 51 and 52
     Consider a two level quantum system with energies $\epsilon_1=0 \text{and} \epsilon_2 = \epsilon$
     \item[51.] The Helmholtz free energy of the system is given by\hfill (2009)
     \begin{enumerate}[label=(\Alph*)]
        \item $-k_BTln\brak{1+e^{\frac{-\epsilon}{k_BT}}}$
        \item $k_BTln\brak{1+e^{\frac{-\epsilon}{k_BT}}}$
        \item $\frac{3}{2}k_B T$
        \item $\epsilon - k_BT$
     \end{enumerate}
     \item [52.]  The specific heat of the system is given by\hfill (2009)
     \begin{enumerate}[label=(\Alph*)]
        \item $\frac{\epsilon}{k_BT}\frac{e^{\frac{-\epsilon}{k_BT}}}{\brak{1+e^{\frac{-\epsilon}{k_BT}}}^2}$
        \item $\frac{\epsilon^2}{k_BT^2}\frac{e^{\frac{-\epsilon}{k_BT}}}{\brak{1+e^{\frac{-\epsilon}{k_BT}}}}$
        \item $-\frac{\epsilon^2e^{\frac{-\epsilon}{k_BT}}}{\brak{1+e^{\frac{-\epsilon}{k_BT}}}^2}$
        \item  $\frac{\epsilon^2}{k_BT^2}\frac{e^{\frac{-\epsilon}{k_BT}}}{\brak{1+e^{\frac{-\epsilon}{k_BT}}}^2}$
     \end{enumerate}
     Common data questions 53 and 54
     A free particle of mass m moves along the x-direction. At t = 0, the normalized wave function of the particle is given by $\psi\brak{x,0}=\frac{1}{\brak{2\pi\alpha}^{1/4}}exp{-\frac{x^2}{4\alpha^2}+ix}$, where $\alpha$ is a real constant 
     \item[53.] The expectation value of the momentum, in this state is \hfill (2009)
     \begin{enumerate}[label=(\Alph*)]
        \item $\hslash \alpha$
        \item $\hslash \sqrt{\alpha}$
        \item $\alpha$
        \item $\frac{\hslash}{\sqrt{\alpha}}$
     \end{enumerate}
     \item[54.] The expectation value of the particle energy is \hfill (2009)
     \begin{enumerate}[label=(\Alph*)]
        \item $\frac{\hslash^2}{2m}\frac{1}{2\alpha^{3/2}}$
        \item $\frac{\hslash^2}{2m}\alpha^2$
        \item $\frac{\hslash^2}{2m}\frac{4\alpha^2+1}{4\alpha^{3/2}}$
         \item $\frac{\hslash^2}{8m\alpha^{3/2} }$
     \end{enumerate}
     Common data questions 55 and 56
     Consider the Zeeman splitting of a single electron system for the 3d to 3p electric dipole transition
     \item[55.] The Zeeman spectrum is \hfill (2009)
     \begin{enumerate}[label=(\Alph*)]
        
        \item Randomly polarized
        \item only $\pi$ polarized
        \item only $\sigma$ polarized
        \item both $\pi$ and $\sigma$ polarized
     \end{enumerate}
     \item[56.]  The fine structure line having the longest wavelength will split into \hfill (2009)
     \begin{enumerate}[label=(\Alph*)]
        \item  17 components
        \item  10 components
        \item  8 components
        \item  4 components
     \end{enumerate}
     \textbf{Linked Answer Questions}
     Statement for Linked Answer Questions 57 ans 58:
    The primitive translation vectors of the face centered cubic (fcc) lattice are 
    $$\hat{a}_1 = \frac{a}{2}\brak{\hat{j}+\hat{k}};\hat{a}_2 = \frac{a}{2}\brak{\hat{i}+\hat{k}};\hat{a}_1 = \frac{a}{2}\brak{\hat{j}+\hat{i}}$$
    \item[57.] The primitive transition vectors of the fccreciprocal lattice are \hfill (2009)
    \begin{enumerate}[label=(\Alph*)]
        \item $\hat{b}_1 = \frac{2\pi}{a}\brak{\hat{j}+\hat{k}-\hat{i}};\hat{b}_2 = \frac{2\pi}{a}\brak{-\hat{j}+\hat{k}+\hat{i}};\hat{b}_3 = \frac{2\pi}{a}\brak{\hat{j}-\hat{k}+\hat{i}}$
        \item $\hat{b}_1 = \frac{\pi}{a}\brak{\hat{j}+\hat{k}-\hat{i}};\hat{b}_2 = \frac{\pi}{a}\brak{-\hat{j}+\hat{k}+\hat{i}};\hat{b}_3 = \frac{\pi}{a}\brak{\hat{j}-\hat{k}+\hat{i}}$
        \item $\hat{b}_1 = \frac{\pi}{2a}\brak{\hat{j}+\hat{k}-\hat{i}};\hat{b}_2 = \frac{\pi}{2a}\brak{-\hat{j}+\hat{k}+\hat{i}};\hat{b}_3 = \frac{\pi}{2a}\brak{\hat{j}-\hat{k}+\hat{i}}$
        \item $\hat{b}_1 = \frac{3\pi}{a}\brak{\hat{j}+\hat{k}-\hat{i}};\hat{b}_2 = \frac{3\pi}{a}\brak{-\hat{j}+\hat{k}+\hat{i}};\hat{b}_3 = \frac{3\pi}{a}\brak{\hat{j}-\hat{k}+\hat{i}}$
    \end{enumerate}
    \item[58.] The volume of the primitive cell of the fcc reciprocal lattice is \hfill (2009)
    \begin{enumerate}[label=(\Alph*)]
        \item $4\brak{\frac{\pi}{a}}^3$
        \item $4\brak{\frac{2\pi}{a}}^3$
        \item $4\brak{\frac{\pi}{2a}}^3$
        \item $4\brak{\frac{3\pi}{a}}^3$
    \end{enumerate}
    
    Statement for Linked Answer Questions 59 and 60:
    The Karnaugh map of logic circuit shown is below 
    \newpage
    \begin{figure}[!ht]
        \centering
        \resizebox{0.3\textwidth}{!}{%
    \begin{circuitikz}
    \tikzstyle{every node}=[font=\normalsize]
    
    \draw  (1.75,10) rectangle (4.75,6);
    \draw [short] (3.25,10) -- (3.25,6);
    \draw [short] (1.75,9) -- (4.75,9);
    \draw [short] (1.75,8) -- (4.75,8);
    \draw [short] (1.75,7) -- (4.75,7);
    \draw [short] (1,10.75) -- (1.75,10);
    \node [font=\normalsize] at (2.5,10.25) {$\bar{R}$};
    \node [font=\normalsize] at (4,10.25) {R};
    \node [font=\normalsize] at (1.25,9.5) {$\bar{P}\bar{Q}$};
    \node [font=\normalsize] at (1.25,7.5) {PQ};
    \node [font=\normalsize] at (1.25,6.5) {$P\bar{Q}$};
    \node [font=\normalsize] at (1.25,8.5) {$\bar{P}$Q};
    \node [font=\normalsize] at (2.5,9.5) {1};
    \node [font=\normalsize] at (4,9.5) {1};
    \node [font=\normalsize] at (2.5,8.5) {1};
    \node [font=\normalsize] at (2.5,6.5) {1};
    \node [font=\normalsize] at (4,6.5) {1};
    \end{circuitikz}
    }%  % Specify the path to your TikZ file
        \caption{1}
        \label{fig1}
    \end{figure}
    %image
    \item[59.] The minimized logic expression for the above map is \hfill (2009)
    \begin{enumerate}[label=(\Alph*)]
        \item $Y=\bar{PR} + \bar{Q}$
        \item $Y=\bar{Q} .PR$
        \item $Y=PR + \bar{Q}$
        \item $Y=\bar{PR}.Q$
    \end{enumerate}
    \item[60.] The corresponding logic implementation using gates is given as: \hfill (2009)
         \begin{figure}[!ht]
        \centering
        \resizebox{0.3\textwidth}{!}{%
    \begin{circuitikz}
    \tikzstyle{every node}=[font=\normalsize]
   
    \draw (6.5,12.25) node[ieeestd not port, anchor=in](port){} (port.out) to[short] (8.25,12.25);
    \draw (port.in) to[short] (6.25,12.25);
    \draw (6.5,11) node[ieeestd not port, anchor=in](port){} (port.out) to[short] (8.25,11);
    \draw (port.in) to[short] (6.25,11);
    \draw (6.5,9.75) node[ieeestd not port, anchor=in](port){} (port.out) to[short] (8.25,9.75);
    \draw (port.in) to[short] (6.25,9.75);
    \draw (8.75,12) to[short] (9,12);
    \draw (8.75,11.5) to[short] (9,11.5);
    \draw (9,12) node[ieeestd nor port, anchor=in 1, scale=0.89](port){} (port.out) to[short] (10.75,11.75);
    \draw [short] (8.25,12.25) -- (8.75,12);
    \draw [short] (8.25,11) -- (8.75,11.5);
    \draw [short] (8,9.75) -- (11.5,9.75);
    \draw [short] (10.75,11.75) -- (11.5,11.75);
    \draw [short] (11.5,11.75) -- (13,10.75);
    \draw [short] (11.5,9.75) -- (13,10.25);
    \draw (13,10.75) to[short] (13.25,10.75);
    \draw (13,10.25) to[short] (13.25,10.25);
    \draw (13.25,10.75) node[ieeestd nor port, anchor=in 1, scale=0.89](port){} (port.out) to[short] (15,10.5);
    \draw [short] (15,10.5) -- (16,10.5);
    \node [font=\normalsize] at (6,12.25) {P};
    \node [font=\normalsize] at (6,11) {R};
    \node [font=\normalsize] at (6,9.75) {Q};
    \node [font=\normalsize] at (16.5,10.5) {Y};
    \end{circuitikz}
    }%  % Specify the path to your TikZ file
        \caption{option1}
        %\label{fig2}
    \end{figure}
    \begin{figure}[!ht]
        \centering
        \resizebox{0.3\textwidth}{!}{%
        \begin{circuitikz}
        \tikzstyle{every node}=[font=\normalsize]
        
        \draw (6.5,12.25) node[ieeestd not port, anchor=in](port){} (port.out) to[short] (8.25,12.25);
        \draw (port.in) to[short] (6.25,12.25);
        \draw (6.5,11) node[ieeestd not port, anchor=in](port){} (port.out) to[short] (8.25,11);
        \draw (port.in) to[short] (6.25,11);
        \draw (6.5,9.75) node[ieeestd not port, anchor=in](port){} (port.out) to[short] (8.25,9.75);
        \draw (port.in) to[short] (6.25,9.75);
        \draw [short] (8.25,12.25) -- (8.75,12);
        \draw [short] (8.25,11) -- (8.75,11.5);
        \draw [short] (8,9.75) -- (11.5,9.75);
        \draw [short] (10.75,11.75) -- (11.5,11.75);
        \draw [short] (11.5,11.75) -- (13,10.75);
        \draw [short] (11.5,9.75) -- (13,10.25);
        \draw [short] (15,10.5) -- (16,10.5);
        \node [font=\normalsize] at (6,12.25) {P};
        \node [font=\normalsize] at (6,11) {R};
        \node [font=\normalsize] at (6,9.75) {Q};
        \node [font=\normalsize] at (16.5,10.5) {Y};
        \draw (8.75,12) to[short] (9,12);
        \draw (8.75,11.5) to[short] (9,11.5);
        \draw (9,12) node[ieeestd and port, anchor=in 1, scale=0.89](port){} (port.out) to[short] (10.75,11.75);
        \draw (13,10.75) to[short] (13.25,10.75);
        \draw (13,10.25) to[short] (13.25,10.25);
        \draw (13.25,10.75) node[ieeestd or port, anchor=in 1, scale=0.89](port){} (port.out) to[short] (15,10.5);
        \end{circuitikz}
        }% % Specify the path to your TikZ file
        \caption{option2}
        %\label{fig3}
    \end{figure}
    \begin{figure}[!ht]
        \centering
        \resizebox{0.3\textwidth}{!}{%
    \begin{circuitikz}
    \tikzstyle{every node}=[font=\normalsize]
    
    \draw (6.5,12.25) node[ieeestd not port, anchor=in](port){} (port.out) to[short] (8.25,12.25);
    \draw (port.in) to[short] (6.25,12.25);
    \draw (6.5,11) node[ieeestd not port, anchor=in](port){} (port.out) to[short] (8.25,11);
    \draw (port.in) to[short] (6.25,11);
    \draw (6.5,9.75) node[ieeestd not port, anchor=in](port){} (port.out) to[short] (8.25,9.75);
    \draw (port.in) to[short] (6.25,9.75);
    \draw [short] (8.25,12.25) -- (8.75,12);
    \draw [short] (8.25,11) -- (8.75,11.5);
    \draw [short] (8,9.75) -- (11.5,9.75);
    \draw [short] (10.75,11.75) -- (11.5,11.75);
    \draw [short] (11.5,11.75) -- (13,10.75);
    \draw [short] (11.5,9.75) -- (13,10.25);
    \draw [short] (15,10.5) -- (16,10.5);
    \node [font=\normalsize] at (6,12.25) {P};
    \node [font=\normalsize] at (6,11) {R};
    \node [font=\normalsize] at (6,9.75) {Q};
    \node [font=\normalsize] at (16.5,10.5) {Y};
    \draw (8.75,12) to[short] (9,12);
    \draw (8.75,11.5) to[short] (9,11.5);
    \draw (9,12) node[ieeestd nand port, anchor=in 1, scale=0.89](port){} (port.out) to[short] (10.75,11.75);
    \draw (13,10.75) to[short] (13.25,10.75);
    \draw (13,10.25) to[short] (13.25,10.25);
    \draw (13.25,10.75) node[ieeestd nor port, anchor=in 1, scale=0.89](port){} (port.out) to[short] (15,10.5);
    \end{circuitikz}
    }%  % Specify the path to your TikZ file
        \caption{option3}
       % \label{fig4}
    \end{figure}
    \begin{figure}[!ht]
        \centering
        \resizebox{0.3\textwidth}{!}{%
    \begin{circuitikz}
    \tikzstyle{every node}=[font=\normalsize]
    
    \draw (6.5,12.25) node[ieeestd not port, anchor=in](port){} (port.out) to[short] (8.25,12.25);
    \draw (port.in) to[short] (6.25,12.25);
    \draw (6.5,11) node[ieeestd not port, anchor=in](port){} (port.out) to[short] (8.25,11);
    \draw (port.in) to[short] (6.25,11);
    \draw (6.5,9.75) node[ieeestd not port, anchor=in](port){} (port.out) to[short] (8.25,9.75);
    \draw (port.in) to[short] (6.25,9.75);
    \draw [short] (8.25,12.25) -- (8.75,12);
    \draw [short] (8.25,11) -- (8.75,11.5);
    \draw [short] (8,9.75) -- (11.5,9.75);
    \draw [short] (10.75,11.75) -- (11.5,11.75);
    \draw [short] (11.5,11.75) -- (13,10.75);
    \draw [short] (11.5,9.75) -- (13,10.25);
    \draw [short] (15,10.5) -- (16,10.5);
    \node [font=\normalsize] at (6,12.25) {P};
    \node [font=\normalsize] at (6,11) {R};
    \node [font=\normalsize] at (6,9.75) {Q};
    \node [font=\normalsize] at (16.5,10.5) {Y};
    \draw (8.75,12) to[short] (9,12);
    \draw (8.75,11.5) to[short] (9,11.5);
    \draw (9,12) node[ieeestd nand port, anchor=in 1, scale=0.89](port){} (port.out) to[short] (10.75,11.75);
    \draw (13,10.75) to[short] (13.25,10.75);
    \draw (13,10.25) to[short] (13.25,10.25);
    \draw (13.25,10.75) node[ieeestd xor port, anchor=in 1, scale=0.89](port){} (port.out) to[short] (15,10.5);
    \end{circuitikz}
    }%  % Specify the path to your TikZ file
        \caption{option4}
       % \label{fig5}
    \end{figure}



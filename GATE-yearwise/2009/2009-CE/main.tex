\iffalse
\chapter{2009}
\author{AI24BTECH11009}
\section{ce}
\fi

\item Water flows through a 100 mm diameter pipe with a velocity of 0.015 m/sec. If the kinematic viscosity of water is $1.13 \times 10^{-6}$ $\text{m}^2$/sec, the friction factor of the pipe material is
    \begin{enumerate}
        \item 0.0015
        \item 0.032
        \item 0.037
        \item 0.048 \\
    \end{enumerate}
\item A rectangular open channel of width 4.5 m is carrying a discharge of 100 $\text{m}^3$/sec. The critical depth of the channel is
\begin{enumerate}
    \item 7.09 m
    \item 3.69 m
    \item 2.16 m
    \item 1.31 m \\
\end{enumerate}
\item Water ($\gamma_w$ = 9.879 kN/$\text{m}^3$) flows with a flow rate of 0.3 $\text{m}^3$/sec through a pipe AB of 10 m length and of uniform cross section. The end 'B' is above end 'A' and the pipe makes an angle of 30\degree to the horizontal. For a pressure of 12 kN/$\text{m}^2$ at the end 'B', the corresponding pressure at end 'A' is
\begin{enumerate}
    \item 12.0 kN/$\text{m}^2$
    \item 17.0 kN/$\text{m}^2$
    \item 56.4 kN/$\text{m}^2$
    \item 61.4 kN/$\text{m}^2$ \\
\end{enumerate}
\item An agricultural land of 437 ha is to be irrigated for a particular crop. The base period of the crop is 90 days and the total depth of water required by the crop is 105 cm. If a rainfall of 15 cm occurs during the base period, the duty of irrigation water is
 \begin{enumerate}
     \item 437 ha/cumec
     \item 486 ha/cumec
     \item 741 ha/cumec
     \item 864 ha/cumec \\
 \end{enumerate}
\item The correct match of \textbf{Column I} with \textbf{Column II} is 
\begin{table}[h!]
  \centering
  \begin{tabular}[12pt]{ |c| c|}
    \hline
    \textbf{Column I} & \textbf{Column II}\\ 
    \hline
    P. Coriolis effect & 1. Rotation of earth \\
    \hline 
    Q. Fumigation & 2. Lapse rate and vertical temperature profile \\
    \hline
    R. Ozone layer & 3. Inversion \\
    \hline
    S. Maximum mixing depth (mixing height) & 4. Dobson\\
    \hline
    \end{tabular}

\end{table} 
\begin{enumerate}
    \item P-2, Q-1, R-4, S-3
    \item P-2, Q-1, R-3, S-4
    \item P-1, Q-3, R-2, S-4
    \item P-1, Q-3, R-4, S-2 \\
\end{enumerate}
\item A horizontal flow primary clarifier treats wastewater in which 10\%, 60\% and 30\% of particles have settling velocities of 0.1 mm/s, 0.2 mm/s and 1.0 mm/s respectively. What would be the total percentage of particles removed if clarifier operates at a Surface Overflow Rate (SOR) of 43.2 $\text{m}^3/\text{m}^2\cdot$d ?
\begin{enumerate}
    \item 43 \%
    \item 56 \%
    \item 86 \%
    \item 100 \% \\
\end{enumerate}
\item An aerobic reactor receives wastewater at a flow rate of 500 $\text{m}^3$/d having a COD of 2000 mg/L. The effluent COD is 400 mg/L. Assuming that wastewater contains 80\% biodegradable waste, the daily volume of methane produced by the reactor is
\begin{enumerate}
    \item 0.224 $\text{m}^3$
    \item 0.280 $\text{m}^3$
    \item 224 $\text{m}^3$
    \item 280 $\text{m}^3$ \\
\end{enumerate}
\item The correct match of \textbf{Column I} with \textbf{Column II} is
\begin{table}[h!]
  \centering
  \begin{tabular}[12pt]{ |c| c|}
    \hline
    \textbf{Column I} & \textbf{Column II}\\ 
    \hline
    P. Grit chamber & 1. Zone settling \\
    \hline 
    Q. Secondary settling tank & 2. Stoke's Law \\
    \hline
    R. Activated sludge process & 3. Aerobic \\
    \hline
    S. Trickling Filter & 4. Contact stabilisation \\
    \hline
    \end{tabular}

\end{table} 
 \begin{enumerate}
     \item P-1, Q-2, R-3, S-4
     \item P-2, Q-1, R-3, S-4
     \item P-1, Q-2, R-4, S-3
     \item P-2, Q-1, R-4, S-3 \\
 \end{enumerate}
\item Which of the following stress combinations are appropriate in identifying the critical condition for the design of concrete pavements ?
\begin{table}[h!]
  \centering
  \begin{tabular}[12pt]{ |c| c|}
    \hline
    \textbf{Type of Stress} & \textbf{Location}\\ 
    \hline
    P. Load & 1. Corner \\
    \hline 
    Q. Temperature & 2. Edge \\
    \hline
     & 3. Interior \\
    \hline
    \end{tabular}

\end{table} 
\begin{enumerate}
     \item P-2, Q-3
     \item P-1, Q-3
     \item P-3, Q-1
     \item P-2, Q-2 \\
 \end{enumerate}
\item A crest vertical curve joins two gradients of \brak{+3\%} and \brak{-2\%} for a design speed of 80 km/h and the corresponding stopping sight distance of 120 m. The height of driver's eye and the object above the road surface are 1.20 m and 0.15 m respectively. The curve length (which is less than stopping sight distance) to be provided is
\begin{enumerate}
    \item 120 m
    \item 152 m
    \item 163 m
    \item 240 m \\
\end{enumerate}
\item On a specific highway, the speed-density relationship follows the Greenberg's model $\sbrak{v = v_f \ln{\brak{\frac{k_j}{k}}}}$, where $v_f$ and $k_j$ are the free flow speed and jam density respectively. When the highway is operating at a capacity, the density obtained as per this model is
\begin{enumerate}
    \item $e \cdot k_j$
    \item $k_j$
    \item $\frac{k_j}{2}$
    \item $\frac{k_j}{e}$ \\
\end{enumerate}
\item A three-phase traffic signal at an intersection is designed for flows shown in the figure below. There are six groups of flows identified by the numbers 1 through 6. Among these 1, 3, 4, and 6 are through flows and, 2 and 5 are right turning. Which phasing scheme is \textbf{\underline{not feasible}} ?
 \begin{figure}[!ht]
\centering
\resizebox{0.5\textwidth}{!}{%
\begin{circuitikz}
\tikzstyle{every node}=[font=\large]
\draw  (4.75,17) ellipse (0.25cm and 1cm);
\draw  (5,12.5) ellipse (0.25cm and 1cm);
\draw  (7.5,14.5) ellipse (1cm and 0.25cm);
\draw  (2.25,14.5) ellipse (1cm and 0.25cm);
\draw [short] (4,18) -- (4,15.5);
\draw [short] (4,15.5) -- (1,15.5);
\draw [short] (1,13.5) -- (4,13.5);
\draw [short] (4,13.5) -- (4,11);
\draw [short] (5.75,18) -- (5.75,15.5);
\draw [short] (5.75,15.5) -- (8.5,15.5);
\draw [short] (5.75,13.5) -- (8.5,13.5);
\draw [short] (5.75,13.5) -- (5.75,11);
\draw [short] (5,17.75) -- (4.5,17.5);
\draw [short] (4.5,17.25) -- (5,17.5);
\draw [short] (4.5,17) -- (5,17.25);
\draw [short] (4.5,16.75) -- (5,17);
\draw [short] (4.5,16.5) -- (5,16.75);
\draw [short] (4.5,16.25) -- (5,16.5);
\draw [short] (1.5,14.75) -- (1.75,14.25);
\draw [short] (1.75,14.75) -- (2,14.25);
\draw [short] (2,14.75) -- (2.25,14.25);
\draw [short] (2.25,14.75) -- (2.5,14.25);
\draw [short] (2.5,14.75) -- (2.75,14.25);
\draw [short] (2.75,14.75) -- (3,14.25);
\draw [short] (6.75,14.75) -- (7,14.25);
\draw [short] (7,14.75) -- (7.25,14.25);
\draw [short] (7.25,14.75) -- (7.5,14.25);
\draw [short] (7.5,14.75) -- (7.75,14.25);
\draw [short] (7.75,14.75) -- (8,14.25);
\draw [short] (8,14.75) -- (8.25,14.25);
\draw [short] (4.75,13) -- (5.25,13.25);
\draw [short] (4.75,12.75) -- (5.25,13);
\draw [short] (4.75,12.5) -- (5.25,12.75);
\draw [short] (4.75,12.25) -- (5.25,12.5);
\draw [short] (4.75,12) -- (5.25,12.25);
\draw [short] (4.75,11.75) -- (5.25,12);
\draw [->, >=Stealth] (1,15) -- (3.5,15);
\draw [->, >=Stealth] (8.75,14) -- (6,14);
\draw [->, >=Stealth] (5.5,18) -- (5.5,15.5);
\draw [->, >=Stealth] (5.25,18) .. controls (5.25,17) and (5.5,16) .. (4.75,15.5) ;
\draw [->, >=Stealth] (4.25,11.25) -- (4.25,13.75);
\draw [->, >=Stealth] (4.5,11.25) .. controls (4.5,12.5) and (4.25,13.25) .. (5,13.75) ;
\node [font=\large] at (4.25,11) {1};
\node [font=\large] at (4.5,11) {2};
\node [font=\large] at (9,14) {3};
\node [font=\large] at (5.5,18.25) {4};
\node [font=\large] at (5.25,18.25) {5};
\node [font=\large] at (0.75,15) {6};
\end{circuitikz}

}%
\end{figure}
\begin{table}[h!]
  \centering
  \begin{tabular}[12pt]{ |c| c| c| c|}
    \hline
    \textbf{Combination choice} & \textbf{Phase I} & \textbf{Phase II} & \textbf{Phase III} \\ 
    \hline
    P & 1, 4 & 2, 5 & 3, 6 \\
    \hline 
    Q & 1, 2 & 4, 5 & 3, 6 \\
    \hline
    R & 2, 5 & 1, 3 & 4, 6 \\
    \hline
    S & 1, 4 & 2, 6 & 3, 5 \\
    \hline
    \end{tabular}

\end{table} 
\begin{enumerate}
    \item P
    \item Q
    \item R
    \item S \\
\end{enumerate}

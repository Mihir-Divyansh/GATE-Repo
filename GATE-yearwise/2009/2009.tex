\iffalse
    \author{EE24BTECH11061}
    \section{ma}
    \chapter{2009}
 \fi
%\begin{enumerate}
\item A simple pendulum, consisting of a bob of mass $m$ connected with a string of length $a$, is oscillating in a vertical plane. If the string is making an angle $\theta$ with the vertical, then the expression for the Lagrangian is given as
\begin{multicols}{2}
    \begin{enumerate}
        \item $ma^2\brak{\dot{\theta}^2 - \frac{2g}{a} \sin^2{\frac{\theta}{2}}}$
        \item $2mga \sin^2{\frac{\theta}{2}}$
        \item $ma^2\brak{\frac{\dot{\theta}^2}{2} - \frac{2g}{a} \sin^2{\frac{\theta}{2}}}$
        \item $\frac{ma}{2}\brak{\dot{\theta}^2 - \frac{2g}{a} \cos{\theta}}$
    \end{enumerate}
\end{multicols}

\item The extremal of the functional $\int_0^1 \brak{y+x^2 + \frac{y^{\prime 2}}{4}}, dx$, $y\brak{0} = 0$, $y\brak{1} = 0$ is
\begin{multicols}{2}
    \begin{enumerate}
        \item $4\brak{x^2-x}$
        \item $3\brak{x^2-x}$
        \item $2\brak{x^2-x}$
        \item $x^2-x$
    \end{enumerate}
\end{multicols}

\subsection*{Common Data for Questions 51 \& 52:}
Let $T \colon \mathbb{R}^3 \rightarrow \mathbb{R}^3$ be the linear transformation defined by
\begin{align*}
    T\brak{x_1, x_2, x_3} = \brak{x_1 + 3 x_2+2x_3, 3x_1 + 4x_2 + x_3, 2x_1 + x_2 - x_3}
\end{align*}
\item The dimension of the range space of $T^2$ is
\begin{multicols}{2}
    \begin{enumerate}
        \item 0
        \item 1
        \item 2
        \item 3
    \end{enumerate}
\end{multicols}
\item The dimension of the null space of $T^3$ is
\begin{multicols}{2}
    \begin{enumerate}
        \item 0
        \item 1
        \item 2
        \item 3
    \end{enumerate}
\end{multicols}

\subsection*{Common Data for Questions 53 \& 54:}
Let $y_1\brak{x} = 1+x$ and $y_2\brak{x} = e^x$ be two solutions of $y^{{\prime}{\prime}}\brak{x} + P\brak{x}y^{\prime}\brak{x} + Q\brak{x} y\brak{x} = 0$
\item $P\brak{x} = $
\begin{multicols}{2}
    \begin{enumerate}
        \item $1+x$
        \item $-1-x$
        \item $\frac{1+x}{x}$
        \item $\frac{-1-x}{x}$
    \end{enumerate}
\end{multicols}
\item The set of initial conditions for which the above differential equation has NO solution is
\begin{multicols}{2}
    \begin{enumerate}
        \item $y\brak{0} = 2, y^{\prime}\brak{0} = 1$
        \item $y\brak{1} = 0, y^{\prime}\brak{1} = 1$
        \item $y\brak{1} = 1, y^{\prime}\brak{1} = 0$
        \item $y\brak{2} = 1, y^{\prime}\brak{2} = 2$
    \end{enumerate}
\end{multicols}

\subsection*{Common Data for Questions 55 \& 56:}
Let $X$ and $Y$ be random variables having the joint probability density function
\begin{align*}
    f\brak{x,y} = 
    \begin{cases}
        \frac{1}{\sqrt{2 \pi y}} e^{\frac{-1}{2y}\brak{x-y}^2}, & \text{ if } -\infty < x < \infty, 0 < y < 1\\
        0, & \text{ otherwise }
    \end{cases}
\end{align*}
\item The variance of the random variable $X$ is
\begin{multicols}{2}
    \begin{enumerate}
        \item $\frac{1}{12}$
        \item $\frac{1}{4}$
        \item $\frac{7}{12}$
        \item $\frac{5}{12}$
    \end{enumerate}
\end{multicols}
\item The covariance between the random variables $X$ adn $Y$ is
\begin{multicols}{2}
    \begin{enumerate}
        \item $\frac{1}{3}$
        \item $\frac{1}{4}$
        \item $\frac{1}{6}$
        \item $\frac{1}{12}$
    \end{enumerate}
\end{multicols}

\subsection*{Statement for Linked Answer Questions 57 \& 58:}
Consider the function $f\brak{z} = \frac{e^{\iota z}}{z\brak{z^2+1}}$
\item The residue of $f$ at the isolated singular point in the upper half plane $\cbrak{z = x+ \iota y \in \mathbb{C} \colon y > 0}$ is
\begin{multicols}{2}
    \begin{enumerate}
        \item $\frac{-1}{2e}$
        \item $\frac{-1}{e}$
        \item $\frac{e}{2}$
        \item $1$
    \end{enumerate}
\end{multicols}
\item The Cauchy Principal Value of the integral $\int_{-\infty} ^ {\infty} \frac{\sin {x}, dx }{x\brak{x^2 + 1}}$ is
\begin{multicols}{2}
    \begin{enumerate}
        \item $-2 \pi \brak{1 + 2e^{-1}}$
        \item $\pi \brak{1 - e ^{-1}}$
        \item $2 \pi \brak{1+e}$
        \item $-\pi \brak{1 + e^{-1}}$
    \end{enumerate}
\end{multicols}

\subsection*{Statement for Linked Answer Questions 59 \& 60:}
Let $f\brak{x,y} = kxy - x^3y - xy^3$ for $\brak{x,y} \in \mathbb{R}$, where $k$ is a real constant. The directional derivative of $f$ at the point $\brak{1,2}$ in the direction of the unit vector $u = \brak{\frac{-1}{\sqrt{2}}, \frac{-1}{\sqrt{2}}}$ is $\frac{15}{\sqrt{2}}$.
\item The value of $k$ is
\begin{multicols}{2}
    \begin{enumerate}
        \item 2
        \item 4
        \item 1
        \item -1
    \end{enumerate}
\end{multicols}
\item The value of $f$ at a local minimum in the rectangular region $R = \cbrak{\brak{x,y} \in \mathbb{R}^2 \colon \abs{x}< \frac{3}{2} , \abs{y} < \frac{3}{2}}$ is
\begin{multicols}{2}
    \begin{enumerate}
        \item -2
        \item -3
        \item $\frac{-7}{8}$
        \item 0
    \end{enumerate}
\end{multicols}
%\end{enumerate}
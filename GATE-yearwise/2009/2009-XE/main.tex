\iffalse
\chapter{2009}
\author{AI24BTECH11009}
\section{xe}
\fi

\item Under what conditions is the equation $\Delta \cdot \rho \overrightarrow{V} = 0$ valid ?\\
P : Steady incompressible flow \\
Q : Unsteady incompressible flow \\
R : Steady compressible flow \\
S : Unsteady compressible flow 
    \begin{enumerate}
        \item P,Q,R
        \item Q,R,S
        \item P,R,S
        \item P,Q,S \\
    \end{enumerate}
\item Stream function CANNOT be defined for
\begin{enumerate}
    \item two dimensional incompressible flow
    \item two dimensional compressible flow
    \item three dimensional incompressible flow
    \item axisymmetric incompressible flow \\
\end{enumerate}
\item Which one of the following is an irrotational flow ?
\begin{enumerate}
    \item Free vortex flow
    \item Forced vortex flow
    \item Couette flow
    \item Wake flow \\
\end{enumerate}
\item Under strong wind conditions, electrical cables can be subjected to wind-induced oscillations. Which one of the following non-dimensional numbersis relevant to this problem ?
 \begin{enumerate}
     \item Froude number
     \item Weber number
     \item Faraday number
     \item Strouhal number \\
 \end{enumerate}
\item Dimples are made on golf balls for which of the following reasons ? \\
P : to make the ball travel a longer distance \\
Q : to make the flow over the ball turbulent \\
R : to make the flow over the ball laminar \\
S : to create a separated boundary layer flow over the ball
\begin{enumerate}
    \item P, Q
    \item Q, S
    \item R, S
    \item P, R \\
\end{enumerate}
\item In a 2-D boundary layer flow, $x$ and $y$ are the streamwise and wall-normal coordinates, respectively. If $u$ denotes the velocity along $x$ direction, which one of the following represents the condition at the point of flow separation ?
\begin{enumerate}
    \item $\frac{\partial u}{\partial x} = 0$
    \item $\frac{\partial u}{\partial y} = 0$
    \item $\frac{\partial^2 u}{\partial x^2} = 0$
    \item $\frac{\partial^2 u}{\partial y^2} = 0$ \\
\end{enumerate}
\item Which one among the following boundary layer flows is the LEAST susceptible to flow separation ?
\begin{enumerate}
    \item turbulent boundary layer in a favourable pressure gradient
    \item laminar boundary layer in a favourable pressure gradient
    \item turbulent boundary layer in an adverse pressure gradient
    \item laminar boundary layer in an adverse pressure gradient \\
\end{enumerate}
\item Air from the blower of a hairdryer flows between two identical elliptical cylinders suspended freely, for two cases shown in the figure. The cylinders would move 
\begin{figure}[!ht]
\centering
\resizebox{0.5\textwidth}{!}{%
\begin{circuitikz}
\tikzstyle{every node}=[font=\LARGE]
\draw  (4.5,12.25) ellipse (1cm and 0.5cm);
\draw  (4.5,10.5) ellipse (1cm and 0.5cm);
\draw  (9.75,12.5) ellipse (0.5cm and 0.75cm);
\draw  (9.75,10.25) ellipse (0.5cm and 0.75cm);
\draw [->, >=Stealth] (2.25,11.75) -- (3.5,11.75);
\draw [->, >=Stealth] (2.25,11.5) -- (3.5,11.5);
\draw [->, >=Stealth] (2.25,11.25) -- (3.5,11.25);
\draw [->, >=Stealth] (2.25,11) -- (3.5,11);
\draw [->, >=Stealth] (7.75,11.75) -- (9,11.75);
\draw [->, >=Stealth] (7.75,11.5) -- (9,11.5);
\draw [->, >=Stealth] (7.75,11.25) -- (9,11.25);
\draw [->, >=Stealth] (7.75,11) -- (9,11);
\node [font=\normalsize] at (8,12) {Air flow};
\node [font=\normalsize] at (2.25,12) {Air flow};
\node [font=\normalsize] at (4.25,9.5) {Case 1};
\node [font=\normalsize] at (9.75,9.25) {Case 2};
\end{circuitikz}

}%
\end{figure}
 \begin{enumerate}
    \item away from each other for Case 1 and towards each other for Case 2
    \item towards each other for Case 1 and away from each other for Case 2
    \item away from each other for Case 1 and away from each other for Case 2
    \item towards each other for Case 1 and towards each other for Case 2 \\
 \end{enumerate}
\item A 40 cm cubical block slides on oil (viscosity = 0.80 Pa.s), over a large plane horizontal surface. If the oil film between the block and the surface has a uniform thickness of 0.4 mm, what will be the force required to drag the block at 4 m/s ? Ignore the end effects and treat the flow as two dimensional.
\begin{enumerate}
     \item 1280 N
     \item 1640 N
     \item 1920 N
     \item 2560 N \\
 \end{enumerate}
\item For a floating body, G, B, and M represent centre of gravity, centre of buoyancy, and the metacentre, respectively. The body will be stable if
\begin{enumerate}
    \item G is located above B
    \item B is located above M
    \item M is located above B
    \item M is located above G \\
\end{enumerate}
\item A nozzle has inlet and outlet diameters of 10 cm and 5 cm, respectively. If it discharges air at a steady rate of 0.1 $\text{m}^3$/s into atmosphere, the gauge pressure (static) at the nozzle inlet will be
\begin{enumerate}
    \item 1.26 kPa
    \item 1.46 kPa
    \item 3.52 kPa
    \item 3.92 kPa \\
\end{enumerate}
\item Consider incompressible flow through a two-dimensional open channel. At a certain section A-A, the velocity profile is parabolic. Neglecting air resistance at the free surface, find the volume flow rate per unit width of the channel.
 \begin{figure}[!ht]
\centering
\resizebox{0.5\textwidth}{!}{%
\begin{circuitikz}
\tikzstyle{every node}=[font=\normalsize]
\draw [short] (4.25,11.5) -- (10.25,11.5);
\draw [short] (4.25,8.75) -- (10.25,8.75);
\draw [short] (7,11.5) -- (7,8.75);
\draw [dashed] (7,8.75) -- (7,8);
\draw [dashed] (7,12) -- (7,11.25);
\draw [<->, >=Stealth] (5.75,11.5) -- (5.75,8.75);
\draw [short] (7,8.75) .. controls (8.25,9.75) and (8.5,10) .. (9,11.5);
\draw [->, >=Stealth] (7,9.25) -- (7.75,9.25);
\draw [->, >=Stealth] (7,9.75) -- (8.25,9.75);
\draw [->, >=Stealth] (7,10.25) -- (8.5,10.25);
\draw [->, >=Stealth] (7,11.25) -- (9,11.25);
\draw [->, >=Stealth] (7,10.75) -- (8.75,10.75);
\node [font=\normalsize] at (5.25,10.25) {2 m};
\node [font=\normalsize] at (4.75,12) {Air};
\node [font=\normalsize] at (4.75,11) {Water};
\node [font=\normalsize] at (7,12.25) {A};
\node [font=\normalsize] at (7,8) {A};
\node [font=\normalsize] at (8,11.75) {V=10 m/s};
\end{circuitikz}

}%
\end{figure}
 \begin{enumerate}
    \item 10 $\text{m}^3$/s
    \item 13.33 $\text{m}^3$/s
    \item 20 $\text{m}^3$/s
    \item 33.33 $\text{m}^3$/s \\
\end{enumerate}

\iffalse
\chapter{2022}
\author{AI24BTECH11028}
\section{ee}
\fi

    \item $e^A$  denotes the exponential of a square matrix A. Suppose $\lambda$ is an eigenvalue and $v$ is the corresponding eigen-vector of matrix A. \\
    Consider the following two statements:\\
    Statement 1: $e^{\lambda}$ is an eigenvalue of $e^A$.\\
    Statement 2: $v$ is an eigenvector of $e^A$.\\
    Which one of the following options is correct? 
    \begin{enumerate}
        \item Statement 1 is true and statement 2 is false
        \item Statement 1 is false and statement 2 is true
        \item Both the statements are correct. 
        \item Both the statements are false. 
    \end{enumerate}

    \item Let $f(x) = \int_0^x e^t \brak{t - 1}\brak{t - 2} \, dt.$. Then $f\brak{x}$ decreases in the interval
    \begin{enumerate}
        \item $x \in \brak{1,2}$
        \item $x \in \brak{2,3}$
        \item $x \in \brak{0,1}$
        \item $x \in \brak{0.5,1}$
    \end{enumerate}

    \item Consider a matrix
    \begin{align*}
        A = 
            \begin{bmatrix}
            1 & 0 & 0 \\
            0 & 4 & -2 \\
            0 & 1 & 1
            \end{bmatrix}        
    \end{align*}
    The matrix $A$ satisfies the equation $6A^{-1} = A^2 + cA + dI$, where $c$ and $d$ are scalars and $I$ is the identity matrix. Then $\brak{c + d}$ is equal to
    \begin{enumerate}
        \item 5
        \item 17
        \item -6
        \item 11
    \end{enumerate}

    \item The fuel cost functions in rupees/hour for two 600 MW thermal power plants are given by
    \begin{align*}
        Plant 1: C_1 = 350 + 6P_1 + 0.004P_1^2
        Plant 2: C_2 = 450 + aP_2 + 0.003P_2^2
    \end{align*}
    where $P_1$ and $P_2$ are power generated by plant 1 and plant 2, respectively, in MW and $a$ is constant. The incremental cost of power $\brak{\lambda}$ is 8 rupees per MWh. The two thermal power plants together meet a total power demand of 550 MW. The optimal generation of plant 1 and plant 2 in MW, respectively, are 
    \begin{enumerate}
        \item 200, 350 
        \item 250, 300 
        \item 325, 225
        \item 350, 200
    \end{enumerate}

    \item  The current gain $\brak{I_{out}/I_{in}}$in the circuit with an ideal current amplifier given below is
    \begin{figure}[!ht]
\centering
\resizebox{0.3\textwidth}{!}{%
\begin{circuitikz}
\tikzstyle{every node}=[font=\normalsize]
\draw (1.25,8.75) to (1.25,8) node[ground]{};
\draw (1.25,9.5) to[american current source] (1.25,9);
\draw (1.25,8.75) to[short] (1.25,9.25);
\draw (1.25,9.5) to[short] (1.25,10.25);
\draw (1.25,10.25) to[short] (2.75,10.25);
\draw (4.25,9.75) node[op amp,scale=1] (opamp2) {};
\draw (opamp2.+) to[short] (2.75,9.25);
\draw  (opamp2.-) to[short] (2.75,10.25);
\draw (5.45,9.75) to[short](5.75,9.75);
\draw (2.75,9.25) to (2.75,9) node[ground]{};
\draw (5.75,9.75) to[C] (6.75,9.75);
\draw (7.5,9.75) to (7.5,8.5) node[ground]{};
\draw (6.75,9.75) to[short] (7.5,9.75);
\draw (2.25,10.25) to[short] (2.25,12.25);
\draw (5.5,9.75) to[short] (5.5,12.25);
\draw (2.25,12.25) to[C] (5.5,12.25);
\draw [->, >=Stealth] (6.75,9.5) -- (7.25,9.5);
\node [font=\small] at (3.75,12.75) {$C_r$};
\node [font=\small] at (0.75,9.25) {$I_{in}$};
\node [font=\small] at (6.25,10.25) {$C_e$};
\node [font=\small] at (7,9.25) {$I_{out}$};
\node [font=\normalsize] at (4,9.75) {$\infty$};
\end{circuitikz}
}%

\label{fig:my_label}
\end{figure}
\begin{enumerate}
    \item $\frac{C_f}{C_c}$
    \item $\frac{C_f}{C_c}$
    \item $\frac{C_f}{C_c}$
    \item $\frac{C_f}{C_c}$
\end{enumerate}

\item If the magnetic field intensity (H) in a conducting region is given by the expression $H = x^2 \hat{i}+ x^2y^2\hat{j}+ x^2y^2z^2\hat{k} A/m$. The magnitude of the current density, in $A/m^2$, at x = 1m, y = 2m, and z = 1m is
\begin{enumerate}
    \item 8
    \item 12
    \item 16
    \item 20
\end{enumerate}

\item Let a causal LTI system be governed by the following differential equation $y\brak{t} + \frac{1}{4}\frac{dy}{dt} = 2x\brak{t}$, where $x\brak{t}$ and $y\brak{t}$ are the input and output respectively. Its impulse response is 
\begin{enumerate}
    \item $2e^{-\frac{1}{4}u\brak{t}}$
    \item $2e^{-4u\brak{t}}$
    \item $8e^{-\frac{1}{4}u\brak{t}}$
    \item $2e^{-4u\brak{t}}$
\end{enumerate}

\item Let an input $x\brak{t} = 2\sin\brak{10\pi t} + 5\cos\brak{15\pi t} + 7\sin\brak{42\pi t} + 4\cos\brak{45\pi t}$ is passed through an LTI system having an impulse response,
\begin{align*}
    h\brak{t} = 2\brak{\frac{2\sin\brak{10\pi t}}{\pi t}}\cos\brak{40\pi t}
\end{align*}
The output of the system is
\begin{enumerate}
    \item $2\sin\brak{10\pi t} + 5\cos\brak{15\pi t}$
    \item $5\cos\brak{15\pi t} + 7\sin\brak{42\pi t}$
    \item $7\sin\brak{42\pi t} + 4\cos\brak{45\pi t}$
    \item $2\sin\brak{10\pi t} + 4\cos\brak{45\pi t}$
\end{enumerate}

\item Consider the system as shown below
\begin{figure}[!ht]
\centering
\resizebox{0.3\textwidth}{!}{%
\begin{circuitikz}
\tikzstyle{every node}=[font=\Large]
\draw  (1.75,11.75) rectangle (5.5,9.25);
\draw [->, >=Stealth] (0.25,10.75) -- (1.25,10.75);
\draw [->, >=Stealth] (5.5,10.75) -- (6.5,10.75);
\draw (1,10.75) to[short] (1.75,10.75);
\draw (6.5,10.75) to[short] (7,10.75);
\node [font=\Large] at (1,11.25) {$x(t)$};
\node [font=\Large] at (6.5,11.25) {$y(t)$};
\end{circuitikz}
}%

\label{fig:my_label}
\end{figure}
where $y\brak{t} = x\brak{e^t}$. The system is
\begin{enumerate}
    \item linear and causal.
    \item linear and non-causal. 
    \item non-linear and causal. 
    \item non-linear and non-causal.
\end{enumerate}

\item The discrete time Fourier series representation of a signal $x[n]$ with period $N$ is written as  $x[n] = \sum_{k=0}^{N-1} a_k e^{j(2\pi kn / N)}.$ A discrete time periodic signal with period $N=3$, has the non-zero Fourier series coefficients: $a_{-3} = 2$ and $a_4 = 1$. The signal is:
\begin{enumerate}
    \item $2 + 2e^{-j\frac{2\pi}{6}n} \cos\left(\frac{2\pi}{6}n\right)$
    \item $1 + 2e^{-j\frac{2\pi}{6}n} \cos\left(\frac{2\pi}{6}n\right)$
    \item $1 + 2e^{j\frac{2\pi}{6}n} \cos\left(\frac{2\pi}{6}n\right)$
    \item $2 + 2e^{j\frac{2\pi}{6}n} \cos\left(\frac{2\pi}{6}n\right)$
\end{enumerate}

\item Let $f(x, y, z) = 4x^2 + 7xy + 3xz^2$. The direction in which the function $f(x, y, z)$ increases most rapidly at point $P = (1, 0, 2)$ is:
\begin{enumerate}
    \item $20\hat{i} + 7\hat{j}$
    \item $20\hat{i} + 7\hat{j} + 12\hat{k}$
    \item $20\hat{i} + 12\hat{k}$
    \item $20\hat{i}$
\end{enumerate}

\item Let $R$ be a region in the first quadrant of the $xy$ plane enclosed by a closed curve $C$ considered in counter-clockwise direction. Which of the following expressions does not represent the area of the region $R$?
\begin{figure}[!ht]
\centering
\resizebox{0.3\textwidth}{!}{%
\begin{circuitikz}
\tikzstyle{every node}=[font=\normalsize]
\draw [->, >=Stealth] (1.5,7.75) -- (5.75,7.75);
\draw [->, >=Stealth] (1.5,7.75) -- (1.5,11.5);
\draw  (3.75,9.75) ellipse (1.25cm and 0.75cm);
\draw [->, >=Stealth] (4,10.5) -- (3.75,10.5);
\node [font=\small] at (1.25,11.25) {$y$};
\node [font=\small] at (5.75,7.5) {$x$};
\node [font=\normalsize] at (3.5,10.75) {$C$};
\node [font=\normalsize] at (3.25,9.75) {$R$};
\end{circuitikz}
}%

\label{fig:my_label}
\end{figure}
\begin{enumerate}
    \item $\int\!\!\!\int_R dx \, dy$
    \item $\oint_C x \, dy$
    \item $\oint_C y \, dx$
    \item $\frac{1}{2} \oint_C (x \, dy - y \, dx)$
\end{enumerate}

\item  Let $\hat{E}(x, y, z) = 2x z^2 \, \hat{i} + 5y \, \hat{j} + 3z \, \hat{k}$. The value of 
$\iiint_V (\nabla \cdot \overrightarrow{E}) , dV,$
where $V$ is the volume enclosed by the unit cube defined by $0 \leq x \leq 1$, $0 \leq y \leq 1$, and $0 \leq z \leq 1$, is:
\begin{enumerate}
    \item $3$
    \item $8$
    \item $10$
    \item $12$
\end{enumerate}

    



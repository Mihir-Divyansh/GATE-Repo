 \iffalse
        \chapter{2010}
        \author{AI24BTECH11034}
        \section{ph}
 \fi
\item The solution of the differential equation for $y\brak{t}$:  $\frac{d^2y}{dt^2} - y = 2 \cosh\brak{t}$, subject to the initial conditions $y\brak{0} = 0$ and $\frac{dy}{dt}|_{t=0} = 0$, is:
\begin{multicols}{2}
\begin{enumerate}
    \item $\frac{1}{2}\cosh\brak{t} + t\sinh\brak{t}$
    \item $-\sinh\brak{t} + t\cosh\brak{t}$
    \item $t\cosh\brak{t}$
    \item $t\sinh\brak{t}$
\end{enumerate}
\end{multicols}

\item Given the recurrence relation for the Legendre polynomials:
\begin{align*}
\brak{2n+1}P_n\brak{x} = \brak{n+1}P_{n+1}\brak{x} + nP_{n-1}\brak{x}
\end{align*}
which of the following integrals has a non-zero value?
\begin{multicols}{2}
\begin{enumerate}
    \item $\int_{-1}^{1} x^2 P_n\brak{x} P_{n+1}\brak{x} dx$
    \item $\int_{-1}^{1} x P_n\brak{x} P_{n+2}\brak{x} dx$
    \item $\int_{-1}^{1} x \sbrak{P_n\brak{x}}^2 dx$
    \item $\int_{-1}^{1} x^2 P_n\brak{x} P_{n+2}\brak{x} dx$
\end{enumerate}
\end{multicols}

\item For a two-dimensional free electron gas, the electronic density n and the Fermi energy $E_f$ are related by:
\begin{multicols}{2}
\begin{enumerate}
\item n = $\frac{\brak{2mE_f}^{\frac{3}{2}}}{3\pi^2\hbar^3}$\\

\item n = $\frac{mE_f}{\pi\hbar^2}$

\item n = $\frac{mE_f}{2\pi\hbar^2}$\\

\item n = $\frac{2^{\frac{3}{2}}\brak{mE_f}^{\frac{1}{2}}}{\pi\hbar}$

\end{enumerate}
\end{multicols}

\item Far away from any of the resonance frequencies of a medium, the real part of the dielectric permittivity is
\begin{enumerate}
\item Always independent of frequency
\item Monotonically decreasing with frequency
\item Monotonically increasing with frequency
\item A non-monotonic function of frequency
\end{enumerate}

\item The ground state wavefunction of a deuteron is in a superposition of s and d states. Which of the following is NOT true as a consequence?
\begin{enumerate}
\item It has a non-zero quadrupole moment
\item The neutron-proton potential is non-central
\item The orbital wavefunction is not spherically symmetric
\item The Hamiltonian does not conserve the total angular momentum
\end{enumerate}

\item The first three energy levels of $^{228}\text{Th}_{90}$ are shown below:

\begin{figure}[!ht]
\centering
\resizebox{3cm}{3cm}{%
\begin{circuitikz}
\tikzstyle{every node}=[font=\small]
\draw [<->, >=Stealth] (6.5,10.75) -- (6.5,8.25);
\draw [<->, >=Stealth] (5.25,9.25) -- (7.75,9.25);
\draw [<->, >=Stealth] (5.5,8.5) -- (7.5,10);
\draw [<->, >=Stealth] (5.5,10.25) -- (7.25,8.5);
\draw [<->, >=Stealth] (7,8.25) -- (6,10.5);
\draw [<->, >=Stealth] (5.25,9.75) -- (7.75,8.75);
\draw [<->, >=Stealth] (6,8.25) -- (7,10.5);
\draw [dashed] (3.25,9.25) -- (5.25,9.25);
\draw [dashed] (7.75,9.25) -- (8.75,9.25);
\draw [dashed] (6.5,10.5) -- (6.5,11.25);
\draw [->, >=Stealth] (6.5,11) -- (6.5,12.75);
\draw [->, >=Stealth] (8.75,9.25) -- (10.5,9.25);
\draw [dashed] (5.5,10.25) -- (3.75,11.75);
\draw [->, >=Stealth] (2.5,12.5) -- (3.25,11.75);
\draw [->, >=Stealth] (2.75,12.75) -- (3.5,12);
\draw [->, >=Stealth] (3,13) -- (3.75,12.25);
\draw [->, >=Stealth] (2.25,12.25) -- (3,11.5);
\node [font=\Huge] at (6.5,9.5) {.};
\node [font=\small] at (6.5,13) {$y$};
\node [font=\small] at (10.75,9.25) {$x$};
\node [font=\small] at (4.75,10) {$30\degree$};
\node [font=\small] at (4.25,12) {$U=1\;\frac{cm}{s}$};
\end{circuitikz}
}%
\end{figure}

The expected spin-parity and energy of the next level are given by:
\begin{multicols}{2}
\begin{enumerate}
\item $\brak{6^+, 400 \text{ keV}}$
\item $\brak{6^+, 300 \text{ keV}}$
\item $\brak{2^+, 400 \text{ keV}}$
\item $\brak{4^+, 300 \text{ keV}}$
\end{enumerate} 
\end{multicols}

\item The quark content of $\Sigma^+, K, \pi$ and p is indicated:
\begin{align*}
|\Sigma^+\rangle &= |uus\rangle; \quad |K^+\rangle = |us\rangle; \quad |\pi^+\rangle = |ud\rangle; \quad |p\rangle = |uud\rangle.
\end{align*}
In the process, $\pi^{-} + p \rightarrow K^{-} + \Sigma^{\prime}$, considering strong interactions only, which of the following statements is true?

\begin{enumerate}
    \item The process is allowed because $\Delta S = 0$.
    \item The process is allowed because $\Delta I_{y} = 0$.
    \item The process is not allowed because $\Delta S \neq 0$ and $\Delta I_{z} \neq 0$.
    \item The process is not allowed because the baryon number is violated.
\end{enumerate}

\item The three principal moments of inertia of a methanol (CH$_3$OH) molecule have the property $I_x = I_y = I$ and $I_z \neq I$. The rotational energy eigenvalues are
\begin{multicols}{2}
\begin{enumerate}
    \item $\frac{\hbar^2}{2I}l\brak{l+1} + \frac{\hbar^2 m_i^2}{2}\brak{\frac{1}{I_i} - \frac{1}{I}}$\\
    \item $\frac{\hbar^2}{2I}l\brak{l+1}$
    \item $\frac{\hbar^2 m_i^2}{2}\brak{\frac{1}{I_i} - \frac{1}{I}}$\\
    \item $\frac{\hbar^2}{2I}l\brak{l+1} + \frac{\hbar^2 m_i^2}{2}\brak{\frac{1}{I_i} + \frac{1}{I}}$
\end{enumerate}
\end{multicols}
\item A particle of mass $m$ is confined in the potential 

\begin{equation*}
V(x) = \begin{cases}
\frac{1}{2}m\omega^2 x^2 & \text{for } x > 0, \\
\infty & \text{for } x \leq 0.
\end{cases}
\end{equation*}

Let the wavefunction of the particle is given by

\begin{align*}
\psi(x) = -\frac{1}{\sqrt{5}}\psi_0 + \frac{2}{\sqrt{5}}\psi_1,
\end{align*}

where $\psi_0$ and $\psi_1$ are the eigenfunctions of the ground state and the first excited state, respectively. The expectation value of the energy is


\begin{multicols}{4}
\begin{enumerate}
\item $\frac{31}{10}\hbar\omega$ 
\item $\frac{25}{10}\hbar\omega$ 
\item $\frac{13}{10}\hbar\omega$
\item $\frac{11}{10}\hbar\omega$
\end{enumerate}
\end{multicols}

\item Match the typical spectra of stable molecules with the corresponding wave-number range:
\begin{multicols}{2}
			\begin{enumerate}[label=(\Alph*)]
                
				\item Electronic spectra
				\item Rotational spectra
                    \item Molecular dissociation
			\end{enumerate}
			\columnbreak
			\begin{enumerate}[label=(\arabic*)]
				\item $10^6 cm^{-1}$ and above
				\item $10^5 - 10^6 cm^{-1}$
				\item $10^0 - 10^2 cm^{-1}$
                    
			\end{enumerate}
		\end{multicols}

      \begin{multicols}{2}
        \begin{enumerate}
            \item $A-2,B-1,C-3$
            \item $A-2,B-3,C-1$
            \item $A-3,B-2,C-1$
            \item $A-1,B-2,C-3$
        \end{enumerate}
    \end{multicols}
 
\item Consider the operations $P: \vec{r} \rightarrow -\vec{r}$ (parity) and $T: t \rightarrow -t$ (time-reversal). For the electric and magnetic fields $\vec{E}$ and $\vec{B}$, which of the following set of transformations is correct?
\begin{multicols}{2}
\begin{enumerate}
\item $P: \vec{E} \to -\vec{E}, \vec{B} \to \vec{B}$;\\ $T: \vec{E} \to \vec{E}, \vec{B} \to -\vec{B}$
\item $P: \vec{E} \to \vec{E}, \vec{B} \to \vec{B}$;\\ $T: \vec{E} \to \vec{E}, \vec{B} \to \vec{B}$
\item $P: \vec{E} \to -\vec{E}, \vec{B} \to \vec{B}$;\\ $T: \vec{E} \to -\vec{E}, \vec{B} \to -\vec{B}$
\item $P: \vec{E} \to \vec{E}, \vec{B} \to -\vec{B}$; \\$T: \vec{E} \to -\vec{E}, \vec{B} \to \vec{B}$
\end{enumerate}
\end{multicols}

\item Two magnetic dipoles of magnitude $m$ each are placed in a plane as shown.


The energy of interaction is given by:
\begin{multicols}{2}
\begin{enumerate}
\item Zero
\item $\dfrac{\mu_0}{4\pi} \dfrac{m^2}{d^3}$
\item $\dfrac{3\mu_0}{2\pi} \dfrac{m^2}{d^3}$
\item $-\dfrac{3\mu_0}{8\pi} \dfrac{m^2}{d^3}$
\end{enumerate}
\end{multicols}

\item Consider a conducting loop of radius $a$ and total loop resistance $R$ placed in a region with a magnetic field $B$, thereby enclosing a flux $\Phi_0$. The loop is connected to an electronic circuit as shown, the capacitor being initially uncharged.


If the loop is pulled out of the region of the magnetic field at a constant speed $v$, the final output voltage $V_{out}$ is independent of:
\begin{multicols}{2}
\begin{enumerate}
 \item $\Phi_0$
 \item $R$
 \item $u$
 \item $C$
 \end{enumerate}
 \end{multicols}

 


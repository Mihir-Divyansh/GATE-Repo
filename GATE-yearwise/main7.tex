\iffalse
\title{Assignment7}
\author{ee24btech11064}
\chapter{2016}
\section{ph}
\fi

%\begin{enumerate}
\item Protons and $\alpha$- particles of equal initial momenta are scattered off a gold foil in a Rutherford scattering experiment. The scattering for proton on gold and $\alpha$-particle on gold are $\sigma_p$ and $\sigma_{\alpha}$ respectively. The ratio $\frac{\sigma_\alpha}{\sigma_p}$ is \underline{\hspace{2cm}}. 

\vspace{0.5cm}
\item For the digital circuit given below, the output $X$ is 
\begin{figure}[!ht]
\centering
\scalebox{0.8}{ % Adjust the scaling factor as needed
\begin{circuitikz}
\tikzstyle{every node}=[font=\footnotesize]

\draw (4.5,8.25) to[short] (4.75,8.25);
\draw (4.5,7.75) to[short] (4.75,7.75);
\draw (4.75,8.25) node[ieeestd or port, anchor=in 1, scale=0.89](port){} (port.out) to[short] (6.5,8);
\draw (4.75,9.75) node[ieeestd not port, anchor=in](port){} (port.out) to[short] (6.5,9.75);
\draw (port.in) to[short] (4.5,9.75);
\draw (7.75,9.25) to[short] (8,9.25);
\draw (7.75,8.75) to[short] (8,8.75);
\draw (8,9.25) node[ieeestd nand port, anchor=in 1, scale=0.89](port){} (port.out) to[short] (9.75,9);
\draw [short] (4.5,9.75) -- (4.75,9.75)node[pos=0.5,left, fill=white]{A};
\draw [short] (4.5,8.25) -- (4.75,8.25)node[pos=0,left, fill=white]{B};
\draw [short] (4.5,7.75) -- (4.75,7.75)node[pos=0,left, fill=white]{C};
\draw [short] (9.75,9) -- (10.25,9)node[pos=1,right, fill=white]{X};
\draw (6.5,8) to[short] (6.5,8.75);
\draw (6.5,9.75) to[short] (6.5,9.25);
\draw (6.5,9.25) to[short] (8,9.25);
\draw (6.5,8.75) to[short] (8,8.75);
\end{circuitikz}
}
\end{figure}
\begin{multicols}{2}
  \begin{enumerate}
      \item $\overline{\overline{A} + B \cdot C}$
      \item $\overline{\overline{A} \cdot (B + C)}$
      \item $\overline{A} \cdot (B + C)$
      \item $A + \overline{(B \cdot C)}$
  \end{enumerate}
    
\end{multicols}
\vspace{0.5cm}
\item The Fermi energies of two metals $X$and $Y$ are $5ev$ and $7ev$ and their Debye temperatues are 170K and 340K, respectively. The moalr specific heats of these metals at constant volume at low temparatues can be written ad $\brak{C_v}x=\gamma_X T+A_X T^3$ and $\brak{C_v}Y=\gamma_Y T+A_Y T^3$, where $\gamma$ and A are constants. Assuming that the thermal effective mass of the elctrons in the two metals are same,which of the follwoing is correct ?
\begin{multicols}{2}
\begin{enumerate}
    \item $\frac{\gamma_X}{\gamma_Y}=\frac{7}{5}$, $\frac{A_X}{A_Y}=8$
    \item $\frac{\gamma_X}{\gamma_Y}=\frac{5}{7}$, $\frac{A_X}{A_Y}=\frac{1}{8}$
    \item $\frac{\gamma_X}{\gamma_Y}=\frac{7}{5}$, $\frac{A_X}{A_Y}=\frac{1}{8}$
    \item $\frac{\gamma_X}{\gamma_Y}=\frac{5}{7}$, $\frac{A_X}{A_Y}=8$
\end{enumerate}
\end{multicols}
\vspace{0.5cm}
\item A two-level system has energies zero and E. The level with zero energy is non-degenerate, while the level with energy E is triply degenerate. The mean energy of a classical particle in this system at temperature T is
\begin{multicols}{2}
    \begin{enumerate}
        \item $\frac{Ee^{-E/k_b T}}{1+3e^{-E/k_b T}}$
        \item $\frac{3Ee^{-E/k_b T}}{1+e^{-E/k_b T}}$
        \item $\frac{Ee^{-E/k_b T}}{1+e^{-E/k_b T}}$
        \item $\frac{3Ee^{-E/k_b T}}{1+3e^{-E/k_b T}}$
    \end{enumerate}
\end{multicols}

\vspace{0.5cm}
\item A partical of rest mass M is moving along the positive x-direction.It decays into two photons $\gamma_1$ and $\gamma_2$ as shown in the figure. The energy of $\gamma_1$ is $1GeV$ and the energy of $\gamma_2$ is $0.82 GeV$. The value of M (in units of GeV/$c^2$ is \underline{\hspace{2cm}}. Give your answer upto two decimal places)

\begin{circuitikz}
\centering
\tikzstyle{every node}=[font=\normalsize]
\draw (5.75,9) to[short] (5.75,9);
\draw [ line width=0.9pt](3.25,9.5) to[short] (6.5,9.5);
\draw [->, >=Stealth] (3.25,9.5) -- (5.25,9.5)node[pos=0.9,above, fill=white]{M};
\draw [line width=0.6pt, dashed] (6.5,9.5) -- (9,9.5);
\draw [->, >=Stealth, dashed] (6.5,9.5) -- (8.25,11)node[pos=1,right, fill=white]{$\gamma_1$};
\draw [->, >=Stealth, dashed] (6.5,9.5) -- (8,8)node[pos=1,right, fill=white]{$\gamma_2$};
\draw [short] (7.5,9) -- (7.5,9)node[pos=0.5, fill=white]{$60^0$};
\draw [short] (7,10) .. controls (7.25,10) and (7.5,10) .. (7.5,9.5)node[pos=0.45,right, fill=white]{$45^0$};
\draw [short] (7,9) .. controls (7.25,9.25) and (7.25,9.25) .. (7.25,9.5);
\end{circuitikz}
\vspace{0.5cm}
\item If x and p are the x components of the position and the momentum operators of a particle respectively, the commutator [$x^2,p^2$] is 
\begin{multicols}{2}
\begin{enumerate}
    \item $ih(xp-px)$
    \item $2ih(xp-px)$
    \item  $ih(xp+px)$
    \item  $2ih(xp+px)$
\end{enumerate}
\end{multicols}
\vspace{0.5cm}
\item The x-y plane is the boundary between free space and a magnetic material with relative permeability $\mu_r$. The magnetic field in the free space is $B_x\hat{i}+B_z\hat{k}$. The magnetic field in the magnetic material is 
\begin{multicols}{2}
    \begin{enumerate}
        \item $B_x\hat{i}+B_z\hat{k}$
        \item $B_x\hat{i}+\mu_r B_z\hat{k}$
        \item $\frac{1}{\mu_r}B_x\hat{i}+B_z\hat{k}$
        \item $\mu_rB_x\hat{i}+B_z\hat{k}$
    \end{enumerate}
\end{multicols}
\vspace{0.5cm}

\item Let \( |l, m \rangle \) be the simultaneous eigenstates of \( L^2 \) and \( L_z \). Here \( \vec{L} \) is the angular momentum operator with Cartesian components \( (L_x, L_y, L_z) \), \( l \) is the angular momentum quantum number and \( m \) is the azimuthal quantum number. The value of \( \langle 1, 0 | (L_x + iL_y) | 1, -1 \rangle \) is
\begin{multicols}{2}
\begin{enumerate}
    \item[(A)] $0$
    \item[(B)] $h$
    \item[(C)] $\sqrt{2}h$
    \item[(D)] $\sqrt{3}h$
\end{enumerate}
\end{multicols}

\vspace{0.5cm}

\item For the parity operator P, whoch of the following statements is NOT true?
\begin{multicols}{2}
    \begin{enumerate}
        \item $P^+=P$
        \item $P^2=-P$
        \item $P^2=I$
        \item $P^+=P^{-1}$
    \end{enumerate}
\end{multicols}
\vspace{0.5cm}

\item For the transistor shown in the figure, assume $V_{BE}=0.7V$ and $\beta_{dc}=100$. If $V_{in}=5V$, $V_{out}\brak{\text{in Volts}}$ is \underline{\hspace{2cm}}. (Give your answer upto one decimal place)

\begin{figure}[H]
\centering
\resizebox{0.4\textwidth}{!}{%
\begin{circuitikz}
\tikzstyle{every node}=[font=\normalsize]

% NPN Transistor
\draw (10.5,10.25) to[Tnpn, transistors/scale=1.19] (10.5,13.25);

% Output Voltage Node
\draw (10.5,12.25) to[short, -o] (11.75,12.25) node[right] {$V_{\text{out}}$};

% Resistor between nodes (7,11.75) and (9.5,11.75)
\draw (7,11.75) to[R, l={\normalsize 200k $\Omega$}] (9.5,11.75);

% Resistor between transistor and ground
\draw (10.5,10.25) to[R, l={\normalsize 1k $\Omega$}] (10.5,8.5);

% Resistor above transistor
\draw (10.5,15) to[R, l={\normalsize 3k $\Omega$}] (10.5,13.25);

% Power Supply Node
\draw (10.5,15) to[short, -o] (10.5,15.25) node[above] {10V};

% Input Voltage Node
\draw (7,11.75) to[short, -o] (6.75,11.75) node[above] {$V_{\text{in}}$};

% Ground Node
\draw (10.5,8.5) to (10.5,8.25) node[ground] {};
\end{circuitikz}
}%
\end{figure}
\vspace{0.5cm}
\item The state of a system is given by 
\begin{align*}
    |\psi \rangle = |\phi_1\rangle+2|\phi_2\rangle+3|\phi_3\rangle
\end{align*}
where $|\phi_1\rangle$, $|\phi_2\rangle$ and $|\phi_3\rangle$ form an orthonormal set. The probability of finding the system in the state $|\phi_2\rangle$ is  \underline{\hspace{2cm}}. (Give your answer upto two decimal places)
\vspace{0.5cm}
\item According to the nuclear shee model, the respective groudn state spin-parity values of $^{15}_8O$ and $^{17}_8O$ nuclei are 
\begin{multicols}{2}
    \begin{enumerate}
        \item $\frac{1^{+}}{2},\frac{1^{-}}{2}$
        \item $\frac{1^-}{2},\frac{5^+}{2}$
        \item $\frac{3^-}{2},\frac{5^+}{2}$
        \item $\frac{3^-}{2},\frac{1^-}{2}$
    \end{enumerate}
\end{multicols}
\vspace{0.5cm}
\item A particle of mass $m$ and energy $E$, moving in the positive x direction, is incident on a step potential at $x=0$, as indicated in the figure. The height of the potential is $V_0$, where $V_0>E$. At $x=x_0$, where $x_0>0$, the probability of finding the electron is $\frac{1}{e}$ times the probability of finding it at $x=0$. If $\alpha=\sqrt{\frac{2m\brak{V_0-E}}{H^2}}$, the value of $x_0$ is 
\begin{figure}[H]
\centering
\resizebox{0.4\textwidth}{!}{%
\begin{circuitikz}
\tikzstyle{every node}=[font=\normalsize]
\draw (5.75,10.75) to[short] (8.5,10.75);
\draw (8.5,10.75) to[short] (8.5,13);
\draw (8.5,13) to[short] (11.25,13);
\draw [dashed] (8.5,10.75) -- (11.25,10.75);
\draw [short] (10.25,10.75) -- (10.25,11);
\draw [short] (10.25,10.75) -- (10.25,10.5)node[pos=0.5, fill=white]{$x=x_0$};
\draw [->, >=Stealth] (5.75,12.25) -- (7.5,12.25)node[pos=0.5,above, fill=white]{E};
\draw [short] (9.5,13) -- (10.25,13)node[pos=0.5,above, fill=white]{$V_0$};
\draw [short] (7.75,10.75) -- (8.5,10.75)node[pos=0.5,below, fill=white]{$x=x_0$};
\end{circuitikz}
}%
\end{figure}
\begin{multicols}{2}
    \begin{enumerate}
        \item $\frac{2}{\alpha}$
        \item $\frac{1}{\alpha}$
        \item $\frac{1}{2\alpha}$
        \item $\frac{1}{4\alpha}$
    \end{enumerate}
\end{multicols}
%\end{enumerate}

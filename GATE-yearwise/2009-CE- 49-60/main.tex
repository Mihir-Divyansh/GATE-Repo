
\iffalse
\title{2009-CE- 49-60}
\author{EE24BTECH11016 - DHWANITH M DODDAHUNDI}
\section{ce}
\chapter{2009}
\fi

% \maketitle
% \newpage
% \bigskip

\item The magnetic bearing of a line AB was N 59\degree 30' W in the year 1967, when the declination was 4\degree 10' E. If the present declination is 3\degree W, the whole circle bearing of the line is
\begin{enumerate}
    \item 299\degree20'
    \item 307\degree20'
    \item 293\degree40'
    \item 301\degree40'

\end{enumerate}
\item Determine the correctness or otherwise of the following \textbf{Assertion\sbrak{a}} and \textbf{Reason\sbrak{r}}: \\
\textbf{Assertion\sbrak{a}}: Curvature correction must be applied when sights are long \\
\textbf{Reason\sbrak{r}}: Line of collimation is not a level line but is tangential to the level line 
\begin{enumerate}
    \item Both \sbrak{a} and \sbrak{r} are true and \sbrak{r} is the correct reason for \sbrak{a}
    \item Both \sbrak{a} and \sbrak{r} are true but \sbrak{r} is \textbf{not} the correct reason for \sbrak{a}
    \item Both \sbrak{a} and \sbrak{r} are false
    \item \sbrak{a} is false but [r] is true \\
\end{enumerate} 
 \textbf{Common data questions} \\
 \textbf{Common data for questions 51 and 52} \\
 Examine the test arrangement and the soil properties given below:
\begin{figure}[H]
			\centering
			\begin{figure}[H]
\centering
\resizebox{6cm}{!}{%
\begin{circuitikz}
\tikzstyle{every node}=[font=\Huge]
\draw [line width=0.7pt, short] (1.5,13) -- (1.5,11);
\draw [line width=0.7pt, short] (1.5,12.25) -- (5.75,12.25);
\draw [line width=0.7pt, short] (1.5,12) -- (5.75,12);
\draw [line width=0.7pt, short] (5.75,12.5) -- (5.75,11.75);
\draw [line width=0.7pt, short] (5.75,12.5) -- (13.25,12.5);
\draw [line width=0.7pt, short] (5.75,11.75) -- (13.25,11.75);
\draw [line width=0.7pt, short] (13.25,12.5) -- (13.25,11.75);
\draw [line width=0.7pt, short] (1,12.5) -- (1.5,13);
\draw [line width=0.7pt, short] (1.5,12.75) -- (1,12.25);
\draw [line width=0.7pt, short] (1.5,12.5) -- (1,12);
\draw [line width=0.7pt, short] (1.5,12.25) -- (1,11.75);
\draw [line width=0.7pt, short] (1.5,12) -- (1,11.5);
\draw [line width=0.7pt, short] (1.5,11.75) -- (1,11.25);
\draw [line width=0.7pt, short] (1.5,11.5) -- (1,11);
\draw [line width=0.7pt, short] (1.5,11.25) -- (1,10.75);
\node [font=\Huge] at (0.5,12) {P};
\node [font=\Huge] at (5.5,13) {Q};
\node [font=\Huge] at (9.25,13) {Rigid};
\node [font=\Huge] at (3.5,13) {El};
\draw [line width=0.7pt, ->, >=Stealth] (12.75,14.5) -- (12.75,12.5);
\node [font=\Huge] at (13.75,12) {R};
\node [font=\Huge] at (12.5,15) {W};
\draw [line width=0.7pt, <->, >=Stealth] (1.5,10.5) -- (6,10.5)node[pos=0.5, fill=white]{L};
\draw [line width=0.7pt, <->, >=Stealth] (6,10.5) -- (13.25,10.5)node[pos=0.5, fill=white]{L};
\draw [line width=0.7pt, short] (6,11.25) -- (6,9.75);
\draw [line width=0.7pt, short] (1.5,10.75) -- (1.5,9.75);
\draw [line width=0.7pt, short] (13.25,10.75) -- (13.25,9.75);
\draw [line width=0.7pt, short] (13.25,11.25) -- (13.25,10.25);
\end{circuitikz}
}%

\end{figure}
		\end{figure}

 
 \item The maximum pressure that can be applied with a factor of safety of 3 through the concrete block, ensuring no bearing capacity failure in soil using Terzaghi's bearing capacity equation without considering the shape factor, depth factor and inclination factor is 
 
 \begin{enumerate}
     \item 26.67 kPa
     \item 60 kPa
     \item 90 kPa
     \item 120 kPa

 \end{enumerate}
 \item The maximum resistance offered by the soil through skin friction while pulling out the pile from the ground is
 \begin{enumerate}
     \item 104.9 kN
      \item 209.8 kN
     \item 236 kN
     \item 472 kN \\
\end{enumerate}
 \textbf{Common data for questions 53 and 54} \\
 Following chemical species were reported for water sample from a well:
 \begin{table}[H]
  \centering
  \begin{tabular}{c|c}
    \hline
    \textbf{Species} & \textbf{Concentration(milli equivalent/L)}\\
    \hline
    Chloride$(Cl^{-})$ & 15\\
    Sulphate$(SO_{4}^{2-}$ & 15 \\
    Carbonate$(CO_{3}^{2-}$ & 05 \\
    BiCarbonate$(HC0_{3}^{-}$ & 30 \\
    Calcium$(Ca^{2+})$ & 12 \\
    Magnesium$(Mg^{2+})$ & 18 \\ 
    pH & 8.5 \\
    \hline
    \end{tabular}
  \end{table}
  \item Total hardness in mg/L as $CaCO_{3}$  is
  \begin{enumerate}
      \item 1500
      \item 2000
      \item 3000
      \item 5000
  \end{enumerate}
  \item Alkalinity present in the wat in mg/L as $CaCO_{3}$ is 
  \begin{enumerate}
      \item 250
      \item 1500
      \item 1750
      \item 5000 \\
  \end{enumerate}
  \textbf{Common data for questions 55 and 56} \\
  One hour triangular unit of hydrograph of a watershed has a peak discharge of 60 $m^{3}/sec.cm$ at 10 hours and time base of 30 hours. The $\phi$ index is 0.4 cm per hour and base flow is 15 $m^{3}/sec$ 
  \item The catchment area of the watershed is 
  \begin{enumerate}
      \item 3.24 $km^{2}$
    \item 32.4 $km^{2}$
      \item 324 $km^{2}$
      \item 3240 $km^{2}$
  \end{enumerate}
  \item If there is rainfall of 5.4 cm in 1 hour, the ordinate of the flood hydrograph at $15^{th}$ hour is
  \begin{enumerate}
      \item 225 $m^{3}/sec$
      \item 240 $m^{3}/sec$
      \item 249 $m^{3}/sec$
      \item 258 $m^{3}/sec$ \\
  \end{enumerate}
  \textbf{Linked answer questions} \\
  \textbf{Statement for Linked answer question 57 and 58} \\
  In the cantilever beam PQR shown in figure below, the segment PQ has flexural rigidity El and the segment QR has infinite flexural rigidity.

  \begin{figure}[H]
			\centering
			\begin{figure}[H]
\usetikzlibrary{lindenmayersystems} 
\centering
\resizebox{6cm}{!}{%
\begin{circuitikz}
\tikzstyle{every node}=[font=\Huge]

\draw [line width=0.7pt, short] (-0.25,19.5) -- (7,19.5);
\draw [line width=0.7pt, short] (7,19.5) -- (7,20.5);
\draw [line width=0.7pt, short] (7,20.5) -- (8.5,20.5);
\draw [line width=0.7pt, short] (8.5,20.5) -- (8.5,19.75);
\draw [line width=0.7pt, short] (8.5,19.5) -- (17,19.5);
\draw [line width=0.7pt, short] (0.75,19.5) -- (0.75,20.75);
\draw [line width=0.7pt, short] (0.75,20.75) -- (2.75,20.75);
\draw [line width=0.7pt, short] (2.75,20.75) -- (2.75,19.5);
\draw [line width=0.7pt, short] (1.25,20.75) -- (1.25,22.25);
\draw [line width=0.7pt, short] (2,21) -- (2,20.75);
\draw [line width=0.7pt, short] (2,21) -- (2,22.25);
\draw [line width=0.7pt, short] (1.25,22.25) -- (2,22.25);
\draw [line width=0.7pt, short] (1.5,22.25) -- (1.5,23);
\draw [line width=0.7pt, short] (1.5,23) -- (1.75,23);
\draw [line width=0.7pt, short] (1.75,23) -- (1.75,22.25);
\draw [line width=0.7pt, short] (0.75,23) -- (15.5,23);
\draw [line width=0.7pt, short] (13.5,22) -- (13.5,19.5);
\draw [line width=0.7pt, short] (15.75,22) -- (15.75,19.75);
\draw [line width=0.7pt, short] (15.75,20) -- (15.75,19.75);
\draw [line width=0.7pt, short] (15.75,20) -- (15.75,19.5);
\draw [line width=0.7pt, short] (13.5,22) -- (15.75,22);
\draw [line width=0.7pt, short] (14.5,23) -- (14.25,22);
\draw [line width=0.7pt, short] (14.5,23) -- (15,22);
\draw [line width=0.7pt, short] (15.5,23) -- (16.25,23);
\draw [line width=0.7pt, short] (0.75,23) -- (-0.25,23);
\draw [line width=0.7pt, short] (-0.25,23) -- (-0.25,24);
\draw [line width=0.7pt, short] (-0.25,24) -- (16.25,24);
\draw [line width=0.7pt, short] (16.25,24) -- (16.25,23);
\draw [line width=0.7pt, short] (8.5,19.75) -- (8.5,15);
\draw [line width=0.7pt, short] (7,19.5) -- (7,15);
\draw [line width=0.7pt, short] (7,15) -- (8.5,15);
\draw [line width=0.7pt, short] (-0.25,19.5) -- (-1.5,19.5);
\draw [line width=0.7pt, short] (7,24) -- (9,24);
\draw [line width=0.7pt, short] (7.25,24) -- (7.25,19.75);
\draw [line width=0.7pt, short] (8.5,24) -- (10.5,24);
\draw [line width=0.7pt, short] (8.25,24) -- (8.25,19.75);
\draw [line width=0.7pt, short] (6.5,24) -- (6.5,24.5);
\draw [line width=0.7pt, short] (6.5,24.5) -- (9,24.5);
\draw [line width=0.7pt, short] (9,24.5) -- (9,24);
\draw [line width=0.7pt, short] (-1,19.5) -- (-1.5,18.75);
\draw [line width=0.7pt, short] (-1,19.5) -- (-0.5,18.75);
\draw [line width=0.7pt, short] (-0.5,19.5) -- (-1,18.75);
\draw [line width=0.7pt, short] (-0.5,19.5) -- (0.25,18.75);
\draw [line width=0.7pt, short] (0.25,19.5) -- (-0.25,18.75);
\draw [line width=0.7pt, short] (0.25,19.5) -- (1,18.75);
\draw [line width=0.7pt, short] (0.25,18.75) -- (1,19.5);
\draw [line width=0.7pt, short] (7,24.5) -- (7,25);
\draw [line width=0.7pt, short] (7,25) -- (7.5,25);
\draw [line width=0.7pt, short] (7.5,25) -- (7.5,24.5);
\draw [line width=0.7pt, short] (8,24.5) -- (8,25);
\draw [line width=0.7pt, short] (8,25) -- (8.5,25);
\draw [line width=0.7pt, short] (8.5,25) -- (8.5,24.5);
\draw [line width=0.7pt, short] (7.25,25) -- (7.25,25.5);
\draw [line width=0.7pt, short] (8.25,25) -- (8.25,25.5);
\draw [line width=0.7pt, dashed] (7.75,23) -- (7.75,20.5);
\draw [line width=0.7pt, dashed] (7.75,24.5) -- (7.75,26.5);
\draw [line width=0.7pt, <->, >=Stealth] (7.75,26.25) -- (2,26.25)node[pos=0.5, fill=white]{5m};
\draw [line width=0.7pt, <->, >=Stealth] (7.75,26.25) -- (14.25,26.25)node[pos=0.5, fill=white]{5m};
\draw [line width=0.7pt, dashed] (14.5,19) -- (14.5,26.75);
\draw [line width=0.7pt, dashed] (1.75,24) -- (1.75,27.25);
\draw [line width=0.7pt, dashed] (1.5,20.75) -- (1.5,18.25);
\node [font=\Large] at (2,18) {Saturated dense sand};
\node [font=\Large] at (0.75,17) {$\gamma_{sat} = 18N/m^3$};
\node [font=\Large] at (0.75,16.3) {$\phi = 35^{\degree}$};
\node [font=\Large] at (2.4,16.3) {$c = 0kPa$};
\node [font=\Large] at (2.29,16.3) {$,$};
\node [font=\Large] at (0,15.7) {$N_{\gamma} = 40 $};
\node [font=\Large] at (0.75,15.2) {$\gamma_w = 10kN/m^3$};
\draw [line width=0.7pt, <->, >=Stealth] (6.5,19.5) -- (6.5,15)node[pos=0.5, fill=white]{5m};
\node [font=\LARGE] at (4.5,22.5) {Concrete block};
\node [font=\LARGE] at (4.5,21.75) {$1.5 \times 1.0 \times 0.6m$};
\node [font=\LARGE] at (5.75,21) {high};
\draw [line width=0.7pt, ->, >=Stealth] (3.5,21.25) -- (2.5,20);
\draw [line width=0.7pt, ->, >=Stealth] (9.25,18) -- (7.75,18);
\node [font=\LARGE] at (11.75,23.5) {Rigid Steel Beam};
\node [font=\LARGE] at (13.25,18) {500 mm diameter bored pile };
\node [font=\LARGE] at (11.75,17) {Angle of friction };
\node [font=\LARGE] at (15.75,17) {$(\delta) = 24^{\degree}$};
\node [font=\LARGE] at (12.25,16) {Earth pressure coefficient };
\node [font=\LARGE] at (17,16) {(K) = 1.5};
\node [font=\LARGE] at (11,20.25) {G.W.T};
\draw [line width=0.7pt, ->, >=Stealth] (9.75,19.75) -- (9.75,19.5);
\draw [line width=2pt, ->, >=Stealth] (13.75,18.5) -- (15.75,18.5);
\draw [line width=2pt, ->, >=Stealth] (13.75,18.5) -- (15.75,18.5);
\end{circuitikz}
}

\end{figure}
			
		\end{figure}

\item The deflection and slope of the beam at 'Q' are respectively 
\begin{enumerate}
    \item $\frac{5WL^{3}}{6El}$ and $\frac{3WL^{2}}{2El}$
        \item $\frac{WL^{3}}{3El}$ and $\frac{WL^{2}}{2El}$
    \item $\frac{WL^{3}}{2El}$ and $\frac{WL^{2}}{El}$
    \item $\frac{WL^{3}}{3El}$ and $\frac{3WL^{2}}{2El}$        
\end{enumerate}
\item The deflection of the beam at 'R' is
\begin{enumerate}
    \item $\frac{8WL^{3}}{El}$
        \item $\frac{5WL^{3}}{6El}$
    \item $\frac{7WL^{3}}{3El}$
    \item $\frac{8WL^{3}}{6El}$ \\
\end{enumerate}
\textbf{Linked answer question 59 and 60} 
\item A saturated undisturbed sample from a clay strata has moisture content of 22.22\% and specific weight of 2.7. Assuming $\gamma_{w}$ = 10$kN/m^{3}$, the void ratio and the saturated unit weight of the clay, respectively are
\begin{enumerate}
    \item 0.6 and 16.875 $kN/m^{3}$
        \item 0.3 and 20.625 $kN/m^{3}$
    \item 0.6 and 20.625 $kN/m^{3}$
    \item 0.3 and 16.975 $kN/m^{3}$
\end{enumerate}
\item Using the properties of the clay layer derived from the above question, the consolidation settlement of the same clay layer under a square footing (neglecting its self weight) with additional data shown in the figure below (assume the stress distribution of IH:2V from the edge of the footing and $\gamma_{w}$ = 10$kN/m^{3}$) is 

\begin{figure}[H]
			\centering
			\begin{figure}[H]
\centering
\resizebox{6cm}{!}{%
\begin{circuitikz}
\tikzstyle{every node}=[font=\Huge]
\draw [line width=1.3pt, ->, >=Stealth] (5.25,16.25) -- (5.25,16);
\draw [line width=0.7pt, short] (-2.75,11.5) -- (9,11.5);
\draw [line width=0.7pt, short] (-3,13.5) -- (9,13.5);
\draw [line width=0.7pt, <->, >=Stealth] (0,13.5) -- (0,11.5);
\draw [line width=0.7pt, short] (1.5,14.75) -- (4.75,14.75);
\draw [line width=0.7pt, short] (1.5,14.75) -- (1.5,14);
\draw [line width=0.7pt, short] (1.5,14) -- (4.75,14);
\draw [line width=0.7pt, short] (4.75,14.75) -- (4.75,14);
\draw [line width=0.7pt, short] (2.75,14.75) -- (2.75,16.75);
\draw [line width=0.7pt, short] (2.75,16.75) -- (3.5,16.75);
\draw [line width=0.7pt, short] (3.5,16.75) -- (3.5,14.75);
\draw [line width=0.7pt, <->, >=Stealth] (1.5,13.75) -- (4.75,13.75);
\draw [line width=0.7pt, ->, >=Stealth] (3.25,18.25) -- (3.25,16.75);
\draw [line width=0.7pt, short] (-3,16) -- (2.75,16);
\draw [line width=0.7pt, short] (3.5,16) -- (9,16);
\node [font=\Large] at (5.75,12.75) {Stiff Clay};
\node [font=\Large] at (5.25,12) {Compression Index };
\node [font=\Large] at (8.5,12) {$(C_c) = 0.4$};
\node [font=\Large] at (5.75,10.75) {Dense sand};
\node [font=\Large] at (-1,12.5) {$1.0 m$};
\draw [ line width=0.7pt](-0.5,14.5) to[short] (1,14.5);
\draw [line width=0.7pt, <->, >=Stealth] (0.25,16) -- (0.25,14.5);
\draw [line width=0.7pt, <->, >=Stealth] (0.5,14.5) -- (0.5,13.5);
\node [font=\Large] at (6.5,15.25) {Saturated sand};
\node [font=\Large] at (6.75,14.5) {$\gamma_{sat} = 18 kN/m^3$};
\node [font=\Large] at (-0.25,14) {$1.0 m$};
\node [font=\Large] at (-0.5,15.5) {$1.0 m$};
\node [font=\Large] at (3,19) {$200kN$};
\node [font=\Large] at (3.25,13.25) {$2m \times 2m$};
\draw [line width=0.7pt, <->, >=Stealth] (1.25,14) -- (1.25,13.5);
\draw [line width=0.7pt, <->, >=Stealth] (5,14) -- (5,13.5);
\node [font=\Large] at (6.75,16.5) {G.W.T};
\end{circuitikz}
}%

\end{figure}
			
		\end{figure}

\begin{enumerate}
    \item 32.78 mm
        \item 61.75 mm
    \item 79.5 mm
    \item 131.13 mm
\end{enumerate}

